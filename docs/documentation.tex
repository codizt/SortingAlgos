\documentclass[12pt]{article}
\usepackage{xcolor}
\usepackage[utf8]{inputenc}
\usepackage{minted}
\usepackage{epigraph}
\usepackage{lmodern}
\usepackage{pgf}

\title{Analysis of Various Sorting Algorithms \\ 19CSE212}
\author{Deebakkarthi C R\\\color{magenta}\texttt{CB.EN.U4CSE20613}
\and
Devaraja G\\\color{magenta}\texttt{CB.EN.U4CSE20614}
\and
Krisha Vardhni M\\\color{magenta}\texttt{CB.EN.U4CSE20633}
}


\begin{document}
\begin{titlepage}
\maketitle
\end{titlepage}
\tableofcontents
\pagebreak
\begin{abstract}
	\epigraph{
		\textbf{Sort} (\textit{verb})}
	{To put a number of things in order or separate them into groups
}
	We come across sorting a lot in our day to day lives. For
	example a teacher asking the students to stand in the
	order of their height, a supermarket assistant
	replenishing stocks in the shelves, a deck of cards
	getting sorted, and the sort by price, popularity, etc.
	feature in ecommerce websites. All these involve
	arranging a set of things based on a set of rules. In
	relevance to computer science, sorting plays a key role
	in searching as it’s very efficient to search for an
	element in a sorted list rather than an unsorted one. The
	rules(algorithms) to sort efficiently have been a hot
	topic in computer science for decades and due to its
	relevance, numerous people have worked on building
	various sorting algorithms, as a result we now have a
	large number of those, out of which only a few are
	dominant in practical implications. We’ll be analyzing
	five of the most common algorithms in this report. \\


	Before we get into that we need to understand the notion of time
	and space complexity. Computers are fast but even they
	can only do so much. They are limited by the number of
	instructions they can run per unit time and the amount of
	storage they can use. In building these algorithms it's
	crucial that we consider these as they play a major role
	in the speed and space taken. In general the running time
	takes precedence over the space required as modern PCs
	have ample storage space available for most of the
	everyday tasks.
\end{abstract}
\section{Sorting Algorithms}
\subsection{Bubble Sort}
\subsubsection*{Principle}
In this sorting technique, the array is sequentially scanned
several times and during each iteration, the pairs of consecutive
elements are compared and interchanged(if required), to bring
them into ascending order. At the end of the first iteration the
largest element in the array is pushed to the end. It is an easy
but time consuming method when a large number of swaps are
required to take place.
\subsubsection*{Code}
\begin{minted}{python}
def bubbleSort(unsortedList):
	swap = 0
	itr = 0
	comp = 0
	tracemalloc.start()
	t_s = perf_counter_ns()
	for i in range(len(unsortedList)):
		for j in range(len(unsortedList) - 1 - i):
		itr += 1
		comp += 1
		if unsortedList[j] > unsortedList[j + 1]:
			t = unsortedList[j]
			unsortedList[j] = unsortedList[j + 1]
			unsortedList[j + 1] = t
			swap += 1
	t_e = perf_counter_ns()
	mem = tracemalloc.get_traced_memory()[1]
	tracemalloc.stop()
	return {"Time":t_e-t_s,
	"Memory":mem,
	"Comparisons":comp,
	"Swaps":swap,
	"Iterations":itr}
\end{minted}
\subsubsection*{Graphs}
%% Creator: Matplotlib, PGF backend
%%
%% To include the figure in your LaTeX document, write
%%   \input{<filename>.pgf}
%%
%% Make sure the required packages are loaded in your preamble
%%   \usepackage{pgf}
%%
%% Also ensure that all the required font packages are loaded; for instance,
%% the lmodern package is sometimes necessary when using math font.
%%   \usepackage{lmodern}
%%
%% Figures using additional raster images can only be included by \input if
%% they are in the same directory as the main LaTeX file. For loading figures
%% from other directories you can use the `import` package
%%   \usepackage{import}
%%
%% and then include the figures with
%%   \import{<path to file>}{<filename>.pgf}
%%
%% Matplotlib used the following preamble
%%   \usepackage{fontspec}
%%   \setmainfont{DejaVuSerif.ttf}[Path=\detokenize{/home/dbk/.local/lib/python3.10/site-packages/matplotlib/mpl-data/fonts/ttf/}]
%%   \setsansfont{DejaVuSans.ttf}[Path=\detokenize{/home/dbk/.local/lib/python3.10/site-packages/matplotlib/mpl-data/fonts/ttf/}]
%%   \setmonofont{DejaVuSansMono.ttf}[Path=\detokenize{/home/dbk/.local/lib/python3.10/site-packages/matplotlib/mpl-data/fonts/ttf/}]
%%
\begingroup%
\makeatletter%
\begin{pgfpicture}%
\pgfpathrectangle{\pgfpointorigin}{\pgfqpoint{6.400000in}{4.800000in}}%
\pgfusepath{use as bounding box, clip}%
\begin{pgfscope}%
\pgfsetbuttcap%
\pgfsetmiterjoin%
\definecolor{currentfill}{rgb}{1.000000,1.000000,1.000000}%
\pgfsetfillcolor{currentfill}%
\pgfsetlinewidth{0.000000pt}%
\definecolor{currentstroke}{rgb}{1.000000,1.000000,1.000000}%
\pgfsetstrokecolor{currentstroke}%
\pgfsetdash{}{0pt}%
\pgfpathmoveto{\pgfqpoint{0.000000in}{0.000000in}}%
\pgfpathlineto{\pgfqpoint{6.400000in}{0.000000in}}%
\pgfpathlineto{\pgfqpoint{6.400000in}{4.800000in}}%
\pgfpathlineto{\pgfqpoint{0.000000in}{4.800000in}}%
\pgfpathlineto{\pgfqpoint{0.000000in}{0.000000in}}%
\pgfpathclose%
\pgfusepath{fill}%
\end{pgfscope}%
\begin{pgfscope}%
\pgfsetbuttcap%
\pgfsetmiterjoin%
\definecolor{currentfill}{rgb}{1.000000,1.000000,1.000000}%
\pgfsetfillcolor{currentfill}%
\pgfsetlinewidth{0.000000pt}%
\definecolor{currentstroke}{rgb}{0.000000,0.000000,0.000000}%
\pgfsetstrokecolor{currentstroke}%
\pgfsetstrokeopacity{0.000000}%
\pgfsetdash{}{0pt}%
\pgfpathmoveto{\pgfqpoint{0.800000in}{0.528000in}}%
\pgfpathlineto{\pgfqpoint{5.760000in}{0.528000in}}%
\pgfpathlineto{\pgfqpoint{5.760000in}{4.224000in}}%
\pgfpathlineto{\pgfqpoint{0.800000in}{4.224000in}}%
\pgfpathlineto{\pgfqpoint{0.800000in}{0.528000in}}%
\pgfpathclose%
\pgfusepath{fill}%
\end{pgfscope}%
\begin{pgfscope}%
\pgfsetbuttcap%
\pgfsetroundjoin%
\definecolor{currentfill}{rgb}{0.000000,0.000000,0.000000}%
\pgfsetfillcolor{currentfill}%
\pgfsetlinewidth{0.803000pt}%
\definecolor{currentstroke}{rgb}{0.000000,0.000000,0.000000}%
\pgfsetstrokecolor{currentstroke}%
\pgfsetdash{}{0pt}%
\pgfsys@defobject{currentmarker}{\pgfqpoint{0.000000in}{-0.048611in}}{\pgfqpoint{0.000000in}{0.000000in}}{%
\pgfpathmoveto{\pgfqpoint{0.000000in}{0.000000in}}%
\pgfpathlineto{\pgfqpoint{0.000000in}{-0.048611in}}%
\pgfusepath{stroke,fill}%
}%
\begin{pgfscope}%
\pgfsys@transformshift{1.020945in}{0.528000in}%
\pgfsys@useobject{currentmarker}{}%
\end{pgfscope}%
\end{pgfscope}%
\begin{pgfscope}%
\definecolor{textcolor}{rgb}{0.000000,0.000000,0.000000}%
\pgfsetstrokecolor{textcolor}%
\pgfsetfillcolor{textcolor}%
\pgftext[x=1.020945in,y=0.430778in,,top]{\color{textcolor}\ttfamily\fontsize{10.000000}{12.000000}\selectfont 0}%
\end{pgfscope}%
\begin{pgfscope}%
\pgfsetbuttcap%
\pgfsetroundjoin%
\definecolor{currentfill}{rgb}{0.000000,0.000000,0.000000}%
\pgfsetfillcolor{currentfill}%
\pgfsetlinewidth{0.803000pt}%
\definecolor{currentstroke}{rgb}{0.000000,0.000000,0.000000}%
\pgfsetstrokecolor{currentstroke}%
\pgfsetdash{}{0pt}%
\pgfsys@defobject{currentmarker}{\pgfqpoint{0.000000in}{-0.048611in}}{\pgfqpoint{0.000000in}{0.000000in}}{%
\pgfpathmoveto{\pgfqpoint{0.000000in}{0.000000in}}%
\pgfpathlineto{\pgfqpoint{0.000000in}{-0.048611in}}%
\pgfusepath{stroke,fill}%
}%
\begin{pgfscope}%
\pgfsys@transformshift{1.922764in}{0.528000in}%
\pgfsys@useobject{currentmarker}{}%
\end{pgfscope}%
\end{pgfscope}%
\begin{pgfscope}%
\definecolor{textcolor}{rgb}{0.000000,0.000000,0.000000}%
\pgfsetstrokecolor{textcolor}%
\pgfsetfillcolor{textcolor}%
\pgftext[x=1.922764in,y=0.430778in,,top]{\color{textcolor}\ttfamily\fontsize{10.000000}{12.000000}\selectfont 200}%
\end{pgfscope}%
\begin{pgfscope}%
\pgfsetbuttcap%
\pgfsetroundjoin%
\definecolor{currentfill}{rgb}{0.000000,0.000000,0.000000}%
\pgfsetfillcolor{currentfill}%
\pgfsetlinewidth{0.803000pt}%
\definecolor{currentstroke}{rgb}{0.000000,0.000000,0.000000}%
\pgfsetstrokecolor{currentstroke}%
\pgfsetdash{}{0pt}%
\pgfsys@defobject{currentmarker}{\pgfqpoint{0.000000in}{-0.048611in}}{\pgfqpoint{0.000000in}{0.000000in}}{%
\pgfpathmoveto{\pgfqpoint{0.000000in}{0.000000in}}%
\pgfpathlineto{\pgfqpoint{0.000000in}{-0.048611in}}%
\pgfusepath{stroke,fill}%
}%
\begin{pgfscope}%
\pgfsys@transformshift{2.824582in}{0.528000in}%
\pgfsys@useobject{currentmarker}{}%
\end{pgfscope}%
\end{pgfscope}%
\begin{pgfscope}%
\definecolor{textcolor}{rgb}{0.000000,0.000000,0.000000}%
\pgfsetstrokecolor{textcolor}%
\pgfsetfillcolor{textcolor}%
\pgftext[x=2.824582in,y=0.430778in,,top]{\color{textcolor}\ttfamily\fontsize{10.000000}{12.000000}\selectfont 400}%
\end{pgfscope}%
\begin{pgfscope}%
\pgfsetbuttcap%
\pgfsetroundjoin%
\definecolor{currentfill}{rgb}{0.000000,0.000000,0.000000}%
\pgfsetfillcolor{currentfill}%
\pgfsetlinewidth{0.803000pt}%
\definecolor{currentstroke}{rgb}{0.000000,0.000000,0.000000}%
\pgfsetstrokecolor{currentstroke}%
\pgfsetdash{}{0pt}%
\pgfsys@defobject{currentmarker}{\pgfqpoint{0.000000in}{-0.048611in}}{\pgfqpoint{0.000000in}{0.000000in}}{%
\pgfpathmoveto{\pgfqpoint{0.000000in}{0.000000in}}%
\pgfpathlineto{\pgfqpoint{0.000000in}{-0.048611in}}%
\pgfusepath{stroke,fill}%
}%
\begin{pgfscope}%
\pgfsys@transformshift{3.726400in}{0.528000in}%
\pgfsys@useobject{currentmarker}{}%
\end{pgfscope}%
\end{pgfscope}%
\begin{pgfscope}%
\definecolor{textcolor}{rgb}{0.000000,0.000000,0.000000}%
\pgfsetstrokecolor{textcolor}%
\pgfsetfillcolor{textcolor}%
\pgftext[x=3.726400in,y=0.430778in,,top]{\color{textcolor}\ttfamily\fontsize{10.000000}{12.000000}\selectfont 600}%
\end{pgfscope}%
\begin{pgfscope}%
\pgfsetbuttcap%
\pgfsetroundjoin%
\definecolor{currentfill}{rgb}{0.000000,0.000000,0.000000}%
\pgfsetfillcolor{currentfill}%
\pgfsetlinewidth{0.803000pt}%
\definecolor{currentstroke}{rgb}{0.000000,0.000000,0.000000}%
\pgfsetstrokecolor{currentstroke}%
\pgfsetdash{}{0pt}%
\pgfsys@defobject{currentmarker}{\pgfqpoint{0.000000in}{-0.048611in}}{\pgfqpoint{0.000000in}{0.000000in}}{%
\pgfpathmoveto{\pgfqpoint{0.000000in}{0.000000in}}%
\pgfpathlineto{\pgfqpoint{0.000000in}{-0.048611in}}%
\pgfusepath{stroke,fill}%
}%
\begin{pgfscope}%
\pgfsys@transformshift{4.628218in}{0.528000in}%
\pgfsys@useobject{currentmarker}{}%
\end{pgfscope}%
\end{pgfscope}%
\begin{pgfscope}%
\definecolor{textcolor}{rgb}{0.000000,0.000000,0.000000}%
\pgfsetstrokecolor{textcolor}%
\pgfsetfillcolor{textcolor}%
\pgftext[x=4.628218in,y=0.430778in,,top]{\color{textcolor}\ttfamily\fontsize{10.000000}{12.000000}\selectfont 800}%
\end{pgfscope}%
\begin{pgfscope}%
\pgfsetbuttcap%
\pgfsetroundjoin%
\definecolor{currentfill}{rgb}{0.000000,0.000000,0.000000}%
\pgfsetfillcolor{currentfill}%
\pgfsetlinewidth{0.803000pt}%
\definecolor{currentstroke}{rgb}{0.000000,0.000000,0.000000}%
\pgfsetstrokecolor{currentstroke}%
\pgfsetdash{}{0pt}%
\pgfsys@defobject{currentmarker}{\pgfqpoint{0.000000in}{-0.048611in}}{\pgfqpoint{0.000000in}{0.000000in}}{%
\pgfpathmoveto{\pgfqpoint{0.000000in}{0.000000in}}%
\pgfpathlineto{\pgfqpoint{0.000000in}{-0.048611in}}%
\pgfusepath{stroke,fill}%
}%
\begin{pgfscope}%
\pgfsys@transformshift{5.530036in}{0.528000in}%
\pgfsys@useobject{currentmarker}{}%
\end{pgfscope}%
\end{pgfscope}%
\begin{pgfscope}%
\definecolor{textcolor}{rgb}{0.000000,0.000000,0.000000}%
\pgfsetstrokecolor{textcolor}%
\pgfsetfillcolor{textcolor}%
\pgftext[x=5.530036in,y=0.430778in,,top]{\color{textcolor}\ttfamily\fontsize{10.000000}{12.000000}\selectfont 1000}%
\end{pgfscope}%
\begin{pgfscope}%
\definecolor{textcolor}{rgb}{0.000000,0.000000,0.000000}%
\pgfsetstrokecolor{textcolor}%
\pgfsetfillcolor{textcolor}%
\pgftext[x=3.280000in,y=0.240063in,,top]{\color{textcolor}\ttfamily\fontsize{10.000000}{12.000000}\selectfont Size of Array}%
\end{pgfscope}%
\begin{pgfscope}%
\pgfsetbuttcap%
\pgfsetroundjoin%
\definecolor{currentfill}{rgb}{0.000000,0.000000,0.000000}%
\pgfsetfillcolor{currentfill}%
\pgfsetlinewidth{0.803000pt}%
\definecolor{currentstroke}{rgb}{0.000000,0.000000,0.000000}%
\pgfsetstrokecolor{currentstroke}%
\pgfsetdash{}{0pt}%
\pgfsys@defobject{currentmarker}{\pgfqpoint{-0.048611in}{0.000000in}}{\pgfqpoint{-0.000000in}{0.000000in}}{%
\pgfpathmoveto{\pgfqpoint{-0.000000in}{0.000000in}}%
\pgfpathlineto{\pgfqpoint{-0.048611in}{0.000000in}}%
\pgfusepath{stroke,fill}%
}%
\begin{pgfscope}%
\pgfsys@transformshift{0.800000in}{0.684574in}%
\pgfsys@useobject{currentmarker}{}%
\end{pgfscope}%
\end{pgfscope}%
\begin{pgfscope}%
\definecolor{textcolor}{rgb}{0.000000,0.000000,0.000000}%
\pgfsetstrokecolor{textcolor}%
\pgfsetfillcolor{textcolor}%
\pgftext[x=0.368305in, y=0.631440in, left, base]{\color{textcolor}\ttfamily\fontsize{10.000000}{12.000000}\selectfont 0.00}%
\end{pgfscope}%
\begin{pgfscope}%
\pgfsetbuttcap%
\pgfsetroundjoin%
\definecolor{currentfill}{rgb}{0.000000,0.000000,0.000000}%
\pgfsetfillcolor{currentfill}%
\pgfsetlinewidth{0.803000pt}%
\definecolor{currentstroke}{rgb}{0.000000,0.000000,0.000000}%
\pgfsetstrokecolor{currentstroke}%
\pgfsetdash{}{0pt}%
\pgfsys@defobject{currentmarker}{\pgfqpoint{-0.048611in}{0.000000in}}{\pgfqpoint{-0.000000in}{0.000000in}}{%
\pgfpathmoveto{\pgfqpoint{-0.000000in}{0.000000in}}%
\pgfpathlineto{\pgfqpoint{-0.048611in}{0.000000in}}%
\pgfusepath{stroke,fill}%
}%
\begin{pgfscope}%
\pgfsys@transformshift{0.800000in}{1.172698in}%
\pgfsys@useobject{currentmarker}{}%
\end{pgfscope}%
\end{pgfscope}%
\begin{pgfscope}%
\definecolor{textcolor}{rgb}{0.000000,0.000000,0.000000}%
\pgfsetstrokecolor{textcolor}%
\pgfsetfillcolor{textcolor}%
\pgftext[x=0.368305in, y=1.119563in, left, base]{\color{textcolor}\ttfamily\fontsize{10.000000}{12.000000}\selectfont 0.25}%
\end{pgfscope}%
\begin{pgfscope}%
\pgfsetbuttcap%
\pgfsetroundjoin%
\definecolor{currentfill}{rgb}{0.000000,0.000000,0.000000}%
\pgfsetfillcolor{currentfill}%
\pgfsetlinewidth{0.803000pt}%
\definecolor{currentstroke}{rgb}{0.000000,0.000000,0.000000}%
\pgfsetstrokecolor{currentstroke}%
\pgfsetdash{}{0pt}%
\pgfsys@defobject{currentmarker}{\pgfqpoint{-0.048611in}{0.000000in}}{\pgfqpoint{-0.000000in}{0.000000in}}{%
\pgfpathmoveto{\pgfqpoint{-0.000000in}{0.000000in}}%
\pgfpathlineto{\pgfqpoint{-0.048611in}{0.000000in}}%
\pgfusepath{stroke,fill}%
}%
\begin{pgfscope}%
\pgfsys@transformshift{0.800000in}{1.660821in}%
\pgfsys@useobject{currentmarker}{}%
\end{pgfscope}%
\end{pgfscope}%
\begin{pgfscope}%
\definecolor{textcolor}{rgb}{0.000000,0.000000,0.000000}%
\pgfsetstrokecolor{textcolor}%
\pgfsetfillcolor{textcolor}%
\pgftext[x=0.368305in, y=1.607687in, left, base]{\color{textcolor}\ttfamily\fontsize{10.000000}{12.000000}\selectfont 0.50}%
\end{pgfscope}%
\begin{pgfscope}%
\pgfsetbuttcap%
\pgfsetroundjoin%
\definecolor{currentfill}{rgb}{0.000000,0.000000,0.000000}%
\pgfsetfillcolor{currentfill}%
\pgfsetlinewidth{0.803000pt}%
\definecolor{currentstroke}{rgb}{0.000000,0.000000,0.000000}%
\pgfsetstrokecolor{currentstroke}%
\pgfsetdash{}{0pt}%
\pgfsys@defobject{currentmarker}{\pgfqpoint{-0.048611in}{0.000000in}}{\pgfqpoint{-0.000000in}{0.000000in}}{%
\pgfpathmoveto{\pgfqpoint{-0.000000in}{0.000000in}}%
\pgfpathlineto{\pgfqpoint{-0.048611in}{0.000000in}}%
\pgfusepath{stroke,fill}%
}%
\begin{pgfscope}%
\pgfsys@transformshift{0.800000in}{2.148945in}%
\pgfsys@useobject{currentmarker}{}%
\end{pgfscope}%
\end{pgfscope}%
\begin{pgfscope}%
\definecolor{textcolor}{rgb}{0.000000,0.000000,0.000000}%
\pgfsetstrokecolor{textcolor}%
\pgfsetfillcolor{textcolor}%
\pgftext[x=0.368305in, y=2.095810in, left, base]{\color{textcolor}\ttfamily\fontsize{10.000000}{12.000000}\selectfont 0.75}%
\end{pgfscope}%
\begin{pgfscope}%
\pgfsetbuttcap%
\pgfsetroundjoin%
\definecolor{currentfill}{rgb}{0.000000,0.000000,0.000000}%
\pgfsetfillcolor{currentfill}%
\pgfsetlinewidth{0.803000pt}%
\definecolor{currentstroke}{rgb}{0.000000,0.000000,0.000000}%
\pgfsetstrokecolor{currentstroke}%
\pgfsetdash{}{0pt}%
\pgfsys@defobject{currentmarker}{\pgfqpoint{-0.048611in}{0.000000in}}{\pgfqpoint{-0.000000in}{0.000000in}}{%
\pgfpathmoveto{\pgfqpoint{-0.000000in}{0.000000in}}%
\pgfpathlineto{\pgfqpoint{-0.048611in}{0.000000in}}%
\pgfusepath{stroke,fill}%
}%
\begin{pgfscope}%
\pgfsys@transformshift{0.800000in}{2.637068in}%
\pgfsys@useobject{currentmarker}{}%
\end{pgfscope}%
\end{pgfscope}%
\begin{pgfscope}%
\definecolor{textcolor}{rgb}{0.000000,0.000000,0.000000}%
\pgfsetstrokecolor{textcolor}%
\pgfsetfillcolor{textcolor}%
\pgftext[x=0.368305in, y=2.583933in, left, base]{\color{textcolor}\ttfamily\fontsize{10.000000}{12.000000}\selectfont 1.00}%
\end{pgfscope}%
\begin{pgfscope}%
\pgfsetbuttcap%
\pgfsetroundjoin%
\definecolor{currentfill}{rgb}{0.000000,0.000000,0.000000}%
\pgfsetfillcolor{currentfill}%
\pgfsetlinewidth{0.803000pt}%
\definecolor{currentstroke}{rgb}{0.000000,0.000000,0.000000}%
\pgfsetstrokecolor{currentstroke}%
\pgfsetdash{}{0pt}%
\pgfsys@defobject{currentmarker}{\pgfqpoint{-0.048611in}{0.000000in}}{\pgfqpoint{-0.000000in}{0.000000in}}{%
\pgfpathmoveto{\pgfqpoint{-0.000000in}{0.000000in}}%
\pgfpathlineto{\pgfqpoint{-0.048611in}{0.000000in}}%
\pgfusepath{stroke,fill}%
}%
\begin{pgfscope}%
\pgfsys@transformshift{0.800000in}{3.125191in}%
\pgfsys@useobject{currentmarker}{}%
\end{pgfscope}%
\end{pgfscope}%
\begin{pgfscope}%
\definecolor{textcolor}{rgb}{0.000000,0.000000,0.000000}%
\pgfsetstrokecolor{textcolor}%
\pgfsetfillcolor{textcolor}%
\pgftext[x=0.368305in, y=3.072057in, left, base]{\color{textcolor}\ttfamily\fontsize{10.000000}{12.000000}\selectfont 1.25}%
\end{pgfscope}%
\begin{pgfscope}%
\pgfsetbuttcap%
\pgfsetroundjoin%
\definecolor{currentfill}{rgb}{0.000000,0.000000,0.000000}%
\pgfsetfillcolor{currentfill}%
\pgfsetlinewidth{0.803000pt}%
\definecolor{currentstroke}{rgb}{0.000000,0.000000,0.000000}%
\pgfsetstrokecolor{currentstroke}%
\pgfsetdash{}{0pt}%
\pgfsys@defobject{currentmarker}{\pgfqpoint{-0.048611in}{0.000000in}}{\pgfqpoint{-0.000000in}{0.000000in}}{%
\pgfpathmoveto{\pgfqpoint{-0.000000in}{0.000000in}}%
\pgfpathlineto{\pgfqpoint{-0.048611in}{0.000000in}}%
\pgfusepath{stroke,fill}%
}%
\begin{pgfscope}%
\pgfsys@transformshift{0.800000in}{3.613315in}%
\pgfsys@useobject{currentmarker}{}%
\end{pgfscope}%
\end{pgfscope}%
\begin{pgfscope}%
\definecolor{textcolor}{rgb}{0.000000,0.000000,0.000000}%
\pgfsetstrokecolor{textcolor}%
\pgfsetfillcolor{textcolor}%
\pgftext[x=0.368305in, y=3.560180in, left, base]{\color{textcolor}\ttfamily\fontsize{10.000000}{12.000000}\selectfont 1.50}%
\end{pgfscope}%
\begin{pgfscope}%
\pgfsetbuttcap%
\pgfsetroundjoin%
\definecolor{currentfill}{rgb}{0.000000,0.000000,0.000000}%
\pgfsetfillcolor{currentfill}%
\pgfsetlinewidth{0.803000pt}%
\definecolor{currentstroke}{rgb}{0.000000,0.000000,0.000000}%
\pgfsetstrokecolor{currentstroke}%
\pgfsetdash{}{0pt}%
\pgfsys@defobject{currentmarker}{\pgfqpoint{-0.048611in}{0.000000in}}{\pgfqpoint{-0.000000in}{0.000000in}}{%
\pgfpathmoveto{\pgfqpoint{-0.000000in}{0.000000in}}%
\pgfpathlineto{\pgfqpoint{-0.048611in}{0.000000in}}%
\pgfusepath{stroke,fill}%
}%
\begin{pgfscope}%
\pgfsys@transformshift{0.800000in}{4.101438in}%
\pgfsys@useobject{currentmarker}{}%
\end{pgfscope}%
\end{pgfscope}%
\begin{pgfscope}%
\definecolor{textcolor}{rgb}{0.000000,0.000000,0.000000}%
\pgfsetstrokecolor{textcolor}%
\pgfsetfillcolor{textcolor}%
\pgftext[x=0.368305in, y=4.048304in, left, base]{\color{textcolor}\ttfamily\fontsize{10.000000}{12.000000}\selectfont 1.75}%
\end{pgfscope}%
\begin{pgfscope}%
\definecolor{textcolor}{rgb}{0.000000,0.000000,0.000000}%
\pgfsetstrokecolor{textcolor}%
\pgfsetfillcolor{textcolor}%
\pgftext[x=0.312750in,y=2.376000in,,bottom,rotate=90.000000]{\color{textcolor}\ttfamily\fontsize{10.000000}{12.000000}\selectfont Time}%
\end{pgfscope}%
\begin{pgfscope}%
\definecolor{textcolor}{rgb}{0.000000,0.000000,0.000000}%
\pgfsetstrokecolor{textcolor}%
\pgfsetfillcolor{textcolor}%
\pgftext[x=0.800000in,y=4.265667in,left,base]{\color{textcolor}\ttfamily\fontsize{10.000000}{12.000000}\selectfont 1e9}%
\end{pgfscope}%
\begin{pgfscope}%
\pgfpathrectangle{\pgfqpoint{0.800000in}{0.528000in}}{\pgfqpoint{4.960000in}{3.696000in}}%
\pgfusepath{clip}%
\pgfsetrectcap%
\pgfsetroundjoin%
\pgfsetlinewidth{1.505625pt}%
\definecolor{currentstroke}{rgb}{0.000000,1.000000,0.498039}%
\pgfsetstrokecolor{currentstroke}%
\pgfsetdash{}{0pt}%
\pgfpathmoveto{\pgfqpoint{1.025455in}{0.696000in}}%
\pgfpathlineto{\pgfqpoint{1.029964in}{0.697566in}}%
\pgfpathlineto{\pgfqpoint{1.034473in}{0.701542in}}%
\pgfpathlineto{\pgfqpoint{1.038982in}{0.700412in}}%
\pgfpathlineto{\pgfqpoint{1.048000in}{0.700173in}}%
\pgfpathlineto{\pgfqpoint{1.052509in}{0.698066in}}%
\pgfpathlineto{\pgfqpoint{1.057018in}{0.698076in}}%
\pgfpathlineto{\pgfqpoint{1.061527in}{0.700157in}}%
\pgfpathlineto{\pgfqpoint{1.075055in}{0.700203in}}%
\pgfpathlineto{\pgfqpoint{1.079564in}{0.702397in}}%
\pgfpathlineto{\pgfqpoint{1.084073in}{0.700446in}}%
\pgfpathlineto{\pgfqpoint{1.093091in}{0.701750in}}%
\pgfpathlineto{\pgfqpoint{1.097600in}{0.704124in}}%
\pgfpathlineto{\pgfqpoint{1.102109in}{0.701175in}}%
\pgfpathlineto{\pgfqpoint{1.115636in}{0.705491in}}%
\pgfpathlineto{\pgfqpoint{1.120145in}{0.704023in}}%
\pgfpathlineto{\pgfqpoint{1.124655in}{0.704108in}}%
\pgfpathlineto{\pgfqpoint{1.129164in}{0.702945in}}%
\pgfpathlineto{\pgfqpoint{1.138182in}{0.708398in}}%
\pgfpathlineto{\pgfqpoint{1.142691in}{0.704362in}}%
\pgfpathlineto{\pgfqpoint{1.147200in}{0.708689in}}%
\pgfpathlineto{\pgfqpoint{1.151709in}{0.704915in}}%
\pgfpathlineto{\pgfqpoint{1.156218in}{0.708976in}}%
\pgfpathlineto{\pgfqpoint{1.160727in}{0.706645in}}%
\pgfpathlineto{\pgfqpoint{1.165236in}{0.708878in}}%
\pgfpathlineto{\pgfqpoint{1.169745in}{0.707357in}}%
\pgfpathlineto{\pgfqpoint{1.174255in}{0.708974in}}%
\pgfpathlineto{\pgfqpoint{1.178764in}{0.706599in}}%
\pgfpathlineto{\pgfqpoint{1.183273in}{0.708920in}}%
\pgfpathlineto{\pgfqpoint{1.187782in}{0.708205in}}%
\pgfpathlineto{\pgfqpoint{1.192291in}{0.710877in}}%
\pgfpathlineto{\pgfqpoint{1.196800in}{0.708540in}}%
\pgfpathlineto{\pgfqpoint{1.210327in}{0.711259in}}%
\pgfpathlineto{\pgfqpoint{1.223855in}{0.711056in}}%
\pgfpathlineto{\pgfqpoint{1.228364in}{0.724375in}}%
\pgfpathlineto{\pgfqpoint{1.232873in}{0.712126in}}%
\pgfpathlineto{\pgfqpoint{1.237382in}{0.713671in}}%
\pgfpathlineto{\pgfqpoint{1.246400in}{0.732322in}}%
\pgfpathlineto{\pgfqpoint{1.250909in}{0.717203in}}%
\pgfpathlineto{\pgfqpoint{1.255418in}{0.719366in}}%
\pgfpathlineto{\pgfqpoint{1.259927in}{0.715499in}}%
\pgfpathlineto{\pgfqpoint{1.264436in}{0.721154in}}%
\pgfpathlineto{\pgfqpoint{1.268945in}{0.723906in}}%
\pgfpathlineto{\pgfqpoint{1.273455in}{0.719521in}}%
\pgfpathlineto{\pgfqpoint{1.282473in}{0.719510in}}%
\pgfpathlineto{\pgfqpoint{1.291491in}{0.723057in}}%
\pgfpathlineto{\pgfqpoint{1.296000in}{0.721978in}}%
\pgfpathlineto{\pgfqpoint{1.300509in}{0.719060in}}%
\pgfpathlineto{\pgfqpoint{1.323055in}{0.720769in}}%
\pgfpathlineto{\pgfqpoint{1.327564in}{0.719750in}}%
\pgfpathlineto{\pgfqpoint{1.332073in}{0.722588in}}%
\pgfpathlineto{\pgfqpoint{1.336582in}{0.722773in}}%
\pgfpathlineto{\pgfqpoint{1.341091in}{0.743407in}}%
\pgfpathlineto{\pgfqpoint{1.345600in}{0.757248in}}%
\pgfpathlineto{\pgfqpoint{1.354618in}{0.742997in}}%
\pgfpathlineto{\pgfqpoint{1.359127in}{0.763478in}}%
\pgfpathlineto{\pgfqpoint{1.363636in}{0.773903in}}%
\pgfpathlineto{\pgfqpoint{1.368145in}{0.787855in}}%
\pgfpathlineto{\pgfqpoint{1.377164in}{0.753791in}}%
\pgfpathlineto{\pgfqpoint{1.381673in}{0.744549in}}%
\pgfpathlineto{\pgfqpoint{1.386182in}{0.751217in}}%
\pgfpathlineto{\pgfqpoint{1.390691in}{0.764048in}}%
\pgfpathlineto{\pgfqpoint{1.395200in}{0.785652in}}%
\pgfpathlineto{\pgfqpoint{1.399709in}{0.776846in}}%
\pgfpathlineto{\pgfqpoint{1.404218in}{0.773587in}}%
\pgfpathlineto{\pgfqpoint{1.408727in}{0.732651in}}%
\pgfpathlineto{\pgfqpoint{1.413236in}{0.740270in}}%
\pgfpathlineto{\pgfqpoint{1.417745in}{0.737784in}}%
\pgfpathlineto{\pgfqpoint{1.422255in}{0.731416in}}%
\pgfpathlineto{\pgfqpoint{1.431273in}{0.729921in}}%
\pgfpathlineto{\pgfqpoint{1.440291in}{0.733355in}}%
\pgfpathlineto{\pgfqpoint{1.453818in}{0.734471in}}%
\pgfpathlineto{\pgfqpoint{1.458327in}{0.734919in}}%
\pgfpathlineto{\pgfqpoint{1.462836in}{0.733943in}}%
\pgfpathlineto{\pgfqpoint{1.471855in}{0.735237in}}%
\pgfpathlineto{\pgfqpoint{1.476364in}{0.739175in}}%
\pgfpathlineto{\pgfqpoint{1.480873in}{0.739186in}}%
\pgfpathlineto{\pgfqpoint{1.485382in}{0.737748in}}%
\pgfpathlineto{\pgfqpoint{1.489891in}{0.738853in}}%
\pgfpathlineto{\pgfqpoint{1.494400in}{0.736511in}}%
\pgfpathlineto{\pgfqpoint{1.498909in}{0.741523in}}%
\pgfpathlineto{\pgfqpoint{1.503418in}{0.752293in}}%
\pgfpathlineto{\pgfqpoint{1.512436in}{0.841092in}}%
\pgfpathlineto{\pgfqpoint{1.516945in}{0.749151in}}%
\pgfpathlineto{\pgfqpoint{1.521455in}{0.740594in}}%
\pgfpathlineto{\pgfqpoint{1.525964in}{0.742634in}}%
\pgfpathlineto{\pgfqpoint{1.534982in}{0.742482in}}%
\pgfpathlineto{\pgfqpoint{1.539491in}{0.744005in}}%
\pgfpathlineto{\pgfqpoint{1.548509in}{0.758974in}}%
\pgfpathlineto{\pgfqpoint{1.553018in}{0.763762in}}%
\pgfpathlineto{\pgfqpoint{1.557527in}{0.752770in}}%
\pgfpathlineto{\pgfqpoint{1.562036in}{0.746004in}}%
\pgfpathlineto{\pgfqpoint{1.566545in}{0.748455in}}%
\pgfpathlineto{\pgfqpoint{1.571055in}{0.754268in}}%
\pgfpathlineto{\pgfqpoint{1.575564in}{0.757866in}}%
\pgfpathlineto{\pgfqpoint{1.580073in}{0.755150in}}%
\pgfpathlineto{\pgfqpoint{1.584582in}{0.756896in}}%
\pgfpathlineto{\pgfqpoint{1.589091in}{0.769203in}}%
\pgfpathlineto{\pgfqpoint{1.593600in}{0.817006in}}%
\pgfpathlineto{\pgfqpoint{1.598109in}{0.765312in}}%
\pgfpathlineto{\pgfqpoint{1.602618in}{0.820951in}}%
\pgfpathlineto{\pgfqpoint{1.607127in}{0.781578in}}%
\pgfpathlineto{\pgfqpoint{1.611636in}{0.803865in}}%
\pgfpathlineto{\pgfqpoint{1.616145in}{0.800209in}}%
\pgfpathlineto{\pgfqpoint{1.620655in}{0.856555in}}%
\pgfpathlineto{\pgfqpoint{1.625164in}{0.836287in}}%
\pgfpathlineto{\pgfqpoint{1.629673in}{0.849290in}}%
\pgfpathlineto{\pgfqpoint{1.634182in}{0.778492in}}%
\pgfpathlineto{\pgfqpoint{1.638691in}{0.756913in}}%
\pgfpathlineto{\pgfqpoint{1.643200in}{0.757096in}}%
\pgfpathlineto{\pgfqpoint{1.665745in}{0.766837in}}%
\pgfpathlineto{\pgfqpoint{1.670255in}{0.812534in}}%
\pgfpathlineto{\pgfqpoint{1.674764in}{0.845151in}}%
\pgfpathlineto{\pgfqpoint{1.679273in}{0.821130in}}%
\pgfpathlineto{\pgfqpoint{1.683782in}{0.822786in}}%
\pgfpathlineto{\pgfqpoint{1.688291in}{0.763674in}}%
\pgfpathlineto{\pgfqpoint{1.697309in}{0.774117in}}%
\pgfpathlineto{\pgfqpoint{1.701818in}{0.775088in}}%
\pgfpathlineto{\pgfqpoint{1.706327in}{0.791867in}}%
\pgfpathlineto{\pgfqpoint{1.710836in}{0.786345in}}%
\pgfpathlineto{\pgfqpoint{1.715345in}{0.800955in}}%
\pgfpathlineto{\pgfqpoint{1.719855in}{0.846843in}}%
\pgfpathlineto{\pgfqpoint{1.724364in}{0.769745in}}%
\pgfpathlineto{\pgfqpoint{1.728873in}{0.774270in}}%
\pgfpathlineto{\pgfqpoint{1.733382in}{0.800598in}}%
\pgfpathlineto{\pgfqpoint{1.737891in}{0.770630in}}%
\pgfpathlineto{\pgfqpoint{1.742400in}{0.780313in}}%
\pgfpathlineto{\pgfqpoint{1.746909in}{0.775589in}}%
\pgfpathlineto{\pgfqpoint{1.751418in}{0.779928in}}%
\pgfpathlineto{\pgfqpoint{1.755927in}{0.861218in}}%
\pgfpathlineto{\pgfqpoint{1.760436in}{0.788643in}}%
\pgfpathlineto{\pgfqpoint{1.764945in}{0.791400in}}%
\pgfpathlineto{\pgfqpoint{1.769455in}{0.796169in}}%
\pgfpathlineto{\pgfqpoint{1.773964in}{0.789462in}}%
\pgfpathlineto{\pgfqpoint{1.778473in}{0.792756in}}%
\pgfpathlineto{\pgfqpoint{1.787491in}{0.863268in}}%
\pgfpathlineto{\pgfqpoint{1.792000in}{0.832673in}}%
\pgfpathlineto{\pgfqpoint{1.796509in}{0.811520in}}%
\pgfpathlineto{\pgfqpoint{1.801018in}{0.827566in}}%
\pgfpathlineto{\pgfqpoint{1.805527in}{0.815275in}}%
\pgfpathlineto{\pgfqpoint{1.810036in}{0.789974in}}%
\pgfpathlineto{\pgfqpoint{1.814545in}{0.784976in}}%
\pgfpathlineto{\pgfqpoint{1.819055in}{0.797585in}}%
\pgfpathlineto{\pgfqpoint{1.823564in}{0.886497in}}%
\pgfpathlineto{\pgfqpoint{1.828073in}{0.795696in}}%
\pgfpathlineto{\pgfqpoint{1.832582in}{0.798558in}}%
\pgfpathlineto{\pgfqpoint{1.837091in}{0.828858in}}%
\pgfpathlineto{\pgfqpoint{1.841600in}{0.813082in}}%
\pgfpathlineto{\pgfqpoint{1.846109in}{0.787571in}}%
\pgfpathlineto{\pgfqpoint{1.850618in}{0.810881in}}%
\pgfpathlineto{\pgfqpoint{1.855127in}{0.918538in}}%
\pgfpathlineto{\pgfqpoint{1.859636in}{0.823379in}}%
\pgfpathlineto{\pgfqpoint{1.864145in}{0.790510in}}%
\pgfpathlineto{\pgfqpoint{1.873164in}{0.792566in}}%
\pgfpathlineto{\pgfqpoint{1.877673in}{0.791844in}}%
\pgfpathlineto{\pgfqpoint{1.882182in}{0.823574in}}%
\pgfpathlineto{\pgfqpoint{1.886691in}{0.901310in}}%
\pgfpathlineto{\pgfqpoint{1.891200in}{0.793727in}}%
\pgfpathlineto{\pgfqpoint{1.895709in}{0.794162in}}%
\pgfpathlineto{\pgfqpoint{1.900218in}{0.796627in}}%
\pgfpathlineto{\pgfqpoint{1.904727in}{0.949128in}}%
\pgfpathlineto{\pgfqpoint{1.909236in}{0.832421in}}%
\pgfpathlineto{\pgfqpoint{1.913745in}{0.805443in}}%
\pgfpathlineto{\pgfqpoint{1.922764in}{0.800342in}}%
\pgfpathlineto{\pgfqpoint{1.927273in}{0.803051in}}%
\pgfpathlineto{\pgfqpoint{1.931782in}{0.804433in}}%
\pgfpathlineto{\pgfqpoint{1.936291in}{0.808599in}}%
\pgfpathlineto{\pgfqpoint{1.940800in}{0.803202in}}%
\pgfpathlineto{\pgfqpoint{1.945309in}{0.808140in}}%
\pgfpathlineto{\pgfqpoint{1.949818in}{0.810444in}}%
\pgfpathlineto{\pgfqpoint{1.954327in}{0.816068in}}%
\pgfpathlineto{\pgfqpoint{1.958836in}{0.866581in}}%
\pgfpathlineto{\pgfqpoint{1.963345in}{0.818367in}}%
\pgfpathlineto{\pgfqpoint{1.967855in}{0.813787in}}%
\pgfpathlineto{\pgfqpoint{1.972364in}{0.817539in}}%
\pgfpathlineto{\pgfqpoint{1.981382in}{0.846885in}}%
\pgfpathlineto{\pgfqpoint{1.985891in}{0.829571in}}%
\pgfpathlineto{\pgfqpoint{1.999418in}{0.816362in}}%
\pgfpathlineto{\pgfqpoint{2.003927in}{0.823664in}}%
\pgfpathlineto{\pgfqpoint{2.008436in}{0.839632in}}%
\pgfpathlineto{\pgfqpoint{2.012945in}{0.818564in}}%
\pgfpathlineto{\pgfqpoint{2.017455in}{0.851562in}}%
\pgfpathlineto{\pgfqpoint{2.021964in}{0.821223in}}%
\pgfpathlineto{\pgfqpoint{2.026473in}{0.903042in}}%
\pgfpathlineto{\pgfqpoint{2.030982in}{0.865993in}}%
\pgfpathlineto{\pgfqpoint{2.040000in}{0.839438in}}%
\pgfpathlineto{\pgfqpoint{2.044509in}{0.948408in}}%
\pgfpathlineto{\pgfqpoint{2.049018in}{0.863512in}}%
\pgfpathlineto{\pgfqpoint{2.053527in}{0.916311in}}%
\pgfpathlineto{\pgfqpoint{2.058036in}{0.852712in}}%
\pgfpathlineto{\pgfqpoint{2.062545in}{0.871189in}}%
\pgfpathlineto{\pgfqpoint{2.067055in}{0.841712in}}%
\pgfpathlineto{\pgfqpoint{2.071564in}{0.838772in}}%
\pgfpathlineto{\pgfqpoint{2.076073in}{0.841482in}}%
\pgfpathlineto{\pgfqpoint{2.080582in}{0.859810in}}%
\pgfpathlineto{\pgfqpoint{2.085091in}{0.841573in}}%
\pgfpathlineto{\pgfqpoint{2.089600in}{0.853763in}}%
\pgfpathlineto{\pgfqpoint{2.094109in}{0.846727in}}%
\pgfpathlineto{\pgfqpoint{2.098618in}{0.851858in}}%
\pgfpathlineto{\pgfqpoint{2.103127in}{0.864722in}}%
\pgfpathlineto{\pgfqpoint{2.107636in}{0.851550in}}%
\pgfpathlineto{\pgfqpoint{2.112145in}{0.888770in}}%
\pgfpathlineto{\pgfqpoint{2.116655in}{0.866980in}}%
\pgfpathlineto{\pgfqpoint{2.121164in}{0.977159in}}%
\pgfpathlineto{\pgfqpoint{2.125673in}{0.952840in}}%
\pgfpathlineto{\pgfqpoint{2.130182in}{0.898078in}}%
\pgfpathlineto{\pgfqpoint{2.134691in}{1.027350in}}%
\pgfpathlineto{\pgfqpoint{2.139200in}{0.968968in}}%
\pgfpathlineto{\pgfqpoint{2.143709in}{0.979945in}}%
\pgfpathlineto{\pgfqpoint{2.148218in}{0.899682in}}%
\pgfpathlineto{\pgfqpoint{2.152727in}{0.842934in}}%
\pgfpathlineto{\pgfqpoint{2.157236in}{0.852285in}}%
\pgfpathlineto{\pgfqpoint{2.161745in}{0.880467in}}%
\pgfpathlineto{\pgfqpoint{2.166255in}{0.883298in}}%
\pgfpathlineto{\pgfqpoint{2.170764in}{1.051735in}}%
\pgfpathlineto{\pgfqpoint{2.175273in}{0.905386in}}%
\pgfpathlineto{\pgfqpoint{2.179782in}{1.029832in}}%
\pgfpathlineto{\pgfqpoint{2.184291in}{1.022856in}}%
\pgfpathlineto{\pgfqpoint{2.188800in}{0.876419in}}%
\pgfpathlineto{\pgfqpoint{2.193309in}{1.117471in}}%
\pgfpathlineto{\pgfqpoint{2.197818in}{1.066440in}}%
\pgfpathlineto{\pgfqpoint{2.202327in}{1.136896in}}%
\pgfpathlineto{\pgfqpoint{2.206836in}{0.976202in}}%
\pgfpathlineto{\pgfqpoint{2.211345in}{0.946888in}}%
\pgfpathlineto{\pgfqpoint{2.215855in}{0.981565in}}%
\pgfpathlineto{\pgfqpoint{2.220364in}{0.922790in}}%
\pgfpathlineto{\pgfqpoint{2.224873in}{0.894186in}}%
\pgfpathlineto{\pgfqpoint{2.229382in}{0.914776in}}%
\pgfpathlineto{\pgfqpoint{2.233891in}{0.958968in}}%
\pgfpathlineto{\pgfqpoint{2.238400in}{0.913116in}}%
\pgfpathlineto{\pgfqpoint{2.242909in}{0.896878in}}%
\pgfpathlineto{\pgfqpoint{2.247418in}{0.890736in}}%
\pgfpathlineto{\pgfqpoint{2.251927in}{0.899535in}}%
\pgfpathlineto{\pgfqpoint{2.256436in}{0.891822in}}%
\pgfpathlineto{\pgfqpoint{2.260945in}{0.917119in}}%
\pgfpathlineto{\pgfqpoint{2.265455in}{0.904254in}}%
\pgfpathlineto{\pgfqpoint{2.269964in}{0.906869in}}%
\pgfpathlineto{\pgfqpoint{2.274473in}{1.041415in}}%
\pgfpathlineto{\pgfqpoint{2.278982in}{0.916238in}}%
\pgfpathlineto{\pgfqpoint{2.283491in}{0.905234in}}%
\pgfpathlineto{\pgfqpoint{2.288000in}{0.910703in}}%
\pgfpathlineto{\pgfqpoint{2.292509in}{0.903726in}}%
\pgfpathlineto{\pgfqpoint{2.297018in}{0.913172in}}%
\pgfpathlineto{\pgfqpoint{2.301527in}{0.914313in}}%
\pgfpathlineto{\pgfqpoint{2.306036in}{0.907834in}}%
\pgfpathlineto{\pgfqpoint{2.310545in}{0.907054in}}%
\pgfpathlineto{\pgfqpoint{2.315055in}{0.908867in}}%
\pgfpathlineto{\pgfqpoint{2.319564in}{1.037816in}}%
\pgfpathlineto{\pgfqpoint{2.324073in}{0.913952in}}%
\pgfpathlineto{\pgfqpoint{2.337600in}{0.926626in}}%
\pgfpathlineto{\pgfqpoint{2.342109in}{0.924837in}}%
\pgfpathlineto{\pgfqpoint{2.346618in}{0.920576in}}%
\pgfpathlineto{\pgfqpoint{2.351127in}{0.918628in}}%
\pgfpathlineto{\pgfqpoint{2.355636in}{0.923097in}}%
\pgfpathlineto{\pgfqpoint{2.360145in}{0.936781in}}%
\pgfpathlineto{\pgfqpoint{2.364655in}{0.931155in}}%
\pgfpathlineto{\pgfqpoint{2.369164in}{0.928061in}}%
\pgfpathlineto{\pgfqpoint{2.373673in}{0.940386in}}%
\pgfpathlineto{\pgfqpoint{2.378182in}{0.933611in}}%
\pgfpathlineto{\pgfqpoint{2.382691in}{0.946508in}}%
\pgfpathlineto{\pgfqpoint{2.391709in}{0.928951in}}%
\pgfpathlineto{\pgfqpoint{2.396218in}{0.935730in}}%
\pgfpathlineto{\pgfqpoint{2.400727in}{0.938198in}}%
\pgfpathlineto{\pgfqpoint{2.405236in}{0.947944in}}%
\pgfpathlineto{\pgfqpoint{2.409745in}{0.950278in}}%
\pgfpathlineto{\pgfqpoint{2.414255in}{0.994521in}}%
\pgfpathlineto{\pgfqpoint{2.418764in}{0.945317in}}%
\pgfpathlineto{\pgfqpoint{2.423273in}{0.946532in}}%
\pgfpathlineto{\pgfqpoint{2.427782in}{0.941441in}}%
\pgfpathlineto{\pgfqpoint{2.432291in}{0.953594in}}%
\pgfpathlineto{\pgfqpoint{2.436800in}{0.953420in}}%
\pgfpathlineto{\pgfqpoint{2.441309in}{0.956917in}}%
\pgfpathlineto{\pgfqpoint{2.445818in}{0.976913in}}%
\pgfpathlineto{\pgfqpoint{2.450327in}{0.962521in}}%
\pgfpathlineto{\pgfqpoint{2.454836in}{0.961337in}}%
\pgfpathlineto{\pgfqpoint{2.459345in}{0.976053in}}%
\pgfpathlineto{\pgfqpoint{2.463855in}{0.967005in}}%
\pgfpathlineto{\pgfqpoint{2.468364in}{0.978211in}}%
\pgfpathlineto{\pgfqpoint{2.472873in}{0.968923in}}%
\pgfpathlineto{\pgfqpoint{2.477382in}{0.985855in}}%
\pgfpathlineto{\pgfqpoint{2.481891in}{0.960020in}}%
\pgfpathlineto{\pgfqpoint{2.486400in}{0.967229in}}%
\pgfpathlineto{\pgfqpoint{2.490909in}{1.001088in}}%
\pgfpathlineto{\pgfqpoint{2.495418in}{0.975349in}}%
\pgfpathlineto{\pgfqpoint{2.499927in}{1.000626in}}%
\pgfpathlineto{\pgfqpoint{2.504436in}{0.983478in}}%
\pgfpathlineto{\pgfqpoint{2.508945in}{0.978094in}}%
\pgfpathlineto{\pgfqpoint{2.513455in}{0.979219in}}%
\pgfpathlineto{\pgfqpoint{2.517964in}{0.975190in}}%
\pgfpathlineto{\pgfqpoint{2.522473in}{0.984960in}}%
\pgfpathlineto{\pgfqpoint{2.526982in}{0.968631in}}%
\pgfpathlineto{\pgfqpoint{2.536000in}{0.998933in}}%
\pgfpathlineto{\pgfqpoint{2.540509in}{0.984612in}}%
\pgfpathlineto{\pgfqpoint{2.545018in}{1.003433in}}%
\pgfpathlineto{\pgfqpoint{2.549527in}{0.982970in}}%
\pgfpathlineto{\pgfqpoint{2.554036in}{0.996159in}}%
\pgfpathlineto{\pgfqpoint{2.558545in}{0.992468in}}%
\pgfpathlineto{\pgfqpoint{2.563055in}{0.994938in}}%
\pgfpathlineto{\pgfqpoint{2.567564in}{0.988043in}}%
\pgfpathlineto{\pgfqpoint{2.572073in}{0.984145in}}%
\pgfpathlineto{\pgfqpoint{2.576582in}{0.997734in}}%
\pgfpathlineto{\pgfqpoint{2.581091in}{0.993898in}}%
\pgfpathlineto{\pgfqpoint{2.585600in}{0.996891in}}%
\pgfpathlineto{\pgfqpoint{2.590109in}{0.992830in}}%
\pgfpathlineto{\pgfqpoint{2.594618in}{1.018225in}}%
\pgfpathlineto{\pgfqpoint{2.599127in}{0.998931in}}%
\pgfpathlineto{\pgfqpoint{2.603636in}{0.999058in}}%
\pgfpathlineto{\pgfqpoint{2.608145in}{1.010562in}}%
\pgfpathlineto{\pgfqpoint{2.612655in}{1.006531in}}%
\pgfpathlineto{\pgfqpoint{2.617164in}{1.011808in}}%
\pgfpathlineto{\pgfqpoint{2.621673in}{1.005637in}}%
\pgfpathlineto{\pgfqpoint{2.626182in}{1.013554in}}%
\pgfpathlineto{\pgfqpoint{2.630691in}{1.017903in}}%
\pgfpathlineto{\pgfqpoint{2.635200in}{1.017486in}}%
\pgfpathlineto{\pgfqpoint{2.639709in}{1.014084in}}%
\pgfpathlineto{\pgfqpoint{2.644218in}{1.017029in}}%
\pgfpathlineto{\pgfqpoint{2.648727in}{1.027428in}}%
\pgfpathlineto{\pgfqpoint{2.653236in}{1.018943in}}%
\pgfpathlineto{\pgfqpoint{2.657745in}{1.019565in}}%
\pgfpathlineto{\pgfqpoint{2.662255in}{1.027678in}}%
\pgfpathlineto{\pgfqpoint{2.666764in}{1.018198in}}%
\pgfpathlineto{\pgfqpoint{2.671273in}{1.020016in}}%
\pgfpathlineto{\pgfqpoint{2.675782in}{1.026150in}}%
\pgfpathlineto{\pgfqpoint{2.680291in}{1.047466in}}%
\pgfpathlineto{\pgfqpoint{2.684800in}{1.025668in}}%
\pgfpathlineto{\pgfqpoint{2.689309in}{1.052118in}}%
\pgfpathlineto{\pgfqpoint{2.693818in}{1.039083in}}%
\pgfpathlineto{\pgfqpoint{2.698327in}{1.043414in}}%
\pgfpathlineto{\pgfqpoint{2.702836in}{1.051287in}}%
\pgfpathlineto{\pgfqpoint{2.707345in}{1.053282in}}%
\pgfpathlineto{\pgfqpoint{2.711855in}{1.051667in}}%
\pgfpathlineto{\pgfqpoint{2.716364in}{1.053441in}}%
\pgfpathlineto{\pgfqpoint{2.720873in}{1.050758in}}%
\pgfpathlineto{\pgfqpoint{2.725382in}{1.085903in}}%
\pgfpathlineto{\pgfqpoint{2.729891in}{1.052347in}}%
\pgfpathlineto{\pgfqpoint{2.734400in}{1.065952in}}%
\pgfpathlineto{\pgfqpoint{2.738909in}{1.057040in}}%
\pgfpathlineto{\pgfqpoint{2.743418in}{1.056829in}}%
\pgfpathlineto{\pgfqpoint{2.747927in}{1.060969in}}%
\pgfpathlineto{\pgfqpoint{2.752436in}{1.077575in}}%
\pgfpathlineto{\pgfqpoint{2.756945in}{1.063624in}}%
\pgfpathlineto{\pgfqpoint{2.765964in}{1.069983in}}%
\pgfpathlineto{\pgfqpoint{2.770473in}{1.063070in}}%
\pgfpathlineto{\pgfqpoint{2.774982in}{1.068172in}}%
\pgfpathlineto{\pgfqpoint{2.779491in}{1.063239in}}%
\pgfpathlineto{\pgfqpoint{2.788509in}{1.079286in}}%
\pgfpathlineto{\pgfqpoint{2.793018in}{1.072461in}}%
\pgfpathlineto{\pgfqpoint{2.797527in}{1.071910in}}%
\pgfpathlineto{\pgfqpoint{2.802036in}{1.073608in}}%
\pgfpathlineto{\pgfqpoint{2.806545in}{1.089958in}}%
\pgfpathlineto{\pgfqpoint{2.811055in}{1.091526in}}%
\pgfpathlineto{\pgfqpoint{2.815564in}{1.082056in}}%
\pgfpathlineto{\pgfqpoint{2.820073in}{1.103146in}}%
\pgfpathlineto{\pgfqpoint{2.824582in}{1.084337in}}%
\pgfpathlineto{\pgfqpoint{2.829091in}{1.077595in}}%
\pgfpathlineto{\pgfqpoint{2.833600in}{1.090600in}}%
\pgfpathlineto{\pgfqpoint{2.838109in}{1.093781in}}%
\pgfpathlineto{\pgfqpoint{2.842618in}{1.114388in}}%
\pgfpathlineto{\pgfqpoint{2.847127in}{1.101475in}}%
\pgfpathlineto{\pgfqpoint{2.851636in}{1.098483in}}%
\pgfpathlineto{\pgfqpoint{2.856145in}{1.090621in}}%
\pgfpathlineto{\pgfqpoint{2.860655in}{1.096819in}}%
\pgfpathlineto{\pgfqpoint{2.865164in}{1.093286in}}%
\pgfpathlineto{\pgfqpoint{2.869673in}{1.100213in}}%
\pgfpathlineto{\pgfqpoint{2.874182in}{1.125384in}}%
\pgfpathlineto{\pgfqpoint{2.878691in}{1.201171in}}%
\pgfpathlineto{\pgfqpoint{2.883200in}{1.134942in}}%
\pgfpathlineto{\pgfqpoint{2.887709in}{1.113870in}}%
\pgfpathlineto{\pgfqpoint{2.892218in}{1.136139in}}%
\pgfpathlineto{\pgfqpoint{2.896727in}{1.140713in}}%
\pgfpathlineto{\pgfqpoint{2.905745in}{1.167548in}}%
\pgfpathlineto{\pgfqpoint{2.910255in}{1.160362in}}%
\pgfpathlineto{\pgfqpoint{2.919273in}{1.191902in}}%
\pgfpathlineto{\pgfqpoint{2.923782in}{1.182396in}}%
\pgfpathlineto{\pgfqpoint{2.928291in}{1.144052in}}%
\pgfpathlineto{\pgfqpoint{2.932800in}{1.156841in}}%
\pgfpathlineto{\pgfqpoint{2.937309in}{1.106327in}}%
\pgfpathlineto{\pgfqpoint{2.941818in}{1.136741in}}%
\pgfpathlineto{\pgfqpoint{2.946327in}{1.158063in}}%
\pgfpathlineto{\pgfqpoint{2.950836in}{1.134144in}}%
\pgfpathlineto{\pgfqpoint{2.955345in}{1.132341in}}%
\pgfpathlineto{\pgfqpoint{2.959855in}{1.142508in}}%
\pgfpathlineto{\pgfqpoint{2.964364in}{1.145295in}}%
\pgfpathlineto{\pgfqpoint{2.973382in}{1.147979in}}%
\pgfpathlineto{\pgfqpoint{2.982400in}{1.163408in}}%
\pgfpathlineto{\pgfqpoint{2.986909in}{1.157349in}}%
\pgfpathlineto{\pgfqpoint{2.991418in}{1.156856in}}%
\pgfpathlineto{\pgfqpoint{2.995927in}{1.176493in}}%
\pgfpathlineto{\pgfqpoint{3.000436in}{1.145089in}}%
\pgfpathlineto{\pgfqpoint{3.004945in}{1.163557in}}%
\pgfpathlineto{\pgfqpoint{3.009455in}{1.173233in}}%
\pgfpathlineto{\pgfqpoint{3.013964in}{1.168027in}}%
\pgfpathlineto{\pgfqpoint{3.018473in}{1.176842in}}%
\pgfpathlineto{\pgfqpoint{3.022982in}{1.193478in}}%
\pgfpathlineto{\pgfqpoint{3.027491in}{1.151937in}}%
\pgfpathlineto{\pgfqpoint{3.032000in}{1.178306in}}%
\pgfpathlineto{\pgfqpoint{3.036509in}{1.171717in}}%
\pgfpathlineto{\pgfqpoint{3.041018in}{1.186583in}}%
\pgfpathlineto{\pgfqpoint{3.045527in}{1.221051in}}%
\pgfpathlineto{\pgfqpoint{3.050036in}{1.183795in}}%
\pgfpathlineto{\pgfqpoint{3.054545in}{1.178725in}}%
\pgfpathlineto{\pgfqpoint{3.059055in}{1.192141in}}%
\pgfpathlineto{\pgfqpoint{3.063564in}{1.220677in}}%
\pgfpathlineto{\pgfqpoint{3.068073in}{1.224467in}}%
\pgfpathlineto{\pgfqpoint{3.072582in}{1.205267in}}%
\pgfpathlineto{\pgfqpoint{3.077091in}{1.201064in}}%
\pgfpathlineto{\pgfqpoint{3.081600in}{1.214127in}}%
\pgfpathlineto{\pgfqpoint{3.086109in}{1.206132in}}%
\pgfpathlineto{\pgfqpoint{3.090618in}{1.225116in}}%
\pgfpathlineto{\pgfqpoint{3.099636in}{1.188442in}}%
\pgfpathlineto{\pgfqpoint{3.104145in}{1.182486in}}%
\pgfpathlineto{\pgfqpoint{3.108655in}{1.238911in}}%
\pgfpathlineto{\pgfqpoint{3.113164in}{1.216972in}}%
\pgfpathlineto{\pgfqpoint{3.117673in}{1.225865in}}%
\pgfpathlineto{\pgfqpoint{3.122182in}{1.211802in}}%
\pgfpathlineto{\pgfqpoint{3.126691in}{1.212700in}}%
\pgfpathlineto{\pgfqpoint{3.131200in}{1.257530in}}%
\pgfpathlineto{\pgfqpoint{3.135709in}{1.273988in}}%
\pgfpathlineto{\pgfqpoint{3.140218in}{1.224220in}}%
\pgfpathlineto{\pgfqpoint{3.144727in}{1.220664in}}%
\pgfpathlineto{\pgfqpoint{3.149236in}{1.219736in}}%
\pgfpathlineto{\pgfqpoint{3.153745in}{1.239356in}}%
\pgfpathlineto{\pgfqpoint{3.158255in}{1.228128in}}%
\pgfpathlineto{\pgfqpoint{3.162764in}{1.263125in}}%
\pgfpathlineto{\pgfqpoint{3.167273in}{1.242329in}}%
\pgfpathlineto{\pgfqpoint{3.171782in}{1.251356in}}%
\pgfpathlineto{\pgfqpoint{3.176291in}{1.230608in}}%
\pgfpathlineto{\pgfqpoint{3.180800in}{1.235939in}}%
\pgfpathlineto{\pgfqpoint{3.185309in}{1.229685in}}%
\pgfpathlineto{\pgfqpoint{3.189818in}{1.262220in}}%
\pgfpathlineto{\pgfqpoint{3.194327in}{1.255072in}}%
\pgfpathlineto{\pgfqpoint{3.198836in}{1.244173in}}%
\pgfpathlineto{\pgfqpoint{3.207855in}{1.230326in}}%
\pgfpathlineto{\pgfqpoint{3.212364in}{1.275486in}}%
\pgfpathlineto{\pgfqpoint{3.216873in}{1.258262in}}%
\pgfpathlineto{\pgfqpoint{3.221382in}{1.282209in}}%
\pgfpathlineto{\pgfqpoint{3.225891in}{1.248063in}}%
\pgfpathlineto{\pgfqpoint{3.239418in}{1.285549in}}%
\pgfpathlineto{\pgfqpoint{3.243927in}{1.265984in}}%
\pgfpathlineto{\pgfqpoint{3.248436in}{1.293931in}}%
\pgfpathlineto{\pgfqpoint{3.252945in}{1.286184in}}%
\pgfpathlineto{\pgfqpoint{3.257455in}{1.269722in}}%
\pgfpathlineto{\pgfqpoint{3.261964in}{1.307464in}}%
\pgfpathlineto{\pgfqpoint{3.266473in}{1.286131in}}%
\pgfpathlineto{\pgfqpoint{3.270982in}{1.318293in}}%
\pgfpathlineto{\pgfqpoint{3.275491in}{1.286543in}}%
\pgfpathlineto{\pgfqpoint{3.280000in}{1.282253in}}%
\pgfpathlineto{\pgfqpoint{3.284509in}{1.300167in}}%
\pgfpathlineto{\pgfqpoint{3.289018in}{1.333540in}}%
\pgfpathlineto{\pgfqpoint{3.293527in}{1.318730in}}%
\pgfpathlineto{\pgfqpoint{3.298036in}{1.295669in}}%
\pgfpathlineto{\pgfqpoint{3.302545in}{1.333554in}}%
\pgfpathlineto{\pgfqpoint{3.307055in}{1.309924in}}%
\pgfpathlineto{\pgfqpoint{3.311564in}{1.339264in}}%
\pgfpathlineto{\pgfqpoint{3.316073in}{1.301520in}}%
\pgfpathlineto{\pgfqpoint{3.320582in}{1.382755in}}%
\pgfpathlineto{\pgfqpoint{3.325091in}{1.325099in}}%
\pgfpathlineto{\pgfqpoint{3.329600in}{1.304389in}}%
\pgfpathlineto{\pgfqpoint{3.334109in}{1.357819in}}%
\pgfpathlineto{\pgfqpoint{3.338618in}{1.314976in}}%
\pgfpathlineto{\pgfqpoint{3.343127in}{1.300732in}}%
\pgfpathlineto{\pgfqpoint{3.347636in}{1.330834in}}%
\pgfpathlineto{\pgfqpoint{3.352145in}{1.328193in}}%
\pgfpathlineto{\pgfqpoint{3.356655in}{1.342277in}}%
\pgfpathlineto{\pgfqpoint{3.361164in}{1.325143in}}%
\pgfpathlineto{\pgfqpoint{3.365673in}{1.344678in}}%
\pgfpathlineto{\pgfqpoint{3.370182in}{1.336322in}}%
\pgfpathlineto{\pgfqpoint{3.374691in}{1.314516in}}%
\pgfpathlineto{\pgfqpoint{3.379200in}{1.368457in}}%
\pgfpathlineto{\pgfqpoint{3.383709in}{1.338096in}}%
\pgfpathlineto{\pgfqpoint{3.388218in}{1.339861in}}%
\pgfpathlineto{\pgfqpoint{3.392727in}{1.356740in}}%
\pgfpathlineto{\pgfqpoint{3.397236in}{1.361076in}}%
\pgfpathlineto{\pgfqpoint{3.401745in}{1.367754in}}%
\pgfpathlineto{\pgfqpoint{3.406255in}{1.347563in}}%
\pgfpathlineto{\pgfqpoint{3.410764in}{1.348117in}}%
\pgfpathlineto{\pgfqpoint{3.415273in}{1.396950in}}%
\pgfpathlineto{\pgfqpoint{3.419782in}{1.353029in}}%
\pgfpathlineto{\pgfqpoint{3.424291in}{1.362832in}}%
\pgfpathlineto{\pgfqpoint{3.428800in}{1.392930in}}%
\pgfpathlineto{\pgfqpoint{3.433309in}{1.407472in}}%
\pgfpathlineto{\pgfqpoint{3.437818in}{1.385152in}}%
\pgfpathlineto{\pgfqpoint{3.442327in}{1.396578in}}%
\pgfpathlineto{\pgfqpoint{3.446836in}{1.372336in}}%
\pgfpathlineto{\pgfqpoint{3.451345in}{1.439640in}}%
\pgfpathlineto{\pgfqpoint{3.460364in}{1.382197in}}%
\pgfpathlineto{\pgfqpoint{3.464873in}{1.400558in}}%
\pgfpathlineto{\pgfqpoint{3.469382in}{1.377452in}}%
\pgfpathlineto{\pgfqpoint{3.473891in}{1.408133in}}%
\pgfpathlineto{\pgfqpoint{3.478400in}{1.412163in}}%
\pgfpathlineto{\pgfqpoint{3.482909in}{1.409744in}}%
\pgfpathlineto{\pgfqpoint{3.487418in}{1.403202in}}%
\pgfpathlineto{\pgfqpoint{3.491927in}{1.458422in}}%
\pgfpathlineto{\pgfqpoint{3.496436in}{1.412610in}}%
\pgfpathlineto{\pgfqpoint{3.500945in}{1.392249in}}%
\pgfpathlineto{\pgfqpoint{3.505455in}{1.442126in}}%
\pgfpathlineto{\pgfqpoint{3.509964in}{1.417164in}}%
\pgfpathlineto{\pgfqpoint{3.514473in}{1.420539in}}%
\pgfpathlineto{\pgfqpoint{3.518982in}{1.450436in}}%
\pgfpathlineto{\pgfqpoint{3.523491in}{1.439570in}}%
\pgfpathlineto{\pgfqpoint{3.528000in}{1.437260in}}%
\pgfpathlineto{\pgfqpoint{3.532509in}{1.451632in}}%
\pgfpathlineto{\pgfqpoint{3.537018in}{1.457015in}}%
\pgfpathlineto{\pgfqpoint{3.546036in}{1.441916in}}%
\pgfpathlineto{\pgfqpoint{3.550545in}{1.438260in}}%
\pgfpathlineto{\pgfqpoint{3.559564in}{1.497778in}}%
\pgfpathlineto{\pgfqpoint{3.564073in}{1.447244in}}%
\pgfpathlineto{\pgfqpoint{3.568582in}{1.455588in}}%
\pgfpathlineto{\pgfqpoint{3.573091in}{1.457248in}}%
\pgfpathlineto{\pgfqpoint{3.577600in}{1.438785in}}%
\pgfpathlineto{\pgfqpoint{3.582109in}{1.487376in}}%
\pgfpathlineto{\pgfqpoint{3.586618in}{1.457841in}}%
\pgfpathlineto{\pgfqpoint{3.591127in}{1.479053in}}%
\pgfpathlineto{\pgfqpoint{3.595636in}{1.467317in}}%
\pgfpathlineto{\pgfqpoint{3.600145in}{1.482095in}}%
\pgfpathlineto{\pgfqpoint{3.604655in}{1.501126in}}%
\pgfpathlineto{\pgfqpoint{3.609164in}{1.491122in}}%
\pgfpathlineto{\pgfqpoint{3.613673in}{1.477531in}}%
\pgfpathlineto{\pgfqpoint{3.618182in}{1.483793in}}%
\pgfpathlineto{\pgfqpoint{3.622691in}{1.470855in}}%
\pgfpathlineto{\pgfqpoint{3.631709in}{1.488427in}}%
\pgfpathlineto{\pgfqpoint{3.636218in}{1.522654in}}%
\pgfpathlineto{\pgfqpoint{3.640727in}{1.494096in}}%
\pgfpathlineto{\pgfqpoint{3.645236in}{1.521311in}}%
\pgfpathlineto{\pgfqpoint{3.649745in}{1.481982in}}%
\pgfpathlineto{\pgfqpoint{3.654255in}{1.492692in}}%
\pgfpathlineto{\pgfqpoint{3.658764in}{1.544684in}}%
\pgfpathlineto{\pgfqpoint{3.663273in}{1.480371in}}%
\pgfpathlineto{\pgfqpoint{3.667782in}{1.496393in}}%
\pgfpathlineto{\pgfqpoint{3.676800in}{1.539004in}}%
\pgfpathlineto{\pgfqpoint{3.681309in}{1.551556in}}%
\pgfpathlineto{\pgfqpoint{3.685818in}{1.540745in}}%
\pgfpathlineto{\pgfqpoint{3.690327in}{1.522701in}}%
\pgfpathlineto{\pgfqpoint{3.694836in}{1.530025in}}%
\pgfpathlineto{\pgfqpoint{3.699345in}{1.573934in}}%
\pgfpathlineto{\pgfqpoint{3.703855in}{1.548095in}}%
\pgfpathlineto{\pgfqpoint{3.708364in}{1.503623in}}%
\pgfpathlineto{\pgfqpoint{3.717382in}{1.566491in}}%
\pgfpathlineto{\pgfqpoint{3.721891in}{1.556068in}}%
\pgfpathlineto{\pgfqpoint{3.726400in}{1.565276in}}%
\pgfpathlineto{\pgfqpoint{3.730909in}{1.595108in}}%
\pgfpathlineto{\pgfqpoint{3.735418in}{1.528937in}}%
\pgfpathlineto{\pgfqpoint{3.739927in}{1.555902in}}%
\pgfpathlineto{\pgfqpoint{3.744436in}{1.556096in}}%
\pgfpathlineto{\pgfqpoint{3.748945in}{1.529990in}}%
\pgfpathlineto{\pgfqpoint{3.753455in}{1.546026in}}%
\pgfpathlineto{\pgfqpoint{3.757964in}{1.574634in}}%
\pgfpathlineto{\pgfqpoint{3.762473in}{1.582992in}}%
\pgfpathlineto{\pgfqpoint{3.766982in}{1.585799in}}%
\pgfpathlineto{\pgfqpoint{3.771491in}{1.570225in}}%
\pgfpathlineto{\pgfqpoint{3.776000in}{1.584544in}}%
\pgfpathlineto{\pgfqpoint{3.780509in}{1.580411in}}%
\pgfpathlineto{\pgfqpoint{3.785018in}{1.564199in}}%
\pgfpathlineto{\pgfqpoint{3.789527in}{1.679903in}}%
\pgfpathlineto{\pgfqpoint{3.794036in}{1.742873in}}%
\pgfpathlineto{\pgfqpoint{3.798545in}{1.899836in}}%
\pgfpathlineto{\pgfqpoint{3.803055in}{2.290542in}}%
\pgfpathlineto{\pgfqpoint{3.807564in}{1.700081in}}%
\pgfpathlineto{\pgfqpoint{3.812073in}{1.741923in}}%
\pgfpathlineto{\pgfqpoint{3.816582in}{2.008431in}}%
\pgfpathlineto{\pgfqpoint{3.821091in}{1.633767in}}%
\pgfpathlineto{\pgfqpoint{3.825600in}{1.558138in}}%
\pgfpathlineto{\pgfqpoint{3.830109in}{1.564981in}}%
\pgfpathlineto{\pgfqpoint{3.834618in}{1.588324in}}%
\pgfpathlineto{\pgfqpoint{3.839127in}{1.582200in}}%
\pgfpathlineto{\pgfqpoint{3.843636in}{1.958040in}}%
\pgfpathlineto{\pgfqpoint{3.848145in}{1.734116in}}%
\pgfpathlineto{\pgfqpoint{3.852655in}{1.649652in}}%
\pgfpathlineto{\pgfqpoint{3.861673in}{1.597015in}}%
\pgfpathlineto{\pgfqpoint{3.866182in}{1.648526in}}%
\pgfpathlineto{\pgfqpoint{3.870691in}{1.620897in}}%
\pgfpathlineto{\pgfqpoint{3.879709in}{1.616302in}}%
\pgfpathlineto{\pgfqpoint{3.884218in}{1.607459in}}%
\pgfpathlineto{\pgfqpoint{3.888727in}{1.675881in}}%
\pgfpathlineto{\pgfqpoint{3.893236in}{1.649116in}}%
\pgfpathlineto{\pgfqpoint{3.897745in}{1.652761in}}%
\pgfpathlineto{\pgfqpoint{3.902255in}{1.634530in}}%
\pgfpathlineto{\pgfqpoint{3.906764in}{1.673723in}}%
\pgfpathlineto{\pgfqpoint{3.911273in}{1.649178in}}%
\pgfpathlineto{\pgfqpoint{3.915782in}{1.701265in}}%
\pgfpathlineto{\pgfqpoint{3.920291in}{1.709547in}}%
\pgfpathlineto{\pgfqpoint{3.924800in}{1.645805in}}%
\pgfpathlineto{\pgfqpoint{3.929309in}{1.661514in}}%
\pgfpathlineto{\pgfqpoint{3.933818in}{1.672129in}}%
\pgfpathlineto{\pgfqpoint{3.938327in}{1.676638in}}%
\pgfpathlineto{\pgfqpoint{3.947345in}{1.646333in}}%
\pgfpathlineto{\pgfqpoint{3.951855in}{1.643900in}}%
\pgfpathlineto{\pgfqpoint{3.956364in}{1.669011in}}%
\pgfpathlineto{\pgfqpoint{3.960873in}{1.667089in}}%
\pgfpathlineto{\pgfqpoint{3.965382in}{1.695023in}}%
\pgfpathlineto{\pgfqpoint{3.969891in}{1.701376in}}%
\pgfpathlineto{\pgfqpoint{3.974400in}{1.683119in}}%
\pgfpathlineto{\pgfqpoint{3.978909in}{1.719760in}}%
\pgfpathlineto{\pgfqpoint{3.983418in}{1.696002in}}%
\pgfpathlineto{\pgfqpoint{3.987927in}{1.730077in}}%
\pgfpathlineto{\pgfqpoint{3.992436in}{1.724061in}}%
\pgfpathlineto{\pgfqpoint{3.996945in}{1.695224in}}%
\pgfpathlineto{\pgfqpoint{4.001455in}{1.724501in}}%
\pgfpathlineto{\pgfqpoint{4.005964in}{1.693104in}}%
\pgfpathlineto{\pgfqpoint{4.010473in}{1.676685in}}%
\pgfpathlineto{\pgfqpoint{4.014982in}{1.674113in}}%
\pgfpathlineto{\pgfqpoint{4.019491in}{1.704288in}}%
\pgfpathlineto{\pgfqpoint{4.024000in}{1.696008in}}%
\pgfpathlineto{\pgfqpoint{4.028509in}{1.699291in}}%
\pgfpathlineto{\pgfqpoint{4.033018in}{1.694597in}}%
\pgfpathlineto{\pgfqpoint{4.037527in}{1.725154in}}%
\pgfpathlineto{\pgfqpoint{4.042036in}{1.720024in}}%
\pgfpathlineto{\pgfqpoint{4.046545in}{1.732418in}}%
\pgfpathlineto{\pgfqpoint{4.051055in}{1.739345in}}%
\pgfpathlineto{\pgfqpoint{4.055564in}{1.732136in}}%
\pgfpathlineto{\pgfqpoint{4.060073in}{1.720405in}}%
\pgfpathlineto{\pgfqpoint{4.064582in}{1.737850in}}%
\pgfpathlineto{\pgfqpoint{4.069091in}{1.747165in}}%
\pgfpathlineto{\pgfqpoint{4.073600in}{1.723717in}}%
\pgfpathlineto{\pgfqpoint{4.078109in}{1.775182in}}%
\pgfpathlineto{\pgfqpoint{4.082618in}{1.793402in}}%
\pgfpathlineto{\pgfqpoint{4.087127in}{1.734354in}}%
\pgfpathlineto{\pgfqpoint{4.091636in}{1.775798in}}%
\pgfpathlineto{\pgfqpoint{4.096145in}{1.746140in}}%
\pgfpathlineto{\pgfqpoint{4.100655in}{1.783973in}}%
\pgfpathlineto{\pgfqpoint{4.105164in}{1.776275in}}%
\pgfpathlineto{\pgfqpoint{4.109673in}{1.785458in}}%
\pgfpathlineto{\pgfqpoint{4.114182in}{1.776198in}}%
\pgfpathlineto{\pgfqpoint{4.118691in}{1.826881in}}%
\pgfpathlineto{\pgfqpoint{4.123200in}{1.775434in}}%
\pgfpathlineto{\pgfqpoint{4.127709in}{1.815022in}}%
\pgfpathlineto{\pgfqpoint{4.132218in}{1.737616in}}%
\pgfpathlineto{\pgfqpoint{4.136727in}{1.838145in}}%
\pgfpathlineto{\pgfqpoint{4.141236in}{1.782613in}}%
\pgfpathlineto{\pgfqpoint{4.145745in}{1.853336in}}%
\pgfpathlineto{\pgfqpoint{4.150255in}{1.760552in}}%
\pgfpathlineto{\pgfqpoint{4.154764in}{1.768310in}}%
\pgfpathlineto{\pgfqpoint{4.159273in}{1.843914in}}%
\pgfpathlineto{\pgfqpoint{4.163782in}{1.793585in}}%
\pgfpathlineto{\pgfqpoint{4.168291in}{1.798796in}}%
\pgfpathlineto{\pgfqpoint{4.172800in}{1.819032in}}%
\pgfpathlineto{\pgfqpoint{4.177309in}{1.789414in}}%
\pgfpathlineto{\pgfqpoint{4.181818in}{1.810361in}}%
\pgfpathlineto{\pgfqpoint{4.195345in}{1.841306in}}%
\pgfpathlineto{\pgfqpoint{4.199855in}{1.782707in}}%
\pgfpathlineto{\pgfqpoint{4.204364in}{1.819033in}}%
\pgfpathlineto{\pgfqpoint{4.208873in}{1.806734in}}%
\pgfpathlineto{\pgfqpoint{4.213382in}{1.831219in}}%
\pgfpathlineto{\pgfqpoint{4.217891in}{1.845196in}}%
\pgfpathlineto{\pgfqpoint{4.222400in}{1.902992in}}%
\pgfpathlineto{\pgfqpoint{4.226909in}{1.827982in}}%
\pgfpathlineto{\pgfqpoint{4.231418in}{1.846407in}}%
\pgfpathlineto{\pgfqpoint{4.235927in}{1.819336in}}%
\pgfpathlineto{\pgfqpoint{4.240436in}{1.871941in}}%
\pgfpathlineto{\pgfqpoint{4.244945in}{1.858016in}}%
\pgfpathlineto{\pgfqpoint{4.249455in}{1.874390in}}%
\pgfpathlineto{\pgfqpoint{4.253964in}{1.942276in}}%
\pgfpathlineto{\pgfqpoint{4.258473in}{1.878138in}}%
\pgfpathlineto{\pgfqpoint{4.262982in}{1.841673in}}%
\pgfpathlineto{\pgfqpoint{4.267491in}{1.857022in}}%
\pgfpathlineto{\pgfqpoint{4.272000in}{1.859924in}}%
\pgfpathlineto{\pgfqpoint{4.276509in}{1.919056in}}%
\pgfpathlineto{\pgfqpoint{4.281018in}{1.851558in}}%
\pgfpathlineto{\pgfqpoint{4.285527in}{1.889795in}}%
\pgfpathlineto{\pgfqpoint{4.294545in}{1.874078in}}%
\pgfpathlineto{\pgfqpoint{4.299055in}{1.886549in}}%
\pgfpathlineto{\pgfqpoint{4.303564in}{1.930775in}}%
\pgfpathlineto{\pgfqpoint{4.308073in}{1.958167in}}%
\pgfpathlineto{\pgfqpoint{4.312582in}{1.869724in}}%
\pgfpathlineto{\pgfqpoint{4.321600in}{1.925044in}}%
\pgfpathlineto{\pgfqpoint{4.326109in}{1.956834in}}%
\pgfpathlineto{\pgfqpoint{4.330618in}{1.903574in}}%
\pgfpathlineto{\pgfqpoint{4.335127in}{1.899535in}}%
\pgfpathlineto{\pgfqpoint{4.339636in}{1.921515in}}%
\pgfpathlineto{\pgfqpoint{4.344145in}{1.865154in}}%
\pgfpathlineto{\pgfqpoint{4.348655in}{2.009418in}}%
\pgfpathlineto{\pgfqpoint{4.353164in}{1.937597in}}%
\pgfpathlineto{\pgfqpoint{4.357673in}{1.912775in}}%
\pgfpathlineto{\pgfqpoint{4.366691in}{1.921452in}}%
\pgfpathlineto{\pgfqpoint{4.371200in}{1.935401in}}%
\pgfpathlineto{\pgfqpoint{4.375709in}{1.923804in}}%
\pgfpathlineto{\pgfqpoint{4.380218in}{1.962293in}}%
\pgfpathlineto{\pgfqpoint{4.384727in}{1.937121in}}%
\pgfpathlineto{\pgfqpoint{4.389236in}{1.953596in}}%
\pgfpathlineto{\pgfqpoint{4.393745in}{1.920192in}}%
\pgfpathlineto{\pgfqpoint{4.398255in}{1.928807in}}%
\pgfpathlineto{\pgfqpoint{4.402764in}{1.941528in}}%
\pgfpathlineto{\pgfqpoint{4.407273in}{1.980025in}}%
\pgfpathlineto{\pgfqpoint{4.411782in}{2.000263in}}%
\pgfpathlineto{\pgfqpoint{4.416291in}{1.966096in}}%
\pgfpathlineto{\pgfqpoint{4.420800in}{2.034001in}}%
\pgfpathlineto{\pgfqpoint{4.425309in}{1.980936in}}%
\pgfpathlineto{\pgfqpoint{4.429818in}{1.969872in}}%
\pgfpathlineto{\pgfqpoint{4.434327in}{1.973905in}}%
\pgfpathlineto{\pgfqpoint{4.438836in}{1.960496in}}%
\pgfpathlineto{\pgfqpoint{4.443345in}{1.965586in}}%
\pgfpathlineto{\pgfqpoint{4.447855in}{1.985962in}}%
\pgfpathlineto{\pgfqpoint{4.452364in}{1.992697in}}%
\pgfpathlineto{\pgfqpoint{4.456873in}{1.994553in}}%
\pgfpathlineto{\pgfqpoint{4.461382in}{1.984711in}}%
\pgfpathlineto{\pgfqpoint{4.465891in}{2.008661in}}%
\pgfpathlineto{\pgfqpoint{4.470400in}{1.986285in}}%
\pgfpathlineto{\pgfqpoint{4.474909in}{1.997061in}}%
\pgfpathlineto{\pgfqpoint{4.479418in}{2.034391in}}%
\pgfpathlineto{\pgfqpoint{4.483927in}{1.995644in}}%
\pgfpathlineto{\pgfqpoint{4.488436in}{2.017639in}}%
\pgfpathlineto{\pgfqpoint{4.492945in}{2.024202in}}%
\pgfpathlineto{\pgfqpoint{4.497455in}{2.043306in}}%
\pgfpathlineto{\pgfqpoint{4.501964in}{2.023430in}}%
\pgfpathlineto{\pgfqpoint{4.506473in}{2.031375in}}%
\pgfpathlineto{\pgfqpoint{4.510982in}{2.075900in}}%
\pgfpathlineto{\pgfqpoint{4.515491in}{2.021301in}}%
\pgfpathlineto{\pgfqpoint{4.520000in}{2.038663in}}%
\pgfpathlineto{\pgfqpoint{4.524509in}{2.041372in}}%
\pgfpathlineto{\pgfqpoint{4.529018in}{2.110497in}}%
\pgfpathlineto{\pgfqpoint{4.533527in}{2.070149in}}%
\pgfpathlineto{\pgfqpoint{4.538036in}{2.058661in}}%
\pgfpathlineto{\pgfqpoint{4.542545in}{2.089418in}}%
\pgfpathlineto{\pgfqpoint{4.547055in}{2.053506in}}%
\pgfpathlineto{\pgfqpoint{4.551564in}{2.067771in}}%
\pgfpathlineto{\pgfqpoint{4.556073in}{2.102356in}}%
\pgfpathlineto{\pgfqpoint{4.560582in}{2.044790in}}%
\pgfpathlineto{\pgfqpoint{4.565091in}{2.084696in}}%
\pgfpathlineto{\pgfqpoint{4.569600in}{2.055086in}}%
\pgfpathlineto{\pgfqpoint{4.574109in}{2.087337in}}%
\pgfpathlineto{\pgfqpoint{4.578618in}{2.067297in}}%
\pgfpathlineto{\pgfqpoint{4.587636in}{2.172523in}}%
\pgfpathlineto{\pgfqpoint{4.592145in}{2.203227in}}%
\pgfpathlineto{\pgfqpoint{4.596655in}{2.225898in}}%
\pgfpathlineto{\pgfqpoint{4.601164in}{2.169495in}}%
\pgfpathlineto{\pgfqpoint{4.605673in}{2.213178in}}%
\pgfpathlineto{\pgfqpoint{4.610182in}{2.628091in}}%
\pgfpathlineto{\pgfqpoint{4.614691in}{2.170927in}}%
\pgfpathlineto{\pgfqpoint{4.619200in}{2.129296in}}%
\pgfpathlineto{\pgfqpoint{4.623709in}{2.188069in}}%
\pgfpathlineto{\pgfqpoint{4.628218in}{2.219166in}}%
\pgfpathlineto{\pgfqpoint{4.632727in}{2.286709in}}%
\pgfpathlineto{\pgfqpoint{4.637236in}{2.124518in}}%
\pgfpathlineto{\pgfqpoint{4.641745in}{2.272195in}}%
\pgfpathlineto{\pgfqpoint{4.646255in}{2.727837in}}%
\pgfpathlineto{\pgfqpoint{4.655273in}{2.201026in}}%
\pgfpathlineto{\pgfqpoint{4.659782in}{2.254023in}}%
\pgfpathlineto{\pgfqpoint{4.664291in}{2.204570in}}%
\pgfpathlineto{\pgfqpoint{4.668800in}{2.969749in}}%
\pgfpathlineto{\pgfqpoint{4.673309in}{2.694715in}}%
\pgfpathlineto{\pgfqpoint{4.677818in}{2.307227in}}%
\pgfpathlineto{\pgfqpoint{4.682327in}{2.168473in}}%
\pgfpathlineto{\pgfqpoint{4.691345in}{2.347375in}}%
\pgfpathlineto{\pgfqpoint{4.695855in}{2.334146in}}%
\pgfpathlineto{\pgfqpoint{4.700364in}{2.394103in}}%
\pgfpathlineto{\pgfqpoint{4.704873in}{2.420920in}}%
\pgfpathlineto{\pgfqpoint{4.709382in}{2.539586in}}%
\pgfpathlineto{\pgfqpoint{4.713891in}{2.829911in}}%
\pgfpathlineto{\pgfqpoint{4.718400in}{2.220552in}}%
\pgfpathlineto{\pgfqpoint{4.722909in}{2.417971in}}%
\pgfpathlineto{\pgfqpoint{4.727418in}{2.397311in}}%
\pgfpathlineto{\pgfqpoint{4.731927in}{2.328806in}}%
\pgfpathlineto{\pgfqpoint{4.736436in}{2.446918in}}%
\pgfpathlineto{\pgfqpoint{4.740945in}{2.805250in}}%
\pgfpathlineto{\pgfqpoint{4.745455in}{2.313460in}}%
\pgfpathlineto{\pgfqpoint{4.749964in}{2.530013in}}%
\pgfpathlineto{\pgfqpoint{4.754473in}{2.340339in}}%
\pgfpathlineto{\pgfqpoint{4.758982in}{2.485710in}}%
\pgfpathlineto{\pgfqpoint{4.763491in}{2.071672in}}%
\pgfpathlineto{\pgfqpoint{4.768000in}{2.061442in}}%
\pgfpathlineto{\pgfqpoint{4.772509in}{2.082043in}}%
\pgfpathlineto{\pgfqpoint{4.777018in}{2.058251in}}%
\pgfpathlineto{\pgfqpoint{4.781527in}{2.154386in}}%
\pgfpathlineto{\pgfqpoint{4.786036in}{2.136423in}}%
\pgfpathlineto{\pgfqpoint{4.790545in}{2.069685in}}%
\pgfpathlineto{\pgfqpoint{4.795055in}{2.107711in}}%
\pgfpathlineto{\pgfqpoint{4.804073in}{2.214501in}}%
\pgfpathlineto{\pgfqpoint{4.808582in}{3.198384in}}%
\pgfpathlineto{\pgfqpoint{4.813091in}{2.952499in}}%
\pgfpathlineto{\pgfqpoint{4.817600in}{2.106307in}}%
\pgfpathlineto{\pgfqpoint{4.822109in}{2.076861in}}%
\pgfpathlineto{\pgfqpoint{4.826618in}{2.153694in}}%
\pgfpathlineto{\pgfqpoint{4.831127in}{2.771329in}}%
\pgfpathlineto{\pgfqpoint{4.835636in}{2.275305in}}%
\pgfpathlineto{\pgfqpoint{4.840145in}{2.140961in}}%
\pgfpathlineto{\pgfqpoint{4.844655in}{2.248045in}}%
\pgfpathlineto{\pgfqpoint{4.849164in}{3.057859in}}%
\pgfpathlineto{\pgfqpoint{4.853673in}{2.345213in}}%
\pgfpathlineto{\pgfqpoint{4.858182in}{2.353354in}}%
\pgfpathlineto{\pgfqpoint{4.862691in}{2.341777in}}%
\pgfpathlineto{\pgfqpoint{4.867200in}{2.590389in}}%
\pgfpathlineto{\pgfqpoint{4.871709in}{3.626265in}}%
\pgfpathlineto{\pgfqpoint{4.876218in}{2.760858in}}%
\pgfpathlineto{\pgfqpoint{4.880727in}{2.611354in}}%
\pgfpathlineto{\pgfqpoint{4.885236in}{2.556350in}}%
\pgfpathlineto{\pgfqpoint{4.889745in}{2.302296in}}%
\pgfpathlineto{\pgfqpoint{4.894255in}{2.367520in}}%
\pgfpathlineto{\pgfqpoint{4.898764in}{2.323422in}}%
\pgfpathlineto{\pgfqpoint{4.903273in}{2.405655in}}%
\pgfpathlineto{\pgfqpoint{4.907782in}{2.374499in}}%
\pgfpathlineto{\pgfqpoint{4.912291in}{2.327402in}}%
\pgfpathlineto{\pgfqpoint{4.916800in}{2.392618in}}%
\pgfpathlineto{\pgfqpoint{4.921309in}{2.419115in}}%
\pgfpathlineto{\pgfqpoint{4.925818in}{2.374437in}}%
\pgfpathlineto{\pgfqpoint{4.930327in}{2.383931in}}%
\pgfpathlineto{\pgfqpoint{4.934836in}{2.390188in}}%
\pgfpathlineto{\pgfqpoint{4.939345in}{2.366512in}}%
\pgfpathlineto{\pgfqpoint{4.943855in}{2.546802in}}%
\pgfpathlineto{\pgfqpoint{4.948364in}{2.567544in}}%
\pgfpathlineto{\pgfqpoint{4.952873in}{2.397116in}}%
\pgfpathlineto{\pgfqpoint{4.957382in}{2.407623in}}%
\pgfpathlineto{\pgfqpoint{4.961891in}{2.440630in}}%
\pgfpathlineto{\pgfqpoint{4.966400in}{2.431849in}}%
\pgfpathlineto{\pgfqpoint{4.970909in}{2.464215in}}%
\pgfpathlineto{\pgfqpoint{4.975418in}{2.398981in}}%
\pgfpathlineto{\pgfqpoint{4.979927in}{2.400348in}}%
\pgfpathlineto{\pgfqpoint{4.984436in}{2.455223in}}%
\pgfpathlineto{\pgfqpoint{4.988945in}{2.375658in}}%
\pgfpathlineto{\pgfqpoint{4.993455in}{2.431871in}}%
\pgfpathlineto{\pgfqpoint{5.002473in}{2.458552in}}%
\pgfpathlineto{\pgfqpoint{5.006982in}{2.450330in}}%
\pgfpathlineto{\pgfqpoint{5.011491in}{2.401821in}}%
\pgfpathlineto{\pgfqpoint{5.016000in}{2.454195in}}%
\pgfpathlineto{\pgfqpoint{5.020509in}{2.387407in}}%
\pgfpathlineto{\pgfqpoint{5.025018in}{2.406368in}}%
\pgfpathlineto{\pgfqpoint{5.029527in}{2.438952in}}%
\pgfpathlineto{\pgfqpoint{5.034036in}{2.435094in}}%
\pgfpathlineto{\pgfqpoint{5.038545in}{2.467393in}}%
\pgfpathlineto{\pgfqpoint{5.043055in}{2.558597in}}%
\pgfpathlineto{\pgfqpoint{5.047564in}{2.531881in}}%
\pgfpathlineto{\pgfqpoint{5.052073in}{2.443951in}}%
\pgfpathlineto{\pgfqpoint{5.056582in}{2.635604in}}%
\pgfpathlineto{\pgfqpoint{5.065600in}{2.498973in}}%
\pgfpathlineto{\pgfqpoint{5.070109in}{2.485509in}}%
\pgfpathlineto{\pgfqpoint{5.074618in}{2.538487in}}%
\pgfpathlineto{\pgfqpoint{5.079127in}{2.443567in}}%
\pgfpathlineto{\pgfqpoint{5.083636in}{2.603080in}}%
\pgfpathlineto{\pgfqpoint{5.088145in}{2.477841in}}%
\pgfpathlineto{\pgfqpoint{5.092655in}{2.504268in}}%
\pgfpathlineto{\pgfqpoint{5.097164in}{2.581769in}}%
\pgfpathlineto{\pgfqpoint{5.101673in}{2.507028in}}%
\pgfpathlineto{\pgfqpoint{5.106182in}{2.499016in}}%
\pgfpathlineto{\pgfqpoint{5.110691in}{2.548438in}}%
\pgfpathlineto{\pgfqpoint{5.115200in}{2.557462in}}%
\pgfpathlineto{\pgfqpoint{5.119709in}{2.534791in}}%
\pgfpathlineto{\pgfqpoint{5.124218in}{2.608213in}}%
\pgfpathlineto{\pgfqpoint{5.128727in}{2.553747in}}%
\pgfpathlineto{\pgfqpoint{5.133236in}{2.515587in}}%
\pgfpathlineto{\pgfqpoint{5.137745in}{2.574119in}}%
\pgfpathlineto{\pgfqpoint{5.142255in}{2.580128in}}%
\pgfpathlineto{\pgfqpoint{5.146764in}{2.583305in}}%
\pgfpathlineto{\pgfqpoint{5.151273in}{3.685075in}}%
\pgfpathlineto{\pgfqpoint{5.155782in}{3.075802in}}%
\pgfpathlineto{\pgfqpoint{5.160291in}{4.056000in}}%
\pgfpathlineto{\pgfqpoint{5.164800in}{3.072674in}}%
\pgfpathlineto{\pgfqpoint{5.169309in}{2.912382in}}%
\pgfpathlineto{\pgfqpoint{5.173818in}{2.665769in}}%
\pgfpathlineto{\pgfqpoint{5.178327in}{2.582174in}}%
\pgfpathlineto{\pgfqpoint{5.182836in}{2.609270in}}%
\pgfpathlineto{\pgfqpoint{5.187345in}{2.628671in}}%
\pgfpathlineto{\pgfqpoint{5.191855in}{2.679942in}}%
\pgfpathlineto{\pgfqpoint{5.196364in}{2.642415in}}%
\pgfpathlineto{\pgfqpoint{5.200873in}{2.641476in}}%
\pgfpathlineto{\pgfqpoint{5.205382in}{2.653199in}}%
\pgfpathlineto{\pgfqpoint{5.209891in}{2.590354in}}%
\pgfpathlineto{\pgfqpoint{5.218909in}{2.673203in}}%
\pgfpathlineto{\pgfqpoint{5.223418in}{2.678558in}}%
\pgfpathlineto{\pgfqpoint{5.227927in}{2.648932in}}%
\pgfpathlineto{\pgfqpoint{5.232436in}{2.648576in}}%
\pgfpathlineto{\pgfqpoint{5.241455in}{2.748167in}}%
\pgfpathlineto{\pgfqpoint{5.245964in}{2.685324in}}%
\pgfpathlineto{\pgfqpoint{5.250473in}{3.557244in}}%
\pgfpathlineto{\pgfqpoint{5.254982in}{2.761023in}}%
\pgfpathlineto{\pgfqpoint{5.259491in}{3.885630in}}%
\pgfpathlineto{\pgfqpoint{5.264000in}{3.256012in}}%
\pgfpathlineto{\pgfqpoint{5.273018in}{2.850533in}}%
\pgfpathlineto{\pgfqpoint{5.277527in}{2.705477in}}%
\pgfpathlineto{\pgfqpoint{5.282036in}{2.745609in}}%
\pgfpathlineto{\pgfqpoint{5.286545in}{2.678369in}}%
\pgfpathlineto{\pgfqpoint{5.291055in}{2.803877in}}%
\pgfpathlineto{\pgfqpoint{5.295564in}{2.689275in}}%
\pgfpathlineto{\pgfqpoint{5.300073in}{2.676786in}}%
\pgfpathlineto{\pgfqpoint{5.304582in}{2.716977in}}%
\pgfpathlineto{\pgfqpoint{5.309091in}{2.774820in}}%
\pgfpathlineto{\pgfqpoint{5.313600in}{2.665187in}}%
\pgfpathlineto{\pgfqpoint{5.318109in}{2.688780in}}%
\pgfpathlineto{\pgfqpoint{5.322618in}{2.676210in}}%
\pgfpathlineto{\pgfqpoint{5.327127in}{2.750980in}}%
\pgfpathlineto{\pgfqpoint{5.331636in}{2.733289in}}%
\pgfpathlineto{\pgfqpoint{5.336145in}{2.732791in}}%
\pgfpathlineto{\pgfqpoint{5.340655in}{2.730302in}}%
\pgfpathlineto{\pgfqpoint{5.345164in}{2.748342in}}%
\pgfpathlineto{\pgfqpoint{5.349673in}{2.647897in}}%
\pgfpathlineto{\pgfqpoint{5.354182in}{2.720891in}}%
\pgfpathlineto{\pgfqpoint{5.358691in}{2.827585in}}%
\pgfpathlineto{\pgfqpoint{5.363200in}{2.805015in}}%
\pgfpathlineto{\pgfqpoint{5.367709in}{2.793195in}}%
\pgfpathlineto{\pgfqpoint{5.372218in}{2.884981in}}%
\pgfpathlineto{\pgfqpoint{5.376727in}{2.723106in}}%
\pgfpathlineto{\pgfqpoint{5.381236in}{2.746411in}}%
\pgfpathlineto{\pgfqpoint{5.385745in}{2.760852in}}%
\pgfpathlineto{\pgfqpoint{5.390255in}{2.755036in}}%
\pgfpathlineto{\pgfqpoint{5.394764in}{2.799465in}}%
\pgfpathlineto{\pgfqpoint{5.399273in}{2.766869in}}%
\pgfpathlineto{\pgfqpoint{5.403782in}{2.815558in}}%
\pgfpathlineto{\pgfqpoint{5.408291in}{2.804744in}}%
\pgfpathlineto{\pgfqpoint{5.412800in}{2.695305in}}%
\pgfpathlineto{\pgfqpoint{5.421818in}{2.791380in}}%
\pgfpathlineto{\pgfqpoint{5.430836in}{2.913716in}}%
\pgfpathlineto{\pgfqpoint{5.435345in}{2.821611in}}%
\pgfpathlineto{\pgfqpoint{5.439855in}{2.775005in}}%
\pgfpathlineto{\pgfqpoint{5.444364in}{2.860609in}}%
\pgfpathlineto{\pgfqpoint{5.448873in}{2.794449in}}%
\pgfpathlineto{\pgfqpoint{5.453382in}{2.823447in}}%
\pgfpathlineto{\pgfqpoint{5.457891in}{2.843133in}}%
\pgfpathlineto{\pgfqpoint{5.462400in}{2.818798in}}%
\pgfpathlineto{\pgfqpoint{5.466909in}{2.862380in}}%
\pgfpathlineto{\pgfqpoint{5.471418in}{2.833155in}}%
\pgfpathlineto{\pgfqpoint{5.475927in}{2.823407in}}%
\pgfpathlineto{\pgfqpoint{5.480436in}{2.962398in}}%
\pgfpathlineto{\pgfqpoint{5.484945in}{2.836191in}}%
\pgfpathlineto{\pgfqpoint{5.489455in}{2.852066in}}%
\pgfpathlineto{\pgfqpoint{5.493964in}{2.913997in}}%
\pgfpathlineto{\pgfqpoint{5.498473in}{2.805982in}}%
\pgfpathlineto{\pgfqpoint{5.502982in}{2.918078in}}%
\pgfpathlineto{\pgfqpoint{5.507491in}{2.843687in}}%
\pgfpathlineto{\pgfqpoint{5.512000in}{2.862945in}}%
\pgfpathlineto{\pgfqpoint{5.516509in}{2.876850in}}%
\pgfpathlineto{\pgfqpoint{5.521018in}{2.958271in}}%
\pgfpathlineto{\pgfqpoint{5.525527in}{2.959261in}}%
\pgfpathlineto{\pgfqpoint{5.530036in}{2.928020in}}%
\pgfpathlineto{\pgfqpoint{5.534545in}{2.967009in}}%
\pgfpathlineto{\pgfqpoint{5.534545in}{2.967009in}}%
\pgfusepath{stroke}%
\end{pgfscope}%
\begin{pgfscope}%
\pgfsetrectcap%
\pgfsetmiterjoin%
\pgfsetlinewidth{0.803000pt}%
\definecolor{currentstroke}{rgb}{0.000000,0.000000,0.000000}%
\pgfsetstrokecolor{currentstroke}%
\pgfsetdash{}{0pt}%
\pgfpathmoveto{\pgfqpoint{0.800000in}{0.528000in}}%
\pgfpathlineto{\pgfqpoint{0.800000in}{4.224000in}}%
\pgfusepath{stroke}%
\end{pgfscope}%
\begin{pgfscope}%
\pgfsetrectcap%
\pgfsetmiterjoin%
\pgfsetlinewidth{0.803000pt}%
\definecolor{currentstroke}{rgb}{0.000000,0.000000,0.000000}%
\pgfsetstrokecolor{currentstroke}%
\pgfsetdash{}{0pt}%
\pgfpathmoveto{\pgfqpoint{5.760000in}{0.528000in}}%
\pgfpathlineto{\pgfqpoint{5.760000in}{4.224000in}}%
\pgfusepath{stroke}%
\end{pgfscope}%
\begin{pgfscope}%
\pgfsetrectcap%
\pgfsetmiterjoin%
\pgfsetlinewidth{0.803000pt}%
\definecolor{currentstroke}{rgb}{0.000000,0.000000,0.000000}%
\pgfsetstrokecolor{currentstroke}%
\pgfsetdash{}{0pt}%
\pgfpathmoveto{\pgfqpoint{0.800000in}{0.528000in}}%
\pgfpathlineto{\pgfqpoint{5.760000in}{0.528000in}}%
\pgfusepath{stroke}%
\end{pgfscope}%
\begin{pgfscope}%
\pgfsetrectcap%
\pgfsetmiterjoin%
\pgfsetlinewidth{0.803000pt}%
\definecolor{currentstroke}{rgb}{0.000000,0.000000,0.000000}%
\pgfsetstrokecolor{currentstroke}%
\pgfsetdash{}{0pt}%
\pgfpathmoveto{\pgfqpoint{0.800000in}{4.224000in}}%
\pgfpathlineto{\pgfqpoint{5.760000in}{4.224000in}}%
\pgfusepath{stroke}%
\end{pgfscope}%
\begin{pgfscope}%
\definecolor{textcolor}{rgb}{0.000000,0.000000,0.000000}%
\pgfsetstrokecolor{textcolor}%
\pgfsetfillcolor{textcolor}%
\pgftext[x=3.280000in,y=4.307333in,,base]{\color{textcolor}\ttfamily\fontsize{12.000000}{14.400000}\selectfont Bubble Sort Time vs Input size}%
\end{pgfscope}%
\begin{pgfscope}%
\pgfsetbuttcap%
\pgfsetmiterjoin%
\definecolor{currentfill}{rgb}{1.000000,1.000000,1.000000}%
\pgfsetfillcolor{currentfill}%
\pgfsetfillopacity{0.800000}%
\pgfsetlinewidth{1.003750pt}%
\definecolor{currentstroke}{rgb}{0.800000,0.800000,0.800000}%
\pgfsetstrokecolor{currentstroke}%
\pgfsetstrokeopacity{0.800000}%
\pgfsetdash{}{0pt}%
\pgfpathmoveto{\pgfqpoint{0.897222in}{3.908286in}}%
\pgfpathlineto{\pgfqpoint{1.843376in}{3.908286in}}%
\pgfpathquadraticcurveto{\pgfqpoint{1.871153in}{3.908286in}}{\pgfqpoint{1.871153in}{3.936063in}}%
\pgfpathlineto{\pgfqpoint{1.871153in}{4.126778in}}%
\pgfpathquadraticcurveto{\pgfqpoint{1.871153in}{4.154556in}}{\pgfqpoint{1.843376in}{4.154556in}}%
\pgfpathlineto{\pgfqpoint{0.897222in}{4.154556in}}%
\pgfpathquadraticcurveto{\pgfqpoint{0.869444in}{4.154556in}}{\pgfqpoint{0.869444in}{4.126778in}}%
\pgfpathlineto{\pgfqpoint{0.869444in}{3.936063in}}%
\pgfpathquadraticcurveto{\pgfqpoint{0.869444in}{3.908286in}}{\pgfqpoint{0.897222in}{3.908286in}}%
\pgfpathlineto{\pgfqpoint{0.897222in}{3.908286in}}%
\pgfpathclose%
\pgfusepath{stroke,fill}%
\end{pgfscope}%
\begin{pgfscope}%
\pgfsetrectcap%
\pgfsetroundjoin%
\pgfsetlinewidth{1.505625pt}%
\definecolor{currentstroke}{rgb}{0.000000,1.000000,0.498039}%
\pgfsetstrokecolor{currentstroke}%
\pgfsetdash{}{0pt}%
\pgfpathmoveto{\pgfqpoint{0.925000in}{4.041342in}}%
\pgfpathlineto{\pgfqpoint{1.063889in}{4.041342in}}%
\pgfpathlineto{\pgfqpoint{1.202778in}{4.041342in}}%
\pgfusepath{stroke}%
\end{pgfscope}%
\begin{pgfscope}%
\definecolor{textcolor}{rgb}{0.000000,0.000000,0.000000}%
\pgfsetstrokecolor{textcolor}%
\pgfsetfillcolor{textcolor}%
\pgftext[x=1.313889in,y=3.992731in,left,base]{\color{textcolor}\ttfamily\fontsize{10.000000}{12.000000}\selectfont Bubble}%
\end{pgfscope}%
\end{pgfpicture}%
\makeatother%
\endgroup%

%% Creator: Matplotlib, PGF backend
%%
%% To include the figure in your LaTeX document, write
%%   \input{<filename>.pgf}
%%
%% Make sure the required packages are loaded in your preamble
%%   \usepackage{pgf}
%%
%% Also ensure that all the required font packages are loaded; for instance,
%% the lmodern package is sometimes necessary when using math font.
%%   \usepackage{lmodern}
%%
%% Figures using additional raster images can only be included by \input if
%% they are in the same directory as the main LaTeX file. For loading figures
%% from other directories you can use the `import` package
%%   \usepackage{import}
%%
%% and then include the figures with
%%   \import{<path to file>}{<filename>.pgf}
%%
%% Matplotlib used the following preamble
%%   \usepackage{fontspec}
%%   \setmainfont{DejaVuSerif.ttf}[Path=\detokenize{/home/dbk/.local/lib/python3.10/site-packages/matplotlib/mpl-data/fonts/ttf/}]
%%   \setsansfont{DejaVuSans.ttf}[Path=\detokenize{/home/dbk/.local/lib/python3.10/site-packages/matplotlib/mpl-data/fonts/ttf/}]
%%   \setmonofont{DejaVuSansMono.ttf}[Path=\detokenize{/home/dbk/.local/lib/python3.10/site-packages/matplotlib/mpl-data/fonts/ttf/}]
%%
\begingroup%
\makeatletter%
\begin{pgfpicture}%
\pgfpathrectangle{\pgfpointorigin}{\pgfqpoint{6.400000in}{4.800000in}}%
\pgfusepath{use as bounding box, clip}%
\begin{pgfscope}%
\pgfsetbuttcap%
\pgfsetmiterjoin%
\definecolor{currentfill}{rgb}{1.000000,1.000000,1.000000}%
\pgfsetfillcolor{currentfill}%
\pgfsetlinewidth{0.000000pt}%
\definecolor{currentstroke}{rgb}{1.000000,1.000000,1.000000}%
\pgfsetstrokecolor{currentstroke}%
\pgfsetdash{}{0pt}%
\pgfpathmoveto{\pgfqpoint{0.000000in}{0.000000in}}%
\pgfpathlineto{\pgfqpoint{6.400000in}{0.000000in}}%
\pgfpathlineto{\pgfqpoint{6.400000in}{4.800000in}}%
\pgfpathlineto{\pgfqpoint{0.000000in}{4.800000in}}%
\pgfpathlineto{\pgfqpoint{0.000000in}{0.000000in}}%
\pgfpathclose%
\pgfusepath{fill}%
\end{pgfscope}%
\begin{pgfscope}%
\pgfsetbuttcap%
\pgfsetmiterjoin%
\definecolor{currentfill}{rgb}{1.000000,1.000000,1.000000}%
\pgfsetfillcolor{currentfill}%
\pgfsetlinewidth{0.000000pt}%
\definecolor{currentstroke}{rgb}{0.000000,0.000000,0.000000}%
\pgfsetstrokecolor{currentstroke}%
\pgfsetstrokeopacity{0.000000}%
\pgfsetdash{}{0pt}%
\pgfpathmoveto{\pgfqpoint{0.800000in}{0.528000in}}%
\pgfpathlineto{\pgfqpoint{5.760000in}{0.528000in}}%
\pgfpathlineto{\pgfqpoint{5.760000in}{4.224000in}}%
\pgfpathlineto{\pgfqpoint{0.800000in}{4.224000in}}%
\pgfpathlineto{\pgfqpoint{0.800000in}{0.528000in}}%
\pgfpathclose%
\pgfusepath{fill}%
\end{pgfscope}%
\begin{pgfscope}%
\pgfsetbuttcap%
\pgfsetroundjoin%
\definecolor{currentfill}{rgb}{0.000000,0.000000,0.000000}%
\pgfsetfillcolor{currentfill}%
\pgfsetlinewidth{0.803000pt}%
\definecolor{currentstroke}{rgb}{0.000000,0.000000,0.000000}%
\pgfsetstrokecolor{currentstroke}%
\pgfsetdash{}{0pt}%
\pgfsys@defobject{currentmarker}{\pgfqpoint{0.000000in}{-0.048611in}}{\pgfqpoint{0.000000in}{0.000000in}}{%
\pgfpathmoveto{\pgfqpoint{0.000000in}{0.000000in}}%
\pgfpathlineto{\pgfqpoint{0.000000in}{-0.048611in}}%
\pgfusepath{stroke,fill}%
}%
\begin{pgfscope}%
\pgfsys@transformshift{1.020945in}{0.528000in}%
\pgfsys@useobject{currentmarker}{}%
\end{pgfscope}%
\end{pgfscope}%
\begin{pgfscope}%
\definecolor{textcolor}{rgb}{0.000000,0.000000,0.000000}%
\pgfsetstrokecolor{textcolor}%
\pgfsetfillcolor{textcolor}%
\pgftext[x=1.020945in,y=0.430778in,,top]{\color{textcolor}\ttfamily\fontsize{10.000000}{12.000000}\selectfont 0}%
\end{pgfscope}%
\begin{pgfscope}%
\pgfsetbuttcap%
\pgfsetroundjoin%
\definecolor{currentfill}{rgb}{0.000000,0.000000,0.000000}%
\pgfsetfillcolor{currentfill}%
\pgfsetlinewidth{0.803000pt}%
\definecolor{currentstroke}{rgb}{0.000000,0.000000,0.000000}%
\pgfsetstrokecolor{currentstroke}%
\pgfsetdash{}{0pt}%
\pgfsys@defobject{currentmarker}{\pgfqpoint{0.000000in}{-0.048611in}}{\pgfqpoint{0.000000in}{0.000000in}}{%
\pgfpathmoveto{\pgfqpoint{0.000000in}{0.000000in}}%
\pgfpathlineto{\pgfqpoint{0.000000in}{-0.048611in}}%
\pgfusepath{stroke,fill}%
}%
\begin{pgfscope}%
\pgfsys@transformshift{1.922764in}{0.528000in}%
\pgfsys@useobject{currentmarker}{}%
\end{pgfscope}%
\end{pgfscope}%
\begin{pgfscope}%
\definecolor{textcolor}{rgb}{0.000000,0.000000,0.000000}%
\pgfsetstrokecolor{textcolor}%
\pgfsetfillcolor{textcolor}%
\pgftext[x=1.922764in,y=0.430778in,,top]{\color{textcolor}\ttfamily\fontsize{10.000000}{12.000000}\selectfont 200}%
\end{pgfscope}%
\begin{pgfscope}%
\pgfsetbuttcap%
\pgfsetroundjoin%
\definecolor{currentfill}{rgb}{0.000000,0.000000,0.000000}%
\pgfsetfillcolor{currentfill}%
\pgfsetlinewidth{0.803000pt}%
\definecolor{currentstroke}{rgb}{0.000000,0.000000,0.000000}%
\pgfsetstrokecolor{currentstroke}%
\pgfsetdash{}{0pt}%
\pgfsys@defobject{currentmarker}{\pgfqpoint{0.000000in}{-0.048611in}}{\pgfqpoint{0.000000in}{0.000000in}}{%
\pgfpathmoveto{\pgfqpoint{0.000000in}{0.000000in}}%
\pgfpathlineto{\pgfqpoint{0.000000in}{-0.048611in}}%
\pgfusepath{stroke,fill}%
}%
\begin{pgfscope}%
\pgfsys@transformshift{2.824582in}{0.528000in}%
\pgfsys@useobject{currentmarker}{}%
\end{pgfscope}%
\end{pgfscope}%
\begin{pgfscope}%
\definecolor{textcolor}{rgb}{0.000000,0.000000,0.000000}%
\pgfsetstrokecolor{textcolor}%
\pgfsetfillcolor{textcolor}%
\pgftext[x=2.824582in,y=0.430778in,,top]{\color{textcolor}\ttfamily\fontsize{10.000000}{12.000000}\selectfont 400}%
\end{pgfscope}%
\begin{pgfscope}%
\pgfsetbuttcap%
\pgfsetroundjoin%
\definecolor{currentfill}{rgb}{0.000000,0.000000,0.000000}%
\pgfsetfillcolor{currentfill}%
\pgfsetlinewidth{0.803000pt}%
\definecolor{currentstroke}{rgb}{0.000000,0.000000,0.000000}%
\pgfsetstrokecolor{currentstroke}%
\pgfsetdash{}{0pt}%
\pgfsys@defobject{currentmarker}{\pgfqpoint{0.000000in}{-0.048611in}}{\pgfqpoint{0.000000in}{0.000000in}}{%
\pgfpathmoveto{\pgfqpoint{0.000000in}{0.000000in}}%
\pgfpathlineto{\pgfqpoint{0.000000in}{-0.048611in}}%
\pgfusepath{stroke,fill}%
}%
\begin{pgfscope}%
\pgfsys@transformshift{3.726400in}{0.528000in}%
\pgfsys@useobject{currentmarker}{}%
\end{pgfscope}%
\end{pgfscope}%
\begin{pgfscope}%
\definecolor{textcolor}{rgb}{0.000000,0.000000,0.000000}%
\pgfsetstrokecolor{textcolor}%
\pgfsetfillcolor{textcolor}%
\pgftext[x=3.726400in,y=0.430778in,,top]{\color{textcolor}\ttfamily\fontsize{10.000000}{12.000000}\selectfont 600}%
\end{pgfscope}%
\begin{pgfscope}%
\pgfsetbuttcap%
\pgfsetroundjoin%
\definecolor{currentfill}{rgb}{0.000000,0.000000,0.000000}%
\pgfsetfillcolor{currentfill}%
\pgfsetlinewidth{0.803000pt}%
\definecolor{currentstroke}{rgb}{0.000000,0.000000,0.000000}%
\pgfsetstrokecolor{currentstroke}%
\pgfsetdash{}{0pt}%
\pgfsys@defobject{currentmarker}{\pgfqpoint{0.000000in}{-0.048611in}}{\pgfqpoint{0.000000in}{0.000000in}}{%
\pgfpathmoveto{\pgfqpoint{0.000000in}{0.000000in}}%
\pgfpathlineto{\pgfqpoint{0.000000in}{-0.048611in}}%
\pgfusepath{stroke,fill}%
}%
\begin{pgfscope}%
\pgfsys@transformshift{4.628218in}{0.528000in}%
\pgfsys@useobject{currentmarker}{}%
\end{pgfscope}%
\end{pgfscope}%
\begin{pgfscope}%
\definecolor{textcolor}{rgb}{0.000000,0.000000,0.000000}%
\pgfsetstrokecolor{textcolor}%
\pgfsetfillcolor{textcolor}%
\pgftext[x=4.628218in,y=0.430778in,,top]{\color{textcolor}\ttfamily\fontsize{10.000000}{12.000000}\selectfont 800}%
\end{pgfscope}%
\begin{pgfscope}%
\pgfsetbuttcap%
\pgfsetroundjoin%
\definecolor{currentfill}{rgb}{0.000000,0.000000,0.000000}%
\pgfsetfillcolor{currentfill}%
\pgfsetlinewidth{0.803000pt}%
\definecolor{currentstroke}{rgb}{0.000000,0.000000,0.000000}%
\pgfsetstrokecolor{currentstroke}%
\pgfsetdash{}{0pt}%
\pgfsys@defobject{currentmarker}{\pgfqpoint{0.000000in}{-0.048611in}}{\pgfqpoint{0.000000in}{0.000000in}}{%
\pgfpathmoveto{\pgfqpoint{0.000000in}{0.000000in}}%
\pgfpathlineto{\pgfqpoint{0.000000in}{-0.048611in}}%
\pgfusepath{stroke,fill}%
}%
\begin{pgfscope}%
\pgfsys@transformshift{5.530036in}{0.528000in}%
\pgfsys@useobject{currentmarker}{}%
\end{pgfscope}%
\end{pgfscope}%
\begin{pgfscope}%
\definecolor{textcolor}{rgb}{0.000000,0.000000,0.000000}%
\pgfsetstrokecolor{textcolor}%
\pgfsetfillcolor{textcolor}%
\pgftext[x=5.530036in,y=0.430778in,,top]{\color{textcolor}\ttfamily\fontsize{10.000000}{12.000000}\selectfont 1000}%
\end{pgfscope}%
\begin{pgfscope}%
\definecolor{textcolor}{rgb}{0.000000,0.000000,0.000000}%
\pgfsetstrokecolor{textcolor}%
\pgfsetfillcolor{textcolor}%
\pgftext[x=3.280000in,y=0.240063in,,top]{\color{textcolor}\ttfamily\fontsize{10.000000}{12.000000}\selectfont Size of Array}%
\end{pgfscope}%
\begin{pgfscope}%
\pgfsetbuttcap%
\pgfsetroundjoin%
\definecolor{currentfill}{rgb}{0.000000,0.000000,0.000000}%
\pgfsetfillcolor{currentfill}%
\pgfsetlinewidth{0.803000pt}%
\definecolor{currentstroke}{rgb}{0.000000,0.000000,0.000000}%
\pgfsetstrokecolor{currentstroke}%
\pgfsetdash{}{0pt}%
\pgfsys@defobject{currentmarker}{\pgfqpoint{-0.048611in}{0.000000in}}{\pgfqpoint{-0.000000in}{0.000000in}}{%
\pgfpathmoveto{\pgfqpoint{-0.000000in}{0.000000in}}%
\pgfpathlineto{\pgfqpoint{-0.048611in}{0.000000in}}%
\pgfusepath{stroke,fill}%
}%
\begin{pgfscope}%
\pgfsys@transformshift{0.800000in}{0.576000in}%
\pgfsys@useobject{currentmarker}{}%
\end{pgfscope}%
\end{pgfscope}%
\begin{pgfscope}%
\definecolor{textcolor}{rgb}{0.000000,0.000000,0.000000}%
\pgfsetstrokecolor{textcolor}%
\pgfsetfillcolor{textcolor}%
\pgftext[x=0.451923in, y=0.522865in, left, base]{\color{textcolor}\ttfamily\fontsize{10.000000}{12.000000}\selectfont 280}%
\end{pgfscope}%
\begin{pgfscope}%
\pgfsetbuttcap%
\pgfsetroundjoin%
\definecolor{currentfill}{rgb}{0.000000,0.000000,0.000000}%
\pgfsetfillcolor{currentfill}%
\pgfsetlinewidth{0.803000pt}%
\definecolor{currentstroke}{rgb}{0.000000,0.000000,0.000000}%
\pgfsetstrokecolor{currentstroke}%
\pgfsetdash{}{0pt}%
\pgfsys@defobject{currentmarker}{\pgfqpoint{-0.048611in}{0.000000in}}{\pgfqpoint{-0.000000in}{0.000000in}}{%
\pgfpathmoveto{\pgfqpoint{-0.000000in}{0.000000in}}%
\pgfpathlineto{\pgfqpoint{-0.048611in}{0.000000in}}%
\pgfusepath{stroke,fill}%
}%
\begin{pgfscope}%
\pgfsys@transformshift{0.800000in}{1.176000in}%
\pgfsys@useobject{currentmarker}{}%
\end{pgfscope}%
\end{pgfscope}%
\begin{pgfscope}%
\definecolor{textcolor}{rgb}{0.000000,0.000000,0.000000}%
\pgfsetstrokecolor{textcolor}%
\pgfsetfillcolor{textcolor}%
\pgftext[x=0.451923in, y=1.122865in, left, base]{\color{textcolor}\ttfamily\fontsize{10.000000}{12.000000}\selectfont 300}%
\end{pgfscope}%
\begin{pgfscope}%
\pgfsetbuttcap%
\pgfsetroundjoin%
\definecolor{currentfill}{rgb}{0.000000,0.000000,0.000000}%
\pgfsetfillcolor{currentfill}%
\pgfsetlinewidth{0.803000pt}%
\definecolor{currentstroke}{rgb}{0.000000,0.000000,0.000000}%
\pgfsetstrokecolor{currentstroke}%
\pgfsetdash{}{0pt}%
\pgfsys@defobject{currentmarker}{\pgfqpoint{-0.048611in}{0.000000in}}{\pgfqpoint{-0.000000in}{0.000000in}}{%
\pgfpathmoveto{\pgfqpoint{-0.000000in}{0.000000in}}%
\pgfpathlineto{\pgfqpoint{-0.048611in}{0.000000in}}%
\pgfusepath{stroke,fill}%
}%
\begin{pgfscope}%
\pgfsys@transformshift{0.800000in}{1.776000in}%
\pgfsys@useobject{currentmarker}{}%
\end{pgfscope}%
\end{pgfscope}%
\begin{pgfscope}%
\definecolor{textcolor}{rgb}{0.000000,0.000000,0.000000}%
\pgfsetstrokecolor{textcolor}%
\pgfsetfillcolor{textcolor}%
\pgftext[x=0.451923in, y=1.722865in, left, base]{\color{textcolor}\ttfamily\fontsize{10.000000}{12.000000}\selectfont 320}%
\end{pgfscope}%
\begin{pgfscope}%
\pgfsetbuttcap%
\pgfsetroundjoin%
\definecolor{currentfill}{rgb}{0.000000,0.000000,0.000000}%
\pgfsetfillcolor{currentfill}%
\pgfsetlinewidth{0.803000pt}%
\definecolor{currentstroke}{rgb}{0.000000,0.000000,0.000000}%
\pgfsetstrokecolor{currentstroke}%
\pgfsetdash{}{0pt}%
\pgfsys@defobject{currentmarker}{\pgfqpoint{-0.048611in}{0.000000in}}{\pgfqpoint{-0.000000in}{0.000000in}}{%
\pgfpathmoveto{\pgfqpoint{-0.000000in}{0.000000in}}%
\pgfpathlineto{\pgfqpoint{-0.048611in}{0.000000in}}%
\pgfusepath{stroke,fill}%
}%
\begin{pgfscope}%
\pgfsys@transformshift{0.800000in}{2.376000in}%
\pgfsys@useobject{currentmarker}{}%
\end{pgfscope}%
\end{pgfscope}%
\begin{pgfscope}%
\definecolor{textcolor}{rgb}{0.000000,0.000000,0.000000}%
\pgfsetstrokecolor{textcolor}%
\pgfsetfillcolor{textcolor}%
\pgftext[x=0.451923in, y=2.322865in, left, base]{\color{textcolor}\ttfamily\fontsize{10.000000}{12.000000}\selectfont 340}%
\end{pgfscope}%
\begin{pgfscope}%
\pgfsetbuttcap%
\pgfsetroundjoin%
\definecolor{currentfill}{rgb}{0.000000,0.000000,0.000000}%
\pgfsetfillcolor{currentfill}%
\pgfsetlinewidth{0.803000pt}%
\definecolor{currentstroke}{rgb}{0.000000,0.000000,0.000000}%
\pgfsetstrokecolor{currentstroke}%
\pgfsetdash{}{0pt}%
\pgfsys@defobject{currentmarker}{\pgfqpoint{-0.048611in}{0.000000in}}{\pgfqpoint{-0.000000in}{0.000000in}}{%
\pgfpathmoveto{\pgfqpoint{-0.000000in}{0.000000in}}%
\pgfpathlineto{\pgfqpoint{-0.048611in}{0.000000in}}%
\pgfusepath{stroke,fill}%
}%
\begin{pgfscope}%
\pgfsys@transformshift{0.800000in}{2.976000in}%
\pgfsys@useobject{currentmarker}{}%
\end{pgfscope}%
\end{pgfscope}%
\begin{pgfscope}%
\definecolor{textcolor}{rgb}{0.000000,0.000000,0.000000}%
\pgfsetstrokecolor{textcolor}%
\pgfsetfillcolor{textcolor}%
\pgftext[x=0.451923in, y=2.922865in, left, base]{\color{textcolor}\ttfamily\fontsize{10.000000}{12.000000}\selectfont 360}%
\end{pgfscope}%
\begin{pgfscope}%
\pgfsetbuttcap%
\pgfsetroundjoin%
\definecolor{currentfill}{rgb}{0.000000,0.000000,0.000000}%
\pgfsetfillcolor{currentfill}%
\pgfsetlinewidth{0.803000pt}%
\definecolor{currentstroke}{rgb}{0.000000,0.000000,0.000000}%
\pgfsetstrokecolor{currentstroke}%
\pgfsetdash{}{0pt}%
\pgfsys@defobject{currentmarker}{\pgfqpoint{-0.048611in}{0.000000in}}{\pgfqpoint{-0.000000in}{0.000000in}}{%
\pgfpathmoveto{\pgfqpoint{-0.000000in}{0.000000in}}%
\pgfpathlineto{\pgfqpoint{-0.048611in}{0.000000in}}%
\pgfusepath{stroke,fill}%
}%
\begin{pgfscope}%
\pgfsys@transformshift{0.800000in}{3.576000in}%
\pgfsys@useobject{currentmarker}{}%
\end{pgfscope}%
\end{pgfscope}%
\begin{pgfscope}%
\definecolor{textcolor}{rgb}{0.000000,0.000000,0.000000}%
\pgfsetstrokecolor{textcolor}%
\pgfsetfillcolor{textcolor}%
\pgftext[x=0.451923in, y=3.522865in, left, base]{\color{textcolor}\ttfamily\fontsize{10.000000}{12.000000}\selectfont 380}%
\end{pgfscope}%
\begin{pgfscope}%
\pgfsetbuttcap%
\pgfsetroundjoin%
\definecolor{currentfill}{rgb}{0.000000,0.000000,0.000000}%
\pgfsetfillcolor{currentfill}%
\pgfsetlinewidth{0.803000pt}%
\definecolor{currentstroke}{rgb}{0.000000,0.000000,0.000000}%
\pgfsetstrokecolor{currentstroke}%
\pgfsetdash{}{0pt}%
\pgfsys@defobject{currentmarker}{\pgfqpoint{-0.048611in}{0.000000in}}{\pgfqpoint{-0.000000in}{0.000000in}}{%
\pgfpathmoveto{\pgfqpoint{-0.000000in}{0.000000in}}%
\pgfpathlineto{\pgfqpoint{-0.048611in}{0.000000in}}%
\pgfusepath{stroke,fill}%
}%
\begin{pgfscope}%
\pgfsys@transformshift{0.800000in}{4.176000in}%
\pgfsys@useobject{currentmarker}{}%
\end{pgfscope}%
\end{pgfscope}%
\begin{pgfscope}%
\definecolor{textcolor}{rgb}{0.000000,0.000000,0.000000}%
\pgfsetstrokecolor{textcolor}%
\pgfsetfillcolor{textcolor}%
\pgftext[x=0.451923in, y=4.122865in, left, base]{\color{textcolor}\ttfamily\fontsize{10.000000}{12.000000}\selectfont 400}%
\end{pgfscope}%
\begin{pgfscope}%
\definecolor{textcolor}{rgb}{0.000000,0.000000,0.000000}%
\pgfsetstrokecolor{textcolor}%
\pgfsetfillcolor{textcolor}%
\pgftext[x=0.396368in,y=2.376000in,,bottom,rotate=90.000000]{\color{textcolor}\ttfamily\fontsize{10.000000}{12.000000}\selectfont Memory}%
\end{pgfscope}%
\begin{pgfscope}%
\pgfpathrectangle{\pgfqpoint{0.800000in}{0.528000in}}{\pgfqpoint{4.960000in}{3.696000in}}%
\pgfusepath{clip}%
\pgfsetrectcap%
\pgfsetroundjoin%
\pgfsetlinewidth{1.505625pt}%
\definecolor{currentstroke}{rgb}{0.000000,1.000000,0.498039}%
\pgfsetstrokecolor{currentstroke}%
\pgfsetdash{}{0pt}%
\pgfpathmoveto{\pgfqpoint{1.025455in}{0.696000in}}%
\pgfpathlineto{\pgfqpoint{1.733382in}{0.696000in}}%
\pgfpathlineto{\pgfqpoint{1.746909in}{3.216000in}}%
\pgfpathlineto{\pgfqpoint{2.892218in}{3.216000in}}%
\pgfpathlineto{\pgfqpoint{2.896727in}{4.056000in}}%
\pgfpathlineto{\pgfqpoint{5.534545in}{4.056000in}}%
\pgfpathlineto{\pgfqpoint{5.534545in}{4.056000in}}%
\pgfusepath{stroke}%
\end{pgfscope}%
\begin{pgfscope}%
\pgfsetrectcap%
\pgfsetmiterjoin%
\pgfsetlinewidth{0.803000pt}%
\definecolor{currentstroke}{rgb}{0.000000,0.000000,0.000000}%
\pgfsetstrokecolor{currentstroke}%
\pgfsetdash{}{0pt}%
\pgfpathmoveto{\pgfqpoint{0.800000in}{0.528000in}}%
\pgfpathlineto{\pgfqpoint{0.800000in}{4.224000in}}%
\pgfusepath{stroke}%
\end{pgfscope}%
\begin{pgfscope}%
\pgfsetrectcap%
\pgfsetmiterjoin%
\pgfsetlinewidth{0.803000pt}%
\definecolor{currentstroke}{rgb}{0.000000,0.000000,0.000000}%
\pgfsetstrokecolor{currentstroke}%
\pgfsetdash{}{0pt}%
\pgfpathmoveto{\pgfqpoint{5.760000in}{0.528000in}}%
\pgfpathlineto{\pgfqpoint{5.760000in}{4.224000in}}%
\pgfusepath{stroke}%
\end{pgfscope}%
\begin{pgfscope}%
\pgfsetrectcap%
\pgfsetmiterjoin%
\pgfsetlinewidth{0.803000pt}%
\definecolor{currentstroke}{rgb}{0.000000,0.000000,0.000000}%
\pgfsetstrokecolor{currentstroke}%
\pgfsetdash{}{0pt}%
\pgfpathmoveto{\pgfqpoint{0.800000in}{0.528000in}}%
\pgfpathlineto{\pgfqpoint{5.760000in}{0.528000in}}%
\pgfusepath{stroke}%
\end{pgfscope}%
\begin{pgfscope}%
\pgfsetrectcap%
\pgfsetmiterjoin%
\pgfsetlinewidth{0.803000pt}%
\definecolor{currentstroke}{rgb}{0.000000,0.000000,0.000000}%
\pgfsetstrokecolor{currentstroke}%
\pgfsetdash{}{0pt}%
\pgfpathmoveto{\pgfqpoint{0.800000in}{4.224000in}}%
\pgfpathlineto{\pgfqpoint{5.760000in}{4.224000in}}%
\pgfusepath{stroke}%
\end{pgfscope}%
\begin{pgfscope}%
\definecolor{textcolor}{rgb}{0.000000,0.000000,0.000000}%
\pgfsetstrokecolor{textcolor}%
\pgfsetfillcolor{textcolor}%
\pgftext[x=3.280000in,y=4.307333in,,base]{\color{textcolor}\ttfamily\fontsize{12.000000}{14.400000}\selectfont Bubble Sort Memory vs Input size}%
\end{pgfscope}%
\begin{pgfscope}%
\pgfsetbuttcap%
\pgfsetmiterjoin%
\definecolor{currentfill}{rgb}{1.000000,1.000000,1.000000}%
\pgfsetfillcolor{currentfill}%
\pgfsetfillopacity{0.800000}%
\pgfsetlinewidth{1.003750pt}%
\definecolor{currentstroke}{rgb}{0.800000,0.800000,0.800000}%
\pgfsetstrokecolor{currentstroke}%
\pgfsetstrokeopacity{0.800000}%
\pgfsetdash{}{0pt}%
\pgfpathmoveto{\pgfqpoint{0.897222in}{3.908286in}}%
\pgfpathlineto{\pgfqpoint{1.843376in}{3.908286in}}%
\pgfpathquadraticcurveto{\pgfqpoint{1.871153in}{3.908286in}}{\pgfqpoint{1.871153in}{3.936063in}}%
\pgfpathlineto{\pgfqpoint{1.871153in}{4.126778in}}%
\pgfpathquadraticcurveto{\pgfqpoint{1.871153in}{4.154556in}}{\pgfqpoint{1.843376in}{4.154556in}}%
\pgfpathlineto{\pgfqpoint{0.897222in}{4.154556in}}%
\pgfpathquadraticcurveto{\pgfqpoint{0.869444in}{4.154556in}}{\pgfqpoint{0.869444in}{4.126778in}}%
\pgfpathlineto{\pgfqpoint{0.869444in}{3.936063in}}%
\pgfpathquadraticcurveto{\pgfqpoint{0.869444in}{3.908286in}}{\pgfqpoint{0.897222in}{3.908286in}}%
\pgfpathlineto{\pgfqpoint{0.897222in}{3.908286in}}%
\pgfpathclose%
\pgfusepath{stroke,fill}%
\end{pgfscope}%
\begin{pgfscope}%
\pgfsetrectcap%
\pgfsetroundjoin%
\pgfsetlinewidth{1.505625pt}%
\definecolor{currentstroke}{rgb}{0.000000,1.000000,0.498039}%
\pgfsetstrokecolor{currentstroke}%
\pgfsetdash{}{0pt}%
\pgfpathmoveto{\pgfqpoint{0.925000in}{4.041342in}}%
\pgfpathlineto{\pgfqpoint{1.063889in}{4.041342in}}%
\pgfpathlineto{\pgfqpoint{1.202778in}{4.041342in}}%
\pgfusepath{stroke}%
\end{pgfscope}%
\begin{pgfscope}%
\definecolor{textcolor}{rgb}{0.000000,0.000000,0.000000}%
\pgfsetstrokecolor{textcolor}%
\pgfsetfillcolor{textcolor}%
\pgftext[x=1.313889in,y=3.992731in,left,base]{\color{textcolor}\ttfamily\fontsize{10.000000}{12.000000}\selectfont Bubble}%
\end{pgfscope}%
\end{pgfpicture}%
\makeatother%
\endgroup%

%% Creator: Matplotlib, PGF backend
%%
%% To include the figure in your LaTeX document, write
%%   \input{<filename>.pgf}
%%
%% Make sure the required packages are loaded in your preamble
%%   \usepackage{pgf}
%%
%% Also ensure that all the required font packages are loaded; for instance,
%% the lmodern package is sometimes necessary when using math font.
%%   \usepackage{lmodern}
%%
%% Figures using additional raster images can only be included by \input if
%% they are in the same directory as the main LaTeX file. For loading figures
%% from other directories you can use the `import` package
%%   \usepackage{import}
%%
%% and then include the figures with
%%   \import{<path to file>}{<filename>.pgf}
%%
%% Matplotlib used the following preamble
%%   \usepackage{fontspec}
%%   \setmainfont{DejaVuSerif.ttf}[Path=\detokenize{/home/dbk/.local/lib/python3.10/site-packages/matplotlib/mpl-data/fonts/ttf/}]
%%   \setsansfont{DejaVuSans.ttf}[Path=\detokenize{/home/dbk/.local/lib/python3.10/site-packages/matplotlib/mpl-data/fonts/ttf/}]
%%   \setmonofont{DejaVuSansMono.ttf}[Path=\detokenize{/home/dbk/.local/lib/python3.10/site-packages/matplotlib/mpl-data/fonts/ttf/}]
%%
\begingroup%
\makeatletter%
\begin{pgfpicture}%
\pgfpathrectangle{\pgfpointorigin}{\pgfqpoint{6.400000in}{4.800000in}}%
\pgfusepath{use as bounding box, clip}%
\begin{pgfscope}%
\pgfsetbuttcap%
\pgfsetmiterjoin%
\definecolor{currentfill}{rgb}{1.000000,1.000000,1.000000}%
\pgfsetfillcolor{currentfill}%
\pgfsetlinewidth{0.000000pt}%
\definecolor{currentstroke}{rgb}{1.000000,1.000000,1.000000}%
\pgfsetstrokecolor{currentstroke}%
\pgfsetdash{}{0pt}%
\pgfpathmoveto{\pgfqpoint{0.000000in}{0.000000in}}%
\pgfpathlineto{\pgfqpoint{6.400000in}{0.000000in}}%
\pgfpathlineto{\pgfqpoint{6.400000in}{4.800000in}}%
\pgfpathlineto{\pgfqpoint{0.000000in}{4.800000in}}%
\pgfpathlineto{\pgfqpoint{0.000000in}{0.000000in}}%
\pgfpathclose%
\pgfusepath{fill}%
\end{pgfscope}%
\begin{pgfscope}%
\pgfsetbuttcap%
\pgfsetmiterjoin%
\definecolor{currentfill}{rgb}{1.000000,1.000000,1.000000}%
\pgfsetfillcolor{currentfill}%
\pgfsetlinewidth{0.000000pt}%
\definecolor{currentstroke}{rgb}{0.000000,0.000000,0.000000}%
\pgfsetstrokecolor{currentstroke}%
\pgfsetstrokeopacity{0.000000}%
\pgfsetdash{}{0pt}%
\pgfpathmoveto{\pgfqpoint{0.800000in}{0.528000in}}%
\pgfpathlineto{\pgfqpoint{5.760000in}{0.528000in}}%
\pgfpathlineto{\pgfqpoint{5.760000in}{4.224000in}}%
\pgfpathlineto{\pgfqpoint{0.800000in}{4.224000in}}%
\pgfpathlineto{\pgfqpoint{0.800000in}{0.528000in}}%
\pgfpathclose%
\pgfusepath{fill}%
\end{pgfscope}%
\begin{pgfscope}%
\pgfsetbuttcap%
\pgfsetroundjoin%
\definecolor{currentfill}{rgb}{0.000000,0.000000,0.000000}%
\pgfsetfillcolor{currentfill}%
\pgfsetlinewidth{0.803000pt}%
\definecolor{currentstroke}{rgb}{0.000000,0.000000,0.000000}%
\pgfsetstrokecolor{currentstroke}%
\pgfsetdash{}{0pt}%
\pgfsys@defobject{currentmarker}{\pgfqpoint{0.000000in}{-0.048611in}}{\pgfqpoint{0.000000in}{0.000000in}}{%
\pgfpathmoveto{\pgfqpoint{0.000000in}{0.000000in}}%
\pgfpathlineto{\pgfqpoint{0.000000in}{-0.048611in}}%
\pgfusepath{stroke,fill}%
}%
\begin{pgfscope}%
\pgfsys@transformshift{1.020945in}{0.528000in}%
\pgfsys@useobject{currentmarker}{}%
\end{pgfscope}%
\end{pgfscope}%
\begin{pgfscope}%
\definecolor{textcolor}{rgb}{0.000000,0.000000,0.000000}%
\pgfsetstrokecolor{textcolor}%
\pgfsetfillcolor{textcolor}%
\pgftext[x=1.020945in,y=0.430778in,,top]{\color{textcolor}\ttfamily\fontsize{10.000000}{12.000000}\selectfont 0}%
\end{pgfscope}%
\begin{pgfscope}%
\pgfsetbuttcap%
\pgfsetroundjoin%
\definecolor{currentfill}{rgb}{0.000000,0.000000,0.000000}%
\pgfsetfillcolor{currentfill}%
\pgfsetlinewidth{0.803000pt}%
\definecolor{currentstroke}{rgb}{0.000000,0.000000,0.000000}%
\pgfsetstrokecolor{currentstroke}%
\pgfsetdash{}{0pt}%
\pgfsys@defobject{currentmarker}{\pgfqpoint{0.000000in}{-0.048611in}}{\pgfqpoint{0.000000in}{0.000000in}}{%
\pgfpathmoveto{\pgfqpoint{0.000000in}{0.000000in}}%
\pgfpathlineto{\pgfqpoint{0.000000in}{-0.048611in}}%
\pgfusepath{stroke,fill}%
}%
\begin{pgfscope}%
\pgfsys@transformshift{1.922764in}{0.528000in}%
\pgfsys@useobject{currentmarker}{}%
\end{pgfscope}%
\end{pgfscope}%
\begin{pgfscope}%
\definecolor{textcolor}{rgb}{0.000000,0.000000,0.000000}%
\pgfsetstrokecolor{textcolor}%
\pgfsetfillcolor{textcolor}%
\pgftext[x=1.922764in,y=0.430778in,,top]{\color{textcolor}\ttfamily\fontsize{10.000000}{12.000000}\selectfont 200}%
\end{pgfscope}%
\begin{pgfscope}%
\pgfsetbuttcap%
\pgfsetroundjoin%
\definecolor{currentfill}{rgb}{0.000000,0.000000,0.000000}%
\pgfsetfillcolor{currentfill}%
\pgfsetlinewidth{0.803000pt}%
\definecolor{currentstroke}{rgb}{0.000000,0.000000,0.000000}%
\pgfsetstrokecolor{currentstroke}%
\pgfsetdash{}{0pt}%
\pgfsys@defobject{currentmarker}{\pgfqpoint{0.000000in}{-0.048611in}}{\pgfqpoint{0.000000in}{0.000000in}}{%
\pgfpathmoveto{\pgfqpoint{0.000000in}{0.000000in}}%
\pgfpathlineto{\pgfqpoint{0.000000in}{-0.048611in}}%
\pgfusepath{stroke,fill}%
}%
\begin{pgfscope}%
\pgfsys@transformshift{2.824582in}{0.528000in}%
\pgfsys@useobject{currentmarker}{}%
\end{pgfscope}%
\end{pgfscope}%
\begin{pgfscope}%
\definecolor{textcolor}{rgb}{0.000000,0.000000,0.000000}%
\pgfsetstrokecolor{textcolor}%
\pgfsetfillcolor{textcolor}%
\pgftext[x=2.824582in,y=0.430778in,,top]{\color{textcolor}\ttfamily\fontsize{10.000000}{12.000000}\selectfont 400}%
\end{pgfscope}%
\begin{pgfscope}%
\pgfsetbuttcap%
\pgfsetroundjoin%
\definecolor{currentfill}{rgb}{0.000000,0.000000,0.000000}%
\pgfsetfillcolor{currentfill}%
\pgfsetlinewidth{0.803000pt}%
\definecolor{currentstroke}{rgb}{0.000000,0.000000,0.000000}%
\pgfsetstrokecolor{currentstroke}%
\pgfsetdash{}{0pt}%
\pgfsys@defobject{currentmarker}{\pgfqpoint{0.000000in}{-0.048611in}}{\pgfqpoint{0.000000in}{0.000000in}}{%
\pgfpathmoveto{\pgfqpoint{0.000000in}{0.000000in}}%
\pgfpathlineto{\pgfqpoint{0.000000in}{-0.048611in}}%
\pgfusepath{stroke,fill}%
}%
\begin{pgfscope}%
\pgfsys@transformshift{3.726400in}{0.528000in}%
\pgfsys@useobject{currentmarker}{}%
\end{pgfscope}%
\end{pgfscope}%
\begin{pgfscope}%
\definecolor{textcolor}{rgb}{0.000000,0.000000,0.000000}%
\pgfsetstrokecolor{textcolor}%
\pgfsetfillcolor{textcolor}%
\pgftext[x=3.726400in,y=0.430778in,,top]{\color{textcolor}\ttfamily\fontsize{10.000000}{12.000000}\selectfont 600}%
\end{pgfscope}%
\begin{pgfscope}%
\pgfsetbuttcap%
\pgfsetroundjoin%
\definecolor{currentfill}{rgb}{0.000000,0.000000,0.000000}%
\pgfsetfillcolor{currentfill}%
\pgfsetlinewidth{0.803000pt}%
\definecolor{currentstroke}{rgb}{0.000000,0.000000,0.000000}%
\pgfsetstrokecolor{currentstroke}%
\pgfsetdash{}{0pt}%
\pgfsys@defobject{currentmarker}{\pgfqpoint{0.000000in}{-0.048611in}}{\pgfqpoint{0.000000in}{0.000000in}}{%
\pgfpathmoveto{\pgfqpoint{0.000000in}{0.000000in}}%
\pgfpathlineto{\pgfqpoint{0.000000in}{-0.048611in}}%
\pgfusepath{stroke,fill}%
}%
\begin{pgfscope}%
\pgfsys@transformshift{4.628218in}{0.528000in}%
\pgfsys@useobject{currentmarker}{}%
\end{pgfscope}%
\end{pgfscope}%
\begin{pgfscope}%
\definecolor{textcolor}{rgb}{0.000000,0.000000,0.000000}%
\pgfsetstrokecolor{textcolor}%
\pgfsetfillcolor{textcolor}%
\pgftext[x=4.628218in,y=0.430778in,,top]{\color{textcolor}\ttfamily\fontsize{10.000000}{12.000000}\selectfont 800}%
\end{pgfscope}%
\begin{pgfscope}%
\pgfsetbuttcap%
\pgfsetroundjoin%
\definecolor{currentfill}{rgb}{0.000000,0.000000,0.000000}%
\pgfsetfillcolor{currentfill}%
\pgfsetlinewidth{0.803000pt}%
\definecolor{currentstroke}{rgb}{0.000000,0.000000,0.000000}%
\pgfsetstrokecolor{currentstroke}%
\pgfsetdash{}{0pt}%
\pgfsys@defobject{currentmarker}{\pgfqpoint{0.000000in}{-0.048611in}}{\pgfqpoint{0.000000in}{0.000000in}}{%
\pgfpathmoveto{\pgfqpoint{0.000000in}{0.000000in}}%
\pgfpathlineto{\pgfqpoint{0.000000in}{-0.048611in}}%
\pgfusepath{stroke,fill}%
}%
\begin{pgfscope}%
\pgfsys@transformshift{5.530036in}{0.528000in}%
\pgfsys@useobject{currentmarker}{}%
\end{pgfscope}%
\end{pgfscope}%
\begin{pgfscope}%
\definecolor{textcolor}{rgb}{0.000000,0.000000,0.000000}%
\pgfsetstrokecolor{textcolor}%
\pgfsetfillcolor{textcolor}%
\pgftext[x=5.530036in,y=0.430778in,,top]{\color{textcolor}\ttfamily\fontsize{10.000000}{12.000000}\selectfont 1000}%
\end{pgfscope}%
\begin{pgfscope}%
\definecolor{textcolor}{rgb}{0.000000,0.000000,0.000000}%
\pgfsetstrokecolor{textcolor}%
\pgfsetfillcolor{textcolor}%
\pgftext[x=3.280000in,y=0.240063in,,top]{\color{textcolor}\ttfamily\fontsize{10.000000}{12.000000}\selectfont Size of Array}%
\end{pgfscope}%
\begin{pgfscope}%
\pgfsetbuttcap%
\pgfsetroundjoin%
\definecolor{currentfill}{rgb}{0.000000,0.000000,0.000000}%
\pgfsetfillcolor{currentfill}%
\pgfsetlinewidth{0.803000pt}%
\definecolor{currentstroke}{rgb}{0.000000,0.000000,0.000000}%
\pgfsetstrokecolor{currentstroke}%
\pgfsetdash{}{0pt}%
\pgfsys@defobject{currentmarker}{\pgfqpoint{-0.048611in}{0.000000in}}{\pgfqpoint{-0.000000in}{0.000000in}}{%
\pgfpathmoveto{\pgfqpoint{-0.000000in}{0.000000in}}%
\pgfpathlineto{\pgfqpoint{-0.048611in}{0.000000in}}%
\pgfusepath{stroke,fill}%
}%
\begin{pgfscope}%
\pgfsys@transformshift{0.800000in}{0.668257in}%
\pgfsys@useobject{currentmarker}{}%
\end{pgfscope}%
\end{pgfscope}%
\begin{pgfscope}%
\definecolor{textcolor}{rgb}{0.000000,0.000000,0.000000}%
\pgfsetstrokecolor{textcolor}%
\pgfsetfillcolor{textcolor}%
\pgftext[x=0.619160in, y=0.615122in, left, base]{\color{textcolor}\ttfamily\fontsize{10.000000}{12.000000}\selectfont 0}%
\end{pgfscope}%
\begin{pgfscope}%
\pgfsetbuttcap%
\pgfsetroundjoin%
\definecolor{currentfill}{rgb}{0.000000,0.000000,0.000000}%
\pgfsetfillcolor{currentfill}%
\pgfsetlinewidth{0.803000pt}%
\definecolor{currentstroke}{rgb}{0.000000,0.000000,0.000000}%
\pgfsetstrokecolor{currentstroke}%
\pgfsetdash{}{0pt}%
\pgfsys@defobject{currentmarker}{\pgfqpoint{-0.048611in}{0.000000in}}{\pgfqpoint{-0.000000in}{0.000000in}}{%
\pgfpathmoveto{\pgfqpoint{-0.000000in}{0.000000in}}%
\pgfpathlineto{\pgfqpoint{-0.048611in}{0.000000in}}%
\pgfusepath{stroke,fill}%
}%
\begin{pgfscope}%
\pgfsys@transformshift{0.800000in}{1.228724in}%
\pgfsys@useobject{currentmarker}{}%
\end{pgfscope}%
\end{pgfscope}%
\begin{pgfscope}%
\definecolor{textcolor}{rgb}{0.000000,0.000000,0.000000}%
\pgfsetstrokecolor{textcolor}%
\pgfsetfillcolor{textcolor}%
\pgftext[x=0.201069in, y=1.175589in, left, base]{\color{textcolor}\ttfamily\fontsize{10.000000}{12.000000}\selectfont 100000}%
\end{pgfscope}%
\begin{pgfscope}%
\pgfsetbuttcap%
\pgfsetroundjoin%
\definecolor{currentfill}{rgb}{0.000000,0.000000,0.000000}%
\pgfsetfillcolor{currentfill}%
\pgfsetlinewidth{0.803000pt}%
\definecolor{currentstroke}{rgb}{0.000000,0.000000,0.000000}%
\pgfsetstrokecolor{currentstroke}%
\pgfsetdash{}{0pt}%
\pgfsys@defobject{currentmarker}{\pgfqpoint{-0.048611in}{0.000000in}}{\pgfqpoint{-0.000000in}{0.000000in}}{%
\pgfpathmoveto{\pgfqpoint{-0.000000in}{0.000000in}}%
\pgfpathlineto{\pgfqpoint{-0.048611in}{0.000000in}}%
\pgfusepath{stroke,fill}%
}%
\begin{pgfscope}%
\pgfsys@transformshift{0.800000in}{1.789191in}%
\pgfsys@useobject{currentmarker}{}%
\end{pgfscope}%
\end{pgfscope}%
\begin{pgfscope}%
\definecolor{textcolor}{rgb}{0.000000,0.000000,0.000000}%
\pgfsetstrokecolor{textcolor}%
\pgfsetfillcolor{textcolor}%
\pgftext[x=0.201069in, y=1.736056in, left, base]{\color{textcolor}\ttfamily\fontsize{10.000000}{12.000000}\selectfont 200000}%
\end{pgfscope}%
\begin{pgfscope}%
\pgfsetbuttcap%
\pgfsetroundjoin%
\definecolor{currentfill}{rgb}{0.000000,0.000000,0.000000}%
\pgfsetfillcolor{currentfill}%
\pgfsetlinewidth{0.803000pt}%
\definecolor{currentstroke}{rgb}{0.000000,0.000000,0.000000}%
\pgfsetstrokecolor{currentstroke}%
\pgfsetdash{}{0pt}%
\pgfsys@defobject{currentmarker}{\pgfqpoint{-0.048611in}{0.000000in}}{\pgfqpoint{-0.000000in}{0.000000in}}{%
\pgfpathmoveto{\pgfqpoint{-0.000000in}{0.000000in}}%
\pgfpathlineto{\pgfqpoint{-0.048611in}{0.000000in}}%
\pgfusepath{stroke,fill}%
}%
\begin{pgfscope}%
\pgfsys@transformshift{0.800000in}{2.349658in}%
\pgfsys@useobject{currentmarker}{}%
\end{pgfscope}%
\end{pgfscope}%
\begin{pgfscope}%
\definecolor{textcolor}{rgb}{0.000000,0.000000,0.000000}%
\pgfsetstrokecolor{textcolor}%
\pgfsetfillcolor{textcolor}%
\pgftext[x=0.201069in, y=2.296524in, left, base]{\color{textcolor}\ttfamily\fontsize{10.000000}{12.000000}\selectfont 300000}%
\end{pgfscope}%
\begin{pgfscope}%
\pgfsetbuttcap%
\pgfsetroundjoin%
\definecolor{currentfill}{rgb}{0.000000,0.000000,0.000000}%
\pgfsetfillcolor{currentfill}%
\pgfsetlinewidth{0.803000pt}%
\definecolor{currentstroke}{rgb}{0.000000,0.000000,0.000000}%
\pgfsetstrokecolor{currentstroke}%
\pgfsetdash{}{0pt}%
\pgfsys@defobject{currentmarker}{\pgfqpoint{-0.048611in}{0.000000in}}{\pgfqpoint{-0.000000in}{0.000000in}}{%
\pgfpathmoveto{\pgfqpoint{-0.000000in}{0.000000in}}%
\pgfpathlineto{\pgfqpoint{-0.048611in}{0.000000in}}%
\pgfusepath{stroke,fill}%
}%
\begin{pgfscope}%
\pgfsys@transformshift{0.800000in}{2.910125in}%
\pgfsys@useobject{currentmarker}{}%
\end{pgfscope}%
\end{pgfscope}%
\begin{pgfscope}%
\definecolor{textcolor}{rgb}{0.000000,0.000000,0.000000}%
\pgfsetstrokecolor{textcolor}%
\pgfsetfillcolor{textcolor}%
\pgftext[x=0.201069in, y=2.856991in, left, base]{\color{textcolor}\ttfamily\fontsize{10.000000}{12.000000}\selectfont 400000}%
\end{pgfscope}%
\begin{pgfscope}%
\pgfsetbuttcap%
\pgfsetroundjoin%
\definecolor{currentfill}{rgb}{0.000000,0.000000,0.000000}%
\pgfsetfillcolor{currentfill}%
\pgfsetlinewidth{0.803000pt}%
\definecolor{currentstroke}{rgb}{0.000000,0.000000,0.000000}%
\pgfsetstrokecolor{currentstroke}%
\pgfsetdash{}{0pt}%
\pgfsys@defobject{currentmarker}{\pgfqpoint{-0.048611in}{0.000000in}}{\pgfqpoint{-0.000000in}{0.000000in}}{%
\pgfpathmoveto{\pgfqpoint{-0.000000in}{0.000000in}}%
\pgfpathlineto{\pgfqpoint{-0.048611in}{0.000000in}}%
\pgfusepath{stroke,fill}%
}%
\begin{pgfscope}%
\pgfsys@transformshift{0.800000in}{3.470592in}%
\pgfsys@useobject{currentmarker}{}%
\end{pgfscope}%
\end{pgfscope}%
\begin{pgfscope}%
\definecolor{textcolor}{rgb}{0.000000,0.000000,0.000000}%
\pgfsetstrokecolor{textcolor}%
\pgfsetfillcolor{textcolor}%
\pgftext[x=0.201069in, y=3.417458in, left, base]{\color{textcolor}\ttfamily\fontsize{10.000000}{12.000000}\selectfont 500000}%
\end{pgfscope}%
\begin{pgfscope}%
\pgfsetbuttcap%
\pgfsetroundjoin%
\definecolor{currentfill}{rgb}{0.000000,0.000000,0.000000}%
\pgfsetfillcolor{currentfill}%
\pgfsetlinewidth{0.803000pt}%
\definecolor{currentstroke}{rgb}{0.000000,0.000000,0.000000}%
\pgfsetstrokecolor{currentstroke}%
\pgfsetdash{}{0pt}%
\pgfsys@defobject{currentmarker}{\pgfqpoint{-0.048611in}{0.000000in}}{\pgfqpoint{-0.000000in}{0.000000in}}{%
\pgfpathmoveto{\pgfqpoint{-0.000000in}{0.000000in}}%
\pgfpathlineto{\pgfqpoint{-0.048611in}{0.000000in}}%
\pgfusepath{stroke,fill}%
}%
\begin{pgfscope}%
\pgfsys@transformshift{0.800000in}{4.031059in}%
\pgfsys@useobject{currentmarker}{}%
\end{pgfscope}%
\end{pgfscope}%
\begin{pgfscope}%
\definecolor{textcolor}{rgb}{0.000000,0.000000,0.000000}%
\pgfsetstrokecolor{textcolor}%
\pgfsetfillcolor{textcolor}%
\pgftext[x=0.201069in, y=3.977925in, left, base]{\color{textcolor}\ttfamily\fontsize{10.000000}{12.000000}\selectfont 600000}%
\end{pgfscope}%
\begin{pgfscope}%
\definecolor{textcolor}{rgb}{0.000000,0.000000,0.000000}%
\pgfsetstrokecolor{textcolor}%
\pgfsetfillcolor{textcolor}%
\pgftext[x=0.145513in,y=2.376000in,,bottom,rotate=90.000000]{\color{textcolor}\ttfamily\fontsize{10.000000}{12.000000}\selectfont Comparisons}%
\end{pgfscope}%
\begin{pgfscope}%
\pgfpathrectangle{\pgfqpoint{0.800000in}{0.528000in}}{\pgfqpoint{4.960000in}{3.696000in}}%
\pgfusepath{clip}%
\pgfsetrectcap%
\pgfsetroundjoin%
\pgfsetlinewidth{1.505625pt}%
\definecolor{currentstroke}{rgb}{0.000000,1.000000,0.498039}%
\pgfsetstrokecolor{currentstroke}%
\pgfsetdash{}{0pt}%
\pgfpathmoveto{\pgfqpoint{1.025455in}{0.696000in}}%
\pgfpathlineto{\pgfqpoint{1.115636in}{0.708274in}}%
\pgfpathlineto{\pgfqpoint{1.205818in}{0.722790in}}%
\pgfpathlineto{\pgfqpoint{1.296000in}{0.739548in}}%
\pgfpathlineto{\pgfqpoint{1.386182in}{0.758548in}}%
\pgfpathlineto{\pgfqpoint{1.476364in}{0.779790in}}%
\pgfpathlineto{\pgfqpoint{1.566545in}{0.803273in}}%
\pgfpathlineto{\pgfqpoint{1.656727in}{0.828999in}}%
\pgfpathlineto{\pgfqpoint{1.746909in}{0.856966in}}%
\pgfpathlineto{\pgfqpoint{1.837091in}{0.887175in}}%
\pgfpathlineto{\pgfqpoint{1.927273in}{0.919626in}}%
\pgfpathlineto{\pgfqpoint{2.021964in}{0.956113in}}%
\pgfpathlineto{\pgfqpoint{2.116655in}{0.995071in}}%
\pgfpathlineto{\pgfqpoint{2.211345in}{1.036501in}}%
\pgfpathlineto{\pgfqpoint{2.306036in}{1.080402in}}%
\pgfpathlineto{\pgfqpoint{2.400727in}{1.126775in}}%
\pgfpathlineto{\pgfqpoint{2.495418in}{1.175620in}}%
\pgfpathlineto{\pgfqpoint{2.590109in}{1.226936in}}%
\pgfpathlineto{\pgfqpoint{2.684800in}{1.280724in}}%
\pgfpathlineto{\pgfqpoint{2.779491in}{1.336984in}}%
\pgfpathlineto{\pgfqpoint{2.878691in}{1.398573in}}%
\pgfpathlineto{\pgfqpoint{2.977891in}{1.462876in}}%
\pgfpathlineto{\pgfqpoint{3.077091in}{1.529891in}}%
\pgfpathlineto{\pgfqpoint{3.176291in}{1.599619in}}%
\pgfpathlineto{\pgfqpoint{3.275491in}{1.672059in}}%
\pgfpathlineto{\pgfqpoint{3.374691in}{1.747212in}}%
\pgfpathlineto{\pgfqpoint{3.473891in}{1.825078in}}%
\pgfpathlineto{\pgfqpoint{3.577600in}{1.909383in}}%
\pgfpathlineto{\pgfqpoint{3.681309in}{1.996653in}}%
\pgfpathlineto{\pgfqpoint{3.785018in}{2.086889in}}%
\pgfpathlineto{\pgfqpoint{3.888727in}{2.180089in}}%
\pgfpathlineto{\pgfqpoint{3.992436in}{2.276254in}}%
\pgfpathlineto{\pgfqpoint{4.096145in}{2.375383in}}%
\pgfpathlineto{\pgfqpoint{4.204364in}{2.481984in}}%
\pgfpathlineto{\pgfqpoint{4.312582in}{2.591813in}}%
\pgfpathlineto{\pgfqpoint{4.420800in}{2.704871in}}%
\pgfpathlineto{\pgfqpoint{4.529018in}{2.821157in}}%
\pgfpathlineto{\pgfqpoint{4.637236in}{2.940671in}}%
\pgfpathlineto{\pgfqpoint{4.745455in}{3.063413in}}%
\pgfpathlineto{\pgfqpoint{4.858182in}{3.194702in}}%
\pgfpathlineto{\pgfqpoint{4.970909in}{3.329495in}}%
\pgfpathlineto{\pgfqpoint{5.083636in}{3.467790in}}%
\pgfpathlineto{\pgfqpoint{5.196364in}{3.609588in}}%
\pgfpathlineto{\pgfqpoint{5.309091in}{3.754889in}}%
\pgfpathlineto{\pgfqpoint{5.426327in}{3.909718in}}%
\pgfpathlineto{\pgfqpoint{5.534545in}{4.056000in}}%
\pgfpathlineto{\pgfqpoint{5.534545in}{4.056000in}}%
\pgfusepath{stroke}%
\end{pgfscope}%
\begin{pgfscope}%
\pgfsetrectcap%
\pgfsetmiterjoin%
\pgfsetlinewidth{0.803000pt}%
\definecolor{currentstroke}{rgb}{0.000000,0.000000,0.000000}%
\pgfsetstrokecolor{currentstroke}%
\pgfsetdash{}{0pt}%
\pgfpathmoveto{\pgfqpoint{0.800000in}{0.528000in}}%
\pgfpathlineto{\pgfqpoint{0.800000in}{4.224000in}}%
\pgfusepath{stroke}%
\end{pgfscope}%
\begin{pgfscope}%
\pgfsetrectcap%
\pgfsetmiterjoin%
\pgfsetlinewidth{0.803000pt}%
\definecolor{currentstroke}{rgb}{0.000000,0.000000,0.000000}%
\pgfsetstrokecolor{currentstroke}%
\pgfsetdash{}{0pt}%
\pgfpathmoveto{\pgfqpoint{5.760000in}{0.528000in}}%
\pgfpathlineto{\pgfqpoint{5.760000in}{4.224000in}}%
\pgfusepath{stroke}%
\end{pgfscope}%
\begin{pgfscope}%
\pgfsetrectcap%
\pgfsetmiterjoin%
\pgfsetlinewidth{0.803000pt}%
\definecolor{currentstroke}{rgb}{0.000000,0.000000,0.000000}%
\pgfsetstrokecolor{currentstroke}%
\pgfsetdash{}{0pt}%
\pgfpathmoveto{\pgfqpoint{0.800000in}{0.528000in}}%
\pgfpathlineto{\pgfqpoint{5.760000in}{0.528000in}}%
\pgfusepath{stroke}%
\end{pgfscope}%
\begin{pgfscope}%
\pgfsetrectcap%
\pgfsetmiterjoin%
\pgfsetlinewidth{0.803000pt}%
\definecolor{currentstroke}{rgb}{0.000000,0.000000,0.000000}%
\pgfsetstrokecolor{currentstroke}%
\pgfsetdash{}{0pt}%
\pgfpathmoveto{\pgfqpoint{0.800000in}{4.224000in}}%
\pgfpathlineto{\pgfqpoint{5.760000in}{4.224000in}}%
\pgfusepath{stroke}%
\end{pgfscope}%
\begin{pgfscope}%
\definecolor{textcolor}{rgb}{0.000000,0.000000,0.000000}%
\pgfsetstrokecolor{textcolor}%
\pgfsetfillcolor{textcolor}%
\pgftext[x=3.280000in,y=4.307333in,,base]{\color{textcolor}\ttfamily\fontsize{12.000000}{14.400000}\selectfont Bubble Sort Comparisons vs Input size}%
\end{pgfscope}%
\begin{pgfscope}%
\pgfsetbuttcap%
\pgfsetmiterjoin%
\definecolor{currentfill}{rgb}{1.000000,1.000000,1.000000}%
\pgfsetfillcolor{currentfill}%
\pgfsetfillopacity{0.800000}%
\pgfsetlinewidth{1.003750pt}%
\definecolor{currentstroke}{rgb}{0.800000,0.800000,0.800000}%
\pgfsetstrokecolor{currentstroke}%
\pgfsetstrokeopacity{0.800000}%
\pgfsetdash{}{0pt}%
\pgfpathmoveto{\pgfqpoint{0.897222in}{3.908286in}}%
\pgfpathlineto{\pgfqpoint{1.843376in}{3.908286in}}%
\pgfpathquadraticcurveto{\pgfqpoint{1.871153in}{3.908286in}}{\pgfqpoint{1.871153in}{3.936063in}}%
\pgfpathlineto{\pgfqpoint{1.871153in}{4.126778in}}%
\pgfpathquadraticcurveto{\pgfqpoint{1.871153in}{4.154556in}}{\pgfqpoint{1.843376in}{4.154556in}}%
\pgfpathlineto{\pgfqpoint{0.897222in}{4.154556in}}%
\pgfpathquadraticcurveto{\pgfqpoint{0.869444in}{4.154556in}}{\pgfqpoint{0.869444in}{4.126778in}}%
\pgfpathlineto{\pgfqpoint{0.869444in}{3.936063in}}%
\pgfpathquadraticcurveto{\pgfqpoint{0.869444in}{3.908286in}}{\pgfqpoint{0.897222in}{3.908286in}}%
\pgfpathlineto{\pgfqpoint{0.897222in}{3.908286in}}%
\pgfpathclose%
\pgfusepath{stroke,fill}%
\end{pgfscope}%
\begin{pgfscope}%
\pgfsetrectcap%
\pgfsetroundjoin%
\pgfsetlinewidth{1.505625pt}%
\definecolor{currentstroke}{rgb}{0.000000,1.000000,0.498039}%
\pgfsetstrokecolor{currentstroke}%
\pgfsetdash{}{0pt}%
\pgfpathmoveto{\pgfqpoint{0.925000in}{4.041342in}}%
\pgfpathlineto{\pgfqpoint{1.063889in}{4.041342in}}%
\pgfpathlineto{\pgfqpoint{1.202778in}{4.041342in}}%
\pgfusepath{stroke}%
\end{pgfscope}%
\begin{pgfscope}%
\definecolor{textcolor}{rgb}{0.000000,0.000000,0.000000}%
\pgfsetstrokecolor{textcolor}%
\pgfsetfillcolor{textcolor}%
\pgftext[x=1.313889in,y=3.992731in,left,base]{\color{textcolor}\ttfamily\fontsize{10.000000}{12.000000}\selectfont Bubble}%
\end{pgfscope}%
\end{pgfpicture}%
\makeatother%
\endgroup%

%% Creator: Matplotlib, PGF backend
%%
%% To include the figure in your LaTeX document, write
%%   \input{<filename>.pgf}
%%
%% Make sure the required packages are loaded in your preamble
%%   \usepackage{pgf}
%%
%% Also ensure that all the required font packages are loaded; for instance,
%% the lmodern package is sometimes necessary when using math font.
%%   \usepackage{lmodern}
%%
%% Figures using additional raster images can only be included by \input if
%% they are in the same directory as the main LaTeX file. For loading figures
%% from other directories you can use the `import` package
%%   \usepackage{import}
%%
%% and then include the figures with
%%   \import{<path to file>}{<filename>.pgf}
%%
%% Matplotlib used the following preamble
%%   \usepackage{fontspec}
%%   \setmainfont{DejaVuSerif.ttf}[Path=\detokenize{/home/dbk/.local/lib/python3.10/site-packages/matplotlib/mpl-data/fonts/ttf/}]
%%   \setsansfont{DejaVuSans.ttf}[Path=\detokenize{/home/dbk/.local/lib/python3.10/site-packages/matplotlib/mpl-data/fonts/ttf/}]
%%   \setmonofont{DejaVuSansMono.ttf}[Path=\detokenize{/home/dbk/.local/lib/python3.10/site-packages/matplotlib/mpl-data/fonts/ttf/}]
%%
\begingroup%
\makeatletter%
\begin{pgfpicture}%
\pgfpathrectangle{\pgfpointorigin}{\pgfqpoint{6.400000in}{4.800000in}}%
\pgfusepath{use as bounding box, clip}%
\begin{pgfscope}%
\pgfsetbuttcap%
\pgfsetmiterjoin%
\definecolor{currentfill}{rgb}{1.000000,1.000000,1.000000}%
\pgfsetfillcolor{currentfill}%
\pgfsetlinewidth{0.000000pt}%
\definecolor{currentstroke}{rgb}{1.000000,1.000000,1.000000}%
\pgfsetstrokecolor{currentstroke}%
\pgfsetdash{}{0pt}%
\pgfpathmoveto{\pgfqpoint{0.000000in}{0.000000in}}%
\pgfpathlineto{\pgfqpoint{6.400000in}{0.000000in}}%
\pgfpathlineto{\pgfqpoint{6.400000in}{4.800000in}}%
\pgfpathlineto{\pgfqpoint{0.000000in}{4.800000in}}%
\pgfpathlineto{\pgfqpoint{0.000000in}{0.000000in}}%
\pgfpathclose%
\pgfusepath{fill}%
\end{pgfscope}%
\begin{pgfscope}%
\pgfsetbuttcap%
\pgfsetmiterjoin%
\definecolor{currentfill}{rgb}{1.000000,1.000000,1.000000}%
\pgfsetfillcolor{currentfill}%
\pgfsetlinewidth{0.000000pt}%
\definecolor{currentstroke}{rgb}{0.000000,0.000000,0.000000}%
\pgfsetstrokecolor{currentstroke}%
\pgfsetstrokeopacity{0.000000}%
\pgfsetdash{}{0pt}%
\pgfpathmoveto{\pgfqpoint{0.800000in}{0.528000in}}%
\pgfpathlineto{\pgfqpoint{5.760000in}{0.528000in}}%
\pgfpathlineto{\pgfqpoint{5.760000in}{4.224000in}}%
\pgfpathlineto{\pgfqpoint{0.800000in}{4.224000in}}%
\pgfpathlineto{\pgfqpoint{0.800000in}{0.528000in}}%
\pgfpathclose%
\pgfusepath{fill}%
\end{pgfscope}%
\begin{pgfscope}%
\pgfsetbuttcap%
\pgfsetroundjoin%
\definecolor{currentfill}{rgb}{0.000000,0.000000,0.000000}%
\pgfsetfillcolor{currentfill}%
\pgfsetlinewidth{0.803000pt}%
\definecolor{currentstroke}{rgb}{0.000000,0.000000,0.000000}%
\pgfsetstrokecolor{currentstroke}%
\pgfsetdash{}{0pt}%
\pgfsys@defobject{currentmarker}{\pgfqpoint{0.000000in}{-0.048611in}}{\pgfqpoint{0.000000in}{0.000000in}}{%
\pgfpathmoveto{\pgfqpoint{0.000000in}{0.000000in}}%
\pgfpathlineto{\pgfqpoint{0.000000in}{-0.048611in}}%
\pgfusepath{stroke,fill}%
}%
\begin{pgfscope}%
\pgfsys@transformshift{1.020945in}{0.528000in}%
\pgfsys@useobject{currentmarker}{}%
\end{pgfscope}%
\end{pgfscope}%
\begin{pgfscope}%
\definecolor{textcolor}{rgb}{0.000000,0.000000,0.000000}%
\pgfsetstrokecolor{textcolor}%
\pgfsetfillcolor{textcolor}%
\pgftext[x=1.020945in,y=0.430778in,,top]{\color{textcolor}\ttfamily\fontsize{10.000000}{12.000000}\selectfont 0}%
\end{pgfscope}%
\begin{pgfscope}%
\pgfsetbuttcap%
\pgfsetroundjoin%
\definecolor{currentfill}{rgb}{0.000000,0.000000,0.000000}%
\pgfsetfillcolor{currentfill}%
\pgfsetlinewidth{0.803000pt}%
\definecolor{currentstroke}{rgb}{0.000000,0.000000,0.000000}%
\pgfsetstrokecolor{currentstroke}%
\pgfsetdash{}{0pt}%
\pgfsys@defobject{currentmarker}{\pgfqpoint{0.000000in}{-0.048611in}}{\pgfqpoint{0.000000in}{0.000000in}}{%
\pgfpathmoveto{\pgfqpoint{0.000000in}{0.000000in}}%
\pgfpathlineto{\pgfqpoint{0.000000in}{-0.048611in}}%
\pgfusepath{stroke,fill}%
}%
\begin{pgfscope}%
\pgfsys@transformshift{1.922764in}{0.528000in}%
\pgfsys@useobject{currentmarker}{}%
\end{pgfscope}%
\end{pgfscope}%
\begin{pgfscope}%
\definecolor{textcolor}{rgb}{0.000000,0.000000,0.000000}%
\pgfsetstrokecolor{textcolor}%
\pgfsetfillcolor{textcolor}%
\pgftext[x=1.922764in,y=0.430778in,,top]{\color{textcolor}\ttfamily\fontsize{10.000000}{12.000000}\selectfont 200}%
\end{pgfscope}%
\begin{pgfscope}%
\pgfsetbuttcap%
\pgfsetroundjoin%
\definecolor{currentfill}{rgb}{0.000000,0.000000,0.000000}%
\pgfsetfillcolor{currentfill}%
\pgfsetlinewidth{0.803000pt}%
\definecolor{currentstroke}{rgb}{0.000000,0.000000,0.000000}%
\pgfsetstrokecolor{currentstroke}%
\pgfsetdash{}{0pt}%
\pgfsys@defobject{currentmarker}{\pgfqpoint{0.000000in}{-0.048611in}}{\pgfqpoint{0.000000in}{0.000000in}}{%
\pgfpathmoveto{\pgfqpoint{0.000000in}{0.000000in}}%
\pgfpathlineto{\pgfqpoint{0.000000in}{-0.048611in}}%
\pgfusepath{stroke,fill}%
}%
\begin{pgfscope}%
\pgfsys@transformshift{2.824582in}{0.528000in}%
\pgfsys@useobject{currentmarker}{}%
\end{pgfscope}%
\end{pgfscope}%
\begin{pgfscope}%
\definecolor{textcolor}{rgb}{0.000000,0.000000,0.000000}%
\pgfsetstrokecolor{textcolor}%
\pgfsetfillcolor{textcolor}%
\pgftext[x=2.824582in,y=0.430778in,,top]{\color{textcolor}\ttfamily\fontsize{10.000000}{12.000000}\selectfont 400}%
\end{pgfscope}%
\begin{pgfscope}%
\pgfsetbuttcap%
\pgfsetroundjoin%
\definecolor{currentfill}{rgb}{0.000000,0.000000,0.000000}%
\pgfsetfillcolor{currentfill}%
\pgfsetlinewidth{0.803000pt}%
\definecolor{currentstroke}{rgb}{0.000000,0.000000,0.000000}%
\pgfsetstrokecolor{currentstroke}%
\pgfsetdash{}{0pt}%
\pgfsys@defobject{currentmarker}{\pgfqpoint{0.000000in}{-0.048611in}}{\pgfqpoint{0.000000in}{0.000000in}}{%
\pgfpathmoveto{\pgfqpoint{0.000000in}{0.000000in}}%
\pgfpathlineto{\pgfqpoint{0.000000in}{-0.048611in}}%
\pgfusepath{stroke,fill}%
}%
\begin{pgfscope}%
\pgfsys@transformshift{3.726400in}{0.528000in}%
\pgfsys@useobject{currentmarker}{}%
\end{pgfscope}%
\end{pgfscope}%
\begin{pgfscope}%
\definecolor{textcolor}{rgb}{0.000000,0.000000,0.000000}%
\pgfsetstrokecolor{textcolor}%
\pgfsetfillcolor{textcolor}%
\pgftext[x=3.726400in,y=0.430778in,,top]{\color{textcolor}\ttfamily\fontsize{10.000000}{12.000000}\selectfont 600}%
\end{pgfscope}%
\begin{pgfscope}%
\pgfsetbuttcap%
\pgfsetroundjoin%
\definecolor{currentfill}{rgb}{0.000000,0.000000,0.000000}%
\pgfsetfillcolor{currentfill}%
\pgfsetlinewidth{0.803000pt}%
\definecolor{currentstroke}{rgb}{0.000000,0.000000,0.000000}%
\pgfsetstrokecolor{currentstroke}%
\pgfsetdash{}{0pt}%
\pgfsys@defobject{currentmarker}{\pgfqpoint{0.000000in}{-0.048611in}}{\pgfqpoint{0.000000in}{0.000000in}}{%
\pgfpathmoveto{\pgfqpoint{0.000000in}{0.000000in}}%
\pgfpathlineto{\pgfqpoint{0.000000in}{-0.048611in}}%
\pgfusepath{stroke,fill}%
}%
\begin{pgfscope}%
\pgfsys@transformshift{4.628218in}{0.528000in}%
\pgfsys@useobject{currentmarker}{}%
\end{pgfscope}%
\end{pgfscope}%
\begin{pgfscope}%
\definecolor{textcolor}{rgb}{0.000000,0.000000,0.000000}%
\pgfsetstrokecolor{textcolor}%
\pgfsetfillcolor{textcolor}%
\pgftext[x=4.628218in,y=0.430778in,,top]{\color{textcolor}\ttfamily\fontsize{10.000000}{12.000000}\selectfont 800}%
\end{pgfscope}%
\begin{pgfscope}%
\pgfsetbuttcap%
\pgfsetroundjoin%
\definecolor{currentfill}{rgb}{0.000000,0.000000,0.000000}%
\pgfsetfillcolor{currentfill}%
\pgfsetlinewidth{0.803000pt}%
\definecolor{currentstroke}{rgb}{0.000000,0.000000,0.000000}%
\pgfsetstrokecolor{currentstroke}%
\pgfsetdash{}{0pt}%
\pgfsys@defobject{currentmarker}{\pgfqpoint{0.000000in}{-0.048611in}}{\pgfqpoint{0.000000in}{0.000000in}}{%
\pgfpathmoveto{\pgfqpoint{0.000000in}{0.000000in}}%
\pgfpathlineto{\pgfqpoint{0.000000in}{-0.048611in}}%
\pgfusepath{stroke,fill}%
}%
\begin{pgfscope}%
\pgfsys@transformshift{5.530036in}{0.528000in}%
\pgfsys@useobject{currentmarker}{}%
\end{pgfscope}%
\end{pgfscope}%
\begin{pgfscope}%
\definecolor{textcolor}{rgb}{0.000000,0.000000,0.000000}%
\pgfsetstrokecolor{textcolor}%
\pgfsetfillcolor{textcolor}%
\pgftext[x=5.530036in,y=0.430778in,,top]{\color{textcolor}\ttfamily\fontsize{10.000000}{12.000000}\selectfont 1000}%
\end{pgfscope}%
\begin{pgfscope}%
\definecolor{textcolor}{rgb}{0.000000,0.000000,0.000000}%
\pgfsetstrokecolor{textcolor}%
\pgfsetfillcolor{textcolor}%
\pgftext[x=3.280000in,y=0.240063in,,top]{\color{textcolor}\ttfamily\fontsize{10.000000}{12.000000}\selectfont Size of Array}%
\end{pgfscope}%
\begin{pgfscope}%
\pgfsetbuttcap%
\pgfsetroundjoin%
\definecolor{currentfill}{rgb}{0.000000,0.000000,0.000000}%
\pgfsetfillcolor{currentfill}%
\pgfsetlinewidth{0.803000pt}%
\definecolor{currentstroke}{rgb}{0.000000,0.000000,0.000000}%
\pgfsetstrokecolor{currentstroke}%
\pgfsetdash{}{0pt}%
\pgfsys@defobject{currentmarker}{\pgfqpoint{-0.048611in}{0.000000in}}{\pgfqpoint{-0.000000in}{0.000000in}}{%
\pgfpathmoveto{\pgfqpoint{-0.000000in}{0.000000in}}%
\pgfpathlineto{\pgfqpoint{-0.048611in}{0.000000in}}%
\pgfusepath{stroke,fill}%
}%
\begin{pgfscope}%
\pgfsys@transformshift{0.800000in}{0.675273in}%
\pgfsys@useobject{currentmarker}{}%
\end{pgfscope}%
\end{pgfscope}%
\begin{pgfscope}%
\definecolor{textcolor}{rgb}{0.000000,0.000000,0.000000}%
\pgfsetstrokecolor{textcolor}%
\pgfsetfillcolor{textcolor}%
\pgftext[x=0.619160in, y=0.622138in, left, base]{\color{textcolor}\ttfamily\fontsize{10.000000}{12.000000}\selectfont 0}%
\end{pgfscope}%
\begin{pgfscope}%
\pgfsetbuttcap%
\pgfsetroundjoin%
\definecolor{currentfill}{rgb}{0.000000,0.000000,0.000000}%
\pgfsetfillcolor{currentfill}%
\pgfsetlinewidth{0.803000pt}%
\definecolor{currentstroke}{rgb}{0.000000,0.000000,0.000000}%
\pgfsetstrokecolor{currentstroke}%
\pgfsetdash{}{0pt}%
\pgfsys@defobject{currentmarker}{\pgfqpoint{-0.048611in}{0.000000in}}{\pgfqpoint{-0.000000in}{0.000000in}}{%
\pgfpathmoveto{\pgfqpoint{-0.000000in}{0.000000in}}%
\pgfpathlineto{\pgfqpoint{-0.048611in}{0.000000in}}%
\pgfusepath{stroke,fill}%
}%
\begin{pgfscope}%
\pgfsys@transformshift{0.800000in}{1.228879in}%
\pgfsys@useobject{currentmarker}{}%
\end{pgfscope}%
\end{pgfscope}%
\begin{pgfscope}%
\definecolor{textcolor}{rgb}{0.000000,0.000000,0.000000}%
\pgfsetstrokecolor{textcolor}%
\pgfsetfillcolor{textcolor}%
\pgftext[x=0.284687in, y=1.175744in, left, base]{\color{textcolor}\ttfamily\fontsize{10.000000}{12.000000}\selectfont 50000}%
\end{pgfscope}%
\begin{pgfscope}%
\pgfsetbuttcap%
\pgfsetroundjoin%
\definecolor{currentfill}{rgb}{0.000000,0.000000,0.000000}%
\pgfsetfillcolor{currentfill}%
\pgfsetlinewidth{0.803000pt}%
\definecolor{currentstroke}{rgb}{0.000000,0.000000,0.000000}%
\pgfsetstrokecolor{currentstroke}%
\pgfsetdash{}{0pt}%
\pgfsys@defobject{currentmarker}{\pgfqpoint{-0.048611in}{0.000000in}}{\pgfqpoint{-0.000000in}{0.000000in}}{%
\pgfpathmoveto{\pgfqpoint{-0.000000in}{0.000000in}}%
\pgfpathlineto{\pgfqpoint{-0.048611in}{0.000000in}}%
\pgfusepath{stroke,fill}%
}%
\begin{pgfscope}%
\pgfsys@transformshift{0.800000in}{1.782485in}%
\pgfsys@useobject{currentmarker}{}%
\end{pgfscope}%
\end{pgfscope}%
\begin{pgfscope}%
\definecolor{textcolor}{rgb}{0.000000,0.000000,0.000000}%
\pgfsetstrokecolor{textcolor}%
\pgfsetfillcolor{textcolor}%
\pgftext[x=0.201069in, y=1.729350in, left, base]{\color{textcolor}\ttfamily\fontsize{10.000000}{12.000000}\selectfont 100000}%
\end{pgfscope}%
\begin{pgfscope}%
\pgfsetbuttcap%
\pgfsetroundjoin%
\definecolor{currentfill}{rgb}{0.000000,0.000000,0.000000}%
\pgfsetfillcolor{currentfill}%
\pgfsetlinewidth{0.803000pt}%
\definecolor{currentstroke}{rgb}{0.000000,0.000000,0.000000}%
\pgfsetstrokecolor{currentstroke}%
\pgfsetdash{}{0pt}%
\pgfsys@defobject{currentmarker}{\pgfqpoint{-0.048611in}{0.000000in}}{\pgfqpoint{-0.000000in}{0.000000in}}{%
\pgfpathmoveto{\pgfqpoint{-0.000000in}{0.000000in}}%
\pgfpathlineto{\pgfqpoint{-0.048611in}{0.000000in}}%
\pgfusepath{stroke,fill}%
}%
\begin{pgfscope}%
\pgfsys@transformshift{0.800000in}{2.336091in}%
\pgfsys@useobject{currentmarker}{}%
\end{pgfscope}%
\end{pgfscope}%
\begin{pgfscope}%
\definecolor{textcolor}{rgb}{0.000000,0.000000,0.000000}%
\pgfsetstrokecolor{textcolor}%
\pgfsetfillcolor{textcolor}%
\pgftext[x=0.201069in, y=2.282956in, left, base]{\color{textcolor}\ttfamily\fontsize{10.000000}{12.000000}\selectfont 150000}%
\end{pgfscope}%
\begin{pgfscope}%
\pgfsetbuttcap%
\pgfsetroundjoin%
\definecolor{currentfill}{rgb}{0.000000,0.000000,0.000000}%
\pgfsetfillcolor{currentfill}%
\pgfsetlinewidth{0.803000pt}%
\definecolor{currentstroke}{rgb}{0.000000,0.000000,0.000000}%
\pgfsetstrokecolor{currentstroke}%
\pgfsetdash{}{0pt}%
\pgfsys@defobject{currentmarker}{\pgfqpoint{-0.048611in}{0.000000in}}{\pgfqpoint{-0.000000in}{0.000000in}}{%
\pgfpathmoveto{\pgfqpoint{-0.000000in}{0.000000in}}%
\pgfpathlineto{\pgfqpoint{-0.048611in}{0.000000in}}%
\pgfusepath{stroke,fill}%
}%
\begin{pgfscope}%
\pgfsys@transformshift{0.800000in}{2.889696in}%
\pgfsys@useobject{currentmarker}{}%
\end{pgfscope}%
\end{pgfscope}%
\begin{pgfscope}%
\definecolor{textcolor}{rgb}{0.000000,0.000000,0.000000}%
\pgfsetstrokecolor{textcolor}%
\pgfsetfillcolor{textcolor}%
\pgftext[x=0.201069in, y=2.836562in, left, base]{\color{textcolor}\ttfamily\fontsize{10.000000}{12.000000}\selectfont 200000}%
\end{pgfscope}%
\begin{pgfscope}%
\pgfsetbuttcap%
\pgfsetroundjoin%
\definecolor{currentfill}{rgb}{0.000000,0.000000,0.000000}%
\pgfsetfillcolor{currentfill}%
\pgfsetlinewidth{0.803000pt}%
\definecolor{currentstroke}{rgb}{0.000000,0.000000,0.000000}%
\pgfsetstrokecolor{currentstroke}%
\pgfsetdash{}{0pt}%
\pgfsys@defobject{currentmarker}{\pgfqpoint{-0.048611in}{0.000000in}}{\pgfqpoint{-0.000000in}{0.000000in}}{%
\pgfpathmoveto{\pgfqpoint{-0.000000in}{0.000000in}}%
\pgfpathlineto{\pgfqpoint{-0.048611in}{0.000000in}}%
\pgfusepath{stroke,fill}%
}%
\begin{pgfscope}%
\pgfsys@transformshift{0.800000in}{3.443302in}%
\pgfsys@useobject{currentmarker}{}%
\end{pgfscope}%
\end{pgfscope}%
\begin{pgfscope}%
\definecolor{textcolor}{rgb}{0.000000,0.000000,0.000000}%
\pgfsetstrokecolor{textcolor}%
\pgfsetfillcolor{textcolor}%
\pgftext[x=0.201069in, y=3.390168in, left, base]{\color{textcolor}\ttfamily\fontsize{10.000000}{12.000000}\selectfont 250000}%
\end{pgfscope}%
\begin{pgfscope}%
\pgfsetbuttcap%
\pgfsetroundjoin%
\definecolor{currentfill}{rgb}{0.000000,0.000000,0.000000}%
\pgfsetfillcolor{currentfill}%
\pgfsetlinewidth{0.803000pt}%
\definecolor{currentstroke}{rgb}{0.000000,0.000000,0.000000}%
\pgfsetstrokecolor{currentstroke}%
\pgfsetdash{}{0pt}%
\pgfsys@defobject{currentmarker}{\pgfqpoint{-0.048611in}{0.000000in}}{\pgfqpoint{-0.000000in}{0.000000in}}{%
\pgfpathmoveto{\pgfqpoint{-0.000000in}{0.000000in}}%
\pgfpathlineto{\pgfqpoint{-0.048611in}{0.000000in}}%
\pgfusepath{stroke,fill}%
}%
\begin{pgfscope}%
\pgfsys@transformshift{0.800000in}{3.996908in}%
\pgfsys@useobject{currentmarker}{}%
\end{pgfscope}%
\end{pgfscope}%
\begin{pgfscope}%
\definecolor{textcolor}{rgb}{0.000000,0.000000,0.000000}%
\pgfsetstrokecolor{textcolor}%
\pgfsetfillcolor{textcolor}%
\pgftext[x=0.201069in, y=3.943774in, left, base]{\color{textcolor}\ttfamily\fontsize{10.000000}{12.000000}\selectfont 300000}%
\end{pgfscope}%
\begin{pgfscope}%
\definecolor{textcolor}{rgb}{0.000000,0.000000,0.000000}%
\pgfsetstrokecolor{textcolor}%
\pgfsetfillcolor{textcolor}%
\pgftext[x=0.145513in,y=2.376000in,,bottom,rotate=90.000000]{\color{textcolor}\ttfamily\fontsize{10.000000}{12.000000}\selectfont Swaps}%
\end{pgfscope}%
\begin{pgfscope}%
\pgfpathrectangle{\pgfqpoint{0.800000in}{0.528000in}}{\pgfqpoint{4.960000in}{3.696000in}}%
\pgfusepath{clip}%
\pgfsetrectcap%
\pgfsetroundjoin%
\pgfsetlinewidth{1.505625pt}%
\definecolor{currentstroke}{rgb}{0.000000,1.000000,0.498039}%
\pgfsetstrokecolor{currentstroke}%
\pgfsetdash{}{0pt}%
\pgfpathmoveto{\pgfqpoint{1.025455in}{0.696000in}}%
\pgfpathlineto{\pgfqpoint{1.034473in}{0.703595in}}%
\pgfpathlineto{\pgfqpoint{1.038982in}{0.703274in}}%
\pgfpathlineto{\pgfqpoint{1.043491in}{0.705445in}}%
\pgfpathlineto{\pgfqpoint{1.048000in}{0.709652in}}%
\pgfpathlineto{\pgfqpoint{1.052509in}{0.704647in}}%
\pgfpathlineto{\pgfqpoint{1.057018in}{0.702865in}}%
\pgfpathlineto{\pgfqpoint{1.066036in}{0.709696in}}%
\pgfpathlineto{\pgfqpoint{1.070545in}{0.710062in}}%
\pgfpathlineto{\pgfqpoint{1.075055in}{0.705887in}}%
\pgfpathlineto{\pgfqpoint{1.079564in}{0.707482in}}%
\pgfpathlineto{\pgfqpoint{1.088582in}{0.713162in}}%
\pgfpathlineto{\pgfqpoint{1.097600in}{0.709342in}}%
\pgfpathlineto{\pgfqpoint{1.102109in}{0.711501in}}%
\pgfpathlineto{\pgfqpoint{1.106618in}{0.716428in}}%
\pgfpathlineto{\pgfqpoint{1.115636in}{0.714645in}}%
\pgfpathlineto{\pgfqpoint{1.120145in}{0.719650in}}%
\pgfpathlineto{\pgfqpoint{1.129164in}{0.713173in}}%
\pgfpathlineto{\pgfqpoint{1.133673in}{0.715509in}}%
\pgfpathlineto{\pgfqpoint{1.138182in}{0.721455in}}%
\pgfpathlineto{\pgfqpoint{1.142691in}{0.718797in}}%
\pgfpathlineto{\pgfqpoint{1.147200in}{0.723636in}}%
\pgfpathlineto{\pgfqpoint{1.151709in}{0.716494in}}%
\pgfpathlineto{\pgfqpoint{1.156218in}{0.723392in}}%
\pgfpathlineto{\pgfqpoint{1.160727in}{0.723271in}}%
\pgfpathlineto{\pgfqpoint{1.165236in}{0.721311in}}%
\pgfpathlineto{\pgfqpoint{1.169745in}{0.727212in}}%
\pgfpathlineto{\pgfqpoint{1.174255in}{0.726127in}}%
\pgfpathlineto{\pgfqpoint{1.178764in}{0.720093in}}%
\pgfpathlineto{\pgfqpoint{1.183273in}{0.727666in}}%
\pgfpathlineto{\pgfqpoint{1.187782in}{0.721145in}}%
\pgfpathlineto{\pgfqpoint{1.196800in}{0.728342in}}%
\pgfpathlineto{\pgfqpoint{1.201309in}{0.725275in}}%
\pgfpathlineto{\pgfqpoint{1.205818in}{0.729715in}}%
\pgfpathlineto{\pgfqpoint{1.214836in}{0.730224in}}%
\pgfpathlineto{\pgfqpoint{1.219345in}{0.734841in}}%
\pgfpathlineto{\pgfqpoint{1.228364in}{0.730844in}}%
\pgfpathlineto{\pgfqpoint{1.232873in}{0.738207in}}%
\pgfpathlineto{\pgfqpoint{1.237382in}{0.735240in}}%
\pgfpathlineto{\pgfqpoint{1.246400in}{0.731807in}}%
\pgfpathlineto{\pgfqpoint{1.250909in}{0.737620in}}%
\pgfpathlineto{\pgfqpoint{1.255418in}{0.738251in}}%
\pgfpathlineto{\pgfqpoint{1.259927in}{0.741573in}}%
\pgfpathlineto{\pgfqpoint{1.264436in}{0.733645in}}%
\pgfpathlineto{\pgfqpoint{1.268945in}{0.741374in}}%
\pgfpathlineto{\pgfqpoint{1.273455in}{0.739967in}}%
\pgfpathlineto{\pgfqpoint{1.277964in}{0.739956in}}%
\pgfpathlineto{\pgfqpoint{1.282473in}{0.744219in}}%
\pgfpathlineto{\pgfqpoint{1.291491in}{0.742304in}}%
\pgfpathlineto{\pgfqpoint{1.296000in}{0.756664in}}%
\pgfpathlineto{\pgfqpoint{1.300509in}{0.745193in}}%
\pgfpathlineto{\pgfqpoint{1.305018in}{0.740753in}}%
\pgfpathlineto{\pgfqpoint{1.318545in}{0.753985in}}%
\pgfpathlineto{\pgfqpoint{1.323055in}{0.754151in}}%
\pgfpathlineto{\pgfqpoint{1.327564in}{0.742669in}}%
\pgfpathlineto{\pgfqpoint{1.332073in}{0.757882in}}%
\pgfpathlineto{\pgfqpoint{1.336582in}{0.754749in}}%
\pgfpathlineto{\pgfqpoint{1.341091in}{0.759775in}}%
\pgfpathlineto{\pgfqpoint{1.345600in}{0.758425in}}%
\pgfpathlineto{\pgfqpoint{1.350109in}{0.760163in}}%
\pgfpathlineto{\pgfqpoint{1.354618in}{0.751283in}}%
\pgfpathlineto{\pgfqpoint{1.359127in}{0.758236in}}%
\pgfpathlineto{\pgfqpoint{1.363636in}{0.758015in}}%
\pgfpathlineto{\pgfqpoint{1.368145in}{0.763728in}}%
\pgfpathlineto{\pgfqpoint{1.372655in}{0.763806in}}%
\pgfpathlineto{\pgfqpoint{1.377164in}{0.755347in}}%
\pgfpathlineto{\pgfqpoint{1.381673in}{0.761348in}}%
\pgfpathlineto{\pgfqpoint{1.386182in}{0.763186in}}%
\pgfpathlineto{\pgfqpoint{1.395200in}{0.757506in}}%
\pgfpathlineto{\pgfqpoint{1.399709in}{0.763163in}}%
\pgfpathlineto{\pgfqpoint{1.404218in}{0.771722in}}%
\pgfpathlineto{\pgfqpoint{1.408727in}{0.767581in}}%
\pgfpathlineto{\pgfqpoint{1.422255in}{0.772708in}}%
\pgfpathlineto{\pgfqpoint{1.426764in}{0.770903in}}%
\pgfpathlineto{\pgfqpoint{1.431273in}{0.765710in}}%
\pgfpathlineto{\pgfqpoint{1.435782in}{0.773715in}}%
\pgfpathlineto{\pgfqpoint{1.444800in}{0.779583in}}%
\pgfpathlineto{\pgfqpoint{1.449309in}{0.787112in}}%
\pgfpathlineto{\pgfqpoint{1.458327in}{0.780923in}}%
\pgfpathlineto{\pgfqpoint{1.462836in}{0.777391in}}%
\pgfpathlineto{\pgfqpoint{1.467345in}{0.778620in}}%
\pgfpathlineto{\pgfqpoint{1.471855in}{0.782196in}}%
\pgfpathlineto{\pgfqpoint{1.476364in}{0.792272in}}%
\pgfpathlineto{\pgfqpoint{1.485382in}{0.780259in}}%
\pgfpathlineto{\pgfqpoint{1.489891in}{0.785131in}}%
\pgfpathlineto{\pgfqpoint{1.494400in}{0.781499in}}%
\pgfpathlineto{\pgfqpoint{1.498909in}{0.792815in}}%
\pgfpathlineto{\pgfqpoint{1.503418in}{0.792959in}}%
\pgfpathlineto{\pgfqpoint{1.507927in}{0.787323in}}%
\pgfpathlineto{\pgfqpoint{1.512436in}{0.800034in}}%
\pgfpathlineto{\pgfqpoint{1.516945in}{0.791309in}}%
\pgfpathlineto{\pgfqpoint{1.521455in}{0.794918in}}%
\pgfpathlineto{\pgfqpoint{1.525964in}{0.792095in}}%
\pgfpathlineto{\pgfqpoint{1.530473in}{0.793745in}}%
\pgfpathlineto{\pgfqpoint{1.534982in}{0.799978in}}%
\pgfpathlineto{\pgfqpoint{1.539491in}{0.796468in}}%
\pgfpathlineto{\pgfqpoint{1.544000in}{0.807297in}}%
\pgfpathlineto{\pgfqpoint{1.548509in}{0.807862in}}%
\pgfpathlineto{\pgfqpoint{1.553018in}{0.810862in}}%
\pgfpathlineto{\pgfqpoint{1.557527in}{0.807408in}}%
\pgfpathlineto{\pgfqpoint{1.562036in}{0.808747in}}%
\pgfpathlineto{\pgfqpoint{1.571055in}{0.808183in}}%
\pgfpathlineto{\pgfqpoint{1.575564in}{0.814760in}}%
\pgfpathlineto{\pgfqpoint{1.580073in}{0.802691in}}%
\pgfpathlineto{\pgfqpoint{1.584582in}{0.809456in}}%
\pgfpathlineto{\pgfqpoint{1.589091in}{0.820451in}}%
\pgfpathlineto{\pgfqpoint{1.593600in}{0.823075in}}%
\pgfpathlineto{\pgfqpoint{1.598109in}{0.817339in}}%
\pgfpathlineto{\pgfqpoint{1.602618in}{0.820251in}}%
\pgfpathlineto{\pgfqpoint{1.607127in}{0.830526in}}%
\pgfpathlineto{\pgfqpoint{1.611636in}{0.817727in}}%
\pgfpathlineto{\pgfqpoint{1.616145in}{0.814018in}}%
\pgfpathlineto{\pgfqpoint{1.620655in}{0.828766in}}%
\pgfpathlineto{\pgfqpoint{1.625164in}{0.817361in}}%
\pgfpathlineto{\pgfqpoint{1.629673in}{0.816431in}}%
\pgfpathlineto{\pgfqpoint{1.634182in}{0.835852in}}%
\pgfpathlineto{\pgfqpoint{1.638691in}{0.832010in}}%
\pgfpathlineto{\pgfqpoint{1.643200in}{0.832608in}}%
\pgfpathlineto{\pgfqpoint{1.647709in}{0.834689in}}%
\pgfpathlineto{\pgfqpoint{1.652218in}{0.831168in}}%
\pgfpathlineto{\pgfqpoint{1.656727in}{0.837092in}}%
\pgfpathlineto{\pgfqpoint{1.661236in}{0.832408in}}%
\pgfpathlineto{\pgfqpoint{1.665745in}{0.831478in}}%
\pgfpathlineto{\pgfqpoint{1.670255in}{0.844687in}}%
\pgfpathlineto{\pgfqpoint{1.674764in}{0.844842in}}%
\pgfpathlineto{\pgfqpoint{1.679273in}{0.852316in}}%
\pgfpathlineto{\pgfqpoint{1.683782in}{0.846182in}}%
\pgfpathlineto{\pgfqpoint{1.688291in}{0.823241in}}%
\pgfpathlineto{\pgfqpoint{1.697309in}{0.852272in}}%
\pgfpathlineto{\pgfqpoint{1.701818in}{0.853246in}}%
\pgfpathlineto{\pgfqpoint{1.706327in}{0.856059in}}%
\pgfpathlineto{\pgfqpoint{1.710836in}{0.853412in}}%
\pgfpathlineto{\pgfqpoint{1.715345in}{0.841798in}}%
\pgfpathlineto{\pgfqpoint{1.719855in}{0.849902in}}%
\pgfpathlineto{\pgfqpoint{1.724364in}{0.867463in}}%
\pgfpathlineto{\pgfqpoint{1.728873in}{0.862004in}}%
\pgfpathlineto{\pgfqpoint{1.733382in}{0.859878in}}%
\pgfpathlineto{\pgfqpoint{1.737891in}{0.860543in}}%
\pgfpathlineto{\pgfqpoint{1.742400in}{0.854508in}}%
\pgfpathlineto{\pgfqpoint{1.746909in}{0.855438in}}%
\pgfpathlineto{\pgfqpoint{1.751418in}{0.857664in}}%
\pgfpathlineto{\pgfqpoint{1.755927in}{0.854608in}}%
\pgfpathlineto{\pgfqpoint{1.764945in}{0.876188in}}%
\pgfpathlineto{\pgfqpoint{1.769455in}{0.859679in}}%
\pgfpathlineto{\pgfqpoint{1.773964in}{0.867540in}}%
\pgfpathlineto{\pgfqpoint{1.778473in}{0.868836in}}%
\pgfpathlineto{\pgfqpoint{1.782982in}{0.863809in}}%
\pgfpathlineto{\pgfqpoint{1.787491in}{0.877738in}}%
\pgfpathlineto{\pgfqpoint{1.792000in}{0.869821in}}%
\pgfpathlineto{\pgfqpoint{1.801018in}{0.891146in}}%
\pgfpathlineto{\pgfqpoint{1.805527in}{0.867673in}}%
\pgfpathlineto{\pgfqpoint{1.810036in}{0.884636in}}%
\pgfpathlineto{\pgfqpoint{1.814545in}{0.882244in}}%
\pgfpathlineto{\pgfqpoint{1.819055in}{0.893360in}}%
\pgfpathlineto{\pgfqpoint{1.828073in}{0.896195in}}%
\pgfpathlineto{\pgfqpoint{1.832582in}{0.895675in}}%
\pgfpathlineto{\pgfqpoint{1.837091in}{0.892685in}}%
\pgfpathlineto{\pgfqpoint{1.841600in}{0.891622in}}%
\pgfpathlineto{\pgfqpoint{1.846109in}{0.882953in}}%
\pgfpathlineto{\pgfqpoint{1.850618in}{0.886451in}}%
\pgfpathlineto{\pgfqpoint{1.855127in}{0.903004in}}%
\pgfpathlineto{\pgfqpoint{1.859636in}{0.907156in}}%
\pgfpathlineto{\pgfqpoint{1.864145in}{0.908839in}}%
\pgfpathlineto{\pgfqpoint{1.868655in}{0.908297in}}%
\pgfpathlineto{\pgfqpoint{1.873164in}{0.895951in}}%
\pgfpathlineto{\pgfqpoint{1.877673in}{0.899052in}}%
\pgfpathlineto{\pgfqpoint{1.886691in}{0.923908in}}%
\pgfpathlineto{\pgfqpoint{1.891200in}{0.898874in}}%
\pgfpathlineto{\pgfqpoint{1.895709in}{0.912858in}}%
\pgfpathlineto{\pgfqpoint{1.900218in}{0.917741in}}%
\pgfpathlineto{\pgfqpoint{1.904727in}{0.925326in}}%
\pgfpathlineto{\pgfqpoint{1.909236in}{0.906725in}}%
\pgfpathlineto{\pgfqpoint{1.913745in}{0.927208in}}%
\pgfpathlineto{\pgfqpoint{1.918255in}{0.929356in}}%
\pgfpathlineto{\pgfqpoint{1.922764in}{0.921406in}}%
\pgfpathlineto{\pgfqpoint{1.927273in}{0.933541in}}%
\pgfpathlineto{\pgfqpoint{1.931782in}{0.931083in}}%
\pgfpathlineto{\pgfqpoint{1.936291in}{0.935623in}}%
\pgfpathlineto{\pgfqpoint{1.940800in}{0.923244in}}%
\pgfpathlineto{\pgfqpoint{1.945309in}{0.941413in}}%
\pgfpathlineto{\pgfqpoint{1.949818in}{0.939398in}}%
\pgfpathlineto{\pgfqpoint{1.958836in}{0.937981in}}%
\pgfpathlineto{\pgfqpoint{1.963345in}{0.936719in}}%
\pgfpathlineto{\pgfqpoint{1.967855in}{0.953017in}}%
\pgfpathlineto{\pgfqpoint{1.972364in}{0.944602in}}%
\pgfpathlineto{\pgfqpoint{1.976873in}{0.954346in}}%
\pgfpathlineto{\pgfqpoint{1.981382in}{0.952430in}}%
\pgfpathlineto{\pgfqpoint{1.985891in}{0.954534in}}%
\pgfpathlineto{\pgfqpoint{1.990400in}{0.958276in}}%
\pgfpathlineto{\pgfqpoint{1.999418in}{0.940838in}}%
\pgfpathlineto{\pgfqpoint{2.003927in}{0.942598in}}%
\pgfpathlineto{\pgfqpoint{2.008436in}{0.948278in}}%
\pgfpathlineto{\pgfqpoint{2.012945in}{0.963259in}}%
\pgfpathlineto{\pgfqpoint{2.017455in}{0.969282in}}%
\pgfpathlineto{\pgfqpoint{2.021964in}{0.952021in}}%
\pgfpathlineto{\pgfqpoint{2.026473in}{0.944026in}}%
\pgfpathlineto{\pgfqpoint{2.030982in}{0.972681in}}%
\pgfpathlineto{\pgfqpoint{2.035491in}{0.986743in}}%
\pgfpathlineto{\pgfqpoint{2.040000in}{0.955010in}}%
\pgfpathlineto{\pgfqpoint{2.044509in}{0.958387in}}%
\pgfpathlineto{\pgfqpoint{2.049018in}{0.973999in}}%
\pgfpathlineto{\pgfqpoint{2.053527in}{0.967510in}}%
\pgfpathlineto{\pgfqpoint{2.058036in}{0.978682in}}%
\pgfpathlineto{\pgfqpoint{2.062545in}{0.981627in}}%
\pgfpathlineto{\pgfqpoint{2.067055in}{0.982292in}}%
\pgfpathlineto{\pgfqpoint{2.071564in}{0.969725in}}%
\pgfpathlineto{\pgfqpoint{2.076073in}{1.011832in}}%
\pgfpathlineto{\pgfqpoint{2.080582in}{0.968053in}}%
\pgfpathlineto{\pgfqpoint{2.089600in}{0.969526in}}%
\pgfpathlineto{\pgfqpoint{2.094109in}{0.992556in}}%
\pgfpathlineto{\pgfqpoint{2.098618in}{0.990208in}}%
\pgfpathlineto{\pgfqpoint{2.103127in}{1.001923in}}%
\pgfpathlineto{\pgfqpoint{2.107636in}{0.994316in}}%
\pgfpathlineto{\pgfqpoint{2.112145in}{0.981638in}}%
\pgfpathlineto{\pgfqpoint{2.116655in}{1.019483in}}%
\pgfpathlineto{\pgfqpoint{2.121164in}{1.006861in}}%
\pgfpathlineto{\pgfqpoint{2.125673in}{1.010116in}}%
\pgfpathlineto{\pgfqpoint{2.130182in}{1.022140in}}%
\pgfpathlineto{\pgfqpoint{2.134691in}{0.999631in}}%
\pgfpathlineto{\pgfqpoint{2.139200in}{1.013626in}}%
\pgfpathlineto{\pgfqpoint{2.143709in}{1.007027in}}%
\pgfpathlineto{\pgfqpoint{2.148218in}{1.020103in}}%
\pgfpathlineto{\pgfqpoint{2.152727in}{1.002886in}}%
\pgfpathlineto{\pgfqpoint{2.161745in}{1.000616in}}%
\pgfpathlineto{\pgfqpoint{2.166255in}{0.985370in}}%
\pgfpathlineto{\pgfqpoint{2.170764in}{1.027566in}}%
\pgfpathlineto{\pgfqpoint{2.175273in}{1.033312in}}%
\pgfpathlineto{\pgfqpoint{2.179782in}{1.032770in}}%
\pgfpathlineto{\pgfqpoint{2.184291in}{1.026558in}}%
\pgfpathlineto{\pgfqpoint{2.188800in}{1.029481in}}%
\pgfpathlineto{\pgfqpoint{2.193309in}{1.036113in}}%
\pgfpathlineto{\pgfqpoint{2.197818in}{1.020037in}}%
\pgfpathlineto{\pgfqpoint{2.202327in}{1.023436in}}%
\pgfpathlineto{\pgfqpoint{2.206836in}{1.049389in}}%
\pgfpathlineto{\pgfqpoint{2.211345in}{1.022550in}}%
\pgfpathlineto{\pgfqpoint{2.215855in}{1.044229in}}%
\pgfpathlineto{\pgfqpoint{2.220364in}{1.031607in}}%
\pgfpathlineto{\pgfqpoint{2.224873in}{1.027123in}}%
\pgfpathlineto{\pgfqpoint{2.229382in}{1.073150in}}%
\pgfpathlineto{\pgfqpoint{2.233891in}{1.044451in}}%
\pgfpathlineto{\pgfqpoint{2.238400in}{1.043786in}}%
\pgfpathlineto{\pgfqpoint{2.242909in}{1.066861in}}%
\pgfpathlineto{\pgfqpoint{2.247418in}{1.047108in}}%
\pgfpathlineto{\pgfqpoint{2.251927in}{1.072485in}}%
\pgfpathlineto{\pgfqpoint{2.256436in}{1.037453in}}%
\pgfpathlineto{\pgfqpoint{2.260945in}{1.056442in}}%
\pgfpathlineto{\pgfqpoint{2.269964in}{1.080789in}}%
\pgfpathlineto{\pgfqpoint{2.274473in}{1.053618in}}%
\pgfpathlineto{\pgfqpoint{2.278982in}{1.082096in}}%
\pgfpathlineto{\pgfqpoint{2.283491in}{1.069152in}}%
\pgfpathlineto{\pgfqpoint{2.288000in}{1.051802in}}%
\pgfpathlineto{\pgfqpoint{2.292509in}{1.064048in}}%
\pgfpathlineto{\pgfqpoint{2.297018in}{1.087588in}}%
\pgfpathlineto{\pgfqpoint{2.301527in}{1.091784in}}%
\pgfpathlineto{\pgfqpoint{2.310545in}{1.105657in}}%
\pgfpathlineto{\pgfqpoint{2.315055in}{1.067525in}}%
\pgfpathlineto{\pgfqpoint{2.319564in}{1.081044in}}%
\pgfpathlineto{\pgfqpoint{2.324073in}{1.075995in}}%
\pgfpathlineto{\pgfqpoint{2.328582in}{1.090843in}}%
\pgfpathlineto{\pgfqpoint{2.333091in}{1.096013in}}%
\pgfpathlineto{\pgfqpoint{2.337600in}{1.110728in}}%
\pgfpathlineto{\pgfqpoint{2.342109in}{1.097298in}}%
\pgfpathlineto{\pgfqpoint{2.346618in}{1.106798in}}%
\pgfpathlineto{\pgfqpoint{2.351127in}{1.080490in}}%
\pgfpathlineto{\pgfqpoint{2.355636in}{1.090212in}}%
\pgfpathlineto{\pgfqpoint{2.360145in}{1.122719in}}%
\pgfpathlineto{\pgfqpoint{2.364655in}{1.130127in}}%
\pgfpathlineto{\pgfqpoint{2.369164in}{1.087056in}}%
\pgfpathlineto{\pgfqpoint{2.373673in}{1.118767in}}%
\pgfpathlineto{\pgfqpoint{2.378182in}{1.131278in}}%
\pgfpathlineto{\pgfqpoint{2.382691in}{1.114116in}}%
\pgfpathlineto{\pgfqpoint{2.387200in}{1.118911in}}%
\pgfpathlineto{\pgfqpoint{2.391709in}{1.127170in}}%
\pgfpathlineto{\pgfqpoint{2.396218in}{1.111957in}}%
\pgfpathlineto{\pgfqpoint{2.400727in}{1.147964in}}%
\pgfpathlineto{\pgfqpoint{2.405236in}{1.127303in}}%
\pgfpathlineto{\pgfqpoint{2.409745in}{1.119376in}}%
\pgfpathlineto{\pgfqpoint{2.414255in}{1.154829in}}%
\pgfpathlineto{\pgfqpoint{2.418764in}{1.113186in}}%
\pgfpathlineto{\pgfqpoint{2.423273in}{1.114770in}}%
\pgfpathlineto{\pgfqpoint{2.432291in}{1.135087in}}%
\pgfpathlineto{\pgfqpoint{2.436800in}{1.135021in}}%
\pgfpathlineto{\pgfqpoint{2.441309in}{1.147964in}}%
\pgfpathlineto{\pgfqpoint{2.445818in}{1.146015in}}%
\pgfpathlineto{\pgfqpoint{2.450327in}{1.159778in}}%
\pgfpathlineto{\pgfqpoint{2.454836in}{1.155349in}}%
\pgfpathlineto{\pgfqpoint{2.459345in}{1.134190in}}%
\pgfpathlineto{\pgfqpoint{2.463855in}{1.159999in}}%
\pgfpathlineto{\pgfqpoint{2.468364in}{1.149403in}}%
\pgfpathlineto{\pgfqpoint{2.472873in}{1.168270in}}%
\pgfpathlineto{\pgfqpoint{2.477382in}{1.175035in}}%
\pgfpathlineto{\pgfqpoint{2.481891in}{1.155836in}}%
\pgfpathlineto{\pgfqpoint{2.486400in}{1.165513in}}%
\pgfpathlineto{\pgfqpoint{2.490909in}{1.200324in}}%
\pgfpathlineto{\pgfqpoint{2.495418in}{1.164362in}}%
\pgfpathlineto{\pgfqpoint{2.499927in}{1.171116in}}%
\pgfpathlineto{\pgfqpoint{2.504436in}{1.224948in}}%
\pgfpathlineto{\pgfqpoint{2.508945in}{1.188477in}}%
\pgfpathlineto{\pgfqpoint{2.513455in}{1.184668in}}%
\pgfpathlineto{\pgfqpoint{2.517964in}{1.206934in}}%
\pgfpathlineto{\pgfqpoint{2.522473in}{1.177117in}}%
\pgfpathlineto{\pgfqpoint{2.526982in}{1.168857in}}%
\pgfpathlineto{\pgfqpoint{2.531491in}{1.188886in}}%
\pgfpathlineto{\pgfqpoint{2.536000in}{1.185653in}}%
\pgfpathlineto{\pgfqpoint{2.540509in}{1.194201in}}%
\pgfpathlineto{\pgfqpoint{2.545018in}{1.215991in}}%
\pgfpathlineto{\pgfqpoint{2.549527in}{1.218726in}}%
\pgfpathlineto{\pgfqpoint{2.554036in}{1.251676in}}%
\pgfpathlineto{\pgfqpoint{2.558545in}{1.175998in}}%
\pgfpathlineto{\pgfqpoint{2.563055in}{1.189684in}}%
\pgfpathlineto{\pgfqpoint{2.567564in}{1.216057in}}%
\pgfpathlineto{\pgfqpoint{2.572073in}{1.206425in}}%
\pgfpathlineto{\pgfqpoint{2.576582in}{1.236043in}}%
\pgfpathlineto{\pgfqpoint{2.581091in}{1.211595in}}%
\pgfpathlineto{\pgfqpoint{2.585600in}{1.202804in}}%
\pgfpathlineto{\pgfqpoint{2.590109in}{1.233883in}}%
\pgfpathlineto{\pgfqpoint{2.594618in}{1.214474in}}%
\pgfpathlineto{\pgfqpoint{2.603636in}{1.237803in}}%
\pgfpathlineto{\pgfqpoint{2.608145in}{1.241623in}}%
\pgfpathlineto{\pgfqpoint{2.612655in}{1.269879in}}%
\pgfpathlineto{\pgfqpoint{2.617164in}{1.276522in}}%
\pgfpathlineto{\pgfqpoint{2.621673in}{1.212160in}}%
\pgfpathlineto{\pgfqpoint{2.626182in}{1.275116in}}%
\pgfpathlineto{\pgfqpoint{2.630691in}{1.286664in}}%
\pgfpathlineto{\pgfqpoint{2.635200in}{1.237471in}}%
\pgfpathlineto{\pgfqpoint{2.639709in}{1.237316in}}%
\pgfpathlineto{\pgfqpoint{2.644218in}{1.249019in}}%
\pgfpathlineto{\pgfqpoint{2.648727in}{1.266092in}}%
\pgfpathlineto{\pgfqpoint{2.653236in}{1.267598in}}%
\pgfpathlineto{\pgfqpoint{2.657745in}{1.278083in}}%
\pgfpathlineto{\pgfqpoint{2.662255in}{1.243262in}}%
\pgfpathlineto{\pgfqpoint{2.666764in}{1.260412in}}%
\pgfpathlineto{\pgfqpoint{2.671273in}{1.258120in}}%
\pgfpathlineto{\pgfqpoint{2.675782in}{1.269469in}}%
\pgfpathlineto{\pgfqpoint{2.680291in}{1.288115in}}%
\pgfpathlineto{\pgfqpoint{2.684800in}{1.268351in}}%
\pgfpathlineto{\pgfqpoint{2.689309in}{1.276367in}}%
\pgfpathlineto{\pgfqpoint{2.693818in}{1.293784in}}%
\pgfpathlineto{\pgfqpoint{2.698327in}{1.316825in}}%
\pgfpathlineto{\pgfqpoint{2.702836in}{1.251798in}}%
\pgfpathlineto{\pgfqpoint{2.707345in}{1.336289in}}%
\pgfpathlineto{\pgfqpoint{2.711855in}{1.309517in}}%
\pgfpathlineto{\pgfqpoint{2.716364in}{1.275293in}}%
\pgfpathlineto{\pgfqpoint{2.720873in}{1.317101in}}%
\pgfpathlineto{\pgfqpoint{2.725382in}{1.320944in}}%
\pgfpathlineto{\pgfqpoint{2.729891in}{1.309008in}}%
\pgfpathlineto{\pgfqpoint{2.734400in}{1.333156in}}%
\pgfpathlineto{\pgfqpoint{2.738909in}{1.316216in}}%
\pgfpathlineto{\pgfqpoint{2.743418in}{1.312097in}}%
\pgfpathlineto{\pgfqpoint{2.747927in}{1.324553in}}%
\pgfpathlineto{\pgfqpoint{2.752436in}{1.322006in}}%
\pgfpathlineto{\pgfqpoint{2.756945in}{1.305874in}}%
\pgfpathlineto{\pgfqpoint{2.761455in}{1.331141in}}%
\pgfpathlineto{\pgfqpoint{2.765964in}{1.329070in}}%
\pgfpathlineto{\pgfqpoint{2.770473in}{1.284859in}}%
\pgfpathlineto{\pgfqpoint{2.774982in}{1.316659in}}%
\pgfpathlineto{\pgfqpoint{2.779491in}{1.326347in}}%
\pgfpathlineto{\pgfqpoint{2.784000in}{1.316803in}}%
\pgfpathlineto{\pgfqpoint{2.793018in}{1.321176in}}%
\pgfpathlineto{\pgfqpoint{2.797527in}{1.342346in}}%
\pgfpathlineto{\pgfqpoint{2.806545in}{1.371233in}}%
\pgfpathlineto{\pgfqpoint{2.811055in}{1.374289in}}%
\pgfpathlineto{\pgfqpoint{2.815564in}{1.352754in}}%
\pgfpathlineto{\pgfqpoint{2.820073in}{1.392945in}}%
\pgfpathlineto{\pgfqpoint{2.824582in}{1.374521in}}%
\pgfpathlineto{\pgfqpoint{2.829091in}{1.331252in}}%
\pgfpathlineto{\pgfqpoint{2.833600in}{1.368321in}}%
\pgfpathlineto{\pgfqpoint{2.838109in}{1.381685in}}%
\pgfpathlineto{\pgfqpoint{2.842618in}{1.412255in}}%
\pgfpathlineto{\pgfqpoint{2.847127in}{1.400408in}}%
\pgfpathlineto{\pgfqpoint{2.851636in}{1.366051in}}%
\pgfpathlineto{\pgfqpoint{2.856145in}{1.344649in}}%
\pgfpathlineto{\pgfqpoint{2.860655in}{1.403874in}}%
\pgfpathlineto{\pgfqpoint{2.865164in}{1.397673in}}%
\pgfpathlineto{\pgfqpoint{2.869673in}{1.402789in}}%
\pgfpathlineto{\pgfqpoint{2.874182in}{1.387343in}}%
\pgfpathlineto{\pgfqpoint{2.878691in}{1.406963in}}%
\pgfpathlineto{\pgfqpoint{2.883200in}{1.387553in}}%
\pgfpathlineto{\pgfqpoint{2.887709in}{1.377954in}}%
\pgfpathlineto{\pgfqpoint{2.892218in}{1.391993in}}%
\pgfpathlineto{\pgfqpoint{2.901236in}{1.413540in}}%
\pgfpathlineto{\pgfqpoint{2.905745in}{1.403940in}}%
\pgfpathlineto{\pgfqpoint{2.910255in}{1.419585in}}%
\pgfpathlineto{\pgfqpoint{2.914764in}{1.412942in}}%
\pgfpathlineto{\pgfqpoint{2.919273in}{1.435285in}}%
\pgfpathlineto{\pgfqpoint{2.923782in}{1.421611in}}%
\pgfpathlineto{\pgfqpoint{2.928291in}{1.429406in}}%
\pgfpathlineto{\pgfqpoint{2.932800in}{1.410185in}}%
\pgfpathlineto{\pgfqpoint{2.937309in}{1.379504in}}%
\pgfpathlineto{\pgfqpoint{2.941818in}{1.442825in}}%
\pgfpathlineto{\pgfqpoint{2.946327in}{1.457164in}}%
\pgfpathlineto{\pgfqpoint{2.950836in}{1.410971in}}%
\pgfpathlineto{\pgfqpoint{2.955345in}{1.418445in}}%
\pgfpathlineto{\pgfqpoint{2.959855in}{1.421689in}}%
\pgfpathlineto{\pgfqpoint{2.964364in}{1.461216in}}%
\pgfpathlineto{\pgfqpoint{2.968873in}{1.451118in}}%
\pgfpathlineto{\pgfqpoint{2.973382in}{1.518348in}}%
\pgfpathlineto{\pgfqpoint{2.977891in}{1.473573in}}%
\pgfpathlineto{\pgfqpoint{2.982400in}{1.466099in}}%
\pgfpathlineto{\pgfqpoint{2.986909in}{1.407162in}}%
\pgfpathlineto{\pgfqpoint{2.991418in}{1.473982in}}%
\pgfpathlineto{\pgfqpoint{2.995927in}{1.486505in}}%
\pgfpathlineto{\pgfqpoint{3.000436in}{1.407849in}}%
\pgfpathlineto{\pgfqpoint{3.004945in}{1.460009in}}%
\pgfpathlineto{\pgfqpoint{3.009455in}{1.494864in}}%
\pgfpathlineto{\pgfqpoint{3.013964in}{1.483571in}}%
\pgfpathlineto{\pgfqpoint{3.018473in}{1.543238in}}%
\pgfpathlineto{\pgfqpoint{3.022982in}{1.526420in}}%
\pgfpathlineto{\pgfqpoint{3.027491in}{1.453222in}}%
\pgfpathlineto{\pgfqpoint{3.032000in}{1.513997in}}%
\pgfpathlineto{\pgfqpoint{3.036509in}{1.513720in}}%
\pgfpathlineto{\pgfqpoint{3.041018in}{1.521825in}}%
\pgfpathlineto{\pgfqpoint{3.045527in}{1.520419in}}%
\pgfpathlineto{\pgfqpoint{3.050036in}{1.515968in}}%
\pgfpathlineto{\pgfqpoint{3.054545in}{1.474624in}}%
\pgfpathlineto{\pgfqpoint{3.059055in}{1.493680in}}%
\pgfpathlineto{\pgfqpoint{3.063564in}{1.547191in}}%
\pgfpathlineto{\pgfqpoint{3.068073in}{1.547501in}}%
\pgfpathlineto{\pgfqpoint{3.072582in}{1.535942in}}%
\pgfpathlineto{\pgfqpoint{3.077091in}{1.481688in}}%
\pgfpathlineto{\pgfqpoint{3.081600in}{1.532033in}}%
\pgfpathlineto{\pgfqpoint{3.086109in}{1.515447in}}%
\pgfpathlineto{\pgfqpoint{3.090618in}{1.524659in}}%
\pgfpathlineto{\pgfqpoint{3.095127in}{1.574528in}}%
\pgfpathlineto{\pgfqpoint{3.099636in}{1.481356in}}%
\pgfpathlineto{\pgfqpoint{3.104145in}{1.506944in}}%
\pgfpathlineto{\pgfqpoint{3.108655in}{1.569911in}}%
\pgfpathlineto{\pgfqpoint{3.113164in}{1.578880in}}%
\pgfpathlineto{\pgfqpoint{3.117673in}{1.571683in}}%
\pgfpathlineto{\pgfqpoint{3.122182in}{1.550889in}}%
\pgfpathlineto{\pgfqpoint{3.126691in}{1.507265in}}%
\pgfpathlineto{\pgfqpoint{3.131200in}{1.560732in}}%
\pgfpathlineto{\pgfqpoint{3.135709in}{1.521437in}}%
\pgfpathlineto{\pgfqpoint{3.140218in}{1.570343in}}%
\pgfpathlineto{\pgfqpoint{3.144727in}{1.520607in}}%
\pgfpathlineto{\pgfqpoint{3.149236in}{1.570066in}}%
\pgfpathlineto{\pgfqpoint{3.153745in}{1.558695in}}%
\pgfpathlineto{\pgfqpoint{3.158255in}{1.563035in}}%
\pgfpathlineto{\pgfqpoint{3.162764in}{1.612262in}}%
\pgfpathlineto{\pgfqpoint{3.167273in}{1.604312in}}%
\pgfpathlineto{\pgfqpoint{3.171782in}{1.615971in}}%
\pgfpathlineto{\pgfqpoint{3.176291in}{1.609627in}}%
\pgfpathlineto{\pgfqpoint{3.180800in}{1.574351in}}%
\pgfpathlineto{\pgfqpoint{3.185309in}{1.590240in}}%
\pgfpathlineto{\pgfqpoint{3.189818in}{1.640917in}}%
\pgfpathlineto{\pgfqpoint{3.194327in}{1.616768in}}%
\pgfpathlineto{\pgfqpoint{3.198836in}{1.632856in}}%
\pgfpathlineto{\pgfqpoint{3.203345in}{1.593694in}}%
\pgfpathlineto{\pgfqpoint{3.207855in}{1.602884in}}%
\pgfpathlineto{\pgfqpoint{3.212364in}{1.623223in}}%
\pgfpathlineto{\pgfqpoint{3.216873in}{1.622393in}}%
\pgfpathlineto{\pgfqpoint{3.221382in}{1.660647in}}%
\pgfpathlineto{\pgfqpoint{3.225891in}{1.600681in}}%
\pgfpathlineto{\pgfqpoint{3.230400in}{1.611033in}}%
\pgfpathlineto{\pgfqpoint{3.234909in}{1.627663in}}%
\pgfpathlineto{\pgfqpoint{3.239418in}{1.648966in}}%
\pgfpathlineto{\pgfqpoint{3.243927in}{1.597957in}}%
\pgfpathlineto{\pgfqpoint{3.252945in}{1.679414in}}%
\pgfpathlineto{\pgfqpoint{3.257455in}{1.632004in}}%
\pgfpathlineto{\pgfqpoint{3.261964in}{1.706707in}}%
\pgfpathlineto{\pgfqpoint{3.266473in}{1.662485in}}%
\pgfpathlineto{\pgfqpoint{3.275491in}{1.670158in}}%
\pgfpathlineto{\pgfqpoint{3.280000in}{1.687563in}}%
\pgfpathlineto{\pgfqpoint{3.284509in}{1.662286in}}%
\pgfpathlineto{\pgfqpoint{3.289018in}{1.707338in}}%
\pgfpathlineto{\pgfqpoint{3.293527in}{1.696388in}}%
\pgfpathlineto{\pgfqpoint{3.298036in}{1.663028in}}%
\pgfpathlineto{\pgfqpoint{3.302545in}{1.749468in}}%
\pgfpathlineto{\pgfqpoint{3.307055in}{1.732206in}}%
\pgfpathlineto{\pgfqpoint{3.316073in}{1.708877in}}%
\pgfpathlineto{\pgfqpoint{3.320582in}{1.746943in}}%
\pgfpathlineto{\pgfqpoint{3.325091in}{1.687973in}}%
\pgfpathlineto{\pgfqpoint{3.329600in}{1.670479in}}%
\pgfpathlineto{\pgfqpoint{3.334109in}{1.740001in}}%
\pgfpathlineto{\pgfqpoint{3.343127in}{1.688892in}}%
\pgfpathlineto{\pgfqpoint{3.347636in}{1.724068in}}%
\pgfpathlineto{\pgfqpoint{3.352145in}{1.718909in}}%
\pgfpathlineto{\pgfqpoint{3.356655in}{1.754505in}}%
\pgfpathlineto{\pgfqpoint{3.361164in}{1.730678in}}%
\pgfpathlineto{\pgfqpoint{3.365673in}{1.744042in}}%
\pgfpathlineto{\pgfqpoint{3.370182in}{1.768025in}}%
\pgfpathlineto{\pgfqpoint{3.374691in}{1.673757in}}%
\pgfpathlineto{\pgfqpoint{3.379200in}{1.771247in}}%
\pgfpathlineto{\pgfqpoint{3.383709in}{1.738351in}}%
\pgfpathlineto{\pgfqpoint{3.388218in}{1.719086in}}%
\pgfpathlineto{\pgfqpoint{3.401745in}{1.793513in}}%
\pgfpathlineto{\pgfqpoint{3.406255in}{1.799480in}}%
\pgfpathlineto{\pgfqpoint{3.410764in}{1.740776in}}%
\pgfpathlineto{\pgfqpoint{3.415273in}{1.739315in}}%
\pgfpathlineto{\pgfqpoint{3.419782in}{1.790501in}}%
\pgfpathlineto{\pgfqpoint{3.424291in}{1.783581in}}%
\pgfpathlineto{\pgfqpoint{3.428800in}{1.807087in}}%
\pgfpathlineto{\pgfqpoint{3.433309in}{1.812501in}}%
\pgfpathlineto{\pgfqpoint{3.437818in}{1.821182in}}%
\pgfpathlineto{\pgfqpoint{3.442327in}{1.811826in}}%
\pgfpathlineto{\pgfqpoint{3.446836in}{1.760318in}}%
\pgfpathlineto{\pgfqpoint{3.451345in}{1.772298in}}%
\pgfpathlineto{\pgfqpoint{3.455855in}{1.853501in}}%
\pgfpathlineto{\pgfqpoint{3.464873in}{1.789848in}}%
\pgfpathlineto{\pgfqpoint{3.469382in}{1.816354in}}%
\pgfpathlineto{\pgfqpoint{3.473891in}{1.853745in}}%
\pgfpathlineto{\pgfqpoint{3.478400in}{1.827382in}}%
\pgfpathlineto{\pgfqpoint{3.482909in}{1.758735in}}%
\pgfpathlineto{\pgfqpoint{3.487418in}{1.827914in}}%
\pgfpathlineto{\pgfqpoint{3.491927in}{1.873929in}}%
\pgfpathlineto{\pgfqpoint{3.496436in}{1.865171in}}%
\pgfpathlineto{\pgfqpoint{3.500945in}{1.771988in}}%
\pgfpathlineto{\pgfqpoint{3.505455in}{1.832686in}}%
\pgfpathlineto{\pgfqpoint{3.509964in}{1.802104in}}%
\pgfpathlineto{\pgfqpoint{3.514473in}{1.853468in}}%
\pgfpathlineto{\pgfqpoint{3.518982in}{1.879908in}}%
\pgfpathlineto{\pgfqpoint{3.523491in}{1.864031in}}%
\pgfpathlineto{\pgfqpoint{3.528000in}{1.854653in}}%
\pgfpathlineto{\pgfqpoint{3.532509in}{1.912560in}}%
\pgfpathlineto{\pgfqpoint{3.537018in}{1.876963in}}%
\pgfpathlineto{\pgfqpoint{3.541527in}{1.915649in}}%
\pgfpathlineto{\pgfqpoint{3.546036in}{1.912516in}}%
\pgfpathlineto{\pgfqpoint{3.550545in}{1.862082in}}%
\pgfpathlineto{\pgfqpoint{3.555055in}{1.859004in}}%
\pgfpathlineto{\pgfqpoint{3.559564in}{1.927817in}}%
\pgfpathlineto{\pgfqpoint{3.564073in}{1.867640in}}%
\pgfpathlineto{\pgfqpoint{3.568582in}{1.896649in}}%
\pgfpathlineto{\pgfqpoint{3.573091in}{1.877240in}}%
\pgfpathlineto{\pgfqpoint{3.577600in}{1.911486in}}%
\pgfpathlineto{\pgfqpoint{3.582109in}{1.891456in}}%
\pgfpathlineto{\pgfqpoint{3.586618in}{1.890704in}}%
\pgfpathlineto{\pgfqpoint{3.591127in}{1.928316in}}%
\pgfpathlineto{\pgfqpoint{3.595636in}{1.896948in}}%
\pgfpathlineto{\pgfqpoint{3.604655in}{1.938037in}}%
\pgfpathlineto{\pgfqpoint{3.609164in}{1.879266in}}%
\pgfpathlineto{\pgfqpoint{3.613673in}{1.930419in}}%
\pgfpathlineto{\pgfqpoint{3.618182in}{1.932014in}}%
\pgfpathlineto{\pgfqpoint{3.622691in}{1.948965in}}%
\pgfpathlineto{\pgfqpoint{3.627200in}{1.920742in}}%
\pgfpathlineto{\pgfqpoint{3.631709in}{1.977055in}}%
\pgfpathlineto{\pgfqpoint{3.636218in}{1.924230in}}%
\pgfpathlineto{\pgfqpoint{3.640727in}{1.925116in}}%
\pgfpathlineto{\pgfqpoint{3.645236in}{1.959063in}}%
\pgfpathlineto{\pgfqpoint{3.649745in}{1.953172in}}%
\pgfpathlineto{\pgfqpoint{3.654255in}{1.939864in}}%
\pgfpathlineto{\pgfqpoint{3.658764in}{1.996088in}}%
\pgfpathlineto{\pgfqpoint{3.663273in}{1.943152in}}%
\pgfpathlineto{\pgfqpoint{3.672291in}{1.997372in}}%
\pgfpathlineto{\pgfqpoint{3.681309in}{1.947659in}}%
\pgfpathlineto{\pgfqpoint{3.685818in}{2.052921in}}%
\pgfpathlineto{\pgfqpoint{3.690327in}{2.001801in}}%
\pgfpathlineto{\pgfqpoint{3.694836in}{1.983045in}}%
\pgfpathlineto{\pgfqpoint{3.699345in}{2.051360in}}%
\pgfpathlineto{\pgfqpoint{3.703855in}{1.989223in}}%
\pgfpathlineto{\pgfqpoint{3.708364in}{1.996708in}}%
\pgfpathlineto{\pgfqpoint{3.712873in}{2.017845in}}%
\pgfpathlineto{\pgfqpoint{3.717382in}{1.975007in}}%
\pgfpathlineto{\pgfqpoint{3.721891in}{2.069596in}}%
\pgfpathlineto{\pgfqpoint{3.726400in}{2.111017in}}%
\pgfpathlineto{\pgfqpoint{3.730909in}{2.114759in}}%
\pgfpathlineto{\pgfqpoint{3.735418in}{2.039745in}}%
\pgfpathlineto{\pgfqpoint{3.739927in}{2.017966in}}%
\pgfpathlineto{\pgfqpoint{3.744436in}{2.039103in}}%
\pgfpathlineto{\pgfqpoint{3.748945in}{2.006507in}}%
\pgfpathlineto{\pgfqpoint{3.753455in}{2.020890in}}%
\pgfpathlineto{\pgfqpoint{3.757964in}{2.063772in}}%
\pgfpathlineto{\pgfqpoint{3.762473in}{2.064104in}}%
\pgfpathlineto{\pgfqpoint{3.766982in}{2.038837in}}%
\pgfpathlineto{\pgfqpoint{3.771491in}{2.087931in}}%
\pgfpathlineto{\pgfqpoint{3.776000in}{2.043986in}}%
\pgfpathlineto{\pgfqpoint{3.780509in}{2.058944in}}%
\pgfpathlineto{\pgfqpoint{3.785018in}{2.092072in}}%
\pgfpathlineto{\pgfqpoint{3.789527in}{2.091308in}}%
\pgfpathlineto{\pgfqpoint{3.798545in}{2.131068in}}%
\pgfpathlineto{\pgfqpoint{3.803055in}{2.129197in}}%
\pgfpathlineto{\pgfqpoint{3.807564in}{2.098671in}}%
\pgfpathlineto{\pgfqpoint{3.816582in}{2.140679in}}%
\pgfpathlineto{\pgfqpoint{3.821091in}{2.102369in}}%
\pgfpathlineto{\pgfqpoint{3.830109in}{2.054006in}}%
\pgfpathlineto{\pgfqpoint{3.839127in}{2.095394in}}%
\pgfpathlineto{\pgfqpoint{3.843636in}{2.134877in}}%
\pgfpathlineto{\pgfqpoint{3.848145in}{2.158605in}}%
\pgfpathlineto{\pgfqpoint{3.852655in}{2.173530in}}%
\pgfpathlineto{\pgfqpoint{3.857164in}{2.148097in}}%
\pgfpathlineto{\pgfqpoint{3.861673in}{2.159191in}}%
\pgfpathlineto{\pgfqpoint{3.866182in}{2.109555in}}%
\pgfpathlineto{\pgfqpoint{3.870691in}{2.158139in}}%
\pgfpathlineto{\pgfqpoint{3.875200in}{2.160542in}}%
\pgfpathlineto{\pgfqpoint{3.879709in}{2.178158in}}%
\pgfpathlineto{\pgfqpoint{3.884218in}{2.154486in}}%
\pgfpathlineto{\pgfqpoint{3.888727in}{2.117782in}}%
\pgfpathlineto{\pgfqpoint{3.893236in}{2.210798in}}%
\pgfpathlineto{\pgfqpoint{3.897745in}{2.172090in}}%
\pgfpathlineto{\pgfqpoint{3.902255in}{2.147865in}}%
\pgfpathlineto{\pgfqpoint{3.906764in}{2.154109in}}%
\pgfpathlineto{\pgfqpoint{3.911273in}{2.240173in}}%
\pgfpathlineto{\pgfqpoint{3.915782in}{2.140169in}}%
\pgfpathlineto{\pgfqpoint{3.920291in}{2.240516in}}%
\pgfpathlineto{\pgfqpoint{3.924800in}{2.146802in}}%
\pgfpathlineto{\pgfqpoint{3.929309in}{2.144377in}}%
\pgfpathlineto{\pgfqpoint{3.933818in}{2.192009in}}%
\pgfpathlineto{\pgfqpoint{3.938327in}{2.276855in}}%
\pgfpathlineto{\pgfqpoint{3.942836in}{2.280409in}}%
\pgfpathlineto{\pgfqpoint{3.947345in}{2.315308in}}%
\pgfpathlineto{\pgfqpoint{3.951855in}{2.247646in}}%
\pgfpathlineto{\pgfqpoint{3.956364in}{2.228193in}}%
\pgfpathlineto{\pgfqpoint{3.960873in}{2.219844in}}%
\pgfpathlineto{\pgfqpoint{3.965382in}{2.205340in}}%
\pgfpathlineto{\pgfqpoint{3.969891in}{2.271939in}}%
\pgfpathlineto{\pgfqpoint{3.974400in}{2.281837in}}%
\pgfpathlineto{\pgfqpoint{3.978909in}{2.251632in}}%
\pgfpathlineto{\pgfqpoint{3.983418in}{2.186916in}}%
\pgfpathlineto{\pgfqpoint{3.987927in}{2.240117in}}%
\pgfpathlineto{\pgfqpoint{3.992436in}{2.213766in}}%
\pgfpathlineto{\pgfqpoint{3.996945in}{2.306340in}}%
\pgfpathlineto{\pgfqpoint{4.001455in}{2.245111in}}%
\pgfpathlineto{\pgfqpoint{4.005964in}{2.224439in}}%
\pgfpathlineto{\pgfqpoint{4.010473in}{2.285602in}}%
\pgfpathlineto{\pgfqpoint{4.014982in}{2.248466in}}%
\pgfpathlineto{\pgfqpoint{4.019491in}{2.325450in}}%
\pgfpathlineto{\pgfqpoint{4.024000in}{2.248754in}}%
\pgfpathlineto{\pgfqpoint{4.028509in}{2.265052in}}%
\pgfpathlineto{\pgfqpoint{4.033018in}{2.233895in}}%
\pgfpathlineto{\pgfqpoint{4.037527in}{2.327454in}}%
\pgfpathlineto{\pgfqpoint{4.042036in}{2.336688in}}%
\pgfpathlineto{\pgfqpoint{4.046545in}{2.288912in}}%
\pgfpathlineto{\pgfqpoint{4.051055in}{2.366406in}}%
\pgfpathlineto{\pgfqpoint{4.055564in}{2.372684in}}%
\pgfpathlineto{\pgfqpoint{4.060073in}{2.330078in}}%
\pgfpathlineto{\pgfqpoint{4.064582in}{2.274286in}}%
\pgfpathlineto{\pgfqpoint{4.069091in}{2.343011in}}%
\pgfpathlineto{\pgfqpoint{4.073600in}{2.318685in}}%
\pgfpathlineto{\pgfqpoint{4.078109in}{2.325129in}}%
\pgfpathlineto{\pgfqpoint{4.082618in}{2.318785in}}%
\pgfpathlineto{\pgfqpoint{4.087127in}{2.330244in}}%
\pgfpathlineto{\pgfqpoint{4.091636in}{2.382527in}}%
\pgfpathlineto{\pgfqpoint{4.096145in}{2.333865in}}%
\pgfpathlineto{\pgfqpoint{4.105164in}{2.447055in}}%
\pgfpathlineto{\pgfqpoint{4.109673in}{2.329015in}}%
\pgfpathlineto{\pgfqpoint{4.118691in}{2.427779in}}%
\pgfpathlineto{\pgfqpoint{4.123200in}{2.404848in}}%
\pgfpathlineto{\pgfqpoint{4.127709in}{2.448495in}}%
\pgfpathlineto{\pgfqpoint{4.132218in}{2.316482in}}%
\pgfpathlineto{\pgfqpoint{4.136727in}{2.382638in}}%
\pgfpathlineto{\pgfqpoint{4.145745in}{2.468469in}}%
\pgfpathlineto{\pgfqpoint{4.150255in}{2.446059in}}%
\pgfpathlineto{\pgfqpoint{4.154764in}{2.392193in}}%
\pgfpathlineto{\pgfqpoint{4.159273in}{2.457308in}}%
\pgfpathlineto{\pgfqpoint{4.163782in}{2.456345in}}%
\pgfpathlineto{\pgfqpoint{4.168291in}{2.397652in}}%
\pgfpathlineto{\pgfqpoint{4.172800in}{2.427812in}}%
\pgfpathlineto{\pgfqpoint{4.177309in}{2.421257in}}%
\pgfpathlineto{\pgfqpoint{4.181818in}{2.432983in}}%
\pgfpathlineto{\pgfqpoint{4.195345in}{2.555939in}}%
\pgfpathlineto{\pgfqpoint{4.199855in}{2.445793in}}%
\pgfpathlineto{\pgfqpoint{4.204364in}{2.518338in}}%
\pgfpathlineto{\pgfqpoint{4.208873in}{2.393632in}}%
\pgfpathlineto{\pgfqpoint{4.213382in}{2.491809in}}%
\pgfpathlineto{\pgfqpoint{4.217891in}{2.500855in}}%
\pgfpathlineto{\pgfqpoint{4.226909in}{2.457120in}}%
\pgfpathlineto{\pgfqpoint{4.231418in}{2.498729in}}%
\pgfpathlineto{\pgfqpoint{4.235927in}{2.457142in}}%
\pgfpathlineto{\pgfqpoint{4.240436in}{2.509037in}}%
\pgfpathlineto{\pgfqpoint{4.244945in}{2.484302in}}%
\pgfpathlineto{\pgfqpoint{4.249455in}{2.495551in}}%
\pgfpathlineto{\pgfqpoint{4.253964in}{2.543405in}}%
\pgfpathlineto{\pgfqpoint{4.258473in}{2.551233in}}%
\pgfpathlineto{\pgfqpoint{4.262982in}{2.516533in}}%
\pgfpathlineto{\pgfqpoint{4.267491in}{2.532443in}}%
\pgfpathlineto{\pgfqpoint{4.272000in}{2.603250in}}%
\pgfpathlineto{\pgfqpoint{4.276509in}{2.536352in}}%
\pgfpathlineto{\pgfqpoint{4.281018in}{2.437799in}}%
\pgfpathlineto{\pgfqpoint{4.285527in}{2.573587in}}%
\pgfpathlineto{\pgfqpoint{4.290036in}{2.605420in}}%
\pgfpathlineto{\pgfqpoint{4.294545in}{2.538544in}}%
\pgfpathlineto{\pgfqpoint{4.299055in}{2.616669in}}%
\pgfpathlineto{\pgfqpoint{4.303564in}{2.629358in}}%
\pgfpathlineto{\pgfqpoint{4.308073in}{2.616160in}}%
\pgfpathlineto{\pgfqpoint{4.312582in}{2.538932in}}%
\pgfpathlineto{\pgfqpoint{4.321600in}{2.590605in}}%
\pgfpathlineto{\pgfqpoint{4.326109in}{2.674244in}}%
\pgfpathlineto{\pgfqpoint{4.330618in}{2.554521in}}%
\pgfpathlineto{\pgfqpoint{4.335127in}{2.629557in}}%
\pgfpathlineto{\pgfqpoint{4.339636in}{2.583264in}}%
\pgfpathlineto{\pgfqpoint{4.344145in}{2.501907in}}%
\pgfpathlineto{\pgfqpoint{4.348655in}{2.605243in}}%
\pgfpathlineto{\pgfqpoint{4.353164in}{2.665874in}}%
\pgfpathlineto{\pgfqpoint{4.357673in}{2.557810in}}%
\pgfpathlineto{\pgfqpoint{4.362182in}{2.548963in}}%
\pgfpathlineto{\pgfqpoint{4.366691in}{2.622305in}}%
\pgfpathlineto{\pgfqpoint{4.371200in}{2.588568in}}%
\pgfpathlineto{\pgfqpoint{4.375709in}{2.639489in}}%
\pgfpathlineto{\pgfqpoint{4.380218in}{2.668188in}}%
\pgfpathlineto{\pgfqpoint{4.384727in}{2.642777in}}%
\pgfpathlineto{\pgfqpoint{4.389236in}{2.671830in}}%
\pgfpathlineto{\pgfqpoint{4.393745in}{2.667379in}}%
\pgfpathlineto{\pgfqpoint{4.398255in}{2.586929in}}%
\pgfpathlineto{\pgfqpoint{4.402764in}{2.603593in}}%
\pgfpathlineto{\pgfqpoint{4.407273in}{2.688184in}}%
\pgfpathlineto{\pgfqpoint{4.416291in}{2.720503in}}%
\pgfpathlineto{\pgfqpoint{4.420800in}{2.699079in}}%
\pgfpathlineto{\pgfqpoint{4.425309in}{2.641305in}}%
\pgfpathlineto{\pgfqpoint{4.429818in}{2.700430in}}%
\pgfpathlineto{\pgfqpoint{4.434327in}{2.689490in}}%
\pgfpathlineto{\pgfqpoint{4.438836in}{2.683467in}}%
\pgfpathlineto{\pgfqpoint{4.443345in}{2.697174in}}%
\pgfpathlineto{\pgfqpoint{4.447855in}{2.691284in}}%
\pgfpathlineto{\pgfqpoint{4.452364in}{2.747541in}}%
\pgfpathlineto{\pgfqpoint{4.456873in}{2.701382in}}%
\pgfpathlineto{\pgfqpoint{4.461382in}{2.764703in}}%
\pgfpathlineto{\pgfqpoint{4.465891in}{2.671697in}}%
\pgfpathlineto{\pgfqpoint{4.470400in}{2.682858in}}%
\pgfpathlineto{\pgfqpoint{4.474909in}{2.657293in}}%
\pgfpathlineto{\pgfqpoint{4.479418in}{2.797599in}}%
\pgfpathlineto{\pgfqpoint{4.483927in}{2.705146in}}%
\pgfpathlineto{\pgfqpoint{4.488436in}{2.761935in}}%
\pgfpathlineto{\pgfqpoint{4.492945in}{2.896395in}}%
\pgfpathlineto{\pgfqpoint{4.497455in}{2.824216in}}%
\pgfpathlineto{\pgfqpoint{4.501964in}{2.820772in}}%
\pgfpathlineto{\pgfqpoint{4.506473in}{2.820341in}}%
\pgfpathlineto{\pgfqpoint{4.510982in}{2.830660in}}%
\pgfpathlineto{\pgfqpoint{4.515491in}{2.739536in}}%
\pgfpathlineto{\pgfqpoint{4.520000in}{2.777824in}}%
\pgfpathlineto{\pgfqpoint{4.524509in}{2.772863in}}%
\pgfpathlineto{\pgfqpoint{4.529018in}{2.815004in}}%
\pgfpathlineto{\pgfqpoint{4.533527in}{2.818879in}}%
\pgfpathlineto{\pgfqpoint{4.538036in}{2.797355in}}%
\pgfpathlineto{\pgfqpoint{4.542545in}{2.858263in}}%
\pgfpathlineto{\pgfqpoint{4.547055in}{2.870608in}}%
\pgfpathlineto{\pgfqpoint{4.551564in}{2.816122in}}%
\pgfpathlineto{\pgfqpoint{4.556073in}{2.721600in}}%
\pgfpathlineto{\pgfqpoint{4.565091in}{2.855915in}}%
\pgfpathlineto{\pgfqpoint{4.569600in}{2.826851in}}%
\pgfpathlineto{\pgfqpoint{4.574109in}{2.841787in}}%
\pgfpathlineto{\pgfqpoint{4.578618in}{2.818835in}}%
\pgfpathlineto{\pgfqpoint{4.583127in}{2.830970in}}%
\pgfpathlineto{\pgfqpoint{4.587636in}{2.947039in}}%
\pgfpathlineto{\pgfqpoint{4.592145in}{2.840370in}}%
\pgfpathlineto{\pgfqpoint{4.596655in}{2.904932in}}%
\pgfpathlineto{\pgfqpoint{4.601164in}{2.818669in}}%
\pgfpathlineto{\pgfqpoint{4.605673in}{2.863046in}}%
\pgfpathlineto{\pgfqpoint{4.610182in}{2.869047in}}%
\pgfpathlineto{\pgfqpoint{4.614691in}{2.863588in}}%
\pgfpathlineto{\pgfqpoint{4.619200in}{2.874362in}}%
\pgfpathlineto{\pgfqpoint{4.623709in}{2.922570in}}%
\pgfpathlineto{\pgfqpoint{4.628218in}{2.945732in}}%
\pgfpathlineto{\pgfqpoint{4.632727in}{2.933343in}}%
\pgfpathlineto{\pgfqpoint{4.637236in}{2.863998in}}%
\pgfpathlineto{\pgfqpoint{4.641745in}{2.913136in}}%
\pgfpathlineto{\pgfqpoint{4.646255in}{2.837215in}}%
\pgfpathlineto{\pgfqpoint{4.650764in}{2.872081in}}%
\pgfpathlineto{\pgfqpoint{4.655273in}{2.871228in}}%
\pgfpathlineto{\pgfqpoint{4.659782in}{2.958100in}}%
\pgfpathlineto{\pgfqpoint{4.664291in}{2.926412in}}%
\pgfpathlineto{\pgfqpoint{4.668800in}{2.986932in}}%
\pgfpathlineto{\pgfqpoint{4.673309in}{2.913280in}}%
\pgfpathlineto{\pgfqpoint{4.677818in}{3.024167in}}%
\pgfpathlineto{\pgfqpoint{4.682327in}{2.981196in}}%
\pgfpathlineto{\pgfqpoint{4.686836in}{2.982160in}}%
\pgfpathlineto{\pgfqpoint{4.691345in}{2.955686in}}%
\pgfpathlineto{\pgfqpoint{4.695855in}{2.986887in}}%
\pgfpathlineto{\pgfqpoint{4.704873in}{2.943518in}}%
\pgfpathlineto{\pgfqpoint{4.709382in}{2.993132in}}%
\pgfpathlineto{\pgfqpoint{4.713891in}{3.002167in}}%
\pgfpathlineto{\pgfqpoint{4.718400in}{3.029725in}}%
\pgfpathlineto{\pgfqpoint{4.722909in}{2.987806in}}%
\pgfpathlineto{\pgfqpoint{4.727418in}{2.992169in}}%
\pgfpathlineto{\pgfqpoint{4.731927in}{2.934627in}}%
\pgfpathlineto{\pgfqpoint{4.736436in}{3.108282in}}%
\pgfpathlineto{\pgfqpoint{4.740945in}{3.066175in}}%
\pgfpathlineto{\pgfqpoint{4.745455in}{2.995801in}}%
\pgfpathlineto{\pgfqpoint{4.749964in}{3.028076in}}%
\pgfpathlineto{\pgfqpoint{4.754473in}{3.017613in}}%
\pgfpathlineto{\pgfqpoint{4.758982in}{3.036524in}}%
\pgfpathlineto{\pgfqpoint{4.763491in}{3.191744in}}%
\pgfpathlineto{\pgfqpoint{4.772509in}{3.069131in}}%
\pgfpathlineto{\pgfqpoint{4.777018in}{3.030777in}}%
\pgfpathlineto{\pgfqpoint{4.781527in}{3.123108in}}%
\pgfpathlineto{\pgfqpoint{4.786036in}{2.995933in}}%
\pgfpathlineto{\pgfqpoint{4.790545in}{2.989689in}}%
\pgfpathlineto{\pgfqpoint{4.795055in}{3.158705in}}%
\pgfpathlineto{\pgfqpoint{4.799564in}{3.002975in}}%
\pgfpathlineto{\pgfqpoint{4.804073in}{3.148374in}}%
\pgfpathlineto{\pgfqpoint{4.808582in}{3.118147in}}%
\pgfpathlineto{\pgfqpoint{4.813091in}{3.140580in}}%
\pgfpathlineto{\pgfqpoint{4.817600in}{3.125200in}}%
\pgfpathlineto{\pgfqpoint{4.822109in}{3.143658in}}%
\pgfpathlineto{\pgfqpoint{4.826618in}{3.227861in}}%
\pgfpathlineto{\pgfqpoint{4.831127in}{3.242687in}}%
\pgfpathlineto{\pgfqpoint{4.835636in}{3.150024in}}%
\pgfpathlineto{\pgfqpoint{4.840145in}{3.133239in}}%
\pgfpathlineto{\pgfqpoint{4.844655in}{3.123052in}}%
\pgfpathlineto{\pgfqpoint{4.849164in}{3.194833in}}%
\pgfpathlineto{\pgfqpoint{4.853673in}{3.165824in}}%
\pgfpathlineto{\pgfqpoint{4.858182in}{3.127127in}}%
\pgfpathlineto{\pgfqpoint{4.862691in}{3.180915in}}%
\pgfpathlineto{\pgfqpoint{4.867200in}{3.248588in}}%
\pgfpathlineto{\pgfqpoint{4.871709in}{3.246562in}}%
\pgfpathlineto{\pgfqpoint{4.876218in}{3.174261in}}%
\pgfpathlineto{\pgfqpoint{4.880727in}{3.186252in}}%
\pgfpathlineto{\pgfqpoint{4.885236in}{3.314002in}}%
\pgfpathlineto{\pgfqpoint{4.894255in}{3.128943in}}%
\pgfpathlineto{\pgfqpoint{4.898764in}{3.277143in}}%
\pgfpathlineto{\pgfqpoint{4.903273in}{3.262185in}}%
\pgfpathlineto{\pgfqpoint{4.907782in}{3.287064in}}%
\pgfpathlineto{\pgfqpoint{4.912291in}{3.253238in}}%
\pgfpathlineto{\pgfqpoint{4.916800in}{3.199793in}}%
\pgfpathlineto{\pgfqpoint{4.921309in}{3.303771in}}%
\pgfpathlineto{\pgfqpoint{4.930327in}{3.211341in}}%
\pgfpathlineto{\pgfqpoint{4.934836in}{3.287783in}}%
\pgfpathlineto{\pgfqpoint{4.939345in}{3.248821in}}%
\pgfpathlineto{\pgfqpoint{4.943855in}{3.263746in}}%
\pgfpathlineto{\pgfqpoint{4.948364in}{3.344450in}}%
\pgfpathlineto{\pgfqpoint{4.952873in}{3.245410in}}%
\pgfpathlineto{\pgfqpoint{4.957382in}{3.224750in}}%
\pgfpathlineto{\pgfqpoint{4.961891in}{3.333400in}}%
\pgfpathlineto{\pgfqpoint{4.966400in}{3.255718in}}%
\pgfpathlineto{\pgfqpoint{4.970909in}{3.373404in}}%
\pgfpathlineto{\pgfqpoint{4.975418in}{3.387089in}}%
\pgfpathlineto{\pgfqpoint{4.979927in}{3.271219in}}%
\pgfpathlineto{\pgfqpoint{4.984436in}{3.377135in}}%
\pgfpathlineto{\pgfqpoint{4.988945in}{3.288547in}}%
\pgfpathlineto{\pgfqpoint{4.993455in}{3.385594in}}%
\pgfpathlineto{\pgfqpoint{4.997964in}{3.320313in}}%
\pgfpathlineto{\pgfqpoint{5.002473in}{3.353696in}}%
\pgfpathlineto{\pgfqpoint{5.006982in}{3.397896in}}%
\pgfpathlineto{\pgfqpoint{5.011491in}{3.196018in}}%
\pgfpathlineto{\pgfqpoint{5.016000in}{3.465004in}}%
\pgfpathlineto{\pgfqpoint{5.020509in}{3.398017in}}%
\pgfpathlineto{\pgfqpoint{5.025018in}{3.290418in}}%
\pgfpathlineto{\pgfqpoint{5.029527in}{3.419874in}}%
\pgfpathlineto{\pgfqpoint{5.034036in}{3.401815in}}%
\pgfpathlineto{\pgfqpoint{5.038545in}{3.378132in}}%
\pgfpathlineto{\pgfqpoint{5.043055in}{3.401239in}}%
\pgfpathlineto{\pgfqpoint{5.047564in}{3.415943in}}%
\pgfpathlineto{\pgfqpoint{5.052073in}{3.381254in}}%
\pgfpathlineto{\pgfqpoint{5.056582in}{3.311699in}}%
\pgfpathlineto{\pgfqpoint{5.061091in}{3.302786in}}%
\pgfpathlineto{\pgfqpoint{5.065600in}{3.458726in}}%
\pgfpathlineto{\pgfqpoint{5.070109in}{3.375131in}}%
\pgfpathlineto{\pgfqpoint{5.074618in}{3.432441in}}%
\pgfpathlineto{\pgfqpoint{5.079127in}{3.454895in}}%
\pgfpathlineto{\pgfqpoint{5.083636in}{3.456622in}}%
\pgfpathlineto{\pgfqpoint{5.088145in}{3.463985in}}%
\pgfpathlineto{\pgfqpoint{5.092655in}{3.522534in}}%
\pgfpathlineto{\pgfqpoint{5.097164in}{3.539973in}}%
\pgfpathlineto{\pgfqpoint{5.101673in}{3.480051in}}%
\pgfpathlineto{\pgfqpoint{5.106182in}{3.386314in}}%
\pgfpathlineto{\pgfqpoint{5.110691in}{3.418490in}}%
\pgfpathlineto{\pgfqpoint{5.115200in}{3.521017in}}%
\pgfpathlineto{\pgfqpoint{5.119709in}{3.378364in}}%
\pgfpathlineto{\pgfqpoint{5.128727in}{3.535976in}}%
\pgfpathlineto{\pgfqpoint{5.133236in}{3.499228in}}%
\pgfpathlineto{\pgfqpoint{5.137745in}{3.453079in}}%
\pgfpathlineto{\pgfqpoint{5.142255in}{3.523885in}}%
\pgfpathlineto{\pgfqpoint{5.146764in}{3.506258in}}%
\pgfpathlineto{\pgfqpoint{5.151273in}{3.509336in}}%
\pgfpathlineto{\pgfqpoint{5.155782in}{3.485565in}}%
\pgfpathlineto{\pgfqpoint{5.164800in}{3.636743in}}%
\pgfpathlineto{\pgfqpoint{5.169309in}{3.552219in}}%
\pgfpathlineto{\pgfqpoint{5.173818in}{3.538069in}}%
\pgfpathlineto{\pgfqpoint{5.178327in}{3.563977in}}%
\pgfpathlineto{\pgfqpoint{5.182836in}{3.568805in}}%
\pgfpathlineto{\pgfqpoint{5.187345in}{3.540814in}}%
\pgfpathlineto{\pgfqpoint{5.191855in}{3.593695in}}%
\pgfpathlineto{\pgfqpoint{5.196364in}{3.536518in}}%
\pgfpathlineto{\pgfqpoint{5.200873in}{3.609473in}}%
\pgfpathlineto{\pgfqpoint{5.205382in}{3.754783in}}%
\pgfpathlineto{\pgfqpoint{5.209891in}{3.655245in}}%
\pgfpathlineto{\pgfqpoint{5.214400in}{3.595466in}}%
\pgfpathlineto{\pgfqpoint{5.218909in}{3.618076in}}%
\pgfpathlineto{\pgfqpoint{5.223418in}{3.609284in}}%
\pgfpathlineto{\pgfqpoint{5.227927in}{3.518559in}}%
\pgfpathlineto{\pgfqpoint{5.232436in}{3.694573in}}%
\pgfpathlineto{\pgfqpoint{5.236945in}{3.692691in}}%
\pgfpathlineto{\pgfqpoint{5.241455in}{3.588701in}}%
\pgfpathlineto{\pgfqpoint{5.245964in}{3.593949in}}%
\pgfpathlineto{\pgfqpoint{5.250473in}{3.774536in}}%
\pgfpathlineto{\pgfqpoint{5.254982in}{3.603538in}}%
\pgfpathlineto{\pgfqpoint{5.259491in}{3.583486in}}%
\pgfpathlineto{\pgfqpoint{5.264000in}{3.735196in}}%
\pgfpathlineto{\pgfqpoint{5.268509in}{3.652698in}}%
\pgfpathlineto{\pgfqpoint{5.273018in}{3.634363in}}%
\pgfpathlineto{\pgfqpoint{5.277527in}{3.747785in}}%
\pgfpathlineto{\pgfqpoint{5.282036in}{3.675230in}}%
\pgfpathlineto{\pgfqpoint{5.286545in}{3.719651in}}%
\pgfpathlineto{\pgfqpoint{5.291055in}{3.716097in}}%
\pgfpathlineto{\pgfqpoint{5.295564in}{3.698481in}}%
\pgfpathlineto{\pgfqpoint{5.300073in}{3.729395in}}%
\pgfpathlineto{\pgfqpoint{5.304582in}{3.743656in}}%
\pgfpathlineto{\pgfqpoint{5.309091in}{3.739315in}}%
\pgfpathlineto{\pgfqpoint{5.313600in}{3.771081in}}%
\pgfpathlineto{\pgfqpoint{5.318109in}{3.701117in}}%
\pgfpathlineto{\pgfqpoint{5.322618in}{3.771889in}}%
\pgfpathlineto{\pgfqpoint{5.327127in}{3.698814in}}%
\pgfpathlineto{\pgfqpoint{5.336145in}{3.844999in}}%
\pgfpathlineto{\pgfqpoint{5.340655in}{3.677699in}}%
\pgfpathlineto{\pgfqpoint{5.345164in}{3.754340in}}%
\pgfpathlineto{\pgfqpoint{5.349673in}{3.758149in}}%
\pgfpathlineto{\pgfqpoint{5.354182in}{3.870077in}}%
\pgfpathlineto{\pgfqpoint{5.358691in}{3.935425in}}%
\pgfpathlineto{\pgfqpoint{5.363200in}{3.833428in}}%
\pgfpathlineto{\pgfqpoint{5.367709in}{3.828457in}}%
\pgfpathlineto{\pgfqpoint{5.372218in}{3.844578in}}%
\pgfpathlineto{\pgfqpoint{5.376727in}{3.828878in}}%
\pgfpathlineto{\pgfqpoint{5.381236in}{3.751561in}}%
\pgfpathlineto{\pgfqpoint{5.385745in}{3.923865in}}%
\pgfpathlineto{\pgfqpoint{5.390255in}{3.694285in}}%
\pgfpathlineto{\pgfqpoint{5.394764in}{3.847578in}}%
\pgfpathlineto{\pgfqpoint{5.399273in}{3.810033in}}%
\pgfpathlineto{\pgfqpoint{5.403782in}{3.876543in}}%
\pgfpathlineto{\pgfqpoint{5.408291in}{3.847877in}}%
\pgfpathlineto{\pgfqpoint{5.412800in}{3.833550in}}%
\pgfpathlineto{\pgfqpoint{5.417309in}{3.927796in}}%
\pgfpathlineto{\pgfqpoint{5.421818in}{3.829985in}}%
\pgfpathlineto{\pgfqpoint{5.426327in}{3.929446in}}%
\pgfpathlineto{\pgfqpoint{5.430836in}{3.925438in}}%
\pgfpathlineto{\pgfqpoint{5.435345in}{3.972483in}}%
\pgfpathlineto{\pgfqpoint{5.439855in}{3.850048in}}%
\pgfpathlineto{\pgfqpoint{5.444364in}{3.693909in}}%
\pgfpathlineto{\pgfqpoint{5.448873in}{3.956583in}}%
\pgfpathlineto{\pgfqpoint{5.453382in}{3.883596in}}%
\pgfpathlineto{\pgfqpoint{5.457891in}{3.889221in}}%
\pgfpathlineto{\pgfqpoint{5.466909in}{4.033645in}}%
\pgfpathlineto{\pgfqpoint{5.475927in}{3.859625in}}%
\pgfpathlineto{\pgfqpoint{5.480436in}{3.994494in}}%
\pgfpathlineto{\pgfqpoint{5.484945in}{4.016949in}}%
\pgfpathlineto{\pgfqpoint{5.489455in}{4.029305in}}%
\pgfpathlineto{\pgfqpoint{5.493964in}{3.894823in}}%
\pgfpathlineto{\pgfqpoint{5.498473in}{3.916779in}}%
\pgfpathlineto{\pgfqpoint{5.502982in}{3.977953in}}%
\pgfpathlineto{\pgfqpoint{5.507491in}{3.980145in}}%
\pgfpathlineto{\pgfqpoint{5.512000in}{4.056000in}}%
\pgfpathlineto{\pgfqpoint{5.516509in}{3.927663in}}%
\pgfpathlineto{\pgfqpoint{5.521018in}{3.973801in}}%
\pgfpathlineto{\pgfqpoint{5.525527in}{3.964721in}}%
\pgfpathlineto{\pgfqpoint{5.530036in}{3.869357in}}%
\pgfpathlineto{\pgfqpoint{5.534545in}{4.016760in}}%
\pgfpathlineto{\pgfqpoint{5.534545in}{4.016760in}}%
\pgfusepath{stroke}%
\end{pgfscope}%
\begin{pgfscope}%
\pgfsetrectcap%
\pgfsetmiterjoin%
\pgfsetlinewidth{0.803000pt}%
\definecolor{currentstroke}{rgb}{0.000000,0.000000,0.000000}%
\pgfsetstrokecolor{currentstroke}%
\pgfsetdash{}{0pt}%
\pgfpathmoveto{\pgfqpoint{0.800000in}{0.528000in}}%
\pgfpathlineto{\pgfqpoint{0.800000in}{4.224000in}}%
\pgfusepath{stroke}%
\end{pgfscope}%
\begin{pgfscope}%
\pgfsetrectcap%
\pgfsetmiterjoin%
\pgfsetlinewidth{0.803000pt}%
\definecolor{currentstroke}{rgb}{0.000000,0.000000,0.000000}%
\pgfsetstrokecolor{currentstroke}%
\pgfsetdash{}{0pt}%
\pgfpathmoveto{\pgfqpoint{5.760000in}{0.528000in}}%
\pgfpathlineto{\pgfqpoint{5.760000in}{4.224000in}}%
\pgfusepath{stroke}%
\end{pgfscope}%
\begin{pgfscope}%
\pgfsetrectcap%
\pgfsetmiterjoin%
\pgfsetlinewidth{0.803000pt}%
\definecolor{currentstroke}{rgb}{0.000000,0.000000,0.000000}%
\pgfsetstrokecolor{currentstroke}%
\pgfsetdash{}{0pt}%
\pgfpathmoveto{\pgfqpoint{0.800000in}{0.528000in}}%
\pgfpathlineto{\pgfqpoint{5.760000in}{0.528000in}}%
\pgfusepath{stroke}%
\end{pgfscope}%
\begin{pgfscope}%
\pgfsetrectcap%
\pgfsetmiterjoin%
\pgfsetlinewidth{0.803000pt}%
\definecolor{currentstroke}{rgb}{0.000000,0.000000,0.000000}%
\pgfsetstrokecolor{currentstroke}%
\pgfsetdash{}{0pt}%
\pgfpathmoveto{\pgfqpoint{0.800000in}{4.224000in}}%
\pgfpathlineto{\pgfqpoint{5.760000in}{4.224000in}}%
\pgfusepath{stroke}%
\end{pgfscope}%
\begin{pgfscope}%
\definecolor{textcolor}{rgb}{0.000000,0.000000,0.000000}%
\pgfsetstrokecolor{textcolor}%
\pgfsetfillcolor{textcolor}%
\pgftext[x=3.280000in,y=4.307333in,,base]{\color{textcolor}\ttfamily\fontsize{12.000000}{14.400000}\selectfont Bubble Sort Swaps vs Input size}%
\end{pgfscope}%
\begin{pgfscope}%
\pgfsetbuttcap%
\pgfsetmiterjoin%
\definecolor{currentfill}{rgb}{1.000000,1.000000,1.000000}%
\pgfsetfillcolor{currentfill}%
\pgfsetfillopacity{0.800000}%
\pgfsetlinewidth{1.003750pt}%
\definecolor{currentstroke}{rgb}{0.800000,0.800000,0.800000}%
\pgfsetstrokecolor{currentstroke}%
\pgfsetstrokeopacity{0.800000}%
\pgfsetdash{}{0pt}%
\pgfpathmoveto{\pgfqpoint{0.897222in}{3.908286in}}%
\pgfpathlineto{\pgfqpoint{1.843376in}{3.908286in}}%
\pgfpathquadraticcurveto{\pgfqpoint{1.871153in}{3.908286in}}{\pgfqpoint{1.871153in}{3.936063in}}%
\pgfpathlineto{\pgfqpoint{1.871153in}{4.126778in}}%
\pgfpathquadraticcurveto{\pgfqpoint{1.871153in}{4.154556in}}{\pgfqpoint{1.843376in}{4.154556in}}%
\pgfpathlineto{\pgfqpoint{0.897222in}{4.154556in}}%
\pgfpathquadraticcurveto{\pgfqpoint{0.869444in}{4.154556in}}{\pgfqpoint{0.869444in}{4.126778in}}%
\pgfpathlineto{\pgfqpoint{0.869444in}{3.936063in}}%
\pgfpathquadraticcurveto{\pgfqpoint{0.869444in}{3.908286in}}{\pgfqpoint{0.897222in}{3.908286in}}%
\pgfpathlineto{\pgfqpoint{0.897222in}{3.908286in}}%
\pgfpathclose%
\pgfusepath{stroke,fill}%
\end{pgfscope}%
\begin{pgfscope}%
\pgfsetrectcap%
\pgfsetroundjoin%
\pgfsetlinewidth{1.505625pt}%
\definecolor{currentstroke}{rgb}{0.000000,1.000000,0.498039}%
\pgfsetstrokecolor{currentstroke}%
\pgfsetdash{}{0pt}%
\pgfpathmoveto{\pgfqpoint{0.925000in}{4.041342in}}%
\pgfpathlineto{\pgfqpoint{1.063889in}{4.041342in}}%
\pgfpathlineto{\pgfqpoint{1.202778in}{4.041342in}}%
\pgfusepath{stroke}%
\end{pgfscope}%
\begin{pgfscope}%
\definecolor{textcolor}{rgb}{0.000000,0.000000,0.000000}%
\pgfsetstrokecolor{textcolor}%
\pgfsetfillcolor{textcolor}%
\pgftext[x=1.313889in,y=3.992731in,left,base]{\color{textcolor}\ttfamily\fontsize{10.000000}{12.000000}\selectfont Bubble}%
\end{pgfscope}%
\end{pgfpicture}%
\makeatother%
\endgroup%

\input{../pgf/bs_i.pgf}
\subsubsection*{Insights}
It has a runtime complexity of $O(N^2)$. Its space complexity is
$O(1)$.
\subsection{Selection Sort}
\subsubsection{Principle}
To sort an array in ascending order using selection sort, in each
iteration we repeatedly select the smallest element from the
unsorted part of the array and append it to the end of the sorted
part. In the first iteration, take the 0th element as minimum and
compare with the next element. If it is larger than the next
element, assign the 1st element as minimum and repeat till the
end of the array. At the end of the first iteration we get the
smallest element of the array. At this stage place the minimum
element in the beginning of the array and proceed to the next
iteration to find the next smallest value.
\subsubsection*{Code}
\begin{minted}{python}
def selectionSort(unsortedList):
    swap = 0
    itr = 0
    comp = 0
    tracemalloc.start()
    t_s = perf_counter_ns()
    for i in range(len(unsortedList)):
        minIndex = i
        for j in range(i + 1, len(unsortedList)):
            itr += 1
            comp += 1
            if unsortedList[minIndex] > unsortedList[j]:
                minIndex = j
        (unsortedList[i], unsortedList[minIndex]) = \
            (unsortedList[minIndex], unsortedList[i])
        swap += 1
    t_e = perf_counter_ns()
    mem = tracemalloc.get_traced_memory()[1]
    tracemalloc.stop()
    return {"Time":t_e-t_s,
            "Memory":mem,
            "Comparisons":comp,
            "Swaps":swap,
            "Iterations":itr}
\end{minted}
\subsubsection*{Graphs}
%% Creator: Matplotlib, PGF backend
%%
%% To include the figure in your LaTeX document, write
%%   \input{<filename>.pgf}
%%
%% Make sure the required packages are loaded in your preamble
%%   \usepackage{pgf}
%%
%% Also ensure that all the required font packages are loaded; for instance,
%% the lmodern package is sometimes necessary when using math font.
%%   \usepackage{lmodern}
%%
%% Figures using additional raster images can only be included by \input if
%% they are in the same directory as the main LaTeX file. For loading figures
%% from other directories you can use the `import` package
%%   \usepackage{import}
%%
%% and then include the figures with
%%   \import{<path to file>}{<filename>.pgf}
%%
%% Matplotlib used the following preamble
%%   \usepackage{fontspec}
%%   \setmainfont{DejaVuSerif.ttf}[Path=\detokenize{/home/dbk/.local/lib/python3.10/site-packages/matplotlib/mpl-data/fonts/ttf/}]
%%   \setsansfont{DejaVuSans.ttf}[Path=\detokenize{/home/dbk/.local/lib/python3.10/site-packages/matplotlib/mpl-data/fonts/ttf/}]
%%   \setmonofont{DejaVuSansMono.ttf}[Path=\detokenize{/home/dbk/.local/lib/python3.10/site-packages/matplotlib/mpl-data/fonts/ttf/}]
%%
\begingroup%
\makeatletter%
\begin{pgfpicture}%
\pgfpathrectangle{\pgfpointorigin}{\pgfqpoint{6.400000in}{4.800000in}}%
\pgfusepath{use as bounding box, clip}%
\begin{pgfscope}%
\pgfsetbuttcap%
\pgfsetmiterjoin%
\definecolor{currentfill}{rgb}{1.000000,1.000000,1.000000}%
\pgfsetfillcolor{currentfill}%
\pgfsetlinewidth{0.000000pt}%
\definecolor{currentstroke}{rgb}{1.000000,1.000000,1.000000}%
\pgfsetstrokecolor{currentstroke}%
\pgfsetdash{}{0pt}%
\pgfpathmoveto{\pgfqpoint{0.000000in}{0.000000in}}%
\pgfpathlineto{\pgfqpoint{6.400000in}{0.000000in}}%
\pgfpathlineto{\pgfqpoint{6.400000in}{4.800000in}}%
\pgfpathlineto{\pgfqpoint{0.000000in}{4.800000in}}%
\pgfpathlineto{\pgfqpoint{0.000000in}{0.000000in}}%
\pgfpathclose%
\pgfusepath{fill}%
\end{pgfscope}%
\begin{pgfscope}%
\pgfsetbuttcap%
\pgfsetmiterjoin%
\definecolor{currentfill}{rgb}{1.000000,1.000000,1.000000}%
\pgfsetfillcolor{currentfill}%
\pgfsetlinewidth{0.000000pt}%
\definecolor{currentstroke}{rgb}{0.000000,0.000000,0.000000}%
\pgfsetstrokecolor{currentstroke}%
\pgfsetstrokeopacity{0.000000}%
\pgfsetdash{}{0pt}%
\pgfpathmoveto{\pgfqpoint{0.800000in}{0.528000in}}%
\pgfpathlineto{\pgfqpoint{5.760000in}{0.528000in}}%
\pgfpathlineto{\pgfqpoint{5.760000in}{4.224000in}}%
\pgfpathlineto{\pgfqpoint{0.800000in}{4.224000in}}%
\pgfpathlineto{\pgfqpoint{0.800000in}{0.528000in}}%
\pgfpathclose%
\pgfusepath{fill}%
\end{pgfscope}%
\begin{pgfscope}%
\pgfsetbuttcap%
\pgfsetroundjoin%
\definecolor{currentfill}{rgb}{0.000000,0.000000,0.000000}%
\pgfsetfillcolor{currentfill}%
\pgfsetlinewidth{0.803000pt}%
\definecolor{currentstroke}{rgb}{0.000000,0.000000,0.000000}%
\pgfsetstrokecolor{currentstroke}%
\pgfsetdash{}{0pt}%
\pgfsys@defobject{currentmarker}{\pgfqpoint{0.000000in}{-0.048611in}}{\pgfqpoint{0.000000in}{0.000000in}}{%
\pgfpathmoveto{\pgfqpoint{0.000000in}{0.000000in}}%
\pgfpathlineto{\pgfqpoint{0.000000in}{-0.048611in}}%
\pgfusepath{stroke,fill}%
}%
\begin{pgfscope}%
\pgfsys@transformshift{1.020945in}{0.528000in}%
\pgfsys@useobject{currentmarker}{}%
\end{pgfscope}%
\end{pgfscope}%
\begin{pgfscope}%
\definecolor{textcolor}{rgb}{0.000000,0.000000,0.000000}%
\pgfsetstrokecolor{textcolor}%
\pgfsetfillcolor{textcolor}%
\pgftext[x=1.020945in,y=0.430778in,,top]{\color{textcolor}\ttfamily\fontsize{10.000000}{12.000000}\selectfont 0}%
\end{pgfscope}%
\begin{pgfscope}%
\pgfsetbuttcap%
\pgfsetroundjoin%
\definecolor{currentfill}{rgb}{0.000000,0.000000,0.000000}%
\pgfsetfillcolor{currentfill}%
\pgfsetlinewidth{0.803000pt}%
\definecolor{currentstroke}{rgb}{0.000000,0.000000,0.000000}%
\pgfsetstrokecolor{currentstroke}%
\pgfsetdash{}{0pt}%
\pgfsys@defobject{currentmarker}{\pgfqpoint{0.000000in}{-0.048611in}}{\pgfqpoint{0.000000in}{0.000000in}}{%
\pgfpathmoveto{\pgfqpoint{0.000000in}{0.000000in}}%
\pgfpathlineto{\pgfqpoint{0.000000in}{-0.048611in}}%
\pgfusepath{stroke,fill}%
}%
\begin{pgfscope}%
\pgfsys@transformshift{1.922764in}{0.528000in}%
\pgfsys@useobject{currentmarker}{}%
\end{pgfscope}%
\end{pgfscope}%
\begin{pgfscope}%
\definecolor{textcolor}{rgb}{0.000000,0.000000,0.000000}%
\pgfsetstrokecolor{textcolor}%
\pgfsetfillcolor{textcolor}%
\pgftext[x=1.922764in,y=0.430778in,,top]{\color{textcolor}\ttfamily\fontsize{10.000000}{12.000000}\selectfont 200}%
\end{pgfscope}%
\begin{pgfscope}%
\pgfsetbuttcap%
\pgfsetroundjoin%
\definecolor{currentfill}{rgb}{0.000000,0.000000,0.000000}%
\pgfsetfillcolor{currentfill}%
\pgfsetlinewidth{0.803000pt}%
\definecolor{currentstroke}{rgb}{0.000000,0.000000,0.000000}%
\pgfsetstrokecolor{currentstroke}%
\pgfsetdash{}{0pt}%
\pgfsys@defobject{currentmarker}{\pgfqpoint{0.000000in}{-0.048611in}}{\pgfqpoint{0.000000in}{0.000000in}}{%
\pgfpathmoveto{\pgfqpoint{0.000000in}{0.000000in}}%
\pgfpathlineto{\pgfqpoint{0.000000in}{-0.048611in}}%
\pgfusepath{stroke,fill}%
}%
\begin{pgfscope}%
\pgfsys@transformshift{2.824582in}{0.528000in}%
\pgfsys@useobject{currentmarker}{}%
\end{pgfscope}%
\end{pgfscope}%
\begin{pgfscope}%
\definecolor{textcolor}{rgb}{0.000000,0.000000,0.000000}%
\pgfsetstrokecolor{textcolor}%
\pgfsetfillcolor{textcolor}%
\pgftext[x=2.824582in,y=0.430778in,,top]{\color{textcolor}\ttfamily\fontsize{10.000000}{12.000000}\selectfont 400}%
\end{pgfscope}%
\begin{pgfscope}%
\pgfsetbuttcap%
\pgfsetroundjoin%
\definecolor{currentfill}{rgb}{0.000000,0.000000,0.000000}%
\pgfsetfillcolor{currentfill}%
\pgfsetlinewidth{0.803000pt}%
\definecolor{currentstroke}{rgb}{0.000000,0.000000,0.000000}%
\pgfsetstrokecolor{currentstroke}%
\pgfsetdash{}{0pt}%
\pgfsys@defobject{currentmarker}{\pgfqpoint{0.000000in}{-0.048611in}}{\pgfqpoint{0.000000in}{0.000000in}}{%
\pgfpathmoveto{\pgfqpoint{0.000000in}{0.000000in}}%
\pgfpathlineto{\pgfqpoint{0.000000in}{-0.048611in}}%
\pgfusepath{stroke,fill}%
}%
\begin{pgfscope}%
\pgfsys@transformshift{3.726400in}{0.528000in}%
\pgfsys@useobject{currentmarker}{}%
\end{pgfscope}%
\end{pgfscope}%
\begin{pgfscope}%
\definecolor{textcolor}{rgb}{0.000000,0.000000,0.000000}%
\pgfsetstrokecolor{textcolor}%
\pgfsetfillcolor{textcolor}%
\pgftext[x=3.726400in,y=0.430778in,,top]{\color{textcolor}\ttfamily\fontsize{10.000000}{12.000000}\selectfont 600}%
\end{pgfscope}%
\begin{pgfscope}%
\pgfsetbuttcap%
\pgfsetroundjoin%
\definecolor{currentfill}{rgb}{0.000000,0.000000,0.000000}%
\pgfsetfillcolor{currentfill}%
\pgfsetlinewidth{0.803000pt}%
\definecolor{currentstroke}{rgb}{0.000000,0.000000,0.000000}%
\pgfsetstrokecolor{currentstroke}%
\pgfsetdash{}{0pt}%
\pgfsys@defobject{currentmarker}{\pgfqpoint{0.000000in}{-0.048611in}}{\pgfqpoint{0.000000in}{0.000000in}}{%
\pgfpathmoveto{\pgfqpoint{0.000000in}{0.000000in}}%
\pgfpathlineto{\pgfqpoint{0.000000in}{-0.048611in}}%
\pgfusepath{stroke,fill}%
}%
\begin{pgfscope}%
\pgfsys@transformshift{4.628218in}{0.528000in}%
\pgfsys@useobject{currentmarker}{}%
\end{pgfscope}%
\end{pgfscope}%
\begin{pgfscope}%
\definecolor{textcolor}{rgb}{0.000000,0.000000,0.000000}%
\pgfsetstrokecolor{textcolor}%
\pgfsetfillcolor{textcolor}%
\pgftext[x=4.628218in,y=0.430778in,,top]{\color{textcolor}\ttfamily\fontsize{10.000000}{12.000000}\selectfont 800}%
\end{pgfscope}%
\begin{pgfscope}%
\pgfsetbuttcap%
\pgfsetroundjoin%
\definecolor{currentfill}{rgb}{0.000000,0.000000,0.000000}%
\pgfsetfillcolor{currentfill}%
\pgfsetlinewidth{0.803000pt}%
\definecolor{currentstroke}{rgb}{0.000000,0.000000,0.000000}%
\pgfsetstrokecolor{currentstroke}%
\pgfsetdash{}{0pt}%
\pgfsys@defobject{currentmarker}{\pgfqpoint{0.000000in}{-0.048611in}}{\pgfqpoint{0.000000in}{0.000000in}}{%
\pgfpathmoveto{\pgfqpoint{0.000000in}{0.000000in}}%
\pgfpathlineto{\pgfqpoint{0.000000in}{-0.048611in}}%
\pgfusepath{stroke,fill}%
}%
\begin{pgfscope}%
\pgfsys@transformshift{5.530036in}{0.528000in}%
\pgfsys@useobject{currentmarker}{}%
\end{pgfscope}%
\end{pgfscope}%
\begin{pgfscope}%
\definecolor{textcolor}{rgb}{0.000000,0.000000,0.000000}%
\pgfsetstrokecolor{textcolor}%
\pgfsetfillcolor{textcolor}%
\pgftext[x=5.530036in,y=0.430778in,,top]{\color{textcolor}\ttfamily\fontsize{10.000000}{12.000000}\selectfont 1000}%
\end{pgfscope}%
\begin{pgfscope}%
\definecolor{textcolor}{rgb}{0.000000,0.000000,0.000000}%
\pgfsetstrokecolor{textcolor}%
\pgfsetfillcolor{textcolor}%
\pgftext[x=3.280000in,y=0.240063in,,top]{\color{textcolor}\ttfamily\fontsize{10.000000}{12.000000}\selectfont Size of Array}%
\end{pgfscope}%
\begin{pgfscope}%
\pgfsetbuttcap%
\pgfsetroundjoin%
\definecolor{currentfill}{rgb}{0.000000,0.000000,0.000000}%
\pgfsetfillcolor{currentfill}%
\pgfsetlinewidth{0.803000pt}%
\definecolor{currentstroke}{rgb}{0.000000,0.000000,0.000000}%
\pgfsetstrokecolor{currentstroke}%
\pgfsetdash{}{0pt}%
\pgfsys@defobject{currentmarker}{\pgfqpoint{-0.048611in}{0.000000in}}{\pgfqpoint{-0.000000in}{0.000000in}}{%
\pgfpathmoveto{\pgfqpoint{-0.000000in}{0.000000in}}%
\pgfpathlineto{\pgfqpoint{-0.048611in}{0.000000in}}%
\pgfusepath{stroke,fill}%
}%
\begin{pgfscope}%
\pgfsys@transformshift{0.800000in}{0.681638in}%
\pgfsys@useobject{currentmarker}{}%
\end{pgfscope}%
\end{pgfscope}%
\begin{pgfscope}%
\definecolor{textcolor}{rgb}{0.000000,0.000000,0.000000}%
\pgfsetstrokecolor{textcolor}%
\pgfsetfillcolor{textcolor}%
\pgftext[x=0.451923in, y=0.628503in, left, base]{\color{textcolor}\ttfamily\fontsize{10.000000}{12.000000}\selectfont 0.0}%
\end{pgfscope}%
\begin{pgfscope}%
\pgfsetbuttcap%
\pgfsetroundjoin%
\definecolor{currentfill}{rgb}{0.000000,0.000000,0.000000}%
\pgfsetfillcolor{currentfill}%
\pgfsetlinewidth{0.803000pt}%
\definecolor{currentstroke}{rgb}{0.000000,0.000000,0.000000}%
\pgfsetstrokecolor{currentstroke}%
\pgfsetdash{}{0pt}%
\pgfsys@defobject{currentmarker}{\pgfqpoint{-0.048611in}{0.000000in}}{\pgfqpoint{-0.000000in}{0.000000in}}{%
\pgfpathmoveto{\pgfqpoint{-0.000000in}{0.000000in}}%
\pgfpathlineto{\pgfqpoint{-0.048611in}{0.000000in}}%
\pgfusepath{stroke,fill}%
}%
\begin{pgfscope}%
\pgfsys@transformshift{0.800000in}{1.355753in}%
\pgfsys@useobject{currentmarker}{}%
\end{pgfscope}%
\end{pgfscope}%
\begin{pgfscope}%
\definecolor{textcolor}{rgb}{0.000000,0.000000,0.000000}%
\pgfsetstrokecolor{textcolor}%
\pgfsetfillcolor{textcolor}%
\pgftext[x=0.451923in, y=1.302619in, left, base]{\color{textcolor}\ttfamily\fontsize{10.000000}{12.000000}\selectfont 0.2}%
\end{pgfscope}%
\begin{pgfscope}%
\pgfsetbuttcap%
\pgfsetroundjoin%
\definecolor{currentfill}{rgb}{0.000000,0.000000,0.000000}%
\pgfsetfillcolor{currentfill}%
\pgfsetlinewidth{0.803000pt}%
\definecolor{currentstroke}{rgb}{0.000000,0.000000,0.000000}%
\pgfsetstrokecolor{currentstroke}%
\pgfsetdash{}{0pt}%
\pgfsys@defobject{currentmarker}{\pgfqpoint{-0.048611in}{0.000000in}}{\pgfqpoint{-0.000000in}{0.000000in}}{%
\pgfpathmoveto{\pgfqpoint{-0.000000in}{0.000000in}}%
\pgfpathlineto{\pgfqpoint{-0.048611in}{0.000000in}}%
\pgfusepath{stroke,fill}%
}%
\begin{pgfscope}%
\pgfsys@transformshift{0.800000in}{2.029869in}%
\pgfsys@useobject{currentmarker}{}%
\end{pgfscope}%
\end{pgfscope}%
\begin{pgfscope}%
\definecolor{textcolor}{rgb}{0.000000,0.000000,0.000000}%
\pgfsetstrokecolor{textcolor}%
\pgfsetfillcolor{textcolor}%
\pgftext[x=0.451923in, y=1.976734in, left, base]{\color{textcolor}\ttfamily\fontsize{10.000000}{12.000000}\selectfont 0.4}%
\end{pgfscope}%
\begin{pgfscope}%
\pgfsetbuttcap%
\pgfsetroundjoin%
\definecolor{currentfill}{rgb}{0.000000,0.000000,0.000000}%
\pgfsetfillcolor{currentfill}%
\pgfsetlinewidth{0.803000pt}%
\definecolor{currentstroke}{rgb}{0.000000,0.000000,0.000000}%
\pgfsetstrokecolor{currentstroke}%
\pgfsetdash{}{0pt}%
\pgfsys@defobject{currentmarker}{\pgfqpoint{-0.048611in}{0.000000in}}{\pgfqpoint{-0.000000in}{0.000000in}}{%
\pgfpathmoveto{\pgfqpoint{-0.000000in}{0.000000in}}%
\pgfpathlineto{\pgfqpoint{-0.048611in}{0.000000in}}%
\pgfusepath{stroke,fill}%
}%
\begin{pgfscope}%
\pgfsys@transformshift{0.800000in}{2.703984in}%
\pgfsys@useobject{currentmarker}{}%
\end{pgfscope}%
\end{pgfscope}%
\begin{pgfscope}%
\definecolor{textcolor}{rgb}{0.000000,0.000000,0.000000}%
\pgfsetstrokecolor{textcolor}%
\pgfsetfillcolor{textcolor}%
\pgftext[x=0.451923in, y=2.650850in, left, base]{\color{textcolor}\ttfamily\fontsize{10.000000}{12.000000}\selectfont 0.6}%
\end{pgfscope}%
\begin{pgfscope}%
\pgfsetbuttcap%
\pgfsetroundjoin%
\definecolor{currentfill}{rgb}{0.000000,0.000000,0.000000}%
\pgfsetfillcolor{currentfill}%
\pgfsetlinewidth{0.803000pt}%
\definecolor{currentstroke}{rgb}{0.000000,0.000000,0.000000}%
\pgfsetstrokecolor{currentstroke}%
\pgfsetdash{}{0pt}%
\pgfsys@defobject{currentmarker}{\pgfqpoint{-0.048611in}{0.000000in}}{\pgfqpoint{-0.000000in}{0.000000in}}{%
\pgfpathmoveto{\pgfqpoint{-0.000000in}{0.000000in}}%
\pgfpathlineto{\pgfqpoint{-0.048611in}{0.000000in}}%
\pgfusepath{stroke,fill}%
}%
\begin{pgfscope}%
\pgfsys@transformshift{0.800000in}{3.378099in}%
\pgfsys@useobject{currentmarker}{}%
\end{pgfscope}%
\end{pgfscope}%
\begin{pgfscope}%
\definecolor{textcolor}{rgb}{0.000000,0.000000,0.000000}%
\pgfsetstrokecolor{textcolor}%
\pgfsetfillcolor{textcolor}%
\pgftext[x=0.451923in, y=3.324965in, left, base]{\color{textcolor}\ttfamily\fontsize{10.000000}{12.000000}\selectfont 0.8}%
\end{pgfscope}%
\begin{pgfscope}%
\pgfsetbuttcap%
\pgfsetroundjoin%
\definecolor{currentfill}{rgb}{0.000000,0.000000,0.000000}%
\pgfsetfillcolor{currentfill}%
\pgfsetlinewidth{0.803000pt}%
\definecolor{currentstroke}{rgb}{0.000000,0.000000,0.000000}%
\pgfsetstrokecolor{currentstroke}%
\pgfsetdash{}{0pt}%
\pgfsys@defobject{currentmarker}{\pgfqpoint{-0.048611in}{0.000000in}}{\pgfqpoint{-0.000000in}{0.000000in}}{%
\pgfpathmoveto{\pgfqpoint{-0.000000in}{0.000000in}}%
\pgfpathlineto{\pgfqpoint{-0.048611in}{0.000000in}}%
\pgfusepath{stroke,fill}%
}%
\begin{pgfscope}%
\pgfsys@transformshift{0.800000in}{4.052215in}%
\pgfsys@useobject{currentmarker}{}%
\end{pgfscope}%
\end{pgfscope}%
\begin{pgfscope}%
\definecolor{textcolor}{rgb}{0.000000,0.000000,0.000000}%
\pgfsetstrokecolor{textcolor}%
\pgfsetfillcolor{textcolor}%
\pgftext[x=0.451923in, y=3.999080in, left, base]{\color{textcolor}\ttfamily\fontsize{10.000000}{12.000000}\selectfont 1.0}%
\end{pgfscope}%
\begin{pgfscope}%
\definecolor{textcolor}{rgb}{0.000000,0.000000,0.000000}%
\pgfsetstrokecolor{textcolor}%
\pgfsetfillcolor{textcolor}%
\pgftext[x=0.396368in,y=2.376000in,,bottom,rotate=90.000000]{\color{textcolor}\ttfamily\fontsize{10.000000}{12.000000}\selectfont Time}%
\end{pgfscope}%
\begin{pgfscope}%
\definecolor{textcolor}{rgb}{0.000000,0.000000,0.000000}%
\pgfsetstrokecolor{textcolor}%
\pgfsetfillcolor{textcolor}%
\pgftext[x=0.800000in,y=4.265667in,left,base]{\color{textcolor}\ttfamily\fontsize{10.000000}{12.000000}\selectfont 1e9}%
\end{pgfscope}%
\begin{pgfscope}%
\pgfpathrectangle{\pgfqpoint{0.800000in}{0.528000in}}{\pgfqpoint{4.960000in}{3.696000in}}%
\pgfusepath{clip}%
\pgfsetrectcap%
\pgfsetroundjoin%
\pgfsetlinewidth{1.505625pt}%
\definecolor{currentstroke}{rgb}{0.000000,1.000000,0.498039}%
\pgfsetstrokecolor{currentstroke}%
\pgfsetdash{}{0pt}%
\pgfpathmoveto{\pgfqpoint{1.025455in}{0.696000in}}%
\pgfpathlineto{\pgfqpoint{1.043491in}{0.696898in}}%
\pgfpathlineto{\pgfqpoint{1.052509in}{0.701884in}}%
\pgfpathlineto{\pgfqpoint{1.057018in}{0.700149in}}%
\pgfpathlineto{\pgfqpoint{1.061527in}{0.699838in}}%
\pgfpathlineto{\pgfqpoint{1.066036in}{0.702509in}}%
\pgfpathlineto{\pgfqpoint{1.075055in}{0.699555in}}%
\pgfpathlineto{\pgfqpoint{1.079564in}{0.702772in}}%
\pgfpathlineto{\pgfqpoint{1.084073in}{0.704066in}}%
\pgfpathlineto{\pgfqpoint{1.088582in}{0.703963in}}%
\pgfpathlineto{\pgfqpoint{1.093091in}{0.700661in}}%
\pgfpathlineto{\pgfqpoint{1.102109in}{0.706978in}}%
\pgfpathlineto{\pgfqpoint{1.106618in}{0.706091in}}%
\pgfpathlineto{\pgfqpoint{1.111127in}{0.701581in}}%
\pgfpathlineto{\pgfqpoint{1.120145in}{0.706833in}}%
\pgfpathlineto{\pgfqpoint{1.124655in}{0.704887in}}%
\pgfpathlineto{\pgfqpoint{1.129164in}{0.707088in}}%
\pgfpathlineto{\pgfqpoint{1.133673in}{0.704299in}}%
\pgfpathlineto{\pgfqpoint{1.151709in}{0.706232in}}%
\pgfpathlineto{\pgfqpoint{1.156218in}{0.708529in}}%
\pgfpathlineto{\pgfqpoint{1.160727in}{0.705986in}}%
\pgfpathlineto{\pgfqpoint{1.165236in}{0.709093in}}%
\pgfpathlineto{\pgfqpoint{1.169745in}{0.706716in}}%
\pgfpathlineto{\pgfqpoint{1.174255in}{0.710379in}}%
\pgfpathlineto{\pgfqpoint{1.178764in}{0.712343in}}%
\pgfpathlineto{\pgfqpoint{1.187782in}{0.709950in}}%
\pgfpathlineto{\pgfqpoint{1.192291in}{0.713385in}}%
\pgfpathlineto{\pgfqpoint{1.196800in}{0.713906in}}%
\pgfpathlineto{\pgfqpoint{1.201309in}{0.717257in}}%
\pgfpathlineto{\pgfqpoint{1.205818in}{0.717629in}}%
\pgfpathlineto{\pgfqpoint{1.219345in}{0.710821in}}%
\pgfpathlineto{\pgfqpoint{1.223855in}{0.740447in}}%
\pgfpathlineto{\pgfqpoint{1.228364in}{0.712856in}}%
\pgfpathlineto{\pgfqpoint{1.232873in}{0.714475in}}%
\pgfpathlineto{\pgfqpoint{1.237382in}{0.718430in}}%
\pgfpathlineto{\pgfqpoint{1.241891in}{0.759569in}}%
\pgfpathlineto{\pgfqpoint{1.246400in}{0.716938in}}%
\pgfpathlineto{\pgfqpoint{1.250909in}{0.715265in}}%
\pgfpathlineto{\pgfqpoint{1.259927in}{0.721535in}}%
\pgfpathlineto{\pgfqpoint{1.264436in}{0.717531in}}%
\pgfpathlineto{\pgfqpoint{1.268945in}{0.718014in}}%
\pgfpathlineto{\pgfqpoint{1.277964in}{0.722116in}}%
\pgfpathlineto{\pgfqpoint{1.282473in}{0.718955in}}%
\pgfpathlineto{\pgfqpoint{1.286982in}{0.721544in}}%
\pgfpathlineto{\pgfqpoint{1.291491in}{0.719766in}}%
\pgfpathlineto{\pgfqpoint{1.296000in}{0.722148in}}%
\pgfpathlineto{\pgfqpoint{1.300509in}{0.718305in}}%
\pgfpathlineto{\pgfqpoint{1.305018in}{0.720864in}}%
\pgfpathlineto{\pgfqpoint{1.318545in}{0.721051in}}%
\pgfpathlineto{\pgfqpoint{1.332073in}{0.722615in}}%
\pgfpathlineto{\pgfqpoint{1.336582in}{0.726899in}}%
\pgfpathlineto{\pgfqpoint{1.341091in}{0.757931in}}%
\pgfpathlineto{\pgfqpoint{1.350109in}{0.771519in}}%
\pgfpathlineto{\pgfqpoint{1.354618in}{0.785927in}}%
\pgfpathlineto{\pgfqpoint{1.359127in}{0.780351in}}%
\pgfpathlineto{\pgfqpoint{1.363636in}{0.793269in}}%
\pgfpathlineto{\pgfqpoint{1.368145in}{0.800660in}}%
\pgfpathlineto{\pgfqpoint{1.372655in}{0.755042in}}%
\pgfpathlineto{\pgfqpoint{1.377164in}{0.754288in}}%
\pgfpathlineto{\pgfqpoint{1.381673in}{0.751317in}}%
\pgfpathlineto{\pgfqpoint{1.386182in}{0.754522in}}%
\pgfpathlineto{\pgfqpoint{1.395200in}{0.802185in}}%
\pgfpathlineto{\pgfqpoint{1.399709in}{0.789945in}}%
\pgfpathlineto{\pgfqpoint{1.404218in}{0.733158in}}%
\pgfpathlineto{\pgfqpoint{1.408727in}{0.732624in}}%
\pgfpathlineto{\pgfqpoint{1.413236in}{0.744490in}}%
\pgfpathlineto{\pgfqpoint{1.417745in}{0.732598in}}%
\pgfpathlineto{\pgfqpoint{1.444800in}{0.735395in}}%
\pgfpathlineto{\pgfqpoint{1.462836in}{0.737441in}}%
\pgfpathlineto{\pgfqpoint{1.485382in}{0.743019in}}%
\pgfpathlineto{\pgfqpoint{1.489891in}{0.741970in}}%
\pgfpathlineto{\pgfqpoint{1.494400in}{0.745421in}}%
\pgfpathlineto{\pgfqpoint{1.498909in}{0.743264in}}%
\pgfpathlineto{\pgfqpoint{1.503418in}{0.751604in}}%
\pgfpathlineto{\pgfqpoint{1.507927in}{0.836791in}}%
\pgfpathlineto{\pgfqpoint{1.512436in}{0.828021in}}%
\pgfpathlineto{\pgfqpoint{1.516945in}{0.744512in}}%
\pgfpathlineto{\pgfqpoint{1.525964in}{0.746743in}}%
\pgfpathlineto{\pgfqpoint{1.534982in}{0.746334in}}%
\pgfpathlineto{\pgfqpoint{1.539491in}{0.755248in}}%
\pgfpathlineto{\pgfqpoint{1.544000in}{0.754166in}}%
\pgfpathlineto{\pgfqpoint{1.548509in}{0.757012in}}%
\pgfpathlineto{\pgfqpoint{1.553018in}{0.757136in}}%
\pgfpathlineto{\pgfqpoint{1.557527in}{0.759499in}}%
\pgfpathlineto{\pgfqpoint{1.562036in}{0.750822in}}%
\pgfpathlineto{\pgfqpoint{1.566545in}{0.756340in}}%
\pgfpathlineto{\pgfqpoint{1.571055in}{0.766184in}}%
\pgfpathlineto{\pgfqpoint{1.575564in}{0.759079in}}%
\pgfpathlineto{\pgfqpoint{1.580073in}{0.761168in}}%
\pgfpathlineto{\pgfqpoint{1.584582in}{0.769778in}}%
\pgfpathlineto{\pgfqpoint{1.589091in}{0.800976in}}%
\pgfpathlineto{\pgfqpoint{1.593600in}{0.796748in}}%
\pgfpathlineto{\pgfqpoint{1.602618in}{0.779940in}}%
\pgfpathlineto{\pgfqpoint{1.607127in}{0.798137in}}%
\pgfpathlineto{\pgfqpoint{1.611636in}{0.828918in}}%
\pgfpathlineto{\pgfqpoint{1.616145in}{0.835081in}}%
\pgfpathlineto{\pgfqpoint{1.620655in}{0.853650in}}%
\pgfpathlineto{\pgfqpoint{1.629673in}{0.781909in}}%
\pgfpathlineto{\pgfqpoint{1.634182in}{0.776885in}}%
\pgfpathlineto{\pgfqpoint{1.638691in}{0.761230in}}%
\pgfpathlineto{\pgfqpoint{1.643200in}{0.765871in}}%
\pgfpathlineto{\pgfqpoint{1.647709in}{0.762001in}}%
\pgfpathlineto{\pgfqpoint{1.652218in}{0.766017in}}%
\pgfpathlineto{\pgfqpoint{1.656727in}{0.766456in}}%
\pgfpathlineto{\pgfqpoint{1.661236in}{0.769168in}}%
\pgfpathlineto{\pgfqpoint{1.665745in}{0.769728in}}%
\pgfpathlineto{\pgfqpoint{1.670255in}{0.832604in}}%
\pgfpathlineto{\pgfqpoint{1.674764in}{0.810059in}}%
\pgfpathlineto{\pgfqpoint{1.679273in}{0.862794in}}%
\pgfpathlineto{\pgfqpoint{1.683782in}{0.768561in}}%
\pgfpathlineto{\pgfqpoint{1.688291in}{0.780491in}}%
\pgfpathlineto{\pgfqpoint{1.692800in}{0.780763in}}%
\pgfpathlineto{\pgfqpoint{1.697309in}{0.783689in}}%
\pgfpathlineto{\pgfqpoint{1.701818in}{0.803574in}}%
\pgfpathlineto{\pgfqpoint{1.706327in}{0.805754in}}%
\pgfpathlineto{\pgfqpoint{1.710836in}{0.801302in}}%
\pgfpathlineto{\pgfqpoint{1.715345in}{0.813881in}}%
\pgfpathlineto{\pgfqpoint{1.719855in}{0.795488in}}%
\pgfpathlineto{\pgfqpoint{1.724364in}{0.784630in}}%
\pgfpathlineto{\pgfqpoint{1.728873in}{0.783055in}}%
\pgfpathlineto{\pgfqpoint{1.733382in}{0.779140in}}%
\pgfpathlineto{\pgfqpoint{1.737891in}{0.783416in}}%
\pgfpathlineto{\pgfqpoint{1.742400in}{0.782974in}}%
\pgfpathlineto{\pgfqpoint{1.746909in}{0.786177in}}%
\pgfpathlineto{\pgfqpoint{1.751418in}{0.865025in}}%
\pgfpathlineto{\pgfqpoint{1.755927in}{0.847214in}}%
\pgfpathlineto{\pgfqpoint{1.760436in}{0.818296in}}%
\pgfpathlineto{\pgfqpoint{1.764945in}{0.798659in}}%
\pgfpathlineto{\pgfqpoint{1.769455in}{0.795218in}}%
\pgfpathlineto{\pgfqpoint{1.773964in}{0.797221in}}%
\pgfpathlineto{\pgfqpoint{1.778473in}{0.801826in}}%
\pgfpathlineto{\pgfqpoint{1.782982in}{0.916298in}}%
\pgfpathlineto{\pgfqpoint{1.787491in}{0.819952in}}%
\pgfpathlineto{\pgfqpoint{1.792000in}{0.867586in}}%
\pgfpathlineto{\pgfqpoint{1.796509in}{0.856392in}}%
\pgfpathlineto{\pgfqpoint{1.801018in}{0.812481in}}%
\pgfpathlineto{\pgfqpoint{1.805527in}{0.867671in}}%
\pgfpathlineto{\pgfqpoint{1.810036in}{0.805659in}}%
\pgfpathlineto{\pgfqpoint{1.814545in}{0.800335in}}%
\pgfpathlineto{\pgfqpoint{1.819055in}{0.837768in}}%
\pgfpathlineto{\pgfqpoint{1.823564in}{0.815179in}}%
\pgfpathlineto{\pgfqpoint{1.828073in}{0.804828in}}%
\pgfpathlineto{\pgfqpoint{1.832582in}{0.802270in}}%
\pgfpathlineto{\pgfqpoint{1.837091in}{0.817640in}}%
\pgfpathlineto{\pgfqpoint{1.841600in}{0.808969in}}%
\pgfpathlineto{\pgfqpoint{1.846109in}{0.803984in}}%
\pgfpathlineto{\pgfqpoint{1.850618in}{0.868988in}}%
\pgfpathlineto{\pgfqpoint{1.855127in}{0.877705in}}%
\pgfpathlineto{\pgfqpoint{1.859636in}{0.828951in}}%
\pgfpathlineto{\pgfqpoint{1.864145in}{0.801940in}}%
\pgfpathlineto{\pgfqpoint{1.877673in}{0.808087in}}%
\pgfpathlineto{\pgfqpoint{1.882182in}{0.887683in}}%
\pgfpathlineto{\pgfqpoint{1.886691in}{0.813353in}}%
\pgfpathlineto{\pgfqpoint{1.891200in}{0.808374in}}%
\pgfpathlineto{\pgfqpoint{1.895709in}{0.811580in}}%
\pgfpathlineto{\pgfqpoint{1.900218in}{0.849111in}}%
\pgfpathlineto{\pgfqpoint{1.904727in}{0.918212in}}%
\pgfpathlineto{\pgfqpoint{1.909236in}{0.828820in}}%
\pgfpathlineto{\pgfqpoint{1.913745in}{0.819033in}}%
\pgfpathlineto{\pgfqpoint{1.918255in}{0.813816in}}%
\pgfpathlineto{\pgfqpoint{1.922764in}{0.817065in}}%
\pgfpathlineto{\pgfqpoint{1.927273in}{0.817369in}}%
\pgfpathlineto{\pgfqpoint{1.931782in}{0.819648in}}%
\pgfpathlineto{\pgfqpoint{1.940800in}{0.820220in}}%
\pgfpathlineto{\pgfqpoint{1.945309in}{0.823123in}}%
\pgfpathlineto{\pgfqpoint{1.949818in}{0.829282in}}%
\pgfpathlineto{\pgfqpoint{1.954327in}{0.878960in}}%
\pgfpathlineto{\pgfqpoint{1.958836in}{0.836443in}}%
\pgfpathlineto{\pgfqpoint{1.963345in}{0.840845in}}%
\pgfpathlineto{\pgfqpoint{1.967855in}{0.826589in}}%
\pgfpathlineto{\pgfqpoint{1.972364in}{0.842245in}}%
\pgfpathlineto{\pgfqpoint{1.976873in}{0.882890in}}%
\pgfpathlineto{\pgfqpoint{1.981382in}{0.831349in}}%
\pgfpathlineto{\pgfqpoint{1.985891in}{0.861487in}}%
\pgfpathlineto{\pgfqpoint{1.990400in}{0.876021in}}%
\pgfpathlineto{\pgfqpoint{1.994909in}{0.835321in}}%
\pgfpathlineto{\pgfqpoint{1.999418in}{0.833227in}}%
\pgfpathlineto{\pgfqpoint{2.003927in}{0.837496in}}%
\pgfpathlineto{\pgfqpoint{2.008436in}{0.862449in}}%
\pgfpathlineto{\pgfqpoint{2.012945in}{0.843040in}}%
\pgfpathlineto{\pgfqpoint{2.017455in}{0.853831in}}%
\pgfpathlineto{\pgfqpoint{2.021964in}{0.850063in}}%
\pgfpathlineto{\pgfqpoint{2.030982in}{0.869116in}}%
\pgfpathlineto{\pgfqpoint{2.035491in}{0.852141in}}%
\pgfpathlineto{\pgfqpoint{2.044509in}{0.878882in}}%
\pgfpathlineto{\pgfqpoint{2.049018in}{0.888850in}}%
\pgfpathlineto{\pgfqpoint{2.053527in}{0.961758in}}%
\pgfpathlineto{\pgfqpoint{2.058036in}{0.867284in}}%
\pgfpathlineto{\pgfqpoint{2.062545in}{0.865822in}}%
\pgfpathlineto{\pgfqpoint{2.067055in}{0.866734in}}%
\pgfpathlineto{\pgfqpoint{2.071564in}{0.862768in}}%
\pgfpathlineto{\pgfqpoint{2.076073in}{0.862538in}}%
\pgfpathlineto{\pgfqpoint{2.080582in}{0.871301in}}%
\pgfpathlineto{\pgfqpoint{2.085091in}{0.929115in}}%
\pgfpathlineto{\pgfqpoint{2.089600in}{0.861048in}}%
\pgfpathlineto{\pgfqpoint{2.094109in}{0.905749in}}%
\pgfpathlineto{\pgfqpoint{2.098618in}{0.866103in}}%
\pgfpathlineto{\pgfqpoint{2.103127in}{0.897385in}}%
\pgfpathlineto{\pgfqpoint{2.107636in}{1.028388in}}%
\pgfpathlineto{\pgfqpoint{2.112145in}{0.880894in}}%
\pgfpathlineto{\pgfqpoint{2.116655in}{0.919153in}}%
\pgfpathlineto{\pgfqpoint{2.121164in}{0.984006in}}%
\pgfpathlineto{\pgfqpoint{2.125673in}{0.922108in}}%
\pgfpathlineto{\pgfqpoint{2.130182in}{0.995104in}}%
\pgfpathlineto{\pgfqpoint{2.134691in}{1.169749in}}%
\pgfpathlineto{\pgfqpoint{2.139200in}{1.070746in}}%
\pgfpathlineto{\pgfqpoint{2.143709in}{0.930990in}}%
\pgfpathlineto{\pgfqpoint{2.148218in}{0.907352in}}%
\pgfpathlineto{\pgfqpoint{2.152727in}{0.873114in}}%
\pgfpathlineto{\pgfqpoint{2.157236in}{0.901512in}}%
\pgfpathlineto{\pgfqpoint{2.161745in}{0.915756in}}%
\pgfpathlineto{\pgfqpoint{2.166255in}{1.012170in}}%
\pgfpathlineto{\pgfqpoint{2.170764in}{0.938369in}}%
\pgfpathlineto{\pgfqpoint{2.175273in}{1.091710in}}%
\pgfpathlineto{\pgfqpoint{2.184291in}{0.918887in}}%
\pgfpathlineto{\pgfqpoint{2.188800in}{1.098095in}}%
\pgfpathlineto{\pgfqpoint{2.193309in}{1.059099in}}%
\pgfpathlineto{\pgfqpoint{2.197818in}{1.171782in}}%
\pgfpathlineto{\pgfqpoint{2.206836in}{0.933659in}}%
\pgfpathlineto{\pgfqpoint{2.211345in}{1.051931in}}%
\pgfpathlineto{\pgfqpoint{2.215855in}{0.924900in}}%
\pgfpathlineto{\pgfqpoint{2.220364in}{0.950241in}}%
\pgfpathlineto{\pgfqpoint{2.224873in}{0.954666in}}%
\pgfpathlineto{\pgfqpoint{2.229382in}{0.974651in}}%
\pgfpathlineto{\pgfqpoint{2.233891in}{0.933315in}}%
\pgfpathlineto{\pgfqpoint{2.238400in}{0.932227in}}%
\pgfpathlineto{\pgfqpoint{2.242909in}{0.937308in}}%
\pgfpathlineto{\pgfqpoint{2.247418in}{0.929581in}}%
\pgfpathlineto{\pgfqpoint{2.251927in}{0.927040in}}%
\pgfpathlineto{\pgfqpoint{2.260945in}{0.929404in}}%
\pgfpathlineto{\pgfqpoint{2.265455in}{0.945642in}}%
\pgfpathlineto{\pgfqpoint{2.269964in}{0.973779in}}%
\pgfpathlineto{\pgfqpoint{2.274473in}{0.943097in}}%
\pgfpathlineto{\pgfqpoint{2.278982in}{0.983080in}}%
\pgfpathlineto{\pgfqpoint{2.283491in}{0.932071in}}%
\pgfpathlineto{\pgfqpoint{2.288000in}{0.932038in}}%
\pgfpathlineto{\pgfqpoint{2.292509in}{0.946136in}}%
\pgfpathlineto{\pgfqpoint{2.297018in}{0.955212in}}%
\pgfpathlineto{\pgfqpoint{2.301527in}{0.938781in}}%
\pgfpathlineto{\pgfqpoint{2.306036in}{0.953306in}}%
\pgfpathlineto{\pgfqpoint{2.310545in}{0.954620in}}%
\pgfpathlineto{\pgfqpoint{2.315055in}{0.941953in}}%
\pgfpathlineto{\pgfqpoint{2.319564in}{1.070083in}}%
\pgfpathlineto{\pgfqpoint{2.324073in}{0.954852in}}%
\pgfpathlineto{\pgfqpoint{2.328582in}{0.960447in}}%
\pgfpathlineto{\pgfqpoint{2.333091in}{0.961225in}}%
\pgfpathlineto{\pgfqpoint{2.337600in}{0.950332in}}%
\pgfpathlineto{\pgfqpoint{2.342109in}{0.949321in}}%
\pgfpathlineto{\pgfqpoint{2.346618in}{0.954574in}}%
\pgfpathlineto{\pgfqpoint{2.351127in}{0.965737in}}%
\pgfpathlineto{\pgfqpoint{2.355636in}{0.962194in}}%
\pgfpathlineto{\pgfqpoint{2.360145in}{0.973683in}}%
\pgfpathlineto{\pgfqpoint{2.364655in}{0.966743in}}%
\pgfpathlineto{\pgfqpoint{2.369164in}{0.980725in}}%
\pgfpathlineto{\pgfqpoint{2.373673in}{0.970312in}}%
\pgfpathlineto{\pgfqpoint{2.382691in}{0.978184in}}%
\pgfpathlineto{\pgfqpoint{2.387200in}{0.972166in}}%
\pgfpathlineto{\pgfqpoint{2.391709in}{0.979663in}}%
\pgfpathlineto{\pgfqpoint{2.396218in}{1.006432in}}%
\pgfpathlineto{\pgfqpoint{2.400727in}{1.022329in}}%
\pgfpathlineto{\pgfqpoint{2.405236in}{0.985888in}}%
\pgfpathlineto{\pgfqpoint{2.409745in}{0.983155in}}%
\pgfpathlineto{\pgfqpoint{2.414255in}{0.987320in}}%
\pgfpathlineto{\pgfqpoint{2.418764in}{0.977746in}}%
\pgfpathlineto{\pgfqpoint{2.423273in}{0.987330in}}%
\pgfpathlineto{\pgfqpoint{2.427782in}{1.014571in}}%
\pgfpathlineto{\pgfqpoint{2.432291in}{0.972428in}}%
\pgfpathlineto{\pgfqpoint{2.436800in}{0.993636in}}%
\pgfpathlineto{\pgfqpoint{2.441309in}{0.998205in}}%
\pgfpathlineto{\pgfqpoint{2.450327in}{1.013658in}}%
\pgfpathlineto{\pgfqpoint{2.454836in}{1.010343in}}%
\pgfpathlineto{\pgfqpoint{2.459345in}{1.033809in}}%
\pgfpathlineto{\pgfqpoint{2.463855in}{1.003948in}}%
\pgfpathlineto{\pgfqpoint{2.468364in}{1.011492in}}%
\pgfpathlineto{\pgfqpoint{2.472873in}{0.993445in}}%
\pgfpathlineto{\pgfqpoint{2.477382in}{1.043703in}}%
\pgfpathlineto{\pgfqpoint{2.481891in}{1.055869in}}%
\pgfpathlineto{\pgfqpoint{2.486400in}{1.020223in}}%
\pgfpathlineto{\pgfqpoint{2.490909in}{1.020641in}}%
\pgfpathlineto{\pgfqpoint{2.499927in}{1.028321in}}%
\pgfpathlineto{\pgfqpoint{2.504436in}{1.014875in}}%
\pgfpathlineto{\pgfqpoint{2.508945in}{1.022715in}}%
\pgfpathlineto{\pgfqpoint{2.513455in}{1.009562in}}%
\pgfpathlineto{\pgfqpoint{2.517964in}{1.024920in}}%
\pgfpathlineto{\pgfqpoint{2.522473in}{1.019445in}}%
\pgfpathlineto{\pgfqpoint{2.526982in}{1.057069in}}%
\pgfpathlineto{\pgfqpoint{2.531491in}{1.029275in}}%
\pgfpathlineto{\pgfqpoint{2.536000in}{1.030035in}}%
\pgfpathlineto{\pgfqpoint{2.540509in}{1.022689in}}%
\pgfpathlineto{\pgfqpoint{2.545018in}{1.031190in}}%
\pgfpathlineto{\pgfqpoint{2.549527in}{1.030567in}}%
\pgfpathlineto{\pgfqpoint{2.554036in}{1.034983in}}%
\pgfpathlineto{\pgfqpoint{2.558545in}{1.034084in}}%
\pgfpathlineto{\pgfqpoint{2.563055in}{1.062805in}}%
\pgfpathlineto{\pgfqpoint{2.567564in}{1.035474in}}%
\pgfpathlineto{\pgfqpoint{2.572073in}{1.045140in}}%
\pgfpathlineto{\pgfqpoint{2.576582in}{1.047160in}}%
\pgfpathlineto{\pgfqpoint{2.581091in}{1.040848in}}%
\pgfpathlineto{\pgfqpoint{2.585600in}{1.040313in}}%
\pgfpathlineto{\pgfqpoint{2.590109in}{1.078307in}}%
\pgfpathlineto{\pgfqpoint{2.594618in}{1.051673in}}%
\pgfpathlineto{\pgfqpoint{2.599127in}{1.045333in}}%
\pgfpathlineto{\pgfqpoint{2.603636in}{1.054659in}}%
\pgfpathlineto{\pgfqpoint{2.608145in}{1.056375in}}%
\pgfpathlineto{\pgfqpoint{2.612655in}{1.054385in}}%
\pgfpathlineto{\pgfqpoint{2.617164in}{1.049794in}}%
\pgfpathlineto{\pgfqpoint{2.621673in}{1.054277in}}%
\pgfpathlineto{\pgfqpoint{2.626182in}{1.063468in}}%
\pgfpathlineto{\pgfqpoint{2.630691in}{1.066248in}}%
\pgfpathlineto{\pgfqpoint{2.639709in}{1.061451in}}%
\pgfpathlineto{\pgfqpoint{2.644218in}{1.065905in}}%
\pgfpathlineto{\pgfqpoint{2.648727in}{1.072874in}}%
\pgfpathlineto{\pgfqpoint{2.653236in}{1.060740in}}%
\pgfpathlineto{\pgfqpoint{2.657745in}{1.076803in}}%
\pgfpathlineto{\pgfqpoint{2.662255in}{1.083224in}}%
\pgfpathlineto{\pgfqpoint{2.666764in}{1.072594in}}%
\pgfpathlineto{\pgfqpoint{2.671273in}{1.128471in}}%
\pgfpathlineto{\pgfqpoint{2.675782in}{1.085327in}}%
\pgfpathlineto{\pgfqpoint{2.680291in}{1.095786in}}%
\pgfpathlineto{\pgfqpoint{2.684800in}{1.094886in}}%
\pgfpathlineto{\pgfqpoint{2.689309in}{1.088644in}}%
\pgfpathlineto{\pgfqpoint{2.693818in}{1.099613in}}%
\pgfpathlineto{\pgfqpoint{2.698327in}{1.082883in}}%
\pgfpathlineto{\pgfqpoint{2.702836in}{1.091174in}}%
\pgfpathlineto{\pgfqpoint{2.707345in}{1.104955in}}%
\pgfpathlineto{\pgfqpoint{2.711855in}{1.126766in}}%
\pgfpathlineto{\pgfqpoint{2.716364in}{1.101314in}}%
\pgfpathlineto{\pgfqpoint{2.725382in}{1.104297in}}%
\pgfpathlineto{\pgfqpoint{2.729891in}{1.110531in}}%
\pgfpathlineto{\pgfqpoint{2.734400in}{1.143373in}}%
\pgfpathlineto{\pgfqpoint{2.738909in}{1.105120in}}%
\pgfpathlineto{\pgfqpoint{2.743418in}{1.109775in}}%
\pgfpathlineto{\pgfqpoint{2.752436in}{1.109985in}}%
\pgfpathlineto{\pgfqpoint{2.756945in}{1.119308in}}%
\pgfpathlineto{\pgfqpoint{2.765964in}{1.117891in}}%
\pgfpathlineto{\pgfqpoint{2.770473in}{1.113990in}}%
\pgfpathlineto{\pgfqpoint{2.774982in}{1.124605in}}%
\pgfpathlineto{\pgfqpoint{2.779491in}{1.127811in}}%
\pgfpathlineto{\pgfqpoint{2.784000in}{1.122127in}}%
\pgfpathlineto{\pgfqpoint{2.788509in}{1.120232in}}%
\pgfpathlineto{\pgfqpoint{2.793018in}{1.125856in}}%
\pgfpathlineto{\pgfqpoint{2.797527in}{1.111221in}}%
\pgfpathlineto{\pgfqpoint{2.806545in}{1.132355in}}%
\pgfpathlineto{\pgfqpoint{2.811055in}{1.133496in}}%
\pgfpathlineto{\pgfqpoint{2.815564in}{1.140481in}}%
\pgfpathlineto{\pgfqpoint{2.820073in}{1.155042in}}%
\pgfpathlineto{\pgfqpoint{2.824582in}{1.131208in}}%
\pgfpathlineto{\pgfqpoint{2.829091in}{1.180459in}}%
\pgfpathlineto{\pgfqpoint{2.833600in}{1.161902in}}%
\pgfpathlineto{\pgfqpoint{2.838109in}{1.136587in}}%
\pgfpathlineto{\pgfqpoint{2.842618in}{1.148805in}}%
\pgfpathlineto{\pgfqpoint{2.847127in}{1.153256in}}%
\pgfpathlineto{\pgfqpoint{2.851636in}{1.145972in}}%
\pgfpathlineto{\pgfqpoint{2.856145in}{1.163535in}}%
\pgfpathlineto{\pgfqpoint{2.860655in}{1.164205in}}%
\pgfpathlineto{\pgfqpoint{2.865164in}{1.146992in}}%
\pgfpathlineto{\pgfqpoint{2.869673in}{1.144797in}}%
\pgfpathlineto{\pgfqpoint{2.874182in}{1.179155in}}%
\pgfpathlineto{\pgfqpoint{2.878691in}{1.195754in}}%
\pgfpathlineto{\pgfqpoint{2.883200in}{1.259942in}}%
\pgfpathlineto{\pgfqpoint{2.887709in}{1.205618in}}%
\pgfpathlineto{\pgfqpoint{2.892218in}{1.183553in}}%
\pgfpathlineto{\pgfqpoint{2.896727in}{1.198113in}}%
\pgfpathlineto{\pgfqpoint{2.901236in}{1.186298in}}%
\pgfpathlineto{\pgfqpoint{2.905745in}{1.180942in}}%
\pgfpathlineto{\pgfqpoint{2.910255in}{1.205310in}}%
\pgfpathlineto{\pgfqpoint{2.914764in}{1.210061in}}%
\pgfpathlineto{\pgfqpoint{2.919273in}{1.185064in}}%
\pgfpathlineto{\pgfqpoint{2.923782in}{1.190056in}}%
\pgfpathlineto{\pgfqpoint{2.928291in}{1.192993in}}%
\pgfpathlineto{\pgfqpoint{2.932800in}{1.182524in}}%
\pgfpathlineto{\pgfqpoint{2.937309in}{1.189140in}}%
\pgfpathlineto{\pgfqpoint{2.941818in}{1.213420in}}%
\pgfpathlineto{\pgfqpoint{2.946327in}{1.211465in}}%
\pgfpathlineto{\pgfqpoint{2.950836in}{1.189925in}}%
\pgfpathlineto{\pgfqpoint{2.959855in}{1.194222in}}%
\pgfpathlineto{\pgfqpoint{2.964364in}{1.220124in}}%
\pgfpathlineto{\pgfqpoint{2.968873in}{1.300086in}}%
\pgfpathlineto{\pgfqpoint{2.973382in}{1.198193in}}%
\pgfpathlineto{\pgfqpoint{2.977891in}{1.202461in}}%
\pgfpathlineto{\pgfqpoint{2.982400in}{1.353195in}}%
\pgfpathlineto{\pgfqpoint{2.986909in}{1.207814in}}%
\pgfpathlineto{\pgfqpoint{2.991418in}{1.199708in}}%
\pgfpathlineto{\pgfqpoint{2.995927in}{1.224876in}}%
\pgfpathlineto{\pgfqpoint{3.000436in}{1.198231in}}%
\pgfpathlineto{\pgfqpoint{3.004945in}{1.214042in}}%
\pgfpathlineto{\pgfqpoint{3.009455in}{1.206627in}}%
\pgfpathlineto{\pgfqpoint{3.013964in}{1.244354in}}%
\pgfpathlineto{\pgfqpoint{3.018473in}{1.213997in}}%
\pgfpathlineto{\pgfqpoint{3.022982in}{1.224295in}}%
\pgfpathlineto{\pgfqpoint{3.027491in}{1.229241in}}%
\pgfpathlineto{\pgfqpoint{3.032000in}{1.229829in}}%
\pgfpathlineto{\pgfqpoint{3.036509in}{1.218806in}}%
\pgfpathlineto{\pgfqpoint{3.041018in}{1.241190in}}%
\pgfpathlineto{\pgfqpoint{3.045527in}{1.236007in}}%
\pgfpathlineto{\pgfqpoint{3.054545in}{1.236631in}}%
\pgfpathlineto{\pgfqpoint{3.059055in}{1.257405in}}%
\pgfpathlineto{\pgfqpoint{3.068073in}{1.270612in}}%
\pgfpathlineto{\pgfqpoint{3.072582in}{1.260488in}}%
\pgfpathlineto{\pgfqpoint{3.077091in}{1.257842in}}%
\pgfpathlineto{\pgfqpoint{3.081600in}{1.311208in}}%
\pgfpathlineto{\pgfqpoint{3.086109in}{1.281770in}}%
\pgfpathlineto{\pgfqpoint{3.090618in}{1.243730in}}%
\pgfpathlineto{\pgfqpoint{3.095127in}{1.267631in}}%
\pgfpathlineto{\pgfqpoint{3.099636in}{1.244917in}}%
\pgfpathlineto{\pgfqpoint{3.104145in}{1.280559in}}%
\pgfpathlineto{\pgfqpoint{3.108655in}{1.272076in}}%
\pgfpathlineto{\pgfqpoint{3.113164in}{1.266075in}}%
\pgfpathlineto{\pgfqpoint{3.117673in}{1.278847in}}%
\pgfpathlineto{\pgfqpoint{3.122182in}{1.276689in}}%
\pgfpathlineto{\pgfqpoint{3.126691in}{1.264892in}}%
\pgfpathlineto{\pgfqpoint{3.131200in}{1.299464in}}%
\pgfpathlineto{\pgfqpoint{3.135709in}{1.287387in}}%
\pgfpathlineto{\pgfqpoint{3.144727in}{1.312151in}}%
\pgfpathlineto{\pgfqpoint{3.153745in}{1.327352in}}%
\pgfpathlineto{\pgfqpoint{3.158255in}{1.293026in}}%
\pgfpathlineto{\pgfqpoint{3.162764in}{1.299514in}}%
\pgfpathlineto{\pgfqpoint{3.167273in}{1.293874in}}%
\pgfpathlineto{\pgfqpoint{3.171782in}{1.303856in}}%
\pgfpathlineto{\pgfqpoint{3.176291in}{1.323132in}}%
\pgfpathlineto{\pgfqpoint{3.180800in}{1.294514in}}%
\pgfpathlineto{\pgfqpoint{3.185309in}{1.290307in}}%
\pgfpathlineto{\pgfqpoint{3.189818in}{1.301078in}}%
\pgfpathlineto{\pgfqpoint{3.194327in}{1.321150in}}%
\pgfpathlineto{\pgfqpoint{3.203345in}{1.305542in}}%
\pgfpathlineto{\pgfqpoint{3.207855in}{1.307140in}}%
\pgfpathlineto{\pgfqpoint{3.212364in}{1.328384in}}%
\pgfpathlineto{\pgfqpoint{3.216873in}{1.320808in}}%
\pgfpathlineto{\pgfqpoint{3.221382in}{1.323096in}}%
\pgfpathlineto{\pgfqpoint{3.230400in}{1.360767in}}%
\pgfpathlineto{\pgfqpoint{3.234909in}{1.330309in}}%
\pgfpathlineto{\pgfqpoint{3.239418in}{1.338529in}}%
\pgfpathlineto{\pgfqpoint{3.243927in}{1.352483in}}%
\pgfpathlineto{\pgfqpoint{3.248436in}{1.350587in}}%
\pgfpathlineto{\pgfqpoint{3.252945in}{1.391435in}}%
\pgfpathlineto{\pgfqpoint{3.257455in}{1.328145in}}%
\pgfpathlineto{\pgfqpoint{3.261964in}{1.344265in}}%
\pgfpathlineto{\pgfqpoint{3.266473in}{1.346592in}}%
\pgfpathlineto{\pgfqpoint{3.270982in}{1.366501in}}%
\pgfpathlineto{\pgfqpoint{3.275491in}{1.404222in}}%
\pgfpathlineto{\pgfqpoint{3.280000in}{1.357236in}}%
\pgfpathlineto{\pgfqpoint{3.284509in}{1.351524in}}%
\pgfpathlineto{\pgfqpoint{3.289018in}{1.351315in}}%
\pgfpathlineto{\pgfqpoint{3.293527in}{1.396128in}}%
\pgfpathlineto{\pgfqpoint{3.298036in}{1.367587in}}%
\pgfpathlineto{\pgfqpoint{3.302545in}{1.371151in}}%
\pgfpathlineto{\pgfqpoint{3.307055in}{1.365024in}}%
\pgfpathlineto{\pgfqpoint{3.311564in}{1.427764in}}%
\pgfpathlineto{\pgfqpoint{3.316073in}{1.372176in}}%
\pgfpathlineto{\pgfqpoint{3.320582in}{1.376614in}}%
\pgfpathlineto{\pgfqpoint{3.325091in}{1.487269in}}%
\pgfpathlineto{\pgfqpoint{3.329600in}{1.374436in}}%
\pgfpathlineto{\pgfqpoint{3.334109in}{1.397854in}}%
\pgfpathlineto{\pgfqpoint{3.338618in}{1.446329in}}%
\pgfpathlineto{\pgfqpoint{3.343127in}{1.403806in}}%
\pgfpathlineto{\pgfqpoint{3.347636in}{1.386332in}}%
\pgfpathlineto{\pgfqpoint{3.352145in}{1.390809in}}%
\pgfpathlineto{\pgfqpoint{3.356655in}{1.462638in}}%
\pgfpathlineto{\pgfqpoint{3.361164in}{1.403529in}}%
\pgfpathlineto{\pgfqpoint{3.365673in}{1.390015in}}%
\pgfpathlineto{\pgfqpoint{3.370182in}{1.438590in}}%
\pgfpathlineto{\pgfqpoint{3.374691in}{1.398211in}}%
\pgfpathlineto{\pgfqpoint{3.379200in}{1.397674in}}%
\pgfpathlineto{\pgfqpoint{3.383709in}{1.459304in}}%
\pgfpathlineto{\pgfqpoint{3.392727in}{1.410669in}}%
\pgfpathlineto{\pgfqpoint{3.397236in}{1.414097in}}%
\pgfpathlineto{\pgfqpoint{3.401745in}{1.402960in}}%
\pgfpathlineto{\pgfqpoint{3.406255in}{1.440764in}}%
\pgfpathlineto{\pgfqpoint{3.410764in}{1.452111in}}%
\pgfpathlineto{\pgfqpoint{3.415273in}{1.427533in}}%
\pgfpathlineto{\pgfqpoint{3.419782in}{1.489096in}}%
\pgfpathlineto{\pgfqpoint{3.424291in}{1.425396in}}%
\pgfpathlineto{\pgfqpoint{3.428800in}{1.437448in}}%
\pgfpathlineto{\pgfqpoint{3.433309in}{1.493049in}}%
\pgfpathlineto{\pgfqpoint{3.437818in}{1.439374in}}%
\pgfpathlineto{\pgfqpoint{3.442327in}{1.481916in}}%
\pgfpathlineto{\pgfqpoint{3.446836in}{1.451012in}}%
\pgfpathlineto{\pgfqpoint{3.451345in}{1.450584in}}%
\pgfpathlineto{\pgfqpoint{3.455855in}{1.517594in}}%
\pgfpathlineto{\pgfqpoint{3.460364in}{1.443463in}}%
\pgfpathlineto{\pgfqpoint{3.464873in}{1.469504in}}%
\pgfpathlineto{\pgfqpoint{3.469382in}{1.446201in}}%
\pgfpathlineto{\pgfqpoint{3.478400in}{1.506072in}}%
\pgfpathlineto{\pgfqpoint{3.482909in}{1.467928in}}%
\pgfpathlineto{\pgfqpoint{3.487418in}{1.478202in}}%
\pgfpathlineto{\pgfqpoint{3.491927in}{1.473052in}}%
\pgfpathlineto{\pgfqpoint{3.496436in}{1.505392in}}%
\pgfpathlineto{\pgfqpoint{3.500945in}{1.471664in}}%
\pgfpathlineto{\pgfqpoint{3.505455in}{1.502677in}}%
\pgfpathlineto{\pgfqpoint{3.509964in}{1.480896in}}%
\pgfpathlineto{\pgfqpoint{3.514473in}{1.485183in}}%
\pgfpathlineto{\pgfqpoint{3.518982in}{1.493429in}}%
\pgfpathlineto{\pgfqpoint{3.523491in}{1.536351in}}%
\pgfpathlineto{\pgfqpoint{3.528000in}{1.492483in}}%
\pgfpathlineto{\pgfqpoint{3.532509in}{1.491824in}}%
\pgfpathlineto{\pgfqpoint{3.537018in}{1.510729in}}%
\pgfpathlineto{\pgfqpoint{3.541527in}{1.559033in}}%
\pgfpathlineto{\pgfqpoint{3.546036in}{1.513731in}}%
\pgfpathlineto{\pgfqpoint{3.550545in}{1.508451in}}%
\pgfpathlineto{\pgfqpoint{3.555055in}{1.519653in}}%
\pgfpathlineto{\pgfqpoint{3.559564in}{1.534256in}}%
\pgfpathlineto{\pgfqpoint{3.564073in}{1.577358in}}%
\pgfpathlineto{\pgfqpoint{3.568582in}{1.494237in}}%
\pgfpathlineto{\pgfqpoint{3.573091in}{1.555982in}}%
\pgfpathlineto{\pgfqpoint{3.577600in}{1.523189in}}%
\pgfpathlineto{\pgfqpoint{3.582109in}{1.514213in}}%
\pgfpathlineto{\pgfqpoint{3.586618in}{1.536855in}}%
\pgfpathlineto{\pgfqpoint{3.591127in}{1.579631in}}%
\pgfpathlineto{\pgfqpoint{3.595636in}{1.539653in}}%
\pgfpathlineto{\pgfqpoint{3.600145in}{1.540698in}}%
\pgfpathlineto{\pgfqpoint{3.609164in}{1.558071in}}%
\pgfpathlineto{\pgfqpoint{3.613673in}{1.589176in}}%
\pgfpathlineto{\pgfqpoint{3.618182in}{1.604081in}}%
\pgfpathlineto{\pgfqpoint{3.622691in}{1.552407in}}%
\pgfpathlineto{\pgfqpoint{3.627200in}{1.541394in}}%
\pgfpathlineto{\pgfqpoint{3.631709in}{1.624026in}}%
\pgfpathlineto{\pgfqpoint{3.636218in}{1.540469in}}%
\pgfpathlineto{\pgfqpoint{3.640727in}{1.585771in}}%
\pgfpathlineto{\pgfqpoint{3.645236in}{1.582830in}}%
\pgfpathlineto{\pgfqpoint{3.649745in}{1.568809in}}%
\pgfpathlineto{\pgfqpoint{3.654255in}{1.605789in}}%
\pgfpathlineto{\pgfqpoint{3.658764in}{1.563524in}}%
\pgfpathlineto{\pgfqpoint{3.663273in}{1.585078in}}%
\pgfpathlineto{\pgfqpoint{3.667782in}{1.635660in}}%
\pgfpathlineto{\pgfqpoint{3.672291in}{1.571171in}}%
\pgfpathlineto{\pgfqpoint{3.676800in}{1.605917in}}%
\pgfpathlineto{\pgfqpoint{3.681309in}{1.620011in}}%
\pgfpathlineto{\pgfqpoint{3.685818in}{1.693976in}}%
\pgfpathlineto{\pgfqpoint{3.690327in}{1.606950in}}%
\pgfpathlineto{\pgfqpoint{3.694836in}{1.605832in}}%
\pgfpathlineto{\pgfqpoint{3.699345in}{1.615116in}}%
\pgfpathlineto{\pgfqpoint{3.703855in}{1.602486in}}%
\pgfpathlineto{\pgfqpoint{3.712873in}{1.635244in}}%
\pgfpathlineto{\pgfqpoint{3.717382in}{1.605909in}}%
\pgfpathlineto{\pgfqpoint{3.721891in}{1.617663in}}%
\pgfpathlineto{\pgfqpoint{3.726400in}{1.637408in}}%
\pgfpathlineto{\pgfqpoint{3.730909in}{1.663274in}}%
\pgfpathlineto{\pgfqpoint{3.735418in}{1.615746in}}%
\pgfpathlineto{\pgfqpoint{3.739927in}{1.640637in}}%
\pgfpathlineto{\pgfqpoint{3.744436in}{1.681766in}}%
\pgfpathlineto{\pgfqpoint{3.748945in}{1.704759in}}%
\pgfpathlineto{\pgfqpoint{3.753455in}{1.611741in}}%
\pgfpathlineto{\pgfqpoint{3.757964in}{1.718177in}}%
\pgfpathlineto{\pgfqpoint{3.762473in}{1.629307in}}%
\pgfpathlineto{\pgfqpoint{3.766982in}{1.683866in}}%
\pgfpathlineto{\pgfqpoint{3.771491in}{1.643738in}}%
\pgfpathlineto{\pgfqpoint{3.776000in}{1.620121in}}%
\pgfpathlineto{\pgfqpoint{3.780509in}{1.617430in}}%
\pgfpathlineto{\pgfqpoint{3.785018in}{1.895422in}}%
\pgfpathlineto{\pgfqpoint{3.789527in}{2.269783in}}%
\pgfpathlineto{\pgfqpoint{3.794036in}{1.825936in}}%
\pgfpathlineto{\pgfqpoint{3.798545in}{2.466465in}}%
\pgfpathlineto{\pgfqpoint{3.803055in}{2.566343in}}%
\pgfpathlineto{\pgfqpoint{3.807564in}{2.084894in}}%
\pgfpathlineto{\pgfqpoint{3.812073in}{2.168110in}}%
\pgfpathlineto{\pgfqpoint{3.816582in}{1.722295in}}%
\pgfpathlineto{\pgfqpoint{3.821091in}{1.802090in}}%
\pgfpathlineto{\pgfqpoint{3.825600in}{1.634232in}}%
\pgfpathlineto{\pgfqpoint{3.830109in}{1.844355in}}%
\pgfpathlineto{\pgfqpoint{3.834618in}{1.688997in}}%
\pgfpathlineto{\pgfqpoint{3.839127in}{1.966303in}}%
\pgfpathlineto{\pgfqpoint{3.848145in}{1.641636in}}%
\pgfpathlineto{\pgfqpoint{3.861673in}{1.764106in}}%
\pgfpathlineto{\pgfqpoint{3.866182in}{1.694536in}}%
\pgfpathlineto{\pgfqpoint{3.870691in}{1.718246in}}%
\pgfpathlineto{\pgfqpoint{3.875200in}{1.688845in}}%
\pgfpathlineto{\pgfqpoint{3.879709in}{1.685037in}}%
\pgfpathlineto{\pgfqpoint{3.884218in}{1.748025in}}%
\pgfpathlineto{\pgfqpoint{3.888727in}{1.682995in}}%
\pgfpathlineto{\pgfqpoint{3.893236in}{1.699474in}}%
\pgfpathlineto{\pgfqpoint{3.897745in}{1.724665in}}%
\pgfpathlineto{\pgfqpoint{3.902255in}{1.722423in}}%
\pgfpathlineto{\pgfqpoint{3.906764in}{1.749341in}}%
\pgfpathlineto{\pgfqpoint{3.911273in}{1.742435in}}%
\pgfpathlineto{\pgfqpoint{3.915782in}{1.786771in}}%
\pgfpathlineto{\pgfqpoint{3.920291in}{1.785575in}}%
\pgfpathlineto{\pgfqpoint{3.924800in}{1.757873in}}%
\pgfpathlineto{\pgfqpoint{3.929309in}{1.781803in}}%
\pgfpathlineto{\pgfqpoint{3.933818in}{1.780912in}}%
\pgfpathlineto{\pgfqpoint{3.938327in}{1.744891in}}%
\pgfpathlineto{\pgfqpoint{3.942836in}{1.776240in}}%
\pgfpathlineto{\pgfqpoint{3.947345in}{1.733236in}}%
\pgfpathlineto{\pgfqpoint{3.951855in}{1.713294in}}%
\pgfpathlineto{\pgfqpoint{3.960873in}{1.767821in}}%
\pgfpathlineto{\pgfqpoint{3.965382in}{1.758300in}}%
\pgfpathlineto{\pgfqpoint{3.969891in}{1.895941in}}%
\pgfpathlineto{\pgfqpoint{3.974400in}{1.765206in}}%
\pgfpathlineto{\pgfqpoint{3.978909in}{1.783469in}}%
\pgfpathlineto{\pgfqpoint{3.983418in}{1.806895in}}%
\pgfpathlineto{\pgfqpoint{3.987927in}{1.790757in}}%
\pgfpathlineto{\pgfqpoint{3.992436in}{1.827696in}}%
\pgfpathlineto{\pgfqpoint{3.996945in}{1.790873in}}%
\pgfpathlineto{\pgfqpoint{4.001455in}{1.791047in}}%
\pgfpathlineto{\pgfqpoint{4.005964in}{1.771555in}}%
\pgfpathlineto{\pgfqpoint{4.010473in}{1.828444in}}%
\pgfpathlineto{\pgfqpoint{4.014982in}{1.783631in}}%
\pgfpathlineto{\pgfqpoint{4.019491in}{1.781387in}}%
\pgfpathlineto{\pgfqpoint{4.024000in}{1.797512in}}%
\pgfpathlineto{\pgfqpoint{4.028509in}{1.778814in}}%
\pgfpathlineto{\pgfqpoint{4.033018in}{1.814374in}}%
\pgfpathlineto{\pgfqpoint{4.037527in}{1.809500in}}%
\pgfpathlineto{\pgfqpoint{4.042036in}{1.823032in}}%
\pgfpathlineto{\pgfqpoint{4.046545in}{1.789905in}}%
\pgfpathlineto{\pgfqpoint{4.051055in}{1.806393in}}%
\pgfpathlineto{\pgfqpoint{4.055564in}{1.829420in}}%
\pgfpathlineto{\pgfqpoint{4.060073in}{1.927369in}}%
\pgfpathlineto{\pgfqpoint{4.064582in}{1.841541in}}%
\pgfpathlineto{\pgfqpoint{4.069091in}{1.844675in}}%
\pgfpathlineto{\pgfqpoint{4.073600in}{1.825227in}}%
\pgfpathlineto{\pgfqpoint{4.078109in}{1.824467in}}%
\pgfpathlineto{\pgfqpoint{4.082618in}{1.875682in}}%
\pgfpathlineto{\pgfqpoint{4.087127in}{1.857409in}}%
\pgfpathlineto{\pgfqpoint{4.091636in}{1.905593in}}%
\pgfpathlineto{\pgfqpoint{4.096145in}{1.898368in}}%
\pgfpathlineto{\pgfqpoint{4.100655in}{1.852436in}}%
\pgfpathlineto{\pgfqpoint{4.105164in}{1.828233in}}%
\pgfpathlineto{\pgfqpoint{4.109673in}{1.830563in}}%
\pgfpathlineto{\pgfqpoint{4.114182in}{1.854383in}}%
\pgfpathlineto{\pgfqpoint{4.118691in}{1.847585in}}%
\pgfpathlineto{\pgfqpoint{4.123200in}{1.832588in}}%
\pgfpathlineto{\pgfqpoint{4.127709in}{1.865866in}}%
\pgfpathlineto{\pgfqpoint{4.132218in}{1.883166in}}%
\pgfpathlineto{\pgfqpoint{4.136727in}{1.856739in}}%
\pgfpathlineto{\pgfqpoint{4.145745in}{1.873725in}}%
\pgfpathlineto{\pgfqpoint{4.150255in}{2.003765in}}%
\pgfpathlineto{\pgfqpoint{4.159273in}{1.874387in}}%
\pgfpathlineto{\pgfqpoint{4.163782in}{1.884129in}}%
\pgfpathlineto{\pgfqpoint{4.168291in}{1.944909in}}%
\pgfpathlineto{\pgfqpoint{4.172800in}{1.919280in}}%
\pgfpathlineto{\pgfqpoint{4.177309in}{1.910583in}}%
\pgfpathlineto{\pgfqpoint{4.181818in}{1.908242in}}%
\pgfpathlineto{\pgfqpoint{4.186327in}{1.919119in}}%
\pgfpathlineto{\pgfqpoint{4.190836in}{1.920235in}}%
\pgfpathlineto{\pgfqpoint{4.195345in}{1.890270in}}%
\pgfpathlineto{\pgfqpoint{4.199855in}{1.877848in}}%
\pgfpathlineto{\pgfqpoint{4.204364in}{1.947700in}}%
\pgfpathlineto{\pgfqpoint{4.208873in}{1.923983in}}%
\pgfpathlineto{\pgfqpoint{4.213382in}{1.959108in}}%
\pgfpathlineto{\pgfqpoint{4.217891in}{1.963246in}}%
\pgfpathlineto{\pgfqpoint{4.222400in}{1.915622in}}%
\pgfpathlineto{\pgfqpoint{4.226909in}{1.960256in}}%
\pgfpathlineto{\pgfqpoint{4.231418in}{1.963269in}}%
\pgfpathlineto{\pgfqpoint{4.235927in}{1.906342in}}%
\pgfpathlineto{\pgfqpoint{4.240436in}{1.971101in}}%
\pgfpathlineto{\pgfqpoint{4.244945in}{1.923367in}}%
\pgfpathlineto{\pgfqpoint{4.249455in}{1.926058in}}%
\pgfpathlineto{\pgfqpoint{4.253964in}{1.954259in}}%
\pgfpathlineto{\pgfqpoint{4.258473in}{2.007898in}}%
\pgfpathlineto{\pgfqpoint{4.267491in}{1.918596in}}%
\pgfpathlineto{\pgfqpoint{4.272000in}{2.027409in}}%
\pgfpathlineto{\pgfqpoint{4.276509in}{1.961294in}}%
\pgfpathlineto{\pgfqpoint{4.281018in}{1.962764in}}%
\pgfpathlineto{\pgfqpoint{4.285527in}{2.012617in}}%
\pgfpathlineto{\pgfqpoint{4.290036in}{2.008843in}}%
\pgfpathlineto{\pgfqpoint{4.294545in}{1.986740in}}%
\pgfpathlineto{\pgfqpoint{4.299055in}{2.007719in}}%
\pgfpathlineto{\pgfqpoint{4.303564in}{1.951229in}}%
\pgfpathlineto{\pgfqpoint{4.308073in}{2.018594in}}%
\pgfpathlineto{\pgfqpoint{4.312582in}{2.015618in}}%
\pgfpathlineto{\pgfqpoint{4.317091in}{1.962170in}}%
\pgfpathlineto{\pgfqpoint{4.321600in}{2.000427in}}%
\pgfpathlineto{\pgfqpoint{4.326109in}{1.982872in}}%
\pgfpathlineto{\pgfqpoint{4.330618in}{2.036800in}}%
\pgfpathlineto{\pgfqpoint{4.335127in}{1.987684in}}%
\pgfpathlineto{\pgfqpoint{4.339636in}{2.066317in}}%
\pgfpathlineto{\pgfqpoint{4.344145in}{2.012681in}}%
\pgfpathlineto{\pgfqpoint{4.348655in}{1.995788in}}%
\pgfpathlineto{\pgfqpoint{4.353164in}{2.002545in}}%
\pgfpathlineto{\pgfqpoint{4.357673in}{2.001272in}}%
\pgfpathlineto{\pgfqpoint{4.362182in}{2.050682in}}%
\pgfpathlineto{\pgfqpoint{4.366691in}{2.021006in}}%
\pgfpathlineto{\pgfqpoint{4.371200in}{2.028272in}}%
\pgfpathlineto{\pgfqpoint{4.375709in}{2.052586in}}%
\pgfpathlineto{\pgfqpoint{4.380218in}{2.048472in}}%
\pgfpathlineto{\pgfqpoint{4.384727in}{2.031886in}}%
\pgfpathlineto{\pgfqpoint{4.389236in}{2.103477in}}%
\pgfpathlineto{\pgfqpoint{4.393745in}{2.044057in}}%
\pgfpathlineto{\pgfqpoint{4.398255in}{2.048705in}}%
\pgfpathlineto{\pgfqpoint{4.402764in}{2.131847in}}%
\pgfpathlineto{\pgfqpoint{4.411782in}{2.074826in}}%
\pgfpathlineto{\pgfqpoint{4.416291in}{2.055462in}}%
\pgfpathlineto{\pgfqpoint{4.420800in}{2.105570in}}%
\pgfpathlineto{\pgfqpoint{4.425309in}{2.041077in}}%
\pgfpathlineto{\pgfqpoint{4.429818in}{2.080518in}}%
\pgfpathlineto{\pgfqpoint{4.434327in}{2.066568in}}%
\pgfpathlineto{\pgfqpoint{4.438836in}{2.144953in}}%
\pgfpathlineto{\pgfqpoint{4.443345in}{2.070615in}}%
\pgfpathlineto{\pgfqpoint{4.447855in}{2.057648in}}%
\pgfpathlineto{\pgfqpoint{4.452364in}{2.091967in}}%
\pgfpathlineto{\pgfqpoint{4.456873in}{2.083052in}}%
\pgfpathlineto{\pgfqpoint{4.461382in}{2.060925in}}%
\pgfpathlineto{\pgfqpoint{4.465891in}{2.109414in}}%
\pgfpathlineto{\pgfqpoint{4.470400in}{2.110878in}}%
\pgfpathlineto{\pgfqpoint{4.474909in}{2.071746in}}%
\pgfpathlineto{\pgfqpoint{4.479418in}{2.085975in}}%
\pgfpathlineto{\pgfqpoint{4.483927in}{2.115789in}}%
\pgfpathlineto{\pgfqpoint{4.488436in}{2.096865in}}%
\pgfpathlineto{\pgfqpoint{4.492945in}{2.120800in}}%
\pgfpathlineto{\pgfqpoint{4.497455in}{2.100944in}}%
\pgfpathlineto{\pgfqpoint{4.501964in}{2.104152in}}%
\pgfpathlineto{\pgfqpoint{4.506473in}{2.094850in}}%
\pgfpathlineto{\pgfqpoint{4.510982in}{2.119149in}}%
\pgfpathlineto{\pgfqpoint{4.515491in}{2.128391in}}%
\pgfpathlineto{\pgfqpoint{4.520000in}{2.143605in}}%
\pgfpathlineto{\pgfqpoint{4.524509in}{2.239364in}}%
\pgfpathlineto{\pgfqpoint{4.529018in}{2.212897in}}%
\pgfpathlineto{\pgfqpoint{4.533527in}{2.132187in}}%
\pgfpathlineto{\pgfqpoint{4.538036in}{2.172298in}}%
\pgfpathlineto{\pgfqpoint{4.547055in}{2.187850in}}%
\pgfpathlineto{\pgfqpoint{4.551564in}{2.184976in}}%
\pgfpathlineto{\pgfqpoint{4.556073in}{2.175111in}}%
\pgfpathlineto{\pgfqpoint{4.560582in}{2.202919in}}%
\pgfpathlineto{\pgfqpoint{4.565091in}{2.162951in}}%
\pgfpathlineto{\pgfqpoint{4.569600in}{2.152591in}}%
\pgfpathlineto{\pgfqpoint{4.574109in}{2.166819in}}%
\pgfpathlineto{\pgfqpoint{4.578618in}{2.162396in}}%
\pgfpathlineto{\pgfqpoint{4.583127in}{2.298247in}}%
\pgfpathlineto{\pgfqpoint{4.587636in}{2.230367in}}%
\pgfpathlineto{\pgfqpoint{4.592145in}{2.296182in}}%
\pgfpathlineto{\pgfqpoint{4.596655in}{2.251618in}}%
\pgfpathlineto{\pgfqpoint{4.601164in}{2.365950in}}%
\pgfpathlineto{\pgfqpoint{4.605673in}{2.279017in}}%
\pgfpathlineto{\pgfqpoint{4.610182in}{2.229587in}}%
\pgfpathlineto{\pgfqpoint{4.614691in}{2.307146in}}%
\pgfpathlineto{\pgfqpoint{4.619200in}{2.199235in}}%
\pgfpathlineto{\pgfqpoint{4.623709in}{2.304113in}}%
\pgfpathlineto{\pgfqpoint{4.628218in}{2.291138in}}%
\pgfpathlineto{\pgfqpoint{4.632727in}{2.221485in}}%
\pgfpathlineto{\pgfqpoint{4.637236in}{2.214170in}}%
\pgfpathlineto{\pgfqpoint{4.641745in}{2.298010in}}%
\pgfpathlineto{\pgfqpoint{4.646255in}{2.462076in}}%
\pgfpathlineto{\pgfqpoint{4.650764in}{2.339065in}}%
\pgfpathlineto{\pgfqpoint{4.655273in}{2.387141in}}%
\pgfpathlineto{\pgfqpoint{4.659782in}{2.378385in}}%
\pgfpathlineto{\pgfqpoint{4.664291in}{2.933593in}}%
\pgfpathlineto{\pgfqpoint{4.668800in}{2.649467in}}%
\pgfpathlineto{\pgfqpoint{4.673309in}{3.060323in}}%
\pgfpathlineto{\pgfqpoint{4.677818in}{2.495971in}}%
\pgfpathlineto{\pgfqpoint{4.682327in}{2.231645in}}%
\pgfpathlineto{\pgfqpoint{4.686836in}{2.376404in}}%
\pgfpathlineto{\pgfqpoint{4.691345in}{2.374360in}}%
\pgfpathlineto{\pgfqpoint{4.695855in}{2.612800in}}%
\pgfpathlineto{\pgfqpoint{4.700364in}{2.327281in}}%
\pgfpathlineto{\pgfqpoint{4.704873in}{2.616132in}}%
\pgfpathlineto{\pgfqpoint{4.709382in}{3.413712in}}%
\pgfpathlineto{\pgfqpoint{4.713891in}{2.440991in}}%
\pgfpathlineto{\pgfqpoint{4.718400in}{2.673843in}}%
\pgfpathlineto{\pgfqpoint{4.722909in}{2.602749in}}%
\pgfpathlineto{\pgfqpoint{4.727418in}{2.474230in}}%
\pgfpathlineto{\pgfqpoint{4.731927in}{2.498279in}}%
\pgfpathlineto{\pgfqpoint{4.736436in}{2.397143in}}%
\pgfpathlineto{\pgfqpoint{4.740945in}{2.424958in}}%
\pgfpathlineto{\pgfqpoint{4.745455in}{2.416817in}}%
\pgfpathlineto{\pgfqpoint{4.749964in}{2.705467in}}%
\pgfpathlineto{\pgfqpoint{4.754473in}{2.576111in}}%
\pgfpathlineto{\pgfqpoint{4.758982in}{2.132678in}}%
\pgfpathlineto{\pgfqpoint{4.763491in}{3.208617in}}%
\pgfpathlineto{\pgfqpoint{4.768000in}{2.136898in}}%
\pgfpathlineto{\pgfqpoint{4.772509in}{2.165847in}}%
\pgfpathlineto{\pgfqpoint{4.777018in}{2.202315in}}%
\pgfpathlineto{\pgfqpoint{4.781527in}{2.224194in}}%
\pgfpathlineto{\pgfqpoint{4.786036in}{2.212784in}}%
\pgfpathlineto{\pgfqpoint{4.790545in}{2.161576in}}%
\pgfpathlineto{\pgfqpoint{4.795055in}{2.174128in}}%
\pgfpathlineto{\pgfqpoint{4.799564in}{2.192484in}}%
\pgfpathlineto{\pgfqpoint{4.804073in}{3.626492in}}%
\pgfpathlineto{\pgfqpoint{4.808582in}{3.170526in}}%
\pgfpathlineto{\pgfqpoint{4.813091in}{2.380974in}}%
\pgfpathlineto{\pgfqpoint{4.817600in}{2.185072in}}%
\pgfpathlineto{\pgfqpoint{4.822109in}{2.312126in}}%
\pgfpathlineto{\pgfqpoint{4.826618in}{2.310882in}}%
\pgfpathlineto{\pgfqpoint{4.831127in}{2.182589in}}%
\pgfpathlineto{\pgfqpoint{4.835636in}{2.443792in}}%
\pgfpathlineto{\pgfqpoint{4.840145in}{2.345585in}}%
\pgfpathlineto{\pgfqpoint{4.844655in}{2.365510in}}%
\pgfpathlineto{\pgfqpoint{4.849164in}{3.263388in}}%
\pgfpathlineto{\pgfqpoint{4.853673in}{2.409085in}}%
\pgfpathlineto{\pgfqpoint{4.858182in}{2.435798in}}%
\pgfpathlineto{\pgfqpoint{4.862691in}{2.474611in}}%
\pgfpathlineto{\pgfqpoint{4.867200in}{2.998924in}}%
\pgfpathlineto{\pgfqpoint{4.871709in}{2.566578in}}%
\pgfpathlineto{\pgfqpoint{4.876218in}{2.441465in}}%
\pgfpathlineto{\pgfqpoint{4.885236in}{2.716933in}}%
\pgfpathlineto{\pgfqpoint{4.889745in}{2.412408in}}%
\pgfpathlineto{\pgfqpoint{4.894255in}{2.452369in}}%
\pgfpathlineto{\pgfqpoint{4.898764in}{2.427161in}}%
\pgfpathlineto{\pgfqpoint{4.903273in}{2.450527in}}%
\pgfpathlineto{\pgfqpoint{4.907782in}{2.426959in}}%
\pgfpathlineto{\pgfqpoint{4.912291in}{2.498695in}}%
\pgfpathlineto{\pgfqpoint{4.916800in}{2.426512in}}%
\pgfpathlineto{\pgfqpoint{4.921309in}{2.474664in}}%
\pgfpathlineto{\pgfqpoint{4.925818in}{2.437442in}}%
\pgfpathlineto{\pgfqpoint{4.930327in}{2.501591in}}%
\pgfpathlineto{\pgfqpoint{4.934836in}{2.521170in}}%
\pgfpathlineto{\pgfqpoint{4.939345in}{2.705700in}}%
\pgfpathlineto{\pgfqpoint{4.943855in}{2.549138in}}%
\pgfpathlineto{\pgfqpoint{4.948364in}{2.519516in}}%
\pgfpathlineto{\pgfqpoint{4.952873in}{2.600021in}}%
\pgfpathlineto{\pgfqpoint{4.957382in}{2.531074in}}%
\pgfpathlineto{\pgfqpoint{4.961891in}{2.538999in}}%
\pgfpathlineto{\pgfqpoint{4.966400in}{2.509448in}}%
\pgfpathlineto{\pgfqpoint{4.970909in}{2.498655in}}%
\pgfpathlineto{\pgfqpoint{4.975418in}{2.473273in}}%
\pgfpathlineto{\pgfqpoint{4.979927in}{2.586236in}}%
\pgfpathlineto{\pgfqpoint{4.984436in}{2.513723in}}%
\pgfpathlineto{\pgfqpoint{4.988945in}{2.621365in}}%
\pgfpathlineto{\pgfqpoint{4.993455in}{2.528985in}}%
\pgfpathlineto{\pgfqpoint{4.997964in}{2.498699in}}%
\pgfpathlineto{\pgfqpoint{5.002473in}{2.569627in}}%
\pgfpathlineto{\pgfqpoint{5.006982in}{2.513994in}}%
\pgfpathlineto{\pgfqpoint{5.011491in}{2.561854in}}%
\pgfpathlineto{\pgfqpoint{5.016000in}{2.490435in}}%
\pgfpathlineto{\pgfqpoint{5.020509in}{2.540453in}}%
\pgfpathlineto{\pgfqpoint{5.025018in}{2.497547in}}%
\pgfpathlineto{\pgfqpoint{5.029527in}{2.590799in}}%
\pgfpathlineto{\pgfqpoint{5.034036in}{2.538586in}}%
\pgfpathlineto{\pgfqpoint{5.038545in}{2.714947in}}%
\pgfpathlineto{\pgfqpoint{5.043055in}{2.769680in}}%
\pgfpathlineto{\pgfqpoint{5.047564in}{2.597448in}}%
\pgfpathlineto{\pgfqpoint{5.052073in}{2.643996in}}%
\pgfpathlineto{\pgfqpoint{5.056582in}{2.778178in}}%
\pgfpathlineto{\pgfqpoint{5.061091in}{2.601757in}}%
\pgfpathlineto{\pgfqpoint{5.065600in}{2.584834in}}%
\pgfpathlineto{\pgfqpoint{5.070109in}{2.563450in}}%
\pgfpathlineto{\pgfqpoint{5.074618in}{2.597096in}}%
\pgfpathlineto{\pgfqpoint{5.079127in}{2.603378in}}%
\pgfpathlineto{\pgfqpoint{5.083636in}{2.617075in}}%
\pgfpathlineto{\pgfqpoint{5.088145in}{2.656616in}}%
\pgfpathlineto{\pgfqpoint{5.092655in}{2.629942in}}%
\pgfpathlineto{\pgfqpoint{5.097164in}{2.626745in}}%
\pgfpathlineto{\pgfqpoint{5.101673in}{2.601563in}}%
\pgfpathlineto{\pgfqpoint{5.106182in}{2.671580in}}%
\pgfpathlineto{\pgfqpoint{5.110691in}{2.655686in}}%
\pgfpathlineto{\pgfqpoint{5.119709in}{2.594912in}}%
\pgfpathlineto{\pgfqpoint{5.124218in}{2.723929in}}%
\pgfpathlineto{\pgfqpoint{5.128727in}{2.595730in}}%
\pgfpathlineto{\pgfqpoint{5.133236in}{2.601461in}}%
\pgfpathlineto{\pgfqpoint{5.142255in}{2.678408in}}%
\pgfpathlineto{\pgfqpoint{5.146764in}{3.293769in}}%
\pgfpathlineto{\pgfqpoint{5.151273in}{3.647793in}}%
\pgfpathlineto{\pgfqpoint{5.155782in}{3.285011in}}%
\pgfpathlineto{\pgfqpoint{5.160291in}{4.056000in}}%
\pgfpathlineto{\pgfqpoint{5.164800in}{3.005514in}}%
\pgfpathlineto{\pgfqpoint{5.169309in}{2.910798in}}%
\pgfpathlineto{\pgfqpoint{5.173818in}{2.731005in}}%
\pgfpathlineto{\pgfqpoint{5.178327in}{3.019783in}}%
\pgfpathlineto{\pgfqpoint{5.182836in}{2.725614in}}%
\pgfpathlineto{\pgfqpoint{5.187345in}{2.766034in}}%
\pgfpathlineto{\pgfqpoint{5.191855in}{2.792312in}}%
\pgfpathlineto{\pgfqpoint{5.196364in}{2.665968in}}%
\pgfpathlineto{\pgfqpoint{5.200873in}{2.753987in}}%
\pgfpathlineto{\pgfqpoint{5.205382in}{2.704995in}}%
\pgfpathlineto{\pgfqpoint{5.209891in}{2.706359in}}%
\pgfpathlineto{\pgfqpoint{5.214400in}{2.758692in}}%
\pgfpathlineto{\pgfqpoint{5.218909in}{2.705097in}}%
\pgfpathlineto{\pgfqpoint{5.223418in}{2.866837in}}%
\pgfpathlineto{\pgfqpoint{5.227927in}{2.829065in}}%
\pgfpathlineto{\pgfqpoint{5.232436in}{2.806336in}}%
\pgfpathlineto{\pgfqpoint{5.236945in}{2.725633in}}%
\pgfpathlineto{\pgfqpoint{5.241455in}{2.709775in}}%
\pgfpathlineto{\pgfqpoint{5.245964in}{3.241621in}}%
\pgfpathlineto{\pgfqpoint{5.250473in}{3.938654in}}%
\pgfpathlineto{\pgfqpoint{5.254982in}{3.562115in}}%
\pgfpathlineto{\pgfqpoint{5.259491in}{2.881789in}}%
\pgfpathlineto{\pgfqpoint{5.268509in}{3.359412in}}%
\pgfpathlineto{\pgfqpoint{5.273018in}{2.782547in}}%
\pgfpathlineto{\pgfqpoint{5.282036in}{2.848577in}}%
\pgfpathlineto{\pgfqpoint{5.291055in}{2.676573in}}%
\pgfpathlineto{\pgfqpoint{5.295564in}{2.805843in}}%
\pgfpathlineto{\pgfqpoint{5.300073in}{2.789351in}}%
\pgfpathlineto{\pgfqpoint{5.304582in}{2.756161in}}%
\pgfpathlineto{\pgfqpoint{5.313600in}{2.840623in}}%
\pgfpathlineto{\pgfqpoint{5.318109in}{2.762844in}}%
\pgfpathlineto{\pgfqpoint{5.322618in}{2.809287in}}%
\pgfpathlineto{\pgfqpoint{5.327127in}{2.753517in}}%
\pgfpathlineto{\pgfqpoint{5.331636in}{2.859328in}}%
\pgfpathlineto{\pgfqpoint{5.336145in}{2.918887in}}%
\pgfpathlineto{\pgfqpoint{5.340655in}{2.742859in}}%
\pgfpathlineto{\pgfqpoint{5.349673in}{2.851109in}}%
\pgfpathlineto{\pgfqpoint{5.354182in}{2.825294in}}%
\pgfpathlineto{\pgfqpoint{5.358691in}{2.813551in}}%
\pgfpathlineto{\pgfqpoint{5.363200in}{2.797825in}}%
\pgfpathlineto{\pgfqpoint{5.367709in}{2.829355in}}%
\pgfpathlineto{\pgfqpoint{5.372218in}{2.810404in}}%
\pgfpathlineto{\pgfqpoint{5.376727in}{2.846545in}}%
\pgfpathlineto{\pgfqpoint{5.381236in}{2.817820in}}%
\pgfpathlineto{\pgfqpoint{5.385745in}{2.816331in}}%
\pgfpathlineto{\pgfqpoint{5.390255in}{2.949396in}}%
\pgfpathlineto{\pgfqpoint{5.394764in}{2.930377in}}%
\pgfpathlineto{\pgfqpoint{5.399273in}{2.866890in}}%
\pgfpathlineto{\pgfqpoint{5.403782in}{2.866931in}}%
\pgfpathlineto{\pgfqpoint{5.408291in}{2.906607in}}%
\pgfpathlineto{\pgfqpoint{5.412800in}{2.865189in}}%
\pgfpathlineto{\pgfqpoint{5.417309in}{2.885957in}}%
\pgfpathlineto{\pgfqpoint{5.421818in}{2.881471in}}%
\pgfpathlineto{\pgfqpoint{5.426327in}{2.911305in}}%
\pgfpathlineto{\pgfqpoint{5.430836in}{2.868605in}}%
\pgfpathlineto{\pgfqpoint{5.435345in}{2.932084in}}%
\pgfpathlineto{\pgfqpoint{5.439855in}{2.929572in}}%
\pgfpathlineto{\pgfqpoint{5.444364in}{2.872250in}}%
\pgfpathlineto{\pgfqpoint{5.448873in}{2.988089in}}%
\pgfpathlineto{\pgfqpoint{5.453382in}{2.890927in}}%
\pgfpathlineto{\pgfqpoint{5.457891in}{3.104535in}}%
\pgfpathlineto{\pgfqpoint{5.462400in}{3.022609in}}%
\pgfpathlineto{\pgfqpoint{5.466909in}{2.920038in}}%
\pgfpathlineto{\pgfqpoint{5.471418in}{2.957758in}}%
\pgfpathlineto{\pgfqpoint{5.475927in}{2.883209in}}%
\pgfpathlineto{\pgfqpoint{5.480436in}{3.129846in}}%
\pgfpathlineto{\pgfqpoint{5.484945in}{3.008763in}}%
\pgfpathlineto{\pgfqpoint{5.489455in}{2.929087in}}%
\pgfpathlineto{\pgfqpoint{5.493964in}{3.014761in}}%
\pgfpathlineto{\pgfqpoint{5.498473in}{2.911375in}}%
\pgfpathlineto{\pgfqpoint{5.502982in}{2.986425in}}%
\pgfpathlineto{\pgfqpoint{5.507491in}{2.961366in}}%
\pgfpathlineto{\pgfqpoint{5.512000in}{2.988775in}}%
\pgfpathlineto{\pgfqpoint{5.516509in}{2.979184in}}%
\pgfpathlineto{\pgfqpoint{5.521018in}{2.993551in}}%
\pgfpathlineto{\pgfqpoint{5.525527in}{3.135508in}}%
\pgfpathlineto{\pgfqpoint{5.530036in}{2.909680in}}%
\pgfpathlineto{\pgfqpoint{5.534545in}{2.969936in}}%
\pgfpathlineto{\pgfqpoint{5.534545in}{2.969936in}}%
\pgfusepath{stroke}%
\end{pgfscope}%
\begin{pgfscope}%
\pgfsetrectcap%
\pgfsetmiterjoin%
\pgfsetlinewidth{0.803000pt}%
\definecolor{currentstroke}{rgb}{0.000000,0.000000,0.000000}%
\pgfsetstrokecolor{currentstroke}%
\pgfsetdash{}{0pt}%
\pgfpathmoveto{\pgfqpoint{0.800000in}{0.528000in}}%
\pgfpathlineto{\pgfqpoint{0.800000in}{4.224000in}}%
\pgfusepath{stroke}%
\end{pgfscope}%
\begin{pgfscope}%
\pgfsetrectcap%
\pgfsetmiterjoin%
\pgfsetlinewidth{0.803000pt}%
\definecolor{currentstroke}{rgb}{0.000000,0.000000,0.000000}%
\pgfsetstrokecolor{currentstroke}%
\pgfsetdash{}{0pt}%
\pgfpathmoveto{\pgfqpoint{5.760000in}{0.528000in}}%
\pgfpathlineto{\pgfqpoint{5.760000in}{4.224000in}}%
\pgfusepath{stroke}%
\end{pgfscope}%
\begin{pgfscope}%
\pgfsetrectcap%
\pgfsetmiterjoin%
\pgfsetlinewidth{0.803000pt}%
\definecolor{currentstroke}{rgb}{0.000000,0.000000,0.000000}%
\pgfsetstrokecolor{currentstroke}%
\pgfsetdash{}{0pt}%
\pgfpathmoveto{\pgfqpoint{0.800000in}{0.528000in}}%
\pgfpathlineto{\pgfqpoint{5.760000in}{0.528000in}}%
\pgfusepath{stroke}%
\end{pgfscope}%
\begin{pgfscope}%
\pgfsetrectcap%
\pgfsetmiterjoin%
\pgfsetlinewidth{0.803000pt}%
\definecolor{currentstroke}{rgb}{0.000000,0.000000,0.000000}%
\pgfsetstrokecolor{currentstroke}%
\pgfsetdash{}{0pt}%
\pgfpathmoveto{\pgfqpoint{0.800000in}{4.224000in}}%
\pgfpathlineto{\pgfqpoint{5.760000in}{4.224000in}}%
\pgfusepath{stroke}%
\end{pgfscope}%
\begin{pgfscope}%
\definecolor{textcolor}{rgb}{0.000000,0.000000,0.000000}%
\pgfsetstrokecolor{textcolor}%
\pgfsetfillcolor{textcolor}%
\pgftext[x=3.280000in,y=4.307333in,,base]{\color{textcolor}\ttfamily\fontsize{12.000000}{14.400000}\selectfont Selection Sort  Time vs Input size}%
\end{pgfscope}%
\begin{pgfscope}%
\pgfsetbuttcap%
\pgfsetmiterjoin%
\definecolor{currentfill}{rgb}{1.000000,1.000000,1.000000}%
\pgfsetfillcolor{currentfill}%
\pgfsetfillopacity{0.800000}%
\pgfsetlinewidth{1.003750pt}%
\definecolor{currentstroke}{rgb}{0.800000,0.800000,0.800000}%
\pgfsetstrokecolor{currentstroke}%
\pgfsetstrokeopacity{0.800000}%
\pgfsetdash{}{0pt}%
\pgfpathmoveto{\pgfqpoint{0.897222in}{3.908286in}}%
\pgfpathlineto{\pgfqpoint{2.094230in}{3.908286in}}%
\pgfpathquadraticcurveto{\pgfqpoint{2.122008in}{3.908286in}}{\pgfqpoint{2.122008in}{3.936063in}}%
\pgfpathlineto{\pgfqpoint{2.122008in}{4.126778in}}%
\pgfpathquadraticcurveto{\pgfqpoint{2.122008in}{4.154556in}}{\pgfqpoint{2.094230in}{4.154556in}}%
\pgfpathlineto{\pgfqpoint{0.897222in}{4.154556in}}%
\pgfpathquadraticcurveto{\pgfqpoint{0.869444in}{4.154556in}}{\pgfqpoint{0.869444in}{4.126778in}}%
\pgfpathlineto{\pgfqpoint{0.869444in}{3.936063in}}%
\pgfpathquadraticcurveto{\pgfqpoint{0.869444in}{3.908286in}}{\pgfqpoint{0.897222in}{3.908286in}}%
\pgfpathlineto{\pgfqpoint{0.897222in}{3.908286in}}%
\pgfpathclose%
\pgfusepath{stroke,fill}%
\end{pgfscope}%
\begin{pgfscope}%
\pgfsetrectcap%
\pgfsetroundjoin%
\pgfsetlinewidth{1.505625pt}%
\definecolor{currentstroke}{rgb}{0.000000,1.000000,0.498039}%
\pgfsetstrokecolor{currentstroke}%
\pgfsetdash{}{0pt}%
\pgfpathmoveto{\pgfqpoint{0.925000in}{4.041342in}}%
\pgfpathlineto{\pgfqpoint{1.063889in}{4.041342in}}%
\pgfpathlineto{\pgfqpoint{1.202778in}{4.041342in}}%
\pgfusepath{stroke}%
\end{pgfscope}%
\begin{pgfscope}%
\definecolor{textcolor}{rgb}{0.000000,0.000000,0.000000}%
\pgfsetstrokecolor{textcolor}%
\pgfsetfillcolor{textcolor}%
\pgftext[x=1.313889in,y=3.992731in,left,base]{\color{textcolor}\ttfamily\fontsize{10.000000}{12.000000}\selectfont Selection}%
\end{pgfscope}%
\end{pgfpicture}%
\makeatother%
\endgroup%

%% Creator: Matplotlib, PGF backend
%%
%% To include the figure in your LaTeX document, write
%%   \input{<filename>.pgf}
%%
%% Make sure the required packages are loaded in your preamble
%%   \usepackage{pgf}
%%
%% Also ensure that all the required font packages are loaded; for instance,
%% the lmodern package is sometimes necessary when using math font.
%%   \usepackage{lmodern}
%%
%% Figures using additional raster images can only be included by \input if
%% they are in the same directory as the main LaTeX file. For loading figures
%% from other directories you can use the `import` package
%%   \usepackage{import}
%%
%% and then include the figures with
%%   \import{<path to file>}{<filename>.pgf}
%%
%% Matplotlib used the following preamble
%%   \usepackage{fontspec}
%%   \setmainfont{DejaVuSerif.ttf}[Path=\detokenize{/home/dbk/.local/lib/python3.10/site-packages/matplotlib/mpl-data/fonts/ttf/}]
%%   \setsansfont{DejaVuSans.ttf}[Path=\detokenize{/home/dbk/.local/lib/python3.10/site-packages/matplotlib/mpl-data/fonts/ttf/}]
%%   \setmonofont{DejaVuSansMono.ttf}[Path=\detokenize{/home/dbk/.local/lib/python3.10/site-packages/matplotlib/mpl-data/fonts/ttf/}]
%%
\begingroup%
\makeatletter%
\begin{pgfpicture}%
\pgfpathrectangle{\pgfpointorigin}{\pgfqpoint{6.400000in}{4.800000in}}%
\pgfusepath{use as bounding box, clip}%
\begin{pgfscope}%
\pgfsetbuttcap%
\pgfsetmiterjoin%
\definecolor{currentfill}{rgb}{1.000000,1.000000,1.000000}%
\pgfsetfillcolor{currentfill}%
\pgfsetlinewidth{0.000000pt}%
\definecolor{currentstroke}{rgb}{1.000000,1.000000,1.000000}%
\pgfsetstrokecolor{currentstroke}%
\pgfsetdash{}{0pt}%
\pgfpathmoveto{\pgfqpoint{0.000000in}{0.000000in}}%
\pgfpathlineto{\pgfqpoint{6.400000in}{0.000000in}}%
\pgfpathlineto{\pgfqpoint{6.400000in}{4.800000in}}%
\pgfpathlineto{\pgfqpoint{0.000000in}{4.800000in}}%
\pgfpathlineto{\pgfqpoint{0.000000in}{0.000000in}}%
\pgfpathclose%
\pgfusepath{fill}%
\end{pgfscope}%
\begin{pgfscope}%
\pgfsetbuttcap%
\pgfsetmiterjoin%
\definecolor{currentfill}{rgb}{1.000000,1.000000,1.000000}%
\pgfsetfillcolor{currentfill}%
\pgfsetlinewidth{0.000000pt}%
\definecolor{currentstroke}{rgb}{0.000000,0.000000,0.000000}%
\pgfsetstrokecolor{currentstroke}%
\pgfsetstrokeopacity{0.000000}%
\pgfsetdash{}{0pt}%
\pgfpathmoveto{\pgfqpoint{0.800000in}{0.528000in}}%
\pgfpathlineto{\pgfqpoint{5.760000in}{0.528000in}}%
\pgfpathlineto{\pgfqpoint{5.760000in}{4.224000in}}%
\pgfpathlineto{\pgfqpoint{0.800000in}{4.224000in}}%
\pgfpathlineto{\pgfqpoint{0.800000in}{0.528000in}}%
\pgfpathclose%
\pgfusepath{fill}%
\end{pgfscope}%
\begin{pgfscope}%
\pgfsetbuttcap%
\pgfsetroundjoin%
\definecolor{currentfill}{rgb}{0.000000,0.000000,0.000000}%
\pgfsetfillcolor{currentfill}%
\pgfsetlinewidth{0.803000pt}%
\definecolor{currentstroke}{rgb}{0.000000,0.000000,0.000000}%
\pgfsetstrokecolor{currentstroke}%
\pgfsetdash{}{0pt}%
\pgfsys@defobject{currentmarker}{\pgfqpoint{0.000000in}{-0.048611in}}{\pgfqpoint{0.000000in}{0.000000in}}{%
\pgfpathmoveto{\pgfqpoint{0.000000in}{0.000000in}}%
\pgfpathlineto{\pgfqpoint{0.000000in}{-0.048611in}}%
\pgfusepath{stroke,fill}%
}%
\begin{pgfscope}%
\pgfsys@transformshift{1.020945in}{0.528000in}%
\pgfsys@useobject{currentmarker}{}%
\end{pgfscope}%
\end{pgfscope}%
\begin{pgfscope}%
\definecolor{textcolor}{rgb}{0.000000,0.000000,0.000000}%
\pgfsetstrokecolor{textcolor}%
\pgfsetfillcolor{textcolor}%
\pgftext[x=1.020945in,y=0.430778in,,top]{\color{textcolor}\ttfamily\fontsize{10.000000}{12.000000}\selectfont 0}%
\end{pgfscope}%
\begin{pgfscope}%
\pgfsetbuttcap%
\pgfsetroundjoin%
\definecolor{currentfill}{rgb}{0.000000,0.000000,0.000000}%
\pgfsetfillcolor{currentfill}%
\pgfsetlinewidth{0.803000pt}%
\definecolor{currentstroke}{rgb}{0.000000,0.000000,0.000000}%
\pgfsetstrokecolor{currentstroke}%
\pgfsetdash{}{0pt}%
\pgfsys@defobject{currentmarker}{\pgfqpoint{0.000000in}{-0.048611in}}{\pgfqpoint{0.000000in}{0.000000in}}{%
\pgfpathmoveto{\pgfqpoint{0.000000in}{0.000000in}}%
\pgfpathlineto{\pgfqpoint{0.000000in}{-0.048611in}}%
\pgfusepath{stroke,fill}%
}%
\begin{pgfscope}%
\pgfsys@transformshift{1.922764in}{0.528000in}%
\pgfsys@useobject{currentmarker}{}%
\end{pgfscope}%
\end{pgfscope}%
\begin{pgfscope}%
\definecolor{textcolor}{rgb}{0.000000,0.000000,0.000000}%
\pgfsetstrokecolor{textcolor}%
\pgfsetfillcolor{textcolor}%
\pgftext[x=1.922764in,y=0.430778in,,top]{\color{textcolor}\ttfamily\fontsize{10.000000}{12.000000}\selectfont 200}%
\end{pgfscope}%
\begin{pgfscope}%
\pgfsetbuttcap%
\pgfsetroundjoin%
\definecolor{currentfill}{rgb}{0.000000,0.000000,0.000000}%
\pgfsetfillcolor{currentfill}%
\pgfsetlinewidth{0.803000pt}%
\definecolor{currentstroke}{rgb}{0.000000,0.000000,0.000000}%
\pgfsetstrokecolor{currentstroke}%
\pgfsetdash{}{0pt}%
\pgfsys@defobject{currentmarker}{\pgfqpoint{0.000000in}{-0.048611in}}{\pgfqpoint{0.000000in}{0.000000in}}{%
\pgfpathmoveto{\pgfqpoint{0.000000in}{0.000000in}}%
\pgfpathlineto{\pgfqpoint{0.000000in}{-0.048611in}}%
\pgfusepath{stroke,fill}%
}%
\begin{pgfscope}%
\pgfsys@transformshift{2.824582in}{0.528000in}%
\pgfsys@useobject{currentmarker}{}%
\end{pgfscope}%
\end{pgfscope}%
\begin{pgfscope}%
\definecolor{textcolor}{rgb}{0.000000,0.000000,0.000000}%
\pgfsetstrokecolor{textcolor}%
\pgfsetfillcolor{textcolor}%
\pgftext[x=2.824582in,y=0.430778in,,top]{\color{textcolor}\ttfamily\fontsize{10.000000}{12.000000}\selectfont 400}%
\end{pgfscope}%
\begin{pgfscope}%
\pgfsetbuttcap%
\pgfsetroundjoin%
\definecolor{currentfill}{rgb}{0.000000,0.000000,0.000000}%
\pgfsetfillcolor{currentfill}%
\pgfsetlinewidth{0.803000pt}%
\definecolor{currentstroke}{rgb}{0.000000,0.000000,0.000000}%
\pgfsetstrokecolor{currentstroke}%
\pgfsetdash{}{0pt}%
\pgfsys@defobject{currentmarker}{\pgfqpoint{0.000000in}{-0.048611in}}{\pgfqpoint{0.000000in}{0.000000in}}{%
\pgfpathmoveto{\pgfqpoint{0.000000in}{0.000000in}}%
\pgfpathlineto{\pgfqpoint{0.000000in}{-0.048611in}}%
\pgfusepath{stroke,fill}%
}%
\begin{pgfscope}%
\pgfsys@transformshift{3.726400in}{0.528000in}%
\pgfsys@useobject{currentmarker}{}%
\end{pgfscope}%
\end{pgfscope}%
\begin{pgfscope}%
\definecolor{textcolor}{rgb}{0.000000,0.000000,0.000000}%
\pgfsetstrokecolor{textcolor}%
\pgfsetfillcolor{textcolor}%
\pgftext[x=3.726400in,y=0.430778in,,top]{\color{textcolor}\ttfamily\fontsize{10.000000}{12.000000}\selectfont 600}%
\end{pgfscope}%
\begin{pgfscope}%
\pgfsetbuttcap%
\pgfsetroundjoin%
\definecolor{currentfill}{rgb}{0.000000,0.000000,0.000000}%
\pgfsetfillcolor{currentfill}%
\pgfsetlinewidth{0.803000pt}%
\definecolor{currentstroke}{rgb}{0.000000,0.000000,0.000000}%
\pgfsetstrokecolor{currentstroke}%
\pgfsetdash{}{0pt}%
\pgfsys@defobject{currentmarker}{\pgfqpoint{0.000000in}{-0.048611in}}{\pgfqpoint{0.000000in}{0.000000in}}{%
\pgfpathmoveto{\pgfqpoint{0.000000in}{0.000000in}}%
\pgfpathlineto{\pgfqpoint{0.000000in}{-0.048611in}}%
\pgfusepath{stroke,fill}%
}%
\begin{pgfscope}%
\pgfsys@transformshift{4.628218in}{0.528000in}%
\pgfsys@useobject{currentmarker}{}%
\end{pgfscope}%
\end{pgfscope}%
\begin{pgfscope}%
\definecolor{textcolor}{rgb}{0.000000,0.000000,0.000000}%
\pgfsetstrokecolor{textcolor}%
\pgfsetfillcolor{textcolor}%
\pgftext[x=4.628218in,y=0.430778in,,top]{\color{textcolor}\ttfamily\fontsize{10.000000}{12.000000}\selectfont 800}%
\end{pgfscope}%
\begin{pgfscope}%
\pgfsetbuttcap%
\pgfsetroundjoin%
\definecolor{currentfill}{rgb}{0.000000,0.000000,0.000000}%
\pgfsetfillcolor{currentfill}%
\pgfsetlinewidth{0.803000pt}%
\definecolor{currentstroke}{rgb}{0.000000,0.000000,0.000000}%
\pgfsetstrokecolor{currentstroke}%
\pgfsetdash{}{0pt}%
\pgfsys@defobject{currentmarker}{\pgfqpoint{0.000000in}{-0.048611in}}{\pgfqpoint{0.000000in}{0.000000in}}{%
\pgfpathmoveto{\pgfqpoint{0.000000in}{0.000000in}}%
\pgfpathlineto{\pgfqpoint{0.000000in}{-0.048611in}}%
\pgfusepath{stroke,fill}%
}%
\begin{pgfscope}%
\pgfsys@transformshift{5.530036in}{0.528000in}%
\pgfsys@useobject{currentmarker}{}%
\end{pgfscope}%
\end{pgfscope}%
\begin{pgfscope}%
\definecolor{textcolor}{rgb}{0.000000,0.000000,0.000000}%
\pgfsetstrokecolor{textcolor}%
\pgfsetfillcolor{textcolor}%
\pgftext[x=5.530036in,y=0.430778in,,top]{\color{textcolor}\ttfamily\fontsize{10.000000}{12.000000}\selectfont 1000}%
\end{pgfscope}%
\begin{pgfscope}%
\definecolor{textcolor}{rgb}{0.000000,0.000000,0.000000}%
\pgfsetstrokecolor{textcolor}%
\pgfsetfillcolor{textcolor}%
\pgftext[x=3.280000in,y=0.240063in,,top]{\color{textcolor}\ttfamily\fontsize{10.000000}{12.000000}\selectfont Size of Array}%
\end{pgfscope}%
\begin{pgfscope}%
\pgfsetbuttcap%
\pgfsetroundjoin%
\definecolor{currentfill}{rgb}{0.000000,0.000000,0.000000}%
\pgfsetfillcolor{currentfill}%
\pgfsetlinewidth{0.803000pt}%
\definecolor{currentstroke}{rgb}{0.000000,0.000000,0.000000}%
\pgfsetstrokecolor{currentstroke}%
\pgfsetdash{}{0pt}%
\pgfsys@defobject{currentmarker}{\pgfqpoint{-0.048611in}{0.000000in}}{\pgfqpoint{-0.000000in}{0.000000in}}{%
\pgfpathmoveto{\pgfqpoint{-0.000000in}{0.000000in}}%
\pgfpathlineto{\pgfqpoint{-0.048611in}{0.000000in}}%
\pgfusepath{stroke,fill}%
}%
\begin{pgfscope}%
\pgfsys@transformshift{0.800000in}{0.556000in}%
\pgfsys@useobject{currentmarker}{}%
\end{pgfscope}%
\end{pgfscope}%
\begin{pgfscope}%
\definecolor{textcolor}{rgb}{0.000000,0.000000,0.000000}%
\pgfsetstrokecolor{textcolor}%
\pgfsetfillcolor{textcolor}%
\pgftext[x=0.451923in, y=0.502865in, left, base]{\color{textcolor}\ttfamily\fontsize{10.000000}{12.000000}\selectfont 225}%
\end{pgfscope}%
\begin{pgfscope}%
\pgfsetbuttcap%
\pgfsetroundjoin%
\definecolor{currentfill}{rgb}{0.000000,0.000000,0.000000}%
\pgfsetfillcolor{currentfill}%
\pgfsetlinewidth{0.803000pt}%
\definecolor{currentstroke}{rgb}{0.000000,0.000000,0.000000}%
\pgfsetstrokecolor{currentstroke}%
\pgfsetdash{}{0pt}%
\pgfsys@defobject{currentmarker}{\pgfqpoint{-0.048611in}{0.000000in}}{\pgfqpoint{-0.000000in}{0.000000in}}{%
\pgfpathmoveto{\pgfqpoint{-0.000000in}{0.000000in}}%
\pgfpathlineto{\pgfqpoint{-0.048611in}{0.000000in}}%
\pgfusepath{stroke,fill}%
}%
\begin{pgfscope}%
\pgfsys@transformshift{0.800000in}{1.056000in}%
\pgfsys@useobject{currentmarker}{}%
\end{pgfscope}%
\end{pgfscope}%
\begin{pgfscope}%
\definecolor{textcolor}{rgb}{0.000000,0.000000,0.000000}%
\pgfsetstrokecolor{textcolor}%
\pgfsetfillcolor{textcolor}%
\pgftext[x=0.451923in, y=1.002865in, left, base]{\color{textcolor}\ttfamily\fontsize{10.000000}{12.000000}\selectfont 250}%
\end{pgfscope}%
\begin{pgfscope}%
\pgfsetbuttcap%
\pgfsetroundjoin%
\definecolor{currentfill}{rgb}{0.000000,0.000000,0.000000}%
\pgfsetfillcolor{currentfill}%
\pgfsetlinewidth{0.803000pt}%
\definecolor{currentstroke}{rgb}{0.000000,0.000000,0.000000}%
\pgfsetstrokecolor{currentstroke}%
\pgfsetdash{}{0pt}%
\pgfsys@defobject{currentmarker}{\pgfqpoint{-0.048611in}{0.000000in}}{\pgfqpoint{-0.000000in}{0.000000in}}{%
\pgfpathmoveto{\pgfqpoint{-0.000000in}{0.000000in}}%
\pgfpathlineto{\pgfqpoint{-0.048611in}{0.000000in}}%
\pgfusepath{stroke,fill}%
}%
\begin{pgfscope}%
\pgfsys@transformshift{0.800000in}{1.556000in}%
\pgfsys@useobject{currentmarker}{}%
\end{pgfscope}%
\end{pgfscope}%
\begin{pgfscope}%
\definecolor{textcolor}{rgb}{0.000000,0.000000,0.000000}%
\pgfsetstrokecolor{textcolor}%
\pgfsetfillcolor{textcolor}%
\pgftext[x=0.451923in, y=1.502865in, left, base]{\color{textcolor}\ttfamily\fontsize{10.000000}{12.000000}\selectfont 275}%
\end{pgfscope}%
\begin{pgfscope}%
\pgfsetbuttcap%
\pgfsetroundjoin%
\definecolor{currentfill}{rgb}{0.000000,0.000000,0.000000}%
\pgfsetfillcolor{currentfill}%
\pgfsetlinewidth{0.803000pt}%
\definecolor{currentstroke}{rgb}{0.000000,0.000000,0.000000}%
\pgfsetstrokecolor{currentstroke}%
\pgfsetdash{}{0pt}%
\pgfsys@defobject{currentmarker}{\pgfqpoint{-0.048611in}{0.000000in}}{\pgfqpoint{-0.000000in}{0.000000in}}{%
\pgfpathmoveto{\pgfqpoint{-0.000000in}{0.000000in}}%
\pgfpathlineto{\pgfqpoint{-0.048611in}{0.000000in}}%
\pgfusepath{stroke,fill}%
}%
\begin{pgfscope}%
\pgfsys@transformshift{0.800000in}{2.056000in}%
\pgfsys@useobject{currentmarker}{}%
\end{pgfscope}%
\end{pgfscope}%
\begin{pgfscope}%
\definecolor{textcolor}{rgb}{0.000000,0.000000,0.000000}%
\pgfsetstrokecolor{textcolor}%
\pgfsetfillcolor{textcolor}%
\pgftext[x=0.451923in, y=2.002865in, left, base]{\color{textcolor}\ttfamily\fontsize{10.000000}{12.000000}\selectfont 300}%
\end{pgfscope}%
\begin{pgfscope}%
\pgfsetbuttcap%
\pgfsetroundjoin%
\definecolor{currentfill}{rgb}{0.000000,0.000000,0.000000}%
\pgfsetfillcolor{currentfill}%
\pgfsetlinewidth{0.803000pt}%
\definecolor{currentstroke}{rgb}{0.000000,0.000000,0.000000}%
\pgfsetstrokecolor{currentstroke}%
\pgfsetdash{}{0pt}%
\pgfsys@defobject{currentmarker}{\pgfqpoint{-0.048611in}{0.000000in}}{\pgfqpoint{-0.000000in}{0.000000in}}{%
\pgfpathmoveto{\pgfqpoint{-0.000000in}{0.000000in}}%
\pgfpathlineto{\pgfqpoint{-0.048611in}{0.000000in}}%
\pgfusepath{stroke,fill}%
}%
\begin{pgfscope}%
\pgfsys@transformshift{0.800000in}{2.556000in}%
\pgfsys@useobject{currentmarker}{}%
\end{pgfscope}%
\end{pgfscope}%
\begin{pgfscope}%
\definecolor{textcolor}{rgb}{0.000000,0.000000,0.000000}%
\pgfsetstrokecolor{textcolor}%
\pgfsetfillcolor{textcolor}%
\pgftext[x=0.451923in, y=2.502865in, left, base]{\color{textcolor}\ttfamily\fontsize{10.000000}{12.000000}\selectfont 325}%
\end{pgfscope}%
\begin{pgfscope}%
\pgfsetbuttcap%
\pgfsetroundjoin%
\definecolor{currentfill}{rgb}{0.000000,0.000000,0.000000}%
\pgfsetfillcolor{currentfill}%
\pgfsetlinewidth{0.803000pt}%
\definecolor{currentstroke}{rgb}{0.000000,0.000000,0.000000}%
\pgfsetstrokecolor{currentstroke}%
\pgfsetdash{}{0pt}%
\pgfsys@defobject{currentmarker}{\pgfqpoint{-0.048611in}{0.000000in}}{\pgfqpoint{-0.000000in}{0.000000in}}{%
\pgfpathmoveto{\pgfqpoint{-0.000000in}{0.000000in}}%
\pgfpathlineto{\pgfqpoint{-0.048611in}{0.000000in}}%
\pgfusepath{stroke,fill}%
}%
\begin{pgfscope}%
\pgfsys@transformshift{0.800000in}{3.056000in}%
\pgfsys@useobject{currentmarker}{}%
\end{pgfscope}%
\end{pgfscope}%
\begin{pgfscope}%
\definecolor{textcolor}{rgb}{0.000000,0.000000,0.000000}%
\pgfsetstrokecolor{textcolor}%
\pgfsetfillcolor{textcolor}%
\pgftext[x=0.451923in, y=3.002865in, left, base]{\color{textcolor}\ttfamily\fontsize{10.000000}{12.000000}\selectfont 350}%
\end{pgfscope}%
\begin{pgfscope}%
\pgfsetbuttcap%
\pgfsetroundjoin%
\definecolor{currentfill}{rgb}{0.000000,0.000000,0.000000}%
\pgfsetfillcolor{currentfill}%
\pgfsetlinewidth{0.803000pt}%
\definecolor{currentstroke}{rgb}{0.000000,0.000000,0.000000}%
\pgfsetstrokecolor{currentstroke}%
\pgfsetdash{}{0pt}%
\pgfsys@defobject{currentmarker}{\pgfqpoint{-0.048611in}{0.000000in}}{\pgfqpoint{-0.000000in}{0.000000in}}{%
\pgfpathmoveto{\pgfqpoint{-0.000000in}{0.000000in}}%
\pgfpathlineto{\pgfqpoint{-0.048611in}{0.000000in}}%
\pgfusepath{stroke,fill}%
}%
\begin{pgfscope}%
\pgfsys@transformshift{0.800000in}{3.556000in}%
\pgfsys@useobject{currentmarker}{}%
\end{pgfscope}%
\end{pgfscope}%
\begin{pgfscope}%
\definecolor{textcolor}{rgb}{0.000000,0.000000,0.000000}%
\pgfsetstrokecolor{textcolor}%
\pgfsetfillcolor{textcolor}%
\pgftext[x=0.451923in, y=3.502865in, left, base]{\color{textcolor}\ttfamily\fontsize{10.000000}{12.000000}\selectfont 375}%
\end{pgfscope}%
\begin{pgfscope}%
\pgfsetbuttcap%
\pgfsetroundjoin%
\definecolor{currentfill}{rgb}{0.000000,0.000000,0.000000}%
\pgfsetfillcolor{currentfill}%
\pgfsetlinewidth{0.803000pt}%
\definecolor{currentstroke}{rgb}{0.000000,0.000000,0.000000}%
\pgfsetstrokecolor{currentstroke}%
\pgfsetdash{}{0pt}%
\pgfsys@defobject{currentmarker}{\pgfqpoint{-0.048611in}{0.000000in}}{\pgfqpoint{-0.000000in}{0.000000in}}{%
\pgfpathmoveto{\pgfqpoint{-0.000000in}{0.000000in}}%
\pgfpathlineto{\pgfqpoint{-0.048611in}{0.000000in}}%
\pgfusepath{stroke,fill}%
}%
\begin{pgfscope}%
\pgfsys@transformshift{0.800000in}{4.056000in}%
\pgfsys@useobject{currentmarker}{}%
\end{pgfscope}%
\end{pgfscope}%
\begin{pgfscope}%
\definecolor{textcolor}{rgb}{0.000000,0.000000,0.000000}%
\pgfsetstrokecolor{textcolor}%
\pgfsetfillcolor{textcolor}%
\pgftext[x=0.451923in, y=4.002865in, left, base]{\color{textcolor}\ttfamily\fontsize{10.000000}{12.000000}\selectfont 400}%
\end{pgfscope}%
\begin{pgfscope}%
\definecolor{textcolor}{rgb}{0.000000,0.000000,0.000000}%
\pgfsetstrokecolor{textcolor}%
\pgfsetfillcolor{textcolor}%
\pgftext[x=0.396368in,y=2.376000in,,bottom,rotate=90.000000]{\color{textcolor}\ttfamily\fontsize{10.000000}{12.000000}\selectfont Memory}%
\end{pgfscope}%
\begin{pgfscope}%
\pgfpathrectangle{\pgfqpoint{0.800000in}{0.528000in}}{\pgfqpoint{4.960000in}{3.696000in}}%
\pgfusepath{clip}%
\pgfsetrectcap%
\pgfsetroundjoin%
\pgfsetlinewidth{1.505625pt}%
\definecolor{currentstroke}{rgb}{0.000000,1.000000,0.498039}%
\pgfsetstrokecolor{currentstroke}%
\pgfsetdash{}{0pt}%
\pgfpathmoveto{\pgfqpoint{1.025455in}{0.696000in}}%
\pgfpathlineto{\pgfqpoint{1.728873in}{0.696000in}}%
\pgfpathlineto{\pgfqpoint{1.733382in}{1.816000in}}%
\pgfpathlineto{\pgfqpoint{1.737891in}{3.496000in}}%
\pgfpathlineto{\pgfqpoint{2.892218in}{3.496000in}}%
\pgfpathlineto{\pgfqpoint{2.896727in}{4.056000in}}%
\pgfpathlineto{\pgfqpoint{5.534545in}{4.056000in}}%
\pgfpathlineto{\pgfqpoint{5.534545in}{4.056000in}}%
\pgfusepath{stroke}%
\end{pgfscope}%
\begin{pgfscope}%
\pgfsetrectcap%
\pgfsetmiterjoin%
\pgfsetlinewidth{0.803000pt}%
\definecolor{currentstroke}{rgb}{0.000000,0.000000,0.000000}%
\pgfsetstrokecolor{currentstroke}%
\pgfsetdash{}{0pt}%
\pgfpathmoveto{\pgfqpoint{0.800000in}{0.528000in}}%
\pgfpathlineto{\pgfqpoint{0.800000in}{4.224000in}}%
\pgfusepath{stroke}%
\end{pgfscope}%
\begin{pgfscope}%
\pgfsetrectcap%
\pgfsetmiterjoin%
\pgfsetlinewidth{0.803000pt}%
\definecolor{currentstroke}{rgb}{0.000000,0.000000,0.000000}%
\pgfsetstrokecolor{currentstroke}%
\pgfsetdash{}{0pt}%
\pgfpathmoveto{\pgfqpoint{5.760000in}{0.528000in}}%
\pgfpathlineto{\pgfqpoint{5.760000in}{4.224000in}}%
\pgfusepath{stroke}%
\end{pgfscope}%
\begin{pgfscope}%
\pgfsetrectcap%
\pgfsetmiterjoin%
\pgfsetlinewidth{0.803000pt}%
\definecolor{currentstroke}{rgb}{0.000000,0.000000,0.000000}%
\pgfsetstrokecolor{currentstroke}%
\pgfsetdash{}{0pt}%
\pgfpathmoveto{\pgfqpoint{0.800000in}{0.528000in}}%
\pgfpathlineto{\pgfqpoint{5.760000in}{0.528000in}}%
\pgfusepath{stroke}%
\end{pgfscope}%
\begin{pgfscope}%
\pgfsetrectcap%
\pgfsetmiterjoin%
\pgfsetlinewidth{0.803000pt}%
\definecolor{currentstroke}{rgb}{0.000000,0.000000,0.000000}%
\pgfsetstrokecolor{currentstroke}%
\pgfsetdash{}{0pt}%
\pgfpathmoveto{\pgfqpoint{0.800000in}{4.224000in}}%
\pgfpathlineto{\pgfqpoint{5.760000in}{4.224000in}}%
\pgfusepath{stroke}%
\end{pgfscope}%
\begin{pgfscope}%
\definecolor{textcolor}{rgb}{0.000000,0.000000,0.000000}%
\pgfsetstrokecolor{textcolor}%
\pgfsetfillcolor{textcolor}%
\pgftext[x=3.280000in,y=4.307333in,,base]{\color{textcolor}\ttfamily\fontsize{12.000000}{14.400000}\selectfont Selection Sort Memory vs Input size}%
\end{pgfscope}%
\begin{pgfscope}%
\pgfsetbuttcap%
\pgfsetmiterjoin%
\definecolor{currentfill}{rgb}{1.000000,1.000000,1.000000}%
\pgfsetfillcolor{currentfill}%
\pgfsetfillopacity{0.800000}%
\pgfsetlinewidth{1.003750pt}%
\definecolor{currentstroke}{rgb}{0.800000,0.800000,0.800000}%
\pgfsetstrokecolor{currentstroke}%
\pgfsetstrokeopacity{0.800000}%
\pgfsetdash{}{0pt}%
\pgfpathmoveto{\pgfqpoint{0.897222in}{3.908286in}}%
\pgfpathlineto{\pgfqpoint{2.094230in}{3.908286in}}%
\pgfpathquadraticcurveto{\pgfqpoint{2.122008in}{3.908286in}}{\pgfqpoint{2.122008in}{3.936063in}}%
\pgfpathlineto{\pgfqpoint{2.122008in}{4.126778in}}%
\pgfpathquadraticcurveto{\pgfqpoint{2.122008in}{4.154556in}}{\pgfqpoint{2.094230in}{4.154556in}}%
\pgfpathlineto{\pgfqpoint{0.897222in}{4.154556in}}%
\pgfpathquadraticcurveto{\pgfqpoint{0.869444in}{4.154556in}}{\pgfqpoint{0.869444in}{4.126778in}}%
\pgfpathlineto{\pgfqpoint{0.869444in}{3.936063in}}%
\pgfpathquadraticcurveto{\pgfqpoint{0.869444in}{3.908286in}}{\pgfqpoint{0.897222in}{3.908286in}}%
\pgfpathlineto{\pgfqpoint{0.897222in}{3.908286in}}%
\pgfpathclose%
\pgfusepath{stroke,fill}%
\end{pgfscope}%
\begin{pgfscope}%
\pgfsetrectcap%
\pgfsetroundjoin%
\pgfsetlinewidth{1.505625pt}%
\definecolor{currentstroke}{rgb}{0.000000,1.000000,0.498039}%
\pgfsetstrokecolor{currentstroke}%
\pgfsetdash{}{0pt}%
\pgfpathmoveto{\pgfqpoint{0.925000in}{4.041342in}}%
\pgfpathlineto{\pgfqpoint{1.063889in}{4.041342in}}%
\pgfpathlineto{\pgfqpoint{1.202778in}{4.041342in}}%
\pgfusepath{stroke}%
\end{pgfscope}%
\begin{pgfscope}%
\definecolor{textcolor}{rgb}{0.000000,0.000000,0.000000}%
\pgfsetstrokecolor{textcolor}%
\pgfsetfillcolor{textcolor}%
\pgftext[x=1.313889in,y=3.992731in,left,base]{\color{textcolor}\ttfamily\fontsize{10.000000}{12.000000}\selectfont Selection}%
\end{pgfscope}%
\end{pgfpicture}%
\makeatother%
\endgroup%

\input{../pgf/ss_c.pgf}
\input{../pgf/ss_s.pgf}
\input{../pgf/ss_i.pgf}
\subsubsection*{Insights}
Selection sort is a faster way to arrange smaller arrays but can
take more time for larger ones, it has a runtime complexity of
$O(N^2)$. However it is faster compared to bubble sort. Its space
complexity is $O(1)$.
\subsection{Insertion Sort}
\subsubsection*{Principle}
In insertion sort the array is split up virtually into a sorted
and unsorted part. Initially the first element is assumed to be
already sorted. We pick elements from the remaining unsorted part
of the array and compare it with each element of the sorted part.
If the selected unsorted element is greater than the sorted
element, it will be placed after the sorted element otherwise it
is placed in front of the sorted element. All the remaining
elements are arranged into the sorted part in a similar way.
\subsubsection*{Code}
\begin{minted}{python}
def insertionSort(unsortedList):
    swap = 0
    itr = 0
    comp = 0
    tracemalloc.start()
    t_s = perf_counter_ns()
    for i in range(1, len(unsortedList)):
        itr += 1
        key = unsortedList[i]
        ptr = i - 1
        comp += 1

        while ptr >= 0 and unsortedList[ptr] > key:
            unsortedList[ptr + 1] = unsortedList[ptr]
            ptr -= 1
            swap += 1
            itr += 1

        unsortedList[ptr + 1] = key
    t_e = perf_counter_ns()
    mem = tracemalloc.get_traced_memory()[1]
    tracemalloc.stop()
    return {"Time":t_e-t_s,
            "Memory":mem,
            "Comparisons":comp,
            "Swaps":swap,
            "Iterations":itr}
\end{minted}
\subsubsection*{Graphs}
%% Creator: Matplotlib, PGF backend
%%
%% To include the figure in your LaTeX document, write
%%   \input{<filename>.pgf}
%%
%% Make sure the required packages are loaded in your preamble
%%   \usepackage{pgf}
%%
%% Also ensure that all the required font packages are loaded; for instance,
%% the lmodern package is sometimes necessary when using math font.
%%   \usepackage{lmodern}
%%
%% Figures using additional raster images can only be included by \input if
%% they are in the same directory as the main LaTeX file. For loading figures
%% from other directories you can use the `import` package
%%   \usepackage{import}
%%
%% and then include the figures with
%%   \import{<path to file>}{<filename>.pgf}
%%
%% Matplotlib used the following preamble
%%   \usepackage{fontspec}
%%   \setmainfont{DejaVuSerif.ttf}[Path=\detokenize{/home/dbk/.local/lib/python3.10/site-packages/matplotlib/mpl-data/fonts/ttf/}]
%%   \setsansfont{DejaVuSans.ttf}[Path=\detokenize{/home/dbk/.local/lib/python3.10/site-packages/matplotlib/mpl-data/fonts/ttf/}]
%%   \setmonofont{DejaVuSansMono.ttf}[Path=\detokenize{/home/dbk/.local/lib/python3.10/site-packages/matplotlib/mpl-data/fonts/ttf/}]
%%
\begingroup%
\makeatletter%
\begin{pgfpicture}%
\pgfpathrectangle{\pgfpointorigin}{\pgfqpoint{6.400000in}{4.800000in}}%
\pgfusepath{use as bounding box, clip}%
\begin{pgfscope}%
\pgfsetbuttcap%
\pgfsetmiterjoin%
\definecolor{currentfill}{rgb}{1.000000,1.000000,1.000000}%
\pgfsetfillcolor{currentfill}%
\pgfsetlinewidth{0.000000pt}%
\definecolor{currentstroke}{rgb}{1.000000,1.000000,1.000000}%
\pgfsetstrokecolor{currentstroke}%
\pgfsetdash{}{0pt}%
\pgfpathmoveto{\pgfqpoint{0.000000in}{0.000000in}}%
\pgfpathlineto{\pgfqpoint{6.400000in}{0.000000in}}%
\pgfpathlineto{\pgfqpoint{6.400000in}{4.800000in}}%
\pgfpathlineto{\pgfqpoint{0.000000in}{4.800000in}}%
\pgfpathlineto{\pgfqpoint{0.000000in}{0.000000in}}%
\pgfpathclose%
\pgfusepath{fill}%
\end{pgfscope}%
\begin{pgfscope}%
\pgfsetbuttcap%
\pgfsetmiterjoin%
\definecolor{currentfill}{rgb}{1.000000,1.000000,1.000000}%
\pgfsetfillcolor{currentfill}%
\pgfsetlinewidth{0.000000pt}%
\definecolor{currentstroke}{rgb}{0.000000,0.000000,0.000000}%
\pgfsetstrokecolor{currentstroke}%
\pgfsetstrokeopacity{0.000000}%
\pgfsetdash{}{0pt}%
\pgfpathmoveto{\pgfqpoint{0.800000in}{0.528000in}}%
\pgfpathlineto{\pgfqpoint{5.760000in}{0.528000in}}%
\pgfpathlineto{\pgfqpoint{5.760000in}{4.224000in}}%
\pgfpathlineto{\pgfqpoint{0.800000in}{4.224000in}}%
\pgfpathlineto{\pgfqpoint{0.800000in}{0.528000in}}%
\pgfpathclose%
\pgfusepath{fill}%
\end{pgfscope}%
\begin{pgfscope}%
\pgfsetbuttcap%
\pgfsetroundjoin%
\definecolor{currentfill}{rgb}{0.000000,0.000000,0.000000}%
\pgfsetfillcolor{currentfill}%
\pgfsetlinewidth{0.803000pt}%
\definecolor{currentstroke}{rgb}{0.000000,0.000000,0.000000}%
\pgfsetstrokecolor{currentstroke}%
\pgfsetdash{}{0pt}%
\pgfsys@defobject{currentmarker}{\pgfqpoint{0.000000in}{-0.048611in}}{\pgfqpoint{0.000000in}{0.000000in}}{%
\pgfpathmoveto{\pgfqpoint{0.000000in}{0.000000in}}%
\pgfpathlineto{\pgfqpoint{0.000000in}{-0.048611in}}%
\pgfusepath{stroke,fill}%
}%
\begin{pgfscope}%
\pgfsys@transformshift{1.020945in}{0.528000in}%
\pgfsys@useobject{currentmarker}{}%
\end{pgfscope}%
\end{pgfscope}%
\begin{pgfscope}%
\definecolor{textcolor}{rgb}{0.000000,0.000000,0.000000}%
\pgfsetstrokecolor{textcolor}%
\pgfsetfillcolor{textcolor}%
\pgftext[x=1.020945in,y=0.430778in,,top]{\color{textcolor}\ttfamily\fontsize{10.000000}{12.000000}\selectfont 0}%
\end{pgfscope}%
\begin{pgfscope}%
\pgfsetbuttcap%
\pgfsetroundjoin%
\definecolor{currentfill}{rgb}{0.000000,0.000000,0.000000}%
\pgfsetfillcolor{currentfill}%
\pgfsetlinewidth{0.803000pt}%
\definecolor{currentstroke}{rgb}{0.000000,0.000000,0.000000}%
\pgfsetstrokecolor{currentstroke}%
\pgfsetdash{}{0pt}%
\pgfsys@defobject{currentmarker}{\pgfqpoint{0.000000in}{-0.048611in}}{\pgfqpoint{0.000000in}{0.000000in}}{%
\pgfpathmoveto{\pgfqpoint{0.000000in}{0.000000in}}%
\pgfpathlineto{\pgfqpoint{0.000000in}{-0.048611in}}%
\pgfusepath{stroke,fill}%
}%
\begin{pgfscope}%
\pgfsys@transformshift{1.922764in}{0.528000in}%
\pgfsys@useobject{currentmarker}{}%
\end{pgfscope}%
\end{pgfscope}%
\begin{pgfscope}%
\definecolor{textcolor}{rgb}{0.000000,0.000000,0.000000}%
\pgfsetstrokecolor{textcolor}%
\pgfsetfillcolor{textcolor}%
\pgftext[x=1.922764in,y=0.430778in,,top]{\color{textcolor}\ttfamily\fontsize{10.000000}{12.000000}\selectfont 200}%
\end{pgfscope}%
\begin{pgfscope}%
\pgfsetbuttcap%
\pgfsetroundjoin%
\definecolor{currentfill}{rgb}{0.000000,0.000000,0.000000}%
\pgfsetfillcolor{currentfill}%
\pgfsetlinewidth{0.803000pt}%
\definecolor{currentstroke}{rgb}{0.000000,0.000000,0.000000}%
\pgfsetstrokecolor{currentstroke}%
\pgfsetdash{}{0pt}%
\pgfsys@defobject{currentmarker}{\pgfqpoint{0.000000in}{-0.048611in}}{\pgfqpoint{0.000000in}{0.000000in}}{%
\pgfpathmoveto{\pgfqpoint{0.000000in}{0.000000in}}%
\pgfpathlineto{\pgfqpoint{0.000000in}{-0.048611in}}%
\pgfusepath{stroke,fill}%
}%
\begin{pgfscope}%
\pgfsys@transformshift{2.824582in}{0.528000in}%
\pgfsys@useobject{currentmarker}{}%
\end{pgfscope}%
\end{pgfscope}%
\begin{pgfscope}%
\definecolor{textcolor}{rgb}{0.000000,0.000000,0.000000}%
\pgfsetstrokecolor{textcolor}%
\pgfsetfillcolor{textcolor}%
\pgftext[x=2.824582in,y=0.430778in,,top]{\color{textcolor}\ttfamily\fontsize{10.000000}{12.000000}\selectfont 400}%
\end{pgfscope}%
\begin{pgfscope}%
\pgfsetbuttcap%
\pgfsetroundjoin%
\definecolor{currentfill}{rgb}{0.000000,0.000000,0.000000}%
\pgfsetfillcolor{currentfill}%
\pgfsetlinewidth{0.803000pt}%
\definecolor{currentstroke}{rgb}{0.000000,0.000000,0.000000}%
\pgfsetstrokecolor{currentstroke}%
\pgfsetdash{}{0pt}%
\pgfsys@defobject{currentmarker}{\pgfqpoint{0.000000in}{-0.048611in}}{\pgfqpoint{0.000000in}{0.000000in}}{%
\pgfpathmoveto{\pgfqpoint{0.000000in}{0.000000in}}%
\pgfpathlineto{\pgfqpoint{0.000000in}{-0.048611in}}%
\pgfusepath{stroke,fill}%
}%
\begin{pgfscope}%
\pgfsys@transformshift{3.726400in}{0.528000in}%
\pgfsys@useobject{currentmarker}{}%
\end{pgfscope}%
\end{pgfscope}%
\begin{pgfscope}%
\definecolor{textcolor}{rgb}{0.000000,0.000000,0.000000}%
\pgfsetstrokecolor{textcolor}%
\pgfsetfillcolor{textcolor}%
\pgftext[x=3.726400in,y=0.430778in,,top]{\color{textcolor}\ttfamily\fontsize{10.000000}{12.000000}\selectfont 600}%
\end{pgfscope}%
\begin{pgfscope}%
\pgfsetbuttcap%
\pgfsetroundjoin%
\definecolor{currentfill}{rgb}{0.000000,0.000000,0.000000}%
\pgfsetfillcolor{currentfill}%
\pgfsetlinewidth{0.803000pt}%
\definecolor{currentstroke}{rgb}{0.000000,0.000000,0.000000}%
\pgfsetstrokecolor{currentstroke}%
\pgfsetdash{}{0pt}%
\pgfsys@defobject{currentmarker}{\pgfqpoint{0.000000in}{-0.048611in}}{\pgfqpoint{0.000000in}{0.000000in}}{%
\pgfpathmoveto{\pgfqpoint{0.000000in}{0.000000in}}%
\pgfpathlineto{\pgfqpoint{0.000000in}{-0.048611in}}%
\pgfusepath{stroke,fill}%
}%
\begin{pgfscope}%
\pgfsys@transformshift{4.628218in}{0.528000in}%
\pgfsys@useobject{currentmarker}{}%
\end{pgfscope}%
\end{pgfscope}%
\begin{pgfscope}%
\definecolor{textcolor}{rgb}{0.000000,0.000000,0.000000}%
\pgfsetstrokecolor{textcolor}%
\pgfsetfillcolor{textcolor}%
\pgftext[x=4.628218in,y=0.430778in,,top]{\color{textcolor}\ttfamily\fontsize{10.000000}{12.000000}\selectfont 800}%
\end{pgfscope}%
\begin{pgfscope}%
\pgfsetbuttcap%
\pgfsetroundjoin%
\definecolor{currentfill}{rgb}{0.000000,0.000000,0.000000}%
\pgfsetfillcolor{currentfill}%
\pgfsetlinewidth{0.803000pt}%
\definecolor{currentstroke}{rgb}{0.000000,0.000000,0.000000}%
\pgfsetstrokecolor{currentstroke}%
\pgfsetdash{}{0pt}%
\pgfsys@defobject{currentmarker}{\pgfqpoint{0.000000in}{-0.048611in}}{\pgfqpoint{0.000000in}{0.000000in}}{%
\pgfpathmoveto{\pgfqpoint{0.000000in}{0.000000in}}%
\pgfpathlineto{\pgfqpoint{0.000000in}{-0.048611in}}%
\pgfusepath{stroke,fill}%
}%
\begin{pgfscope}%
\pgfsys@transformshift{5.530036in}{0.528000in}%
\pgfsys@useobject{currentmarker}{}%
\end{pgfscope}%
\end{pgfscope}%
\begin{pgfscope}%
\definecolor{textcolor}{rgb}{0.000000,0.000000,0.000000}%
\pgfsetstrokecolor{textcolor}%
\pgfsetfillcolor{textcolor}%
\pgftext[x=5.530036in,y=0.430778in,,top]{\color{textcolor}\ttfamily\fontsize{10.000000}{12.000000}\selectfont 1000}%
\end{pgfscope}%
\begin{pgfscope}%
\definecolor{textcolor}{rgb}{0.000000,0.000000,0.000000}%
\pgfsetstrokecolor{textcolor}%
\pgfsetfillcolor{textcolor}%
\pgftext[x=3.280000in,y=0.240063in,,top]{\color{textcolor}\ttfamily\fontsize{10.000000}{12.000000}\selectfont Size of Array}%
\end{pgfscope}%
\begin{pgfscope}%
\pgfsetbuttcap%
\pgfsetroundjoin%
\definecolor{currentfill}{rgb}{0.000000,0.000000,0.000000}%
\pgfsetfillcolor{currentfill}%
\pgfsetlinewidth{0.803000pt}%
\definecolor{currentstroke}{rgb}{0.000000,0.000000,0.000000}%
\pgfsetstrokecolor{currentstroke}%
\pgfsetdash{}{0pt}%
\pgfsys@defobject{currentmarker}{\pgfqpoint{-0.048611in}{0.000000in}}{\pgfqpoint{-0.000000in}{0.000000in}}{%
\pgfpathmoveto{\pgfqpoint{-0.000000in}{0.000000in}}%
\pgfpathlineto{\pgfqpoint{-0.048611in}{0.000000in}}%
\pgfusepath{stroke,fill}%
}%
\begin{pgfscope}%
\pgfsys@transformshift{0.800000in}{0.688042in}%
\pgfsys@useobject{currentmarker}{}%
\end{pgfscope}%
\end{pgfscope}%
\begin{pgfscope}%
\definecolor{textcolor}{rgb}{0.000000,0.000000,0.000000}%
\pgfsetstrokecolor{textcolor}%
\pgfsetfillcolor{textcolor}%
\pgftext[x=0.619160in, y=0.634908in, left, base]{\color{textcolor}\ttfamily\fontsize{10.000000}{12.000000}\selectfont 0}%
\end{pgfscope}%
\begin{pgfscope}%
\pgfsetbuttcap%
\pgfsetroundjoin%
\definecolor{currentfill}{rgb}{0.000000,0.000000,0.000000}%
\pgfsetfillcolor{currentfill}%
\pgfsetlinewidth{0.803000pt}%
\definecolor{currentstroke}{rgb}{0.000000,0.000000,0.000000}%
\pgfsetstrokecolor{currentstroke}%
\pgfsetdash{}{0pt}%
\pgfsys@defobject{currentmarker}{\pgfqpoint{-0.048611in}{0.000000in}}{\pgfqpoint{-0.000000in}{0.000000in}}{%
\pgfpathmoveto{\pgfqpoint{-0.000000in}{0.000000in}}%
\pgfpathlineto{\pgfqpoint{-0.048611in}{0.000000in}}%
\pgfusepath{stroke,fill}%
}%
\begin{pgfscope}%
\pgfsys@transformshift{0.800000in}{1.135451in}%
\pgfsys@useobject{currentmarker}{}%
\end{pgfscope}%
\end{pgfscope}%
\begin{pgfscope}%
\definecolor{textcolor}{rgb}{0.000000,0.000000,0.000000}%
\pgfsetstrokecolor{textcolor}%
\pgfsetfillcolor{textcolor}%
\pgftext[x=0.619160in, y=1.082317in, left, base]{\color{textcolor}\ttfamily\fontsize{10.000000}{12.000000}\selectfont 1}%
\end{pgfscope}%
\begin{pgfscope}%
\pgfsetbuttcap%
\pgfsetroundjoin%
\definecolor{currentfill}{rgb}{0.000000,0.000000,0.000000}%
\pgfsetfillcolor{currentfill}%
\pgfsetlinewidth{0.803000pt}%
\definecolor{currentstroke}{rgb}{0.000000,0.000000,0.000000}%
\pgfsetstrokecolor{currentstroke}%
\pgfsetdash{}{0pt}%
\pgfsys@defobject{currentmarker}{\pgfqpoint{-0.048611in}{0.000000in}}{\pgfqpoint{-0.000000in}{0.000000in}}{%
\pgfpathmoveto{\pgfqpoint{-0.000000in}{0.000000in}}%
\pgfpathlineto{\pgfqpoint{-0.048611in}{0.000000in}}%
\pgfusepath{stroke,fill}%
}%
\begin{pgfscope}%
\pgfsys@transformshift{0.800000in}{1.582860in}%
\pgfsys@useobject{currentmarker}{}%
\end{pgfscope}%
\end{pgfscope}%
\begin{pgfscope}%
\definecolor{textcolor}{rgb}{0.000000,0.000000,0.000000}%
\pgfsetstrokecolor{textcolor}%
\pgfsetfillcolor{textcolor}%
\pgftext[x=0.619160in, y=1.529726in, left, base]{\color{textcolor}\ttfamily\fontsize{10.000000}{12.000000}\selectfont 2}%
\end{pgfscope}%
\begin{pgfscope}%
\pgfsetbuttcap%
\pgfsetroundjoin%
\definecolor{currentfill}{rgb}{0.000000,0.000000,0.000000}%
\pgfsetfillcolor{currentfill}%
\pgfsetlinewidth{0.803000pt}%
\definecolor{currentstroke}{rgb}{0.000000,0.000000,0.000000}%
\pgfsetstrokecolor{currentstroke}%
\pgfsetdash{}{0pt}%
\pgfsys@defobject{currentmarker}{\pgfqpoint{-0.048611in}{0.000000in}}{\pgfqpoint{-0.000000in}{0.000000in}}{%
\pgfpathmoveto{\pgfqpoint{-0.000000in}{0.000000in}}%
\pgfpathlineto{\pgfqpoint{-0.048611in}{0.000000in}}%
\pgfusepath{stroke,fill}%
}%
\begin{pgfscope}%
\pgfsys@transformshift{0.800000in}{2.030269in}%
\pgfsys@useobject{currentmarker}{}%
\end{pgfscope}%
\end{pgfscope}%
\begin{pgfscope}%
\definecolor{textcolor}{rgb}{0.000000,0.000000,0.000000}%
\pgfsetstrokecolor{textcolor}%
\pgfsetfillcolor{textcolor}%
\pgftext[x=0.619160in, y=1.977135in, left, base]{\color{textcolor}\ttfamily\fontsize{10.000000}{12.000000}\selectfont 3}%
\end{pgfscope}%
\begin{pgfscope}%
\pgfsetbuttcap%
\pgfsetroundjoin%
\definecolor{currentfill}{rgb}{0.000000,0.000000,0.000000}%
\pgfsetfillcolor{currentfill}%
\pgfsetlinewidth{0.803000pt}%
\definecolor{currentstroke}{rgb}{0.000000,0.000000,0.000000}%
\pgfsetstrokecolor{currentstroke}%
\pgfsetdash{}{0pt}%
\pgfsys@defobject{currentmarker}{\pgfqpoint{-0.048611in}{0.000000in}}{\pgfqpoint{-0.000000in}{0.000000in}}{%
\pgfpathmoveto{\pgfqpoint{-0.000000in}{0.000000in}}%
\pgfpathlineto{\pgfqpoint{-0.048611in}{0.000000in}}%
\pgfusepath{stroke,fill}%
}%
\begin{pgfscope}%
\pgfsys@transformshift{0.800000in}{2.477678in}%
\pgfsys@useobject{currentmarker}{}%
\end{pgfscope}%
\end{pgfscope}%
\begin{pgfscope}%
\definecolor{textcolor}{rgb}{0.000000,0.000000,0.000000}%
\pgfsetstrokecolor{textcolor}%
\pgfsetfillcolor{textcolor}%
\pgftext[x=0.619160in, y=2.424544in, left, base]{\color{textcolor}\ttfamily\fontsize{10.000000}{12.000000}\selectfont 4}%
\end{pgfscope}%
\begin{pgfscope}%
\pgfsetbuttcap%
\pgfsetroundjoin%
\definecolor{currentfill}{rgb}{0.000000,0.000000,0.000000}%
\pgfsetfillcolor{currentfill}%
\pgfsetlinewidth{0.803000pt}%
\definecolor{currentstroke}{rgb}{0.000000,0.000000,0.000000}%
\pgfsetstrokecolor{currentstroke}%
\pgfsetdash{}{0pt}%
\pgfsys@defobject{currentmarker}{\pgfqpoint{-0.048611in}{0.000000in}}{\pgfqpoint{-0.000000in}{0.000000in}}{%
\pgfpathmoveto{\pgfqpoint{-0.000000in}{0.000000in}}%
\pgfpathlineto{\pgfqpoint{-0.048611in}{0.000000in}}%
\pgfusepath{stroke,fill}%
}%
\begin{pgfscope}%
\pgfsys@transformshift{0.800000in}{2.925087in}%
\pgfsys@useobject{currentmarker}{}%
\end{pgfscope}%
\end{pgfscope}%
\begin{pgfscope}%
\definecolor{textcolor}{rgb}{0.000000,0.000000,0.000000}%
\pgfsetstrokecolor{textcolor}%
\pgfsetfillcolor{textcolor}%
\pgftext[x=0.619160in, y=2.871953in, left, base]{\color{textcolor}\ttfamily\fontsize{10.000000}{12.000000}\selectfont 5}%
\end{pgfscope}%
\begin{pgfscope}%
\pgfsetbuttcap%
\pgfsetroundjoin%
\definecolor{currentfill}{rgb}{0.000000,0.000000,0.000000}%
\pgfsetfillcolor{currentfill}%
\pgfsetlinewidth{0.803000pt}%
\definecolor{currentstroke}{rgb}{0.000000,0.000000,0.000000}%
\pgfsetstrokecolor{currentstroke}%
\pgfsetdash{}{0pt}%
\pgfsys@defobject{currentmarker}{\pgfqpoint{-0.048611in}{0.000000in}}{\pgfqpoint{-0.000000in}{0.000000in}}{%
\pgfpathmoveto{\pgfqpoint{-0.000000in}{0.000000in}}%
\pgfpathlineto{\pgfqpoint{-0.048611in}{0.000000in}}%
\pgfusepath{stroke,fill}%
}%
\begin{pgfscope}%
\pgfsys@transformshift{0.800000in}{3.372496in}%
\pgfsys@useobject{currentmarker}{}%
\end{pgfscope}%
\end{pgfscope}%
\begin{pgfscope}%
\definecolor{textcolor}{rgb}{0.000000,0.000000,0.000000}%
\pgfsetstrokecolor{textcolor}%
\pgfsetfillcolor{textcolor}%
\pgftext[x=0.619160in, y=3.319362in, left, base]{\color{textcolor}\ttfamily\fontsize{10.000000}{12.000000}\selectfont 6}%
\end{pgfscope}%
\begin{pgfscope}%
\pgfsetbuttcap%
\pgfsetroundjoin%
\definecolor{currentfill}{rgb}{0.000000,0.000000,0.000000}%
\pgfsetfillcolor{currentfill}%
\pgfsetlinewidth{0.803000pt}%
\definecolor{currentstroke}{rgb}{0.000000,0.000000,0.000000}%
\pgfsetstrokecolor{currentstroke}%
\pgfsetdash{}{0pt}%
\pgfsys@defobject{currentmarker}{\pgfqpoint{-0.048611in}{0.000000in}}{\pgfqpoint{-0.000000in}{0.000000in}}{%
\pgfpathmoveto{\pgfqpoint{-0.000000in}{0.000000in}}%
\pgfpathlineto{\pgfqpoint{-0.048611in}{0.000000in}}%
\pgfusepath{stroke,fill}%
}%
\begin{pgfscope}%
\pgfsys@transformshift{0.800000in}{3.819905in}%
\pgfsys@useobject{currentmarker}{}%
\end{pgfscope}%
\end{pgfscope}%
\begin{pgfscope}%
\definecolor{textcolor}{rgb}{0.000000,0.000000,0.000000}%
\pgfsetstrokecolor{textcolor}%
\pgfsetfillcolor{textcolor}%
\pgftext[x=0.619160in, y=3.766771in, left, base]{\color{textcolor}\ttfamily\fontsize{10.000000}{12.000000}\selectfont 7}%
\end{pgfscope}%
\begin{pgfscope}%
\definecolor{textcolor}{rgb}{0.000000,0.000000,0.000000}%
\pgfsetstrokecolor{textcolor}%
\pgfsetfillcolor{textcolor}%
\pgftext[x=0.563604in,y=2.376000in,,bottom,rotate=90.000000]{\color{textcolor}\ttfamily\fontsize{10.000000}{12.000000}\selectfont Time}%
\end{pgfscope}%
\begin{pgfscope}%
\definecolor{textcolor}{rgb}{0.000000,0.000000,0.000000}%
\pgfsetstrokecolor{textcolor}%
\pgfsetfillcolor{textcolor}%
\pgftext[x=0.800000in,y=4.265667in,left,base]{\color{textcolor}\ttfamily\fontsize{10.000000}{12.000000}\selectfont 1e8}%
\end{pgfscope}%
\begin{pgfscope}%
\pgfpathrectangle{\pgfqpoint{0.800000in}{0.528000in}}{\pgfqpoint{4.960000in}{3.696000in}}%
\pgfusepath{clip}%
\pgfsetrectcap%
\pgfsetroundjoin%
\pgfsetlinewidth{1.505625pt}%
\definecolor{currentstroke}{rgb}{0.000000,1.000000,0.498039}%
\pgfsetstrokecolor{currentstroke}%
\pgfsetdash{}{0pt}%
\pgfpathmoveto{\pgfqpoint{1.025455in}{0.696000in}}%
\pgfpathlineto{\pgfqpoint{1.034473in}{0.700715in}}%
\pgfpathlineto{\pgfqpoint{1.038982in}{0.699393in}}%
\pgfpathlineto{\pgfqpoint{1.043491in}{0.700031in}}%
\pgfpathlineto{\pgfqpoint{1.048000in}{0.703649in}}%
\pgfpathlineto{\pgfqpoint{1.057018in}{0.700655in}}%
\pgfpathlineto{\pgfqpoint{1.070545in}{0.704685in}}%
\pgfpathlineto{\pgfqpoint{1.075055in}{0.702088in}}%
\pgfpathlineto{\pgfqpoint{1.084073in}{0.704407in}}%
\pgfpathlineto{\pgfqpoint{1.088582in}{0.703123in}}%
\pgfpathlineto{\pgfqpoint{1.093091in}{0.704214in}}%
\pgfpathlineto{\pgfqpoint{1.097600in}{0.701760in}}%
\pgfpathlineto{\pgfqpoint{1.102109in}{0.703273in}}%
\pgfpathlineto{\pgfqpoint{1.106618in}{0.707078in}}%
\pgfpathlineto{\pgfqpoint{1.111127in}{0.704159in}}%
\pgfpathlineto{\pgfqpoint{1.115636in}{0.706825in}}%
\pgfpathlineto{\pgfqpoint{1.120145in}{0.706310in}}%
\pgfpathlineto{\pgfqpoint{1.124655in}{0.709793in}}%
\pgfpathlineto{\pgfqpoint{1.129164in}{0.705381in}}%
\pgfpathlineto{\pgfqpoint{1.133673in}{0.703746in}}%
\pgfpathlineto{\pgfqpoint{1.138182in}{0.708028in}}%
\pgfpathlineto{\pgfqpoint{1.142691in}{0.705165in}}%
\pgfpathlineto{\pgfqpoint{1.147200in}{0.710681in}}%
\pgfpathlineto{\pgfqpoint{1.151709in}{0.704697in}}%
\pgfpathlineto{\pgfqpoint{1.156218in}{0.710068in}}%
\pgfpathlineto{\pgfqpoint{1.160727in}{0.707395in}}%
\pgfpathlineto{\pgfqpoint{1.165236in}{0.709127in}}%
\pgfpathlineto{\pgfqpoint{1.169745in}{0.708609in}}%
\pgfpathlineto{\pgfqpoint{1.174255in}{0.713990in}}%
\pgfpathlineto{\pgfqpoint{1.178764in}{0.708402in}}%
\pgfpathlineto{\pgfqpoint{1.183273in}{0.709064in}}%
\pgfpathlineto{\pgfqpoint{1.187782in}{0.706635in}}%
\pgfpathlineto{\pgfqpoint{1.192291in}{0.707467in}}%
\pgfpathlineto{\pgfqpoint{1.196800in}{0.711349in}}%
\pgfpathlineto{\pgfqpoint{1.201309in}{0.708449in}}%
\pgfpathlineto{\pgfqpoint{1.210327in}{0.709888in}}%
\pgfpathlineto{\pgfqpoint{1.214836in}{0.716207in}}%
\pgfpathlineto{\pgfqpoint{1.219345in}{0.718001in}}%
\pgfpathlineto{\pgfqpoint{1.223855in}{0.712487in}}%
\pgfpathlineto{\pgfqpoint{1.228364in}{0.710517in}}%
\pgfpathlineto{\pgfqpoint{1.232873in}{0.713033in}}%
\pgfpathlineto{\pgfqpoint{1.237382in}{0.724951in}}%
\pgfpathlineto{\pgfqpoint{1.241891in}{0.727155in}}%
\pgfpathlineto{\pgfqpoint{1.246400in}{0.717064in}}%
\pgfpathlineto{\pgfqpoint{1.250909in}{0.713047in}}%
\pgfpathlineto{\pgfqpoint{1.255418in}{0.717368in}}%
\pgfpathlineto{\pgfqpoint{1.259927in}{0.714409in}}%
\pgfpathlineto{\pgfqpoint{1.264436in}{0.718110in}}%
\pgfpathlineto{\pgfqpoint{1.268945in}{0.715057in}}%
\pgfpathlineto{\pgfqpoint{1.273455in}{0.714347in}}%
\pgfpathlineto{\pgfqpoint{1.277964in}{0.720859in}}%
\pgfpathlineto{\pgfqpoint{1.282473in}{0.716097in}}%
\pgfpathlineto{\pgfqpoint{1.291491in}{0.716270in}}%
\pgfpathlineto{\pgfqpoint{1.296000in}{0.721209in}}%
\pgfpathlineto{\pgfqpoint{1.305018in}{0.713876in}}%
\pgfpathlineto{\pgfqpoint{1.309527in}{0.715547in}}%
\pgfpathlineto{\pgfqpoint{1.314036in}{0.718403in}}%
\pgfpathlineto{\pgfqpoint{1.323055in}{0.719782in}}%
\pgfpathlineto{\pgfqpoint{1.327564in}{0.715896in}}%
\pgfpathlineto{\pgfqpoint{1.332073in}{0.721227in}}%
\pgfpathlineto{\pgfqpoint{1.336582in}{0.723095in}}%
\pgfpathlineto{\pgfqpoint{1.341091in}{0.746703in}}%
\pgfpathlineto{\pgfqpoint{1.345600in}{0.745457in}}%
\pgfpathlineto{\pgfqpoint{1.350109in}{0.748399in}}%
\pgfpathlineto{\pgfqpoint{1.354618in}{0.747296in}}%
\pgfpathlineto{\pgfqpoint{1.359127in}{0.757134in}}%
\pgfpathlineto{\pgfqpoint{1.363636in}{0.754107in}}%
\pgfpathlineto{\pgfqpoint{1.368145in}{0.768874in}}%
\pgfpathlineto{\pgfqpoint{1.372655in}{0.737633in}}%
\pgfpathlineto{\pgfqpoint{1.381673in}{0.727886in}}%
\pgfpathlineto{\pgfqpoint{1.386182in}{0.747999in}}%
\pgfpathlineto{\pgfqpoint{1.390691in}{0.738904in}}%
\pgfpathlineto{\pgfqpoint{1.395200in}{0.780317in}}%
\pgfpathlineto{\pgfqpoint{1.399709in}{0.762901in}}%
\pgfpathlineto{\pgfqpoint{1.404218in}{0.739676in}}%
\pgfpathlineto{\pgfqpoint{1.408727in}{0.727415in}}%
\pgfpathlineto{\pgfqpoint{1.413236in}{0.732649in}}%
\pgfpathlineto{\pgfqpoint{1.417745in}{0.727913in}}%
\pgfpathlineto{\pgfqpoint{1.422255in}{0.729727in}}%
\pgfpathlineto{\pgfqpoint{1.431273in}{0.726696in}}%
\pgfpathlineto{\pgfqpoint{1.444800in}{0.729184in}}%
\pgfpathlineto{\pgfqpoint{1.449309in}{0.733063in}}%
\pgfpathlineto{\pgfqpoint{1.458327in}{0.731186in}}%
\pgfpathlineto{\pgfqpoint{1.462836in}{0.729408in}}%
\pgfpathlineto{\pgfqpoint{1.471855in}{0.731009in}}%
\pgfpathlineto{\pgfqpoint{1.476364in}{0.737727in}}%
\pgfpathlineto{\pgfqpoint{1.480873in}{0.732417in}}%
\pgfpathlineto{\pgfqpoint{1.485382in}{0.732782in}}%
\pgfpathlineto{\pgfqpoint{1.489891in}{0.735286in}}%
\pgfpathlineto{\pgfqpoint{1.494400in}{0.732863in}}%
\pgfpathlineto{\pgfqpoint{1.498909in}{0.737848in}}%
\pgfpathlineto{\pgfqpoint{1.503418in}{0.748608in}}%
\pgfpathlineto{\pgfqpoint{1.507927in}{0.788037in}}%
\pgfpathlineto{\pgfqpoint{1.512436in}{0.811252in}}%
\pgfpathlineto{\pgfqpoint{1.516945in}{0.734430in}}%
\pgfpathlineto{\pgfqpoint{1.521455in}{0.736220in}}%
\pgfpathlineto{\pgfqpoint{1.530473in}{0.737191in}}%
\pgfpathlineto{\pgfqpoint{1.534982in}{0.740435in}}%
\pgfpathlineto{\pgfqpoint{1.539491in}{0.740035in}}%
\pgfpathlineto{\pgfqpoint{1.544000in}{0.748525in}}%
\pgfpathlineto{\pgfqpoint{1.548509in}{0.746835in}}%
\pgfpathlineto{\pgfqpoint{1.553018in}{0.752216in}}%
\pgfpathlineto{\pgfqpoint{1.557527in}{0.749082in}}%
\pgfpathlineto{\pgfqpoint{1.562036in}{0.742053in}}%
\pgfpathlineto{\pgfqpoint{1.566545in}{0.741669in}}%
\pgfpathlineto{\pgfqpoint{1.571055in}{0.746764in}}%
\pgfpathlineto{\pgfqpoint{1.575564in}{0.747915in}}%
\pgfpathlineto{\pgfqpoint{1.584582in}{0.758299in}}%
\pgfpathlineto{\pgfqpoint{1.589091in}{0.775194in}}%
\pgfpathlineto{\pgfqpoint{1.593600in}{0.819554in}}%
\pgfpathlineto{\pgfqpoint{1.598109in}{0.760376in}}%
\pgfpathlineto{\pgfqpoint{1.602618in}{0.773928in}}%
\pgfpathlineto{\pgfqpoint{1.607127in}{0.774026in}}%
\pgfpathlineto{\pgfqpoint{1.611636in}{0.796847in}}%
\pgfpathlineto{\pgfqpoint{1.616145in}{0.798147in}}%
\pgfpathlineto{\pgfqpoint{1.620655in}{0.838989in}}%
\pgfpathlineto{\pgfqpoint{1.625164in}{0.774434in}}%
\pgfpathlineto{\pgfqpoint{1.629673in}{0.763404in}}%
\pgfpathlineto{\pgfqpoint{1.634182in}{0.782479in}}%
\pgfpathlineto{\pgfqpoint{1.638691in}{0.750451in}}%
\pgfpathlineto{\pgfqpoint{1.643200in}{0.750534in}}%
\pgfpathlineto{\pgfqpoint{1.652218in}{0.752551in}}%
\pgfpathlineto{\pgfqpoint{1.656727in}{0.758969in}}%
\pgfpathlineto{\pgfqpoint{1.661236in}{0.755167in}}%
\pgfpathlineto{\pgfqpoint{1.665745in}{0.756973in}}%
\pgfpathlineto{\pgfqpoint{1.670255in}{0.820331in}}%
\pgfpathlineto{\pgfqpoint{1.674764in}{0.825169in}}%
\pgfpathlineto{\pgfqpoint{1.679273in}{0.836619in}}%
\pgfpathlineto{\pgfqpoint{1.683782in}{0.761360in}}%
\pgfpathlineto{\pgfqpoint{1.688291in}{0.753349in}}%
\pgfpathlineto{\pgfqpoint{1.697309in}{0.765024in}}%
\pgfpathlineto{\pgfqpoint{1.701818in}{0.787637in}}%
\pgfpathlineto{\pgfqpoint{1.706327in}{0.787358in}}%
\pgfpathlineto{\pgfqpoint{1.710836in}{0.764996in}}%
\pgfpathlineto{\pgfqpoint{1.715345in}{0.782660in}}%
\pgfpathlineto{\pgfqpoint{1.719855in}{0.832655in}}%
\pgfpathlineto{\pgfqpoint{1.724364in}{0.777863in}}%
\pgfpathlineto{\pgfqpoint{1.728873in}{0.766589in}}%
\pgfpathlineto{\pgfqpoint{1.733382in}{0.762491in}}%
\pgfpathlineto{\pgfqpoint{1.737891in}{0.773304in}}%
\pgfpathlineto{\pgfqpoint{1.742400in}{0.772862in}}%
\pgfpathlineto{\pgfqpoint{1.746909in}{0.767881in}}%
\pgfpathlineto{\pgfqpoint{1.751418in}{0.831476in}}%
\pgfpathlineto{\pgfqpoint{1.755927in}{0.783001in}}%
\pgfpathlineto{\pgfqpoint{1.760436in}{0.777673in}}%
\pgfpathlineto{\pgfqpoint{1.764945in}{0.784779in}}%
\pgfpathlineto{\pgfqpoint{1.769455in}{0.771805in}}%
\pgfpathlineto{\pgfqpoint{1.773964in}{0.774720in}}%
\pgfpathlineto{\pgfqpoint{1.778473in}{0.772714in}}%
\pgfpathlineto{\pgfqpoint{1.782982in}{0.885686in}}%
\pgfpathlineto{\pgfqpoint{1.787491in}{0.820226in}}%
\pgfpathlineto{\pgfqpoint{1.792000in}{0.830893in}}%
\pgfpathlineto{\pgfqpoint{1.796509in}{0.798113in}}%
\pgfpathlineto{\pgfqpoint{1.801018in}{0.799610in}}%
\pgfpathlineto{\pgfqpoint{1.805527in}{0.818849in}}%
\pgfpathlineto{\pgfqpoint{1.810036in}{0.776056in}}%
\pgfpathlineto{\pgfqpoint{1.814545in}{0.779868in}}%
\pgfpathlineto{\pgfqpoint{1.819055in}{0.795079in}}%
\pgfpathlineto{\pgfqpoint{1.823564in}{0.787354in}}%
\pgfpathlineto{\pgfqpoint{1.828073in}{0.790607in}}%
\pgfpathlineto{\pgfqpoint{1.832582in}{0.782793in}}%
\pgfpathlineto{\pgfqpoint{1.837091in}{0.812712in}}%
\pgfpathlineto{\pgfqpoint{1.841600in}{0.780136in}}%
\pgfpathlineto{\pgfqpoint{1.846109in}{0.774126in}}%
\pgfpathlineto{\pgfqpoint{1.850618in}{0.804688in}}%
\pgfpathlineto{\pgfqpoint{1.855127in}{0.868868in}}%
\pgfpathlineto{\pgfqpoint{1.859636in}{0.832192in}}%
\pgfpathlineto{\pgfqpoint{1.864145in}{0.781107in}}%
\pgfpathlineto{\pgfqpoint{1.868655in}{0.780973in}}%
\pgfpathlineto{\pgfqpoint{1.873164in}{0.777754in}}%
\pgfpathlineto{\pgfqpoint{1.877673in}{0.778696in}}%
\pgfpathlineto{\pgfqpoint{1.882182in}{0.802318in}}%
\pgfpathlineto{\pgfqpoint{1.886691in}{0.819773in}}%
\pgfpathlineto{\pgfqpoint{1.891200in}{0.778580in}}%
\pgfpathlineto{\pgfqpoint{1.895709in}{0.782253in}}%
\pgfpathlineto{\pgfqpoint{1.900218in}{0.795925in}}%
\pgfpathlineto{\pgfqpoint{1.904727in}{0.887798in}}%
\pgfpathlineto{\pgfqpoint{1.909236in}{0.795019in}}%
\pgfpathlineto{\pgfqpoint{1.913745in}{0.796211in}}%
\pgfpathlineto{\pgfqpoint{1.918255in}{0.792864in}}%
\pgfpathlineto{\pgfqpoint{1.922764in}{0.787868in}}%
\pgfpathlineto{\pgfqpoint{1.927273in}{0.794831in}}%
\pgfpathlineto{\pgfqpoint{1.931782in}{0.793242in}}%
\pgfpathlineto{\pgfqpoint{1.936291in}{0.794584in}}%
\pgfpathlineto{\pgfqpoint{1.940800in}{0.789197in}}%
\pgfpathlineto{\pgfqpoint{1.945309in}{0.798249in}}%
\pgfpathlineto{\pgfqpoint{1.949818in}{0.798931in}}%
\pgfpathlineto{\pgfqpoint{1.954327in}{0.822109in}}%
\pgfpathlineto{\pgfqpoint{1.958836in}{0.818719in}}%
\pgfpathlineto{\pgfqpoint{1.963345in}{0.803457in}}%
\pgfpathlineto{\pgfqpoint{1.967855in}{0.802506in}}%
\pgfpathlineto{\pgfqpoint{1.972364in}{0.805988in}}%
\pgfpathlineto{\pgfqpoint{1.976873in}{0.811880in}}%
\pgfpathlineto{\pgfqpoint{1.981382in}{0.806557in}}%
\pgfpathlineto{\pgfqpoint{1.985891in}{0.803958in}}%
\pgfpathlineto{\pgfqpoint{1.990400in}{0.842357in}}%
\pgfpathlineto{\pgfqpoint{1.994909in}{0.804875in}}%
\pgfpathlineto{\pgfqpoint{1.999418in}{0.797143in}}%
\pgfpathlineto{\pgfqpoint{2.003927in}{0.802920in}}%
\pgfpathlineto{\pgfqpoint{2.008436in}{0.803706in}}%
\pgfpathlineto{\pgfqpoint{2.012945in}{0.817378in}}%
\pgfpathlineto{\pgfqpoint{2.017455in}{0.815636in}}%
\pgfpathlineto{\pgfqpoint{2.021964in}{0.802648in}}%
\pgfpathlineto{\pgfqpoint{2.026473in}{0.817381in}}%
\pgfpathlineto{\pgfqpoint{2.030982in}{0.818536in}}%
\pgfpathlineto{\pgfqpoint{2.035491in}{0.835865in}}%
\pgfpathlineto{\pgfqpoint{2.040000in}{0.825931in}}%
\pgfpathlineto{\pgfqpoint{2.044509in}{0.858092in}}%
\pgfpathlineto{\pgfqpoint{2.049018in}{0.835074in}}%
\pgfpathlineto{\pgfqpoint{2.053527in}{0.884198in}}%
\pgfpathlineto{\pgfqpoint{2.058036in}{0.819905in}}%
\pgfpathlineto{\pgfqpoint{2.062545in}{0.828814in}}%
\pgfpathlineto{\pgfqpoint{2.067055in}{0.827281in}}%
\pgfpathlineto{\pgfqpoint{2.071564in}{0.820072in}}%
\pgfpathlineto{\pgfqpoint{2.076073in}{0.833518in}}%
\pgfpathlineto{\pgfqpoint{2.080582in}{0.914564in}}%
\pgfpathlineto{\pgfqpoint{2.085091in}{0.920611in}}%
\pgfpathlineto{\pgfqpoint{2.089600in}{0.819432in}}%
\pgfpathlineto{\pgfqpoint{2.094109in}{0.822265in}}%
\pgfpathlineto{\pgfqpoint{2.098618in}{0.823618in}}%
\pgfpathlineto{\pgfqpoint{2.103127in}{0.848084in}}%
\pgfpathlineto{\pgfqpoint{2.107636in}{0.830305in}}%
\pgfpathlineto{\pgfqpoint{2.112145in}{0.845560in}}%
\pgfpathlineto{\pgfqpoint{2.116655in}{0.870800in}}%
\pgfpathlineto{\pgfqpoint{2.121164in}{0.948486in}}%
\pgfpathlineto{\pgfqpoint{2.125673in}{0.868412in}}%
\pgfpathlineto{\pgfqpoint{2.130182in}{0.846471in}}%
\pgfpathlineto{\pgfqpoint{2.134691in}{0.997658in}}%
\pgfpathlineto{\pgfqpoint{2.139200in}{0.926259in}}%
\pgfpathlineto{\pgfqpoint{2.143709in}{0.883005in}}%
\pgfpathlineto{\pgfqpoint{2.148218in}{0.936759in}}%
\pgfpathlineto{\pgfqpoint{2.152727in}{0.826142in}}%
\pgfpathlineto{\pgfqpoint{2.157236in}{0.833489in}}%
\pgfpathlineto{\pgfqpoint{2.161745in}{0.935218in}}%
\pgfpathlineto{\pgfqpoint{2.166255in}{0.851335in}}%
\pgfpathlineto{\pgfqpoint{2.170764in}{0.900600in}}%
\pgfpathlineto{\pgfqpoint{2.175273in}{0.894298in}}%
\pgfpathlineto{\pgfqpoint{2.179782in}{0.974885in}}%
\pgfpathlineto{\pgfqpoint{2.184291in}{0.881712in}}%
\pgfpathlineto{\pgfqpoint{2.188800in}{0.856489in}}%
\pgfpathlineto{\pgfqpoint{2.193309in}{1.008003in}}%
\pgfpathlineto{\pgfqpoint{2.197818in}{0.998305in}}%
\pgfpathlineto{\pgfqpoint{2.202327in}{1.001442in}}%
\pgfpathlineto{\pgfqpoint{2.206836in}{0.902842in}}%
\pgfpathlineto{\pgfqpoint{2.211345in}{0.960522in}}%
\pgfpathlineto{\pgfqpoint{2.215855in}{0.878691in}}%
\pgfpathlineto{\pgfqpoint{2.220364in}{0.852856in}}%
\pgfpathlineto{\pgfqpoint{2.224873in}{0.862347in}}%
\pgfpathlineto{\pgfqpoint{2.229382in}{0.918473in}}%
\pgfpathlineto{\pgfqpoint{2.233891in}{0.856950in}}%
\pgfpathlineto{\pgfqpoint{2.238400in}{0.913593in}}%
\pgfpathlineto{\pgfqpoint{2.242909in}{0.879511in}}%
\pgfpathlineto{\pgfqpoint{2.247418in}{0.869877in}}%
\pgfpathlineto{\pgfqpoint{2.251927in}{0.872563in}}%
\pgfpathlineto{\pgfqpoint{2.256436in}{0.862387in}}%
\pgfpathlineto{\pgfqpoint{2.260945in}{0.865674in}}%
\pgfpathlineto{\pgfqpoint{2.265455in}{0.880886in}}%
\pgfpathlineto{\pgfqpoint{2.269964in}{0.876528in}}%
\pgfpathlineto{\pgfqpoint{2.274473in}{0.886878in}}%
\pgfpathlineto{\pgfqpoint{2.278982in}{0.884676in}}%
\pgfpathlineto{\pgfqpoint{2.283491in}{0.903493in}}%
\pgfpathlineto{\pgfqpoint{2.288000in}{0.871429in}}%
\pgfpathlineto{\pgfqpoint{2.292509in}{0.871138in}}%
\pgfpathlineto{\pgfqpoint{2.297018in}{0.883451in}}%
\pgfpathlineto{\pgfqpoint{2.301527in}{0.877313in}}%
\pgfpathlineto{\pgfqpoint{2.306036in}{0.912215in}}%
\pgfpathlineto{\pgfqpoint{2.310545in}{0.892932in}}%
\pgfpathlineto{\pgfqpoint{2.315055in}{0.892347in}}%
\pgfpathlineto{\pgfqpoint{2.319564in}{1.118529in}}%
\pgfpathlineto{\pgfqpoint{2.324073in}{0.891456in}}%
\pgfpathlineto{\pgfqpoint{2.328582in}{0.905728in}}%
\pgfpathlineto{\pgfqpoint{2.333091in}{0.889056in}}%
\pgfpathlineto{\pgfqpoint{2.337600in}{0.903724in}}%
\pgfpathlineto{\pgfqpoint{2.342109in}{0.894566in}}%
\pgfpathlineto{\pgfqpoint{2.346618in}{0.894162in}}%
\pgfpathlineto{\pgfqpoint{2.351127in}{0.875470in}}%
\pgfpathlineto{\pgfqpoint{2.355636in}{0.884396in}}%
\pgfpathlineto{\pgfqpoint{2.360145in}{0.904825in}}%
\pgfpathlineto{\pgfqpoint{2.364655in}{0.903931in}}%
\pgfpathlineto{\pgfqpoint{2.369164in}{0.890579in}}%
\pgfpathlineto{\pgfqpoint{2.373673in}{0.914448in}}%
\pgfpathlineto{\pgfqpoint{2.378182in}{0.930705in}}%
\pgfpathlineto{\pgfqpoint{2.382691in}{0.903489in}}%
\pgfpathlineto{\pgfqpoint{2.387200in}{0.907911in}}%
\pgfpathlineto{\pgfqpoint{2.396218in}{0.891167in}}%
\pgfpathlineto{\pgfqpoint{2.400727in}{0.920966in}}%
\pgfpathlineto{\pgfqpoint{2.405236in}{0.927628in}}%
\pgfpathlineto{\pgfqpoint{2.409745in}{0.910500in}}%
\pgfpathlineto{\pgfqpoint{2.414255in}{0.959682in}}%
\pgfpathlineto{\pgfqpoint{2.418764in}{0.912590in}}%
\pgfpathlineto{\pgfqpoint{2.423273in}{0.903913in}}%
\pgfpathlineto{\pgfqpoint{2.432291in}{0.952362in}}%
\pgfpathlineto{\pgfqpoint{2.436800in}{0.957072in}}%
\pgfpathlineto{\pgfqpoint{2.441309in}{0.911013in}}%
\pgfpathlineto{\pgfqpoint{2.445818in}{0.980830in}}%
\pgfpathlineto{\pgfqpoint{2.450327in}{0.925931in}}%
\pgfpathlineto{\pgfqpoint{2.454836in}{0.959386in}}%
\pgfpathlineto{\pgfqpoint{2.459345in}{0.926267in}}%
\pgfpathlineto{\pgfqpoint{2.463855in}{0.947352in}}%
\pgfpathlineto{\pgfqpoint{2.468364in}{0.911724in}}%
\pgfpathlineto{\pgfqpoint{2.472873in}{0.941091in}}%
\pgfpathlineto{\pgfqpoint{2.477382in}{0.939346in}}%
\pgfpathlineto{\pgfqpoint{2.481891in}{0.922285in}}%
\pgfpathlineto{\pgfqpoint{2.486400in}{0.956857in}}%
\pgfpathlineto{\pgfqpoint{2.490909in}{0.939774in}}%
\pgfpathlineto{\pgfqpoint{2.495418in}{0.954023in}}%
\pgfpathlineto{\pgfqpoint{2.499927in}{0.932508in}}%
\pgfpathlineto{\pgfqpoint{2.504436in}{0.991517in}}%
\pgfpathlineto{\pgfqpoint{2.508945in}{0.927313in}}%
\pgfpathlineto{\pgfqpoint{2.513455in}{0.939907in}}%
\pgfpathlineto{\pgfqpoint{2.517964in}{0.957336in}}%
\pgfpathlineto{\pgfqpoint{2.522473in}{0.946556in}}%
\pgfpathlineto{\pgfqpoint{2.526982in}{0.943886in}}%
\pgfpathlineto{\pgfqpoint{2.531491in}{0.951669in}}%
\pgfpathlineto{\pgfqpoint{2.536000in}{0.968122in}}%
\pgfpathlineto{\pgfqpoint{2.540509in}{1.006541in}}%
\pgfpathlineto{\pgfqpoint{2.545018in}{0.962800in}}%
\pgfpathlineto{\pgfqpoint{2.549527in}{0.968092in}}%
\pgfpathlineto{\pgfqpoint{2.554036in}{1.003222in}}%
\pgfpathlineto{\pgfqpoint{2.558545in}{0.940735in}}%
\pgfpathlineto{\pgfqpoint{2.563055in}{0.927224in}}%
\pgfpathlineto{\pgfqpoint{2.567564in}{0.960341in}}%
\pgfpathlineto{\pgfqpoint{2.572073in}{0.961837in}}%
\pgfpathlineto{\pgfqpoint{2.576582in}{0.969577in}}%
\pgfpathlineto{\pgfqpoint{2.581091in}{0.959332in}}%
\pgfpathlineto{\pgfqpoint{2.585600in}{0.974878in}}%
\pgfpathlineto{\pgfqpoint{2.590109in}{0.951480in}}%
\pgfpathlineto{\pgfqpoint{2.594618in}{0.960566in}}%
\pgfpathlineto{\pgfqpoint{2.599127in}{0.952521in}}%
\pgfpathlineto{\pgfqpoint{2.603636in}{0.964327in}}%
\pgfpathlineto{\pgfqpoint{2.608145in}{0.984076in}}%
\pgfpathlineto{\pgfqpoint{2.612655in}{0.988139in}}%
\pgfpathlineto{\pgfqpoint{2.621673in}{0.967506in}}%
\pgfpathlineto{\pgfqpoint{2.626182in}{0.986305in}}%
\pgfpathlineto{\pgfqpoint{2.630691in}{0.981025in}}%
\pgfpathlineto{\pgfqpoint{2.635200in}{0.971875in}}%
\pgfpathlineto{\pgfqpoint{2.639709in}{0.974747in}}%
\pgfpathlineto{\pgfqpoint{2.644218in}{0.963231in}}%
\pgfpathlineto{\pgfqpoint{2.648727in}{0.980474in}}%
\pgfpathlineto{\pgfqpoint{2.657745in}{0.997711in}}%
\pgfpathlineto{\pgfqpoint{2.662255in}{0.972803in}}%
\pgfpathlineto{\pgfqpoint{2.666764in}{0.989695in}}%
\pgfpathlineto{\pgfqpoint{2.671273in}{0.994680in}}%
\pgfpathlineto{\pgfqpoint{2.675782in}{1.073755in}}%
\pgfpathlineto{\pgfqpoint{2.684800in}{0.981317in}}%
\pgfpathlineto{\pgfqpoint{2.689309in}{0.999496in}}%
\pgfpathlineto{\pgfqpoint{2.693818in}{0.999968in}}%
\pgfpathlineto{\pgfqpoint{2.698327in}{1.019167in}}%
\pgfpathlineto{\pgfqpoint{2.702836in}{0.990772in}}%
\pgfpathlineto{\pgfqpoint{2.707345in}{1.016048in}}%
\pgfpathlineto{\pgfqpoint{2.711855in}{1.003530in}}%
\pgfpathlineto{\pgfqpoint{2.716364in}{0.977723in}}%
\pgfpathlineto{\pgfqpoint{2.720873in}{1.013741in}}%
\pgfpathlineto{\pgfqpoint{2.725382in}{1.016935in}}%
\pgfpathlineto{\pgfqpoint{2.729891in}{1.039039in}}%
\pgfpathlineto{\pgfqpoint{2.734400in}{1.033376in}}%
\pgfpathlineto{\pgfqpoint{2.738909in}{1.015851in}}%
\pgfpathlineto{\pgfqpoint{2.743418in}{1.018096in}}%
\pgfpathlineto{\pgfqpoint{2.747927in}{1.049546in}}%
\pgfpathlineto{\pgfqpoint{2.752436in}{1.029321in}}%
\pgfpathlineto{\pgfqpoint{2.756945in}{1.045540in}}%
\pgfpathlineto{\pgfqpoint{2.761455in}{1.045049in}}%
\pgfpathlineto{\pgfqpoint{2.765964in}{1.013933in}}%
\pgfpathlineto{\pgfqpoint{2.770473in}{1.017894in}}%
\pgfpathlineto{\pgfqpoint{2.774982in}{1.011491in}}%
\pgfpathlineto{\pgfqpoint{2.779491in}{1.010140in}}%
\pgfpathlineto{\pgfqpoint{2.784000in}{1.003913in}}%
\pgfpathlineto{\pgfqpoint{2.788509in}{1.017680in}}%
\pgfpathlineto{\pgfqpoint{2.793018in}{1.017572in}}%
\pgfpathlineto{\pgfqpoint{2.797527in}{1.026110in}}%
\pgfpathlineto{\pgfqpoint{2.802036in}{1.060619in}}%
\pgfpathlineto{\pgfqpoint{2.806545in}{1.034020in}}%
\pgfpathlineto{\pgfqpoint{2.811055in}{1.031737in}}%
\pgfpathlineto{\pgfqpoint{2.815564in}{1.043111in}}%
\pgfpathlineto{\pgfqpoint{2.820073in}{1.096863in}}%
\pgfpathlineto{\pgfqpoint{2.824582in}{1.045446in}}%
\pgfpathlineto{\pgfqpoint{2.829091in}{1.030083in}}%
\pgfpathlineto{\pgfqpoint{2.838109in}{1.062251in}}%
\pgfpathlineto{\pgfqpoint{2.842618in}{1.050536in}}%
\pgfpathlineto{\pgfqpoint{2.847127in}{1.066473in}}%
\pgfpathlineto{\pgfqpoint{2.851636in}{1.059241in}}%
\pgfpathlineto{\pgfqpoint{2.856145in}{1.027448in}}%
\pgfpathlineto{\pgfqpoint{2.860655in}{1.066634in}}%
\pgfpathlineto{\pgfqpoint{2.865164in}{1.050155in}}%
\pgfpathlineto{\pgfqpoint{2.874182in}{1.058622in}}%
\pgfpathlineto{\pgfqpoint{2.878691in}{1.098685in}}%
\pgfpathlineto{\pgfqpoint{2.883200in}{1.064753in}}%
\pgfpathlineto{\pgfqpoint{2.887709in}{1.071057in}}%
\pgfpathlineto{\pgfqpoint{2.892218in}{1.072236in}}%
\pgfpathlineto{\pgfqpoint{2.896727in}{1.092539in}}%
\pgfpathlineto{\pgfqpoint{2.901236in}{1.096017in}}%
\pgfpathlineto{\pgfqpoint{2.905745in}{1.116266in}}%
\pgfpathlineto{\pgfqpoint{2.910255in}{1.076111in}}%
\pgfpathlineto{\pgfqpoint{2.914764in}{1.076370in}}%
\pgfpathlineto{\pgfqpoint{2.919273in}{1.092495in}}%
\pgfpathlineto{\pgfqpoint{2.923782in}{1.086822in}}%
\pgfpathlineto{\pgfqpoint{2.928291in}{1.093494in}}%
\pgfpathlineto{\pgfqpoint{2.932800in}{1.071147in}}%
\pgfpathlineto{\pgfqpoint{2.937309in}{1.062290in}}%
\pgfpathlineto{\pgfqpoint{2.941818in}{1.086834in}}%
\pgfpathlineto{\pgfqpoint{2.946327in}{1.079294in}}%
\pgfpathlineto{\pgfqpoint{2.950836in}{1.064654in}}%
\pgfpathlineto{\pgfqpoint{2.955345in}{1.075761in}}%
\pgfpathlineto{\pgfqpoint{2.959855in}{1.082136in}}%
\pgfpathlineto{\pgfqpoint{2.964364in}{1.092333in}}%
\pgfpathlineto{\pgfqpoint{2.968873in}{1.112058in}}%
\pgfpathlineto{\pgfqpoint{2.973382in}{1.116981in}}%
\pgfpathlineto{\pgfqpoint{2.977891in}{1.094626in}}%
\pgfpathlineto{\pgfqpoint{2.982400in}{1.122258in}}%
\pgfpathlineto{\pgfqpoint{2.986909in}{1.067362in}}%
\pgfpathlineto{\pgfqpoint{2.991418in}{1.104213in}}%
\pgfpathlineto{\pgfqpoint{2.995927in}{1.177689in}}%
\pgfpathlineto{\pgfqpoint{3.000436in}{1.072286in}}%
\pgfpathlineto{\pgfqpoint{3.004945in}{1.092811in}}%
\pgfpathlineto{\pgfqpoint{3.009455in}{1.129926in}}%
\pgfpathlineto{\pgfqpoint{3.013964in}{1.102962in}}%
\pgfpathlineto{\pgfqpoint{3.018473in}{1.133643in}}%
\pgfpathlineto{\pgfqpoint{3.022982in}{1.135576in}}%
\pgfpathlineto{\pgfqpoint{3.027491in}{1.090953in}}%
\pgfpathlineto{\pgfqpoint{3.032000in}{1.130545in}}%
\pgfpathlineto{\pgfqpoint{3.036509in}{1.127764in}}%
\pgfpathlineto{\pgfqpoint{3.041018in}{1.121014in}}%
\pgfpathlineto{\pgfqpoint{3.045527in}{1.132143in}}%
\pgfpathlineto{\pgfqpoint{3.050036in}{1.137755in}}%
\pgfpathlineto{\pgfqpoint{3.054545in}{1.116225in}}%
\pgfpathlineto{\pgfqpoint{3.059055in}{1.124798in}}%
\pgfpathlineto{\pgfqpoint{3.063564in}{1.178771in}}%
\pgfpathlineto{\pgfqpoint{3.068073in}{1.159506in}}%
\pgfpathlineto{\pgfqpoint{3.072582in}{1.168931in}}%
\pgfpathlineto{\pgfqpoint{3.077091in}{1.142518in}}%
\pgfpathlineto{\pgfqpoint{3.081600in}{1.196840in}}%
\pgfpathlineto{\pgfqpoint{3.086109in}{1.139190in}}%
\pgfpathlineto{\pgfqpoint{3.090618in}{1.166575in}}%
\pgfpathlineto{\pgfqpoint{3.095127in}{1.180530in}}%
\pgfpathlineto{\pgfqpoint{3.104145in}{1.134825in}}%
\pgfpathlineto{\pgfqpoint{3.108655in}{1.167760in}}%
\pgfpathlineto{\pgfqpoint{3.113164in}{1.185002in}}%
\pgfpathlineto{\pgfqpoint{3.117673in}{1.156605in}}%
\pgfpathlineto{\pgfqpoint{3.122182in}{1.145037in}}%
\pgfpathlineto{\pgfqpoint{3.126691in}{1.119469in}}%
\pgfpathlineto{\pgfqpoint{3.131200in}{1.201594in}}%
\pgfpathlineto{\pgfqpoint{3.135709in}{1.143069in}}%
\pgfpathlineto{\pgfqpoint{3.140218in}{1.171706in}}%
\pgfpathlineto{\pgfqpoint{3.144727in}{1.143047in}}%
\pgfpathlineto{\pgfqpoint{3.149236in}{1.175594in}}%
\pgfpathlineto{\pgfqpoint{3.153745in}{1.163639in}}%
\pgfpathlineto{\pgfqpoint{3.158255in}{1.164655in}}%
\pgfpathlineto{\pgfqpoint{3.162764in}{1.185195in}}%
\pgfpathlineto{\pgfqpoint{3.167273in}{1.166986in}}%
\pgfpathlineto{\pgfqpoint{3.176291in}{1.185072in}}%
\pgfpathlineto{\pgfqpoint{3.180800in}{1.174390in}}%
\pgfpathlineto{\pgfqpoint{3.185309in}{1.175081in}}%
\pgfpathlineto{\pgfqpoint{3.189818in}{1.192783in}}%
\pgfpathlineto{\pgfqpoint{3.194327in}{1.181811in}}%
\pgfpathlineto{\pgfqpoint{3.198836in}{1.210080in}}%
\pgfpathlineto{\pgfqpoint{3.203345in}{1.190493in}}%
\pgfpathlineto{\pgfqpoint{3.207855in}{1.183248in}}%
\pgfpathlineto{\pgfqpoint{3.212364in}{1.187793in}}%
\pgfpathlineto{\pgfqpoint{3.216873in}{1.182224in}}%
\pgfpathlineto{\pgfqpoint{3.221382in}{1.206928in}}%
\pgfpathlineto{\pgfqpoint{3.225891in}{1.170138in}}%
\pgfpathlineto{\pgfqpoint{3.230400in}{1.167107in}}%
\pgfpathlineto{\pgfqpoint{3.234909in}{1.200623in}}%
\pgfpathlineto{\pgfqpoint{3.239418in}{1.211520in}}%
\pgfpathlineto{\pgfqpoint{3.243927in}{1.189531in}}%
\pgfpathlineto{\pgfqpoint{3.248436in}{1.218280in}}%
\pgfpathlineto{\pgfqpoint{3.252945in}{1.221091in}}%
\pgfpathlineto{\pgfqpoint{3.257455in}{1.219782in}}%
\pgfpathlineto{\pgfqpoint{3.261964in}{1.245269in}}%
\pgfpathlineto{\pgfqpoint{3.266473in}{1.216866in}}%
\pgfpathlineto{\pgfqpoint{3.270982in}{1.207566in}}%
\pgfpathlineto{\pgfqpoint{3.275491in}{1.227327in}}%
\pgfpathlineto{\pgfqpoint{3.280000in}{1.319239in}}%
\pgfpathlineto{\pgfqpoint{3.284509in}{1.192658in}}%
\pgfpathlineto{\pgfqpoint{3.289018in}{1.280812in}}%
\pgfpathlineto{\pgfqpoint{3.293527in}{1.238056in}}%
\pgfpathlineto{\pgfqpoint{3.298036in}{1.235736in}}%
\pgfpathlineto{\pgfqpoint{3.302545in}{1.258218in}}%
\pgfpathlineto{\pgfqpoint{3.307055in}{1.256631in}}%
\pgfpathlineto{\pgfqpoint{3.311564in}{1.248402in}}%
\pgfpathlineto{\pgfqpoint{3.316073in}{1.245100in}}%
\pgfpathlineto{\pgfqpoint{3.320582in}{1.285037in}}%
\pgfpathlineto{\pgfqpoint{3.325091in}{1.266569in}}%
\pgfpathlineto{\pgfqpoint{3.329600in}{1.231083in}}%
\pgfpathlineto{\pgfqpoint{3.334109in}{1.267151in}}%
\pgfpathlineto{\pgfqpoint{3.338618in}{1.242825in}}%
\pgfpathlineto{\pgfqpoint{3.343127in}{1.230205in}}%
\pgfpathlineto{\pgfqpoint{3.347636in}{1.296500in}}%
\pgfpathlineto{\pgfqpoint{3.352145in}{1.244606in}}%
\pgfpathlineto{\pgfqpoint{3.356655in}{1.260851in}}%
\pgfpathlineto{\pgfqpoint{3.361164in}{1.259164in}}%
\pgfpathlineto{\pgfqpoint{3.365673in}{1.245581in}}%
\pgfpathlineto{\pgfqpoint{3.370182in}{1.266445in}}%
\pgfpathlineto{\pgfqpoint{3.374691in}{1.228962in}}%
\pgfpathlineto{\pgfqpoint{3.379200in}{1.289056in}}%
\pgfpathlineto{\pgfqpoint{3.383709in}{1.240717in}}%
\pgfpathlineto{\pgfqpoint{3.388218in}{1.243255in}}%
\pgfpathlineto{\pgfqpoint{3.392727in}{1.253660in}}%
\pgfpathlineto{\pgfqpoint{3.401745in}{1.307632in}}%
\pgfpathlineto{\pgfqpoint{3.406255in}{1.300493in}}%
\pgfpathlineto{\pgfqpoint{3.410764in}{1.264786in}}%
\pgfpathlineto{\pgfqpoint{3.415273in}{1.275795in}}%
\pgfpathlineto{\pgfqpoint{3.419782in}{1.277574in}}%
\pgfpathlineto{\pgfqpoint{3.424291in}{1.294766in}}%
\pgfpathlineto{\pgfqpoint{3.433309in}{1.303669in}}%
\pgfpathlineto{\pgfqpoint{3.437818in}{1.358337in}}%
\pgfpathlineto{\pgfqpoint{3.442327in}{1.304006in}}%
\pgfpathlineto{\pgfqpoint{3.446836in}{1.280610in}}%
\pgfpathlineto{\pgfqpoint{3.451345in}{1.292673in}}%
\pgfpathlineto{\pgfqpoint{3.455855in}{1.309026in}}%
\pgfpathlineto{\pgfqpoint{3.460364in}{1.319811in}}%
\pgfpathlineto{\pgfqpoint{3.464873in}{1.282074in}}%
\pgfpathlineto{\pgfqpoint{3.469382in}{1.355091in}}%
\pgfpathlineto{\pgfqpoint{3.473891in}{1.318741in}}%
\pgfpathlineto{\pgfqpoint{3.478400in}{1.314882in}}%
\pgfpathlineto{\pgfqpoint{3.482909in}{1.288072in}}%
\pgfpathlineto{\pgfqpoint{3.487418in}{1.330983in}}%
\pgfpathlineto{\pgfqpoint{3.491927in}{1.356533in}}%
\pgfpathlineto{\pgfqpoint{3.496436in}{1.336737in}}%
\pgfpathlineto{\pgfqpoint{3.500945in}{1.284207in}}%
\pgfpathlineto{\pgfqpoint{3.505455in}{1.342129in}}%
\pgfpathlineto{\pgfqpoint{3.509964in}{1.367408in}}%
\pgfpathlineto{\pgfqpoint{3.514473in}{1.327990in}}%
\pgfpathlineto{\pgfqpoint{3.518982in}{1.323690in}}%
\pgfpathlineto{\pgfqpoint{3.523491in}{1.347600in}}%
\pgfpathlineto{\pgfqpoint{3.528000in}{1.318016in}}%
\pgfpathlineto{\pgfqpoint{3.532509in}{1.352814in}}%
\pgfpathlineto{\pgfqpoint{3.537018in}{1.332991in}}%
\pgfpathlineto{\pgfqpoint{3.541527in}{1.358948in}}%
\pgfpathlineto{\pgfqpoint{3.546036in}{1.358231in}}%
\pgfpathlineto{\pgfqpoint{3.550545in}{1.345624in}}%
\pgfpathlineto{\pgfqpoint{3.555055in}{1.324792in}}%
\pgfpathlineto{\pgfqpoint{3.559564in}{1.358977in}}%
\pgfpathlineto{\pgfqpoint{3.564073in}{1.349015in}}%
\pgfpathlineto{\pgfqpoint{3.568582in}{1.369733in}}%
\pgfpathlineto{\pgfqpoint{3.573091in}{1.369174in}}%
\pgfpathlineto{\pgfqpoint{3.577600in}{1.362270in}}%
\pgfpathlineto{\pgfqpoint{3.582109in}{1.376826in}}%
\pgfpathlineto{\pgfqpoint{3.586618in}{1.424726in}}%
\pgfpathlineto{\pgfqpoint{3.591127in}{1.360941in}}%
\pgfpathlineto{\pgfqpoint{3.595636in}{1.347373in}}%
\pgfpathlineto{\pgfqpoint{3.600145in}{1.374173in}}%
\pgfpathlineto{\pgfqpoint{3.604655in}{1.388205in}}%
\pgfpathlineto{\pgfqpoint{3.609164in}{1.373722in}}%
\pgfpathlineto{\pgfqpoint{3.613673in}{1.400917in}}%
\pgfpathlineto{\pgfqpoint{3.618182in}{1.405229in}}%
\pgfpathlineto{\pgfqpoint{3.622691in}{1.411400in}}%
\pgfpathlineto{\pgfqpoint{3.627200in}{1.474098in}}%
\pgfpathlineto{\pgfqpoint{3.636218in}{1.374797in}}%
\pgfpathlineto{\pgfqpoint{3.640727in}{1.475634in}}%
\pgfpathlineto{\pgfqpoint{3.645236in}{1.472656in}}%
\pgfpathlineto{\pgfqpoint{3.649745in}{1.462637in}}%
\pgfpathlineto{\pgfqpoint{3.654255in}{1.377689in}}%
\pgfpathlineto{\pgfqpoint{3.663273in}{1.477366in}}%
\pgfpathlineto{\pgfqpoint{3.667782in}{1.410550in}}%
\pgfpathlineto{\pgfqpoint{3.672291in}{1.423596in}}%
\pgfpathlineto{\pgfqpoint{3.676800in}{1.452162in}}%
\pgfpathlineto{\pgfqpoint{3.681309in}{1.416633in}}%
\pgfpathlineto{\pgfqpoint{3.685818in}{1.466460in}}%
\pgfpathlineto{\pgfqpoint{3.690327in}{1.440513in}}%
\pgfpathlineto{\pgfqpoint{3.694836in}{1.430754in}}%
\pgfpathlineto{\pgfqpoint{3.699345in}{1.431321in}}%
\pgfpathlineto{\pgfqpoint{3.703855in}{1.413935in}}%
\pgfpathlineto{\pgfqpoint{3.708364in}{1.479404in}}%
\pgfpathlineto{\pgfqpoint{3.712873in}{1.415620in}}%
\pgfpathlineto{\pgfqpoint{3.717382in}{1.467913in}}%
\pgfpathlineto{\pgfqpoint{3.721891in}{1.557660in}}%
\pgfpathlineto{\pgfqpoint{3.726400in}{1.493780in}}%
\pgfpathlineto{\pgfqpoint{3.730909in}{1.495611in}}%
\pgfpathlineto{\pgfqpoint{3.735418in}{1.435757in}}%
\pgfpathlineto{\pgfqpoint{3.739927in}{1.443827in}}%
\pgfpathlineto{\pgfqpoint{3.744436in}{1.456596in}}%
\pgfpathlineto{\pgfqpoint{3.748945in}{1.447680in}}%
\pgfpathlineto{\pgfqpoint{3.753455in}{1.476978in}}%
\pgfpathlineto{\pgfqpoint{3.757964in}{1.438192in}}%
\pgfpathlineto{\pgfqpoint{3.762473in}{1.445653in}}%
\pgfpathlineto{\pgfqpoint{3.766982in}{1.459766in}}%
\pgfpathlineto{\pgfqpoint{3.771491in}{1.466724in}}%
\pgfpathlineto{\pgfqpoint{3.776000in}{1.441881in}}%
\pgfpathlineto{\pgfqpoint{3.780509in}{1.440320in}}%
\pgfpathlineto{\pgfqpoint{3.785018in}{1.497346in}}%
\pgfpathlineto{\pgfqpoint{3.789527in}{1.680834in}}%
\pgfpathlineto{\pgfqpoint{3.794036in}{1.627002in}}%
\pgfpathlineto{\pgfqpoint{3.798545in}{1.950597in}}%
\pgfpathlineto{\pgfqpoint{3.803055in}{2.439883in}}%
\pgfpathlineto{\pgfqpoint{3.807564in}{1.893160in}}%
\pgfpathlineto{\pgfqpoint{3.812073in}{2.527614in}}%
\pgfpathlineto{\pgfqpoint{3.816582in}{1.802209in}}%
\pgfpathlineto{\pgfqpoint{3.821091in}{1.477787in}}%
\pgfpathlineto{\pgfqpoint{3.825600in}{1.462686in}}%
\pgfpathlineto{\pgfqpoint{3.830109in}{1.595739in}}%
\pgfpathlineto{\pgfqpoint{3.834618in}{1.444205in}}%
\pgfpathlineto{\pgfqpoint{3.839127in}{1.454778in}}%
\pgfpathlineto{\pgfqpoint{3.843636in}{1.707425in}}%
\pgfpathlineto{\pgfqpoint{3.848145in}{1.485361in}}%
\pgfpathlineto{\pgfqpoint{3.852655in}{1.579935in}}%
\pgfpathlineto{\pgfqpoint{3.857164in}{1.556915in}}%
\pgfpathlineto{\pgfqpoint{3.866182in}{1.479944in}}%
\pgfpathlineto{\pgfqpoint{3.870691in}{1.498282in}}%
\pgfpathlineto{\pgfqpoint{3.875200in}{1.524017in}}%
\pgfpathlineto{\pgfqpoint{3.879709in}{1.590314in}}%
\pgfpathlineto{\pgfqpoint{3.884218in}{1.635572in}}%
\pgfpathlineto{\pgfqpoint{3.888727in}{1.519908in}}%
\pgfpathlineto{\pgfqpoint{3.893236in}{1.499406in}}%
\pgfpathlineto{\pgfqpoint{3.897745in}{1.495829in}}%
\pgfpathlineto{\pgfqpoint{3.902255in}{1.475172in}}%
\pgfpathlineto{\pgfqpoint{3.906764in}{1.576118in}}%
\pgfpathlineto{\pgfqpoint{3.911273in}{1.594386in}}%
\pgfpathlineto{\pgfqpoint{3.915782in}{1.533818in}}%
\pgfpathlineto{\pgfqpoint{3.920291in}{1.657115in}}%
\pgfpathlineto{\pgfqpoint{3.924800in}{1.509904in}}%
\pgfpathlineto{\pgfqpoint{3.929309in}{1.513169in}}%
\pgfpathlineto{\pgfqpoint{3.938327in}{1.552845in}}%
\pgfpathlineto{\pgfqpoint{3.942836in}{1.552940in}}%
\pgfpathlineto{\pgfqpoint{3.947345in}{1.694347in}}%
\pgfpathlineto{\pgfqpoint{3.951855in}{1.578761in}}%
\pgfpathlineto{\pgfqpoint{3.956364in}{1.544930in}}%
\pgfpathlineto{\pgfqpoint{3.960873in}{1.529094in}}%
\pgfpathlineto{\pgfqpoint{3.965382in}{1.509174in}}%
\pgfpathlineto{\pgfqpoint{3.969891in}{1.586749in}}%
\pgfpathlineto{\pgfqpoint{3.974400in}{1.566663in}}%
\pgfpathlineto{\pgfqpoint{3.978909in}{1.582435in}}%
\pgfpathlineto{\pgfqpoint{3.983418in}{1.565614in}}%
\pgfpathlineto{\pgfqpoint{3.987927in}{1.575540in}}%
\pgfpathlineto{\pgfqpoint{3.992436in}{1.540347in}}%
\pgfpathlineto{\pgfqpoint{3.996945in}{1.593327in}}%
\pgfpathlineto{\pgfqpoint{4.005964in}{1.539042in}}%
\pgfpathlineto{\pgfqpoint{4.010473in}{1.577289in}}%
\pgfpathlineto{\pgfqpoint{4.014982in}{1.555226in}}%
\pgfpathlineto{\pgfqpoint{4.019491in}{1.605203in}}%
\pgfpathlineto{\pgfqpoint{4.024000in}{1.567727in}}%
\pgfpathlineto{\pgfqpoint{4.028509in}{1.569607in}}%
\pgfpathlineto{\pgfqpoint{4.033018in}{1.642646in}}%
\pgfpathlineto{\pgfqpoint{4.037527in}{1.620211in}}%
\pgfpathlineto{\pgfqpoint{4.042036in}{1.612692in}}%
\pgfpathlineto{\pgfqpoint{4.046545in}{1.572744in}}%
\pgfpathlineto{\pgfqpoint{4.051055in}{1.605980in}}%
\pgfpathlineto{\pgfqpoint{4.055564in}{1.620088in}}%
\pgfpathlineto{\pgfqpoint{4.060073in}{1.612722in}}%
\pgfpathlineto{\pgfqpoint{4.064582in}{1.563662in}}%
\pgfpathlineto{\pgfqpoint{4.069091in}{1.646221in}}%
\pgfpathlineto{\pgfqpoint{4.073600in}{1.626969in}}%
\pgfpathlineto{\pgfqpoint{4.078109in}{1.587361in}}%
\pgfpathlineto{\pgfqpoint{4.082618in}{1.595895in}}%
\pgfpathlineto{\pgfqpoint{4.087127in}{1.634120in}}%
\pgfpathlineto{\pgfqpoint{4.091636in}{1.599902in}}%
\pgfpathlineto{\pgfqpoint{4.096145in}{1.616427in}}%
\pgfpathlineto{\pgfqpoint{4.100655in}{1.641721in}}%
\pgfpathlineto{\pgfqpoint{4.105164in}{1.673899in}}%
\pgfpathlineto{\pgfqpoint{4.109673in}{1.596683in}}%
\pgfpathlineto{\pgfqpoint{4.114182in}{1.627570in}}%
\pgfpathlineto{\pgfqpoint{4.118691in}{1.647105in}}%
\pgfpathlineto{\pgfqpoint{4.123200in}{1.648780in}}%
\pgfpathlineto{\pgfqpoint{4.127709in}{1.675099in}}%
\pgfpathlineto{\pgfqpoint{4.132218in}{1.603576in}}%
\pgfpathlineto{\pgfqpoint{4.136727in}{1.614682in}}%
\pgfpathlineto{\pgfqpoint{4.141236in}{1.657894in}}%
\pgfpathlineto{\pgfqpoint{4.145745in}{1.681975in}}%
\pgfpathlineto{\pgfqpoint{4.150255in}{1.649710in}}%
\pgfpathlineto{\pgfqpoint{4.154764in}{1.640305in}}%
\pgfpathlineto{\pgfqpoint{4.159273in}{1.773550in}}%
\pgfpathlineto{\pgfqpoint{4.168291in}{1.685590in}}%
\pgfpathlineto{\pgfqpoint{4.172800in}{1.665978in}}%
\pgfpathlineto{\pgfqpoint{4.177309in}{1.690399in}}%
\pgfpathlineto{\pgfqpoint{4.181818in}{1.652257in}}%
\pgfpathlineto{\pgfqpoint{4.186327in}{1.703615in}}%
\pgfpathlineto{\pgfqpoint{4.190836in}{1.717893in}}%
\pgfpathlineto{\pgfqpoint{4.195345in}{1.751353in}}%
\pgfpathlineto{\pgfqpoint{4.199855in}{1.694216in}}%
\pgfpathlineto{\pgfqpoint{4.204364in}{1.703306in}}%
\pgfpathlineto{\pgfqpoint{4.208873in}{1.652486in}}%
\pgfpathlineto{\pgfqpoint{4.213382in}{1.698976in}}%
\pgfpathlineto{\pgfqpoint{4.217891in}{1.776711in}}%
\pgfpathlineto{\pgfqpoint{4.222400in}{1.718249in}}%
\pgfpathlineto{\pgfqpoint{4.226909in}{1.681586in}}%
\pgfpathlineto{\pgfqpoint{4.231418in}{1.775159in}}%
\pgfpathlineto{\pgfqpoint{4.235927in}{1.721839in}}%
\pgfpathlineto{\pgfqpoint{4.240436in}{1.731569in}}%
\pgfpathlineto{\pgfqpoint{4.244945in}{1.690639in}}%
\pgfpathlineto{\pgfqpoint{4.249455in}{1.709774in}}%
\pgfpathlineto{\pgfqpoint{4.253964in}{1.764967in}}%
\pgfpathlineto{\pgfqpoint{4.258473in}{1.715622in}}%
\pgfpathlineto{\pgfqpoint{4.262982in}{1.755155in}}%
\pgfpathlineto{\pgfqpoint{4.267491in}{1.705140in}}%
\pgfpathlineto{\pgfqpoint{4.272000in}{1.735939in}}%
\pgfpathlineto{\pgfqpoint{4.276509in}{1.801006in}}%
\pgfpathlineto{\pgfqpoint{4.281018in}{1.688620in}}%
\pgfpathlineto{\pgfqpoint{4.285527in}{1.742185in}}%
\pgfpathlineto{\pgfqpoint{4.294545in}{1.769994in}}%
\pgfpathlineto{\pgfqpoint{4.299055in}{1.792945in}}%
\pgfpathlineto{\pgfqpoint{4.303564in}{1.775880in}}%
\pgfpathlineto{\pgfqpoint{4.308073in}{1.829797in}}%
\pgfpathlineto{\pgfqpoint{4.312582in}{1.720054in}}%
\pgfpathlineto{\pgfqpoint{4.317091in}{1.759027in}}%
\pgfpathlineto{\pgfqpoint{4.321600in}{1.742773in}}%
\pgfpathlineto{\pgfqpoint{4.326109in}{1.787727in}}%
\pgfpathlineto{\pgfqpoint{4.330618in}{1.760564in}}%
\pgfpathlineto{\pgfqpoint{4.335127in}{1.750048in}}%
\pgfpathlineto{\pgfqpoint{4.339636in}{1.775543in}}%
\pgfpathlineto{\pgfqpoint{4.344145in}{1.849963in}}%
\pgfpathlineto{\pgfqpoint{4.348655in}{1.762656in}}%
\pgfpathlineto{\pgfqpoint{4.353164in}{1.851299in}}%
\pgfpathlineto{\pgfqpoint{4.357673in}{1.769889in}}%
\pgfpathlineto{\pgfqpoint{4.362182in}{1.723270in}}%
\pgfpathlineto{\pgfqpoint{4.366691in}{1.780279in}}%
\pgfpathlineto{\pgfqpoint{4.371200in}{1.762745in}}%
\pgfpathlineto{\pgfqpoint{4.375709in}{1.808602in}}%
\pgfpathlineto{\pgfqpoint{4.380218in}{1.783373in}}%
\pgfpathlineto{\pgfqpoint{4.384727in}{1.804305in}}%
\pgfpathlineto{\pgfqpoint{4.389236in}{1.795835in}}%
\pgfpathlineto{\pgfqpoint{4.393745in}{1.853336in}}%
\pgfpathlineto{\pgfqpoint{4.398255in}{1.794619in}}%
\pgfpathlineto{\pgfqpoint{4.402764in}{1.800298in}}%
\pgfpathlineto{\pgfqpoint{4.407273in}{1.861477in}}%
\pgfpathlineto{\pgfqpoint{4.411782in}{1.824402in}}%
\pgfpathlineto{\pgfqpoint{4.416291in}{1.808785in}}%
\pgfpathlineto{\pgfqpoint{4.420800in}{1.884781in}}%
\pgfpathlineto{\pgfqpoint{4.425309in}{1.772854in}}%
\pgfpathlineto{\pgfqpoint{4.429818in}{1.808924in}}%
\pgfpathlineto{\pgfqpoint{4.434327in}{1.898736in}}%
\pgfpathlineto{\pgfqpoint{4.438836in}{1.794017in}}%
\pgfpathlineto{\pgfqpoint{4.443345in}{1.795330in}}%
\pgfpathlineto{\pgfqpoint{4.447855in}{1.816995in}}%
\pgfpathlineto{\pgfqpoint{4.452364in}{1.906533in}}%
\pgfpathlineto{\pgfqpoint{4.456873in}{1.808073in}}%
\pgfpathlineto{\pgfqpoint{4.461382in}{1.878336in}}%
\pgfpathlineto{\pgfqpoint{4.465891in}{1.787397in}}%
\pgfpathlineto{\pgfqpoint{4.470400in}{1.857735in}}%
\pgfpathlineto{\pgfqpoint{4.474909in}{1.800112in}}%
\pgfpathlineto{\pgfqpoint{4.479418in}{1.879759in}}%
\pgfpathlineto{\pgfqpoint{4.483927in}{1.880887in}}%
\pgfpathlineto{\pgfqpoint{4.488436in}{1.842583in}}%
\pgfpathlineto{\pgfqpoint{4.492945in}{1.900925in}}%
\pgfpathlineto{\pgfqpoint{4.497455in}{1.866446in}}%
\pgfpathlineto{\pgfqpoint{4.501964in}{1.857957in}}%
\pgfpathlineto{\pgfqpoint{4.506473in}{1.890845in}}%
\pgfpathlineto{\pgfqpoint{4.510982in}{1.881694in}}%
\pgfpathlineto{\pgfqpoint{4.515491in}{1.833768in}}%
\pgfpathlineto{\pgfqpoint{4.520000in}{1.905557in}}%
\pgfpathlineto{\pgfqpoint{4.524509in}{1.833189in}}%
\pgfpathlineto{\pgfqpoint{4.529018in}{2.054943in}}%
\pgfpathlineto{\pgfqpoint{4.533527in}{1.867976in}}%
\pgfpathlineto{\pgfqpoint{4.538036in}{1.866312in}}%
\pgfpathlineto{\pgfqpoint{4.542545in}{1.986370in}}%
\pgfpathlineto{\pgfqpoint{4.551564in}{1.889651in}}%
\pgfpathlineto{\pgfqpoint{4.556073in}{1.869471in}}%
\pgfpathlineto{\pgfqpoint{4.560582in}{1.875357in}}%
\pgfpathlineto{\pgfqpoint{4.565091in}{1.935332in}}%
\pgfpathlineto{\pgfqpoint{4.569600in}{1.922239in}}%
\pgfpathlineto{\pgfqpoint{4.574109in}{1.913230in}}%
\pgfpathlineto{\pgfqpoint{4.578618in}{1.889268in}}%
\pgfpathlineto{\pgfqpoint{4.583127in}{1.919546in}}%
\pgfpathlineto{\pgfqpoint{4.587636in}{1.982420in}}%
\pgfpathlineto{\pgfqpoint{4.592145in}{1.935194in}}%
\pgfpathlineto{\pgfqpoint{4.596655in}{2.019142in}}%
\pgfpathlineto{\pgfqpoint{4.601164in}{1.919519in}}%
\pgfpathlineto{\pgfqpoint{4.605673in}{1.989577in}}%
\pgfpathlineto{\pgfqpoint{4.610182in}{1.937447in}}%
\pgfpathlineto{\pgfqpoint{4.614691in}{1.956925in}}%
\pgfpathlineto{\pgfqpoint{4.619200in}{2.001836in}}%
\pgfpathlineto{\pgfqpoint{4.623709in}{1.941190in}}%
\pgfpathlineto{\pgfqpoint{4.628218in}{1.946723in}}%
\pgfpathlineto{\pgfqpoint{4.632727in}{1.940873in}}%
\pgfpathlineto{\pgfqpoint{4.637236in}{1.857325in}}%
\pgfpathlineto{\pgfqpoint{4.641745in}{2.103379in}}%
\pgfpathlineto{\pgfqpoint{4.646255in}{2.191622in}}%
\pgfpathlineto{\pgfqpoint{4.650764in}{1.935076in}}%
\pgfpathlineto{\pgfqpoint{4.655273in}{2.046929in}}%
\pgfpathlineto{\pgfqpoint{4.659782in}{2.104004in}}%
\pgfpathlineto{\pgfqpoint{4.664291in}{2.213137in}}%
\pgfpathlineto{\pgfqpoint{4.668800in}{2.002148in}}%
\pgfpathlineto{\pgfqpoint{4.673309in}{2.618146in}}%
\pgfpathlineto{\pgfqpoint{4.677818in}{2.075146in}}%
\pgfpathlineto{\pgfqpoint{4.686836in}{2.038462in}}%
\pgfpathlineto{\pgfqpoint{4.691345in}{2.041457in}}%
\pgfpathlineto{\pgfqpoint{4.695855in}{2.024667in}}%
\pgfpathlineto{\pgfqpoint{4.700364in}{2.025946in}}%
\pgfpathlineto{\pgfqpoint{4.704873in}{2.208320in}}%
\pgfpathlineto{\pgfqpoint{4.709382in}{2.783755in}}%
\pgfpathlineto{\pgfqpoint{4.713891in}{2.088793in}}%
\pgfpathlineto{\pgfqpoint{4.718400in}{2.303940in}}%
\pgfpathlineto{\pgfqpoint{4.722909in}{2.192849in}}%
\pgfpathlineto{\pgfqpoint{4.727418in}{2.221777in}}%
\pgfpathlineto{\pgfqpoint{4.731927in}{2.161563in}}%
\pgfpathlineto{\pgfqpoint{4.736436in}{2.339897in}}%
\pgfpathlineto{\pgfqpoint{4.740945in}{2.134186in}}%
\pgfpathlineto{\pgfqpoint{4.745455in}{2.066492in}}%
\pgfpathlineto{\pgfqpoint{4.749964in}{2.040643in}}%
\pgfpathlineto{\pgfqpoint{4.754473in}{1.988802in}}%
\pgfpathlineto{\pgfqpoint{4.758982in}{1.873705in}}%
\pgfpathlineto{\pgfqpoint{4.763491in}{1.960739in}}%
\pgfpathlineto{\pgfqpoint{4.768000in}{1.917014in}}%
\pgfpathlineto{\pgfqpoint{4.772509in}{1.929077in}}%
\pgfpathlineto{\pgfqpoint{4.777018in}{1.964148in}}%
\pgfpathlineto{\pgfqpoint{4.781527in}{1.950361in}}%
\pgfpathlineto{\pgfqpoint{4.786036in}{1.895727in}}%
\pgfpathlineto{\pgfqpoint{4.790545in}{1.902238in}}%
\pgfpathlineto{\pgfqpoint{4.795055in}{1.948745in}}%
\pgfpathlineto{\pgfqpoint{4.799564in}{1.905035in}}%
\pgfpathlineto{\pgfqpoint{4.804073in}{2.071762in}}%
\pgfpathlineto{\pgfqpoint{4.808582in}{2.498896in}}%
\pgfpathlineto{\pgfqpoint{4.813091in}{2.625495in}}%
\pgfpathlineto{\pgfqpoint{4.817600in}{1.907064in}}%
\pgfpathlineto{\pgfqpoint{4.822109in}{1.927707in}}%
\pgfpathlineto{\pgfqpoint{4.826618in}{1.973205in}}%
\pgfpathlineto{\pgfqpoint{4.831127in}{3.017907in}}%
\pgfpathlineto{\pgfqpoint{4.835636in}{2.197428in}}%
\pgfpathlineto{\pgfqpoint{4.840145in}{1.950125in}}%
\pgfpathlineto{\pgfqpoint{4.844655in}{2.188738in}}%
\pgfpathlineto{\pgfqpoint{4.849164in}{3.670640in}}%
\pgfpathlineto{\pgfqpoint{4.853673in}{2.148516in}}%
\pgfpathlineto{\pgfqpoint{4.858182in}{2.105747in}}%
\pgfpathlineto{\pgfqpoint{4.862691in}{2.134839in}}%
\pgfpathlineto{\pgfqpoint{4.867200in}{2.351561in}}%
\pgfpathlineto{\pgfqpoint{4.871709in}{2.624702in}}%
\pgfpathlineto{\pgfqpoint{4.876218in}{2.307344in}}%
\pgfpathlineto{\pgfqpoint{4.880727in}{2.265453in}}%
\pgfpathlineto{\pgfqpoint{4.885236in}{2.419552in}}%
\pgfpathlineto{\pgfqpoint{4.889745in}{2.130750in}}%
\pgfpathlineto{\pgfqpoint{4.894255in}{2.145697in}}%
\pgfpathlineto{\pgfqpoint{4.903273in}{2.188765in}}%
\pgfpathlineto{\pgfqpoint{4.907782in}{2.255393in}}%
\pgfpathlineto{\pgfqpoint{4.912291in}{2.212267in}}%
\pgfpathlineto{\pgfqpoint{4.916800in}{2.128337in}}%
\pgfpathlineto{\pgfqpoint{4.921309in}{2.157207in}}%
\pgfpathlineto{\pgfqpoint{4.925818in}{2.167386in}}%
\pgfpathlineto{\pgfqpoint{4.934836in}{2.223205in}}%
\pgfpathlineto{\pgfqpoint{4.939345in}{2.207288in}}%
\pgfpathlineto{\pgfqpoint{4.943855in}{2.268106in}}%
\pgfpathlineto{\pgfqpoint{4.948364in}{2.210780in}}%
\pgfpathlineto{\pgfqpoint{4.952873in}{2.222870in}}%
\pgfpathlineto{\pgfqpoint{4.957382in}{2.181395in}}%
\pgfpathlineto{\pgfqpoint{4.961891in}{2.232443in}}%
\pgfpathlineto{\pgfqpoint{4.966400in}{2.170068in}}%
\pgfpathlineto{\pgfqpoint{4.970909in}{2.292057in}}%
\pgfpathlineto{\pgfqpoint{4.975418in}{2.274494in}}%
\pgfpathlineto{\pgfqpoint{4.979927in}{2.127010in}}%
\pgfpathlineto{\pgfqpoint{4.984436in}{2.245242in}}%
\pgfpathlineto{\pgfqpoint{4.988945in}{2.181983in}}%
\pgfpathlineto{\pgfqpoint{4.993455in}{2.256028in}}%
\pgfpathlineto{\pgfqpoint{4.997964in}{2.236067in}}%
\pgfpathlineto{\pgfqpoint{5.002473in}{2.271439in}}%
\pgfpathlineto{\pgfqpoint{5.006982in}{2.319781in}}%
\pgfpathlineto{\pgfqpoint{5.011491in}{2.188497in}}%
\pgfpathlineto{\pgfqpoint{5.016000in}{2.254658in}}%
\pgfpathlineto{\pgfqpoint{5.020509in}{2.231808in}}%
\pgfpathlineto{\pgfqpoint{5.025018in}{2.202897in}}%
\pgfpathlineto{\pgfqpoint{5.029527in}{2.248495in}}%
\pgfpathlineto{\pgfqpoint{5.034036in}{2.306094in}}%
\pgfpathlineto{\pgfqpoint{5.038545in}{2.558473in}}%
\pgfpathlineto{\pgfqpoint{5.043055in}{2.354697in}}%
\pgfpathlineto{\pgfqpoint{5.047564in}{2.254743in}}%
\pgfpathlineto{\pgfqpoint{5.052073in}{2.573269in}}%
\pgfpathlineto{\pgfqpoint{5.056582in}{2.273603in}}%
\pgfpathlineto{\pgfqpoint{5.065600in}{2.317242in}}%
\pgfpathlineto{\pgfqpoint{5.070109in}{2.305818in}}%
\pgfpathlineto{\pgfqpoint{5.074618in}{2.311545in}}%
\pgfpathlineto{\pgfqpoint{5.079127in}{2.303922in}}%
\pgfpathlineto{\pgfqpoint{5.083636in}{2.234557in}}%
\pgfpathlineto{\pgfqpoint{5.088145in}{2.249969in}}%
\pgfpathlineto{\pgfqpoint{5.092655in}{2.283494in}}%
\pgfpathlineto{\pgfqpoint{5.097164in}{2.280021in}}%
\pgfpathlineto{\pgfqpoint{5.101673in}{2.317627in}}%
\pgfpathlineto{\pgfqpoint{5.106182in}{2.248641in}}%
\pgfpathlineto{\pgfqpoint{5.110691in}{2.255146in}}%
\pgfpathlineto{\pgfqpoint{5.115200in}{2.539499in}}%
\pgfpathlineto{\pgfqpoint{5.119709in}{2.207647in}}%
\pgfpathlineto{\pgfqpoint{5.124218in}{2.563998in}}%
\pgfpathlineto{\pgfqpoint{5.128727in}{2.243901in}}%
\pgfpathlineto{\pgfqpoint{5.133236in}{2.253222in}}%
\pgfpathlineto{\pgfqpoint{5.137745in}{2.320462in}}%
\pgfpathlineto{\pgfqpoint{5.142255in}{2.331703in}}%
\pgfpathlineto{\pgfqpoint{5.146764in}{2.521974in}}%
\pgfpathlineto{\pgfqpoint{5.151273in}{4.056000in}}%
\pgfpathlineto{\pgfqpoint{5.155782in}{3.255003in}}%
\pgfpathlineto{\pgfqpoint{5.160291in}{3.814733in}}%
\pgfpathlineto{\pgfqpoint{5.164800in}{2.841482in}}%
\pgfpathlineto{\pgfqpoint{5.169309in}{2.425249in}}%
\pgfpathlineto{\pgfqpoint{5.173818in}{2.513960in}}%
\pgfpathlineto{\pgfqpoint{5.178327in}{2.450060in}}%
\pgfpathlineto{\pgfqpoint{5.187345in}{2.295987in}}%
\pgfpathlineto{\pgfqpoint{5.191855in}{2.410134in}}%
\pgfpathlineto{\pgfqpoint{5.196364in}{2.398178in}}%
\pgfpathlineto{\pgfqpoint{5.200873in}{2.425858in}}%
\pgfpathlineto{\pgfqpoint{5.205382in}{2.494781in}}%
\pgfpathlineto{\pgfqpoint{5.214400in}{2.389539in}}%
\pgfpathlineto{\pgfqpoint{5.218909in}{2.450559in}}%
\pgfpathlineto{\pgfqpoint{5.223418in}{2.351597in}}%
\pgfpathlineto{\pgfqpoint{5.227927in}{2.400590in}}%
\pgfpathlineto{\pgfqpoint{5.232436in}{2.474520in}}%
\pgfpathlineto{\pgfqpoint{5.241455in}{2.424915in}}%
\pgfpathlineto{\pgfqpoint{5.245964in}{2.418853in}}%
\pgfpathlineto{\pgfqpoint{5.250473in}{3.442321in}}%
\pgfpathlineto{\pgfqpoint{5.254982in}{2.407572in}}%
\pgfpathlineto{\pgfqpoint{5.259491in}{3.409553in}}%
\pgfpathlineto{\pgfqpoint{5.264000in}{3.063563in}}%
\pgfpathlineto{\pgfqpoint{5.268509in}{3.251057in}}%
\pgfpathlineto{\pgfqpoint{5.273018in}{2.476744in}}%
\pgfpathlineto{\pgfqpoint{5.277527in}{2.430574in}}%
\pgfpathlineto{\pgfqpoint{5.282036in}{2.561550in}}%
\pgfpathlineto{\pgfqpoint{5.286545in}{2.435137in}}%
\pgfpathlineto{\pgfqpoint{5.291055in}{2.581973in}}%
\pgfpathlineto{\pgfqpoint{5.295564in}{2.471800in}}%
\pgfpathlineto{\pgfqpoint{5.300073in}{2.461246in}}%
\pgfpathlineto{\pgfqpoint{5.304582in}{2.541789in}}%
\pgfpathlineto{\pgfqpoint{5.309091in}{2.430637in}}%
\pgfpathlineto{\pgfqpoint{5.313600in}{2.474787in}}%
\pgfpathlineto{\pgfqpoint{5.318109in}{2.455625in}}%
\pgfpathlineto{\pgfqpoint{5.322618in}{2.471681in}}%
\pgfpathlineto{\pgfqpoint{5.327127in}{2.482080in}}%
\pgfpathlineto{\pgfqpoint{5.331636in}{2.478762in}}%
\pgfpathlineto{\pgfqpoint{5.336145in}{2.615317in}}%
\pgfpathlineto{\pgfqpoint{5.340655in}{2.429752in}}%
\pgfpathlineto{\pgfqpoint{5.345164in}{2.478870in}}%
\pgfpathlineto{\pgfqpoint{5.349673in}{2.435506in}}%
\pgfpathlineto{\pgfqpoint{5.354182in}{2.513567in}}%
\pgfpathlineto{\pgfqpoint{5.358691in}{2.566574in}}%
\pgfpathlineto{\pgfqpoint{5.363200in}{2.511811in}}%
\pgfpathlineto{\pgfqpoint{5.367709in}{2.511127in}}%
\pgfpathlineto{\pgfqpoint{5.372218in}{2.442923in}}%
\pgfpathlineto{\pgfqpoint{5.376727in}{2.560803in}}%
\pgfpathlineto{\pgfqpoint{5.381236in}{2.489629in}}%
\pgfpathlineto{\pgfqpoint{5.385745in}{2.491192in}}%
\pgfpathlineto{\pgfqpoint{5.390255in}{2.383216in}}%
\pgfpathlineto{\pgfqpoint{5.394764in}{2.504686in}}%
\pgfpathlineto{\pgfqpoint{5.399273in}{2.508763in}}%
\pgfpathlineto{\pgfqpoint{5.403782in}{2.709049in}}%
\pgfpathlineto{\pgfqpoint{5.408291in}{2.507401in}}%
\pgfpathlineto{\pgfqpoint{5.412800in}{2.436675in}}%
\pgfpathlineto{\pgfqpoint{5.417309in}{2.580282in}}%
\pgfpathlineto{\pgfqpoint{5.421818in}{2.545393in}}%
\pgfpathlineto{\pgfqpoint{5.426327in}{2.610511in}}%
\pgfpathlineto{\pgfqpoint{5.430836in}{2.706990in}}%
\pgfpathlineto{\pgfqpoint{5.435345in}{2.544253in}}%
\pgfpathlineto{\pgfqpoint{5.444364in}{2.510442in}}%
\pgfpathlineto{\pgfqpoint{5.448873in}{2.587989in}}%
\pgfpathlineto{\pgfqpoint{5.453382in}{2.512640in}}%
\pgfpathlineto{\pgfqpoint{5.457891in}{2.494870in}}%
\pgfpathlineto{\pgfqpoint{5.462400in}{2.618472in}}%
\pgfpathlineto{\pgfqpoint{5.466909in}{2.657392in}}%
\pgfpathlineto{\pgfqpoint{5.471418in}{2.617338in}}%
\pgfpathlineto{\pgfqpoint{5.475927in}{2.520984in}}%
\pgfpathlineto{\pgfqpoint{5.480436in}{2.547914in}}%
\pgfpathlineto{\pgfqpoint{5.484945in}{2.530985in}}%
\pgfpathlineto{\pgfqpoint{5.489455in}{2.647348in}}%
\pgfpathlineto{\pgfqpoint{5.493964in}{2.520556in}}%
\pgfpathlineto{\pgfqpoint{5.498473in}{2.474483in}}%
\pgfpathlineto{\pgfqpoint{5.502982in}{2.770698in}}%
\pgfpathlineto{\pgfqpoint{5.507491in}{2.687990in}}%
\pgfpathlineto{\pgfqpoint{5.512000in}{2.711290in}}%
\pgfpathlineto{\pgfqpoint{5.516509in}{2.568876in}}%
\pgfpathlineto{\pgfqpoint{5.521018in}{2.681145in}}%
\pgfpathlineto{\pgfqpoint{5.525527in}{2.545236in}}%
\pgfpathlineto{\pgfqpoint{5.530036in}{2.677729in}}%
\pgfpathlineto{\pgfqpoint{5.534545in}{2.762529in}}%
\pgfpathlineto{\pgfqpoint{5.534545in}{2.762529in}}%
\pgfusepath{stroke}%
\end{pgfscope}%
\begin{pgfscope}%
\pgfsetrectcap%
\pgfsetmiterjoin%
\pgfsetlinewidth{0.803000pt}%
\definecolor{currentstroke}{rgb}{0.000000,0.000000,0.000000}%
\pgfsetstrokecolor{currentstroke}%
\pgfsetdash{}{0pt}%
\pgfpathmoveto{\pgfqpoint{0.800000in}{0.528000in}}%
\pgfpathlineto{\pgfqpoint{0.800000in}{4.224000in}}%
\pgfusepath{stroke}%
\end{pgfscope}%
\begin{pgfscope}%
\pgfsetrectcap%
\pgfsetmiterjoin%
\pgfsetlinewidth{0.803000pt}%
\definecolor{currentstroke}{rgb}{0.000000,0.000000,0.000000}%
\pgfsetstrokecolor{currentstroke}%
\pgfsetdash{}{0pt}%
\pgfpathmoveto{\pgfqpoint{5.760000in}{0.528000in}}%
\pgfpathlineto{\pgfqpoint{5.760000in}{4.224000in}}%
\pgfusepath{stroke}%
\end{pgfscope}%
\begin{pgfscope}%
\pgfsetrectcap%
\pgfsetmiterjoin%
\pgfsetlinewidth{0.803000pt}%
\definecolor{currentstroke}{rgb}{0.000000,0.000000,0.000000}%
\pgfsetstrokecolor{currentstroke}%
\pgfsetdash{}{0pt}%
\pgfpathmoveto{\pgfqpoint{0.800000in}{0.528000in}}%
\pgfpathlineto{\pgfqpoint{5.760000in}{0.528000in}}%
\pgfusepath{stroke}%
\end{pgfscope}%
\begin{pgfscope}%
\pgfsetrectcap%
\pgfsetmiterjoin%
\pgfsetlinewidth{0.803000pt}%
\definecolor{currentstroke}{rgb}{0.000000,0.000000,0.000000}%
\pgfsetstrokecolor{currentstroke}%
\pgfsetdash{}{0pt}%
\pgfpathmoveto{\pgfqpoint{0.800000in}{4.224000in}}%
\pgfpathlineto{\pgfqpoint{5.760000in}{4.224000in}}%
\pgfusepath{stroke}%
\end{pgfscope}%
\begin{pgfscope}%
\definecolor{textcolor}{rgb}{0.000000,0.000000,0.000000}%
\pgfsetstrokecolor{textcolor}%
\pgfsetfillcolor{textcolor}%
\pgftext[x=3.280000in,y=4.307333in,,base]{\color{textcolor}\ttfamily\fontsize{12.000000}{14.400000}\selectfont Insertion Sort Time vs Input size}%
\end{pgfscope}%
\begin{pgfscope}%
\pgfsetbuttcap%
\pgfsetmiterjoin%
\definecolor{currentfill}{rgb}{1.000000,1.000000,1.000000}%
\pgfsetfillcolor{currentfill}%
\pgfsetfillopacity{0.800000}%
\pgfsetlinewidth{1.003750pt}%
\definecolor{currentstroke}{rgb}{0.800000,0.800000,0.800000}%
\pgfsetstrokecolor{currentstroke}%
\pgfsetstrokeopacity{0.800000}%
\pgfsetdash{}{0pt}%
\pgfpathmoveto{\pgfqpoint{0.897222in}{3.908286in}}%
\pgfpathlineto{\pgfqpoint{2.094230in}{3.908286in}}%
\pgfpathquadraticcurveto{\pgfqpoint{2.122008in}{3.908286in}}{\pgfqpoint{2.122008in}{3.936063in}}%
\pgfpathlineto{\pgfqpoint{2.122008in}{4.126778in}}%
\pgfpathquadraticcurveto{\pgfqpoint{2.122008in}{4.154556in}}{\pgfqpoint{2.094230in}{4.154556in}}%
\pgfpathlineto{\pgfqpoint{0.897222in}{4.154556in}}%
\pgfpathquadraticcurveto{\pgfqpoint{0.869444in}{4.154556in}}{\pgfqpoint{0.869444in}{4.126778in}}%
\pgfpathlineto{\pgfqpoint{0.869444in}{3.936063in}}%
\pgfpathquadraticcurveto{\pgfqpoint{0.869444in}{3.908286in}}{\pgfqpoint{0.897222in}{3.908286in}}%
\pgfpathlineto{\pgfqpoint{0.897222in}{3.908286in}}%
\pgfpathclose%
\pgfusepath{stroke,fill}%
\end{pgfscope}%
\begin{pgfscope}%
\pgfsetrectcap%
\pgfsetroundjoin%
\pgfsetlinewidth{1.505625pt}%
\definecolor{currentstroke}{rgb}{0.000000,1.000000,0.498039}%
\pgfsetstrokecolor{currentstroke}%
\pgfsetdash{}{0pt}%
\pgfpathmoveto{\pgfqpoint{0.925000in}{4.041342in}}%
\pgfpathlineto{\pgfqpoint{1.063889in}{4.041342in}}%
\pgfpathlineto{\pgfqpoint{1.202778in}{4.041342in}}%
\pgfusepath{stroke}%
\end{pgfscope}%
\begin{pgfscope}%
\definecolor{textcolor}{rgb}{0.000000,0.000000,0.000000}%
\pgfsetstrokecolor{textcolor}%
\pgfsetfillcolor{textcolor}%
\pgftext[x=1.313889in,y=3.992731in,left,base]{\color{textcolor}\ttfamily\fontsize{10.000000}{12.000000}\selectfont Insertion}%
\end{pgfscope}%
\end{pgfpicture}%
\makeatother%
\endgroup%

%% Creator: Matplotlib, PGF backend
%%
%% To include the figure in your LaTeX document, write
%%   \input{<filename>.pgf}
%%
%% Make sure the required packages are loaded in your preamble
%%   \usepackage{pgf}
%%
%% Also ensure that all the required font packages are loaded; for instance,
%% the lmodern package is sometimes necessary when using math font.
%%   \usepackage{lmodern}
%%
%% Figures using additional raster images can only be included by \input if
%% they are in the same directory as the main LaTeX file. For loading figures
%% from other directories you can use the `import` package
%%   \usepackage{import}
%%
%% and then include the figures with
%%   \import{<path to file>}{<filename>.pgf}
%%
%% Matplotlib used the following preamble
%%   \usepackage{fontspec}
%%   \setmainfont{DejaVuSerif.ttf}[Path=\detokenize{/home/dbk/.local/lib/python3.10/site-packages/matplotlib/mpl-data/fonts/ttf/}]
%%   \setsansfont{DejaVuSans.ttf}[Path=\detokenize{/home/dbk/.local/lib/python3.10/site-packages/matplotlib/mpl-data/fonts/ttf/}]
%%   \setmonofont{DejaVuSansMono.ttf}[Path=\detokenize{/home/dbk/.local/lib/python3.10/site-packages/matplotlib/mpl-data/fonts/ttf/}]
%%
\begingroup%
\makeatletter%
\begin{pgfpicture}%
\pgfpathrectangle{\pgfpointorigin}{\pgfqpoint{6.400000in}{4.800000in}}%
\pgfusepath{use as bounding box, clip}%
\begin{pgfscope}%
\pgfsetbuttcap%
\pgfsetmiterjoin%
\definecolor{currentfill}{rgb}{1.000000,1.000000,1.000000}%
\pgfsetfillcolor{currentfill}%
\pgfsetlinewidth{0.000000pt}%
\definecolor{currentstroke}{rgb}{1.000000,1.000000,1.000000}%
\pgfsetstrokecolor{currentstroke}%
\pgfsetdash{}{0pt}%
\pgfpathmoveto{\pgfqpoint{0.000000in}{0.000000in}}%
\pgfpathlineto{\pgfqpoint{6.400000in}{0.000000in}}%
\pgfpathlineto{\pgfqpoint{6.400000in}{4.800000in}}%
\pgfpathlineto{\pgfqpoint{0.000000in}{4.800000in}}%
\pgfpathlineto{\pgfqpoint{0.000000in}{0.000000in}}%
\pgfpathclose%
\pgfusepath{fill}%
\end{pgfscope}%
\begin{pgfscope}%
\pgfsetbuttcap%
\pgfsetmiterjoin%
\definecolor{currentfill}{rgb}{1.000000,1.000000,1.000000}%
\pgfsetfillcolor{currentfill}%
\pgfsetlinewidth{0.000000pt}%
\definecolor{currentstroke}{rgb}{0.000000,0.000000,0.000000}%
\pgfsetstrokecolor{currentstroke}%
\pgfsetstrokeopacity{0.000000}%
\pgfsetdash{}{0pt}%
\pgfpathmoveto{\pgfqpoint{0.800000in}{0.528000in}}%
\pgfpathlineto{\pgfqpoint{5.760000in}{0.528000in}}%
\pgfpathlineto{\pgfqpoint{5.760000in}{4.224000in}}%
\pgfpathlineto{\pgfqpoint{0.800000in}{4.224000in}}%
\pgfpathlineto{\pgfqpoint{0.800000in}{0.528000in}}%
\pgfpathclose%
\pgfusepath{fill}%
\end{pgfscope}%
\begin{pgfscope}%
\pgfsetbuttcap%
\pgfsetroundjoin%
\definecolor{currentfill}{rgb}{0.000000,0.000000,0.000000}%
\pgfsetfillcolor{currentfill}%
\pgfsetlinewidth{0.803000pt}%
\definecolor{currentstroke}{rgb}{0.000000,0.000000,0.000000}%
\pgfsetstrokecolor{currentstroke}%
\pgfsetdash{}{0pt}%
\pgfsys@defobject{currentmarker}{\pgfqpoint{0.000000in}{-0.048611in}}{\pgfqpoint{0.000000in}{0.000000in}}{%
\pgfpathmoveto{\pgfqpoint{0.000000in}{0.000000in}}%
\pgfpathlineto{\pgfqpoint{0.000000in}{-0.048611in}}%
\pgfusepath{stroke,fill}%
}%
\begin{pgfscope}%
\pgfsys@transformshift{1.020945in}{0.528000in}%
\pgfsys@useobject{currentmarker}{}%
\end{pgfscope}%
\end{pgfscope}%
\begin{pgfscope}%
\definecolor{textcolor}{rgb}{0.000000,0.000000,0.000000}%
\pgfsetstrokecolor{textcolor}%
\pgfsetfillcolor{textcolor}%
\pgftext[x=1.020945in,y=0.430778in,,top]{\color{textcolor}\ttfamily\fontsize{10.000000}{12.000000}\selectfont 0}%
\end{pgfscope}%
\begin{pgfscope}%
\pgfsetbuttcap%
\pgfsetroundjoin%
\definecolor{currentfill}{rgb}{0.000000,0.000000,0.000000}%
\pgfsetfillcolor{currentfill}%
\pgfsetlinewidth{0.803000pt}%
\definecolor{currentstroke}{rgb}{0.000000,0.000000,0.000000}%
\pgfsetstrokecolor{currentstroke}%
\pgfsetdash{}{0pt}%
\pgfsys@defobject{currentmarker}{\pgfqpoint{0.000000in}{-0.048611in}}{\pgfqpoint{0.000000in}{0.000000in}}{%
\pgfpathmoveto{\pgfqpoint{0.000000in}{0.000000in}}%
\pgfpathlineto{\pgfqpoint{0.000000in}{-0.048611in}}%
\pgfusepath{stroke,fill}%
}%
\begin{pgfscope}%
\pgfsys@transformshift{1.922764in}{0.528000in}%
\pgfsys@useobject{currentmarker}{}%
\end{pgfscope}%
\end{pgfscope}%
\begin{pgfscope}%
\definecolor{textcolor}{rgb}{0.000000,0.000000,0.000000}%
\pgfsetstrokecolor{textcolor}%
\pgfsetfillcolor{textcolor}%
\pgftext[x=1.922764in,y=0.430778in,,top]{\color{textcolor}\ttfamily\fontsize{10.000000}{12.000000}\selectfont 200}%
\end{pgfscope}%
\begin{pgfscope}%
\pgfsetbuttcap%
\pgfsetroundjoin%
\definecolor{currentfill}{rgb}{0.000000,0.000000,0.000000}%
\pgfsetfillcolor{currentfill}%
\pgfsetlinewidth{0.803000pt}%
\definecolor{currentstroke}{rgb}{0.000000,0.000000,0.000000}%
\pgfsetstrokecolor{currentstroke}%
\pgfsetdash{}{0pt}%
\pgfsys@defobject{currentmarker}{\pgfqpoint{0.000000in}{-0.048611in}}{\pgfqpoint{0.000000in}{0.000000in}}{%
\pgfpathmoveto{\pgfqpoint{0.000000in}{0.000000in}}%
\pgfpathlineto{\pgfqpoint{0.000000in}{-0.048611in}}%
\pgfusepath{stroke,fill}%
}%
\begin{pgfscope}%
\pgfsys@transformshift{2.824582in}{0.528000in}%
\pgfsys@useobject{currentmarker}{}%
\end{pgfscope}%
\end{pgfscope}%
\begin{pgfscope}%
\definecolor{textcolor}{rgb}{0.000000,0.000000,0.000000}%
\pgfsetstrokecolor{textcolor}%
\pgfsetfillcolor{textcolor}%
\pgftext[x=2.824582in,y=0.430778in,,top]{\color{textcolor}\ttfamily\fontsize{10.000000}{12.000000}\selectfont 400}%
\end{pgfscope}%
\begin{pgfscope}%
\pgfsetbuttcap%
\pgfsetroundjoin%
\definecolor{currentfill}{rgb}{0.000000,0.000000,0.000000}%
\pgfsetfillcolor{currentfill}%
\pgfsetlinewidth{0.803000pt}%
\definecolor{currentstroke}{rgb}{0.000000,0.000000,0.000000}%
\pgfsetstrokecolor{currentstroke}%
\pgfsetdash{}{0pt}%
\pgfsys@defobject{currentmarker}{\pgfqpoint{0.000000in}{-0.048611in}}{\pgfqpoint{0.000000in}{0.000000in}}{%
\pgfpathmoveto{\pgfqpoint{0.000000in}{0.000000in}}%
\pgfpathlineto{\pgfqpoint{0.000000in}{-0.048611in}}%
\pgfusepath{stroke,fill}%
}%
\begin{pgfscope}%
\pgfsys@transformshift{3.726400in}{0.528000in}%
\pgfsys@useobject{currentmarker}{}%
\end{pgfscope}%
\end{pgfscope}%
\begin{pgfscope}%
\definecolor{textcolor}{rgb}{0.000000,0.000000,0.000000}%
\pgfsetstrokecolor{textcolor}%
\pgfsetfillcolor{textcolor}%
\pgftext[x=3.726400in,y=0.430778in,,top]{\color{textcolor}\ttfamily\fontsize{10.000000}{12.000000}\selectfont 600}%
\end{pgfscope}%
\begin{pgfscope}%
\pgfsetbuttcap%
\pgfsetroundjoin%
\definecolor{currentfill}{rgb}{0.000000,0.000000,0.000000}%
\pgfsetfillcolor{currentfill}%
\pgfsetlinewidth{0.803000pt}%
\definecolor{currentstroke}{rgb}{0.000000,0.000000,0.000000}%
\pgfsetstrokecolor{currentstroke}%
\pgfsetdash{}{0pt}%
\pgfsys@defobject{currentmarker}{\pgfqpoint{0.000000in}{-0.048611in}}{\pgfqpoint{0.000000in}{0.000000in}}{%
\pgfpathmoveto{\pgfqpoint{0.000000in}{0.000000in}}%
\pgfpathlineto{\pgfqpoint{0.000000in}{-0.048611in}}%
\pgfusepath{stroke,fill}%
}%
\begin{pgfscope}%
\pgfsys@transformshift{4.628218in}{0.528000in}%
\pgfsys@useobject{currentmarker}{}%
\end{pgfscope}%
\end{pgfscope}%
\begin{pgfscope}%
\definecolor{textcolor}{rgb}{0.000000,0.000000,0.000000}%
\pgfsetstrokecolor{textcolor}%
\pgfsetfillcolor{textcolor}%
\pgftext[x=4.628218in,y=0.430778in,,top]{\color{textcolor}\ttfamily\fontsize{10.000000}{12.000000}\selectfont 800}%
\end{pgfscope}%
\begin{pgfscope}%
\pgfsetbuttcap%
\pgfsetroundjoin%
\definecolor{currentfill}{rgb}{0.000000,0.000000,0.000000}%
\pgfsetfillcolor{currentfill}%
\pgfsetlinewidth{0.803000pt}%
\definecolor{currentstroke}{rgb}{0.000000,0.000000,0.000000}%
\pgfsetstrokecolor{currentstroke}%
\pgfsetdash{}{0pt}%
\pgfsys@defobject{currentmarker}{\pgfqpoint{0.000000in}{-0.048611in}}{\pgfqpoint{0.000000in}{0.000000in}}{%
\pgfpathmoveto{\pgfqpoint{0.000000in}{0.000000in}}%
\pgfpathlineto{\pgfqpoint{0.000000in}{-0.048611in}}%
\pgfusepath{stroke,fill}%
}%
\begin{pgfscope}%
\pgfsys@transformshift{5.530036in}{0.528000in}%
\pgfsys@useobject{currentmarker}{}%
\end{pgfscope}%
\end{pgfscope}%
\begin{pgfscope}%
\definecolor{textcolor}{rgb}{0.000000,0.000000,0.000000}%
\pgfsetstrokecolor{textcolor}%
\pgfsetfillcolor{textcolor}%
\pgftext[x=5.530036in,y=0.430778in,,top]{\color{textcolor}\ttfamily\fontsize{10.000000}{12.000000}\selectfont 1000}%
\end{pgfscope}%
\begin{pgfscope}%
\definecolor{textcolor}{rgb}{0.000000,0.000000,0.000000}%
\pgfsetstrokecolor{textcolor}%
\pgfsetfillcolor{textcolor}%
\pgftext[x=3.280000in,y=0.240063in,,top]{\color{textcolor}\ttfamily\fontsize{10.000000}{12.000000}\selectfont Size of Array}%
\end{pgfscope}%
\begin{pgfscope}%
\pgfsetbuttcap%
\pgfsetroundjoin%
\definecolor{currentfill}{rgb}{0.000000,0.000000,0.000000}%
\pgfsetfillcolor{currentfill}%
\pgfsetlinewidth{0.803000pt}%
\definecolor{currentstroke}{rgb}{0.000000,0.000000,0.000000}%
\pgfsetstrokecolor{currentstroke}%
\pgfsetdash{}{0pt}%
\pgfsys@defobject{currentmarker}{\pgfqpoint{-0.048611in}{0.000000in}}{\pgfqpoint{-0.000000in}{0.000000in}}{%
\pgfpathmoveto{\pgfqpoint{-0.000000in}{0.000000in}}%
\pgfpathlineto{\pgfqpoint{-0.048611in}{0.000000in}}%
\pgfusepath{stroke,fill}%
}%
\begin{pgfscope}%
\pgfsys@transformshift{0.800000in}{1.069333in}%
\pgfsys@useobject{currentmarker}{}%
\end{pgfscope}%
\end{pgfscope}%
\begin{pgfscope}%
\definecolor{textcolor}{rgb}{0.000000,0.000000,0.000000}%
\pgfsetstrokecolor{textcolor}%
\pgfsetfillcolor{textcolor}%
\pgftext[x=0.451923in, y=1.016199in, left, base]{\color{textcolor}\ttfamily\fontsize{10.000000}{12.000000}\selectfont 200}%
\end{pgfscope}%
\begin{pgfscope}%
\pgfsetbuttcap%
\pgfsetroundjoin%
\definecolor{currentfill}{rgb}{0.000000,0.000000,0.000000}%
\pgfsetfillcolor{currentfill}%
\pgfsetlinewidth{0.803000pt}%
\definecolor{currentstroke}{rgb}{0.000000,0.000000,0.000000}%
\pgfsetstrokecolor{currentstroke}%
\pgfsetdash{}{0pt}%
\pgfsys@defobject{currentmarker}{\pgfqpoint{-0.048611in}{0.000000in}}{\pgfqpoint{-0.000000in}{0.000000in}}{%
\pgfpathmoveto{\pgfqpoint{-0.000000in}{0.000000in}}%
\pgfpathlineto{\pgfqpoint{-0.048611in}{0.000000in}}%
\pgfusepath{stroke,fill}%
}%
\begin{pgfscope}%
\pgfsys@transformshift{0.800000in}{1.691556in}%
\pgfsys@useobject{currentmarker}{}%
\end{pgfscope}%
\end{pgfscope}%
\begin{pgfscope}%
\definecolor{textcolor}{rgb}{0.000000,0.000000,0.000000}%
\pgfsetstrokecolor{textcolor}%
\pgfsetfillcolor{textcolor}%
\pgftext[x=0.451923in, y=1.638421in, left, base]{\color{textcolor}\ttfamily\fontsize{10.000000}{12.000000}\selectfont 220}%
\end{pgfscope}%
\begin{pgfscope}%
\pgfsetbuttcap%
\pgfsetroundjoin%
\definecolor{currentfill}{rgb}{0.000000,0.000000,0.000000}%
\pgfsetfillcolor{currentfill}%
\pgfsetlinewidth{0.803000pt}%
\definecolor{currentstroke}{rgb}{0.000000,0.000000,0.000000}%
\pgfsetstrokecolor{currentstroke}%
\pgfsetdash{}{0pt}%
\pgfsys@defobject{currentmarker}{\pgfqpoint{-0.048611in}{0.000000in}}{\pgfqpoint{-0.000000in}{0.000000in}}{%
\pgfpathmoveto{\pgfqpoint{-0.000000in}{0.000000in}}%
\pgfpathlineto{\pgfqpoint{-0.048611in}{0.000000in}}%
\pgfusepath{stroke,fill}%
}%
\begin{pgfscope}%
\pgfsys@transformshift{0.800000in}{2.313778in}%
\pgfsys@useobject{currentmarker}{}%
\end{pgfscope}%
\end{pgfscope}%
\begin{pgfscope}%
\definecolor{textcolor}{rgb}{0.000000,0.000000,0.000000}%
\pgfsetstrokecolor{textcolor}%
\pgfsetfillcolor{textcolor}%
\pgftext[x=0.451923in, y=2.260643in, left, base]{\color{textcolor}\ttfamily\fontsize{10.000000}{12.000000}\selectfont 240}%
\end{pgfscope}%
\begin{pgfscope}%
\pgfsetbuttcap%
\pgfsetroundjoin%
\definecolor{currentfill}{rgb}{0.000000,0.000000,0.000000}%
\pgfsetfillcolor{currentfill}%
\pgfsetlinewidth{0.803000pt}%
\definecolor{currentstroke}{rgb}{0.000000,0.000000,0.000000}%
\pgfsetstrokecolor{currentstroke}%
\pgfsetdash{}{0pt}%
\pgfsys@defobject{currentmarker}{\pgfqpoint{-0.048611in}{0.000000in}}{\pgfqpoint{-0.000000in}{0.000000in}}{%
\pgfpathmoveto{\pgfqpoint{-0.000000in}{0.000000in}}%
\pgfpathlineto{\pgfqpoint{-0.048611in}{0.000000in}}%
\pgfusepath{stroke,fill}%
}%
\begin{pgfscope}%
\pgfsys@transformshift{0.800000in}{2.936000in}%
\pgfsys@useobject{currentmarker}{}%
\end{pgfscope}%
\end{pgfscope}%
\begin{pgfscope}%
\definecolor{textcolor}{rgb}{0.000000,0.000000,0.000000}%
\pgfsetstrokecolor{textcolor}%
\pgfsetfillcolor{textcolor}%
\pgftext[x=0.451923in, y=2.882865in, left, base]{\color{textcolor}\ttfamily\fontsize{10.000000}{12.000000}\selectfont 260}%
\end{pgfscope}%
\begin{pgfscope}%
\pgfsetbuttcap%
\pgfsetroundjoin%
\definecolor{currentfill}{rgb}{0.000000,0.000000,0.000000}%
\pgfsetfillcolor{currentfill}%
\pgfsetlinewidth{0.803000pt}%
\definecolor{currentstroke}{rgb}{0.000000,0.000000,0.000000}%
\pgfsetstrokecolor{currentstroke}%
\pgfsetdash{}{0pt}%
\pgfsys@defobject{currentmarker}{\pgfqpoint{-0.048611in}{0.000000in}}{\pgfqpoint{-0.000000in}{0.000000in}}{%
\pgfpathmoveto{\pgfqpoint{-0.000000in}{0.000000in}}%
\pgfpathlineto{\pgfqpoint{-0.048611in}{0.000000in}}%
\pgfusepath{stroke,fill}%
}%
\begin{pgfscope}%
\pgfsys@transformshift{0.800000in}{3.558222in}%
\pgfsys@useobject{currentmarker}{}%
\end{pgfscope}%
\end{pgfscope}%
\begin{pgfscope}%
\definecolor{textcolor}{rgb}{0.000000,0.000000,0.000000}%
\pgfsetstrokecolor{textcolor}%
\pgfsetfillcolor{textcolor}%
\pgftext[x=0.451923in, y=3.505088in, left, base]{\color{textcolor}\ttfamily\fontsize{10.000000}{12.000000}\selectfont 280}%
\end{pgfscope}%
\begin{pgfscope}%
\pgfsetbuttcap%
\pgfsetroundjoin%
\definecolor{currentfill}{rgb}{0.000000,0.000000,0.000000}%
\pgfsetfillcolor{currentfill}%
\pgfsetlinewidth{0.803000pt}%
\definecolor{currentstroke}{rgb}{0.000000,0.000000,0.000000}%
\pgfsetstrokecolor{currentstroke}%
\pgfsetdash{}{0pt}%
\pgfsys@defobject{currentmarker}{\pgfqpoint{-0.048611in}{0.000000in}}{\pgfqpoint{-0.000000in}{0.000000in}}{%
\pgfpathmoveto{\pgfqpoint{-0.000000in}{0.000000in}}%
\pgfpathlineto{\pgfqpoint{-0.048611in}{0.000000in}}%
\pgfusepath{stroke,fill}%
}%
\begin{pgfscope}%
\pgfsys@transformshift{0.800000in}{4.180444in}%
\pgfsys@useobject{currentmarker}{}%
\end{pgfscope}%
\end{pgfscope}%
\begin{pgfscope}%
\definecolor{textcolor}{rgb}{0.000000,0.000000,0.000000}%
\pgfsetstrokecolor{textcolor}%
\pgfsetfillcolor{textcolor}%
\pgftext[x=0.451923in, y=4.127310in, left, base]{\color{textcolor}\ttfamily\fontsize{10.000000}{12.000000}\selectfont 300}%
\end{pgfscope}%
\begin{pgfscope}%
\definecolor{textcolor}{rgb}{0.000000,0.000000,0.000000}%
\pgfsetstrokecolor{textcolor}%
\pgfsetfillcolor{textcolor}%
\pgftext[x=0.396368in,y=2.376000in,,bottom,rotate=90.000000]{\color{textcolor}\ttfamily\fontsize{10.000000}{12.000000}\selectfont Memory}%
\end{pgfscope}%
\begin{pgfscope}%
\pgfpathrectangle{\pgfqpoint{0.800000in}{0.528000in}}{\pgfqpoint{4.960000in}{3.696000in}}%
\pgfusepath{clip}%
\pgfsetrectcap%
\pgfsetroundjoin%
\pgfsetlinewidth{1.505625pt}%
\definecolor{currentstroke}{rgb}{0.000000,1.000000,0.498039}%
\pgfsetstrokecolor{currentstroke}%
\pgfsetdash{}{0pt}%
\pgfpathmoveto{\pgfqpoint{1.025455in}{0.696000in}}%
\pgfpathlineto{\pgfqpoint{1.733382in}{0.696000in}}%
\pgfpathlineto{\pgfqpoint{1.737891in}{3.184889in}}%
\pgfpathlineto{\pgfqpoint{1.742400in}{4.056000in}}%
\pgfpathlineto{\pgfqpoint{5.534545in}{4.056000in}}%
\pgfpathlineto{\pgfqpoint{5.534545in}{4.056000in}}%
\pgfusepath{stroke}%
\end{pgfscope}%
\begin{pgfscope}%
\pgfsetrectcap%
\pgfsetmiterjoin%
\pgfsetlinewidth{0.803000pt}%
\definecolor{currentstroke}{rgb}{0.000000,0.000000,0.000000}%
\pgfsetstrokecolor{currentstroke}%
\pgfsetdash{}{0pt}%
\pgfpathmoveto{\pgfqpoint{0.800000in}{0.528000in}}%
\pgfpathlineto{\pgfqpoint{0.800000in}{4.224000in}}%
\pgfusepath{stroke}%
\end{pgfscope}%
\begin{pgfscope}%
\pgfsetrectcap%
\pgfsetmiterjoin%
\pgfsetlinewidth{0.803000pt}%
\definecolor{currentstroke}{rgb}{0.000000,0.000000,0.000000}%
\pgfsetstrokecolor{currentstroke}%
\pgfsetdash{}{0pt}%
\pgfpathmoveto{\pgfqpoint{5.760000in}{0.528000in}}%
\pgfpathlineto{\pgfqpoint{5.760000in}{4.224000in}}%
\pgfusepath{stroke}%
\end{pgfscope}%
\begin{pgfscope}%
\pgfsetrectcap%
\pgfsetmiterjoin%
\pgfsetlinewidth{0.803000pt}%
\definecolor{currentstroke}{rgb}{0.000000,0.000000,0.000000}%
\pgfsetstrokecolor{currentstroke}%
\pgfsetdash{}{0pt}%
\pgfpathmoveto{\pgfqpoint{0.800000in}{0.528000in}}%
\pgfpathlineto{\pgfqpoint{5.760000in}{0.528000in}}%
\pgfusepath{stroke}%
\end{pgfscope}%
\begin{pgfscope}%
\pgfsetrectcap%
\pgfsetmiterjoin%
\pgfsetlinewidth{0.803000pt}%
\definecolor{currentstroke}{rgb}{0.000000,0.000000,0.000000}%
\pgfsetstrokecolor{currentstroke}%
\pgfsetdash{}{0pt}%
\pgfpathmoveto{\pgfqpoint{0.800000in}{4.224000in}}%
\pgfpathlineto{\pgfqpoint{5.760000in}{4.224000in}}%
\pgfusepath{stroke}%
\end{pgfscope}%
\begin{pgfscope}%
\definecolor{textcolor}{rgb}{0.000000,0.000000,0.000000}%
\pgfsetstrokecolor{textcolor}%
\pgfsetfillcolor{textcolor}%
\pgftext[x=3.280000in,y=4.307333in,,base]{\color{textcolor}\ttfamily\fontsize{12.000000}{14.400000}\selectfont Insertion Sort Memory vs Input size}%
\end{pgfscope}%
\begin{pgfscope}%
\pgfsetbuttcap%
\pgfsetmiterjoin%
\definecolor{currentfill}{rgb}{1.000000,1.000000,1.000000}%
\pgfsetfillcolor{currentfill}%
\pgfsetfillopacity{0.800000}%
\pgfsetlinewidth{1.003750pt}%
\definecolor{currentstroke}{rgb}{0.800000,0.800000,0.800000}%
\pgfsetstrokecolor{currentstroke}%
\pgfsetstrokeopacity{0.800000}%
\pgfsetdash{}{0pt}%
\pgfpathmoveto{\pgfqpoint{4.465770in}{0.597444in}}%
\pgfpathlineto{\pgfqpoint{5.662778in}{0.597444in}}%
\pgfpathquadraticcurveto{\pgfqpoint{5.690556in}{0.597444in}}{\pgfqpoint{5.690556in}{0.625222in}}%
\pgfpathlineto{\pgfqpoint{5.690556in}{0.815937in}}%
\pgfpathquadraticcurveto{\pgfqpoint{5.690556in}{0.843714in}}{\pgfqpoint{5.662778in}{0.843714in}}%
\pgfpathlineto{\pgfqpoint{4.465770in}{0.843714in}}%
\pgfpathquadraticcurveto{\pgfqpoint{4.437992in}{0.843714in}}{\pgfqpoint{4.437992in}{0.815937in}}%
\pgfpathlineto{\pgfqpoint{4.437992in}{0.625222in}}%
\pgfpathquadraticcurveto{\pgfqpoint{4.437992in}{0.597444in}}{\pgfqpoint{4.465770in}{0.597444in}}%
\pgfpathlineto{\pgfqpoint{4.465770in}{0.597444in}}%
\pgfpathclose%
\pgfusepath{stroke,fill}%
\end{pgfscope}%
\begin{pgfscope}%
\pgfsetrectcap%
\pgfsetroundjoin%
\pgfsetlinewidth{1.505625pt}%
\definecolor{currentstroke}{rgb}{0.000000,1.000000,0.498039}%
\pgfsetstrokecolor{currentstroke}%
\pgfsetdash{}{0pt}%
\pgfpathmoveto{\pgfqpoint{4.493548in}{0.730501in}}%
\pgfpathlineto{\pgfqpoint{4.632437in}{0.730501in}}%
\pgfpathlineto{\pgfqpoint{4.771325in}{0.730501in}}%
\pgfusepath{stroke}%
\end{pgfscope}%
\begin{pgfscope}%
\definecolor{textcolor}{rgb}{0.000000,0.000000,0.000000}%
\pgfsetstrokecolor{textcolor}%
\pgfsetfillcolor{textcolor}%
\pgftext[x=4.882437in,y=0.681890in,left,base]{\color{textcolor}\ttfamily\fontsize{10.000000}{12.000000}\selectfont Insertion}%
\end{pgfscope}%
\end{pgfpicture}%
\makeatother%
\endgroup%

\input{../pgf/is_c.pgf}
\input{../pgf/is_s.pgf}
%% Creator: Matplotlib, PGF backend
%%
%% To include the figure in your LaTeX document, write
%%   \input{<filename>.pgf}
%%
%% Make sure the required packages are loaded in your preamble
%%   \usepackage{pgf}
%%
%% Also ensure that all the required font packages are loaded; for instance,
%% the lmodern package is sometimes necessary when using math font.
%%   \usepackage{lmodern}
%%
%% Figures using additional raster images can only be included by \input if
%% they are in the same directory as the main LaTeX file. For loading figures
%% from other directories you can use the `import` package
%%   \usepackage{import}
%%
%% and then include the figures with
%%   \import{<path to file>}{<filename>.pgf}
%%
%% Matplotlib used the following preamble
%%   \usepackage{fontspec}
%%   \setmainfont{DejaVuSerif.ttf}[Path=\detokenize{/home/dbk/.local/lib/python3.10/site-packages/matplotlib/mpl-data/fonts/ttf/}]
%%   \setsansfont{DejaVuSans.ttf}[Path=\detokenize{/home/dbk/.local/lib/python3.10/site-packages/matplotlib/mpl-data/fonts/ttf/}]
%%   \setmonofont{DejaVuSansMono.ttf}[Path=\detokenize{/home/dbk/.local/lib/python3.10/site-packages/matplotlib/mpl-data/fonts/ttf/}]
%%
\begingroup%
\makeatletter%
\begin{pgfpicture}%
\pgfpathrectangle{\pgfpointorigin}{\pgfqpoint{6.400000in}{4.800000in}}%
\pgfusepath{use as bounding box, clip}%
\begin{pgfscope}%
\pgfsetbuttcap%
\pgfsetmiterjoin%
\definecolor{currentfill}{rgb}{1.000000,1.000000,1.000000}%
\pgfsetfillcolor{currentfill}%
\pgfsetlinewidth{0.000000pt}%
\definecolor{currentstroke}{rgb}{1.000000,1.000000,1.000000}%
\pgfsetstrokecolor{currentstroke}%
\pgfsetdash{}{0pt}%
\pgfpathmoveto{\pgfqpoint{0.000000in}{0.000000in}}%
\pgfpathlineto{\pgfqpoint{6.400000in}{0.000000in}}%
\pgfpathlineto{\pgfqpoint{6.400000in}{4.800000in}}%
\pgfpathlineto{\pgfqpoint{0.000000in}{4.800000in}}%
\pgfpathlineto{\pgfqpoint{0.000000in}{0.000000in}}%
\pgfpathclose%
\pgfusepath{fill}%
\end{pgfscope}%
\begin{pgfscope}%
\pgfsetbuttcap%
\pgfsetmiterjoin%
\definecolor{currentfill}{rgb}{1.000000,1.000000,1.000000}%
\pgfsetfillcolor{currentfill}%
\pgfsetlinewidth{0.000000pt}%
\definecolor{currentstroke}{rgb}{0.000000,0.000000,0.000000}%
\pgfsetstrokecolor{currentstroke}%
\pgfsetstrokeopacity{0.000000}%
\pgfsetdash{}{0pt}%
\pgfpathmoveto{\pgfqpoint{0.800000in}{0.528000in}}%
\pgfpathlineto{\pgfqpoint{5.760000in}{0.528000in}}%
\pgfpathlineto{\pgfqpoint{5.760000in}{4.224000in}}%
\pgfpathlineto{\pgfqpoint{0.800000in}{4.224000in}}%
\pgfpathlineto{\pgfqpoint{0.800000in}{0.528000in}}%
\pgfpathclose%
\pgfusepath{fill}%
\end{pgfscope}%
\begin{pgfscope}%
\pgfsetbuttcap%
\pgfsetroundjoin%
\definecolor{currentfill}{rgb}{0.000000,0.000000,0.000000}%
\pgfsetfillcolor{currentfill}%
\pgfsetlinewidth{0.803000pt}%
\definecolor{currentstroke}{rgb}{0.000000,0.000000,0.000000}%
\pgfsetstrokecolor{currentstroke}%
\pgfsetdash{}{0pt}%
\pgfsys@defobject{currentmarker}{\pgfqpoint{0.000000in}{-0.048611in}}{\pgfqpoint{0.000000in}{0.000000in}}{%
\pgfpathmoveto{\pgfqpoint{0.000000in}{0.000000in}}%
\pgfpathlineto{\pgfqpoint{0.000000in}{-0.048611in}}%
\pgfusepath{stroke,fill}%
}%
\begin{pgfscope}%
\pgfsys@transformshift{1.020945in}{0.528000in}%
\pgfsys@useobject{currentmarker}{}%
\end{pgfscope}%
\end{pgfscope}%
\begin{pgfscope}%
\definecolor{textcolor}{rgb}{0.000000,0.000000,0.000000}%
\pgfsetstrokecolor{textcolor}%
\pgfsetfillcolor{textcolor}%
\pgftext[x=1.020945in,y=0.430778in,,top]{\color{textcolor}\ttfamily\fontsize{10.000000}{12.000000}\selectfont 0}%
\end{pgfscope}%
\begin{pgfscope}%
\pgfsetbuttcap%
\pgfsetroundjoin%
\definecolor{currentfill}{rgb}{0.000000,0.000000,0.000000}%
\pgfsetfillcolor{currentfill}%
\pgfsetlinewidth{0.803000pt}%
\definecolor{currentstroke}{rgb}{0.000000,0.000000,0.000000}%
\pgfsetstrokecolor{currentstroke}%
\pgfsetdash{}{0pt}%
\pgfsys@defobject{currentmarker}{\pgfqpoint{0.000000in}{-0.048611in}}{\pgfqpoint{0.000000in}{0.000000in}}{%
\pgfpathmoveto{\pgfqpoint{0.000000in}{0.000000in}}%
\pgfpathlineto{\pgfqpoint{0.000000in}{-0.048611in}}%
\pgfusepath{stroke,fill}%
}%
\begin{pgfscope}%
\pgfsys@transformshift{1.922764in}{0.528000in}%
\pgfsys@useobject{currentmarker}{}%
\end{pgfscope}%
\end{pgfscope}%
\begin{pgfscope}%
\definecolor{textcolor}{rgb}{0.000000,0.000000,0.000000}%
\pgfsetstrokecolor{textcolor}%
\pgfsetfillcolor{textcolor}%
\pgftext[x=1.922764in,y=0.430778in,,top]{\color{textcolor}\ttfamily\fontsize{10.000000}{12.000000}\selectfont 200}%
\end{pgfscope}%
\begin{pgfscope}%
\pgfsetbuttcap%
\pgfsetroundjoin%
\definecolor{currentfill}{rgb}{0.000000,0.000000,0.000000}%
\pgfsetfillcolor{currentfill}%
\pgfsetlinewidth{0.803000pt}%
\definecolor{currentstroke}{rgb}{0.000000,0.000000,0.000000}%
\pgfsetstrokecolor{currentstroke}%
\pgfsetdash{}{0pt}%
\pgfsys@defobject{currentmarker}{\pgfqpoint{0.000000in}{-0.048611in}}{\pgfqpoint{0.000000in}{0.000000in}}{%
\pgfpathmoveto{\pgfqpoint{0.000000in}{0.000000in}}%
\pgfpathlineto{\pgfqpoint{0.000000in}{-0.048611in}}%
\pgfusepath{stroke,fill}%
}%
\begin{pgfscope}%
\pgfsys@transformshift{2.824582in}{0.528000in}%
\pgfsys@useobject{currentmarker}{}%
\end{pgfscope}%
\end{pgfscope}%
\begin{pgfscope}%
\definecolor{textcolor}{rgb}{0.000000,0.000000,0.000000}%
\pgfsetstrokecolor{textcolor}%
\pgfsetfillcolor{textcolor}%
\pgftext[x=2.824582in,y=0.430778in,,top]{\color{textcolor}\ttfamily\fontsize{10.000000}{12.000000}\selectfont 400}%
\end{pgfscope}%
\begin{pgfscope}%
\pgfsetbuttcap%
\pgfsetroundjoin%
\definecolor{currentfill}{rgb}{0.000000,0.000000,0.000000}%
\pgfsetfillcolor{currentfill}%
\pgfsetlinewidth{0.803000pt}%
\definecolor{currentstroke}{rgb}{0.000000,0.000000,0.000000}%
\pgfsetstrokecolor{currentstroke}%
\pgfsetdash{}{0pt}%
\pgfsys@defobject{currentmarker}{\pgfqpoint{0.000000in}{-0.048611in}}{\pgfqpoint{0.000000in}{0.000000in}}{%
\pgfpathmoveto{\pgfqpoint{0.000000in}{0.000000in}}%
\pgfpathlineto{\pgfqpoint{0.000000in}{-0.048611in}}%
\pgfusepath{stroke,fill}%
}%
\begin{pgfscope}%
\pgfsys@transformshift{3.726400in}{0.528000in}%
\pgfsys@useobject{currentmarker}{}%
\end{pgfscope}%
\end{pgfscope}%
\begin{pgfscope}%
\definecolor{textcolor}{rgb}{0.000000,0.000000,0.000000}%
\pgfsetstrokecolor{textcolor}%
\pgfsetfillcolor{textcolor}%
\pgftext[x=3.726400in,y=0.430778in,,top]{\color{textcolor}\ttfamily\fontsize{10.000000}{12.000000}\selectfont 600}%
\end{pgfscope}%
\begin{pgfscope}%
\pgfsetbuttcap%
\pgfsetroundjoin%
\definecolor{currentfill}{rgb}{0.000000,0.000000,0.000000}%
\pgfsetfillcolor{currentfill}%
\pgfsetlinewidth{0.803000pt}%
\definecolor{currentstroke}{rgb}{0.000000,0.000000,0.000000}%
\pgfsetstrokecolor{currentstroke}%
\pgfsetdash{}{0pt}%
\pgfsys@defobject{currentmarker}{\pgfqpoint{0.000000in}{-0.048611in}}{\pgfqpoint{0.000000in}{0.000000in}}{%
\pgfpathmoveto{\pgfqpoint{0.000000in}{0.000000in}}%
\pgfpathlineto{\pgfqpoint{0.000000in}{-0.048611in}}%
\pgfusepath{stroke,fill}%
}%
\begin{pgfscope}%
\pgfsys@transformshift{4.628218in}{0.528000in}%
\pgfsys@useobject{currentmarker}{}%
\end{pgfscope}%
\end{pgfscope}%
\begin{pgfscope}%
\definecolor{textcolor}{rgb}{0.000000,0.000000,0.000000}%
\pgfsetstrokecolor{textcolor}%
\pgfsetfillcolor{textcolor}%
\pgftext[x=4.628218in,y=0.430778in,,top]{\color{textcolor}\ttfamily\fontsize{10.000000}{12.000000}\selectfont 800}%
\end{pgfscope}%
\begin{pgfscope}%
\pgfsetbuttcap%
\pgfsetroundjoin%
\definecolor{currentfill}{rgb}{0.000000,0.000000,0.000000}%
\pgfsetfillcolor{currentfill}%
\pgfsetlinewidth{0.803000pt}%
\definecolor{currentstroke}{rgb}{0.000000,0.000000,0.000000}%
\pgfsetstrokecolor{currentstroke}%
\pgfsetdash{}{0pt}%
\pgfsys@defobject{currentmarker}{\pgfqpoint{0.000000in}{-0.048611in}}{\pgfqpoint{0.000000in}{0.000000in}}{%
\pgfpathmoveto{\pgfqpoint{0.000000in}{0.000000in}}%
\pgfpathlineto{\pgfqpoint{0.000000in}{-0.048611in}}%
\pgfusepath{stroke,fill}%
}%
\begin{pgfscope}%
\pgfsys@transformshift{5.530036in}{0.528000in}%
\pgfsys@useobject{currentmarker}{}%
\end{pgfscope}%
\end{pgfscope}%
\begin{pgfscope}%
\definecolor{textcolor}{rgb}{0.000000,0.000000,0.000000}%
\pgfsetstrokecolor{textcolor}%
\pgfsetfillcolor{textcolor}%
\pgftext[x=5.530036in,y=0.430778in,,top]{\color{textcolor}\ttfamily\fontsize{10.000000}{12.000000}\selectfont 1000}%
\end{pgfscope}%
\begin{pgfscope}%
\definecolor{textcolor}{rgb}{0.000000,0.000000,0.000000}%
\pgfsetstrokecolor{textcolor}%
\pgfsetfillcolor{textcolor}%
\pgftext[x=3.280000in,y=0.240063in,,top]{\color{textcolor}\ttfamily\fontsize{10.000000}{12.000000}\selectfont Size of Array}%
\end{pgfscope}%
\begin{pgfscope}%
\pgfsetbuttcap%
\pgfsetroundjoin%
\definecolor{currentfill}{rgb}{0.000000,0.000000,0.000000}%
\pgfsetfillcolor{currentfill}%
\pgfsetlinewidth{0.803000pt}%
\definecolor{currentstroke}{rgb}{0.000000,0.000000,0.000000}%
\pgfsetstrokecolor{currentstroke}%
\pgfsetdash{}{0pt}%
\pgfsys@defobject{currentmarker}{\pgfqpoint{-0.048611in}{0.000000in}}{\pgfqpoint{-0.000000in}{0.000000in}}{%
\pgfpathmoveto{\pgfqpoint{-0.000000in}{0.000000in}}%
\pgfpathlineto{\pgfqpoint{-0.048611in}{0.000000in}}%
\pgfusepath{stroke,fill}%
}%
\begin{pgfscope}%
\pgfsys@transformshift{0.800000in}{0.674248in}%
\pgfsys@useobject{currentmarker}{}%
\end{pgfscope}%
\end{pgfscope}%
\begin{pgfscope}%
\definecolor{textcolor}{rgb}{0.000000,0.000000,0.000000}%
\pgfsetstrokecolor{textcolor}%
\pgfsetfillcolor{textcolor}%
\pgftext[x=0.619160in, y=0.621114in, left, base]{\color{textcolor}\ttfamily\fontsize{10.000000}{12.000000}\selectfont 0}%
\end{pgfscope}%
\begin{pgfscope}%
\pgfsetbuttcap%
\pgfsetroundjoin%
\definecolor{currentfill}{rgb}{0.000000,0.000000,0.000000}%
\pgfsetfillcolor{currentfill}%
\pgfsetlinewidth{0.803000pt}%
\definecolor{currentstroke}{rgb}{0.000000,0.000000,0.000000}%
\pgfsetstrokecolor{currentstroke}%
\pgfsetdash{}{0pt}%
\pgfsys@defobject{currentmarker}{\pgfqpoint{-0.048611in}{0.000000in}}{\pgfqpoint{-0.000000in}{0.000000in}}{%
\pgfpathmoveto{\pgfqpoint{-0.000000in}{0.000000in}}%
\pgfpathlineto{\pgfqpoint{-0.048611in}{0.000000in}}%
\pgfusepath{stroke,fill}%
}%
\begin{pgfscope}%
\pgfsys@transformshift{0.800000in}{1.226045in}%
\pgfsys@useobject{currentmarker}{}%
\end{pgfscope}%
\end{pgfscope}%
\begin{pgfscope}%
\definecolor{textcolor}{rgb}{0.000000,0.000000,0.000000}%
\pgfsetstrokecolor{textcolor}%
\pgfsetfillcolor{textcolor}%
\pgftext[x=0.284687in, y=1.172910in, left, base]{\color{textcolor}\ttfamily\fontsize{10.000000}{12.000000}\selectfont 50000}%
\end{pgfscope}%
\begin{pgfscope}%
\pgfsetbuttcap%
\pgfsetroundjoin%
\definecolor{currentfill}{rgb}{0.000000,0.000000,0.000000}%
\pgfsetfillcolor{currentfill}%
\pgfsetlinewidth{0.803000pt}%
\definecolor{currentstroke}{rgb}{0.000000,0.000000,0.000000}%
\pgfsetstrokecolor{currentstroke}%
\pgfsetdash{}{0pt}%
\pgfsys@defobject{currentmarker}{\pgfqpoint{-0.048611in}{0.000000in}}{\pgfqpoint{-0.000000in}{0.000000in}}{%
\pgfpathmoveto{\pgfqpoint{-0.000000in}{0.000000in}}%
\pgfpathlineto{\pgfqpoint{-0.048611in}{0.000000in}}%
\pgfusepath{stroke,fill}%
}%
\begin{pgfscope}%
\pgfsys@transformshift{0.800000in}{1.777841in}%
\pgfsys@useobject{currentmarker}{}%
\end{pgfscope}%
\end{pgfscope}%
\begin{pgfscope}%
\definecolor{textcolor}{rgb}{0.000000,0.000000,0.000000}%
\pgfsetstrokecolor{textcolor}%
\pgfsetfillcolor{textcolor}%
\pgftext[x=0.201069in, y=1.724707in, left, base]{\color{textcolor}\ttfamily\fontsize{10.000000}{12.000000}\selectfont 100000}%
\end{pgfscope}%
\begin{pgfscope}%
\pgfsetbuttcap%
\pgfsetroundjoin%
\definecolor{currentfill}{rgb}{0.000000,0.000000,0.000000}%
\pgfsetfillcolor{currentfill}%
\pgfsetlinewidth{0.803000pt}%
\definecolor{currentstroke}{rgb}{0.000000,0.000000,0.000000}%
\pgfsetstrokecolor{currentstroke}%
\pgfsetdash{}{0pt}%
\pgfsys@defobject{currentmarker}{\pgfqpoint{-0.048611in}{0.000000in}}{\pgfqpoint{-0.000000in}{0.000000in}}{%
\pgfpathmoveto{\pgfqpoint{-0.000000in}{0.000000in}}%
\pgfpathlineto{\pgfqpoint{-0.048611in}{0.000000in}}%
\pgfusepath{stroke,fill}%
}%
\begin{pgfscope}%
\pgfsys@transformshift{0.800000in}{2.329638in}%
\pgfsys@useobject{currentmarker}{}%
\end{pgfscope}%
\end{pgfscope}%
\begin{pgfscope}%
\definecolor{textcolor}{rgb}{0.000000,0.000000,0.000000}%
\pgfsetstrokecolor{textcolor}%
\pgfsetfillcolor{textcolor}%
\pgftext[x=0.201069in, y=2.276504in, left, base]{\color{textcolor}\ttfamily\fontsize{10.000000}{12.000000}\selectfont 150000}%
\end{pgfscope}%
\begin{pgfscope}%
\pgfsetbuttcap%
\pgfsetroundjoin%
\definecolor{currentfill}{rgb}{0.000000,0.000000,0.000000}%
\pgfsetfillcolor{currentfill}%
\pgfsetlinewidth{0.803000pt}%
\definecolor{currentstroke}{rgb}{0.000000,0.000000,0.000000}%
\pgfsetstrokecolor{currentstroke}%
\pgfsetdash{}{0pt}%
\pgfsys@defobject{currentmarker}{\pgfqpoint{-0.048611in}{0.000000in}}{\pgfqpoint{-0.000000in}{0.000000in}}{%
\pgfpathmoveto{\pgfqpoint{-0.000000in}{0.000000in}}%
\pgfpathlineto{\pgfqpoint{-0.048611in}{0.000000in}}%
\pgfusepath{stroke,fill}%
}%
\begin{pgfscope}%
\pgfsys@transformshift{0.800000in}{2.881435in}%
\pgfsys@useobject{currentmarker}{}%
\end{pgfscope}%
\end{pgfscope}%
\begin{pgfscope}%
\definecolor{textcolor}{rgb}{0.000000,0.000000,0.000000}%
\pgfsetstrokecolor{textcolor}%
\pgfsetfillcolor{textcolor}%
\pgftext[x=0.201069in, y=2.828300in, left, base]{\color{textcolor}\ttfamily\fontsize{10.000000}{12.000000}\selectfont 200000}%
\end{pgfscope}%
\begin{pgfscope}%
\pgfsetbuttcap%
\pgfsetroundjoin%
\definecolor{currentfill}{rgb}{0.000000,0.000000,0.000000}%
\pgfsetfillcolor{currentfill}%
\pgfsetlinewidth{0.803000pt}%
\definecolor{currentstroke}{rgb}{0.000000,0.000000,0.000000}%
\pgfsetstrokecolor{currentstroke}%
\pgfsetdash{}{0pt}%
\pgfsys@defobject{currentmarker}{\pgfqpoint{-0.048611in}{0.000000in}}{\pgfqpoint{-0.000000in}{0.000000in}}{%
\pgfpathmoveto{\pgfqpoint{-0.000000in}{0.000000in}}%
\pgfpathlineto{\pgfqpoint{-0.048611in}{0.000000in}}%
\pgfusepath{stroke,fill}%
}%
\begin{pgfscope}%
\pgfsys@transformshift{0.800000in}{3.433231in}%
\pgfsys@useobject{currentmarker}{}%
\end{pgfscope}%
\end{pgfscope}%
\begin{pgfscope}%
\definecolor{textcolor}{rgb}{0.000000,0.000000,0.000000}%
\pgfsetstrokecolor{textcolor}%
\pgfsetfillcolor{textcolor}%
\pgftext[x=0.201069in, y=3.380097in, left, base]{\color{textcolor}\ttfamily\fontsize{10.000000}{12.000000}\selectfont 250000}%
\end{pgfscope}%
\begin{pgfscope}%
\pgfsetbuttcap%
\pgfsetroundjoin%
\definecolor{currentfill}{rgb}{0.000000,0.000000,0.000000}%
\pgfsetfillcolor{currentfill}%
\pgfsetlinewidth{0.803000pt}%
\definecolor{currentstroke}{rgb}{0.000000,0.000000,0.000000}%
\pgfsetstrokecolor{currentstroke}%
\pgfsetdash{}{0pt}%
\pgfsys@defobject{currentmarker}{\pgfqpoint{-0.048611in}{0.000000in}}{\pgfqpoint{-0.000000in}{0.000000in}}{%
\pgfpathmoveto{\pgfqpoint{-0.000000in}{0.000000in}}%
\pgfpathlineto{\pgfqpoint{-0.048611in}{0.000000in}}%
\pgfusepath{stroke,fill}%
}%
\begin{pgfscope}%
\pgfsys@transformshift{0.800000in}{3.985028in}%
\pgfsys@useobject{currentmarker}{}%
\end{pgfscope}%
\end{pgfscope}%
\begin{pgfscope}%
\definecolor{textcolor}{rgb}{0.000000,0.000000,0.000000}%
\pgfsetstrokecolor{textcolor}%
\pgfsetfillcolor{textcolor}%
\pgftext[x=0.201069in, y=3.931893in, left, base]{\color{textcolor}\ttfamily\fontsize{10.000000}{12.000000}\selectfont 300000}%
\end{pgfscope}%
\begin{pgfscope}%
\definecolor{textcolor}{rgb}{0.000000,0.000000,0.000000}%
\pgfsetstrokecolor{textcolor}%
\pgfsetfillcolor{textcolor}%
\pgftext[x=0.145513in,y=2.376000in,,bottom,rotate=90.000000]{\color{textcolor}\ttfamily\fontsize{10.000000}{12.000000}\selectfont Iterations}%
\end{pgfscope}%
\begin{pgfscope}%
\pgfpathrectangle{\pgfqpoint{0.800000in}{0.528000in}}{\pgfqpoint{4.960000in}{3.696000in}}%
\pgfusepath{clip}%
\pgfsetrectcap%
\pgfsetroundjoin%
\pgfsetlinewidth{1.505625pt}%
\definecolor{currentstroke}{rgb}{0.000000,1.000000,0.498039}%
\pgfsetstrokecolor{currentstroke}%
\pgfsetdash{}{0pt}%
\pgfpathmoveto{\pgfqpoint{1.025455in}{0.696000in}}%
\pgfpathlineto{\pgfqpoint{1.034473in}{0.703593in}}%
\pgfpathlineto{\pgfqpoint{1.038982in}{0.703284in}}%
\pgfpathlineto{\pgfqpoint{1.043491in}{0.705458in}}%
\pgfpathlineto{\pgfqpoint{1.048000in}{0.709662in}}%
\pgfpathlineto{\pgfqpoint{1.052509in}{0.704685in}}%
\pgfpathlineto{\pgfqpoint{1.057018in}{0.702920in}}%
\pgfpathlineto{\pgfqpoint{1.066036in}{0.709751in}}%
\pgfpathlineto{\pgfqpoint{1.070545in}{0.710126in}}%
\pgfpathlineto{\pgfqpoint{1.075055in}{0.705976in}}%
\pgfpathlineto{\pgfqpoint{1.079564in}{0.707577in}}%
\pgfpathlineto{\pgfqpoint{1.088582in}{0.713260in}}%
\pgfpathlineto{\pgfqpoint{1.097600in}{0.709475in}}%
\pgfpathlineto{\pgfqpoint{1.102109in}{0.711638in}}%
\pgfpathlineto{\pgfqpoint{1.106618in}{0.716560in}}%
\pgfpathlineto{\pgfqpoint{1.115636in}{0.714805in}}%
\pgfpathlineto{\pgfqpoint{1.120145in}{0.719805in}}%
\pgfpathlineto{\pgfqpoint{1.129164in}{0.713371in}}%
\pgfpathlineto{\pgfqpoint{1.133673in}{0.715710in}}%
\pgfpathlineto{\pgfqpoint{1.138182in}{0.721648in}}%
\pgfpathlineto{\pgfqpoint{1.142691in}{0.719010in}}%
\pgfpathlineto{\pgfqpoint{1.147200in}{0.723844in}}%
\pgfpathlineto{\pgfqpoint{1.151709in}{0.716737in}}%
\pgfpathlineto{\pgfqpoint{1.156218in}{0.723623in}}%
\pgfpathlineto{\pgfqpoint{1.160727in}{0.723513in}}%
\pgfpathlineto{\pgfqpoint{1.165236in}{0.721570in}}%
\pgfpathlineto{\pgfqpoint{1.169745in}{0.727463in}}%
\pgfpathlineto{\pgfqpoint{1.174255in}{0.726393in}}%
\pgfpathlineto{\pgfqpoint{1.178764in}{0.720389in}}%
\pgfpathlineto{\pgfqpoint{1.183273in}{0.727949in}}%
\pgfpathlineto{\pgfqpoint{1.187782in}{0.721460in}}%
\pgfpathlineto{\pgfqpoint{1.196800in}{0.728655in}}%
\pgfpathlineto{\pgfqpoint{1.201309in}{0.725609in}}%
\pgfpathlineto{\pgfqpoint{1.205818in}{0.730046in}}%
\pgfpathlineto{\pgfqpoint{1.214836in}{0.730576in}}%
\pgfpathlineto{\pgfqpoint{1.219345in}{0.735189in}}%
\pgfpathlineto{\pgfqpoint{1.228364in}{0.731227in}}%
\pgfpathlineto{\pgfqpoint{1.232873in}{0.738577in}}%
\pgfpathlineto{\pgfqpoint{1.237382in}{0.735630in}}%
\pgfpathlineto{\pgfqpoint{1.246400in}{0.732231in}}%
\pgfpathlineto{\pgfqpoint{1.250909in}{0.738036in}}%
\pgfpathlineto{\pgfqpoint{1.255418in}{0.738676in}}%
\pgfpathlineto{\pgfqpoint{1.259927in}{0.741998in}}%
\pgfpathlineto{\pgfqpoint{1.264436in}{0.734107in}}%
\pgfpathlineto{\pgfqpoint{1.268945in}{0.741821in}}%
\pgfpathlineto{\pgfqpoint{1.273455in}{0.740431in}}%
\pgfpathlineto{\pgfqpoint{1.277964in}{0.740431in}}%
\pgfpathlineto{\pgfqpoint{1.282473in}{0.744691in}}%
\pgfpathlineto{\pgfqpoint{1.291491in}{0.742803in}}%
\pgfpathlineto{\pgfqpoint{1.296000in}{0.757128in}}%
\pgfpathlineto{\pgfqpoint{1.300509in}{0.745706in}}%
\pgfpathlineto{\pgfqpoint{1.305018in}{0.741291in}}%
\pgfpathlineto{\pgfqpoint{1.318545in}{0.754513in}}%
\pgfpathlineto{\pgfqpoint{1.323055in}{0.754689in}}%
\pgfpathlineto{\pgfqpoint{1.327564in}{0.743256in}}%
\pgfpathlineto{\pgfqpoint{1.332073in}{0.758430in}}%
\pgfpathlineto{\pgfqpoint{1.336582in}{0.755318in}}%
\pgfpathlineto{\pgfqpoint{1.341091in}{0.760339in}}%
\pgfpathlineto{\pgfqpoint{1.345600in}{0.759004in}}%
\pgfpathlineto{\pgfqpoint{1.350109in}{0.760748in}}%
\pgfpathlineto{\pgfqpoint{1.354618in}{0.751908in}}%
\pgfpathlineto{\pgfqpoint{1.359127in}{0.758850in}}%
\pgfpathlineto{\pgfqpoint{1.363636in}{0.758640in}}%
\pgfpathlineto{\pgfqpoint{1.368145in}{0.764346in}}%
\pgfpathlineto{\pgfqpoint{1.372655in}{0.764434in}}%
\pgfpathlineto{\pgfqpoint{1.377164in}{0.756013in}}%
\pgfpathlineto{\pgfqpoint{1.381673in}{0.762006in}}%
\pgfpathlineto{\pgfqpoint{1.386182in}{0.763849in}}%
\pgfpathlineto{\pgfqpoint{1.395200in}{0.758210in}}%
\pgfpathlineto{\pgfqpoint{1.399709in}{0.763860in}}%
\pgfpathlineto{\pgfqpoint{1.404218in}{0.772402in}}%
\pgfpathlineto{\pgfqpoint{1.408727in}{0.768285in}}%
\pgfpathlineto{\pgfqpoint{1.422255in}{0.773428in}}%
\pgfpathlineto{\pgfqpoint{1.426764in}{0.771640in}}%
\pgfpathlineto{\pgfqpoint{1.431273in}{0.766475in}}%
\pgfpathlineto{\pgfqpoint{1.435782in}{0.774465in}}%
\pgfpathlineto{\pgfqpoint{1.444800in}{0.780337in}}%
\pgfpathlineto{\pgfqpoint{1.449309in}{0.787852in}}%
\pgfpathlineto{\pgfqpoint{1.458327in}{0.781705in}}%
\pgfpathlineto{\pgfqpoint{1.462836in}{0.778196in}}%
\pgfpathlineto{\pgfqpoint{1.467345in}{0.779432in}}%
\pgfpathlineto{\pgfqpoint{1.471855in}{0.783007in}}%
\pgfpathlineto{\pgfqpoint{1.476364in}{0.793061in}}%
\pgfpathlineto{\pgfqpoint{1.485382in}{0.781109in}}%
\pgfpathlineto{\pgfqpoint{1.489891in}{0.785976in}}%
\pgfpathlineto{\pgfqpoint{1.494400in}{0.782367in}}%
\pgfpathlineto{\pgfqpoint{1.498909in}{0.793657in}}%
\pgfpathlineto{\pgfqpoint{1.503418in}{0.793811in}}%
\pgfpathlineto{\pgfqpoint{1.507927in}{0.788205in}}%
\pgfpathlineto{\pgfqpoint{1.512436in}{0.800886in}}%
\pgfpathlineto{\pgfqpoint{1.516945in}{0.792200in}}%
\pgfpathlineto{\pgfqpoint{1.521455in}{0.795809in}}%
\pgfpathlineto{\pgfqpoint{1.525964in}{0.793006in}}%
\pgfpathlineto{\pgfqpoint{1.530473in}{0.794661in}}%
\pgfpathlineto{\pgfqpoint{1.534982in}{0.800886in}}%
\pgfpathlineto{\pgfqpoint{1.539491in}{0.797398in}}%
\pgfpathlineto{\pgfqpoint{1.544000in}{0.808202in}}%
\pgfpathlineto{\pgfqpoint{1.548509in}{0.808776in}}%
\pgfpathlineto{\pgfqpoint{1.553018in}{0.811778in}}%
\pgfpathlineto{\pgfqpoint{1.557527in}{0.808346in}}%
\pgfpathlineto{\pgfqpoint{1.562036in}{0.809692in}}%
\pgfpathlineto{\pgfqpoint{1.571055in}{0.809151in}}%
\pgfpathlineto{\pgfqpoint{1.575564in}{0.815718in}}%
\pgfpathlineto{\pgfqpoint{1.580073in}{0.803700in}}%
\pgfpathlineto{\pgfqpoint{1.584582in}{0.810454in}}%
\pgfpathlineto{\pgfqpoint{1.589091in}{0.821423in}}%
\pgfpathlineto{\pgfqpoint{1.593600in}{0.824050in}}%
\pgfpathlineto{\pgfqpoint{1.598109in}{0.818344in}}%
\pgfpathlineto{\pgfqpoint{1.602618in}{0.821258in}}%
\pgfpathlineto{\pgfqpoint{1.607127in}{0.831510in}}%
\pgfpathlineto{\pgfqpoint{1.611636in}{0.818764in}}%
\pgfpathlineto{\pgfqpoint{1.616145in}{0.815078in}}%
\pgfpathlineto{\pgfqpoint{1.620655in}{0.829789in}}%
\pgfpathlineto{\pgfqpoint{1.625164in}{0.818433in}}%
\pgfpathlineto{\pgfqpoint{1.629673in}{0.817517in}}%
\pgfpathlineto{\pgfqpoint{1.634182in}{0.836885in}}%
\pgfpathlineto{\pgfqpoint{1.638691in}{0.833066in}}%
\pgfpathlineto{\pgfqpoint{1.643200in}{0.833673in}}%
\pgfpathlineto{\pgfqpoint{1.647709in}{0.835759in}}%
\pgfpathlineto{\pgfqpoint{1.652218in}{0.832261in}}%
\pgfpathlineto{\pgfqpoint{1.656727in}{0.838176in}}%
\pgfpathlineto{\pgfqpoint{1.661236in}{0.833519in}}%
\pgfpathlineto{\pgfqpoint{1.665745in}{0.832603in}}%
\pgfpathlineto{\pgfqpoint{1.670255in}{0.845780in}}%
\pgfpathlineto{\pgfqpoint{1.674764in}{0.845945in}}%
\pgfpathlineto{\pgfqpoint{1.679273in}{0.853406in}}%
\pgfpathlineto{\pgfqpoint{1.683782in}{0.847303in}}%
\pgfpathlineto{\pgfqpoint{1.688291in}{0.824447in}}%
\pgfpathlineto{\pgfqpoint{1.697309in}{0.853406in}}%
\pgfpathlineto{\pgfqpoint{1.701818in}{0.854388in}}%
\pgfpathlineto{\pgfqpoint{1.706327in}{0.857202in}}%
\pgfpathlineto{\pgfqpoint{1.710836in}{0.854575in}}%
\pgfpathlineto{\pgfqpoint{1.715345in}{0.843010in}}%
\pgfpathlineto{\pgfqpoint{1.719855in}{0.851099in}}%
\pgfpathlineto{\pgfqpoint{1.724364in}{0.868613in}}%
\pgfpathlineto{\pgfqpoint{1.728873in}{0.863183in}}%
\pgfpathlineto{\pgfqpoint{1.733382in}{0.861075in}}%
\pgfpathlineto{\pgfqpoint{1.737891in}{0.861749in}}%
\pgfpathlineto{\pgfqpoint{1.742400in}{0.855745in}}%
\pgfpathlineto{\pgfqpoint{1.746909in}{0.856683in}}%
\pgfpathlineto{\pgfqpoint{1.751418in}{0.858912in}}%
\pgfpathlineto{\pgfqpoint{1.755927in}{0.855878in}}%
\pgfpathlineto{\pgfqpoint{1.764945in}{0.877409in}}%
\pgfpathlineto{\pgfqpoint{1.769455in}{0.860965in}}%
\pgfpathlineto{\pgfqpoint{1.773964in}{0.868812in}}%
\pgfpathlineto{\pgfqpoint{1.778473in}{0.870114in}}%
\pgfpathlineto{\pgfqpoint{1.782982in}{0.865115in}}%
\pgfpathlineto{\pgfqpoint{1.787491in}{0.879009in}}%
\pgfpathlineto{\pgfqpoint{1.792000in}{0.871129in}}%
\pgfpathlineto{\pgfqpoint{1.801018in}{0.892406in}}%
\pgfpathlineto{\pgfqpoint{1.805527in}{0.869021in}}%
\pgfpathlineto{\pgfqpoint{1.810036in}{0.885939in}}%
\pgfpathlineto{\pgfqpoint{1.814545in}{0.883567in}}%
\pgfpathlineto{\pgfqpoint{1.819055in}{0.894658in}}%
\pgfpathlineto{\pgfqpoint{1.828073in}{0.897505in}}%
\pgfpathlineto{\pgfqpoint{1.832582in}{0.896997in}}%
\pgfpathlineto{\pgfqpoint{1.837091in}{0.894029in}}%
\pgfpathlineto{\pgfqpoint{1.841600in}{0.892980in}}%
\pgfpathlineto{\pgfqpoint{1.846109in}{0.884350in}}%
\pgfpathlineto{\pgfqpoint{1.850618in}{0.887849in}}%
\pgfpathlineto{\pgfqpoint{1.855127in}{0.904358in}}%
\pgfpathlineto{\pgfqpoint{1.859636in}{0.908508in}}%
\pgfpathlineto{\pgfqpoint{1.864145in}{0.910196in}}%
\pgfpathlineto{\pgfqpoint{1.868655in}{0.909667in}}%
\pgfpathlineto{\pgfqpoint{1.873164in}{0.897373in}}%
\pgfpathlineto{\pgfqpoint{1.877673in}{0.900474in}}%
\pgfpathlineto{\pgfqpoint{1.886691in}{0.925271in}}%
\pgfpathlineto{\pgfqpoint{1.891200in}{0.900330in}}%
\pgfpathlineto{\pgfqpoint{1.895709in}{0.914280in}}%
\pgfpathlineto{\pgfqpoint{1.900218in}{0.919158in}}%
\pgfpathlineto{\pgfqpoint{1.904727in}{0.926728in}}%
\pgfpathlineto{\pgfqpoint{1.909236in}{0.908199in}}%
\pgfpathlineto{\pgfqpoint{1.913745in}{0.928626in}}%
\pgfpathlineto{\pgfqpoint{1.918255in}{0.930778in}}%
\pgfpathlineto{\pgfqpoint{1.922764in}{0.922866in}}%
\pgfpathlineto{\pgfqpoint{1.927273in}{0.934972in}}%
\pgfpathlineto{\pgfqpoint{1.931782in}{0.932533in}}%
\pgfpathlineto{\pgfqpoint{1.936291in}{0.937069in}}%
\pgfpathlineto{\pgfqpoint{1.940800in}{0.924742in}}%
\pgfpathlineto{\pgfqpoint{1.945309in}{0.942863in}}%
\pgfpathlineto{\pgfqpoint{1.949818in}{0.940865in}}%
\pgfpathlineto{\pgfqpoint{1.958836in}{0.939475in}}%
\pgfpathlineto{\pgfqpoint{1.963345in}{0.938228in}}%
\pgfpathlineto{\pgfqpoint{1.967855in}{0.954484in}}%
\pgfpathlineto{\pgfqpoint{1.972364in}{0.946107in}}%
\pgfpathlineto{\pgfqpoint{1.976873in}{0.955830in}}%
\pgfpathlineto{\pgfqpoint{1.981382in}{0.953932in}}%
\pgfpathlineto{\pgfqpoint{1.985891in}{0.956040in}}%
\pgfpathlineto{\pgfqpoint{1.990400in}{0.959781in}}%
\pgfpathlineto{\pgfqpoint{1.999418in}{0.942421in}}%
\pgfpathlineto{\pgfqpoint{2.003927in}{0.944187in}}%
\pgfpathlineto{\pgfqpoint{2.008436in}{0.949860in}}%
\pgfpathlineto{\pgfqpoint{2.012945in}{0.964802in}}%
\pgfpathlineto{\pgfqpoint{2.017455in}{0.970817in}}%
\pgfpathlineto{\pgfqpoint{2.021964in}{0.953623in}}%
\pgfpathlineto{\pgfqpoint{2.026473in}{0.945666in}}%
\pgfpathlineto{\pgfqpoint{2.030982in}{0.974238in}}%
\pgfpathlineto{\pgfqpoint{2.035491in}{0.988265in}}%
\pgfpathlineto{\pgfqpoint{2.040000in}{0.956647in}}%
\pgfpathlineto{\pgfqpoint{2.044509in}{0.960024in}}%
\pgfpathlineto{\pgfqpoint{2.049018in}{0.975595in}}%
\pgfpathlineto{\pgfqpoint{2.053527in}{0.969139in}}%
\pgfpathlineto{\pgfqpoint{2.058036in}{0.980286in}}%
\pgfpathlineto{\pgfqpoint{2.062545in}{0.983232in}}%
\pgfpathlineto{\pgfqpoint{2.067055in}{0.983905in}}%
\pgfpathlineto{\pgfqpoint{2.071564in}{0.971391in}}%
\pgfpathlineto{\pgfqpoint{2.076073in}{1.013371in}}%
\pgfpathlineto{\pgfqpoint{2.080582in}{0.969746in}}%
\pgfpathlineto{\pgfqpoint{2.089600in}{0.971236in}}%
\pgfpathlineto{\pgfqpoint{2.094109in}{0.994202in}}%
\pgfpathlineto{\pgfqpoint{2.098618in}{0.991873in}}%
\pgfpathlineto{\pgfqpoint{2.103127in}{1.003560in}}%
\pgfpathlineto{\pgfqpoint{2.107636in}{0.995990in}}%
\pgfpathlineto{\pgfqpoint{2.112145in}{0.983365in}}%
\pgfpathlineto{\pgfqpoint{2.116655in}{1.021096in}}%
\pgfpathlineto{\pgfqpoint{2.121164in}{1.008527in}}%
\pgfpathlineto{\pgfqpoint{2.125673in}{1.011782in}}%
\pgfpathlineto{\pgfqpoint{2.130182in}{1.023778in}}%
\pgfpathlineto{\pgfqpoint{2.134691in}{1.001353in}}%
\pgfpathlineto{\pgfqpoint{2.139200in}{1.015314in}}%
\pgfpathlineto{\pgfqpoint{2.143709in}{1.008747in}}%
\pgfpathlineto{\pgfqpoint{2.148218in}{1.021792in}}%
\pgfpathlineto{\pgfqpoint{2.152727in}{1.004642in}}%
\pgfpathlineto{\pgfqpoint{2.161745in}{1.002402in}}%
\pgfpathlineto{\pgfqpoint{2.166255in}{0.987216in}}%
\pgfpathlineto{\pgfqpoint{2.170764in}{1.029285in}}%
\pgfpathlineto{\pgfqpoint{2.175273in}{1.035024in}}%
\pgfpathlineto{\pgfqpoint{2.179782in}{1.034494in}}%
\pgfpathlineto{\pgfqpoint{2.184291in}{1.028314in}}%
\pgfpathlineto{\pgfqpoint{2.188800in}{1.031239in}}%
\pgfpathlineto{\pgfqpoint{2.193309in}{1.037860in}}%
\pgfpathlineto{\pgfqpoint{2.197818in}{1.021847in}}%
\pgfpathlineto{\pgfqpoint{2.202327in}{1.025246in}}%
\pgfpathlineto{\pgfqpoint{2.206836in}{1.051125in}}%
\pgfpathlineto{\pgfqpoint{2.211345in}{1.024385in}}%
\pgfpathlineto{\pgfqpoint{2.215855in}{1.046005in}}%
\pgfpathlineto{\pgfqpoint{2.220364in}{1.033435in}}%
\pgfpathlineto{\pgfqpoint{2.224873in}{1.028976in}}%
\pgfpathlineto{\pgfqpoint{2.229382in}{1.074864in}}%
\pgfpathlineto{\pgfqpoint{2.233891in}{1.046269in}}%
\pgfpathlineto{\pgfqpoint{2.238400in}{1.045618in}}%
\pgfpathlineto{\pgfqpoint{2.242909in}{1.068628in}}%
\pgfpathlineto{\pgfqpoint{2.247418in}{1.048951in}}%
\pgfpathlineto{\pgfqpoint{2.251927in}{1.074257in}}%
\pgfpathlineto{\pgfqpoint{2.256436in}{1.039350in}}%
\pgfpathlineto{\pgfqpoint{2.260945in}{1.058288in}}%
\pgfpathlineto{\pgfqpoint{2.269964in}{1.082578in}}%
\pgfpathlineto{\pgfqpoint{2.274473in}{1.055507in}}%
\pgfpathlineto{\pgfqpoint{2.278982in}{1.083902in}}%
\pgfpathlineto{\pgfqpoint{2.283491in}{1.071012in}}%
\pgfpathlineto{\pgfqpoint{2.288000in}{1.053730in}}%
\pgfpathlineto{\pgfqpoint{2.292509in}{1.065947in}}%
\pgfpathlineto{\pgfqpoint{2.297018in}{1.089420in}}%
\pgfpathlineto{\pgfqpoint{2.301527in}{1.093614in}}%
\pgfpathlineto{\pgfqpoint{2.310545in}{1.107464in}}%
\pgfpathlineto{\pgfqpoint{2.315055in}{1.069467in}}%
\pgfpathlineto{\pgfqpoint{2.319564in}{1.082953in}}%
\pgfpathlineto{\pgfqpoint{2.324073in}{1.077932in}}%
\pgfpathlineto{\pgfqpoint{2.328582in}{1.092742in}}%
\pgfpathlineto{\pgfqpoint{2.333091in}{1.097907in}}%
\pgfpathlineto{\pgfqpoint{2.337600in}{1.112584in}}%
\pgfpathlineto{\pgfqpoint{2.342109in}{1.099209in}}%
\pgfpathlineto{\pgfqpoint{2.346618in}{1.108689in}}%
\pgfpathlineto{\pgfqpoint{2.351127in}{1.082478in}}%
\pgfpathlineto{\pgfqpoint{2.355636in}{1.092179in}}%
\pgfpathlineto{\pgfqpoint{2.360145in}{1.124591in}}%
\pgfpathlineto{\pgfqpoint{2.364655in}{1.131986in}}%
\pgfpathlineto{\pgfqpoint{2.369164in}{1.089067in}}%
\pgfpathlineto{\pgfqpoint{2.373673in}{1.120685in}}%
\pgfpathlineto{\pgfqpoint{2.378182in}{1.133166in}}%
\pgfpathlineto{\pgfqpoint{2.382691in}{1.116072in}}%
\pgfpathlineto{\pgfqpoint{2.387200in}{1.120861in}}%
\pgfpathlineto{\pgfqpoint{2.391709in}{1.129105in}}%
\pgfpathlineto{\pgfqpoint{2.396218in}{1.113953in}}%
\pgfpathlineto{\pgfqpoint{2.400727in}{1.149853in}}%
\pgfpathlineto{\pgfqpoint{2.405236in}{1.129271in}}%
\pgfpathlineto{\pgfqpoint{2.409745in}{1.121380in}}%
\pgfpathlineto{\pgfqpoint{2.414255in}{1.156728in}}%
\pgfpathlineto{\pgfqpoint{2.418764in}{1.115233in}}%
\pgfpathlineto{\pgfqpoint{2.423273in}{1.116822in}}%
\pgfpathlineto{\pgfqpoint{2.432291in}{1.137095in}}%
\pgfpathlineto{\pgfqpoint{2.436800in}{1.137040in}}%
\pgfpathlineto{\pgfqpoint{2.441309in}{1.149952in}}%
\pgfpathlineto{\pgfqpoint{2.445818in}{1.148021in}}%
\pgfpathlineto{\pgfqpoint{2.450327in}{1.161749in}}%
\pgfpathlineto{\pgfqpoint{2.454836in}{1.157346in}}%
\pgfpathlineto{\pgfqpoint{2.459345in}{1.136267in}}%
\pgfpathlineto{\pgfqpoint{2.463855in}{1.162003in}}%
\pgfpathlineto{\pgfqpoint{2.468364in}{1.151453in}}%
\pgfpathlineto{\pgfqpoint{2.472873in}{1.170269in}}%
\pgfpathlineto{\pgfqpoint{2.477382in}{1.177023in}}%
\pgfpathlineto{\pgfqpoint{2.481891in}{1.157898in}}%
\pgfpathlineto{\pgfqpoint{2.486400in}{1.167554in}}%
\pgfpathlineto{\pgfqpoint{2.490909in}{1.202262in}}%
\pgfpathlineto{\pgfqpoint{2.495418in}{1.166429in}}%
\pgfpathlineto{\pgfqpoint{2.499927in}{1.173172in}}%
\pgfpathlineto{\pgfqpoint{2.504436in}{1.226839in}}%
\pgfpathlineto{\pgfqpoint{2.508945in}{1.190498in}}%
\pgfpathlineto{\pgfqpoint{2.513455in}{1.186713in}}%
\pgfpathlineto{\pgfqpoint{2.517964in}{1.208917in}}%
\pgfpathlineto{\pgfqpoint{2.522473in}{1.179208in}}%
\pgfpathlineto{\pgfqpoint{2.526982in}{1.170987in}}%
\pgfpathlineto{\pgfqpoint{2.531491in}{1.190962in}}%
\pgfpathlineto{\pgfqpoint{2.536000in}{1.187750in}}%
\pgfpathlineto{\pgfqpoint{2.540509in}{1.196281in}}%
\pgfpathlineto{\pgfqpoint{2.545018in}{1.218011in}}%
\pgfpathlineto{\pgfqpoint{2.549527in}{1.220748in}}%
\pgfpathlineto{\pgfqpoint{2.554036in}{1.253602in}}%
\pgfpathlineto{\pgfqpoint{2.558545in}{1.178182in}}%
\pgfpathlineto{\pgfqpoint{2.563055in}{1.191833in}}%
\pgfpathlineto{\pgfqpoint{2.567564in}{1.218132in}}%
\pgfpathlineto{\pgfqpoint{2.572073in}{1.208542in}}%
\pgfpathlineto{\pgfqpoint{2.576582in}{1.238074in}}%
\pgfpathlineto{\pgfqpoint{2.581091in}{1.213718in}}%
\pgfpathlineto{\pgfqpoint{2.585600in}{1.204966in}}%
\pgfpathlineto{\pgfqpoint{2.590109in}{1.235955in}}%
\pgfpathlineto{\pgfqpoint{2.594618in}{1.216620in}}%
\pgfpathlineto{\pgfqpoint{2.603636in}{1.239895in}}%
\pgfpathlineto{\pgfqpoint{2.608145in}{1.243713in}}%
\pgfpathlineto{\pgfqpoint{2.612655in}{1.271888in}}%
\pgfpathlineto{\pgfqpoint{2.617164in}{1.278521in}}%
\pgfpathlineto{\pgfqpoint{2.621673in}{1.214380in}}%
\pgfpathlineto{\pgfqpoint{2.626182in}{1.277141in}}%
\pgfpathlineto{\pgfqpoint{2.630691in}{1.288663in}}%
\pgfpathlineto{\pgfqpoint{2.635200in}{1.239641in}}%
\pgfpathlineto{\pgfqpoint{2.639709in}{1.239498in}}%
\pgfpathlineto{\pgfqpoint{2.644218in}{1.251174in}}%
\pgfpathlineto{\pgfqpoint{2.648727in}{1.268202in}}%
\pgfpathlineto{\pgfqpoint{2.653236in}{1.269714in}}%
\pgfpathlineto{\pgfqpoint{2.657745in}{1.280176in}}%
\pgfpathlineto{\pgfqpoint{2.662255in}{1.245479in}}%
\pgfpathlineto{\pgfqpoint{2.666764in}{1.262585in}}%
\pgfpathlineto{\pgfqpoint{2.671273in}{1.260311in}}%
\pgfpathlineto{\pgfqpoint{2.675782in}{1.271634in}}%
\pgfpathlineto{\pgfqpoint{2.680291in}{1.290230in}}%
\pgfpathlineto{\pgfqpoint{2.684800in}{1.270542in}}%
\pgfpathlineto{\pgfqpoint{2.689309in}{1.278543in}}%
\pgfpathlineto{\pgfqpoint{2.693818in}{1.295913in}}%
\pgfpathlineto{\pgfqpoint{2.698327in}{1.318890in}}%
\pgfpathlineto{\pgfqpoint{2.702836in}{1.254087in}}%
\pgfpathlineto{\pgfqpoint{2.707345in}{1.338313in}}%
\pgfpathlineto{\pgfqpoint{2.711855in}{1.311639in}}%
\pgfpathlineto{\pgfqpoint{2.716364in}{1.277538in}}%
\pgfpathlineto{\pgfqpoint{2.720873in}{1.319221in}}%
\pgfpathlineto{\pgfqpoint{2.725382in}{1.323062in}}%
\pgfpathlineto{\pgfqpoint{2.729891in}{1.311176in}}%
\pgfpathlineto{\pgfqpoint{2.734400in}{1.335256in}}%
\pgfpathlineto{\pgfqpoint{2.738909in}{1.318382in}}%
\pgfpathlineto{\pgfqpoint{2.743418in}{1.314288in}}%
\pgfpathlineto{\pgfqpoint{2.747927in}{1.326715in}}%
\pgfpathlineto{\pgfqpoint{2.752436in}{1.324187in}}%
\pgfpathlineto{\pgfqpoint{2.756945in}{1.308119in}}%
\pgfpathlineto{\pgfqpoint{2.761455in}{1.333314in}}%
\pgfpathlineto{\pgfqpoint{2.765964in}{1.331261in}}%
\pgfpathlineto{\pgfqpoint{2.770473in}{1.287206in}}%
\pgfpathlineto{\pgfqpoint{2.774982in}{1.318912in}}%
\pgfpathlineto{\pgfqpoint{2.779491in}{1.328580in}}%
\pgfpathlineto{\pgfqpoint{2.784000in}{1.319078in}}%
\pgfpathlineto{\pgfqpoint{2.793018in}{1.323459in}}%
\pgfpathlineto{\pgfqpoint{2.797527in}{1.344571in}}%
\pgfpathlineto{\pgfqpoint{2.806545in}{1.373386in}}%
\pgfpathlineto{\pgfqpoint{2.811055in}{1.376442in}}%
\pgfpathlineto{\pgfqpoint{2.815564in}{1.354989in}}%
\pgfpathlineto{\pgfqpoint{2.820073in}{1.395060in}}%
\pgfpathlineto{\pgfqpoint{2.824582in}{1.376707in}}%
\pgfpathlineto{\pgfqpoint{2.829091in}{1.333590in}}%
\pgfpathlineto{\pgfqpoint{2.833600in}{1.370549in}}%
\pgfpathlineto{\pgfqpoint{2.838109in}{1.383881in}}%
\pgfpathlineto{\pgfqpoint{2.842618in}{1.414362in}}%
\pgfpathlineto{\pgfqpoint{2.847127in}{1.402565in}}%
\pgfpathlineto{\pgfqpoint{2.851636in}{1.368331in}}%
\pgfpathlineto{\pgfqpoint{2.856145in}{1.347010in}}%
\pgfpathlineto{\pgfqpoint{2.860655in}{1.406052in}}%
\pgfpathlineto{\pgfqpoint{2.865164in}{1.399883in}}%
\pgfpathlineto{\pgfqpoint{2.869673in}{1.404992in}}%
\pgfpathlineto{\pgfqpoint{2.874182in}{1.389608in}}%
\pgfpathlineto{\pgfqpoint{2.878691in}{1.409175in}}%
\pgfpathlineto{\pgfqpoint{2.883200in}{1.389840in}}%
\pgfpathlineto{\pgfqpoint{2.887709in}{1.380283in}}%
\pgfpathlineto{\pgfqpoint{2.892218in}{1.394288in}}%
\pgfpathlineto{\pgfqpoint{2.901236in}{1.415786in}}%
\pgfpathlineto{\pgfqpoint{2.905745in}{1.406228in}}%
\pgfpathlineto{\pgfqpoint{2.910255in}{1.421833in}}%
\pgfpathlineto{\pgfqpoint{2.914764in}{1.415223in}}%
\pgfpathlineto{\pgfqpoint{2.919273in}{1.437504in}}%
\pgfpathlineto{\pgfqpoint{2.923782in}{1.423886in}}%
\pgfpathlineto{\pgfqpoint{2.928291in}{1.431666in}}%
\pgfpathlineto{\pgfqpoint{2.932800in}{1.412519in}}%
\pgfpathlineto{\pgfqpoint{2.937309in}{1.381949in}}%
\pgfpathlineto{\pgfqpoint{2.941818in}{1.445075in}}%
\pgfpathlineto{\pgfqpoint{2.946327in}{1.459378in}}%
\pgfpathlineto{\pgfqpoint{2.950836in}{1.413347in}}%
\pgfpathlineto{\pgfqpoint{2.955345in}{1.420807in}}%
\pgfpathlineto{\pgfqpoint{2.959855in}{1.424052in}}%
\pgfpathlineto{\pgfqpoint{2.964364in}{1.463461in}}%
\pgfpathlineto{\pgfqpoint{2.968873in}{1.453407in}}%
\pgfpathlineto{\pgfqpoint{2.973382in}{1.520428in}}%
\pgfpathlineto{\pgfqpoint{2.977891in}{1.475810in}}%
\pgfpathlineto{\pgfqpoint{2.982400in}{1.468372in}}%
\pgfpathlineto{\pgfqpoint{2.986909in}{1.409639in}}%
\pgfpathlineto{\pgfqpoint{2.991418in}{1.476251in}}%
\pgfpathlineto{\pgfqpoint{2.995927in}{1.488744in}}%
\pgfpathlineto{\pgfqpoint{3.000436in}{1.410356in}}%
\pgfpathlineto{\pgfqpoint{3.004945in}{1.462357in}}%
\pgfpathlineto{\pgfqpoint{3.009455in}{1.497109in}}%
\pgfpathlineto{\pgfqpoint{3.013964in}{1.485864in}}%
\pgfpathlineto{\pgfqpoint{3.018473in}{1.545347in}}%
\pgfpathlineto{\pgfqpoint{3.022982in}{1.528595in}}%
\pgfpathlineto{\pgfqpoint{3.027491in}{1.455647in}}%
\pgfpathlineto{\pgfqpoint{3.032000in}{1.516235in}}%
\pgfpathlineto{\pgfqpoint{3.036509in}{1.515970in}}%
\pgfpathlineto{\pgfqpoint{3.041018in}{1.524059in}}%
\pgfpathlineto{\pgfqpoint{3.045527in}{1.522669in}}%
\pgfpathlineto{\pgfqpoint{3.050036in}{1.518243in}}%
\pgfpathlineto{\pgfqpoint{3.054545in}{1.477046in}}%
\pgfpathlineto{\pgfqpoint{3.059055in}{1.496050in}}%
\pgfpathlineto{\pgfqpoint{3.063564in}{1.549398in}}%
\pgfpathlineto{\pgfqpoint{3.068073in}{1.549718in}}%
\pgfpathlineto{\pgfqpoint{3.072582in}{1.538207in}}%
\pgfpathlineto{\pgfqpoint{3.077091in}{1.484142in}}%
\pgfpathlineto{\pgfqpoint{3.081600in}{1.534334in}}%
\pgfpathlineto{\pgfqpoint{3.086109in}{1.517813in}}%
\pgfpathlineto{\pgfqpoint{3.090618in}{1.527006in}}%
\pgfpathlineto{\pgfqpoint{3.095127in}{1.576723in}}%
\pgfpathlineto{\pgfqpoint{3.099636in}{1.483866in}}%
\pgfpathlineto{\pgfqpoint{3.104145in}{1.509381in}}%
\pgfpathlineto{\pgfqpoint{3.108655in}{1.572154in}}%
\pgfpathlineto{\pgfqpoint{3.113164in}{1.581104in}}%
\pgfpathlineto{\pgfqpoint{3.117673in}{1.573942in}}%
\pgfpathlineto{\pgfqpoint{3.122182in}{1.553227in}}%
\pgfpathlineto{\pgfqpoint{3.126691in}{1.509757in}}%
\pgfpathlineto{\pgfqpoint{3.131200in}{1.563060in}}%
\pgfpathlineto{\pgfqpoint{3.135709in}{1.523905in}}%
\pgfpathlineto{\pgfqpoint{3.140218in}{1.572661in}}%
\pgfpathlineto{\pgfqpoint{3.144727in}{1.523099in}}%
\pgfpathlineto{\pgfqpoint{3.149236in}{1.572408in}}%
\pgfpathlineto{\pgfqpoint{3.153745in}{1.561085in}}%
\pgfpathlineto{\pgfqpoint{3.158255in}{1.565422in}}%
\pgfpathlineto{\pgfqpoint{3.162764in}{1.614499in}}%
\pgfpathlineto{\pgfqpoint{3.167273in}{1.606586in}}%
\pgfpathlineto{\pgfqpoint{3.171782in}{1.618218in}}%
\pgfpathlineto{\pgfqpoint{3.176291in}{1.611905in}}%
\pgfpathlineto{\pgfqpoint{3.180800in}{1.576756in}}%
\pgfpathlineto{\pgfqpoint{3.185309in}{1.592603in}}%
\pgfpathlineto{\pgfqpoint{3.189818in}{1.643126in}}%
\pgfpathlineto{\pgfqpoint{3.194327in}{1.619067in}}%
\pgfpathlineto{\pgfqpoint{3.198836in}{1.635114in}}%
\pgfpathlineto{\pgfqpoint{3.203345in}{1.596091in}}%
\pgfpathlineto{\pgfqpoint{3.207855in}{1.605262in}}%
\pgfpathlineto{\pgfqpoint{3.212364in}{1.625546in}}%
\pgfpathlineto{\pgfqpoint{3.216873in}{1.624729in}}%
\pgfpathlineto{\pgfqpoint{3.221382in}{1.662869in}}%
\pgfpathlineto{\pgfqpoint{3.225891in}{1.603110in}}%
\pgfpathlineto{\pgfqpoint{3.230400in}{1.613439in}}%
\pgfpathlineto{\pgfqpoint{3.234909in}{1.630026in}}%
\pgfpathlineto{\pgfqpoint{3.239418in}{1.651270in}}%
\pgfpathlineto{\pgfqpoint{3.243927in}{1.600439in}}%
\pgfpathlineto{\pgfqpoint{3.252945in}{1.681652in}}%
\pgfpathlineto{\pgfqpoint{3.257455in}{1.634407in}}%
\pgfpathlineto{\pgfqpoint{3.261964in}{1.708878in}}%
\pgfpathlineto{\pgfqpoint{3.266473in}{1.664811in}}%
\pgfpathlineto{\pgfqpoint{3.275491in}{1.672481in}}%
\pgfpathlineto{\pgfqpoint{3.280000in}{1.689841in}}%
\pgfpathlineto{\pgfqpoint{3.284509in}{1.664657in}}%
\pgfpathlineto{\pgfqpoint{3.289018in}{1.709573in}}%
\pgfpathlineto{\pgfqpoint{3.293527in}{1.698670in}}%
\pgfpathlineto{\pgfqpoint{3.298036in}{1.665429in}}%
\pgfpathlineto{\pgfqpoint{3.302545in}{1.751598in}}%
\pgfpathlineto{\pgfqpoint{3.307055in}{1.734404in}}%
\pgfpathlineto{\pgfqpoint{3.316073in}{1.711173in}}%
\pgfpathlineto{\pgfqpoint{3.320582in}{1.749126in}}%
\pgfpathlineto{\pgfqpoint{3.325091in}{1.690360in}}%
\pgfpathlineto{\pgfqpoint{3.329600in}{1.672934in}}%
\pgfpathlineto{\pgfqpoint{3.334109in}{1.742240in}}%
\pgfpathlineto{\pgfqpoint{3.343127in}{1.691320in}}%
\pgfpathlineto{\pgfqpoint{3.347636in}{1.726392in}}%
\pgfpathlineto{\pgfqpoint{3.352145in}{1.721260in}}%
\pgfpathlineto{\pgfqpoint{3.356655in}{1.756752in}}%
\pgfpathlineto{\pgfqpoint{3.361164in}{1.733013in}}%
\pgfpathlineto{\pgfqpoint{3.365673in}{1.746345in}}%
\pgfpathlineto{\pgfqpoint{3.370182in}{1.770260in}}%
\pgfpathlineto{\pgfqpoint{3.374691in}{1.676311in}}%
\pgfpathlineto{\pgfqpoint{3.379200in}{1.773493in}}%
\pgfpathlineto{\pgfqpoint{3.383709in}{1.740717in}}%
\pgfpathlineto{\pgfqpoint{3.388218in}{1.721525in}}%
\pgfpathlineto{\pgfqpoint{3.401745in}{1.795742in}}%
\pgfpathlineto{\pgfqpoint{3.406255in}{1.801701in}}%
\pgfpathlineto{\pgfqpoint{3.410764in}{1.743200in}}%
\pgfpathlineto{\pgfqpoint{3.415273in}{1.741754in}}%
\pgfpathlineto{\pgfqpoint{3.419782in}{1.792784in}}%
\pgfpathlineto{\pgfqpoint{3.424291in}{1.785898in}}%
\pgfpathlineto{\pgfqpoint{3.428800in}{1.809338in}}%
\pgfpathlineto{\pgfqpoint{3.433309in}{1.814746in}}%
\pgfpathlineto{\pgfqpoint{3.437818in}{1.823409in}}%
\pgfpathlineto{\pgfqpoint{3.442327in}{1.814094in}}%
\pgfpathlineto{\pgfqpoint{3.446836in}{1.762766in}}%
\pgfpathlineto{\pgfqpoint{3.451345in}{1.774718in}}%
\pgfpathlineto{\pgfqpoint{3.455855in}{1.855667in}}%
\pgfpathlineto{\pgfqpoint{3.464873in}{1.792243in}}%
\pgfpathlineto{\pgfqpoint{3.469382in}{1.818674in}}%
\pgfpathlineto{\pgfqpoint{3.473891in}{1.855954in}}%
\pgfpathlineto{\pgfqpoint{3.478400in}{1.829688in}}%
\pgfpathlineto{\pgfqpoint{3.482909in}{1.761276in}}%
\pgfpathlineto{\pgfqpoint{3.487418in}{1.830240in}}%
\pgfpathlineto{\pgfqpoint{3.491927in}{1.876116in}}%
\pgfpathlineto{\pgfqpoint{3.496436in}{1.867398in}}%
\pgfpathlineto{\pgfqpoint{3.500945in}{1.774531in}}%
\pgfpathlineto{\pgfqpoint{3.505455in}{1.835041in}}%
\pgfpathlineto{\pgfqpoint{3.509964in}{1.804570in}}%
\pgfpathlineto{\pgfqpoint{3.514473in}{1.855777in}}%
\pgfpathlineto{\pgfqpoint{3.518982in}{1.882142in}}%
\pgfpathlineto{\pgfqpoint{3.523491in}{1.866328in}}%
\pgfpathlineto{\pgfqpoint{3.528000in}{1.856991in}}%
\pgfpathlineto{\pgfqpoint{3.532509in}{1.914720in}}%
\pgfpathlineto{\pgfqpoint{3.537018in}{1.879251in}}%
\pgfpathlineto{\pgfqpoint{3.541527in}{1.917821in}}%
\pgfpathlineto{\pgfqpoint{3.546036in}{1.914709in}}%
\pgfpathlineto{\pgfqpoint{3.550545in}{1.864451in}}%
\pgfpathlineto{\pgfqpoint{3.555055in}{1.861394in}}%
\pgfpathlineto{\pgfqpoint{3.559564in}{1.929994in}}%
\pgfpathlineto{\pgfqpoint{3.564073in}{1.870025in}}%
\pgfpathlineto{\pgfqpoint{3.568582in}{1.898950in}}%
\pgfpathlineto{\pgfqpoint{3.573091in}{1.879615in}}%
\pgfpathlineto{\pgfqpoint{3.577600in}{1.913760in}}%
\pgfpathlineto{\pgfqpoint{3.582109in}{1.893807in}}%
\pgfpathlineto{\pgfqpoint{3.586618in}{1.893068in}}%
\pgfpathlineto{\pgfqpoint{3.591127in}{1.930568in}}%
\pgfpathlineto{\pgfqpoint{3.595636in}{1.899314in}}%
\pgfpathlineto{\pgfqpoint{3.604655in}{1.940290in}}%
\pgfpathlineto{\pgfqpoint{3.609164in}{1.881723in}}%
\pgfpathlineto{\pgfqpoint{3.613673in}{1.932720in}}%
\pgfpathlineto{\pgfqpoint{3.618182in}{1.934320in}}%
\pgfpathlineto{\pgfqpoint{3.622691in}{1.951227in}}%
\pgfpathlineto{\pgfqpoint{3.627200in}{1.923107in}}%
\pgfpathlineto{\pgfqpoint{3.631709in}{1.979247in}}%
\pgfpathlineto{\pgfqpoint{3.636218in}{1.926606in}}%
\pgfpathlineto{\pgfqpoint{3.640727in}{1.927500in}}%
\pgfpathlineto{\pgfqpoint{3.645236in}{1.961347in}}%
\pgfpathlineto{\pgfqpoint{3.649745in}{1.955487in}}%
\pgfpathlineto{\pgfqpoint{3.654255in}{1.942233in}}%
\pgfpathlineto{\pgfqpoint{3.658764in}{1.998284in}}%
\pgfpathlineto{\pgfqpoint{3.663273in}{1.945532in}}%
\pgfpathlineto{\pgfqpoint{3.672291in}{1.999597in}}%
\pgfpathlineto{\pgfqpoint{3.681309in}{1.950068in}}%
\pgfpathlineto{\pgfqpoint{3.685818in}{2.054998in}}%
\pgfpathlineto{\pgfqpoint{3.690327in}{2.004056in}}%
\pgfpathlineto{\pgfqpoint{3.694836in}{1.985372in}}%
\pgfpathlineto{\pgfqpoint{3.699345in}{2.053475in}}%
\pgfpathlineto{\pgfqpoint{3.703855in}{1.991552in}}%
\pgfpathlineto{\pgfqpoint{3.708364in}{1.999024in}}%
\pgfpathlineto{\pgfqpoint{3.712873in}{2.020102in}}%
\pgfpathlineto{\pgfqpoint{3.717382in}{1.977415in}}%
\pgfpathlineto{\pgfqpoint{3.721891in}{2.071706in}}%
\pgfpathlineto{\pgfqpoint{3.726400in}{2.113003in}}%
\pgfpathlineto{\pgfqpoint{3.730909in}{2.116744in}}%
\pgfpathlineto{\pgfqpoint{3.735418in}{2.041986in}}%
\pgfpathlineto{\pgfqpoint{3.739927in}{2.020290in}}%
\pgfpathlineto{\pgfqpoint{3.744436in}{2.041368in}}%
\pgfpathlineto{\pgfqpoint{3.748945in}{2.008890in}}%
\pgfpathlineto{\pgfqpoint{3.753455in}{2.023236in}}%
\pgfpathlineto{\pgfqpoint{3.757964in}{2.065990in}}%
\pgfpathlineto{\pgfqpoint{3.762473in}{2.066332in}}%
\pgfpathlineto{\pgfqpoint{3.766982in}{2.041159in}}%
\pgfpathlineto{\pgfqpoint{3.771491in}{2.090103in}}%
\pgfpathlineto{\pgfqpoint{3.776000in}{2.046313in}}%
\pgfpathlineto{\pgfqpoint{3.780509in}{2.061233in}}%
\pgfpathlineto{\pgfqpoint{3.785018in}{2.094264in}}%
\pgfpathlineto{\pgfqpoint{3.789527in}{2.093513in}}%
\pgfpathlineto{\pgfqpoint{3.798545in}{2.133165in}}%
\pgfpathlineto{\pgfqpoint{3.803055in}{2.131311in}}%
\pgfpathlineto{\pgfqpoint{3.807564in}{2.100896in}}%
\pgfpathlineto{\pgfqpoint{3.816582in}{2.142789in}}%
\pgfpathlineto{\pgfqpoint{3.821091in}{2.104615in}}%
\pgfpathlineto{\pgfqpoint{3.830109in}{2.056433in}}%
\pgfpathlineto{\pgfqpoint{3.839127in}{2.097707in}}%
\pgfpathlineto{\pgfqpoint{3.843636in}{2.137072in}}%
\pgfpathlineto{\pgfqpoint{3.848145in}{2.160733in}}%
\pgfpathlineto{\pgfqpoint{3.852655in}{2.175621in}}%
\pgfpathlineto{\pgfqpoint{3.857164in}{2.150282in}}%
\pgfpathlineto{\pgfqpoint{3.861673in}{2.161351in}}%
\pgfpathlineto{\pgfqpoint{3.866182in}{2.111888in}}%
\pgfpathlineto{\pgfqpoint{3.870691in}{2.160325in}}%
\pgfpathlineto{\pgfqpoint{3.875200in}{2.162731in}}%
\pgfpathlineto{\pgfqpoint{3.879709in}{2.180300in}}%
\pgfpathlineto{\pgfqpoint{3.884218in}{2.156716in}}%
\pgfpathlineto{\pgfqpoint{3.888727in}{2.120143in}}%
\pgfpathlineto{\pgfqpoint{3.893236in}{2.212867in}}%
\pgfpathlineto{\pgfqpoint{3.897745in}{2.174296in}}%
\pgfpathlineto{\pgfqpoint{3.902255in}{2.150161in}}%
\pgfpathlineto{\pgfqpoint{3.906764in}{2.156396in}}%
\pgfpathlineto{\pgfqpoint{3.911273in}{2.242189in}}%
\pgfpathlineto{\pgfqpoint{3.915782in}{2.142524in}}%
\pgfpathlineto{\pgfqpoint{3.920291in}{2.242554in}}%
\pgfpathlineto{\pgfqpoint{3.924800in}{2.149156in}}%
\pgfpathlineto{\pgfqpoint{3.929309in}{2.146751in}}%
\pgfpathlineto{\pgfqpoint{3.933818in}{2.194238in}}%
\pgfpathlineto{\pgfqpoint{3.938327in}{2.278818in}}%
\pgfpathlineto{\pgfqpoint{3.942836in}{2.282371in}}%
\pgfpathlineto{\pgfqpoint{3.947345in}{2.317167in}}%
\pgfpathlineto{\pgfqpoint{3.951855in}{2.249738in}}%
\pgfpathlineto{\pgfqpoint{3.956364in}{2.230359in}}%
\pgfpathlineto{\pgfqpoint{3.960873in}{2.222049in}}%
\pgfpathlineto{\pgfqpoint{3.965382in}{2.207603in}}%
\pgfpathlineto{\pgfqpoint{3.969891in}{2.273995in}}%
\pgfpathlineto{\pgfqpoint{3.974400in}{2.283872in}}%
\pgfpathlineto{\pgfqpoint{3.978909in}{2.253777in}}%
\pgfpathlineto{\pgfqpoint{3.983418in}{2.189283in}}%
\pgfpathlineto{\pgfqpoint{3.987927in}{2.242322in}}%
\pgfpathlineto{\pgfqpoint{3.992436in}{2.216067in}}%
\pgfpathlineto{\pgfqpoint{3.996945in}{2.308350in}}%
\pgfpathlineto{\pgfqpoint{4.001455in}{2.247332in}}%
\pgfpathlineto{\pgfqpoint{4.005964in}{2.226739in}}%
\pgfpathlineto{\pgfqpoint{4.010473in}{2.287713in}}%
\pgfpathlineto{\pgfqpoint{4.014982in}{2.250709in}}%
\pgfpathlineto{\pgfqpoint{4.019491in}{2.327453in}}%
\pgfpathlineto{\pgfqpoint{4.024000in}{2.251018in}}%
\pgfpathlineto{\pgfqpoint{4.028509in}{2.267274in}}%
\pgfpathlineto{\pgfqpoint{4.033018in}{2.236230in}}%
\pgfpathlineto{\pgfqpoint{4.037527in}{2.329495in}}%
\pgfpathlineto{\pgfqpoint{4.042036in}{2.338710in}}%
\pgfpathlineto{\pgfqpoint{4.046545in}{2.291101in}}%
\pgfpathlineto{\pgfqpoint{4.051055in}{2.368352in}}%
\pgfpathlineto{\pgfqpoint{4.055564in}{2.374621in}}%
\pgfpathlineto{\pgfqpoint{4.060073in}{2.332165in}}%
\pgfpathlineto{\pgfqpoint{4.064582in}{2.276566in}}%
\pgfpathlineto{\pgfqpoint{4.069091in}{2.345077in}}%
\pgfpathlineto{\pgfqpoint{4.073600in}{2.320842in}}%
\pgfpathlineto{\pgfqpoint{4.078109in}{2.327276in}}%
\pgfpathlineto{\pgfqpoint{4.082618in}{2.320964in}}%
\pgfpathlineto{\pgfqpoint{4.087127in}{2.332397in}}%
\pgfpathlineto{\pgfqpoint{4.091636in}{2.384520in}}%
\pgfpathlineto{\pgfqpoint{4.096145in}{2.336028in}}%
\pgfpathlineto{\pgfqpoint{4.105164in}{2.448870in}}%
\pgfpathlineto{\pgfqpoint{4.109673in}{2.331227in}}%
\pgfpathlineto{\pgfqpoint{4.118691in}{2.429690in}}%
\pgfpathlineto{\pgfqpoint{4.123200in}{2.406845in}}%
\pgfpathlineto{\pgfqpoint{4.127709in}{2.450360in}}%
\pgfpathlineto{\pgfqpoint{4.132218in}{2.318790in}}%
\pgfpathlineto{\pgfqpoint{4.136727in}{2.384740in}}%
\pgfpathlineto{\pgfqpoint{4.145745in}{2.470313in}}%
\pgfpathlineto{\pgfqpoint{4.150255in}{2.447987in}}%
\pgfpathlineto{\pgfqpoint{4.154764in}{2.394309in}}%
\pgfpathlineto{\pgfqpoint{4.159273in}{2.459222in}}%
\pgfpathlineto{\pgfqpoint{4.163782in}{2.458273in}}%
\pgfpathlineto{\pgfqpoint{4.168291in}{2.399782in}}%
\pgfpathlineto{\pgfqpoint{4.172800in}{2.429855in}}%
\pgfpathlineto{\pgfqpoint{4.177309in}{2.423333in}}%
\pgfpathlineto{\pgfqpoint{4.181818in}{2.435031in}}%
\pgfpathlineto{\pgfqpoint{4.195345in}{2.557618in}}%
\pgfpathlineto{\pgfqpoint{4.199855in}{2.447844in}}%
\pgfpathlineto{\pgfqpoint{4.204364in}{2.520162in}}%
\pgfpathlineto{\pgfqpoint{4.208873in}{2.395876in}}%
\pgfpathlineto{\pgfqpoint{4.213382in}{2.493742in}}%
\pgfpathlineto{\pgfqpoint{4.217891in}{2.502770in}}%
\pgfpathlineto{\pgfqpoint{4.226909in}{2.459200in}}%
\pgfpathlineto{\pgfqpoint{4.231418in}{2.500684in}}%
\pgfpathlineto{\pgfqpoint{4.235927in}{2.459244in}}%
\pgfpathlineto{\pgfqpoint{4.240436in}{2.510980in}}%
\pgfpathlineto{\pgfqpoint{4.244945in}{2.486337in}}%
\pgfpathlineto{\pgfqpoint{4.249455in}{2.497561in}}%
\pgfpathlineto{\pgfqpoint{4.253964in}{2.545269in}}%
\pgfpathlineto{\pgfqpoint{4.258473in}{2.553083in}}%
\pgfpathlineto{\pgfqpoint{4.262982in}{2.518507in}}%
\pgfpathlineto{\pgfqpoint{4.267491in}{2.534377in}}%
\pgfpathlineto{\pgfqpoint{4.272000in}{2.604962in}}%
\pgfpathlineto{\pgfqpoint{4.276509in}{2.538294in}}%
\pgfpathlineto{\pgfqpoint{4.281018in}{2.440075in}}%
\pgfpathlineto{\pgfqpoint{4.285527in}{2.575430in}}%
\pgfpathlineto{\pgfqpoint{4.290036in}{2.607170in}}%
\pgfpathlineto{\pgfqpoint{4.294545in}{2.540524in}}%
\pgfpathlineto{\pgfqpoint{4.299055in}{2.618404in}}%
\pgfpathlineto{\pgfqpoint{4.303564in}{2.631062in}}%
\pgfpathlineto{\pgfqpoint{4.308073in}{2.617919in}}%
\pgfpathlineto{\pgfqpoint{4.312582in}{2.540954in}}%
\pgfpathlineto{\pgfqpoint{4.321600in}{2.592481in}}%
\pgfpathlineto{\pgfqpoint{4.326109in}{2.675857in}}%
\pgfpathlineto{\pgfqpoint{4.330618in}{2.556537in}}%
\pgfpathlineto{\pgfqpoint{4.335127in}{2.631338in}}%
\pgfpathlineto{\pgfqpoint{4.339636in}{2.585208in}}%
\pgfpathlineto{\pgfqpoint{4.344145in}{2.504127in}}%
\pgfpathlineto{\pgfqpoint{4.348655in}{2.607137in}}%
\pgfpathlineto{\pgfqpoint{4.353164in}{2.667580in}}%
\pgfpathlineto{\pgfqpoint{4.357673in}{2.559881in}}%
\pgfpathlineto{\pgfqpoint{4.362182in}{2.551074in}}%
\pgfpathlineto{\pgfqpoint{4.366691in}{2.624187in}}%
\pgfpathlineto{\pgfqpoint{4.371200in}{2.590572in}}%
\pgfpathlineto{\pgfqpoint{4.375709in}{2.641337in}}%
\pgfpathlineto{\pgfqpoint{4.380218in}{2.669953in}}%
\pgfpathlineto{\pgfqpoint{4.384727in}{2.644637in}}%
\pgfpathlineto{\pgfqpoint{4.389236in}{2.673606in}}%
\pgfpathlineto{\pgfqpoint{4.393745in}{2.669181in}}%
\pgfpathlineto{\pgfqpoint{4.398255in}{2.589005in}}%
\pgfpathlineto{\pgfqpoint{4.402764in}{2.605625in}}%
\pgfpathlineto{\pgfqpoint{4.407273in}{2.689950in}}%
\pgfpathlineto{\pgfqpoint{4.416291in}{2.722186in}}%
\pgfpathlineto{\pgfqpoint{4.420800in}{2.700843in}}%
\pgfpathlineto{\pgfqpoint{4.425309in}{2.643268in}}%
\pgfpathlineto{\pgfqpoint{4.429818in}{2.702211in}}%
\pgfpathlineto{\pgfqpoint{4.434327in}{2.691319in}}%
\pgfpathlineto{\pgfqpoint{4.438836in}{2.685326in}}%
\pgfpathlineto{\pgfqpoint{4.443345in}{2.699000in}}%
\pgfpathlineto{\pgfqpoint{4.447855in}{2.693140in}}%
\pgfpathlineto{\pgfqpoint{4.452364in}{2.749224in}}%
\pgfpathlineto{\pgfqpoint{4.456873in}{2.703226in}}%
\pgfpathlineto{\pgfqpoint{4.461382in}{2.766352in}}%
\pgfpathlineto{\pgfqpoint{4.465891in}{2.673661in}}%
\pgfpathlineto{\pgfqpoint{4.470400in}{2.684796in}}%
\pgfpathlineto{\pgfqpoint{4.474909in}{2.659325in}}%
\pgfpathlineto{\pgfqpoint{4.479418in}{2.799184in}}%
\pgfpathlineto{\pgfqpoint{4.483927in}{2.707045in}}%
\pgfpathlineto{\pgfqpoint{4.488436in}{2.763659in}}%
\pgfpathlineto{\pgfqpoint{4.492945in}{2.897691in}}%
\pgfpathlineto{\pgfqpoint{4.497455in}{2.825758in}}%
\pgfpathlineto{\pgfqpoint{4.501964in}{2.822337in}}%
\pgfpathlineto{\pgfqpoint{4.506473in}{2.821918in}}%
\pgfpathlineto{\pgfqpoint{4.510982in}{2.832214in}}%
\pgfpathlineto{\pgfqpoint{4.515491in}{2.741400in}}%
\pgfpathlineto{\pgfqpoint{4.520000in}{2.779573in}}%
\pgfpathlineto{\pgfqpoint{4.524509in}{2.774640in}}%
\pgfpathlineto{\pgfqpoint{4.529018in}{2.816654in}}%
\pgfpathlineto{\pgfqpoint{4.533527in}{2.820527in}}%
\pgfpathlineto{\pgfqpoint{4.538036in}{2.799085in}}%
\pgfpathlineto{\pgfqpoint{4.542545in}{2.859804in}}%
\pgfpathlineto{\pgfqpoint{4.547055in}{2.872120in}}%
\pgfpathlineto{\pgfqpoint{4.551564in}{2.817824in}}%
\pgfpathlineto{\pgfqpoint{4.556073in}{2.723621in}}%
\pgfpathlineto{\pgfqpoint{4.565091in}{2.857520in}}%
\pgfpathlineto{\pgfqpoint{4.569600in}{2.828562in}}%
\pgfpathlineto{\pgfqpoint{4.574109in}{2.843460in}}%
\pgfpathlineto{\pgfqpoint{4.578618in}{2.820594in}}%
\pgfpathlineto{\pgfqpoint{4.583127in}{2.832700in}}%
\pgfpathlineto{\pgfqpoint{4.587636in}{2.948401in}}%
\pgfpathlineto{\pgfqpoint{4.592145in}{2.842092in}}%
\pgfpathlineto{\pgfqpoint{4.596655in}{2.906453in}}%
\pgfpathlineto{\pgfqpoint{4.601164in}{2.820483in}}%
\pgfpathlineto{\pgfqpoint{4.605673in}{2.864726in}}%
\pgfpathlineto{\pgfqpoint{4.610182in}{2.870719in}}%
\pgfpathlineto{\pgfqpoint{4.614691in}{2.865289in}}%
\pgfpathlineto{\pgfqpoint{4.619200in}{2.876038in}}%
\pgfpathlineto{\pgfqpoint{4.623709in}{2.924100in}}%
\pgfpathlineto{\pgfqpoint{4.628218in}{2.947198in}}%
\pgfpathlineto{\pgfqpoint{4.632727in}{2.934860in}}%
\pgfpathlineto{\pgfqpoint{4.637236in}{2.865753in}}%
\pgfpathlineto{\pgfqpoint{4.641745in}{2.914741in}}%
\pgfpathlineto{\pgfqpoint{4.646255in}{2.839079in}}%
\pgfpathlineto{\pgfqpoint{4.650764in}{2.873842in}}%
\pgfpathlineto{\pgfqpoint{4.655273in}{2.873003in}}%
\pgfpathlineto{\pgfqpoint{4.659782in}{2.959602in}}%
\pgfpathlineto{\pgfqpoint{4.664291in}{2.928028in}}%
\pgfpathlineto{\pgfqpoint{4.668800in}{2.988362in}}%
\pgfpathlineto{\pgfqpoint{4.673309in}{2.914962in}}%
\pgfpathlineto{\pgfqpoint{4.677818in}{3.025498in}}%
\pgfpathlineto{\pgfqpoint{4.682327in}{2.982678in}}%
\pgfpathlineto{\pgfqpoint{4.686836in}{2.983649in}}%
\pgfpathlineto{\pgfqpoint{4.691345in}{2.957274in}}%
\pgfpathlineto{\pgfqpoint{4.695855in}{2.988384in}}%
\pgfpathlineto{\pgfqpoint{4.704873in}{2.945178in}}%
\pgfpathlineto{\pgfqpoint{4.709382in}{2.994641in}}%
\pgfpathlineto{\pgfqpoint{4.713891in}{3.003658in}}%
\pgfpathlineto{\pgfqpoint{4.718400in}{3.031137in}}%
\pgfpathlineto{\pgfqpoint{4.722909in}{2.989366in}}%
\pgfpathlineto{\pgfqpoint{4.727418in}{2.993725in}}%
\pgfpathlineto{\pgfqpoint{4.731927in}{2.936383in}}%
\pgfpathlineto{\pgfqpoint{4.736436in}{3.109481in}}%
\pgfpathlineto{\pgfqpoint{4.740945in}{3.067523in}}%
\pgfpathlineto{\pgfqpoint{4.745455in}{2.997389in}}%
\pgfpathlineto{\pgfqpoint{4.749964in}{3.029570in}}%
\pgfpathlineto{\pgfqpoint{4.754473in}{3.019152in}}%
\pgfpathlineto{\pgfqpoint{4.758982in}{3.038012in}}%
\pgfpathlineto{\pgfqpoint{4.763491in}{3.192736in}}%
\pgfpathlineto{\pgfqpoint{4.772509in}{3.070546in}}%
\pgfpathlineto{\pgfqpoint{4.777018in}{3.032329in}}%
\pgfpathlineto{\pgfqpoint{4.781527in}{3.124369in}}%
\pgfpathlineto{\pgfqpoint{4.786036in}{2.997621in}}%
\pgfpathlineto{\pgfqpoint{4.790545in}{2.991408in}}%
\pgfpathlineto{\pgfqpoint{4.795055in}{3.159882in}}%
\pgfpathlineto{\pgfqpoint{4.799564in}{3.004673in}}%
\pgfpathlineto{\pgfqpoint{4.804073in}{3.149608in}}%
\pgfpathlineto{\pgfqpoint{4.808582in}{3.119491in}}%
\pgfpathlineto{\pgfqpoint{4.813091in}{3.141861in}}%
\pgfpathlineto{\pgfqpoint{4.817600in}{3.126543in}}%
\pgfpathlineto{\pgfqpoint{4.822109in}{3.144951in}}%
\pgfpathlineto{\pgfqpoint{4.826618in}{3.228890in}}%
\pgfpathlineto{\pgfqpoint{4.831127in}{3.243678in}}%
\pgfpathlineto{\pgfqpoint{4.835636in}{3.151329in}}%
\pgfpathlineto{\pgfqpoint{4.840145in}{3.134610in}}%
\pgfpathlineto{\pgfqpoint{4.844655in}{3.124468in}}%
\pgfpathlineto{\pgfqpoint{4.849164in}{3.196025in}}%
\pgfpathlineto{\pgfqpoint{4.853673in}{3.167122in}}%
\pgfpathlineto{\pgfqpoint{4.858182in}{3.128562in}}%
\pgfpathlineto{\pgfqpoint{4.862691in}{3.182186in}}%
\pgfpathlineto{\pgfqpoint{4.867200in}{3.249649in}}%
\pgfpathlineto{\pgfqpoint{4.871709in}{3.247640in}}%
\pgfpathlineto{\pgfqpoint{4.876218in}{3.175586in}}%
\pgfpathlineto{\pgfqpoint{4.880727in}{3.187549in}}%
\pgfpathlineto{\pgfqpoint{4.885236in}{3.314893in}}%
\pgfpathlineto{\pgfqpoint{4.894255in}{3.130460in}}%
\pgfpathlineto{\pgfqpoint{4.898764in}{3.278187in}}%
\pgfpathlineto{\pgfqpoint{4.903273in}{3.263289in}}%
\pgfpathlineto{\pgfqpoint{4.907782in}{3.288098in}}%
\pgfpathlineto{\pgfqpoint{4.912291in}{3.254394in}}%
\pgfpathlineto{\pgfqpoint{4.916800in}{3.201135in}}%
\pgfpathlineto{\pgfqpoint{4.921309in}{3.304784in}}%
\pgfpathlineto{\pgfqpoint{4.930327in}{3.212678in}}%
\pgfpathlineto{\pgfqpoint{4.934836in}{3.288881in}}%
\pgfpathlineto{\pgfqpoint{4.939345in}{3.250057in}}%
\pgfpathlineto{\pgfqpoint{4.943855in}{3.264944in}}%
\pgfpathlineto{\pgfqpoint{4.948364in}{3.345396in}}%
\pgfpathlineto{\pgfqpoint{4.952873in}{3.246691in}}%
\pgfpathlineto{\pgfqpoint{4.957382in}{3.226109in}}%
\pgfpathlineto{\pgfqpoint{4.961891in}{3.334416in}}%
\pgfpathlineto{\pgfqpoint{4.966400in}{3.256998in}}%
\pgfpathlineto{\pgfqpoint{4.970909in}{3.374310in}}%
\pgfpathlineto{\pgfqpoint{4.975418in}{3.387962in}}%
\pgfpathlineto{\pgfqpoint{4.979927in}{3.272482in}}%
\pgfpathlineto{\pgfqpoint{4.984436in}{3.378063in}}%
\pgfpathlineto{\pgfqpoint{4.988945in}{3.289775in}}%
\pgfpathlineto{\pgfqpoint{4.993455in}{3.386516in}}%
\pgfpathlineto{\pgfqpoint{4.997964in}{3.321459in}}%
\pgfpathlineto{\pgfqpoint{5.002473in}{3.354744in}}%
\pgfpathlineto{\pgfqpoint{5.006982in}{3.398810in}}%
\pgfpathlineto{\pgfqpoint{5.011491in}{3.197603in}}%
\pgfpathlineto{\pgfqpoint{5.016000in}{3.465721in}}%
\pgfpathlineto{\pgfqpoint{5.020509in}{3.398965in}}%
\pgfpathlineto{\pgfqpoint{5.025018in}{3.291729in}}%
\pgfpathlineto{\pgfqpoint{5.029527in}{3.420772in}}%
\pgfpathlineto{\pgfqpoint{5.034036in}{3.402783in}}%
\pgfpathlineto{\pgfqpoint{5.038545in}{3.379188in}}%
\pgfpathlineto{\pgfqpoint{5.043055in}{3.402231in}}%
\pgfpathlineto{\pgfqpoint{5.047564in}{3.416898in}}%
\pgfpathlineto{\pgfqpoint{5.052073in}{3.382334in}}%
\pgfpathlineto{\pgfqpoint{5.056582in}{3.313017in}}%
\pgfpathlineto{\pgfqpoint{5.061091in}{3.304144in}}%
\pgfpathlineto{\pgfqpoint{5.065600in}{3.459585in}}%
\pgfpathlineto{\pgfqpoint{5.070109in}{3.376275in}}%
\pgfpathlineto{\pgfqpoint{5.074618in}{3.433408in}}%
\pgfpathlineto{\pgfqpoint{5.079127in}{3.455800in}}%
\pgfpathlineto{\pgfqpoint{5.083636in}{3.457532in}}%
\pgfpathlineto{\pgfqpoint{5.088145in}{3.464882in}}%
\pgfpathlineto{\pgfqpoint{5.092655in}{3.523251in}}%
\pgfpathlineto{\pgfqpoint{5.097164in}{3.540644in}}%
\pgfpathlineto{\pgfqpoint{5.101673in}{3.480929in}}%
\pgfpathlineto{\pgfqpoint{5.106182in}{3.387509in}}%
\pgfpathlineto{\pgfqpoint{5.110691in}{3.419591in}}%
\pgfpathlineto{\pgfqpoint{5.115200in}{3.521795in}}%
\pgfpathlineto{\pgfqpoint{5.119709in}{3.379619in}}%
\pgfpathlineto{\pgfqpoint{5.128727in}{3.536737in}}%
\pgfpathlineto{\pgfqpoint{5.133236in}{3.500120in}}%
\pgfpathlineto{\pgfqpoint{5.137745in}{3.454133in}}%
\pgfpathlineto{\pgfqpoint{5.142255in}{3.524719in}}%
\pgfpathlineto{\pgfqpoint{5.146764in}{3.507161in}}%
\pgfpathlineto{\pgfqpoint{5.151273in}{3.510240in}}%
\pgfpathlineto{\pgfqpoint{5.155782in}{3.486557in}}%
\pgfpathlineto{\pgfqpoint{5.164800in}{3.637264in}}%
\pgfpathlineto{\pgfqpoint{5.169309in}{3.553026in}}%
\pgfpathlineto{\pgfqpoint{5.173818in}{3.538933in}}%
\pgfpathlineto{\pgfqpoint{5.178327in}{3.564769in}}%
\pgfpathlineto{\pgfqpoint{5.182836in}{3.569591in}}%
\pgfpathlineto{\pgfqpoint{5.187345in}{3.541703in}}%
\pgfpathlineto{\pgfqpoint{5.191855in}{3.594422in}}%
\pgfpathlineto{\pgfqpoint{5.196364in}{3.537444in}}%
\pgfpathlineto{\pgfqpoint{5.200873in}{3.610170in}}%
\pgfpathlineto{\pgfqpoint{5.205382in}{3.755017in}}%
\pgfpathlineto{\pgfqpoint{5.209891in}{3.655815in}}%
\pgfpathlineto{\pgfqpoint{5.214400in}{3.596243in}}%
\pgfpathlineto{\pgfqpoint{5.218909in}{3.618789in}}%
\pgfpathlineto{\pgfqpoint{5.223418in}{3.610038in}}%
\pgfpathlineto{\pgfqpoint{5.227927in}{3.519621in}}%
\pgfpathlineto{\pgfqpoint{5.232436in}{3.695070in}}%
\pgfpathlineto{\pgfqpoint{5.236945in}{3.693205in}}%
\pgfpathlineto{\pgfqpoint{5.241455in}{3.589566in}}%
\pgfpathlineto{\pgfqpoint{5.245964in}{3.594808in}}%
\pgfpathlineto{\pgfqpoint{5.250473in}{3.774815in}}%
\pgfpathlineto{\pgfqpoint{5.254982in}{3.604388in}}%
\pgfpathlineto{\pgfqpoint{5.259491in}{3.584413in}}%
\pgfpathlineto{\pgfqpoint{5.264000in}{3.735638in}}%
\pgfpathlineto{\pgfqpoint{5.268509in}{3.653420in}}%
\pgfpathlineto{\pgfqpoint{5.273018in}{3.635156in}}%
\pgfpathlineto{\pgfqpoint{5.277527in}{3.748219in}}%
\pgfpathlineto{\pgfqpoint{5.282036in}{3.675911in}}%
\pgfpathlineto{\pgfqpoint{5.286545in}{3.720199in}}%
\pgfpathlineto{\pgfqpoint{5.291055in}{3.716667in}}%
\pgfpathlineto{\pgfqpoint{5.295564in}{3.699120in}}%
\pgfpathlineto{\pgfqpoint{5.300073in}{3.729943in}}%
\pgfpathlineto{\pgfqpoint{5.304582in}{3.744169in}}%
\pgfpathlineto{\pgfqpoint{5.309091in}{3.739854in}}%
\pgfpathlineto{\pgfqpoint{5.313600in}{3.771527in}}%
\pgfpathlineto{\pgfqpoint{5.318109in}{3.701802in}}%
\pgfpathlineto{\pgfqpoint{5.322618in}{3.772354in}}%
\pgfpathlineto{\pgfqpoint{5.327127in}{3.699528in}}%
\pgfpathlineto{\pgfqpoint{5.336145in}{3.845258in}}%
\pgfpathlineto{\pgfqpoint{5.340655in}{3.678516in}}%
\pgfpathlineto{\pgfqpoint{5.345164in}{3.754918in}}%
\pgfpathlineto{\pgfqpoint{5.349673in}{3.758725in}}%
\pgfpathlineto{\pgfqpoint{5.354182in}{3.870298in}}%
\pgfpathlineto{\pgfqpoint{5.358691in}{3.935443in}}%
\pgfpathlineto{\pgfqpoint{5.363200in}{3.833791in}}%
\pgfpathlineto{\pgfqpoint{5.367709in}{3.828847in}}%
\pgfpathlineto{\pgfqpoint{5.372218in}{3.844927in}}%
\pgfpathlineto{\pgfqpoint{5.376727in}{3.829289in}}%
\pgfpathlineto{\pgfqpoint{5.381236in}{3.752236in}}%
\pgfpathlineto{\pgfqpoint{5.385745in}{3.923988in}}%
\pgfpathlineto{\pgfqpoint{5.390255in}{3.695169in}}%
\pgfpathlineto{\pgfqpoint{5.394764in}{3.847973in}}%
\pgfpathlineto{\pgfqpoint{5.399273in}{3.810561in}}%
\pgfpathlineto{\pgfqpoint{5.403782in}{3.876865in}}%
\pgfpathlineto{\pgfqpoint{5.408291in}{3.848304in}}%
\pgfpathlineto{\pgfqpoint{5.412800in}{3.834034in}}%
\pgfpathlineto{\pgfqpoint{5.417309in}{3.927983in}}%
\pgfpathlineto{\pgfqpoint{5.421818in}{3.830503in}}%
\pgfpathlineto{\pgfqpoint{5.426327in}{3.929650in}}%
\pgfpathlineto{\pgfqpoint{5.430836in}{3.925666in}}%
\pgfpathlineto{\pgfqpoint{5.435345in}{3.972568in}}%
\pgfpathlineto{\pgfqpoint{5.439855in}{3.850544in}}%
\pgfpathlineto{\pgfqpoint{5.444364in}{3.694926in}}%
\pgfpathlineto{\pgfqpoint{5.448873in}{3.956754in}}%
\pgfpathlineto{\pgfqpoint{5.453382in}{3.884016in}}%
\pgfpathlineto{\pgfqpoint{5.457891in}{3.889633in}}%
\pgfpathlineto{\pgfqpoint{5.466909in}{4.033608in}}%
\pgfpathlineto{\pgfqpoint{5.475927in}{3.860178in}}%
\pgfpathlineto{\pgfqpoint{5.480436in}{3.994618in}}%
\pgfpathlineto{\pgfqpoint{5.484945in}{4.017010in}}%
\pgfpathlineto{\pgfqpoint{5.489455in}{4.029337in}}%
\pgfpathlineto{\pgfqpoint{5.493964in}{3.895306in}}%
\pgfpathlineto{\pgfqpoint{5.498473in}{3.917201in}}%
\pgfpathlineto{\pgfqpoint{5.502982in}{3.978186in}}%
\pgfpathlineto{\pgfqpoint{5.507491in}{3.980382in}}%
\pgfpathlineto{\pgfqpoint{5.512000in}{4.056000in}}%
\pgfpathlineto{\pgfqpoint{5.516509in}{3.928094in}}%
\pgfpathlineto{\pgfqpoint{5.521018in}{3.974091in}}%
\pgfpathlineto{\pgfqpoint{5.525527in}{3.965053in}}%
\pgfpathlineto{\pgfqpoint{5.530036in}{3.870011in}}%
\pgfpathlineto{\pgfqpoint{5.534545in}{4.016944in}}%
\pgfpathlineto{\pgfqpoint{5.534545in}{4.016944in}}%
\pgfusepath{stroke}%
\end{pgfscope}%
\begin{pgfscope}%
\pgfsetrectcap%
\pgfsetmiterjoin%
\pgfsetlinewidth{0.803000pt}%
\definecolor{currentstroke}{rgb}{0.000000,0.000000,0.000000}%
\pgfsetstrokecolor{currentstroke}%
\pgfsetdash{}{0pt}%
\pgfpathmoveto{\pgfqpoint{0.800000in}{0.528000in}}%
\pgfpathlineto{\pgfqpoint{0.800000in}{4.224000in}}%
\pgfusepath{stroke}%
\end{pgfscope}%
\begin{pgfscope}%
\pgfsetrectcap%
\pgfsetmiterjoin%
\pgfsetlinewidth{0.803000pt}%
\definecolor{currentstroke}{rgb}{0.000000,0.000000,0.000000}%
\pgfsetstrokecolor{currentstroke}%
\pgfsetdash{}{0pt}%
\pgfpathmoveto{\pgfqpoint{5.760000in}{0.528000in}}%
\pgfpathlineto{\pgfqpoint{5.760000in}{4.224000in}}%
\pgfusepath{stroke}%
\end{pgfscope}%
\begin{pgfscope}%
\pgfsetrectcap%
\pgfsetmiterjoin%
\pgfsetlinewidth{0.803000pt}%
\definecolor{currentstroke}{rgb}{0.000000,0.000000,0.000000}%
\pgfsetstrokecolor{currentstroke}%
\pgfsetdash{}{0pt}%
\pgfpathmoveto{\pgfqpoint{0.800000in}{0.528000in}}%
\pgfpathlineto{\pgfqpoint{5.760000in}{0.528000in}}%
\pgfusepath{stroke}%
\end{pgfscope}%
\begin{pgfscope}%
\pgfsetrectcap%
\pgfsetmiterjoin%
\pgfsetlinewidth{0.803000pt}%
\definecolor{currentstroke}{rgb}{0.000000,0.000000,0.000000}%
\pgfsetstrokecolor{currentstroke}%
\pgfsetdash{}{0pt}%
\pgfpathmoveto{\pgfqpoint{0.800000in}{4.224000in}}%
\pgfpathlineto{\pgfqpoint{5.760000in}{4.224000in}}%
\pgfusepath{stroke}%
\end{pgfscope}%
\begin{pgfscope}%
\definecolor{textcolor}{rgb}{0.000000,0.000000,0.000000}%
\pgfsetstrokecolor{textcolor}%
\pgfsetfillcolor{textcolor}%
\pgftext[x=3.280000in,y=4.307333in,,base]{\color{textcolor}\ttfamily\fontsize{12.000000}{14.400000}\selectfont Insertion Sort Iterations vs Input size}%
\end{pgfscope}%
\begin{pgfscope}%
\pgfsetbuttcap%
\pgfsetmiterjoin%
\definecolor{currentfill}{rgb}{1.000000,1.000000,1.000000}%
\pgfsetfillcolor{currentfill}%
\pgfsetfillopacity{0.800000}%
\pgfsetlinewidth{1.003750pt}%
\definecolor{currentstroke}{rgb}{0.800000,0.800000,0.800000}%
\pgfsetstrokecolor{currentstroke}%
\pgfsetstrokeopacity{0.800000}%
\pgfsetdash{}{0pt}%
\pgfpathmoveto{\pgfqpoint{0.897222in}{3.908286in}}%
\pgfpathlineto{\pgfqpoint{2.094230in}{3.908286in}}%
\pgfpathquadraticcurveto{\pgfqpoint{2.122008in}{3.908286in}}{\pgfqpoint{2.122008in}{3.936063in}}%
\pgfpathlineto{\pgfqpoint{2.122008in}{4.126778in}}%
\pgfpathquadraticcurveto{\pgfqpoint{2.122008in}{4.154556in}}{\pgfqpoint{2.094230in}{4.154556in}}%
\pgfpathlineto{\pgfqpoint{0.897222in}{4.154556in}}%
\pgfpathquadraticcurveto{\pgfqpoint{0.869444in}{4.154556in}}{\pgfqpoint{0.869444in}{4.126778in}}%
\pgfpathlineto{\pgfqpoint{0.869444in}{3.936063in}}%
\pgfpathquadraticcurveto{\pgfqpoint{0.869444in}{3.908286in}}{\pgfqpoint{0.897222in}{3.908286in}}%
\pgfpathlineto{\pgfqpoint{0.897222in}{3.908286in}}%
\pgfpathclose%
\pgfusepath{stroke,fill}%
\end{pgfscope}%
\begin{pgfscope}%
\pgfsetrectcap%
\pgfsetroundjoin%
\pgfsetlinewidth{1.505625pt}%
\definecolor{currentstroke}{rgb}{0.000000,1.000000,0.498039}%
\pgfsetstrokecolor{currentstroke}%
\pgfsetdash{}{0pt}%
\pgfpathmoveto{\pgfqpoint{0.925000in}{4.041342in}}%
\pgfpathlineto{\pgfqpoint{1.063889in}{4.041342in}}%
\pgfpathlineto{\pgfqpoint{1.202778in}{4.041342in}}%
\pgfusepath{stroke}%
\end{pgfscope}%
\begin{pgfscope}%
\definecolor{textcolor}{rgb}{0.000000,0.000000,0.000000}%
\pgfsetstrokecolor{textcolor}%
\pgfsetfillcolor{textcolor}%
\pgftext[x=1.313889in,y=3.992731in,left,base]{\color{textcolor}\ttfamily\fontsize{10.000000}{12.000000}\selectfont Insertion}%
\end{pgfscope}%
\end{pgfpicture}%
\makeatother%
\endgroup%

\subsubsection{Insights}
We can see that insertion sort is a faster way to arrange smaller
arrays but can take more time for larger ones, it has a runtime
complexity of $O(N^2)$. Its space complexity is $O(1)$. The number of
comparisons,swaps and iterations increases with the increase in
input size.
\subsection{Merge Sort}
\subsubsection*{Principle}
Merge sort is based on the divide and conquer principle. It
divides the array into two equal halves and calls itself into
each half recursively and then merges the two sorted halves. The
subarrays are divided repeatedly until they become arrays of size
1. At this point they start merging in the desired
order(ascending or descending) the one element arrays become 2
element arrays which merge again and become 4 elements and so on
until the complete sorted list is reached.
\subsubsection*{Code}
\begin{minted}{python}
def merge_sort(a):
    global t_comp
    global t_swap
    global t_itr
    if len(a) > 1:
        m = len(a) // 2
        l = a[:m]
        r = a[m:]
        merge_sort(l)
        t_itr+=1
        merge_sort(r)
        t_itr+=1
        size_l = len(l)
        size_r = len(r)
        i = j = k = 0
        while True:
            t_comp += 1
            if l[i] <= r[j]:
                a[k] = l[i]
                i += 1
                k += 1
                t_comp += 1
                if i == size_l:
                    while j < size_r:
                        a[k] = r[j]
                        j += 1
                        k += 1
                    return
            else:
                t_swap += 1
                a[k] = r[j]
                j += 1
                k += 1
                t_comp += 1
                if j == size_r:
                    while i < size_l:
                        a[k] = l[i]
                        i += 1
                        k += 1
                    return
\end{minted}
\subsubsection*{Graphs}
%% Creator: Matplotlib, PGF backend
%%
%% To include the figure in your LaTeX document, write
%%   \input{<filename>.pgf}
%%
%% Make sure the required packages are loaded in your preamble
%%   \usepackage{pgf}
%%
%% Also ensure that all the required font packages are loaded; for instance,
%% the lmodern package is sometimes necessary when using math font.
%%   \usepackage{lmodern}
%%
%% Figures using additional raster images can only be included by \input if
%% they are in the same directory as the main LaTeX file. For loading figures
%% from other directories you can use the `import` package
%%   \usepackage{import}
%%
%% and then include the figures with
%%   \import{<path to file>}{<filename>.pgf}
%%
%% Matplotlib used the following preamble
%%   \usepackage{fontspec}
%%   \setmainfont{DejaVuSerif.ttf}[Path=\detokenize{/home/dbk/.local/lib/python3.10/site-packages/matplotlib/mpl-data/fonts/ttf/}]
%%   \setsansfont{DejaVuSans.ttf}[Path=\detokenize{/home/dbk/.local/lib/python3.10/site-packages/matplotlib/mpl-data/fonts/ttf/}]
%%   \setmonofont{DejaVuSansMono.ttf}[Path=\detokenize{/home/dbk/.local/lib/python3.10/site-packages/matplotlib/mpl-data/fonts/ttf/}]
%%
\begingroup%
\makeatletter%
\begin{pgfpicture}%
\pgfpathrectangle{\pgfpointorigin}{\pgfqpoint{6.400000in}{4.800000in}}%
\pgfusepath{use as bounding box, clip}%
\begin{pgfscope}%
\pgfsetbuttcap%
\pgfsetmiterjoin%
\definecolor{currentfill}{rgb}{1.000000,1.000000,1.000000}%
\pgfsetfillcolor{currentfill}%
\pgfsetlinewidth{0.000000pt}%
\definecolor{currentstroke}{rgb}{1.000000,1.000000,1.000000}%
\pgfsetstrokecolor{currentstroke}%
\pgfsetdash{}{0pt}%
\pgfpathmoveto{\pgfqpoint{0.000000in}{0.000000in}}%
\pgfpathlineto{\pgfqpoint{6.400000in}{0.000000in}}%
\pgfpathlineto{\pgfqpoint{6.400000in}{4.800000in}}%
\pgfpathlineto{\pgfqpoint{0.000000in}{4.800000in}}%
\pgfpathlineto{\pgfqpoint{0.000000in}{0.000000in}}%
\pgfpathclose%
\pgfusepath{fill}%
\end{pgfscope}%
\begin{pgfscope}%
\pgfsetbuttcap%
\pgfsetmiterjoin%
\definecolor{currentfill}{rgb}{1.000000,1.000000,1.000000}%
\pgfsetfillcolor{currentfill}%
\pgfsetlinewidth{0.000000pt}%
\definecolor{currentstroke}{rgb}{0.000000,0.000000,0.000000}%
\pgfsetstrokecolor{currentstroke}%
\pgfsetstrokeopacity{0.000000}%
\pgfsetdash{}{0pt}%
\pgfpathmoveto{\pgfqpoint{0.800000in}{0.528000in}}%
\pgfpathlineto{\pgfqpoint{5.760000in}{0.528000in}}%
\pgfpathlineto{\pgfqpoint{5.760000in}{4.224000in}}%
\pgfpathlineto{\pgfqpoint{0.800000in}{4.224000in}}%
\pgfpathlineto{\pgfqpoint{0.800000in}{0.528000in}}%
\pgfpathclose%
\pgfusepath{fill}%
\end{pgfscope}%
\begin{pgfscope}%
\pgfsetbuttcap%
\pgfsetroundjoin%
\definecolor{currentfill}{rgb}{0.000000,0.000000,0.000000}%
\pgfsetfillcolor{currentfill}%
\pgfsetlinewidth{0.803000pt}%
\definecolor{currentstroke}{rgb}{0.000000,0.000000,0.000000}%
\pgfsetstrokecolor{currentstroke}%
\pgfsetdash{}{0pt}%
\pgfsys@defobject{currentmarker}{\pgfqpoint{0.000000in}{-0.048611in}}{\pgfqpoint{0.000000in}{0.000000in}}{%
\pgfpathmoveto{\pgfqpoint{0.000000in}{0.000000in}}%
\pgfpathlineto{\pgfqpoint{0.000000in}{-0.048611in}}%
\pgfusepath{stroke,fill}%
}%
\begin{pgfscope}%
\pgfsys@transformshift{1.020945in}{0.528000in}%
\pgfsys@useobject{currentmarker}{}%
\end{pgfscope}%
\end{pgfscope}%
\begin{pgfscope}%
\definecolor{textcolor}{rgb}{0.000000,0.000000,0.000000}%
\pgfsetstrokecolor{textcolor}%
\pgfsetfillcolor{textcolor}%
\pgftext[x=1.020945in,y=0.430778in,,top]{\color{textcolor}\ttfamily\fontsize{10.000000}{12.000000}\selectfont 0}%
\end{pgfscope}%
\begin{pgfscope}%
\pgfsetbuttcap%
\pgfsetroundjoin%
\definecolor{currentfill}{rgb}{0.000000,0.000000,0.000000}%
\pgfsetfillcolor{currentfill}%
\pgfsetlinewidth{0.803000pt}%
\definecolor{currentstroke}{rgb}{0.000000,0.000000,0.000000}%
\pgfsetstrokecolor{currentstroke}%
\pgfsetdash{}{0pt}%
\pgfsys@defobject{currentmarker}{\pgfqpoint{0.000000in}{-0.048611in}}{\pgfqpoint{0.000000in}{0.000000in}}{%
\pgfpathmoveto{\pgfqpoint{0.000000in}{0.000000in}}%
\pgfpathlineto{\pgfqpoint{0.000000in}{-0.048611in}}%
\pgfusepath{stroke,fill}%
}%
\begin{pgfscope}%
\pgfsys@transformshift{1.922764in}{0.528000in}%
\pgfsys@useobject{currentmarker}{}%
\end{pgfscope}%
\end{pgfscope}%
\begin{pgfscope}%
\definecolor{textcolor}{rgb}{0.000000,0.000000,0.000000}%
\pgfsetstrokecolor{textcolor}%
\pgfsetfillcolor{textcolor}%
\pgftext[x=1.922764in,y=0.430778in,,top]{\color{textcolor}\ttfamily\fontsize{10.000000}{12.000000}\selectfont 200}%
\end{pgfscope}%
\begin{pgfscope}%
\pgfsetbuttcap%
\pgfsetroundjoin%
\definecolor{currentfill}{rgb}{0.000000,0.000000,0.000000}%
\pgfsetfillcolor{currentfill}%
\pgfsetlinewidth{0.803000pt}%
\definecolor{currentstroke}{rgb}{0.000000,0.000000,0.000000}%
\pgfsetstrokecolor{currentstroke}%
\pgfsetdash{}{0pt}%
\pgfsys@defobject{currentmarker}{\pgfqpoint{0.000000in}{-0.048611in}}{\pgfqpoint{0.000000in}{0.000000in}}{%
\pgfpathmoveto{\pgfqpoint{0.000000in}{0.000000in}}%
\pgfpathlineto{\pgfqpoint{0.000000in}{-0.048611in}}%
\pgfusepath{stroke,fill}%
}%
\begin{pgfscope}%
\pgfsys@transformshift{2.824582in}{0.528000in}%
\pgfsys@useobject{currentmarker}{}%
\end{pgfscope}%
\end{pgfscope}%
\begin{pgfscope}%
\definecolor{textcolor}{rgb}{0.000000,0.000000,0.000000}%
\pgfsetstrokecolor{textcolor}%
\pgfsetfillcolor{textcolor}%
\pgftext[x=2.824582in,y=0.430778in,,top]{\color{textcolor}\ttfamily\fontsize{10.000000}{12.000000}\selectfont 400}%
\end{pgfscope}%
\begin{pgfscope}%
\pgfsetbuttcap%
\pgfsetroundjoin%
\definecolor{currentfill}{rgb}{0.000000,0.000000,0.000000}%
\pgfsetfillcolor{currentfill}%
\pgfsetlinewidth{0.803000pt}%
\definecolor{currentstroke}{rgb}{0.000000,0.000000,0.000000}%
\pgfsetstrokecolor{currentstroke}%
\pgfsetdash{}{0pt}%
\pgfsys@defobject{currentmarker}{\pgfqpoint{0.000000in}{-0.048611in}}{\pgfqpoint{0.000000in}{0.000000in}}{%
\pgfpathmoveto{\pgfqpoint{0.000000in}{0.000000in}}%
\pgfpathlineto{\pgfqpoint{0.000000in}{-0.048611in}}%
\pgfusepath{stroke,fill}%
}%
\begin{pgfscope}%
\pgfsys@transformshift{3.726400in}{0.528000in}%
\pgfsys@useobject{currentmarker}{}%
\end{pgfscope}%
\end{pgfscope}%
\begin{pgfscope}%
\definecolor{textcolor}{rgb}{0.000000,0.000000,0.000000}%
\pgfsetstrokecolor{textcolor}%
\pgfsetfillcolor{textcolor}%
\pgftext[x=3.726400in,y=0.430778in,,top]{\color{textcolor}\ttfamily\fontsize{10.000000}{12.000000}\selectfont 600}%
\end{pgfscope}%
\begin{pgfscope}%
\pgfsetbuttcap%
\pgfsetroundjoin%
\definecolor{currentfill}{rgb}{0.000000,0.000000,0.000000}%
\pgfsetfillcolor{currentfill}%
\pgfsetlinewidth{0.803000pt}%
\definecolor{currentstroke}{rgb}{0.000000,0.000000,0.000000}%
\pgfsetstrokecolor{currentstroke}%
\pgfsetdash{}{0pt}%
\pgfsys@defobject{currentmarker}{\pgfqpoint{0.000000in}{-0.048611in}}{\pgfqpoint{0.000000in}{0.000000in}}{%
\pgfpathmoveto{\pgfqpoint{0.000000in}{0.000000in}}%
\pgfpathlineto{\pgfqpoint{0.000000in}{-0.048611in}}%
\pgfusepath{stroke,fill}%
}%
\begin{pgfscope}%
\pgfsys@transformshift{4.628218in}{0.528000in}%
\pgfsys@useobject{currentmarker}{}%
\end{pgfscope}%
\end{pgfscope}%
\begin{pgfscope}%
\definecolor{textcolor}{rgb}{0.000000,0.000000,0.000000}%
\pgfsetstrokecolor{textcolor}%
\pgfsetfillcolor{textcolor}%
\pgftext[x=4.628218in,y=0.430778in,,top]{\color{textcolor}\ttfamily\fontsize{10.000000}{12.000000}\selectfont 800}%
\end{pgfscope}%
\begin{pgfscope}%
\pgfsetbuttcap%
\pgfsetroundjoin%
\definecolor{currentfill}{rgb}{0.000000,0.000000,0.000000}%
\pgfsetfillcolor{currentfill}%
\pgfsetlinewidth{0.803000pt}%
\definecolor{currentstroke}{rgb}{0.000000,0.000000,0.000000}%
\pgfsetstrokecolor{currentstroke}%
\pgfsetdash{}{0pt}%
\pgfsys@defobject{currentmarker}{\pgfqpoint{0.000000in}{-0.048611in}}{\pgfqpoint{0.000000in}{0.000000in}}{%
\pgfpathmoveto{\pgfqpoint{0.000000in}{0.000000in}}%
\pgfpathlineto{\pgfqpoint{0.000000in}{-0.048611in}}%
\pgfusepath{stroke,fill}%
}%
\begin{pgfscope}%
\pgfsys@transformshift{5.530036in}{0.528000in}%
\pgfsys@useobject{currentmarker}{}%
\end{pgfscope}%
\end{pgfscope}%
\begin{pgfscope}%
\definecolor{textcolor}{rgb}{0.000000,0.000000,0.000000}%
\pgfsetstrokecolor{textcolor}%
\pgfsetfillcolor{textcolor}%
\pgftext[x=5.530036in,y=0.430778in,,top]{\color{textcolor}\ttfamily\fontsize{10.000000}{12.000000}\selectfont 1000}%
\end{pgfscope}%
\begin{pgfscope}%
\definecolor{textcolor}{rgb}{0.000000,0.000000,0.000000}%
\pgfsetstrokecolor{textcolor}%
\pgfsetfillcolor{textcolor}%
\pgftext[x=3.280000in,y=0.240063in,,top]{\color{textcolor}\ttfamily\fontsize{10.000000}{12.000000}\selectfont Size of Array}%
\end{pgfscope}%
\begin{pgfscope}%
\pgfsetbuttcap%
\pgfsetroundjoin%
\definecolor{currentfill}{rgb}{0.000000,0.000000,0.000000}%
\pgfsetfillcolor{currentfill}%
\pgfsetlinewidth{0.803000pt}%
\definecolor{currentstroke}{rgb}{0.000000,0.000000,0.000000}%
\pgfsetstrokecolor{currentstroke}%
\pgfsetdash{}{0pt}%
\pgfsys@defobject{currentmarker}{\pgfqpoint{-0.048611in}{0.000000in}}{\pgfqpoint{-0.000000in}{0.000000in}}{%
\pgfpathmoveto{\pgfqpoint{-0.000000in}{0.000000in}}%
\pgfpathlineto{\pgfqpoint{-0.048611in}{0.000000in}}%
\pgfusepath{stroke,fill}%
}%
\begin{pgfscope}%
\pgfsys@transformshift{0.800000in}{0.617360in}%
\pgfsys@useobject{currentmarker}{}%
\end{pgfscope}%
\end{pgfscope}%
\begin{pgfscope}%
\definecolor{textcolor}{rgb}{0.000000,0.000000,0.000000}%
\pgfsetstrokecolor{textcolor}%
\pgfsetfillcolor{textcolor}%
\pgftext[x=0.451923in, y=0.564226in, left, base]{\color{textcolor}\ttfamily\fontsize{10.000000}{12.000000}\selectfont 0.0}%
\end{pgfscope}%
\begin{pgfscope}%
\pgfsetbuttcap%
\pgfsetroundjoin%
\definecolor{currentfill}{rgb}{0.000000,0.000000,0.000000}%
\pgfsetfillcolor{currentfill}%
\pgfsetlinewidth{0.803000pt}%
\definecolor{currentstroke}{rgb}{0.000000,0.000000,0.000000}%
\pgfsetstrokecolor{currentstroke}%
\pgfsetdash{}{0pt}%
\pgfsys@defobject{currentmarker}{\pgfqpoint{-0.048611in}{0.000000in}}{\pgfqpoint{-0.000000in}{0.000000in}}{%
\pgfpathmoveto{\pgfqpoint{-0.000000in}{0.000000in}}%
\pgfpathlineto{\pgfqpoint{-0.048611in}{0.000000in}}%
\pgfusepath{stroke,fill}%
}%
\begin{pgfscope}%
\pgfsys@transformshift{0.800000in}{1.165095in}%
\pgfsys@useobject{currentmarker}{}%
\end{pgfscope}%
\end{pgfscope}%
\begin{pgfscope}%
\definecolor{textcolor}{rgb}{0.000000,0.000000,0.000000}%
\pgfsetstrokecolor{textcolor}%
\pgfsetfillcolor{textcolor}%
\pgftext[x=0.451923in, y=1.111960in, left, base]{\color{textcolor}\ttfamily\fontsize{10.000000}{12.000000}\selectfont 0.5}%
\end{pgfscope}%
\begin{pgfscope}%
\pgfsetbuttcap%
\pgfsetroundjoin%
\definecolor{currentfill}{rgb}{0.000000,0.000000,0.000000}%
\pgfsetfillcolor{currentfill}%
\pgfsetlinewidth{0.803000pt}%
\definecolor{currentstroke}{rgb}{0.000000,0.000000,0.000000}%
\pgfsetstrokecolor{currentstroke}%
\pgfsetdash{}{0pt}%
\pgfsys@defobject{currentmarker}{\pgfqpoint{-0.048611in}{0.000000in}}{\pgfqpoint{-0.000000in}{0.000000in}}{%
\pgfpathmoveto{\pgfqpoint{-0.000000in}{0.000000in}}%
\pgfpathlineto{\pgfqpoint{-0.048611in}{0.000000in}}%
\pgfusepath{stroke,fill}%
}%
\begin{pgfscope}%
\pgfsys@transformshift{0.800000in}{1.712830in}%
\pgfsys@useobject{currentmarker}{}%
\end{pgfscope}%
\end{pgfscope}%
\begin{pgfscope}%
\definecolor{textcolor}{rgb}{0.000000,0.000000,0.000000}%
\pgfsetstrokecolor{textcolor}%
\pgfsetfillcolor{textcolor}%
\pgftext[x=0.451923in, y=1.659695in, left, base]{\color{textcolor}\ttfamily\fontsize{10.000000}{12.000000}\selectfont 1.0}%
\end{pgfscope}%
\begin{pgfscope}%
\pgfsetbuttcap%
\pgfsetroundjoin%
\definecolor{currentfill}{rgb}{0.000000,0.000000,0.000000}%
\pgfsetfillcolor{currentfill}%
\pgfsetlinewidth{0.803000pt}%
\definecolor{currentstroke}{rgb}{0.000000,0.000000,0.000000}%
\pgfsetstrokecolor{currentstroke}%
\pgfsetdash{}{0pt}%
\pgfsys@defobject{currentmarker}{\pgfqpoint{-0.048611in}{0.000000in}}{\pgfqpoint{-0.000000in}{0.000000in}}{%
\pgfpathmoveto{\pgfqpoint{-0.000000in}{0.000000in}}%
\pgfpathlineto{\pgfqpoint{-0.048611in}{0.000000in}}%
\pgfusepath{stroke,fill}%
}%
\begin{pgfscope}%
\pgfsys@transformshift{0.800000in}{2.260564in}%
\pgfsys@useobject{currentmarker}{}%
\end{pgfscope}%
\end{pgfscope}%
\begin{pgfscope}%
\definecolor{textcolor}{rgb}{0.000000,0.000000,0.000000}%
\pgfsetstrokecolor{textcolor}%
\pgfsetfillcolor{textcolor}%
\pgftext[x=0.451923in, y=2.207430in, left, base]{\color{textcolor}\ttfamily\fontsize{10.000000}{12.000000}\selectfont 1.5}%
\end{pgfscope}%
\begin{pgfscope}%
\pgfsetbuttcap%
\pgfsetroundjoin%
\definecolor{currentfill}{rgb}{0.000000,0.000000,0.000000}%
\pgfsetfillcolor{currentfill}%
\pgfsetlinewidth{0.803000pt}%
\definecolor{currentstroke}{rgb}{0.000000,0.000000,0.000000}%
\pgfsetstrokecolor{currentstroke}%
\pgfsetdash{}{0pt}%
\pgfsys@defobject{currentmarker}{\pgfqpoint{-0.048611in}{0.000000in}}{\pgfqpoint{-0.000000in}{0.000000in}}{%
\pgfpathmoveto{\pgfqpoint{-0.000000in}{0.000000in}}%
\pgfpathlineto{\pgfqpoint{-0.048611in}{0.000000in}}%
\pgfusepath{stroke,fill}%
}%
\begin{pgfscope}%
\pgfsys@transformshift{0.800000in}{2.808299in}%
\pgfsys@useobject{currentmarker}{}%
\end{pgfscope}%
\end{pgfscope}%
\begin{pgfscope}%
\definecolor{textcolor}{rgb}{0.000000,0.000000,0.000000}%
\pgfsetstrokecolor{textcolor}%
\pgfsetfillcolor{textcolor}%
\pgftext[x=0.451923in, y=2.755164in, left, base]{\color{textcolor}\ttfamily\fontsize{10.000000}{12.000000}\selectfont 2.0}%
\end{pgfscope}%
\begin{pgfscope}%
\pgfsetbuttcap%
\pgfsetroundjoin%
\definecolor{currentfill}{rgb}{0.000000,0.000000,0.000000}%
\pgfsetfillcolor{currentfill}%
\pgfsetlinewidth{0.803000pt}%
\definecolor{currentstroke}{rgb}{0.000000,0.000000,0.000000}%
\pgfsetstrokecolor{currentstroke}%
\pgfsetdash{}{0pt}%
\pgfsys@defobject{currentmarker}{\pgfqpoint{-0.048611in}{0.000000in}}{\pgfqpoint{-0.000000in}{0.000000in}}{%
\pgfpathmoveto{\pgfqpoint{-0.000000in}{0.000000in}}%
\pgfpathlineto{\pgfqpoint{-0.048611in}{0.000000in}}%
\pgfusepath{stroke,fill}%
}%
\begin{pgfscope}%
\pgfsys@transformshift{0.800000in}{3.356034in}%
\pgfsys@useobject{currentmarker}{}%
\end{pgfscope}%
\end{pgfscope}%
\begin{pgfscope}%
\definecolor{textcolor}{rgb}{0.000000,0.000000,0.000000}%
\pgfsetstrokecolor{textcolor}%
\pgfsetfillcolor{textcolor}%
\pgftext[x=0.451923in, y=3.302899in, left, base]{\color{textcolor}\ttfamily\fontsize{10.000000}{12.000000}\selectfont 2.5}%
\end{pgfscope}%
\begin{pgfscope}%
\pgfsetbuttcap%
\pgfsetroundjoin%
\definecolor{currentfill}{rgb}{0.000000,0.000000,0.000000}%
\pgfsetfillcolor{currentfill}%
\pgfsetlinewidth{0.803000pt}%
\definecolor{currentstroke}{rgb}{0.000000,0.000000,0.000000}%
\pgfsetstrokecolor{currentstroke}%
\pgfsetdash{}{0pt}%
\pgfsys@defobject{currentmarker}{\pgfqpoint{-0.048611in}{0.000000in}}{\pgfqpoint{-0.000000in}{0.000000in}}{%
\pgfpathmoveto{\pgfqpoint{-0.000000in}{0.000000in}}%
\pgfpathlineto{\pgfqpoint{-0.048611in}{0.000000in}}%
\pgfusepath{stroke,fill}%
}%
\begin{pgfscope}%
\pgfsys@transformshift{0.800000in}{3.903768in}%
\pgfsys@useobject{currentmarker}{}%
\end{pgfscope}%
\end{pgfscope}%
\begin{pgfscope}%
\definecolor{textcolor}{rgb}{0.000000,0.000000,0.000000}%
\pgfsetstrokecolor{textcolor}%
\pgfsetfillcolor{textcolor}%
\pgftext[x=0.451923in, y=3.850634in, left, base]{\color{textcolor}\ttfamily\fontsize{10.000000}{12.000000}\selectfont 3.0}%
\end{pgfscope}%
\begin{pgfscope}%
\definecolor{textcolor}{rgb}{0.000000,0.000000,0.000000}%
\pgfsetstrokecolor{textcolor}%
\pgfsetfillcolor{textcolor}%
\pgftext[x=0.396368in,y=2.376000in,,bottom,rotate=90.000000]{\color{textcolor}\ttfamily\fontsize{10.000000}{12.000000}\selectfont Time}%
\end{pgfscope}%
\begin{pgfscope}%
\definecolor{textcolor}{rgb}{0.000000,0.000000,0.000000}%
\pgfsetstrokecolor{textcolor}%
\pgfsetfillcolor{textcolor}%
\pgftext[x=0.800000in,y=4.265667in,left,base]{\color{textcolor}\ttfamily\fontsize{10.000000}{12.000000}\selectfont 1e7}%
\end{pgfscope}%
\begin{pgfscope}%
\pgfpathrectangle{\pgfqpoint{0.800000in}{0.528000in}}{\pgfqpoint{4.960000in}{3.696000in}}%
\pgfusepath{clip}%
\pgfsetrectcap%
\pgfsetroundjoin%
\pgfsetlinewidth{1.505625pt}%
\definecolor{currentstroke}{rgb}{0.000000,1.000000,0.498039}%
\pgfsetstrokecolor{currentstroke}%
\pgfsetdash{}{0pt}%
\pgfpathmoveto{\pgfqpoint{1.025455in}{0.696000in}}%
\pgfpathlineto{\pgfqpoint{1.029964in}{0.699155in}}%
\pgfpathlineto{\pgfqpoint{1.034473in}{0.698289in}}%
\pgfpathlineto{\pgfqpoint{1.038982in}{0.700196in}}%
\pgfpathlineto{\pgfqpoint{1.043491in}{0.778012in}}%
\pgfpathlineto{\pgfqpoint{1.048000in}{0.761392in}}%
\pgfpathlineto{\pgfqpoint{1.052509in}{0.702171in}}%
\pgfpathlineto{\pgfqpoint{1.057018in}{0.701635in}}%
\pgfpathlineto{\pgfqpoint{1.061527in}{0.709670in}}%
\pgfpathlineto{\pgfqpoint{1.066036in}{0.740591in}}%
\pgfpathlineto{\pgfqpoint{1.070545in}{0.705517in}}%
\pgfpathlineto{\pgfqpoint{1.075055in}{0.706607in}}%
\pgfpathlineto{\pgfqpoint{1.079564in}{0.765559in}}%
\pgfpathlineto{\pgfqpoint{1.084073in}{0.709484in}}%
\pgfpathlineto{\pgfqpoint{1.088582in}{0.799311in}}%
\pgfpathlineto{\pgfqpoint{1.093091in}{0.709928in}}%
\pgfpathlineto{\pgfqpoint{1.097600in}{0.773341in}}%
\pgfpathlineto{\pgfqpoint{1.102109in}{0.711244in}}%
\pgfpathlineto{\pgfqpoint{1.111127in}{0.716021in}}%
\pgfpathlineto{\pgfqpoint{1.115636in}{0.715287in}}%
\pgfpathlineto{\pgfqpoint{1.120145in}{0.716362in}}%
\pgfpathlineto{\pgfqpoint{1.124655in}{0.718695in}}%
\pgfpathlineto{\pgfqpoint{1.133673in}{0.719151in}}%
\pgfpathlineto{\pgfqpoint{1.138182in}{0.722454in}}%
\pgfpathlineto{\pgfqpoint{1.142691in}{0.721835in}}%
\pgfpathlineto{\pgfqpoint{1.147200in}{0.724591in}}%
\pgfpathlineto{\pgfqpoint{1.151709in}{0.723292in}}%
\pgfpathlineto{\pgfqpoint{1.160727in}{0.724636in}}%
\pgfpathlineto{\pgfqpoint{1.165236in}{0.729447in}}%
\pgfpathlineto{\pgfqpoint{1.169745in}{0.729564in}}%
\pgfpathlineto{\pgfqpoint{1.174255in}{0.732644in}}%
\pgfpathlineto{\pgfqpoint{1.178764in}{0.733254in}}%
\pgfpathlineto{\pgfqpoint{1.183273in}{0.842822in}}%
\pgfpathlineto{\pgfqpoint{1.187782in}{0.732195in}}%
\pgfpathlineto{\pgfqpoint{1.192291in}{0.732013in}}%
\pgfpathlineto{\pgfqpoint{1.201309in}{0.742018in}}%
\pgfpathlineto{\pgfqpoint{1.205818in}{0.769646in}}%
\pgfpathlineto{\pgfqpoint{1.210327in}{0.744266in}}%
\pgfpathlineto{\pgfqpoint{1.214836in}{0.736348in}}%
\pgfpathlineto{\pgfqpoint{1.219345in}{0.739113in}}%
\pgfpathlineto{\pgfqpoint{1.223855in}{0.861680in}}%
\pgfpathlineto{\pgfqpoint{1.228364in}{0.741671in}}%
\pgfpathlineto{\pgfqpoint{1.232873in}{0.743893in}}%
\pgfpathlineto{\pgfqpoint{1.237382in}{0.752890in}}%
\pgfpathlineto{\pgfqpoint{1.241891in}{0.890467in}}%
\pgfpathlineto{\pgfqpoint{1.246400in}{0.771824in}}%
\pgfpathlineto{\pgfqpoint{1.250909in}{0.746921in}}%
\pgfpathlineto{\pgfqpoint{1.255418in}{0.749828in}}%
\pgfpathlineto{\pgfqpoint{1.259927in}{0.766851in}}%
\pgfpathlineto{\pgfqpoint{1.264436in}{0.770891in}}%
\pgfpathlineto{\pgfqpoint{1.268945in}{0.752966in}}%
\pgfpathlineto{\pgfqpoint{1.273455in}{0.754461in}}%
\pgfpathlineto{\pgfqpoint{1.277964in}{0.773435in}}%
\pgfpathlineto{\pgfqpoint{1.282473in}{0.763479in}}%
\pgfpathlineto{\pgfqpoint{1.286982in}{0.777157in}}%
\pgfpathlineto{\pgfqpoint{1.291491in}{0.759430in}}%
\pgfpathlineto{\pgfqpoint{1.296000in}{0.778520in}}%
\pgfpathlineto{\pgfqpoint{1.300509in}{0.760846in}}%
\pgfpathlineto{\pgfqpoint{1.305018in}{0.759797in}}%
\pgfpathlineto{\pgfqpoint{1.309527in}{0.762239in}}%
\pgfpathlineto{\pgfqpoint{1.323055in}{0.763839in}}%
\pgfpathlineto{\pgfqpoint{1.327564in}{0.767603in}}%
\pgfpathlineto{\pgfqpoint{1.332073in}{0.767545in}}%
\pgfpathlineto{\pgfqpoint{1.336582in}{0.789274in}}%
\pgfpathlineto{\pgfqpoint{1.341091in}{0.799130in}}%
\pgfpathlineto{\pgfqpoint{1.345600in}{0.917220in}}%
\pgfpathlineto{\pgfqpoint{1.350109in}{0.803570in}}%
\pgfpathlineto{\pgfqpoint{1.354618in}{0.978466in}}%
\pgfpathlineto{\pgfqpoint{1.359127in}{0.962129in}}%
\pgfpathlineto{\pgfqpoint{1.363636in}{1.024504in}}%
\pgfpathlineto{\pgfqpoint{1.368145in}{1.044294in}}%
\pgfpathlineto{\pgfqpoint{1.372655in}{0.873887in}}%
\pgfpathlineto{\pgfqpoint{1.377164in}{0.862719in}}%
\pgfpathlineto{\pgfqpoint{1.386182in}{1.031649in}}%
\pgfpathlineto{\pgfqpoint{1.390691in}{1.110926in}}%
\pgfpathlineto{\pgfqpoint{1.395200in}{0.983786in}}%
\pgfpathlineto{\pgfqpoint{1.399709in}{1.029038in}}%
\pgfpathlineto{\pgfqpoint{1.404218in}{0.836118in}}%
\pgfpathlineto{\pgfqpoint{1.408727in}{0.877689in}}%
\pgfpathlineto{\pgfqpoint{1.413236in}{0.791539in}}%
\pgfpathlineto{\pgfqpoint{1.417745in}{0.815892in}}%
\pgfpathlineto{\pgfqpoint{1.422255in}{0.788226in}}%
\pgfpathlineto{\pgfqpoint{1.426764in}{0.789422in}}%
\pgfpathlineto{\pgfqpoint{1.431273in}{0.789296in}}%
\pgfpathlineto{\pgfqpoint{1.435782in}{0.792098in}}%
\pgfpathlineto{\pgfqpoint{1.440291in}{0.791999in}}%
\pgfpathlineto{\pgfqpoint{1.444800in}{0.794869in}}%
\pgfpathlineto{\pgfqpoint{1.449309in}{0.793993in}}%
\pgfpathlineto{\pgfqpoint{1.453818in}{0.796345in}}%
\pgfpathlineto{\pgfqpoint{1.476364in}{0.801651in}}%
\pgfpathlineto{\pgfqpoint{1.480873in}{0.836117in}}%
\pgfpathlineto{\pgfqpoint{1.485382in}{0.802264in}}%
\pgfpathlineto{\pgfqpoint{1.489891in}{0.802140in}}%
\pgfpathlineto{\pgfqpoint{1.494400in}{0.805242in}}%
\pgfpathlineto{\pgfqpoint{1.498909in}{0.806764in}}%
\pgfpathlineto{\pgfqpoint{1.503418in}{0.822463in}}%
\pgfpathlineto{\pgfqpoint{1.507927in}{1.085022in}}%
\pgfpathlineto{\pgfqpoint{1.512436in}{0.856414in}}%
\pgfpathlineto{\pgfqpoint{1.516945in}{0.814072in}}%
\pgfpathlineto{\pgfqpoint{1.521455in}{0.814227in}}%
\pgfpathlineto{\pgfqpoint{1.525964in}{0.816933in}}%
\pgfpathlineto{\pgfqpoint{1.530473in}{0.811611in}}%
\pgfpathlineto{\pgfqpoint{1.534982in}{0.815514in}}%
\pgfpathlineto{\pgfqpoint{1.539491in}{0.835083in}}%
\pgfpathlineto{\pgfqpoint{1.544000in}{0.841405in}}%
\pgfpathlineto{\pgfqpoint{1.548509in}{0.820513in}}%
\pgfpathlineto{\pgfqpoint{1.562036in}{0.821804in}}%
\pgfpathlineto{\pgfqpoint{1.566545in}{0.825470in}}%
\pgfpathlineto{\pgfqpoint{1.571055in}{0.823522in}}%
\pgfpathlineto{\pgfqpoint{1.575564in}{0.964584in}}%
\pgfpathlineto{\pgfqpoint{1.580073in}{0.964288in}}%
\pgfpathlineto{\pgfqpoint{1.584582in}{1.052488in}}%
\pgfpathlineto{\pgfqpoint{1.589091in}{1.031635in}}%
\pgfpathlineto{\pgfqpoint{1.593600in}{0.913625in}}%
\pgfpathlineto{\pgfqpoint{1.598109in}{1.024598in}}%
\pgfpathlineto{\pgfqpoint{1.602618in}{1.033565in}}%
\pgfpathlineto{\pgfqpoint{1.607127in}{1.032644in}}%
\pgfpathlineto{\pgfqpoint{1.611636in}{1.064348in}}%
\pgfpathlineto{\pgfqpoint{1.616145in}{1.190549in}}%
\pgfpathlineto{\pgfqpoint{1.620655in}{1.118378in}}%
\pgfpathlineto{\pgfqpoint{1.625164in}{1.239245in}}%
\pgfpathlineto{\pgfqpoint{1.629673in}{0.863085in}}%
\pgfpathlineto{\pgfqpoint{1.634182in}{0.837797in}}%
\pgfpathlineto{\pgfqpoint{1.638691in}{0.837791in}}%
\pgfpathlineto{\pgfqpoint{1.643200in}{0.841827in}}%
\pgfpathlineto{\pgfqpoint{1.647709in}{0.838772in}}%
\pgfpathlineto{\pgfqpoint{1.652218in}{0.844463in}}%
\pgfpathlineto{\pgfqpoint{1.656727in}{0.844049in}}%
\pgfpathlineto{\pgfqpoint{1.661236in}{0.862221in}}%
\pgfpathlineto{\pgfqpoint{1.665745in}{1.012964in}}%
\pgfpathlineto{\pgfqpoint{1.670255in}{1.048583in}}%
\pgfpathlineto{\pgfqpoint{1.674764in}{0.926732in}}%
\pgfpathlineto{\pgfqpoint{1.679273in}{1.147814in}}%
\pgfpathlineto{\pgfqpoint{1.683782in}{0.853481in}}%
\pgfpathlineto{\pgfqpoint{1.688291in}{0.881180in}}%
\pgfpathlineto{\pgfqpoint{1.692800in}{0.882376in}}%
\pgfpathlineto{\pgfqpoint{1.697309in}{0.897599in}}%
\pgfpathlineto{\pgfqpoint{1.701818in}{1.020697in}}%
\pgfpathlineto{\pgfqpoint{1.706327in}{0.871736in}}%
\pgfpathlineto{\pgfqpoint{1.710836in}{1.138147in}}%
\pgfpathlineto{\pgfqpoint{1.715345in}{1.082433in}}%
\pgfpathlineto{\pgfqpoint{1.719855in}{0.861610in}}%
\pgfpathlineto{\pgfqpoint{1.728873in}{0.864143in}}%
\pgfpathlineto{\pgfqpoint{1.733382in}{0.868919in}}%
\pgfpathlineto{\pgfqpoint{1.737891in}{0.869375in}}%
\pgfpathlineto{\pgfqpoint{1.742400in}{0.886140in}}%
\pgfpathlineto{\pgfqpoint{1.746909in}{0.868026in}}%
\pgfpathlineto{\pgfqpoint{1.751418in}{1.111601in}}%
\pgfpathlineto{\pgfqpoint{1.755927in}{0.971154in}}%
\pgfpathlineto{\pgfqpoint{1.760436in}{0.890032in}}%
\pgfpathlineto{\pgfqpoint{1.764945in}{1.001565in}}%
\pgfpathlineto{\pgfqpoint{1.769455in}{1.029858in}}%
\pgfpathlineto{\pgfqpoint{1.773964in}{0.920505in}}%
\pgfpathlineto{\pgfqpoint{1.778473in}{1.036553in}}%
\pgfpathlineto{\pgfqpoint{1.782982in}{1.283000in}}%
\pgfpathlineto{\pgfqpoint{1.787491in}{0.996836in}}%
\pgfpathlineto{\pgfqpoint{1.792000in}{0.953237in}}%
\pgfpathlineto{\pgfqpoint{1.796509in}{1.146985in}}%
\pgfpathlineto{\pgfqpoint{1.801018in}{0.996845in}}%
\pgfpathlineto{\pgfqpoint{1.805527in}{1.126653in}}%
\pgfpathlineto{\pgfqpoint{1.810036in}{0.881474in}}%
\pgfpathlineto{\pgfqpoint{1.814545in}{0.887252in}}%
\pgfpathlineto{\pgfqpoint{1.819055in}{1.269952in}}%
\pgfpathlineto{\pgfqpoint{1.828073in}{0.896756in}}%
\pgfpathlineto{\pgfqpoint{1.832582in}{0.949633in}}%
\pgfpathlineto{\pgfqpoint{1.837091in}{0.946366in}}%
\pgfpathlineto{\pgfqpoint{1.841600in}{0.892915in}}%
\pgfpathlineto{\pgfqpoint{1.846109in}{1.014198in}}%
\pgfpathlineto{\pgfqpoint{1.850618in}{1.186151in}}%
\pgfpathlineto{\pgfqpoint{1.855127in}{1.061847in}}%
\pgfpathlineto{\pgfqpoint{1.859636in}{0.896463in}}%
\pgfpathlineto{\pgfqpoint{1.864145in}{0.900629in}}%
\pgfpathlineto{\pgfqpoint{1.868655in}{0.902816in}}%
\pgfpathlineto{\pgfqpoint{1.873164in}{0.901973in}}%
\pgfpathlineto{\pgfqpoint{1.877673in}{0.903508in}}%
\pgfpathlineto{\pgfqpoint{1.882182in}{1.195655in}}%
\pgfpathlineto{\pgfqpoint{1.886691in}{0.910370in}}%
\pgfpathlineto{\pgfqpoint{1.891200in}{0.904605in}}%
\pgfpathlineto{\pgfqpoint{1.895709in}{0.906752in}}%
\pgfpathlineto{\pgfqpoint{1.900218in}{0.925734in}}%
\pgfpathlineto{\pgfqpoint{1.904727in}{0.983002in}}%
\pgfpathlineto{\pgfqpoint{1.909236in}{0.909325in}}%
\pgfpathlineto{\pgfqpoint{1.913745in}{0.908547in}}%
\pgfpathlineto{\pgfqpoint{1.918255in}{0.913161in}}%
\pgfpathlineto{\pgfqpoint{1.922764in}{0.914399in}}%
\pgfpathlineto{\pgfqpoint{1.931782in}{0.926682in}}%
\pgfpathlineto{\pgfqpoint{1.936291in}{0.922035in}}%
\pgfpathlineto{\pgfqpoint{1.940800in}{0.922465in}}%
\pgfpathlineto{\pgfqpoint{1.949818in}{0.918696in}}%
\pgfpathlineto{\pgfqpoint{1.954327in}{1.019295in}}%
\pgfpathlineto{\pgfqpoint{1.958836in}{1.023595in}}%
\pgfpathlineto{\pgfqpoint{1.963345in}{0.945748in}}%
\pgfpathlineto{\pgfqpoint{1.967855in}{0.924662in}}%
\pgfpathlineto{\pgfqpoint{1.972364in}{0.923786in}}%
\pgfpathlineto{\pgfqpoint{1.976873in}{1.065304in}}%
\pgfpathlineto{\pgfqpoint{1.981382in}{0.929769in}}%
\pgfpathlineto{\pgfqpoint{1.985891in}{1.009305in}}%
\pgfpathlineto{\pgfqpoint{1.990400in}{0.936729in}}%
\pgfpathlineto{\pgfqpoint{1.994909in}{0.931688in}}%
\pgfpathlineto{\pgfqpoint{1.999418in}{0.935013in}}%
\pgfpathlineto{\pgfqpoint{2.003927in}{1.002544in}}%
\pgfpathlineto{\pgfqpoint{2.008436in}{0.945816in}}%
\pgfpathlineto{\pgfqpoint{2.012945in}{0.938421in}}%
\pgfpathlineto{\pgfqpoint{2.017455in}{1.049969in}}%
\pgfpathlineto{\pgfqpoint{2.021964in}{0.949302in}}%
\pgfpathlineto{\pgfqpoint{2.026473in}{1.206281in}}%
\pgfpathlineto{\pgfqpoint{2.030982in}{0.985862in}}%
\pgfpathlineto{\pgfqpoint{2.035491in}{1.113880in}}%
\pgfpathlineto{\pgfqpoint{2.040000in}{1.087006in}}%
\pgfpathlineto{\pgfqpoint{2.044509in}{1.173420in}}%
\pgfpathlineto{\pgfqpoint{2.049018in}{0.945299in}}%
\pgfpathlineto{\pgfqpoint{2.053527in}{1.040427in}}%
\pgfpathlineto{\pgfqpoint{2.058036in}{1.023749in}}%
\pgfpathlineto{\pgfqpoint{2.062545in}{0.995696in}}%
\pgfpathlineto{\pgfqpoint{2.067055in}{0.995437in}}%
\pgfpathlineto{\pgfqpoint{2.071564in}{0.954697in}}%
\pgfpathlineto{\pgfqpoint{2.076073in}{0.991775in}}%
\pgfpathlineto{\pgfqpoint{2.080582in}{0.986066in}}%
\pgfpathlineto{\pgfqpoint{2.085091in}{0.976013in}}%
\pgfpathlineto{\pgfqpoint{2.089600in}{0.956711in}}%
\pgfpathlineto{\pgfqpoint{2.094109in}{1.052092in}}%
\pgfpathlineto{\pgfqpoint{2.098618in}{0.964757in}}%
\pgfpathlineto{\pgfqpoint{2.107636in}{1.074691in}}%
\pgfpathlineto{\pgfqpoint{2.112145in}{1.009309in}}%
\pgfpathlineto{\pgfqpoint{2.116655in}{1.030070in}}%
\pgfpathlineto{\pgfqpoint{2.121164in}{1.364421in}}%
\pgfpathlineto{\pgfqpoint{2.125673in}{1.238078in}}%
\pgfpathlineto{\pgfqpoint{2.130182in}{1.209514in}}%
\pgfpathlineto{\pgfqpoint{2.134691in}{1.406541in}}%
\pgfpathlineto{\pgfqpoint{2.139200in}{1.664740in}}%
\pgfpathlineto{\pgfqpoint{2.143709in}{1.011112in}}%
\pgfpathlineto{\pgfqpoint{2.148218in}{0.977766in}}%
\pgfpathlineto{\pgfqpoint{2.152727in}{0.972668in}}%
\pgfpathlineto{\pgfqpoint{2.157236in}{1.084473in}}%
\pgfpathlineto{\pgfqpoint{2.161745in}{1.022219in}}%
\pgfpathlineto{\pgfqpoint{2.166255in}{1.589301in}}%
\pgfpathlineto{\pgfqpoint{2.170764in}{1.132813in}}%
\pgfpathlineto{\pgfqpoint{2.175273in}{1.604968in}}%
\pgfpathlineto{\pgfqpoint{2.179782in}{1.460015in}}%
\pgfpathlineto{\pgfqpoint{2.184291in}{1.035098in}}%
\pgfpathlineto{\pgfqpoint{2.188800in}{1.872102in}}%
\pgfpathlineto{\pgfqpoint{2.193309in}{1.339660in}}%
\pgfpathlineto{\pgfqpoint{2.197818in}{1.603608in}}%
\pgfpathlineto{\pgfqpoint{2.202327in}{1.466384in}}%
\pgfpathlineto{\pgfqpoint{2.206836in}{1.114212in}}%
\pgfpathlineto{\pgfqpoint{2.211345in}{1.252453in}}%
\pgfpathlineto{\pgfqpoint{2.215855in}{1.240401in}}%
\pgfpathlineto{\pgfqpoint{2.220364in}{1.201471in}}%
\pgfpathlineto{\pgfqpoint{2.224873in}{1.077496in}}%
\pgfpathlineto{\pgfqpoint{2.229382in}{1.327788in}}%
\pgfpathlineto{\pgfqpoint{2.233891in}{1.060730in}}%
\pgfpathlineto{\pgfqpoint{2.238400in}{1.004552in}}%
\pgfpathlineto{\pgfqpoint{2.242909in}{1.021974in}}%
\pgfpathlineto{\pgfqpoint{2.247418in}{1.006044in}}%
\pgfpathlineto{\pgfqpoint{2.251927in}{1.002405in}}%
\pgfpathlineto{\pgfqpoint{2.256436in}{1.015640in}}%
\pgfpathlineto{\pgfqpoint{2.260945in}{1.054980in}}%
\pgfpathlineto{\pgfqpoint{2.265455in}{1.184576in}}%
\pgfpathlineto{\pgfqpoint{2.269964in}{1.246060in}}%
\pgfpathlineto{\pgfqpoint{2.274473in}{1.006453in}}%
\pgfpathlineto{\pgfqpoint{2.278982in}{1.010049in}}%
\pgfpathlineto{\pgfqpoint{2.283491in}{1.041049in}}%
\pgfpathlineto{\pgfqpoint{2.288000in}{1.088906in}}%
\pgfpathlineto{\pgfqpoint{2.292509in}{1.095153in}}%
\pgfpathlineto{\pgfqpoint{2.297018in}{1.018040in}}%
\pgfpathlineto{\pgfqpoint{2.301527in}{1.226095in}}%
\pgfpathlineto{\pgfqpoint{2.306036in}{1.023314in}}%
\pgfpathlineto{\pgfqpoint{2.315055in}{1.092713in}}%
\pgfpathlineto{\pgfqpoint{2.319564in}{1.214850in}}%
\pgfpathlineto{\pgfqpoint{2.324073in}{1.039846in}}%
\pgfpathlineto{\pgfqpoint{2.328582in}{1.034840in}}%
\pgfpathlineto{\pgfqpoint{2.333091in}{1.081111in}}%
\pgfpathlineto{\pgfqpoint{2.337600in}{1.053783in}}%
\pgfpathlineto{\pgfqpoint{2.342109in}{1.072516in}}%
\pgfpathlineto{\pgfqpoint{2.346618in}{1.032411in}}%
\pgfpathlineto{\pgfqpoint{2.351127in}{1.030641in}}%
\pgfpathlineto{\pgfqpoint{2.355636in}{1.122969in}}%
\pgfpathlineto{\pgfqpoint{2.360145in}{1.039694in}}%
\pgfpathlineto{\pgfqpoint{2.364655in}{1.068455in}}%
\pgfpathlineto{\pgfqpoint{2.369164in}{1.212095in}}%
\pgfpathlineto{\pgfqpoint{2.373673in}{1.042896in}}%
\pgfpathlineto{\pgfqpoint{2.378182in}{1.156663in}}%
\pgfpathlineto{\pgfqpoint{2.382691in}{1.069493in}}%
\pgfpathlineto{\pgfqpoint{2.387200in}{1.563057in}}%
\pgfpathlineto{\pgfqpoint{2.391709in}{1.155150in}}%
\pgfpathlineto{\pgfqpoint{2.396218in}{1.050610in}}%
\pgfpathlineto{\pgfqpoint{2.400727in}{1.079297in}}%
\pgfpathlineto{\pgfqpoint{2.405236in}{1.048266in}}%
\pgfpathlineto{\pgfqpoint{2.409745in}{1.156089in}}%
\pgfpathlineto{\pgfqpoint{2.414255in}{1.086924in}}%
\pgfpathlineto{\pgfqpoint{2.418764in}{1.086332in}}%
\pgfpathlineto{\pgfqpoint{2.423273in}{1.114746in}}%
\pgfpathlineto{\pgfqpoint{2.427782in}{1.109216in}}%
\pgfpathlineto{\pgfqpoint{2.432291in}{1.095252in}}%
\pgfpathlineto{\pgfqpoint{2.436800in}{1.236941in}}%
\pgfpathlineto{\pgfqpoint{2.441309in}{1.066410in}}%
\pgfpathlineto{\pgfqpoint{2.445818in}{1.165173in}}%
\pgfpathlineto{\pgfqpoint{2.450327in}{1.592345in}}%
\pgfpathlineto{\pgfqpoint{2.454836in}{1.085965in}}%
\pgfpathlineto{\pgfqpoint{2.459345in}{1.340384in}}%
\pgfpathlineto{\pgfqpoint{2.463855in}{1.077300in}}%
\pgfpathlineto{\pgfqpoint{2.468364in}{1.083552in}}%
\pgfpathlineto{\pgfqpoint{2.472873in}{1.230486in}}%
\pgfpathlineto{\pgfqpoint{2.477382in}{1.109146in}}%
\pgfpathlineto{\pgfqpoint{2.481891in}{1.069660in}}%
\pgfpathlineto{\pgfqpoint{2.486400in}{1.323903in}}%
\pgfpathlineto{\pgfqpoint{2.490909in}{1.337012in}}%
\pgfpathlineto{\pgfqpoint{2.495418in}{1.080467in}}%
\pgfpathlineto{\pgfqpoint{2.499927in}{1.092816in}}%
\pgfpathlineto{\pgfqpoint{2.504436in}{1.235813in}}%
\pgfpathlineto{\pgfqpoint{2.508945in}{1.084661in}}%
\pgfpathlineto{\pgfqpoint{2.513455in}{1.080136in}}%
\pgfpathlineto{\pgfqpoint{2.517964in}{1.192151in}}%
\pgfpathlineto{\pgfqpoint{2.522473in}{1.081829in}}%
\pgfpathlineto{\pgfqpoint{2.526982in}{1.145485in}}%
\pgfpathlineto{\pgfqpoint{2.531491in}{1.078002in}}%
\pgfpathlineto{\pgfqpoint{2.536000in}{1.127006in}}%
\pgfpathlineto{\pgfqpoint{2.540509in}{1.126313in}}%
\pgfpathlineto{\pgfqpoint{2.549527in}{1.090281in}}%
\pgfpathlineto{\pgfqpoint{2.554036in}{1.089726in}}%
\pgfpathlineto{\pgfqpoint{2.558545in}{1.416740in}}%
\pgfpathlineto{\pgfqpoint{2.563055in}{1.098211in}}%
\pgfpathlineto{\pgfqpoint{2.567564in}{1.136761in}}%
\pgfpathlineto{\pgfqpoint{2.572073in}{1.090147in}}%
\pgfpathlineto{\pgfqpoint{2.576582in}{1.122394in}}%
\pgfpathlineto{\pgfqpoint{2.581091in}{1.100818in}}%
\pgfpathlineto{\pgfqpoint{2.585600in}{1.135422in}}%
\pgfpathlineto{\pgfqpoint{2.594618in}{1.289377in}}%
\pgfpathlineto{\pgfqpoint{2.599127in}{1.326341in}}%
\pgfpathlineto{\pgfqpoint{2.603636in}{1.146308in}}%
\pgfpathlineto{\pgfqpoint{2.608145in}{1.330278in}}%
\pgfpathlineto{\pgfqpoint{2.612655in}{1.414808in}}%
\pgfpathlineto{\pgfqpoint{2.617164in}{1.114092in}}%
\pgfpathlineto{\pgfqpoint{2.626182in}{1.316372in}}%
\pgfpathlineto{\pgfqpoint{2.630691in}{1.163690in}}%
\pgfpathlineto{\pgfqpoint{2.635200in}{1.127535in}}%
\pgfpathlineto{\pgfqpoint{2.639709in}{1.188188in}}%
\pgfpathlineto{\pgfqpoint{2.644218in}{1.107478in}}%
\pgfpathlineto{\pgfqpoint{2.648727in}{1.196700in}}%
\pgfpathlineto{\pgfqpoint{2.653236in}{1.392707in}}%
\pgfpathlineto{\pgfqpoint{2.657745in}{1.132487in}}%
\pgfpathlineto{\pgfqpoint{2.662255in}{1.135521in}}%
\pgfpathlineto{\pgfqpoint{2.666764in}{1.160489in}}%
\pgfpathlineto{\pgfqpoint{2.671273in}{1.166275in}}%
\pgfpathlineto{\pgfqpoint{2.675782in}{1.144763in}}%
\pgfpathlineto{\pgfqpoint{2.680291in}{1.206495in}}%
\pgfpathlineto{\pgfqpoint{2.684800in}{1.332283in}}%
\pgfpathlineto{\pgfqpoint{2.689309in}{1.220418in}}%
\pgfpathlineto{\pgfqpoint{2.693818in}{1.187662in}}%
\pgfpathlineto{\pgfqpoint{2.698327in}{1.131963in}}%
\pgfpathlineto{\pgfqpoint{2.702836in}{1.144147in}}%
\pgfpathlineto{\pgfqpoint{2.707345in}{1.151412in}}%
\pgfpathlineto{\pgfqpoint{2.711855in}{1.446963in}}%
\pgfpathlineto{\pgfqpoint{2.716364in}{1.354848in}}%
\pgfpathlineto{\pgfqpoint{2.720873in}{1.376495in}}%
\pgfpathlineto{\pgfqpoint{2.725382in}{1.207520in}}%
\pgfpathlineto{\pgfqpoint{2.729891in}{1.161698in}}%
\pgfpathlineto{\pgfqpoint{2.734400in}{1.645775in}}%
\pgfpathlineto{\pgfqpoint{2.738909in}{1.189233in}}%
\pgfpathlineto{\pgfqpoint{2.743418in}{1.137150in}}%
\pgfpathlineto{\pgfqpoint{2.747927in}{1.339654in}}%
\pgfpathlineto{\pgfqpoint{2.752436in}{1.179900in}}%
\pgfpathlineto{\pgfqpoint{2.756945in}{1.705166in}}%
\pgfpathlineto{\pgfqpoint{2.761455in}{1.163872in}}%
\pgfpathlineto{\pgfqpoint{2.765964in}{1.141378in}}%
\pgfpathlineto{\pgfqpoint{2.774982in}{1.218128in}}%
\pgfpathlineto{\pgfqpoint{2.779491in}{1.150285in}}%
\pgfpathlineto{\pgfqpoint{2.784000in}{1.165748in}}%
\pgfpathlineto{\pgfqpoint{2.788509in}{1.151080in}}%
\pgfpathlineto{\pgfqpoint{2.793018in}{1.159657in}}%
\pgfpathlineto{\pgfqpoint{2.797527in}{1.694274in}}%
\pgfpathlineto{\pgfqpoint{2.802036in}{1.179937in}}%
\pgfpathlineto{\pgfqpoint{2.806545in}{1.166135in}}%
\pgfpathlineto{\pgfqpoint{2.811055in}{1.290829in}}%
\pgfpathlineto{\pgfqpoint{2.815564in}{1.185779in}}%
\pgfpathlineto{\pgfqpoint{2.820073in}{1.170717in}}%
\pgfpathlineto{\pgfqpoint{2.824582in}{1.317870in}}%
\pgfpathlineto{\pgfqpoint{2.829091in}{1.227447in}}%
\pgfpathlineto{\pgfqpoint{2.833600in}{1.214384in}}%
\pgfpathlineto{\pgfqpoint{2.838109in}{1.240975in}}%
\pgfpathlineto{\pgfqpoint{2.842618in}{1.319784in}}%
\pgfpathlineto{\pgfqpoint{2.847127in}{1.183519in}}%
\pgfpathlineto{\pgfqpoint{2.851636in}{1.449062in}}%
\pgfpathlineto{\pgfqpoint{2.856145in}{1.173794in}}%
\pgfpathlineto{\pgfqpoint{2.860655in}{1.189314in}}%
\pgfpathlineto{\pgfqpoint{2.865164in}{1.287258in}}%
\pgfpathlineto{\pgfqpoint{2.869673in}{1.231389in}}%
\pgfpathlineto{\pgfqpoint{2.874182in}{1.229261in}}%
\pgfpathlineto{\pgfqpoint{2.878691in}{1.178578in}}%
\pgfpathlineto{\pgfqpoint{2.883200in}{1.216516in}}%
\pgfpathlineto{\pgfqpoint{2.887709in}{1.202177in}}%
\pgfpathlineto{\pgfqpoint{2.892218in}{1.250269in}}%
\pgfpathlineto{\pgfqpoint{2.896727in}{1.239466in}}%
\pgfpathlineto{\pgfqpoint{2.901236in}{1.243265in}}%
\pgfpathlineto{\pgfqpoint{2.910255in}{1.240569in}}%
\pgfpathlineto{\pgfqpoint{2.914764in}{1.356932in}}%
\pgfpathlineto{\pgfqpoint{2.919273in}{1.235265in}}%
\pgfpathlineto{\pgfqpoint{2.923782in}{1.223043in}}%
\pgfpathlineto{\pgfqpoint{2.928291in}{1.287868in}}%
\pgfpathlineto{\pgfqpoint{2.932800in}{1.226115in}}%
\pgfpathlineto{\pgfqpoint{2.937309in}{1.340022in}}%
\pgfpathlineto{\pgfqpoint{2.941818in}{1.202788in}}%
\pgfpathlineto{\pgfqpoint{2.946327in}{1.205751in}}%
\pgfpathlineto{\pgfqpoint{2.950836in}{1.525765in}}%
\pgfpathlineto{\pgfqpoint{2.955345in}{1.242545in}}%
\pgfpathlineto{\pgfqpoint{2.959855in}{1.231626in}}%
\pgfpathlineto{\pgfqpoint{2.964364in}{1.362112in}}%
\pgfpathlineto{\pgfqpoint{2.968873in}{1.202234in}}%
\pgfpathlineto{\pgfqpoint{2.973382in}{1.279087in}}%
\pgfpathlineto{\pgfqpoint{2.977891in}{1.268043in}}%
\pgfpathlineto{\pgfqpoint{2.982400in}{1.212193in}}%
\pgfpathlineto{\pgfqpoint{2.986909in}{1.243474in}}%
\pgfpathlineto{\pgfqpoint{2.991418in}{1.666826in}}%
\pgfpathlineto{\pgfqpoint{2.995927in}{1.281435in}}%
\pgfpathlineto{\pgfqpoint{3.000436in}{1.282736in}}%
\pgfpathlineto{\pgfqpoint{3.004945in}{1.488159in}}%
\pgfpathlineto{\pgfqpoint{3.009455in}{1.270056in}}%
\pgfpathlineto{\pgfqpoint{3.018473in}{1.211566in}}%
\pgfpathlineto{\pgfqpoint{3.022982in}{1.238301in}}%
\pgfpathlineto{\pgfqpoint{3.027491in}{1.248814in}}%
\pgfpathlineto{\pgfqpoint{3.032000in}{1.253905in}}%
\pgfpathlineto{\pgfqpoint{3.036509in}{1.311848in}}%
\pgfpathlineto{\pgfqpoint{3.041018in}{1.249509in}}%
\pgfpathlineto{\pgfqpoint{3.045527in}{1.274008in}}%
\pgfpathlineto{\pgfqpoint{3.050036in}{1.431167in}}%
\pgfpathlineto{\pgfqpoint{3.054545in}{1.273625in}}%
\pgfpathlineto{\pgfqpoint{3.059055in}{1.257786in}}%
\pgfpathlineto{\pgfqpoint{3.063564in}{1.417103in}}%
\pgfpathlineto{\pgfqpoint{3.068073in}{1.290469in}}%
\pgfpathlineto{\pgfqpoint{3.072582in}{1.243618in}}%
\pgfpathlineto{\pgfqpoint{3.077091in}{1.490959in}}%
\pgfpathlineto{\pgfqpoint{3.081600in}{1.279533in}}%
\pgfpathlineto{\pgfqpoint{3.086109in}{1.282441in}}%
\pgfpathlineto{\pgfqpoint{3.090618in}{1.310293in}}%
\pgfpathlineto{\pgfqpoint{3.095127in}{1.278178in}}%
\pgfpathlineto{\pgfqpoint{3.099636in}{1.273416in}}%
\pgfpathlineto{\pgfqpoint{3.104145in}{1.336856in}}%
\pgfpathlineto{\pgfqpoint{3.108655in}{1.254450in}}%
\pgfpathlineto{\pgfqpoint{3.113164in}{1.316593in}}%
\pgfpathlineto{\pgfqpoint{3.117673in}{1.297621in}}%
\pgfpathlineto{\pgfqpoint{3.122182in}{1.252025in}}%
\pgfpathlineto{\pgfqpoint{3.126691in}{1.336552in}}%
\pgfpathlineto{\pgfqpoint{3.131200in}{1.673563in}}%
\pgfpathlineto{\pgfqpoint{3.135709in}{1.414544in}}%
\pgfpathlineto{\pgfqpoint{3.140218in}{1.291699in}}%
\pgfpathlineto{\pgfqpoint{3.144727in}{1.282819in}}%
\pgfpathlineto{\pgfqpoint{3.149236in}{1.295084in}}%
\pgfpathlineto{\pgfqpoint{3.153745in}{1.340088in}}%
\pgfpathlineto{\pgfqpoint{3.158255in}{1.651111in}}%
\pgfpathlineto{\pgfqpoint{3.162764in}{1.263612in}}%
\pgfpathlineto{\pgfqpoint{3.167273in}{1.323317in}}%
\pgfpathlineto{\pgfqpoint{3.171782in}{1.354713in}}%
\pgfpathlineto{\pgfqpoint{3.176291in}{1.534660in}}%
\pgfpathlineto{\pgfqpoint{3.180800in}{1.311267in}}%
\pgfpathlineto{\pgfqpoint{3.185309in}{1.273057in}}%
\pgfpathlineto{\pgfqpoint{3.189818in}{1.316442in}}%
\pgfpathlineto{\pgfqpoint{3.194327in}{1.283019in}}%
\pgfpathlineto{\pgfqpoint{3.198836in}{1.724504in}}%
\pgfpathlineto{\pgfqpoint{3.203345in}{1.335959in}}%
\pgfpathlineto{\pgfqpoint{3.207855in}{1.394246in}}%
\pgfpathlineto{\pgfqpoint{3.212364in}{1.299640in}}%
\pgfpathlineto{\pgfqpoint{3.216873in}{1.362734in}}%
\pgfpathlineto{\pgfqpoint{3.221382in}{1.401469in}}%
\pgfpathlineto{\pgfqpoint{3.225891in}{1.305614in}}%
\pgfpathlineto{\pgfqpoint{3.230400in}{1.369934in}}%
\pgfpathlineto{\pgfqpoint{3.234909in}{1.299806in}}%
\pgfpathlineto{\pgfqpoint{3.239418in}{1.324619in}}%
\pgfpathlineto{\pgfqpoint{3.243927in}{1.370179in}}%
\pgfpathlineto{\pgfqpoint{3.248436in}{1.394945in}}%
\pgfpathlineto{\pgfqpoint{3.252945in}{1.512215in}}%
\pgfpathlineto{\pgfqpoint{3.261964in}{1.295220in}}%
\pgfpathlineto{\pgfqpoint{3.266473in}{1.525333in}}%
\pgfpathlineto{\pgfqpoint{3.270982in}{1.328382in}}%
\pgfpathlineto{\pgfqpoint{3.275491in}{1.315066in}}%
\pgfpathlineto{\pgfqpoint{3.280000in}{1.394085in}}%
\pgfpathlineto{\pgfqpoint{3.284509in}{1.542208in}}%
\pgfpathlineto{\pgfqpoint{3.289018in}{1.345260in}}%
\pgfpathlineto{\pgfqpoint{3.293527in}{1.658348in}}%
\pgfpathlineto{\pgfqpoint{3.298036in}{1.370197in}}%
\pgfpathlineto{\pgfqpoint{3.302545in}{1.590373in}}%
\pgfpathlineto{\pgfqpoint{3.307055in}{1.369853in}}%
\pgfpathlineto{\pgfqpoint{3.311564in}{1.437858in}}%
\pgfpathlineto{\pgfqpoint{3.316073in}{1.785005in}}%
\pgfpathlineto{\pgfqpoint{3.320582in}{1.312888in}}%
\pgfpathlineto{\pgfqpoint{3.325091in}{1.417257in}}%
\pgfpathlineto{\pgfqpoint{3.329600in}{1.722329in}}%
\pgfpathlineto{\pgfqpoint{3.334109in}{1.352651in}}%
\pgfpathlineto{\pgfqpoint{3.338618in}{1.592884in}}%
\pgfpathlineto{\pgfqpoint{3.343127in}{1.356864in}}%
\pgfpathlineto{\pgfqpoint{3.347636in}{1.370284in}}%
\pgfpathlineto{\pgfqpoint{3.352145in}{1.573259in}}%
\pgfpathlineto{\pgfqpoint{3.356655in}{1.380630in}}%
\pgfpathlineto{\pgfqpoint{3.361164in}{1.550392in}}%
\pgfpathlineto{\pgfqpoint{3.365673in}{1.450699in}}%
\pgfpathlineto{\pgfqpoint{3.370182in}{1.508240in}}%
\pgfpathlineto{\pgfqpoint{3.379200in}{1.358273in}}%
\pgfpathlineto{\pgfqpoint{3.383709in}{1.343063in}}%
\pgfpathlineto{\pgfqpoint{3.388218in}{1.335872in}}%
\pgfpathlineto{\pgfqpoint{3.392727in}{1.464396in}}%
\pgfpathlineto{\pgfqpoint{3.397236in}{1.400790in}}%
\pgfpathlineto{\pgfqpoint{3.401745in}{1.539532in}}%
\pgfpathlineto{\pgfqpoint{3.406255in}{1.340329in}}%
\pgfpathlineto{\pgfqpoint{3.410764in}{1.474983in}}%
\pgfpathlineto{\pgfqpoint{3.415273in}{1.399860in}}%
\pgfpathlineto{\pgfqpoint{3.419782in}{1.369341in}}%
\pgfpathlineto{\pgfqpoint{3.424291in}{1.347092in}}%
\pgfpathlineto{\pgfqpoint{3.428800in}{1.387132in}}%
\pgfpathlineto{\pgfqpoint{3.433309in}{1.394486in}}%
\pgfpathlineto{\pgfqpoint{3.437818in}{2.023559in}}%
\pgfpathlineto{\pgfqpoint{3.442327in}{1.560753in}}%
\pgfpathlineto{\pgfqpoint{3.446836in}{1.622558in}}%
\pgfpathlineto{\pgfqpoint{3.451345in}{1.378543in}}%
\pgfpathlineto{\pgfqpoint{3.455855in}{1.389395in}}%
\pgfpathlineto{\pgfqpoint{3.460364in}{1.411563in}}%
\pgfpathlineto{\pgfqpoint{3.464873in}{1.467356in}}%
\pgfpathlineto{\pgfqpoint{3.469382in}{1.805852in}}%
\pgfpathlineto{\pgfqpoint{3.473891in}{2.044862in}}%
\pgfpathlineto{\pgfqpoint{3.478400in}{1.397315in}}%
\pgfpathlineto{\pgfqpoint{3.482909in}{1.639865in}}%
\pgfpathlineto{\pgfqpoint{3.487418in}{1.612217in}}%
\pgfpathlineto{\pgfqpoint{3.496436in}{1.412271in}}%
\pgfpathlineto{\pgfqpoint{3.500945in}{1.570617in}}%
\pgfpathlineto{\pgfqpoint{3.505455in}{1.611859in}}%
\pgfpathlineto{\pgfqpoint{3.509964in}{1.376862in}}%
\pgfpathlineto{\pgfqpoint{3.514473in}{1.712807in}}%
\pgfpathlineto{\pgfqpoint{3.518982in}{1.379351in}}%
\pgfpathlineto{\pgfqpoint{3.523491in}{1.441023in}}%
\pgfpathlineto{\pgfqpoint{3.528000in}{1.456131in}}%
\pgfpathlineto{\pgfqpoint{3.537018in}{1.377099in}}%
\pgfpathlineto{\pgfqpoint{3.541527in}{1.398246in}}%
\pgfpathlineto{\pgfqpoint{3.546036in}{1.555622in}}%
\pgfpathlineto{\pgfqpoint{3.550545in}{2.022171in}}%
\pgfpathlineto{\pgfqpoint{3.555055in}{1.384673in}}%
\pgfpathlineto{\pgfqpoint{3.559564in}{1.564414in}}%
\pgfpathlineto{\pgfqpoint{3.564073in}{1.478602in}}%
\pgfpathlineto{\pgfqpoint{3.568582in}{1.467170in}}%
\pgfpathlineto{\pgfqpoint{3.573091in}{1.418891in}}%
\pgfpathlineto{\pgfqpoint{3.577600in}{1.487679in}}%
\pgfpathlineto{\pgfqpoint{3.582109in}{1.421520in}}%
\pgfpathlineto{\pgfqpoint{3.586618in}{1.503521in}}%
\pgfpathlineto{\pgfqpoint{3.591127in}{1.414138in}}%
\pgfpathlineto{\pgfqpoint{3.595636in}{1.468409in}}%
\pgfpathlineto{\pgfqpoint{3.600145in}{1.444311in}}%
\pgfpathlineto{\pgfqpoint{3.604655in}{1.448234in}}%
\pgfpathlineto{\pgfqpoint{3.609164in}{1.429292in}}%
\pgfpathlineto{\pgfqpoint{3.613673in}{1.455073in}}%
\pgfpathlineto{\pgfqpoint{3.618182in}{1.454468in}}%
\pgfpathlineto{\pgfqpoint{3.622691in}{1.464124in}}%
\pgfpathlineto{\pgfqpoint{3.627200in}{1.401628in}}%
\pgfpathlineto{\pgfqpoint{3.631709in}{1.584079in}}%
\pgfpathlineto{\pgfqpoint{3.636218in}{1.452620in}}%
\pgfpathlineto{\pgfqpoint{3.640727in}{1.443814in}}%
\pgfpathlineto{\pgfqpoint{3.649745in}{1.724838in}}%
\pgfpathlineto{\pgfqpoint{3.654255in}{1.414134in}}%
\pgfpathlineto{\pgfqpoint{3.658764in}{1.637738in}}%
\pgfpathlineto{\pgfqpoint{3.663273in}{1.492621in}}%
\pgfpathlineto{\pgfqpoint{3.667782in}{1.435857in}}%
\pgfpathlineto{\pgfqpoint{3.672291in}{1.484492in}}%
\pgfpathlineto{\pgfqpoint{3.676800in}{1.441767in}}%
\pgfpathlineto{\pgfqpoint{3.681309in}{1.657249in}}%
\pgfpathlineto{\pgfqpoint{3.685818in}{1.736839in}}%
\pgfpathlineto{\pgfqpoint{3.690327in}{1.485514in}}%
\pgfpathlineto{\pgfqpoint{3.694836in}{1.501130in}}%
\pgfpathlineto{\pgfqpoint{3.699345in}{1.546665in}}%
\pgfpathlineto{\pgfqpoint{3.703855in}{1.949187in}}%
\pgfpathlineto{\pgfqpoint{3.708364in}{1.468525in}}%
\pgfpathlineto{\pgfqpoint{3.712873in}{1.512537in}}%
\pgfpathlineto{\pgfqpoint{3.717382in}{1.529388in}}%
\pgfpathlineto{\pgfqpoint{3.721891in}{1.465414in}}%
\pgfpathlineto{\pgfqpoint{3.726400in}{1.487404in}}%
\pgfpathlineto{\pgfqpoint{3.730909in}{1.486790in}}%
\pgfpathlineto{\pgfqpoint{3.735418in}{1.541774in}}%
\pgfpathlineto{\pgfqpoint{3.739927in}{1.545152in}}%
\pgfpathlineto{\pgfqpoint{3.744436in}{1.438895in}}%
\pgfpathlineto{\pgfqpoint{3.748945in}{1.662397in}}%
\pgfpathlineto{\pgfqpoint{3.753455in}{1.493645in}}%
\pgfpathlineto{\pgfqpoint{3.757964in}{1.663368in}}%
\pgfpathlineto{\pgfqpoint{3.762473in}{1.662181in}}%
\pgfpathlineto{\pgfqpoint{3.766982in}{1.527187in}}%
\pgfpathlineto{\pgfqpoint{3.780509in}{1.475481in}}%
\pgfpathlineto{\pgfqpoint{3.785018in}{1.618101in}}%
\pgfpathlineto{\pgfqpoint{3.789527in}{3.119266in}}%
\pgfpathlineto{\pgfqpoint{3.794036in}{1.857527in}}%
\pgfpathlineto{\pgfqpoint{3.798545in}{2.499893in}}%
\pgfpathlineto{\pgfqpoint{3.803055in}{2.350382in}}%
\pgfpathlineto{\pgfqpoint{3.807564in}{1.566590in}}%
\pgfpathlineto{\pgfqpoint{3.812073in}{2.305114in}}%
\pgfpathlineto{\pgfqpoint{3.821091in}{1.581719in}}%
\pgfpathlineto{\pgfqpoint{3.825600in}{1.486123in}}%
\pgfpathlineto{\pgfqpoint{3.830109in}{1.818437in}}%
\pgfpathlineto{\pgfqpoint{3.834618in}{1.508942in}}%
\pgfpathlineto{\pgfqpoint{3.839127in}{1.482693in}}%
\pgfpathlineto{\pgfqpoint{3.843636in}{1.841214in}}%
\pgfpathlineto{\pgfqpoint{3.848145in}{1.588083in}}%
\pgfpathlineto{\pgfqpoint{3.852655in}{1.675390in}}%
\pgfpathlineto{\pgfqpoint{3.857164in}{1.484373in}}%
\pgfpathlineto{\pgfqpoint{3.861673in}{1.568696in}}%
\pgfpathlineto{\pgfqpoint{3.866182in}{1.706373in}}%
\pgfpathlineto{\pgfqpoint{3.870691in}{1.512411in}}%
\pgfpathlineto{\pgfqpoint{3.875200in}{1.500122in}}%
\pgfpathlineto{\pgfqpoint{3.879709in}{1.528822in}}%
\pgfpathlineto{\pgfqpoint{3.884218in}{2.202086in}}%
\pgfpathlineto{\pgfqpoint{3.888727in}{1.520621in}}%
\pgfpathlineto{\pgfqpoint{3.893236in}{1.502827in}}%
\pgfpathlineto{\pgfqpoint{3.897745in}{1.891873in}}%
\pgfpathlineto{\pgfqpoint{3.902255in}{1.692581in}}%
\pgfpathlineto{\pgfqpoint{3.906764in}{2.128821in}}%
\pgfpathlineto{\pgfqpoint{3.911273in}{1.678522in}}%
\pgfpathlineto{\pgfqpoint{3.920291in}{1.907167in}}%
\pgfpathlineto{\pgfqpoint{3.924800in}{1.486431in}}%
\pgfpathlineto{\pgfqpoint{3.929309in}{1.560351in}}%
\pgfpathlineto{\pgfqpoint{3.933818in}{1.578994in}}%
\pgfpathlineto{\pgfqpoint{3.938327in}{1.519318in}}%
\pgfpathlineto{\pgfqpoint{3.942836in}{1.740115in}}%
\pgfpathlineto{\pgfqpoint{3.947345in}{1.573189in}}%
\pgfpathlineto{\pgfqpoint{3.951855in}{1.573585in}}%
\pgfpathlineto{\pgfqpoint{3.956364in}{1.641929in}}%
\pgfpathlineto{\pgfqpoint{3.960873in}{1.543716in}}%
\pgfpathlineto{\pgfqpoint{3.965382in}{1.559683in}}%
\pgfpathlineto{\pgfqpoint{3.969891in}{1.520518in}}%
\pgfpathlineto{\pgfqpoint{3.974400in}{1.566232in}}%
\pgfpathlineto{\pgfqpoint{3.978909in}{1.522504in}}%
\pgfpathlineto{\pgfqpoint{3.983418in}{1.837339in}}%
\pgfpathlineto{\pgfqpoint{3.987927in}{1.754732in}}%
\pgfpathlineto{\pgfqpoint{3.992436in}{1.519904in}}%
\pgfpathlineto{\pgfqpoint{3.996945in}{1.620625in}}%
\pgfpathlineto{\pgfqpoint{4.001455in}{1.561926in}}%
\pgfpathlineto{\pgfqpoint{4.005964in}{1.645467in}}%
\pgfpathlineto{\pgfqpoint{4.010473in}{1.759873in}}%
\pgfpathlineto{\pgfqpoint{4.014982in}{1.559269in}}%
\pgfpathlineto{\pgfqpoint{4.019491in}{1.654661in}}%
\pgfpathlineto{\pgfqpoint{4.024000in}{1.810062in}}%
\pgfpathlineto{\pgfqpoint{4.033018in}{1.537096in}}%
\pgfpathlineto{\pgfqpoint{4.042036in}{1.642268in}}%
\pgfpathlineto{\pgfqpoint{4.046545in}{1.571471in}}%
\pgfpathlineto{\pgfqpoint{4.051055in}{1.634656in}}%
\pgfpathlineto{\pgfqpoint{4.060073in}{1.576719in}}%
\pgfpathlineto{\pgfqpoint{4.064582in}{1.640352in}}%
\pgfpathlineto{\pgfqpoint{4.069091in}{1.591126in}}%
\pgfpathlineto{\pgfqpoint{4.073600in}{1.698991in}}%
\pgfpathlineto{\pgfqpoint{4.078109in}{1.633520in}}%
\pgfpathlineto{\pgfqpoint{4.082618in}{1.599014in}}%
\pgfpathlineto{\pgfqpoint{4.087127in}{1.580591in}}%
\pgfpathlineto{\pgfqpoint{4.091636in}{1.778324in}}%
\pgfpathlineto{\pgfqpoint{4.096145in}{1.765725in}}%
\pgfpathlineto{\pgfqpoint{4.100655in}{1.611267in}}%
\pgfpathlineto{\pgfqpoint{4.105164in}{1.580106in}}%
\pgfpathlineto{\pgfqpoint{4.109673in}{1.651843in}}%
\pgfpathlineto{\pgfqpoint{4.114182in}{1.910430in}}%
\pgfpathlineto{\pgfqpoint{4.118691in}{1.703700in}}%
\pgfpathlineto{\pgfqpoint{4.123200in}{1.902932in}}%
\pgfpathlineto{\pgfqpoint{4.127709in}{1.729925in}}%
\pgfpathlineto{\pgfqpoint{4.132218in}{1.602001in}}%
\pgfpathlineto{\pgfqpoint{4.136727in}{1.602606in}}%
\pgfpathlineto{\pgfqpoint{4.141236in}{1.798571in}}%
\pgfpathlineto{\pgfqpoint{4.145745in}{1.696882in}}%
\pgfpathlineto{\pgfqpoint{4.150255in}{1.731952in}}%
\pgfpathlineto{\pgfqpoint{4.154764in}{1.569377in}}%
\pgfpathlineto{\pgfqpoint{4.159273in}{1.695823in}}%
\pgfpathlineto{\pgfqpoint{4.163782in}{1.655212in}}%
\pgfpathlineto{\pgfqpoint{4.168291in}{2.019936in}}%
\pgfpathlineto{\pgfqpoint{4.172800in}{1.570371in}}%
\pgfpathlineto{\pgfqpoint{4.177309in}{1.592834in}}%
\pgfpathlineto{\pgfqpoint{4.181818in}{2.134317in}}%
\pgfpathlineto{\pgfqpoint{4.186327in}{1.637796in}}%
\pgfpathlineto{\pgfqpoint{4.190836in}{1.759418in}}%
\pgfpathlineto{\pgfqpoint{4.195345in}{1.733436in}}%
\pgfpathlineto{\pgfqpoint{4.199855in}{1.776885in}}%
\pgfpathlineto{\pgfqpoint{4.204364in}{1.628487in}}%
\pgfpathlineto{\pgfqpoint{4.208873in}{1.619094in}}%
\pgfpathlineto{\pgfqpoint{4.213382in}{1.673569in}}%
\pgfpathlineto{\pgfqpoint{4.217891in}{1.712787in}}%
\pgfpathlineto{\pgfqpoint{4.222400in}{1.675245in}}%
\pgfpathlineto{\pgfqpoint{4.226909in}{1.790427in}}%
\pgfpathlineto{\pgfqpoint{4.231418in}{1.667362in}}%
\pgfpathlineto{\pgfqpoint{4.235927in}{1.638814in}}%
\pgfpathlineto{\pgfqpoint{4.240436in}{1.742292in}}%
\pgfpathlineto{\pgfqpoint{4.244945in}{1.651125in}}%
\pgfpathlineto{\pgfqpoint{4.249455in}{1.751929in}}%
\pgfpathlineto{\pgfqpoint{4.253964in}{1.666262in}}%
\pgfpathlineto{\pgfqpoint{4.258473in}{1.603829in}}%
\pgfpathlineto{\pgfqpoint{4.262982in}{1.659489in}}%
\pgfpathlineto{\pgfqpoint{4.267491in}{1.874926in}}%
\pgfpathlineto{\pgfqpoint{4.272000in}{1.595363in}}%
\pgfpathlineto{\pgfqpoint{4.276509in}{1.656040in}}%
\pgfpathlineto{\pgfqpoint{4.281018in}{1.686499in}}%
\pgfpathlineto{\pgfqpoint{4.285527in}{1.618598in}}%
\pgfpathlineto{\pgfqpoint{4.290036in}{1.744960in}}%
\pgfpathlineto{\pgfqpoint{4.294545in}{1.746301in}}%
\pgfpathlineto{\pgfqpoint{4.299055in}{1.607981in}}%
\pgfpathlineto{\pgfqpoint{4.303564in}{1.917334in}}%
\pgfpathlineto{\pgfqpoint{4.308073in}{2.380767in}}%
\pgfpathlineto{\pgfqpoint{4.312582in}{1.765595in}}%
\pgfpathlineto{\pgfqpoint{4.317091in}{1.818516in}}%
\pgfpathlineto{\pgfqpoint{4.321600in}{1.656942in}}%
\pgfpathlineto{\pgfqpoint{4.326109in}{1.680834in}}%
\pgfpathlineto{\pgfqpoint{4.330618in}{1.666936in}}%
\pgfpathlineto{\pgfqpoint{4.335127in}{1.710284in}}%
\pgfpathlineto{\pgfqpoint{4.339636in}{1.860025in}}%
\pgfpathlineto{\pgfqpoint{4.344145in}{1.672531in}}%
\pgfpathlineto{\pgfqpoint{4.348655in}{1.840719in}}%
\pgfpathlineto{\pgfqpoint{4.353164in}{1.671723in}}%
\pgfpathlineto{\pgfqpoint{4.357673in}{1.883165in}}%
\pgfpathlineto{\pgfqpoint{4.362182in}{1.907469in}}%
\pgfpathlineto{\pgfqpoint{4.366691in}{1.900296in}}%
\pgfpathlineto{\pgfqpoint{4.371200in}{1.657178in}}%
\pgfpathlineto{\pgfqpoint{4.375709in}{1.652927in}}%
\pgfpathlineto{\pgfqpoint{4.380218in}{1.736645in}}%
\pgfpathlineto{\pgfqpoint{4.384727in}{1.733475in}}%
\pgfpathlineto{\pgfqpoint{4.389236in}{1.662942in}}%
\pgfpathlineto{\pgfqpoint{4.393745in}{1.737549in}}%
\pgfpathlineto{\pgfqpoint{4.398255in}{1.690654in}}%
\pgfpathlineto{\pgfqpoint{4.402764in}{2.010476in}}%
\pgfpathlineto{\pgfqpoint{4.407273in}{1.813892in}}%
\pgfpathlineto{\pgfqpoint{4.411782in}{1.686413in}}%
\pgfpathlineto{\pgfqpoint{4.416291in}{1.751829in}}%
\pgfpathlineto{\pgfqpoint{4.420800in}{1.833468in}}%
\pgfpathlineto{\pgfqpoint{4.425309in}{1.797072in}}%
\pgfpathlineto{\pgfqpoint{4.429818in}{1.921890in}}%
\pgfpathlineto{\pgfqpoint{4.434327in}{1.815193in}}%
\pgfpathlineto{\pgfqpoint{4.438836in}{1.950072in}}%
\pgfpathlineto{\pgfqpoint{4.443345in}{1.692987in}}%
\pgfpathlineto{\pgfqpoint{4.447855in}{2.142512in}}%
\pgfpathlineto{\pgfqpoint{4.452364in}{1.697211in}}%
\pgfpathlineto{\pgfqpoint{4.456873in}{1.681940in}}%
\pgfpathlineto{\pgfqpoint{4.461382in}{1.822285in}}%
\pgfpathlineto{\pgfqpoint{4.465891in}{2.094330in}}%
\pgfpathlineto{\pgfqpoint{4.470400in}{1.696842in}}%
\pgfpathlineto{\pgfqpoint{4.474909in}{1.797630in}}%
\pgfpathlineto{\pgfqpoint{4.479418in}{2.183214in}}%
\pgfpathlineto{\pgfqpoint{4.483927in}{1.712220in}}%
\pgfpathlineto{\pgfqpoint{4.488436in}{1.718692in}}%
\pgfpathlineto{\pgfqpoint{4.492945in}{1.938539in}}%
\pgfpathlineto{\pgfqpoint{4.497455in}{1.764324in}}%
\pgfpathlineto{\pgfqpoint{4.501964in}{1.910197in}}%
\pgfpathlineto{\pgfqpoint{4.506473in}{1.921355in}}%
\pgfpathlineto{\pgfqpoint{4.510982in}{1.973580in}}%
\pgfpathlineto{\pgfqpoint{4.520000in}{1.859214in}}%
\pgfpathlineto{\pgfqpoint{4.524509in}{2.103010in}}%
\pgfpathlineto{\pgfqpoint{4.529018in}{1.692142in}}%
\pgfpathlineto{\pgfqpoint{4.533527in}{1.888976in}}%
\pgfpathlineto{\pgfqpoint{4.538036in}{1.737199in}}%
\pgfpathlineto{\pgfqpoint{4.542545in}{1.963904in}}%
\pgfpathlineto{\pgfqpoint{4.547055in}{1.777168in}}%
\pgfpathlineto{\pgfqpoint{4.551564in}{1.972926in}}%
\pgfpathlineto{\pgfqpoint{4.556073in}{2.009896in}}%
\pgfpathlineto{\pgfqpoint{4.560582in}{1.974450in}}%
\pgfpathlineto{\pgfqpoint{4.565091in}{1.761863in}}%
\pgfpathlineto{\pgfqpoint{4.569600in}{1.740200in}}%
\pgfpathlineto{\pgfqpoint{4.574109in}{1.825666in}}%
\pgfpathlineto{\pgfqpoint{4.578618in}{1.703157in}}%
\pgfpathlineto{\pgfqpoint{4.583127in}{1.856945in}}%
\pgfpathlineto{\pgfqpoint{4.587636in}{1.924872in}}%
\pgfpathlineto{\pgfqpoint{4.592145in}{1.790843in}}%
\pgfpathlineto{\pgfqpoint{4.596655in}{1.788287in}}%
\pgfpathlineto{\pgfqpoint{4.601164in}{2.380925in}}%
\pgfpathlineto{\pgfqpoint{4.605673in}{1.767134in}}%
\pgfpathlineto{\pgfqpoint{4.610182in}{1.935530in}}%
\pgfpathlineto{\pgfqpoint{4.614691in}{1.704604in}}%
\pgfpathlineto{\pgfqpoint{4.619200in}{1.840296in}}%
\pgfpathlineto{\pgfqpoint{4.623709in}{2.191860in}}%
\pgfpathlineto{\pgfqpoint{4.628218in}{3.388799in}}%
\pgfpathlineto{\pgfqpoint{4.632727in}{1.735738in}}%
\pgfpathlineto{\pgfqpoint{4.637236in}{1.857533in}}%
\pgfpathlineto{\pgfqpoint{4.641745in}{1.799911in}}%
\pgfpathlineto{\pgfqpoint{4.646255in}{2.537125in}}%
\pgfpathlineto{\pgfqpoint{4.650764in}{2.026328in}}%
\pgfpathlineto{\pgfqpoint{4.655273in}{1.850748in}}%
\pgfpathlineto{\pgfqpoint{4.659782in}{1.968204in}}%
\pgfpathlineto{\pgfqpoint{4.664291in}{2.375066in}}%
\pgfpathlineto{\pgfqpoint{4.668800in}{3.422748in}}%
\pgfpathlineto{\pgfqpoint{4.673309in}{2.300494in}}%
\pgfpathlineto{\pgfqpoint{4.677818in}{1.860323in}}%
\pgfpathlineto{\pgfqpoint{4.682327in}{2.184605in}}%
\pgfpathlineto{\pgfqpoint{4.686836in}{1.808326in}}%
\pgfpathlineto{\pgfqpoint{4.691345in}{1.827831in}}%
\pgfpathlineto{\pgfqpoint{4.695855in}{1.896190in}}%
\pgfpathlineto{\pgfqpoint{4.700364in}{2.388151in}}%
\pgfpathlineto{\pgfqpoint{4.704873in}{1.851077in}}%
\pgfpathlineto{\pgfqpoint{4.709382in}{4.056000in}}%
\pgfpathlineto{\pgfqpoint{4.713891in}{1.822239in}}%
\pgfpathlineto{\pgfqpoint{4.718400in}{2.540487in}}%
\pgfpathlineto{\pgfqpoint{4.722909in}{2.316086in}}%
\pgfpathlineto{\pgfqpoint{4.727418in}{1.821588in}}%
\pgfpathlineto{\pgfqpoint{4.731927in}{2.059907in}}%
\pgfpathlineto{\pgfqpoint{4.736436in}{1.901768in}}%
\pgfpathlineto{\pgfqpoint{4.740945in}{1.820087in}}%
\pgfpathlineto{\pgfqpoint{4.745455in}{2.891368in}}%
\pgfpathlineto{\pgfqpoint{4.749964in}{2.344352in}}%
\pgfpathlineto{\pgfqpoint{4.754473in}{2.491427in}}%
\pgfpathlineto{\pgfqpoint{4.758982in}{1.750143in}}%
\pgfpathlineto{\pgfqpoint{4.763491in}{1.776430in}}%
\pgfpathlineto{\pgfqpoint{4.768000in}{1.749337in}}%
\pgfpathlineto{\pgfqpoint{4.777018in}{1.752231in}}%
\pgfpathlineto{\pgfqpoint{4.781527in}{1.803463in}}%
\pgfpathlineto{\pgfqpoint{4.786036in}{1.773612in}}%
\pgfpathlineto{\pgfqpoint{4.790545in}{1.757176in}}%
\pgfpathlineto{\pgfqpoint{4.795055in}{1.765574in}}%
\pgfpathlineto{\pgfqpoint{4.799564in}{1.768763in}}%
\pgfpathlineto{\pgfqpoint{4.804073in}{3.241454in}}%
\pgfpathlineto{\pgfqpoint{4.808582in}{1.770359in}}%
\pgfpathlineto{\pgfqpoint{4.817600in}{1.772824in}}%
\pgfpathlineto{\pgfqpoint{4.822109in}{1.796526in}}%
\pgfpathlineto{\pgfqpoint{4.826618in}{2.971818in}}%
\pgfpathlineto{\pgfqpoint{4.831127in}{1.813727in}}%
\pgfpathlineto{\pgfqpoint{4.835636in}{1.942898in}}%
\pgfpathlineto{\pgfqpoint{4.840145in}{1.976159in}}%
\pgfpathlineto{\pgfqpoint{4.844655in}{1.885376in}}%
\pgfpathlineto{\pgfqpoint{4.849164in}{1.927885in}}%
\pgfpathlineto{\pgfqpoint{4.853673in}{2.189159in}}%
\pgfpathlineto{\pgfqpoint{4.858182in}{1.929468in}}%
\pgfpathlineto{\pgfqpoint{4.862691in}{2.076760in}}%
\pgfpathlineto{\pgfqpoint{4.867200in}{3.890032in}}%
\pgfpathlineto{\pgfqpoint{4.876218in}{1.795835in}}%
\pgfpathlineto{\pgfqpoint{4.880727in}{2.514333in}}%
\pgfpathlineto{\pgfqpoint{4.885236in}{1.833177in}}%
\pgfpathlineto{\pgfqpoint{4.889745in}{2.022127in}}%
\pgfpathlineto{\pgfqpoint{4.894255in}{1.939011in}}%
\pgfpathlineto{\pgfqpoint{4.898764in}{1.821912in}}%
\pgfpathlineto{\pgfqpoint{4.907782in}{2.227737in}}%
\pgfpathlineto{\pgfqpoint{4.912291in}{1.824345in}}%
\pgfpathlineto{\pgfqpoint{4.916800in}{1.844120in}}%
\pgfpathlineto{\pgfqpoint{4.921309in}{2.638235in}}%
\pgfpathlineto{\pgfqpoint{4.925818in}{2.027575in}}%
\pgfpathlineto{\pgfqpoint{4.930327in}{1.885020in}}%
\pgfpathlineto{\pgfqpoint{4.934836in}{1.871880in}}%
\pgfpathlineto{\pgfqpoint{4.939345in}{2.334425in}}%
\pgfpathlineto{\pgfqpoint{4.943855in}{1.887271in}}%
\pgfpathlineto{\pgfqpoint{4.948364in}{2.283408in}}%
\pgfpathlineto{\pgfqpoint{4.952873in}{2.297540in}}%
\pgfpathlineto{\pgfqpoint{4.957382in}{2.343206in}}%
\pgfpathlineto{\pgfqpoint{4.961891in}{1.820286in}}%
\pgfpathlineto{\pgfqpoint{4.966400in}{1.813765in}}%
\pgfpathlineto{\pgfqpoint{4.970909in}{1.865808in}}%
\pgfpathlineto{\pgfqpoint{4.975418in}{1.879194in}}%
\pgfpathlineto{\pgfqpoint{4.979927in}{1.842188in}}%
\pgfpathlineto{\pgfqpoint{4.984436in}{1.902386in}}%
\pgfpathlineto{\pgfqpoint{4.988945in}{1.914255in}}%
\pgfpathlineto{\pgfqpoint{4.993455in}{1.837644in}}%
\pgfpathlineto{\pgfqpoint{4.997964in}{1.986129in}}%
\pgfpathlineto{\pgfqpoint{5.002473in}{2.018044in}}%
\pgfpathlineto{\pgfqpoint{5.006982in}{2.199900in}}%
\pgfpathlineto{\pgfqpoint{5.011491in}{2.117465in}}%
\pgfpathlineto{\pgfqpoint{5.016000in}{1.855992in}}%
\pgfpathlineto{\pgfqpoint{5.020509in}{1.865872in}}%
\pgfpathlineto{\pgfqpoint{5.025018in}{1.845769in}}%
\pgfpathlineto{\pgfqpoint{5.029527in}{2.121862in}}%
\pgfpathlineto{\pgfqpoint{5.034036in}{1.856253in}}%
\pgfpathlineto{\pgfqpoint{5.038545in}{1.915484in}}%
\pgfpathlineto{\pgfqpoint{5.043055in}{1.931731in}}%
\pgfpathlineto{\pgfqpoint{5.047564in}{1.975949in}}%
\pgfpathlineto{\pgfqpoint{5.052073in}{1.957763in}}%
\pgfpathlineto{\pgfqpoint{5.056582in}{2.127901in}}%
\pgfpathlineto{\pgfqpoint{5.061091in}{1.896927in}}%
\pgfpathlineto{\pgfqpoint{5.065600in}{1.919528in}}%
\pgfpathlineto{\pgfqpoint{5.070109in}{2.227967in}}%
\pgfpathlineto{\pgfqpoint{5.074618in}{1.909810in}}%
\pgfpathlineto{\pgfqpoint{5.079127in}{1.888812in}}%
\pgfpathlineto{\pgfqpoint{5.083636in}{1.883804in}}%
\pgfpathlineto{\pgfqpoint{5.088145in}{2.162808in}}%
\pgfpathlineto{\pgfqpoint{5.092655in}{1.963294in}}%
\pgfpathlineto{\pgfqpoint{5.097164in}{2.796776in}}%
\pgfpathlineto{\pgfqpoint{5.101673in}{2.115742in}}%
\pgfpathlineto{\pgfqpoint{5.106182in}{1.913391in}}%
\pgfpathlineto{\pgfqpoint{5.110691in}{1.886068in}}%
\pgfpathlineto{\pgfqpoint{5.115200in}{2.342627in}}%
\pgfpathlineto{\pgfqpoint{5.119709in}{2.079474in}}%
\pgfpathlineto{\pgfqpoint{5.124218in}{2.118055in}}%
\pgfpathlineto{\pgfqpoint{5.128727in}{2.014857in}}%
\pgfpathlineto{\pgfqpoint{5.133236in}{2.484999in}}%
\pgfpathlineto{\pgfqpoint{5.137745in}{1.938058in}}%
\pgfpathlineto{\pgfqpoint{5.142255in}{1.922015in}}%
\pgfpathlineto{\pgfqpoint{5.146764in}{2.316547in}}%
\pgfpathlineto{\pgfqpoint{5.151273in}{2.364699in}}%
\pgfpathlineto{\pgfqpoint{5.155782in}{3.433926in}}%
\pgfpathlineto{\pgfqpoint{5.160291in}{2.453661in}}%
\pgfpathlineto{\pgfqpoint{5.164800in}{2.577514in}}%
\pgfpathlineto{\pgfqpoint{5.169309in}{1.945018in}}%
\pgfpathlineto{\pgfqpoint{5.173818in}{1.960908in}}%
\pgfpathlineto{\pgfqpoint{5.178327in}{2.040802in}}%
\pgfpathlineto{\pgfqpoint{5.182836in}{1.918262in}}%
\pgfpathlineto{\pgfqpoint{5.187345in}{2.025817in}}%
\pgfpathlineto{\pgfqpoint{5.191855in}{1.908451in}}%
\pgfpathlineto{\pgfqpoint{5.196364in}{2.558366in}}%
\pgfpathlineto{\pgfqpoint{5.200873in}{2.022238in}}%
\pgfpathlineto{\pgfqpoint{5.205382in}{1.948284in}}%
\pgfpathlineto{\pgfqpoint{5.209891in}{1.996550in}}%
\pgfpathlineto{\pgfqpoint{5.214400in}{2.562491in}}%
\pgfpathlineto{\pgfqpoint{5.223418in}{2.021512in}}%
\pgfpathlineto{\pgfqpoint{5.227927in}{2.563729in}}%
\pgfpathlineto{\pgfqpoint{5.232436in}{2.294470in}}%
\pgfpathlineto{\pgfqpoint{5.236945in}{2.196628in}}%
\pgfpathlineto{\pgfqpoint{5.241455in}{1.970795in}}%
\pgfpathlineto{\pgfqpoint{5.245964in}{2.634824in}}%
\pgfpathlineto{\pgfqpoint{5.250473in}{1.990975in}}%
\pgfpathlineto{\pgfqpoint{5.254982in}{2.393400in}}%
\pgfpathlineto{\pgfqpoint{5.259491in}{2.039349in}}%
\pgfpathlineto{\pgfqpoint{5.264000in}{2.775775in}}%
\pgfpathlineto{\pgfqpoint{5.268509in}{2.305099in}}%
\pgfpathlineto{\pgfqpoint{5.273018in}{2.643140in}}%
\pgfpathlineto{\pgfqpoint{5.277527in}{2.334346in}}%
\pgfpathlineto{\pgfqpoint{5.282036in}{2.722452in}}%
\pgfpathlineto{\pgfqpoint{5.286545in}{2.172126in}}%
\pgfpathlineto{\pgfqpoint{5.291055in}{2.179116in}}%
\pgfpathlineto{\pgfqpoint{5.295564in}{2.001370in}}%
\pgfpathlineto{\pgfqpoint{5.300073in}{2.073749in}}%
\pgfpathlineto{\pgfqpoint{5.304582in}{1.953661in}}%
\pgfpathlineto{\pgfqpoint{5.309091in}{1.995103in}}%
\pgfpathlineto{\pgfqpoint{5.318109in}{1.936165in}}%
\pgfpathlineto{\pgfqpoint{5.322618in}{1.969714in}}%
\pgfpathlineto{\pgfqpoint{5.327127in}{1.975067in}}%
\pgfpathlineto{\pgfqpoint{5.331636in}{2.022046in}}%
\pgfpathlineto{\pgfqpoint{5.336145in}{1.966649in}}%
\pgfpathlineto{\pgfqpoint{5.340655in}{1.974391in}}%
\pgfpathlineto{\pgfqpoint{5.345164in}{1.944576in}}%
\pgfpathlineto{\pgfqpoint{5.349673in}{1.981690in}}%
\pgfpathlineto{\pgfqpoint{5.354182in}{2.082306in}}%
\pgfpathlineto{\pgfqpoint{5.358691in}{1.987562in}}%
\pgfpathlineto{\pgfqpoint{5.363200in}{2.230455in}}%
\pgfpathlineto{\pgfqpoint{5.367709in}{2.406886in}}%
\pgfpathlineto{\pgfqpoint{5.372218in}{1.981447in}}%
\pgfpathlineto{\pgfqpoint{5.376727in}{2.823161in}}%
\pgfpathlineto{\pgfqpoint{5.381236in}{2.015290in}}%
\pgfpathlineto{\pgfqpoint{5.385745in}{1.997935in}}%
\pgfpathlineto{\pgfqpoint{5.390255in}{2.142103in}}%
\pgfpathlineto{\pgfqpoint{5.394764in}{1.955896in}}%
\pgfpathlineto{\pgfqpoint{5.399273in}{2.004949in}}%
\pgfpathlineto{\pgfqpoint{5.403782in}{2.596743in}}%
\pgfpathlineto{\pgfqpoint{5.408291in}{1.987063in}}%
\pgfpathlineto{\pgfqpoint{5.412800in}{2.240830in}}%
\pgfpathlineto{\pgfqpoint{5.417309in}{2.396882in}}%
\pgfpathlineto{\pgfqpoint{5.421818in}{2.156469in}}%
\pgfpathlineto{\pgfqpoint{5.426327in}{2.412685in}}%
\pgfpathlineto{\pgfqpoint{5.430836in}{2.017239in}}%
\pgfpathlineto{\pgfqpoint{5.435345in}{2.550114in}}%
\pgfpathlineto{\pgfqpoint{5.439855in}{2.040703in}}%
\pgfpathlineto{\pgfqpoint{5.444364in}{2.045207in}}%
\pgfpathlineto{\pgfqpoint{5.448873in}{2.148566in}}%
\pgfpathlineto{\pgfqpoint{5.453382in}{2.088118in}}%
\pgfpathlineto{\pgfqpoint{5.457891in}{2.004680in}}%
\pgfpathlineto{\pgfqpoint{5.462400in}{2.116945in}}%
\pgfpathlineto{\pgfqpoint{5.466909in}{2.310393in}}%
\pgfpathlineto{\pgfqpoint{5.471418in}{2.256227in}}%
\pgfpathlineto{\pgfqpoint{5.475927in}{1.998375in}}%
\pgfpathlineto{\pgfqpoint{5.480436in}{2.053517in}}%
\pgfpathlineto{\pgfqpoint{5.484945in}{2.014846in}}%
\pgfpathlineto{\pgfqpoint{5.489455in}{2.028026in}}%
\pgfpathlineto{\pgfqpoint{5.493964in}{1.995647in}}%
\pgfpathlineto{\pgfqpoint{5.498473in}{2.248103in}}%
\pgfpathlineto{\pgfqpoint{5.502982in}{2.030191in}}%
\pgfpathlineto{\pgfqpoint{5.507491in}{2.573051in}}%
\pgfpathlineto{\pgfqpoint{5.512000in}{2.163273in}}%
\pgfpathlineto{\pgfqpoint{5.516509in}{2.002022in}}%
\pgfpathlineto{\pgfqpoint{5.521018in}{2.138969in}}%
\pgfpathlineto{\pgfqpoint{5.525527in}{2.028620in}}%
\pgfpathlineto{\pgfqpoint{5.530036in}{2.041998in}}%
\pgfpathlineto{\pgfqpoint{5.534545in}{1.992018in}}%
\pgfpathlineto{\pgfqpoint{5.534545in}{1.992018in}}%
\pgfusepath{stroke}%
\end{pgfscope}%
\begin{pgfscope}%
\pgfsetrectcap%
\pgfsetmiterjoin%
\pgfsetlinewidth{0.803000pt}%
\definecolor{currentstroke}{rgb}{0.000000,0.000000,0.000000}%
\pgfsetstrokecolor{currentstroke}%
\pgfsetdash{}{0pt}%
\pgfpathmoveto{\pgfqpoint{0.800000in}{0.528000in}}%
\pgfpathlineto{\pgfqpoint{0.800000in}{4.224000in}}%
\pgfusepath{stroke}%
\end{pgfscope}%
\begin{pgfscope}%
\pgfsetrectcap%
\pgfsetmiterjoin%
\pgfsetlinewidth{0.803000pt}%
\definecolor{currentstroke}{rgb}{0.000000,0.000000,0.000000}%
\pgfsetstrokecolor{currentstroke}%
\pgfsetdash{}{0pt}%
\pgfpathmoveto{\pgfqpoint{5.760000in}{0.528000in}}%
\pgfpathlineto{\pgfqpoint{5.760000in}{4.224000in}}%
\pgfusepath{stroke}%
\end{pgfscope}%
\begin{pgfscope}%
\pgfsetrectcap%
\pgfsetmiterjoin%
\pgfsetlinewidth{0.803000pt}%
\definecolor{currentstroke}{rgb}{0.000000,0.000000,0.000000}%
\pgfsetstrokecolor{currentstroke}%
\pgfsetdash{}{0pt}%
\pgfpathmoveto{\pgfqpoint{0.800000in}{0.528000in}}%
\pgfpathlineto{\pgfqpoint{5.760000in}{0.528000in}}%
\pgfusepath{stroke}%
\end{pgfscope}%
\begin{pgfscope}%
\pgfsetrectcap%
\pgfsetmiterjoin%
\pgfsetlinewidth{0.803000pt}%
\definecolor{currentstroke}{rgb}{0.000000,0.000000,0.000000}%
\pgfsetstrokecolor{currentstroke}%
\pgfsetdash{}{0pt}%
\pgfpathmoveto{\pgfqpoint{0.800000in}{4.224000in}}%
\pgfpathlineto{\pgfqpoint{5.760000in}{4.224000in}}%
\pgfusepath{stroke}%
\end{pgfscope}%
\begin{pgfscope}%
\definecolor{textcolor}{rgb}{0.000000,0.000000,0.000000}%
\pgfsetstrokecolor{textcolor}%
\pgfsetfillcolor{textcolor}%
\pgftext[x=3.280000in,y=4.307333in,,base]{\color{textcolor}\ttfamily\fontsize{12.000000}{14.400000}\selectfont Merge Sort Time vs Input size}%
\end{pgfscope}%
\begin{pgfscope}%
\pgfsetbuttcap%
\pgfsetmiterjoin%
\definecolor{currentfill}{rgb}{1.000000,1.000000,1.000000}%
\pgfsetfillcolor{currentfill}%
\pgfsetfillopacity{0.800000}%
\pgfsetlinewidth{1.003750pt}%
\definecolor{currentstroke}{rgb}{0.800000,0.800000,0.800000}%
\pgfsetstrokecolor{currentstroke}%
\pgfsetstrokeopacity{0.800000}%
\pgfsetdash{}{0pt}%
\pgfpathmoveto{\pgfqpoint{4.800243in}{3.907336in}}%
\pgfpathlineto{\pgfqpoint{5.662778in}{3.907336in}}%
\pgfpathquadraticcurveto{\pgfqpoint{5.690556in}{3.907336in}}{\pgfqpoint{5.690556in}{3.935114in}}%
\pgfpathlineto{\pgfqpoint{5.690556in}{4.126778in}}%
\pgfpathquadraticcurveto{\pgfqpoint{5.690556in}{4.154556in}}{\pgfqpoint{5.662778in}{4.154556in}}%
\pgfpathlineto{\pgfqpoint{4.800243in}{4.154556in}}%
\pgfpathquadraticcurveto{\pgfqpoint{4.772465in}{4.154556in}}{\pgfqpoint{4.772465in}{4.126778in}}%
\pgfpathlineto{\pgfqpoint{4.772465in}{3.935114in}}%
\pgfpathquadraticcurveto{\pgfqpoint{4.772465in}{3.907336in}}{\pgfqpoint{4.800243in}{3.907336in}}%
\pgfpathlineto{\pgfqpoint{4.800243in}{3.907336in}}%
\pgfpathclose%
\pgfusepath{stroke,fill}%
\end{pgfscope}%
\begin{pgfscope}%
\pgfsetrectcap%
\pgfsetroundjoin%
\pgfsetlinewidth{1.505625pt}%
\definecolor{currentstroke}{rgb}{0.000000,1.000000,0.498039}%
\pgfsetstrokecolor{currentstroke}%
\pgfsetdash{}{0pt}%
\pgfpathmoveto{\pgfqpoint{4.828020in}{4.041342in}}%
\pgfpathlineto{\pgfqpoint{4.966909in}{4.041342in}}%
\pgfpathlineto{\pgfqpoint{5.105798in}{4.041342in}}%
\pgfusepath{stroke}%
\end{pgfscope}%
\begin{pgfscope}%
\definecolor{textcolor}{rgb}{0.000000,0.000000,0.000000}%
\pgfsetstrokecolor{textcolor}%
\pgfsetfillcolor{textcolor}%
\pgftext[x=5.216909in,y=3.992731in,left,base]{\color{textcolor}\ttfamily\fontsize{10.000000}{12.000000}\selectfont Merge}%
\end{pgfscope}%
\end{pgfpicture}%
\makeatother%
\endgroup%

%% Creator: Matplotlib, PGF backend
%%
%% To include the figure in your LaTeX document, write
%%   \input{<filename>.pgf}
%%
%% Make sure the required packages are loaded in your preamble
%%   \usepackage{pgf}
%%
%% Also ensure that all the required font packages are loaded; for instance,
%% the lmodern package is sometimes necessary when using math font.
%%   \usepackage{lmodern}
%%
%% Figures using additional raster images can only be included by \input if
%% they are in the same directory as the main LaTeX file. For loading figures
%% from other directories you can use the `import` package
%%   \usepackage{import}
%%
%% and then include the figures with
%%   \import{<path to file>}{<filename>.pgf}
%%
%% Matplotlib used the following preamble
%%   \usepackage{fontspec}
%%   \setmainfont{DejaVuSerif.ttf}[Path=\detokenize{/home/dbk/.local/lib/python3.10/site-packages/matplotlib/mpl-data/fonts/ttf/}]
%%   \setsansfont{DejaVuSans.ttf}[Path=\detokenize{/home/dbk/.local/lib/python3.10/site-packages/matplotlib/mpl-data/fonts/ttf/}]
%%   \setmonofont{DejaVuSansMono.ttf}[Path=\detokenize{/home/dbk/.local/lib/python3.10/site-packages/matplotlib/mpl-data/fonts/ttf/}]
%%
\begingroup%
\makeatletter%
\begin{pgfpicture}%
\pgfpathrectangle{\pgfpointorigin}{\pgfqpoint{6.400000in}{4.800000in}}%
\pgfusepath{use as bounding box, clip}%
\begin{pgfscope}%
\pgfsetbuttcap%
\pgfsetmiterjoin%
\definecolor{currentfill}{rgb}{1.000000,1.000000,1.000000}%
\pgfsetfillcolor{currentfill}%
\pgfsetlinewidth{0.000000pt}%
\definecolor{currentstroke}{rgb}{1.000000,1.000000,1.000000}%
\pgfsetstrokecolor{currentstroke}%
\pgfsetdash{}{0pt}%
\pgfpathmoveto{\pgfqpoint{0.000000in}{0.000000in}}%
\pgfpathlineto{\pgfqpoint{6.400000in}{0.000000in}}%
\pgfpathlineto{\pgfqpoint{6.400000in}{4.800000in}}%
\pgfpathlineto{\pgfqpoint{0.000000in}{4.800000in}}%
\pgfpathlineto{\pgfqpoint{0.000000in}{0.000000in}}%
\pgfpathclose%
\pgfusepath{fill}%
\end{pgfscope}%
\begin{pgfscope}%
\pgfsetbuttcap%
\pgfsetmiterjoin%
\definecolor{currentfill}{rgb}{1.000000,1.000000,1.000000}%
\pgfsetfillcolor{currentfill}%
\pgfsetlinewidth{0.000000pt}%
\definecolor{currentstroke}{rgb}{0.000000,0.000000,0.000000}%
\pgfsetstrokecolor{currentstroke}%
\pgfsetstrokeopacity{0.000000}%
\pgfsetdash{}{0pt}%
\pgfpathmoveto{\pgfqpoint{0.800000in}{0.528000in}}%
\pgfpathlineto{\pgfqpoint{5.760000in}{0.528000in}}%
\pgfpathlineto{\pgfqpoint{5.760000in}{4.224000in}}%
\pgfpathlineto{\pgfqpoint{0.800000in}{4.224000in}}%
\pgfpathlineto{\pgfqpoint{0.800000in}{0.528000in}}%
\pgfpathclose%
\pgfusepath{fill}%
\end{pgfscope}%
\begin{pgfscope}%
\pgfsetbuttcap%
\pgfsetroundjoin%
\definecolor{currentfill}{rgb}{0.000000,0.000000,0.000000}%
\pgfsetfillcolor{currentfill}%
\pgfsetlinewidth{0.803000pt}%
\definecolor{currentstroke}{rgb}{0.000000,0.000000,0.000000}%
\pgfsetstrokecolor{currentstroke}%
\pgfsetdash{}{0pt}%
\pgfsys@defobject{currentmarker}{\pgfqpoint{0.000000in}{-0.048611in}}{\pgfqpoint{0.000000in}{0.000000in}}{%
\pgfpathmoveto{\pgfqpoint{0.000000in}{0.000000in}}%
\pgfpathlineto{\pgfqpoint{0.000000in}{-0.048611in}}%
\pgfusepath{stroke,fill}%
}%
\begin{pgfscope}%
\pgfsys@transformshift{1.020945in}{0.528000in}%
\pgfsys@useobject{currentmarker}{}%
\end{pgfscope}%
\end{pgfscope}%
\begin{pgfscope}%
\definecolor{textcolor}{rgb}{0.000000,0.000000,0.000000}%
\pgfsetstrokecolor{textcolor}%
\pgfsetfillcolor{textcolor}%
\pgftext[x=1.020945in,y=0.430778in,,top]{\color{textcolor}\ttfamily\fontsize{10.000000}{12.000000}\selectfont 0}%
\end{pgfscope}%
\begin{pgfscope}%
\pgfsetbuttcap%
\pgfsetroundjoin%
\definecolor{currentfill}{rgb}{0.000000,0.000000,0.000000}%
\pgfsetfillcolor{currentfill}%
\pgfsetlinewidth{0.803000pt}%
\definecolor{currentstroke}{rgb}{0.000000,0.000000,0.000000}%
\pgfsetstrokecolor{currentstroke}%
\pgfsetdash{}{0pt}%
\pgfsys@defobject{currentmarker}{\pgfqpoint{0.000000in}{-0.048611in}}{\pgfqpoint{0.000000in}{0.000000in}}{%
\pgfpathmoveto{\pgfqpoint{0.000000in}{0.000000in}}%
\pgfpathlineto{\pgfqpoint{0.000000in}{-0.048611in}}%
\pgfusepath{stroke,fill}%
}%
\begin{pgfscope}%
\pgfsys@transformshift{1.922764in}{0.528000in}%
\pgfsys@useobject{currentmarker}{}%
\end{pgfscope}%
\end{pgfscope}%
\begin{pgfscope}%
\definecolor{textcolor}{rgb}{0.000000,0.000000,0.000000}%
\pgfsetstrokecolor{textcolor}%
\pgfsetfillcolor{textcolor}%
\pgftext[x=1.922764in,y=0.430778in,,top]{\color{textcolor}\ttfamily\fontsize{10.000000}{12.000000}\selectfont 200}%
\end{pgfscope}%
\begin{pgfscope}%
\pgfsetbuttcap%
\pgfsetroundjoin%
\definecolor{currentfill}{rgb}{0.000000,0.000000,0.000000}%
\pgfsetfillcolor{currentfill}%
\pgfsetlinewidth{0.803000pt}%
\definecolor{currentstroke}{rgb}{0.000000,0.000000,0.000000}%
\pgfsetstrokecolor{currentstroke}%
\pgfsetdash{}{0pt}%
\pgfsys@defobject{currentmarker}{\pgfqpoint{0.000000in}{-0.048611in}}{\pgfqpoint{0.000000in}{0.000000in}}{%
\pgfpathmoveto{\pgfqpoint{0.000000in}{0.000000in}}%
\pgfpathlineto{\pgfqpoint{0.000000in}{-0.048611in}}%
\pgfusepath{stroke,fill}%
}%
\begin{pgfscope}%
\pgfsys@transformshift{2.824582in}{0.528000in}%
\pgfsys@useobject{currentmarker}{}%
\end{pgfscope}%
\end{pgfscope}%
\begin{pgfscope}%
\definecolor{textcolor}{rgb}{0.000000,0.000000,0.000000}%
\pgfsetstrokecolor{textcolor}%
\pgfsetfillcolor{textcolor}%
\pgftext[x=2.824582in,y=0.430778in,,top]{\color{textcolor}\ttfamily\fontsize{10.000000}{12.000000}\selectfont 400}%
\end{pgfscope}%
\begin{pgfscope}%
\pgfsetbuttcap%
\pgfsetroundjoin%
\definecolor{currentfill}{rgb}{0.000000,0.000000,0.000000}%
\pgfsetfillcolor{currentfill}%
\pgfsetlinewidth{0.803000pt}%
\definecolor{currentstroke}{rgb}{0.000000,0.000000,0.000000}%
\pgfsetstrokecolor{currentstroke}%
\pgfsetdash{}{0pt}%
\pgfsys@defobject{currentmarker}{\pgfqpoint{0.000000in}{-0.048611in}}{\pgfqpoint{0.000000in}{0.000000in}}{%
\pgfpathmoveto{\pgfqpoint{0.000000in}{0.000000in}}%
\pgfpathlineto{\pgfqpoint{0.000000in}{-0.048611in}}%
\pgfusepath{stroke,fill}%
}%
\begin{pgfscope}%
\pgfsys@transformshift{3.726400in}{0.528000in}%
\pgfsys@useobject{currentmarker}{}%
\end{pgfscope}%
\end{pgfscope}%
\begin{pgfscope}%
\definecolor{textcolor}{rgb}{0.000000,0.000000,0.000000}%
\pgfsetstrokecolor{textcolor}%
\pgfsetfillcolor{textcolor}%
\pgftext[x=3.726400in,y=0.430778in,,top]{\color{textcolor}\ttfamily\fontsize{10.000000}{12.000000}\selectfont 600}%
\end{pgfscope}%
\begin{pgfscope}%
\pgfsetbuttcap%
\pgfsetroundjoin%
\definecolor{currentfill}{rgb}{0.000000,0.000000,0.000000}%
\pgfsetfillcolor{currentfill}%
\pgfsetlinewidth{0.803000pt}%
\definecolor{currentstroke}{rgb}{0.000000,0.000000,0.000000}%
\pgfsetstrokecolor{currentstroke}%
\pgfsetdash{}{0pt}%
\pgfsys@defobject{currentmarker}{\pgfqpoint{0.000000in}{-0.048611in}}{\pgfqpoint{0.000000in}{0.000000in}}{%
\pgfpathmoveto{\pgfqpoint{0.000000in}{0.000000in}}%
\pgfpathlineto{\pgfqpoint{0.000000in}{-0.048611in}}%
\pgfusepath{stroke,fill}%
}%
\begin{pgfscope}%
\pgfsys@transformshift{4.628218in}{0.528000in}%
\pgfsys@useobject{currentmarker}{}%
\end{pgfscope}%
\end{pgfscope}%
\begin{pgfscope}%
\definecolor{textcolor}{rgb}{0.000000,0.000000,0.000000}%
\pgfsetstrokecolor{textcolor}%
\pgfsetfillcolor{textcolor}%
\pgftext[x=4.628218in,y=0.430778in,,top]{\color{textcolor}\ttfamily\fontsize{10.000000}{12.000000}\selectfont 800}%
\end{pgfscope}%
\begin{pgfscope}%
\pgfsetbuttcap%
\pgfsetroundjoin%
\definecolor{currentfill}{rgb}{0.000000,0.000000,0.000000}%
\pgfsetfillcolor{currentfill}%
\pgfsetlinewidth{0.803000pt}%
\definecolor{currentstroke}{rgb}{0.000000,0.000000,0.000000}%
\pgfsetstrokecolor{currentstroke}%
\pgfsetdash{}{0pt}%
\pgfsys@defobject{currentmarker}{\pgfqpoint{0.000000in}{-0.048611in}}{\pgfqpoint{0.000000in}{0.000000in}}{%
\pgfpathmoveto{\pgfqpoint{0.000000in}{0.000000in}}%
\pgfpathlineto{\pgfqpoint{0.000000in}{-0.048611in}}%
\pgfusepath{stroke,fill}%
}%
\begin{pgfscope}%
\pgfsys@transformshift{5.530036in}{0.528000in}%
\pgfsys@useobject{currentmarker}{}%
\end{pgfscope}%
\end{pgfscope}%
\begin{pgfscope}%
\definecolor{textcolor}{rgb}{0.000000,0.000000,0.000000}%
\pgfsetstrokecolor{textcolor}%
\pgfsetfillcolor{textcolor}%
\pgftext[x=5.530036in,y=0.430778in,,top]{\color{textcolor}\ttfamily\fontsize{10.000000}{12.000000}\selectfont 1000}%
\end{pgfscope}%
\begin{pgfscope}%
\definecolor{textcolor}{rgb}{0.000000,0.000000,0.000000}%
\pgfsetstrokecolor{textcolor}%
\pgfsetfillcolor{textcolor}%
\pgftext[x=3.280000in,y=0.240063in,,top]{\color{textcolor}\ttfamily\fontsize{10.000000}{12.000000}\selectfont Size of Array}%
\end{pgfscope}%
\begin{pgfscope}%
\pgfsetbuttcap%
\pgfsetroundjoin%
\definecolor{currentfill}{rgb}{0.000000,0.000000,0.000000}%
\pgfsetfillcolor{currentfill}%
\pgfsetlinewidth{0.803000pt}%
\definecolor{currentstroke}{rgb}{0.000000,0.000000,0.000000}%
\pgfsetstrokecolor{currentstroke}%
\pgfsetdash{}{0pt}%
\pgfsys@defobject{currentmarker}{\pgfqpoint{-0.048611in}{0.000000in}}{\pgfqpoint{-0.000000in}{0.000000in}}{%
\pgfpathmoveto{\pgfqpoint{-0.000000in}{0.000000in}}%
\pgfpathlineto{\pgfqpoint{-0.048611in}{0.000000in}}%
\pgfusepath{stroke,fill}%
}%
\begin{pgfscope}%
\pgfsys@transformshift{0.800000in}{1.230545in}%
\pgfsys@useobject{currentmarker}{}%
\end{pgfscope}%
\end{pgfscope}%
\begin{pgfscope}%
\definecolor{textcolor}{rgb}{0.000000,0.000000,0.000000}%
\pgfsetstrokecolor{textcolor}%
\pgfsetfillcolor{textcolor}%
\pgftext[x=0.368305in, y=1.177411in, left, base]{\color{textcolor}\ttfamily\fontsize{10.000000}{12.000000}\selectfont 4000}%
\end{pgfscope}%
\begin{pgfscope}%
\pgfsetbuttcap%
\pgfsetroundjoin%
\definecolor{currentfill}{rgb}{0.000000,0.000000,0.000000}%
\pgfsetfillcolor{currentfill}%
\pgfsetlinewidth{0.803000pt}%
\definecolor{currentstroke}{rgb}{0.000000,0.000000,0.000000}%
\pgfsetstrokecolor{currentstroke}%
\pgfsetdash{}{0pt}%
\pgfsys@defobject{currentmarker}{\pgfqpoint{-0.048611in}{0.000000in}}{\pgfqpoint{-0.000000in}{0.000000in}}{%
\pgfpathmoveto{\pgfqpoint{-0.000000in}{0.000000in}}%
\pgfpathlineto{\pgfqpoint{-0.048611in}{0.000000in}}%
\pgfusepath{stroke,fill}%
}%
\begin{pgfscope}%
\pgfsys@transformshift{0.800000in}{1.979209in}%
\pgfsys@useobject{currentmarker}{}%
\end{pgfscope}%
\end{pgfscope}%
\begin{pgfscope}%
\definecolor{textcolor}{rgb}{0.000000,0.000000,0.000000}%
\pgfsetstrokecolor{textcolor}%
\pgfsetfillcolor{textcolor}%
\pgftext[x=0.368305in, y=1.926074in, left, base]{\color{textcolor}\ttfamily\fontsize{10.000000}{12.000000}\selectfont 6000}%
\end{pgfscope}%
\begin{pgfscope}%
\pgfsetbuttcap%
\pgfsetroundjoin%
\definecolor{currentfill}{rgb}{0.000000,0.000000,0.000000}%
\pgfsetfillcolor{currentfill}%
\pgfsetlinewidth{0.803000pt}%
\definecolor{currentstroke}{rgb}{0.000000,0.000000,0.000000}%
\pgfsetstrokecolor{currentstroke}%
\pgfsetdash{}{0pt}%
\pgfsys@defobject{currentmarker}{\pgfqpoint{-0.048611in}{0.000000in}}{\pgfqpoint{-0.000000in}{0.000000in}}{%
\pgfpathmoveto{\pgfqpoint{-0.000000in}{0.000000in}}%
\pgfpathlineto{\pgfqpoint{-0.048611in}{0.000000in}}%
\pgfusepath{stroke,fill}%
}%
\begin{pgfscope}%
\pgfsys@transformshift{0.800000in}{2.727872in}%
\pgfsys@useobject{currentmarker}{}%
\end{pgfscope}%
\end{pgfscope}%
\begin{pgfscope}%
\definecolor{textcolor}{rgb}{0.000000,0.000000,0.000000}%
\pgfsetstrokecolor{textcolor}%
\pgfsetfillcolor{textcolor}%
\pgftext[x=0.368305in, y=2.674737in, left, base]{\color{textcolor}\ttfamily\fontsize{10.000000}{12.000000}\selectfont 8000}%
\end{pgfscope}%
\begin{pgfscope}%
\pgfsetbuttcap%
\pgfsetroundjoin%
\definecolor{currentfill}{rgb}{0.000000,0.000000,0.000000}%
\pgfsetfillcolor{currentfill}%
\pgfsetlinewidth{0.803000pt}%
\definecolor{currentstroke}{rgb}{0.000000,0.000000,0.000000}%
\pgfsetstrokecolor{currentstroke}%
\pgfsetdash{}{0pt}%
\pgfsys@defobject{currentmarker}{\pgfqpoint{-0.048611in}{0.000000in}}{\pgfqpoint{-0.000000in}{0.000000in}}{%
\pgfpathmoveto{\pgfqpoint{-0.000000in}{0.000000in}}%
\pgfpathlineto{\pgfqpoint{-0.048611in}{0.000000in}}%
\pgfusepath{stroke,fill}%
}%
\begin{pgfscope}%
\pgfsys@transformshift{0.800000in}{3.476535in}%
\pgfsys@useobject{currentmarker}{}%
\end{pgfscope}%
\end{pgfscope}%
\begin{pgfscope}%
\definecolor{textcolor}{rgb}{0.000000,0.000000,0.000000}%
\pgfsetstrokecolor{textcolor}%
\pgfsetfillcolor{textcolor}%
\pgftext[x=0.284687in, y=3.423400in, left, base]{\color{textcolor}\ttfamily\fontsize{10.000000}{12.000000}\selectfont 10000}%
\end{pgfscope}%
\begin{pgfscope}%
\definecolor{textcolor}{rgb}{0.000000,0.000000,0.000000}%
\pgfsetstrokecolor{textcolor}%
\pgfsetfillcolor{textcolor}%
\pgftext[x=0.229131in,y=2.376000in,,bottom,rotate=90.000000]{\color{textcolor}\ttfamily\fontsize{10.000000}{12.000000}\selectfont Memory}%
\end{pgfscope}%
\begin{pgfscope}%
\pgfpathrectangle{\pgfqpoint{0.800000in}{0.528000in}}{\pgfqpoint{4.960000in}{3.696000in}}%
\pgfusepath{clip}%
\pgfsetrectcap%
\pgfsetroundjoin%
\pgfsetlinewidth{1.505625pt}%
\definecolor{currentstroke}{rgb}{0.000000,1.000000,0.498039}%
\pgfsetstrokecolor{currentstroke}%
\pgfsetdash{}{0pt}%
\pgfpathmoveto{\pgfqpoint{1.025455in}{2.109476in}}%
\pgfpathlineto{\pgfqpoint{1.029964in}{1.217070in}}%
\pgfpathlineto{\pgfqpoint{1.043491in}{0.696000in}}%
\pgfpathlineto{\pgfqpoint{1.048000in}{0.905251in}}%
\pgfpathlineto{\pgfqpoint{1.052509in}{0.701989in}}%
\pgfpathlineto{\pgfqpoint{1.066036in}{0.710973in}}%
\pgfpathlineto{\pgfqpoint{1.070545in}{0.890652in}}%
\pgfpathlineto{\pgfqpoint{1.075055in}{0.716963in}}%
\pgfpathlineto{\pgfqpoint{1.079564in}{0.896642in}}%
\pgfpathlineto{\pgfqpoint{1.084073in}{0.722952in}}%
\pgfpathlineto{\pgfqpoint{1.088582in}{0.902631in}}%
\pgfpathlineto{\pgfqpoint{1.093091in}{0.728941in}}%
\pgfpathlineto{\pgfqpoint{1.111127in}{0.740920in}}%
\pgfpathlineto{\pgfqpoint{1.115636in}{0.920599in}}%
\pgfpathlineto{\pgfqpoint{1.120145in}{0.746909in}}%
\pgfpathlineto{\pgfqpoint{1.133673in}{0.755893in}}%
\pgfpathlineto{\pgfqpoint{1.138182in}{0.935572in}}%
\pgfpathlineto{\pgfqpoint{1.142691in}{0.761882in}}%
\pgfpathlineto{\pgfqpoint{1.151709in}{0.767872in}}%
\pgfpathlineto{\pgfqpoint{1.156218in}{0.854717in}}%
\pgfpathlineto{\pgfqpoint{1.169745in}{0.863701in}}%
\pgfpathlineto{\pgfqpoint{1.174255in}{0.877176in}}%
\pgfpathlineto{\pgfqpoint{1.336582in}{0.984984in}}%
\pgfpathlineto{\pgfqpoint{1.341091in}{1.341348in}}%
\pgfpathlineto{\pgfqpoint{1.345600in}{0.990973in}}%
\pgfpathlineto{\pgfqpoint{1.377164in}{1.011936in}}%
\pgfpathlineto{\pgfqpoint{1.381673in}{1.191615in}}%
\pgfpathlineto{\pgfqpoint{1.386182in}{1.017925in}}%
\pgfpathlineto{\pgfqpoint{1.598109in}{1.158674in}}%
\pgfpathlineto{\pgfqpoint{1.602618in}{1.338353in}}%
\pgfpathlineto{\pgfqpoint{1.607127in}{1.164663in}}%
\pgfpathlineto{\pgfqpoint{1.728873in}{1.245519in}}%
\pgfpathlineto{\pgfqpoint{1.733382in}{1.332364in}}%
\pgfpathlineto{\pgfqpoint{1.796509in}{1.374289in}}%
\pgfpathlineto{\pgfqpoint{1.801018in}{1.553968in}}%
\pgfpathlineto{\pgfqpoint{1.805527in}{1.380278in}}%
\pgfpathlineto{\pgfqpoint{2.089600in}{1.568941in}}%
\pgfpathlineto{\pgfqpoint{2.094109in}{1.748620in}}%
\pgfpathlineto{\pgfqpoint{2.098618in}{1.574930in}}%
\pgfpathlineto{\pgfqpoint{2.306036in}{1.712684in}}%
\pgfpathlineto{\pgfqpoint{2.310545in}{1.726160in}}%
\pgfpathlineto{\pgfqpoint{2.315055in}{1.718674in}}%
\pgfpathlineto{\pgfqpoint{2.324073in}{1.724663in}}%
\pgfpathlineto{\pgfqpoint{2.328582in}{1.738139in}}%
\pgfpathlineto{\pgfqpoint{2.333091in}{1.730652in}}%
\pgfpathlineto{\pgfqpoint{2.342109in}{1.736642in}}%
\pgfpathlineto{\pgfqpoint{2.346618in}{1.750118in}}%
\pgfpathlineto{\pgfqpoint{2.360145in}{1.759102in}}%
\pgfpathlineto{\pgfqpoint{2.364655in}{1.751615in}}%
\pgfpathlineto{\pgfqpoint{2.369164in}{1.757604in}}%
\pgfpathlineto{\pgfqpoint{2.373673in}{1.768086in}}%
\pgfpathlineto{\pgfqpoint{2.378182in}{1.760599in}}%
\pgfpathlineto{\pgfqpoint{2.382691in}{1.774075in}}%
\pgfpathlineto{\pgfqpoint{2.387200in}{1.777070in}}%
\pgfpathlineto{\pgfqpoint{2.391709in}{1.769583in}}%
\pgfpathlineto{\pgfqpoint{2.396218in}{1.783059in}}%
\pgfpathlineto{\pgfqpoint{2.414255in}{1.795037in}}%
\pgfpathlineto{\pgfqpoint{2.418764in}{1.787551in}}%
\pgfpathlineto{\pgfqpoint{2.423273in}{1.801027in}}%
\pgfpathlineto{\pgfqpoint{2.463855in}{1.827979in}}%
\pgfpathlineto{\pgfqpoint{2.468364in}{1.820492in}}%
\pgfpathlineto{\pgfqpoint{2.472873in}{1.833968in}}%
\pgfpathlineto{\pgfqpoint{2.563055in}{1.893861in}}%
\pgfpathlineto{\pgfqpoint{2.567564in}{2.073540in}}%
\pgfpathlineto{\pgfqpoint{2.572073in}{1.899850in}}%
\pgfpathlineto{\pgfqpoint{2.693818in}{1.980706in}}%
\pgfpathlineto{\pgfqpoint{2.698327in}{2.160385in}}%
\pgfpathlineto{\pgfqpoint{2.702836in}{1.986695in}}%
\pgfpathlineto{\pgfqpoint{2.883200in}{2.106481in}}%
\pgfpathlineto{\pgfqpoint{2.887709in}{2.193326in}}%
\pgfpathlineto{\pgfqpoint{2.892218in}{2.206802in}}%
\pgfpathlineto{\pgfqpoint{4.168291in}{3.054289in}}%
\pgfpathlineto{\pgfqpoint{4.172800in}{3.233968in}}%
\pgfpathlineto{\pgfqpoint{4.177309in}{3.060278in}}%
\pgfpathlineto{\pgfqpoint{4.695855in}{3.404663in}}%
\pgfpathlineto{\pgfqpoint{4.700364in}{3.584342in}}%
\pgfpathlineto{\pgfqpoint{4.704873in}{3.410652in}}%
\pgfpathlineto{\pgfqpoint{4.898764in}{3.539422in}}%
\pgfpathlineto{\pgfqpoint{4.903273in}{3.719102in}}%
\pgfpathlineto{\pgfqpoint{4.907782in}{3.545412in}}%
\pgfpathlineto{\pgfqpoint{5.191855in}{3.734075in}}%
\pgfpathlineto{\pgfqpoint{5.196364in}{3.820920in}}%
\pgfpathlineto{\pgfqpoint{5.200873in}{3.823914in}}%
\pgfpathlineto{\pgfqpoint{5.205382in}{3.837390in}}%
\pgfpathlineto{\pgfqpoint{5.534545in}{4.056000in}}%
\pgfpathlineto{\pgfqpoint{5.534545in}{4.056000in}}%
\pgfusepath{stroke}%
\end{pgfscope}%
\begin{pgfscope}%
\pgfsetrectcap%
\pgfsetmiterjoin%
\pgfsetlinewidth{0.803000pt}%
\definecolor{currentstroke}{rgb}{0.000000,0.000000,0.000000}%
\pgfsetstrokecolor{currentstroke}%
\pgfsetdash{}{0pt}%
\pgfpathmoveto{\pgfqpoint{0.800000in}{0.528000in}}%
\pgfpathlineto{\pgfqpoint{0.800000in}{4.224000in}}%
\pgfusepath{stroke}%
\end{pgfscope}%
\begin{pgfscope}%
\pgfsetrectcap%
\pgfsetmiterjoin%
\pgfsetlinewidth{0.803000pt}%
\definecolor{currentstroke}{rgb}{0.000000,0.000000,0.000000}%
\pgfsetstrokecolor{currentstroke}%
\pgfsetdash{}{0pt}%
\pgfpathmoveto{\pgfqpoint{5.760000in}{0.528000in}}%
\pgfpathlineto{\pgfqpoint{5.760000in}{4.224000in}}%
\pgfusepath{stroke}%
\end{pgfscope}%
\begin{pgfscope}%
\pgfsetrectcap%
\pgfsetmiterjoin%
\pgfsetlinewidth{0.803000pt}%
\definecolor{currentstroke}{rgb}{0.000000,0.000000,0.000000}%
\pgfsetstrokecolor{currentstroke}%
\pgfsetdash{}{0pt}%
\pgfpathmoveto{\pgfqpoint{0.800000in}{0.528000in}}%
\pgfpathlineto{\pgfqpoint{5.760000in}{0.528000in}}%
\pgfusepath{stroke}%
\end{pgfscope}%
\begin{pgfscope}%
\pgfsetrectcap%
\pgfsetmiterjoin%
\pgfsetlinewidth{0.803000pt}%
\definecolor{currentstroke}{rgb}{0.000000,0.000000,0.000000}%
\pgfsetstrokecolor{currentstroke}%
\pgfsetdash{}{0pt}%
\pgfpathmoveto{\pgfqpoint{0.800000in}{4.224000in}}%
\pgfpathlineto{\pgfqpoint{5.760000in}{4.224000in}}%
\pgfusepath{stroke}%
\end{pgfscope}%
\begin{pgfscope}%
\definecolor{textcolor}{rgb}{0.000000,0.000000,0.000000}%
\pgfsetstrokecolor{textcolor}%
\pgfsetfillcolor{textcolor}%
\pgftext[x=3.280000in,y=4.307333in,,base]{\color{textcolor}\ttfamily\fontsize{12.000000}{14.400000}\selectfont Merge Sort Memory vs Input size}%
\end{pgfscope}%
\begin{pgfscope}%
\pgfsetbuttcap%
\pgfsetmiterjoin%
\definecolor{currentfill}{rgb}{1.000000,1.000000,1.000000}%
\pgfsetfillcolor{currentfill}%
\pgfsetfillopacity{0.800000}%
\pgfsetlinewidth{1.003750pt}%
\definecolor{currentstroke}{rgb}{0.800000,0.800000,0.800000}%
\pgfsetstrokecolor{currentstroke}%
\pgfsetstrokeopacity{0.800000}%
\pgfsetdash{}{0pt}%
\pgfpathmoveto{\pgfqpoint{0.897222in}{3.907336in}}%
\pgfpathlineto{\pgfqpoint{1.759758in}{3.907336in}}%
\pgfpathquadraticcurveto{\pgfqpoint{1.787535in}{3.907336in}}{\pgfqpoint{1.787535in}{3.935114in}}%
\pgfpathlineto{\pgfqpoint{1.787535in}{4.126778in}}%
\pgfpathquadraticcurveto{\pgfqpoint{1.787535in}{4.154556in}}{\pgfqpoint{1.759758in}{4.154556in}}%
\pgfpathlineto{\pgfqpoint{0.897222in}{4.154556in}}%
\pgfpathquadraticcurveto{\pgfqpoint{0.869444in}{4.154556in}}{\pgfqpoint{0.869444in}{4.126778in}}%
\pgfpathlineto{\pgfqpoint{0.869444in}{3.935114in}}%
\pgfpathquadraticcurveto{\pgfqpoint{0.869444in}{3.907336in}}{\pgfqpoint{0.897222in}{3.907336in}}%
\pgfpathlineto{\pgfqpoint{0.897222in}{3.907336in}}%
\pgfpathclose%
\pgfusepath{stroke,fill}%
\end{pgfscope}%
\begin{pgfscope}%
\pgfsetrectcap%
\pgfsetroundjoin%
\pgfsetlinewidth{1.505625pt}%
\definecolor{currentstroke}{rgb}{0.000000,1.000000,0.498039}%
\pgfsetstrokecolor{currentstroke}%
\pgfsetdash{}{0pt}%
\pgfpathmoveto{\pgfqpoint{0.925000in}{4.041342in}}%
\pgfpathlineto{\pgfqpoint{1.063889in}{4.041342in}}%
\pgfpathlineto{\pgfqpoint{1.202778in}{4.041342in}}%
\pgfusepath{stroke}%
\end{pgfscope}%
\begin{pgfscope}%
\definecolor{textcolor}{rgb}{0.000000,0.000000,0.000000}%
\pgfsetstrokecolor{textcolor}%
\pgfsetfillcolor{textcolor}%
\pgftext[x=1.313889in,y=3.992731in,left,base]{\color{textcolor}\ttfamily\fontsize{10.000000}{12.000000}\selectfont Merge}%
\end{pgfscope}%
\end{pgfpicture}%
\makeatother%
\endgroup%

%% Creator: Matplotlib, PGF backend
%%
%% To include the figure in your LaTeX document, write
%%   \input{<filename>.pgf}
%%
%% Make sure the required packages are loaded in your preamble
%%   \usepackage{pgf}
%%
%% Also ensure that all the required font packages are loaded; for instance,
%% the lmodern package is sometimes necessary when using math font.
%%   \usepackage{lmodern}
%%
%% Figures using additional raster images can only be included by \input if
%% they are in the same directory as the main LaTeX file. For loading figures
%% from other directories you can use the `import` package
%%   \usepackage{import}
%%
%% and then include the figures with
%%   \import{<path to file>}{<filename>.pgf}
%%
%% Matplotlib used the following preamble
%%   \usepackage{fontspec}
%%   \setmainfont{DejaVuSerif.ttf}[Path=\detokenize{/home/dbk/.local/lib/python3.10/site-packages/matplotlib/mpl-data/fonts/ttf/}]
%%   \setsansfont{DejaVuSans.ttf}[Path=\detokenize{/home/dbk/.local/lib/python3.10/site-packages/matplotlib/mpl-data/fonts/ttf/}]
%%   \setmonofont{DejaVuSansMono.ttf}[Path=\detokenize{/home/dbk/.local/lib/python3.10/site-packages/matplotlib/mpl-data/fonts/ttf/}]
%%
\begingroup%
\makeatletter%
\begin{pgfpicture}%
\pgfpathrectangle{\pgfpointorigin}{\pgfqpoint{6.400000in}{4.800000in}}%
\pgfusepath{use as bounding box, clip}%
\begin{pgfscope}%
\pgfsetbuttcap%
\pgfsetmiterjoin%
\definecolor{currentfill}{rgb}{1.000000,1.000000,1.000000}%
\pgfsetfillcolor{currentfill}%
\pgfsetlinewidth{0.000000pt}%
\definecolor{currentstroke}{rgb}{1.000000,1.000000,1.000000}%
\pgfsetstrokecolor{currentstroke}%
\pgfsetdash{}{0pt}%
\pgfpathmoveto{\pgfqpoint{0.000000in}{0.000000in}}%
\pgfpathlineto{\pgfqpoint{6.400000in}{0.000000in}}%
\pgfpathlineto{\pgfqpoint{6.400000in}{4.800000in}}%
\pgfpathlineto{\pgfqpoint{0.000000in}{4.800000in}}%
\pgfpathlineto{\pgfqpoint{0.000000in}{0.000000in}}%
\pgfpathclose%
\pgfusepath{fill}%
\end{pgfscope}%
\begin{pgfscope}%
\pgfsetbuttcap%
\pgfsetmiterjoin%
\definecolor{currentfill}{rgb}{1.000000,1.000000,1.000000}%
\pgfsetfillcolor{currentfill}%
\pgfsetlinewidth{0.000000pt}%
\definecolor{currentstroke}{rgb}{0.000000,0.000000,0.000000}%
\pgfsetstrokecolor{currentstroke}%
\pgfsetstrokeopacity{0.000000}%
\pgfsetdash{}{0pt}%
\pgfpathmoveto{\pgfqpoint{0.800000in}{0.528000in}}%
\pgfpathlineto{\pgfqpoint{5.760000in}{0.528000in}}%
\pgfpathlineto{\pgfqpoint{5.760000in}{4.224000in}}%
\pgfpathlineto{\pgfqpoint{0.800000in}{4.224000in}}%
\pgfpathlineto{\pgfqpoint{0.800000in}{0.528000in}}%
\pgfpathclose%
\pgfusepath{fill}%
\end{pgfscope}%
\begin{pgfscope}%
\pgfsetbuttcap%
\pgfsetroundjoin%
\definecolor{currentfill}{rgb}{0.000000,0.000000,0.000000}%
\pgfsetfillcolor{currentfill}%
\pgfsetlinewidth{0.803000pt}%
\definecolor{currentstroke}{rgb}{0.000000,0.000000,0.000000}%
\pgfsetstrokecolor{currentstroke}%
\pgfsetdash{}{0pt}%
\pgfsys@defobject{currentmarker}{\pgfqpoint{0.000000in}{-0.048611in}}{\pgfqpoint{0.000000in}{0.000000in}}{%
\pgfpathmoveto{\pgfqpoint{0.000000in}{0.000000in}}%
\pgfpathlineto{\pgfqpoint{0.000000in}{-0.048611in}}%
\pgfusepath{stroke,fill}%
}%
\begin{pgfscope}%
\pgfsys@transformshift{1.020945in}{0.528000in}%
\pgfsys@useobject{currentmarker}{}%
\end{pgfscope}%
\end{pgfscope}%
\begin{pgfscope}%
\definecolor{textcolor}{rgb}{0.000000,0.000000,0.000000}%
\pgfsetstrokecolor{textcolor}%
\pgfsetfillcolor{textcolor}%
\pgftext[x=1.020945in,y=0.430778in,,top]{\color{textcolor}\ttfamily\fontsize{10.000000}{12.000000}\selectfont 0}%
\end{pgfscope}%
\begin{pgfscope}%
\pgfsetbuttcap%
\pgfsetroundjoin%
\definecolor{currentfill}{rgb}{0.000000,0.000000,0.000000}%
\pgfsetfillcolor{currentfill}%
\pgfsetlinewidth{0.803000pt}%
\definecolor{currentstroke}{rgb}{0.000000,0.000000,0.000000}%
\pgfsetstrokecolor{currentstroke}%
\pgfsetdash{}{0pt}%
\pgfsys@defobject{currentmarker}{\pgfqpoint{0.000000in}{-0.048611in}}{\pgfqpoint{0.000000in}{0.000000in}}{%
\pgfpathmoveto{\pgfqpoint{0.000000in}{0.000000in}}%
\pgfpathlineto{\pgfqpoint{0.000000in}{-0.048611in}}%
\pgfusepath{stroke,fill}%
}%
\begin{pgfscope}%
\pgfsys@transformshift{1.922764in}{0.528000in}%
\pgfsys@useobject{currentmarker}{}%
\end{pgfscope}%
\end{pgfscope}%
\begin{pgfscope}%
\definecolor{textcolor}{rgb}{0.000000,0.000000,0.000000}%
\pgfsetstrokecolor{textcolor}%
\pgfsetfillcolor{textcolor}%
\pgftext[x=1.922764in,y=0.430778in,,top]{\color{textcolor}\ttfamily\fontsize{10.000000}{12.000000}\selectfont 200}%
\end{pgfscope}%
\begin{pgfscope}%
\pgfsetbuttcap%
\pgfsetroundjoin%
\definecolor{currentfill}{rgb}{0.000000,0.000000,0.000000}%
\pgfsetfillcolor{currentfill}%
\pgfsetlinewidth{0.803000pt}%
\definecolor{currentstroke}{rgb}{0.000000,0.000000,0.000000}%
\pgfsetstrokecolor{currentstroke}%
\pgfsetdash{}{0pt}%
\pgfsys@defobject{currentmarker}{\pgfqpoint{0.000000in}{-0.048611in}}{\pgfqpoint{0.000000in}{0.000000in}}{%
\pgfpathmoveto{\pgfqpoint{0.000000in}{0.000000in}}%
\pgfpathlineto{\pgfqpoint{0.000000in}{-0.048611in}}%
\pgfusepath{stroke,fill}%
}%
\begin{pgfscope}%
\pgfsys@transformshift{2.824582in}{0.528000in}%
\pgfsys@useobject{currentmarker}{}%
\end{pgfscope}%
\end{pgfscope}%
\begin{pgfscope}%
\definecolor{textcolor}{rgb}{0.000000,0.000000,0.000000}%
\pgfsetstrokecolor{textcolor}%
\pgfsetfillcolor{textcolor}%
\pgftext[x=2.824582in,y=0.430778in,,top]{\color{textcolor}\ttfamily\fontsize{10.000000}{12.000000}\selectfont 400}%
\end{pgfscope}%
\begin{pgfscope}%
\pgfsetbuttcap%
\pgfsetroundjoin%
\definecolor{currentfill}{rgb}{0.000000,0.000000,0.000000}%
\pgfsetfillcolor{currentfill}%
\pgfsetlinewidth{0.803000pt}%
\definecolor{currentstroke}{rgb}{0.000000,0.000000,0.000000}%
\pgfsetstrokecolor{currentstroke}%
\pgfsetdash{}{0pt}%
\pgfsys@defobject{currentmarker}{\pgfqpoint{0.000000in}{-0.048611in}}{\pgfqpoint{0.000000in}{0.000000in}}{%
\pgfpathmoveto{\pgfqpoint{0.000000in}{0.000000in}}%
\pgfpathlineto{\pgfqpoint{0.000000in}{-0.048611in}}%
\pgfusepath{stroke,fill}%
}%
\begin{pgfscope}%
\pgfsys@transformshift{3.726400in}{0.528000in}%
\pgfsys@useobject{currentmarker}{}%
\end{pgfscope}%
\end{pgfscope}%
\begin{pgfscope}%
\definecolor{textcolor}{rgb}{0.000000,0.000000,0.000000}%
\pgfsetstrokecolor{textcolor}%
\pgfsetfillcolor{textcolor}%
\pgftext[x=3.726400in,y=0.430778in,,top]{\color{textcolor}\ttfamily\fontsize{10.000000}{12.000000}\selectfont 600}%
\end{pgfscope}%
\begin{pgfscope}%
\pgfsetbuttcap%
\pgfsetroundjoin%
\definecolor{currentfill}{rgb}{0.000000,0.000000,0.000000}%
\pgfsetfillcolor{currentfill}%
\pgfsetlinewidth{0.803000pt}%
\definecolor{currentstroke}{rgb}{0.000000,0.000000,0.000000}%
\pgfsetstrokecolor{currentstroke}%
\pgfsetdash{}{0pt}%
\pgfsys@defobject{currentmarker}{\pgfqpoint{0.000000in}{-0.048611in}}{\pgfqpoint{0.000000in}{0.000000in}}{%
\pgfpathmoveto{\pgfqpoint{0.000000in}{0.000000in}}%
\pgfpathlineto{\pgfqpoint{0.000000in}{-0.048611in}}%
\pgfusepath{stroke,fill}%
}%
\begin{pgfscope}%
\pgfsys@transformshift{4.628218in}{0.528000in}%
\pgfsys@useobject{currentmarker}{}%
\end{pgfscope}%
\end{pgfscope}%
\begin{pgfscope}%
\definecolor{textcolor}{rgb}{0.000000,0.000000,0.000000}%
\pgfsetstrokecolor{textcolor}%
\pgfsetfillcolor{textcolor}%
\pgftext[x=4.628218in,y=0.430778in,,top]{\color{textcolor}\ttfamily\fontsize{10.000000}{12.000000}\selectfont 800}%
\end{pgfscope}%
\begin{pgfscope}%
\pgfsetbuttcap%
\pgfsetroundjoin%
\definecolor{currentfill}{rgb}{0.000000,0.000000,0.000000}%
\pgfsetfillcolor{currentfill}%
\pgfsetlinewidth{0.803000pt}%
\definecolor{currentstroke}{rgb}{0.000000,0.000000,0.000000}%
\pgfsetstrokecolor{currentstroke}%
\pgfsetdash{}{0pt}%
\pgfsys@defobject{currentmarker}{\pgfqpoint{0.000000in}{-0.048611in}}{\pgfqpoint{0.000000in}{0.000000in}}{%
\pgfpathmoveto{\pgfqpoint{0.000000in}{0.000000in}}%
\pgfpathlineto{\pgfqpoint{0.000000in}{-0.048611in}}%
\pgfusepath{stroke,fill}%
}%
\begin{pgfscope}%
\pgfsys@transformshift{5.530036in}{0.528000in}%
\pgfsys@useobject{currentmarker}{}%
\end{pgfscope}%
\end{pgfscope}%
\begin{pgfscope}%
\definecolor{textcolor}{rgb}{0.000000,0.000000,0.000000}%
\pgfsetstrokecolor{textcolor}%
\pgfsetfillcolor{textcolor}%
\pgftext[x=5.530036in,y=0.430778in,,top]{\color{textcolor}\ttfamily\fontsize{10.000000}{12.000000}\selectfont 1000}%
\end{pgfscope}%
\begin{pgfscope}%
\definecolor{textcolor}{rgb}{0.000000,0.000000,0.000000}%
\pgfsetstrokecolor{textcolor}%
\pgfsetfillcolor{textcolor}%
\pgftext[x=3.280000in,y=0.240063in,,top]{\color{textcolor}\ttfamily\fontsize{10.000000}{12.000000}\selectfont Size of Array}%
\end{pgfscope}%
\begin{pgfscope}%
\pgfsetbuttcap%
\pgfsetroundjoin%
\definecolor{currentfill}{rgb}{0.000000,0.000000,0.000000}%
\pgfsetfillcolor{currentfill}%
\pgfsetlinewidth{0.803000pt}%
\definecolor{currentstroke}{rgb}{0.000000,0.000000,0.000000}%
\pgfsetstrokecolor{currentstroke}%
\pgfsetdash{}{0pt}%
\pgfsys@defobject{currentmarker}{\pgfqpoint{-0.048611in}{0.000000in}}{\pgfqpoint{-0.000000in}{0.000000in}}{%
\pgfpathmoveto{\pgfqpoint{-0.000000in}{0.000000in}}%
\pgfpathlineto{\pgfqpoint{-0.048611in}{0.000000in}}%
\pgfusepath{stroke,fill}%
}%
\begin{pgfscope}%
\pgfsys@transformshift{0.800000in}{1.084043in}%
\pgfsys@useobject{currentmarker}{}%
\end{pgfscope}%
\end{pgfscope}%
\begin{pgfscope}%
\definecolor{textcolor}{rgb}{0.000000,0.000000,0.000000}%
\pgfsetstrokecolor{textcolor}%
\pgfsetfillcolor{textcolor}%
\pgftext[x=0.368305in, y=1.030908in, left, base]{\color{textcolor}\ttfamily\fontsize{10.000000}{12.000000}\selectfont 2000}%
\end{pgfscope}%
\begin{pgfscope}%
\pgfsetbuttcap%
\pgfsetroundjoin%
\definecolor{currentfill}{rgb}{0.000000,0.000000,0.000000}%
\pgfsetfillcolor{currentfill}%
\pgfsetlinewidth{0.803000pt}%
\definecolor{currentstroke}{rgb}{0.000000,0.000000,0.000000}%
\pgfsetstrokecolor{currentstroke}%
\pgfsetdash{}{0pt}%
\pgfsys@defobject{currentmarker}{\pgfqpoint{-0.048611in}{0.000000in}}{\pgfqpoint{-0.000000in}{0.000000in}}{%
\pgfpathmoveto{\pgfqpoint{-0.000000in}{0.000000in}}%
\pgfpathlineto{\pgfqpoint{-0.048611in}{0.000000in}}%
\pgfusepath{stroke,fill}%
}%
\begin{pgfscope}%
\pgfsys@transformshift{0.800000in}{1.691308in}%
\pgfsys@useobject{currentmarker}{}%
\end{pgfscope}%
\end{pgfscope}%
\begin{pgfscope}%
\definecolor{textcolor}{rgb}{0.000000,0.000000,0.000000}%
\pgfsetstrokecolor{textcolor}%
\pgfsetfillcolor{textcolor}%
\pgftext[x=0.368305in, y=1.638174in, left, base]{\color{textcolor}\ttfamily\fontsize{10.000000}{12.000000}\selectfont 4000}%
\end{pgfscope}%
\begin{pgfscope}%
\pgfsetbuttcap%
\pgfsetroundjoin%
\definecolor{currentfill}{rgb}{0.000000,0.000000,0.000000}%
\pgfsetfillcolor{currentfill}%
\pgfsetlinewidth{0.803000pt}%
\definecolor{currentstroke}{rgb}{0.000000,0.000000,0.000000}%
\pgfsetstrokecolor{currentstroke}%
\pgfsetdash{}{0pt}%
\pgfsys@defobject{currentmarker}{\pgfqpoint{-0.048611in}{0.000000in}}{\pgfqpoint{-0.000000in}{0.000000in}}{%
\pgfpathmoveto{\pgfqpoint{-0.000000in}{0.000000in}}%
\pgfpathlineto{\pgfqpoint{-0.048611in}{0.000000in}}%
\pgfusepath{stroke,fill}%
}%
\begin{pgfscope}%
\pgfsys@transformshift{0.800000in}{2.298574in}%
\pgfsys@useobject{currentmarker}{}%
\end{pgfscope}%
\end{pgfscope}%
\begin{pgfscope}%
\definecolor{textcolor}{rgb}{0.000000,0.000000,0.000000}%
\pgfsetstrokecolor{textcolor}%
\pgfsetfillcolor{textcolor}%
\pgftext[x=0.368305in, y=2.245439in, left, base]{\color{textcolor}\ttfamily\fontsize{10.000000}{12.000000}\selectfont 6000}%
\end{pgfscope}%
\begin{pgfscope}%
\pgfsetbuttcap%
\pgfsetroundjoin%
\definecolor{currentfill}{rgb}{0.000000,0.000000,0.000000}%
\pgfsetfillcolor{currentfill}%
\pgfsetlinewidth{0.803000pt}%
\definecolor{currentstroke}{rgb}{0.000000,0.000000,0.000000}%
\pgfsetstrokecolor{currentstroke}%
\pgfsetdash{}{0pt}%
\pgfsys@defobject{currentmarker}{\pgfqpoint{-0.048611in}{0.000000in}}{\pgfqpoint{-0.000000in}{0.000000in}}{%
\pgfpathmoveto{\pgfqpoint{-0.000000in}{0.000000in}}%
\pgfpathlineto{\pgfqpoint{-0.048611in}{0.000000in}}%
\pgfusepath{stroke,fill}%
}%
\begin{pgfscope}%
\pgfsys@transformshift{0.800000in}{2.905839in}%
\pgfsys@useobject{currentmarker}{}%
\end{pgfscope}%
\end{pgfscope}%
\begin{pgfscope}%
\definecolor{textcolor}{rgb}{0.000000,0.000000,0.000000}%
\pgfsetstrokecolor{textcolor}%
\pgfsetfillcolor{textcolor}%
\pgftext[x=0.368305in, y=2.852705in, left, base]{\color{textcolor}\ttfamily\fontsize{10.000000}{12.000000}\selectfont 8000}%
\end{pgfscope}%
\begin{pgfscope}%
\pgfsetbuttcap%
\pgfsetroundjoin%
\definecolor{currentfill}{rgb}{0.000000,0.000000,0.000000}%
\pgfsetfillcolor{currentfill}%
\pgfsetlinewidth{0.803000pt}%
\definecolor{currentstroke}{rgb}{0.000000,0.000000,0.000000}%
\pgfsetstrokecolor{currentstroke}%
\pgfsetdash{}{0pt}%
\pgfsys@defobject{currentmarker}{\pgfqpoint{-0.048611in}{0.000000in}}{\pgfqpoint{-0.000000in}{0.000000in}}{%
\pgfpathmoveto{\pgfqpoint{-0.000000in}{0.000000in}}%
\pgfpathlineto{\pgfqpoint{-0.048611in}{0.000000in}}%
\pgfusepath{stroke,fill}%
}%
\begin{pgfscope}%
\pgfsys@transformshift{0.800000in}{3.513105in}%
\pgfsys@useobject{currentmarker}{}%
\end{pgfscope}%
\end{pgfscope}%
\begin{pgfscope}%
\definecolor{textcolor}{rgb}{0.000000,0.000000,0.000000}%
\pgfsetstrokecolor{textcolor}%
\pgfsetfillcolor{textcolor}%
\pgftext[x=0.284687in, y=3.459970in, left, base]{\color{textcolor}\ttfamily\fontsize{10.000000}{12.000000}\selectfont 10000}%
\end{pgfscope}%
\begin{pgfscope}%
\pgfsetbuttcap%
\pgfsetroundjoin%
\definecolor{currentfill}{rgb}{0.000000,0.000000,0.000000}%
\pgfsetfillcolor{currentfill}%
\pgfsetlinewidth{0.803000pt}%
\definecolor{currentstroke}{rgb}{0.000000,0.000000,0.000000}%
\pgfsetstrokecolor{currentstroke}%
\pgfsetdash{}{0pt}%
\pgfsys@defobject{currentmarker}{\pgfqpoint{-0.048611in}{0.000000in}}{\pgfqpoint{-0.000000in}{0.000000in}}{%
\pgfpathmoveto{\pgfqpoint{-0.000000in}{0.000000in}}%
\pgfpathlineto{\pgfqpoint{-0.048611in}{0.000000in}}%
\pgfusepath{stroke,fill}%
}%
\begin{pgfscope}%
\pgfsys@transformshift{0.800000in}{4.120370in}%
\pgfsys@useobject{currentmarker}{}%
\end{pgfscope}%
\end{pgfscope}%
\begin{pgfscope}%
\definecolor{textcolor}{rgb}{0.000000,0.000000,0.000000}%
\pgfsetstrokecolor{textcolor}%
\pgfsetfillcolor{textcolor}%
\pgftext[x=0.284687in, y=4.067236in, left, base]{\color{textcolor}\ttfamily\fontsize{10.000000}{12.000000}\selectfont 12000}%
\end{pgfscope}%
\begin{pgfscope}%
\definecolor{textcolor}{rgb}{0.000000,0.000000,0.000000}%
\pgfsetstrokecolor{textcolor}%
\pgfsetfillcolor{textcolor}%
\pgftext[x=0.229131in,y=2.376000in,,bottom,rotate=90.000000]{\color{textcolor}\ttfamily\fontsize{10.000000}{12.000000}\selectfont Comparisons}%
\end{pgfscope}%
\begin{pgfscope}%
\pgfpathrectangle{\pgfqpoint{0.800000in}{0.528000in}}{\pgfqpoint{4.960000in}{3.696000in}}%
\pgfusepath{clip}%
\pgfsetrectcap%
\pgfsetroundjoin%
\pgfsetlinewidth{1.505625pt}%
\definecolor{currentstroke}{rgb}{0.000000,1.000000,0.498039}%
\pgfsetstrokecolor{currentstroke}%
\pgfsetdash{}{0pt}%
\pgfpathmoveto{\pgfqpoint{1.025455in}{0.696000in}}%
\pgfpathlineto{\pgfqpoint{1.029964in}{0.701465in}}%
\pgfpathlineto{\pgfqpoint{1.034473in}{0.700251in}}%
\pgfpathlineto{\pgfqpoint{1.043491in}{0.704502in}}%
\pgfpathlineto{\pgfqpoint{1.048000in}{0.709967in}}%
\pgfpathlineto{\pgfqpoint{1.052509in}{0.710574in}}%
\pgfpathlineto{\pgfqpoint{1.057018in}{0.712396in}}%
\pgfpathlineto{\pgfqpoint{1.066036in}{0.719076in}}%
\pgfpathlineto{\pgfqpoint{1.070545in}{0.720898in}}%
\pgfpathlineto{\pgfqpoint{1.075055in}{0.726363in}}%
\pgfpathlineto{\pgfqpoint{1.084073in}{0.731829in}}%
\pgfpathlineto{\pgfqpoint{1.088582in}{0.736687in}}%
\pgfpathlineto{\pgfqpoint{1.093091in}{0.733650in}}%
\pgfpathlineto{\pgfqpoint{1.102109in}{0.740330in}}%
\pgfpathlineto{\pgfqpoint{1.106618in}{0.745796in}}%
\pgfpathlineto{\pgfqpoint{1.111127in}{0.748832in}}%
\pgfpathlineto{\pgfqpoint{1.115636in}{0.749439in}}%
\pgfpathlineto{\pgfqpoint{1.120145in}{0.754905in}}%
\pgfpathlineto{\pgfqpoint{1.124655in}{0.754905in}}%
\pgfpathlineto{\pgfqpoint{1.133673in}{0.762192in}}%
\pgfpathlineto{\pgfqpoint{1.138182in}{0.767657in}}%
\pgfpathlineto{\pgfqpoint{1.142691in}{0.770694in}}%
\pgfpathlineto{\pgfqpoint{1.147200in}{0.776159in}}%
\pgfpathlineto{\pgfqpoint{1.156218in}{0.776159in}}%
\pgfpathlineto{\pgfqpoint{1.165236in}{0.781624in}}%
\pgfpathlineto{\pgfqpoint{1.169745in}{0.786483in}}%
\pgfpathlineto{\pgfqpoint{1.174255in}{0.787697in}}%
\pgfpathlineto{\pgfqpoint{1.183273in}{0.793162in}}%
\pgfpathlineto{\pgfqpoint{1.187782in}{0.788912in}}%
\pgfpathlineto{\pgfqpoint{1.192291in}{0.792555in}}%
\pgfpathlineto{\pgfqpoint{1.196800in}{0.799235in}}%
\pgfpathlineto{\pgfqpoint{1.201309in}{0.796806in}}%
\pgfpathlineto{\pgfqpoint{1.205818in}{0.802879in}}%
\pgfpathlineto{\pgfqpoint{1.219345in}{0.810773in}}%
\pgfpathlineto{\pgfqpoint{1.223855in}{0.816239in}}%
\pgfpathlineto{\pgfqpoint{1.228364in}{0.815631in}}%
\pgfpathlineto{\pgfqpoint{1.237382in}{0.823526in}}%
\pgfpathlineto{\pgfqpoint{1.241891in}{0.822311in}}%
\pgfpathlineto{\pgfqpoint{1.250909in}{0.828384in}}%
\pgfpathlineto{\pgfqpoint{1.255418in}{0.831420in}}%
\pgfpathlineto{\pgfqpoint{1.259927in}{0.839315in}}%
\pgfpathlineto{\pgfqpoint{1.264436in}{0.834457in}}%
\pgfpathlineto{\pgfqpoint{1.273455in}{0.844173in}}%
\pgfpathlineto{\pgfqpoint{1.282473in}{0.850853in}}%
\pgfpathlineto{\pgfqpoint{1.286982in}{0.856925in}}%
\pgfpathlineto{\pgfqpoint{1.296000in}{0.862391in}}%
\pgfpathlineto{\pgfqpoint{1.300509in}{0.862998in}}%
\pgfpathlineto{\pgfqpoint{1.309527in}{0.866642in}}%
\pgfpathlineto{\pgfqpoint{1.314036in}{0.870285in}}%
\pgfpathlineto{\pgfqpoint{1.318545in}{0.871500in}}%
\pgfpathlineto{\pgfqpoint{1.323055in}{0.873929in}}%
\pgfpathlineto{\pgfqpoint{1.327564in}{0.879394in}}%
\pgfpathlineto{\pgfqpoint{1.336582in}{0.884252in}}%
\pgfpathlineto{\pgfqpoint{1.341091in}{0.883038in}}%
\pgfpathlineto{\pgfqpoint{1.345600in}{0.889718in}}%
\pgfpathlineto{\pgfqpoint{1.350109in}{0.892754in}}%
\pgfpathlineto{\pgfqpoint{1.354618in}{0.898219in}}%
\pgfpathlineto{\pgfqpoint{1.359127in}{0.898219in}}%
\pgfpathlineto{\pgfqpoint{1.368145in}{0.904899in}}%
\pgfpathlineto{\pgfqpoint{1.372655in}{0.905507in}}%
\pgfpathlineto{\pgfqpoint{1.377164in}{0.910365in}}%
\pgfpathlineto{\pgfqpoint{1.386182in}{0.924332in}}%
\pgfpathlineto{\pgfqpoint{1.390691in}{0.918259in}}%
\pgfpathlineto{\pgfqpoint{1.413236in}{0.941943in}}%
\pgfpathlineto{\pgfqpoint{1.422255in}{0.944979in}}%
\pgfpathlineto{\pgfqpoint{1.431273in}{0.951052in}}%
\pgfpathlineto{\pgfqpoint{1.435782in}{0.950444in}}%
\pgfpathlineto{\pgfqpoint{1.440291in}{0.960160in}}%
\pgfpathlineto{\pgfqpoint{1.444800in}{0.965626in}}%
\pgfpathlineto{\pgfqpoint{1.449309in}{0.967448in}}%
\pgfpathlineto{\pgfqpoint{1.453818in}{0.963804in}}%
\pgfpathlineto{\pgfqpoint{1.458327in}{0.970484in}}%
\pgfpathlineto{\pgfqpoint{1.467345in}{0.975342in}}%
\pgfpathlineto{\pgfqpoint{1.471855in}{0.975949in}}%
\pgfpathlineto{\pgfqpoint{1.476364in}{0.982022in}}%
\pgfpathlineto{\pgfqpoint{1.480873in}{0.981415in}}%
\pgfpathlineto{\pgfqpoint{1.485382in}{0.983237in}}%
\pgfpathlineto{\pgfqpoint{1.489891in}{0.983844in}}%
\pgfpathlineto{\pgfqpoint{1.494400in}{0.988702in}}%
\pgfpathlineto{\pgfqpoint{1.498909in}{0.991738in}}%
\pgfpathlineto{\pgfqpoint{1.503418in}{0.997811in}}%
\pgfpathlineto{\pgfqpoint{1.512436in}{0.997204in}}%
\pgfpathlineto{\pgfqpoint{1.516945in}{1.004491in}}%
\pgfpathlineto{\pgfqpoint{1.525964in}{1.010564in}}%
\pgfpathlineto{\pgfqpoint{1.530473in}{1.009956in}}%
\pgfpathlineto{\pgfqpoint{1.534982in}{1.015422in}}%
\pgfpathlineto{\pgfqpoint{1.539491in}{1.016029in}}%
\pgfpathlineto{\pgfqpoint{1.544000in}{1.013600in}}%
\pgfpathlineto{\pgfqpoint{1.553018in}{1.031818in}}%
\pgfpathlineto{\pgfqpoint{1.557527in}{1.026352in}}%
\pgfpathlineto{\pgfqpoint{1.562036in}{1.033640in}}%
\pgfpathlineto{\pgfqpoint{1.571055in}{1.038498in}}%
\pgfpathlineto{\pgfqpoint{1.575564in}{1.045785in}}%
\pgfpathlineto{\pgfqpoint{1.580073in}{1.045178in}}%
\pgfpathlineto{\pgfqpoint{1.593600in}{1.053072in}}%
\pgfpathlineto{\pgfqpoint{1.602618in}{1.063396in}}%
\pgfpathlineto{\pgfqpoint{1.607127in}{1.062181in}}%
\pgfpathlineto{\pgfqpoint{1.611636in}{1.068861in}}%
\pgfpathlineto{\pgfqpoint{1.616145in}{1.064610in}}%
\pgfpathlineto{\pgfqpoint{1.620655in}{1.068861in}}%
\pgfpathlineto{\pgfqpoint{1.625164in}{1.075541in}}%
\pgfpathlineto{\pgfqpoint{1.629673in}{1.076755in}}%
\pgfpathlineto{\pgfqpoint{1.634182in}{1.076755in}}%
\pgfpathlineto{\pgfqpoint{1.638691in}{1.084650in}}%
\pgfpathlineto{\pgfqpoint{1.643200in}{1.085257in}}%
\pgfpathlineto{\pgfqpoint{1.652218in}{1.093759in}}%
\pgfpathlineto{\pgfqpoint{1.656727in}{1.097402in}}%
\pgfpathlineto{\pgfqpoint{1.661236in}{1.104690in}}%
\pgfpathlineto{\pgfqpoint{1.665745in}{1.101653in}}%
\pgfpathlineto{\pgfqpoint{1.670255in}{1.107119in}}%
\pgfpathlineto{\pgfqpoint{1.674764in}{1.108333in}}%
\pgfpathlineto{\pgfqpoint{1.683782in}{1.122300in}}%
\pgfpathlineto{\pgfqpoint{1.688291in}{1.118050in}}%
\pgfpathlineto{\pgfqpoint{1.692800in}{1.126551in}}%
\pgfpathlineto{\pgfqpoint{1.697309in}{1.128373in}}%
\pgfpathlineto{\pgfqpoint{1.701818in}{1.131409in}}%
\pgfpathlineto{\pgfqpoint{1.706327in}{1.136267in}}%
\pgfpathlineto{\pgfqpoint{1.710836in}{1.135660in}}%
\pgfpathlineto{\pgfqpoint{1.719855in}{1.149020in}}%
\pgfpathlineto{\pgfqpoint{1.724364in}{1.149020in}}%
\pgfpathlineto{\pgfqpoint{1.728873in}{1.157522in}}%
\pgfpathlineto{\pgfqpoint{1.733382in}{1.159951in}}%
\pgfpathlineto{\pgfqpoint{1.737891in}{1.157522in}}%
\pgfpathlineto{\pgfqpoint{1.742400in}{1.164809in}}%
\pgfpathlineto{\pgfqpoint{1.746909in}{1.167238in}}%
\pgfpathlineto{\pgfqpoint{1.751418in}{1.166023in}}%
\pgfpathlineto{\pgfqpoint{1.755927in}{1.173918in}}%
\pgfpathlineto{\pgfqpoint{1.760436in}{1.169060in}}%
\pgfpathlineto{\pgfqpoint{1.764945in}{1.174525in}}%
\pgfpathlineto{\pgfqpoint{1.769455in}{1.178169in}}%
\pgfpathlineto{\pgfqpoint{1.773964in}{1.175740in}}%
\pgfpathlineto{\pgfqpoint{1.778473in}{1.181205in}}%
\pgfpathlineto{\pgfqpoint{1.782982in}{1.177562in}}%
\pgfpathlineto{\pgfqpoint{1.787491in}{1.184241in}}%
\pgfpathlineto{\pgfqpoint{1.796509in}{1.190921in}}%
\pgfpathlineto{\pgfqpoint{1.801018in}{1.195780in}}%
\pgfpathlineto{\pgfqpoint{1.805527in}{1.194565in}}%
\pgfpathlineto{\pgfqpoint{1.814545in}{1.203067in}}%
\pgfpathlineto{\pgfqpoint{1.819055in}{1.201245in}}%
\pgfpathlineto{\pgfqpoint{1.823564in}{1.206103in}}%
\pgfpathlineto{\pgfqpoint{1.828073in}{1.205496in}}%
\pgfpathlineto{\pgfqpoint{1.832582in}{1.213997in}}%
\pgfpathlineto{\pgfqpoint{1.837091in}{1.212783in}}%
\pgfpathlineto{\pgfqpoint{1.846109in}{1.217034in}}%
\pgfpathlineto{\pgfqpoint{1.850618in}{1.221892in}}%
\pgfpathlineto{\pgfqpoint{1.855127in}{1.231001in}}%
\pgfpathlineto{\pgfqpoint{1.859636in}{1.229786in}}%
\pgfpathlineto{\pgfqpoint{1.864145in}{1.234037in}}%
\pgfpathlineto{\pgfqpoint{1.873164in}{1.237681in}}%
\pgfpathlineto{\pgfqpoint{1.882182in}{1.246790in}}%
\pgfpathlineto{\pgfqpoint{1.886691in}{1.252862in}}%
\pgfpathlineto{\pgfqpoint{1.891200in}{1.250433in}}%
\pgfpathlineto{\pgfqpoint{1.895709in}{1.254077in}}%
\pgfpathlineto{\pgfqpoint{1.900218in}{1.259542in}}%
\pgfpathlineto{\pgfqpoint{1.904727in}{1.258935in}}%
\pgfpathlineto{\pgfqpoint{1.909236in}{1.260757in}}%
\pgfpathlineto{\pgfqpoint{1.913745in}{1.260757in}}%
\pgfpathlineto{\pgfqpoint{1.922764in}{1.271080in}}%
\pgfpathlineto{\pgfqpoint{1.927273in}{1.272902in}}%
\pgfpathlineto{\pgfqpoint{1.931782in}{1.277153in}}%
\pgfpathlineto{\pgfqpoint{1.940800in}{1.280797in}}%
\pgfpathlineto{\pgfqpoint{1.945309in}{1.288084in}}%
\pgfpathlineto{\pgfqpoint{1.949818in}{1.289906in}}%
\pgfpathlineto{\pgfqpoint{1.954327in}{1.295978in}}%
\pgfpathlineto{\pgfqpoint{1.958836in}{1.297800in}}%
\pgfpathlineto{\pgfqpoint{1.963345in}{1.301444in}}%
\pgfpathlineto{\pgfqpoint{1.972364in}{1.304480in}}%
\pgfpathlineto{\pgfqpoint{1.976873in}{1.308124in}}%
\pgfpathlineto{\pgfqpoint{1.981382in}{1.314196in}}%
\pgfpathlineto{\pgfqpoint{1.985891in}{1.314804in}}%
\pgfpathlineto{\pgfqpoint{1.990400in}{1.322698in}}%
\pgfpathlineto{\pgfqpoint{1.994909in}{1.319662in}}%
\pgfpathlineto{\pgfqpoint{1.999418in}{1.323305in}}%
\pgfpathlineto{\pgfqpoint{2.003927in}{1.333629in}}%
\pgfpathlineto{\pgfqpoint{2.008436in}{1.331807in}}%
\pgfpathlineto{\pgfqpoint{2.012945in}{1.343345in}}%
\pgfpathlineto{\pgfqpoint{2.017455in}{1.341523in}}%
\pgfpathlineto{\pgfqpoint{2.030982in}{1.356098in}}%
\pgfpathlineto{\pgfqpoint{2.035491in}{1.354276in}}%
\pgfpathlineto{\pgfqpoint{2.040000in}{1.354883in}}%
\pgfpathlineto{\pgfqpoint{2.044509in}{1.357312in}}%
\pgfpathlineto{\pgfqpoint{2.049018in}{1.364599in}}%
\pgfpathlineto{\pgfqpoint{2.053527in}{1.365207in}}%
\pgfpathlineto{\pgfqpoint{2.058036in}{1.370065in}}%
\pgfpathlineto{\pgfqpoint{2.067055in}{1.368850in}}%
\pgfpathlineto{\pgfqpoint{2.071564in}{1.374316in}}%
\pgfpathlineto{\pgfqpoint{2.076073in}{1.386461in}}%
\pgfpathlineto{\pgfqpoint{2.080582in}{1.382210in}}%
\pgfpathlineto{\pgfqpoint{2.089600in}{1.394963in}}%
\pgfpathlineto{\pgfqpoint{2.094109in}{1.391926in}}%
\pgfpathlineto{\pgfqpoint{2.098618in}{1.391319in}}%
\pgfpathlineto{\pgfqpoint{2.103127in}{1.401035in}}%
\pgfpathlineto{\pgfqpoint{2.107636in}{1.399213in}}%
\pgfpathlineto{\pgfqpoint{2.112145in}{1.404679in}}%
\pgfpathlineto{\pgfqpoint{2.116655in}{1.406501in}}%
\pgfpathlineto{\pgfqpoint{2.121164in}{1.409537in}}%
\pgfpathlineto{\pgfqpoint{2.125673in}{1.410144in}}%
\pgfpathlineto{\pgfqpoint{2.130182in}{1.415610in}}%
\pgfpathlineto{\pgfqpoint{2.134691in}{1.417431in}}%
\pgfpathlineto{\pgfqpoint{2.139200in}{1.425326in}}%
\pgfpathlineto{\pgfqpoint{2.148218in}{1.435042in}}%
\pgfpathlineto{\pgfqpoint{2.152727in}{1.433220in}}%
\pgfpathlineto{\pgfqpoint{2.161745in}{1.445973in}}%
\pgfpathlineto{\pgfqpoint{2.166255in}{1.444758in}}%
\pgfpathlineto{\pgfqpoint{2.170764in}{1.452046in}}%
\pgfpathlineto{\pgfqpoint{2.175273in}{1.453867in}}%
\pgfpathlineto{\pgfqpoint{2.179782in}{1.452046in}}%
\pgfpathlineto{\pgfqpoint{2.184291in}{1.459940in}}%
\pgfpathlineto{\pgfqpoint{2.188800in}{1.458118in}}%
\pgfpathlineto{\pgfqpoint{2.193309in}{1.464191in}}%
\pgfpathlineto{\pgfqpoint{2.197818in}{1.461155in}}%
\pgfpathlineto{\pgfqpoint{2.215855in}{1.487267in}}%
\pgfpathlineto{\pgfqpoint{2.224873in}{1.489089in}}%
\pgfpathlineto{\pgfqpoint{2.242909in}{1.507307in}}%
\pgfpathlineto{\pgfqpoint{2.247418in}{1.504878in}}%
\pgfpathlineto{\pgfqpoint{2.251927in}{1.513379in}}%
\pgfpathlineto{\pgfqpoint{2.256436in}{1.513987in}}%
\pgfpathlineto{\pgfqpoint{2.260945in}{1.524917in}}%
\pgfpathlineto{\pgfqpoint{2.265455in}{1.522488in}}%
\pgfpathlineto{\pgfqpoint{2.269964in}{1.531597in}}%
\pgfpathlineto{\pgfqpoint{2.274473in}{1.527346in}}%
\pgfpathlineto{\pgfqpoint{2.288000in}{1.547386in}}%
\pgfpathlineto{\pgfqpoint{2.292509in}{1.544957in}}%
\pgfpathlineto{\pgfqpoint{2.301527in}{1.562568in}}%
\pgfpathlineto{\pgfqpoint{2.306036in}{1.563782in}}%
\pgfpathlineto{\pgfqpoint{2.310545in}{1.567426in}}%
\pgfpathlineto{\pgfqpoint{2.315055in}{1.564997in}}%
\pgfpathlineto{\pgfqpoint{2.319564in}{1.566211in}}%
\pgfpathlineto{\pgfqpoint{2.324073in}{1.566211in}}%
\pgfpathlineto{\pgfqpoint{2.328582in}{1.575320in}}%
\pgfpathlineto{\pgfqpoint{2.333091in}{1.577142in}}%
\pgfpathlineto{\pgfqpoint{2.337600in}{1.569855in}}%
\pgfpathlineto{\pgfqpoint{2.346618in}{1.589288in}}%
\pgfpathlineto{\pgfqpoint{2.351127in}{1.591717in}}%
\pgfpathlineto{\pgfqpoint{2.355636in}{1.589288in}}%
\pgfpathlineto{\pgfqpoint{2.360145in}{1.596575in}}%
\pgfpathlineto{\pgfqpoint{2.364655in}{1.592931in}}%
\pgfpathlineto{\pgfqpoint{2.369164in}{1.594146in}}%
\pgfpathlineto{\pgfqpoint{2.373673in}{1.603255in}}%
\pgfpathlineto{\pgfqpoint{2.387200in}{1.611149in}}%
\pgfpathlineto{\pgfqpoint{2.391709in}{1.607506in}}%
\pgfpathlineto{\pgfqpoint{2.400727in}{1.618436in}}%
\pgfpathlineto{\pgfqpoint{2.405236in}{1.619044in}}%
\pgfpathlineto{\pgfqpoint{2.409745in}{1.626331in}}%
\pgfpathlineto{\pgfqpoint{2.414255in}{1.624509in}}%
\pgfpathlineto{\pgfqpoint{2.418764in}{1.628760in}}%
\pgfpathlineto{\pgfqpoint{2.423273in}{1.628153in}}%
\pgfpathlineto{\pgfqpoint{2.427782in}{1.637262in}}%
\pgfpathlineto{\pgfqpoint{2.432291in}{1.637262in}}%
\pgfpathlineto{\pgfqpoint{2.436800in}{1.644549in}}%
\pgfpathlineto{\pgfqpoint{2.441309in}{1.645156in}}%
\pgfpathlineto{\pgfqpoint{2.450327in}{1.656694in}}%
\pgfpathlineto{\pgfqpoint{2.454836in}{1.653658in}}%
\pgfpathlineto{\pgfqpoint{2.459345in}{1.654265in}}%
\pgfpathlineto{\pgfqpoint{2.463855in}{1.652443in}}%
\pgfpathlineto{\pgfqpoint{2.468364in}{1.657301in}}%
\pgfpathlineto{\pgfqpoint{2.472873in}{1.668839in}}%
\pgfpathlineto{\pgfqpoint{2.477382in}{1.676127in}}%
\pgfpathlineto{\pgfqpoint{2.481891in}{1.674912in}}%
\pgfpathlineto{\pgfqpoint{2.486400in}{1.666410in}}%
\pgfpathlineto{\pgfqpoint{2.490909in}{1.679770in}}%
\pgfpathlineto{\pgfqpoint{2.495418in}{1.684628in}}%
\pgfpathlineto{\pgfqpoint{2.499927in}{1.687665in}}%
\pgfpathlineto{\pgfqpoint{2.504436in}{1.687057in}}%
\pgfpathlineto{\pgfqpoint{2.508945in}{1.694344in}}%
\pgfpathlineto{\pgfqpoint{2.513455in}{1.685843in}}%
\pgfpathlineto{\pgfqpoint{2.522473in}{1.706490in}}%
\pgfpathlineto{\pgfqpoint{2.526982in}{1.698595in}}%
\pgfpathlineto{\pgfqpoint{2.531491in}{1.707704in}}%
\pgfpathlineto{\pgfqpoint{2.536000in}{1.712562in}}%
\pgfpathlineto{\pgfqpoint{2.540509in}{1.713170in}}%
\pgfpathlineto{\pgfqpoint{2.545018in}{1.718635in}}%
\pgfpathlineto{\pgfqpoint{2.549527in}{1.717421in}}%
\pgfpathlineto{\pgfqpoint{2.554036in}{1.724100in}}%
\pgfpathlineto{\pgfqpoint{2.558545in}{1.722886in}}%
\pgfpathlineto{\pgfqpoint{2.563055in}{1.727744in}}%
\pgfpathlineto{\pgfqpoint{2.567564in}{1.738675in}}%
\pgfpathlineto{\pgfqpoint{2.572073in}{1.743533in}}%
\pgfpathlineto{\pgfqpoint{2.576582in}{1.746569in}}%
\pgfpathlineto{\pgfqpoint{2.581091in}{1.747784in}}%
\pgfpathlineto{\pgfqpoint{2.585600in}{1.750820in}}%
\pgfpathlineto{\pgfqpoint{2.590109in}{1.761144in}}%
\pgfpathlineto{\pgfqpoint{2.594618in}{1.758715in}}%
\pgfpathlineto{\pgfqpoint{2.599127in}{1.769038in}}%
\pgfpathlineto{\pgfqpoint{2.608145in}{1.769038in}}%
\pgfpathlineto{\pgfqpoint{2.612655in}{1.762965in}}%
\pgfpathlineto{\pgfqpoint{2.617164in}{1.775718in}}%
\pgfpathlineto{\pgfqpoint{2.621673in}{1.767216in}}%
\pgfpathlineto{\pgfqpoint{2.626182in}{1.773896in}}%
\pgfpathlineto{\pgfqpoint{2.630691in}{1.785434in}}%
\pgfpathlineto{\pgfqpoint{2.635200in}{1.788471in}}%
\pgfpathlineto{\pgfqpoint{2.639709in}{1.781791in}}%
\pgfpathlineto{\pgfqpoint{2.644218in}{1.787256in}}%
\pgfpathlineto{\pgfqpoint{2.648727in}{1.794543in}}%
\pgfpathlineto{\pgfqpoint{2.653236in}{1.794543in}}%
\pgfpathlineto{\pgfqpoint{2.657745in}{1.799401in}}%
\pgfpathlineto{\pgfqpoint{2.662255in}{1.807296in}}%
\pgfpathlineto{\pgfqpoint{2.671273in}{1.815190in}}%
\pgfpathlineto{\pgfqpoint{2.675782in}{1.813976in}}%
\pgfpathlineto{\pgfqpoint{2.680291in}{1.819441in}}%
\pgfpathlineto{\pgfqpoint{2.684800in}{1.818834in}}%
\pgfpathlineto{\pgfqpoint{2.689309in}{1.819441in}}%
\pgfpathlineto{\pgfqpoint{2.702836in}{1.839481in}}%
\pgfpathlineto{\pgfqpoint{2.707345in}{1.838874in}}%
\pgfpathlineto{\pgfqpoint{2.711855in}{1.841910in}}%
\pgfpathlineto{\pgfqpoint{2.716364in}{1.848590in}}%
\pgfpathlineto{\pgfqpoint{2.720873in}{1.851019in}}%
\pgfpathlineto{\pgfqpoint{2.725382in}{1.859521in}}%
\pgfpathlineto{\pgfqpoint{2.729891in}{1.858306in}}%
\pgfpathlineto{\pgfqpoint{2.734400in}{1.863164in}}%
\pgfpathlineto{\pgfqpoint{2.738909in}{1.864986in}}%
\pgfpathlineto{\pgfqpoint{2.743418in}{1.868022in}}%
\pgfpathlineto{\pgfqpoint{2.747927in}{1.875917in}}%
\pgfpathlineto{\pgfqpoint{2.752436in}{1.878346in}}%
\pgfpathlineto{\pgfqpoint{2.761455in}{1.888062in}}%
\pgfpathlineto{\pgfqpoint{2.765964in}{1.888062in}}%
\pgfpathlineto{\pgfqpoint{2.770473in}{1.885026in}}%
\pgfpathlineto{\pgfqpoint{2.774982in}{1.894135in}}%
\pgfpathlineto{\pgfqpoint{2.779491in}{1.895349in}}%
\pgfpathlineto{\pgfqpoint{2.788509in}{1.901422in}}%
\pgfpathlineto{\pgfqpoint{2.793018in}{1.913567in}}%
\pgfpathlineto{\pgfqpoint{2.802036in}{1.917818in}}%
\pgfpathlineto{\pgfqpoint{2.806545in}{1.916604in}}%
\pgfpathlineto{\pgfqpoint{2.811055in}{1.929356in}}%
\pgfpathlineto{\pgfqpoint{2.815564in}{1.931178in}}%
\pgfpathlineto{\pgfqpoint{2.820073in}{1.929963in}}%
\pgfpathlineto{\pgfqpoint{2.824582in}{1.939680in}}%
\pgfpathlineto{\pgfqpoint{2.829091in}{1.939680in}}%
\pgfpathlineto{\pgfqpoint{2.833600in}{1.946967in}}%
\pgfpathlineto{\pgfqpoint{2.838109in}{1.943931in}}%
\pgfpathlineto{\pgfqpoint{2.842618in}{1.956683in}}%
\pgfpathlineto{\pgfqpoint{2.847127in}{1.956683in}}%
\pgfpathlineto{\pgfqpoint{2.851636in}{1.963363in}}%
\pgfpathlineto{\pgfqpoint{2.856145in}{1.965792in}}%
\pgfpathlineto{\pgfqpoint{2.860655in}{1.976723in}}%
\pgfpathlineto{\pgfqpoint{2.865164in}{1.973687in}}%
\pgfpathlineto{\pgfqpoint{2.869673in}{1.974294in}}%
\pgfpathlineto{\pgfqpoint{2.874182in}{1.982796in}}%
\pgfpathlineto{\pgfqpoint{2.883200in}{1.993119in}}%
\pgfpathlineto{\pgfqpoint{2.887709in}{1.996763in}}%
\pgfpathlineto{\pgfqpoint{2.892218in}{1.991905in}}%
\pgfpathlineto{\pgfqpoint{2.896727in}{1.993726in}}%
\pgfpathlineto{\pgfqpoint{2.901236in}{2.000406in}}%
\pgfpathlineto{\pgfqpoint{2.905745in}{2.004657in}}%
\pgfpathlineto{\pgfqpoint{2.910255in}{2.001621in}}%
\pgfpathlineto{\pgfqpoint{2.914764in}{2.000406in}}%
\pgfpathlineto{\pgfqpoint{2.919273in}{2.011337in}}%
\pgfpathlineto{\pgfqpoint{2.923782in}{2.002835in}}%
\pgfpathlineto{\pgfqpoint{2.928291in}{2.018017in}}%
\pgfpathlineto{\pgfqpoint{2.932800in}{2.013766in}}%
\pgfpathlineto{\pgfqpoint{2.937309in}{2.021053in}}%
\pgfpathlineto{\pgfqpoint{2.941818in}{2.019839in}}%
\pgfpathlineto{\pgfqpoint{2.946327in}{2.023482in}}%
\pgfpathlineto{\pgfqpoint{2.950836in}{2.029555in}}%
\pgfpathlineto{\pgfqpoint{2.955345in}{2.022268in}}%
\pgfpathlineto{\pgfqpoint{2.959855in}{2.033806in}}%
\pgfpathlineto{\pgfqpoint{2.964364in}{2.032591in}}%
\pgfpathlineto{\pgfqpoint{2.973382in}{2.044737in}}%
\pgfpathlineto{\pgfqpoint{2.977891in}{2.041700in}}%
\pgfpathlineto{\pgfqpoint{2.982400in}{2.051417in}}%
\pgfpathlineto{\pgfqpoint{2.991418in}{2.051417in}}%
\pgfpathlineto{\pgfqpoint{2.995927in}{2.050202in}}%
\pgfpathlineto{\pgfqpoint{3.000436in}{2.062955in}}%
\pgfpathlineto{\pgfqpoint{3.004945in}{2.058097in}}%
\pgfpathlineto{\pgfqpoint{3.009455in}{2.064776in}}%
\pgfpathlineto{\pgfqpoint{3.013964in}{2.061740in}}%
\pgfpathlineto{\pgfqpoint{3.018473in}{2.064169in}}%
\pgfpathlineto{\pgfqpoint{3.022982in}{2.070849in}}%
\pgfpathlineto{\pgfqpoint{3.027491in}{2.075100in}}%
\pgfpathlineto{\pgfqpoint{3.032000in}{2.083602in}}%
\pgfpathlineto{\pgfqpoint{3.036509in}{2.077529in}}%
\pgfpathlineto{\pgfqpoint{3.041018in}{2.085423in}}%
\pgfpathlineto{\pgfqpoint{3.045527in}{2.082387in}}%
\pgfpathlineto{\pgfqpoint{3.050036in}{2.087245in}}%
\pgfpathlineto{\pgfqpoint{3.054545in}{2.082994in}}%
\pgfpathlineto{\pgfqpoint{3.059055in}{2.095747in}}%
\pgfpathlineto{\pgfqpoint{3.077091in}{2.107285in}}%
\pgfpathlineto{\pgfqpoint{3.081600in}{2.101212in}}%
\pgfpathlineto{\pgfqpoint{3.086109in}{2.117001in}}%
\pgfpathlineto{\pgfqpoint{3.090618in}{2.113965in}}%
\pgfpathlineto{\pgfqpoint{3.095127in}{2.115179in}}%
\pgfpathlineto{\pgfqpoint{3.099636in}{2.128539in}}%
\pgfpathlineto{\pgfqpoint{3.104145in}{2.122467in}}%
\pgfpathlineto{\pgfqpoint{3.108655in}{2.124896in}}%
\pgfpathlineto{\pgfqpoint{3.122182in}{2.144328in}}%
\pgfpathlineto{\pgfqpoint{3.126691in}{2.152223in}}%
\pgfpathlineto{\pgfqpoint{3.131200in}{2.149186in}}%
\pgfpathlineto{\pgfqpoint{3.135709in}{2.141292in}}%
\pgfpathlineto{\pgfqpoint{3.140218in}{2.155259in}}%
\pgfpathlineto{\pgfqpoint{3.144727in}{2.163153in}}%
\pgfpathlineto{\pgfqpoint{3.149236in}{2.159510in}}%
\pgfpathlineto{\pgfqpoint{3.162764in}{2.177121in}}%
\pgfpathlineto{\pgfqpoint{3.167273in}{2.172870in}}%
\pgfpathlineto{\pgfqpoint{3.171782in}{2.183800in}}%
\pgfpathlineto{\pgfqpoint{3.180800in}{2.175906in}}%
\pgfpathlineto{\pgfqpoint{3.185309in}{2.180764in}}%
\pgfpathlineto{\pgfqpoint{3.189818in}{2.192909in}}%
\pgfpathlineto{\pgfqpoint{3.194327in}{2.197768in}}%
\pgfpathlineto{\pgfqpoint{3.198836in}{2.198375in}}%
\pgfpathlineto{\pgfqpoint{3.203345in}{2.192302in}}%
\pgfpathlineto{\pgfqpoint{3.207855in}{2.209306in}}%
\pgfpathlineto{\pgfqpoint{3.212364in}{2.201411in}}%
\pgfpathlineto{\pgfqpoint{3.216873in}{2.198375in}}%
\pgfpathlineto{\pgfqpoint{3.221382in}{2.209306in}}%
\pgfpathlineto{\pgfqpoint{3.225891in}{2.211735in}}%
\pgfpathlineto{\pgfqpoint{3.230400in}{2.223273in}}%
\pgfpathlineto{\pgfqpoint{3.234909in}{2.213556in}}%
\pgfpathlineto{\pgfqpoint{3.239418in}{2.220844in}}%
\pgfpathlineto{\pgfqpoint{3.243927in}{2.219629in}}%
\pgfpathlineto{\pgfqpoint{3.257455in}{2.237240in}}%
\pgfpathlineto{\pgfqpoint{3.261964in}{2.229345in}}%
\pgfpathlineto{\pgfqpoint{3.266473in}{2.236633in}}%
\pgfpathlineto{\pgfqpoint{3.270982in}{2.246349in}}%
\pgfpathlineto{\pgfqpoint{3.275491in}{2.244527in}}%
\pgfpathlineto{\pgfqpoint{3.280000in}{2.256065in}}%
\pgfpathlineto{\pgfqpoint{3.284509in}{2.254851in}}%
\pgfpathlineto{\pgfqpoint{3.289018in}{2.260316in}}%
\pgfpathlineto{\pgfqpoint{3.293527in}{2.256672in}}%
\pgfpathlineto{\pgfqpoint{3.298036in}{2.273676in}}%
\pgfpathlineto{\pgfqpoint{3.302545in}{2.271247in}}%
\pgfpathlineto{\pgfqpoint{3.307055in}{2.278534in}}%
\pgfpathlineto{\pgfqpoint{3.320582in}{2.279141in}}%
\pgfpathlineto{\pgfqpoint{3.325091in}{2.288250in}}%
\pgfpathlineto{\pgfqpoint{3.329600in}{2.285821in}}%
\pgfpathlineto{\pgfqpoint{3.338618in}{2.293716in}}%
\pgfpathlineto{\pgfqpoint{3.343127in}{2.304646in}}%
\pgfpathlineto{\pgfqpoint{3.347636in}{2.297359in}}%
\pgfpathlineto{\pgfqpoint{3.352145in}{2.318613in}}%
\pgfpathlineto{\pgfqpoint{3.356655in}{2.311933in}}%
\pgfpathlineto{\pgfqpoint{3.361164in}{2.308897in}}%
\pgfpathlineto{\pgfqpoint{3.365673in}{2.317399in}}%
\pgfpathlineto{\pgfqpoint{3.370182in}{2.321650in}}%
\pgfpathlineto{\pgfqpoint{3.374691in}{2.328330in}}%
\pgfpathlineto{\pgfqpoint{3.379200in}{2.339260in}}%
\pgfpathlineto{\pgfqpoint{3.383709in}{2.332581in}}%
\pgfpathlineto{\pgfqpoint{3.388218in}{2.340475in}}%
\pgfpathlineto{\pgfqpoint{3.392727in}{2.342904in}}%
\pgfpathlineto{\pgfqpoint{3.397236in}{2.337439in}}%
\pgfpathlineto{\pgfqpoint{3.406255in}{2.353228in}}%
\pgfpathlineto{\pgfqpoint{3.410764in}{2.349584in}}%
\pgfpathlineto{\pgfqpoint{3.415273in}{2.356871in}}%
\pgfpathlineto{\pgfqpoint{3.419782in}{2.366587in}}%
\pgfpathlineto{\pgfqpoint{3.424291in}{2.367195in}}%
\pgfpathlineto{\pgfqpoint{3.428800in}{2.381162in}}%
\pgfpathlineto{\pgfqpoint{3.433309in}{2.382376in}}%
\pgfpathlineto{\pgfqpoint{3.437818in}{2.380554in}}%
\pgfpathlineto{\pgfqpoint{3.442327in}{2.379947in}}%
\pgfpathlineto{\pgfqpoint{3.446836in}{2.396343in}}%
\pgfpathlineto{\pgfqpoint{3.451345in}{2.403023in}}%
\pgfpathlineto{\pgfqpoint{3.455855in}{2.397558in}}%
\pgfpathlineto{\pgfqpoint{3.460364in}{2.402416in}}%
\pgfpathlineto{\pgfqpoint{3.464873in}{2.404845in}}%
\pgfpathlineto{\pgfqpoint{3.469382in}{2.403023in}}%
\pgfpathlineto{\pgfqpoint{3.473891in}{2.415776in}}%
\pgfpathlineto{\pgfqpoint{3.478400in}{2.403631in}}%
\pgfpathlineto{\pgfqpoint{3.482909in}{2.416990in}}%
\pgfpathlineto{\pgfqpoint{3.487418in}{2.421241in}}%
\pgfpathlineto{\pgfqpoint{3.491927in}{2.420634in}}%
\pgfpathlineto{\pgfqpoint{3.496436in}{2.418205in}}%
\pgfpathlineto{\pgfqpoint{3.500945in}{2.429743in}}%
\pgfpathlineto{\pgfqpoint{3.505455in}{2.430350in}}%
\pgfpathlineto{\pgfqpoint{3.509964in}{2.428528in}}%
\pgfpathlineto{\pgfqpoint{3.514473in}{2.433994in}}%
\pgfpathlineto{\pgfqpoint{3.518982in}{2.435208in}}%
\pgfpathlineto{\pgfqpoint{3.528000in}{2.441888in}}%
\pgfpathlineto{\pgfqpoint{3.532509in}{2.457070in}}%
\pgfpathlineto{\pgfqpoint{3.537018in}{2.453426in}}%
\pgfpathlineto{\pgfqpoint{3.541527in}{2.461321in}}%
\pgfpathlineto{\pgfqpoint{3.546036in}{2.460106in}}%
\pgfpathlineto{\pgfqpoint{3.550545in}{2.454034in}}%
\pgfpathlineto{\pgfqpoint{3.555055in}{2.464357in}}%
\pgfpathlineto{\pgfqpoint{3.559564in}{2.468001in}}%
\pgfpathlineto{\pgfqpoint{3.564073in}{2.468608in}}%
\pgfpathlineto{\pgfqpoint{3.568582in}{2.470430in}}%
\pgfpathlineto{\pgfqpoint{3.573091in}{2.481361in}}%
\pgfpathlineto{\pgfqpoint{3.577600in}{2.480753in}}%
\pgfpathlineto{\pgfqpoint{3.582109in}{2.483182in}}%
\pgfpathlineto{\pgfqpoint{3.591127in}{2.497149in}}%
\pgfpathlineto{\pgfqpoint{3.595636in}{2.483790in}}%
\pgfpathlineto{\pgfqpoint{3.600145in}{2.503222in}}%
\pgfpathlineto{\pgfqpoint{3.604655in}{2.509902in}}%
\pgfpathlineto{\pgfqpoint{3.609164in}{2.506866in}}%
\pgfpathlineto{\pgfqpoint{3.613673in}{2.506258in}}%
\pgfpathlineto{\pgfqpoint{3.618182in}{2.514760in}}%
\pgfpathlineto{\pgfqpoint{3.622691in}{2.511117in}}%
\pgfpathlineto{\pgfqpoint{3.627200in}{2.519618in}}%
\pgfpathlineto{\pgfqpoint{3.631709in}{2.525691in}}%
\pgfpathlineto{\pgfqpoint{3.636218in}{2.520833in}}%
\pgfpathlineto{\pgfqpoint{3.640727in}{2.526298in}}%
\pgfpathlineto{\pgfqpoint{3.645236in}{2.528727in}}%
\pgfpathlineto{\pgfqpoint{3.649745in}{2.539658in}}%
\pgfpathlineto{\pgfqpoint{3.654255in}{2.536014in}}%
\pgfpathlineto{\pgfqpoint{3.658764in}{2.553018in}}%
\pgfpathlineto{\pgfqpoint{3.667782in}{2.543909in}}%
\pgfpathlineto{\pgfqpoint{3.672291in}{2.559698in}}%
\pgfpathlineto{\pgfqpoint{3.676800in}{2.566378in}}%
\pgfpathlineto{\pgfqpoint{3.681309in}{2.562127in}}%
\pgfpathlineto{\pgfqpoint{3.685818in}{2.563949in}}%
\pgfpathlineto{\pgfqpoint{3.690327in}{2.557269in}}%
\pgfpathlineto{\pgfqpoint{3.694836in}{2.575487in}}%
\pgfpathlineto{\pgfqpoint{3.699345in}{2.577916in}}%
\pgfpathlineto{\pgfqpoint{3.703855in}{2.578523in}}%
\pgfpathlineto{\pgfqpoint{3.708364in}{2.587632in}}%
\pgfpathlineto{\pgfqpoint{3.712873in}{2.582774in}}%
\pgfpathlineto{\pgfqpoint{3.717382in}{2.583381in}}%
\pgfpathlineto{\pgfqpoint{3.721891in}{2.597348in}}%
\pgfpathlineto{\pgfqpoint{3.726400in}{2.597956in}}%
\pgfpathlineto{\pgfqpoint{3.730909in}{2.610101in}}%
\pgfpathlineto{\pgfqpoint{3.735418in}{2.610101in}}%
\pgfpathlineto{\pgfqpoint{3.744436in}{2.626497in}}%
\pgfpathlineto{\pgfqpoint{3.748945in}{2.624675in}}%
\pgfpathlineto{\pgfqpoint{3.753455in}{2.624675in}}%
\pgfpathlineto{\pgfqpoint{3.766982in}{2.639857in}}%
\pgfpathlineto{\pgfqpoint{3.771491in}{2.641071in}}%
\pgfpathlineto{\pgfqpoint{3.776000in}{2.633784in}}%
\pgfpathlineto{\pgfqpoint{3.780509in}{2.650180in}}%
\pgfpathlineto{\pgfqpoint{3.785018in}{2.655646in}}%
\pgfpathlineto{\pgfqpoint{3.789527in}{2.650788in}}%
\pgfpathlineto{\pgfqpoint{3.794036in}{2.661111in}}%
\pgfpathlineto{\pgfqpoint{3.798545in}{2.661111in}}%
\pgfpathlineto{\pgfqpoint{3.807564in}{2.663540in}}%
\pgfpathlineto{\pgfqpoint{3.812073in}{2.677507in}}%
\pgfpathlineto{\pgfqpoint{3.816582in}{2.669006in}}%
\pgfpathlineto{\pgfqpoint{3.821091in}{2.682973in}}%
\pgfpathlineto{\pgfqpoint{3.825600in}{2.683580in}}%
\pgfpathlineto{\pgfqpoint{3.830109in}{2.687224in}}%
\pgfpathlineto{\pgfqpoint{3.834618in}{2.686009in}}%
\pgfpathlineto{\pgfqpoint{3.839127in}{2.707263in}}%
\pgfpathlineto{\pgfqpoint{3.843636in}{2.703012in}}%
\pgfpathlineto{\pgfqpoint{3.848145in}{2.707871in}}%
\pgfpathlineto{\pgfqpoint{3.852655in}{2.696940in}}%
\pgfpathlineto{\pgfqpoint{3.857164in}{2.709692in}}%
\pgfpathlineto{\pgfqpoint{3.861673in}{2.703012in}}%
\pgfpathlineto{\pgfqpoint{3.866182in}{2.713336in}}%
\pgfpathlineto{\pgfqpoint{3.870691in}{2.729732in}}%
\pgfpathlineto{\pgfqpoint{3.875200in}{2.712729in}}%
\pgfpathlineto{\pgfqpoint{3.879709in}{2.730339in}}%
\pgfpathlineto{\pgfqpoint{3.884218in}{2.738234in}}%
\pgfpathlineto{\pgfqpoint{3.888727in}{2.737627in}}%
\pgfpathlineto{\pgfqpoint{3.893236in}{2.747343in}}%
\pgfpathlineto{\pgfqpoint{3.897745in}{2.763739in}}%
\pgfpathlineto{\pgfqpoint{3.902255in}{2.754023in}}%
\pgfpathlineto{\pgfqpoint{3.906764in}{2.748557in}}%
\pgfpathlineto{\pgfqpoint{3.911273in}{2.758881in}}%
\pgfpathlineto{\pgfqpoint{3.915782in}{2.757059in}}%
\pgfpathlineto{\pgfqpoint{3.920291in}{2.766775in}}%
\pgfpathlineto{\pgfqpoint{3.924800in}{2.758274in}}%
\pgfpathlineto{\pgfqpoint{3.929309in}{2.763739in}}%
\pgfpathlineto{\pgfqpoint{3.933818in}{2.771633in}}%
\pgfpathlineto{\pgfqpoint{3.938327in}{2.795317in}}%
\pgfpathlineto{\pgfqpoint{3.947345in}{2.785601in}}%
\pgfpathlineto{\pgfqpoint{3.951855in}{2.795924in}}%
\pgfpathlineto{\pgfqpoint{3.956364in}{2.792280in}}%
\pgfpathlineto{\pgfqpoint{3.960873in}{2.799568in}}%
\pgfpathlineto{\pgfqpoint{3.965382in}{2.811713in}}%
\pgfpathlineto{\pgfqpoint{3.969891in}{2.814142in}}%
\pgfpathlineto{\pgfqpoint{3.974400in}{2.819000in}}%
\pgfpathlineto{\pgfqpoint{3.978909in}{2.817178in}}%
\pgfpathlineto{\pgfqpoint{3.983418in}{2.817178in}}%
\pgfpathlineto{\pgfqpoint{3.987927in}{2.834789in}}%
\pgfpathlineto{\pgfqpoint{3.992436in}{2.835396in}}%
\pgfpathlineto{\pgfqpoint{4.001455in}{2.850578in}}%
\pgfpathlineto{\pgfqpoint{4.005964in}{2.846327in}}%
\pgfpathlineto{\pgfqpoint{4.010473in}{2.857865in}}%
\pgfpathlineto{\pgfqpoint{4.014982in}{2.849971in}}%
\pgfpathlineto{\pgfqpoint{4.019491in}{2.862116in}}%
\pgfpathlineto{\pgfqpoint{4.024000in}{2.866367in}}%
\pgfpathlineto{\pgfqpoint{4.028509in}{2.874261in}}%
\pgfpathlineto{\pgfqpoint{4.033018in}{2.877298in}}%
\pgfpathlineto{\pgfqpoint{4.042036in}{2.873047in}}%
\pgfpathlineto{\pgfqpoint{4.046545in}{2.893694in}}%
\pgfpathlineto{\pgfqpoint{4.051055in}{2.902803in}}%
\pgfpathlineto{\pgfqpoint{4.055564in}{2.891872in}}%
\pgfpathlineto{\pgfqpoint{4.064582in}{2.902803in}}%
\pgfpathlineto{\pgfqpoint{4.069091in}{2.902803in}}%
\pgfpathlineto{\pgfqpoint{4.073600in}{2.904017in}}%
\pgfpathlineto{\pgfqpoint{4.078109in}{2.899159in}}%
\pgfpathlineto{\pgfqpoint{4.082618in}{2.914948in}}%
\pgfpathlineto{\pgfqpoint{4.087127in}{2.905839in}}%
\pgfpathlineto{\pgfqpoint{4.091636in}{2.915555in}}%
\pgfpathlineto{\pgfqpoint{4.096145in}{2.907661in}}%
\pgfpathlineto{\pgfqpoint{4.100655in}{2.926486in}}%
\pgfpathlineto{\pgfqpoint{4.105164in}{2.921021in}}%
\pgfpathlineto{\pgfqpoint{4.109673in}{2.923450in}}%
\pgfpathlineto{\pgfqpoint{4.114182in}{2.930130in}}%
\pgfpathlineto{\pgfqpoint{4.118691in}{2.927701in}}%
\pgfpathlineto{\pgfqpoint{4.123200in}{2.936202in}}%
\pgfpathlineto{\pgfqpoint{4.127709in}{2.930130in}}%
\pgfpathlineto{\pgfqpoint{4.132218in}{2.934381in}}%
\pgfpathlineto{\pgfqpoint{4.136727in}{2.936202in}}%
\pgfpathlineto{\pgfqpoint{4.141236in}{2.939846in}}%
\pgfpathlineto{\pgfqpoint{4.145745in}{2.940453in}}%
\pgfpathlineto{\pgfqpoint{4.159273in}{2.961100in}}%
\pgfpathlineto{\pgfqpoint{4.163782in}{2.956849in}}%
\pgfpathlineto{\pgfqpoint{4.168291in}{2.963529in}}%
\pgfpathlineto{\pgfqpoint{4.172800in}{2.964137in}}%
\pgfpathlineto{\pgfqpoint{4.177309in}{2.970817in}}%
\pgfpathlineto{\pgfqpoint{4.181818in}{2.972031in}}%
\pgfpathlineto{\pgfqpoint{4.186327in}{2.972031in}}%
\pgfpathlineto{\pgfqpoint{4.190836in}{2.978711in}}%
\pgfpathlineto{\pgfqpoint{4.199855in}{2.983569in}}%
\pgfpathlineto{\pgfqpoint{4.204364in}{2.989642in}}%
\pgfpathlineto{\pgfqpoint{4.208873in}{2.986605in}}%
\pgfpathlineto{\pgfqpoint{4.213382in}{2.992071in}}%
\pgfpathlineto{\pgfqpoint{4.217891in}{2.986605in}}%
\pgfpathlineto{\pgfqpoint{4.222400in}{2.990249in}}%
\pgfpathlineto{\pgfqpoint{4.226909in}{3.009074in}}%
\pgfpathlineto{\pgfqpoint{4.231418in}{3.000573in}}%
\pgfpathlineto{\pgfqpoint{4.235927in}{3.004823in}}%
\pgfpathlineto{\pgfqpoint{4.240436in}{3.013932in}}%
\pgfpathlineto{\pgfqpoint{4.244945in}{3.011503in}}%
\pgfpathlineto{\pgfqpoint{4.249455in}{3.013325in}}%
\pgfpathlineto{\pgfqpoint{4.262982in}{3.027900in}}%
\pgfpathlineto{\pgfqpoint{4.267491in}{3.029721in}}%
\pgfpathlineto{\pgfqpoint{4.276509in}{3.041259in}}%
\pgfpathlineto{\pgfqpoint{4.281018in}{3.044296in}}%
\pgfpathlineto{\pgfqpoint{4.285527in}{3.044296in}}%
\pgfpathlineto{\pgfqpoint{4.290036in}{3.041867in}}%
\pgfpathlineto{\pgfqpoint{4.294545in}{3.046117in}}%
\pgfpathlineto{\pgfqpoint{4.299055in}{3.040045in}}%
\pgfpathlineto{\pgfqpoint{4.303564in}{3.063728in}}%
\pgfpathlineto{\pgfqpoint{4.308073in}{3.066765in}}%
\pgfpathlineto{\pgfqpoint{4.312582in}{3.064943in}}%
\pgfpathlineto{\pgfqpoint{4.317091in}{3.078910in}}%
\pgfpathlineto{\pgfqpoint{4.321600in}{3.075873in}}%
\pgfpathlineto{\pgfqpoint{4.326109in}{3.077695in}}%
\pgfpathlineto{\pgfqpoint{4.335127in}{3.100771in}}%
\pgfpathlineto{\pgfqpoint{4.344145in}{3.088626in}}%
\pgfpathlineto{\pgfqpoint{4.348655in}{3.101986in}}%
\pgfpathlineto{\pgfqpoint{4.353164in}{3.094699in}}%
\pgfpathlineto{\pgfqpoint{4.357673in}{3.108059in}}%
\pgfpathlineto{\pgfqpoint{4.362182in}{3.101986in}}%
\pgfpathlineto{\pgfqpoint{4.371200in}{3.111702in}}%
\pgfpathlineto{\pgfqpoint{4.375709in}{3.111095in}}%
\pgfpathlineto{\pgfqpoint{4.380218in}{3.117168in}}%
\pgfpathlineto{\pgfqpoint{4.384727in}{3.120811in}}%
\pgfpathlineto{\pgfqpoint{4.389236in}{3.122633in}}%
\pgfpathlineto{\pgfqpoint{4.393745in}{3.125669in}}%
\pgfpathlineto{\pgfqpoint{4.398255in}{3.124455in}}%
\pgfpathlineto{\pgfqpoint{4.402764in}{3.137815in}}%
\pgfpathlineto{\pgfqpoint{4.407273in}{3.145709in}}%
\pgfpathlineto{\pgfqpoint{4.411782in}{3.142673in}}%
\pgfpathlineto{\pgfqpoint{4.416291in}{3.145102in}}%
\pgfpathlineto{\pgfqpoint{4.420800in}{3.152996in}}%
\pgfpathlineto{\pgfqpoint{4.425309in}{3.152996in}}%
\pgfpathlineto{\pgfqpoint{4.429818in}{3.143887in}}%
\pgfpathlineto{\pgfqpoint{4.434327in}{3.154211in}}%
\pgfpathlineto{\pgfqpoint{4.438836in}{3.168178in}}%
\pgfpathlineto{\pgfqpoint{4.443345in}{3.156640in}}%
\pgfpathlineto{\pgfqpoint{4.456873in}{3.177894in}}%
\pgfpathlineto{\pgfqpoint{4.461382in}{3.178501in}}%
\pgfpathlineto{\pgfqpoint{4.465891in}{3.183967in}}%
\pgfpathlineto{\pgfqpoint{4.470400in}{3.180930in}}%
\pgfpathlineto{\pgfqpoint{4.474909in}{3.190647in}}%
\pgfpathlineto{\pgfqpoint{4.479418in}{3.196719in}}%
\pgfpathlineto{\pgfqpoint{4.483927in}{3.187610in}}%
\pgfpathlineto{\pgfqpoint{4.488436in}{3.208865in}}%
\pgfpathlineto{\pgfqpoint{4.492945in}{3.214937in}}%
\pgfpathlineto{\pgfqpoint{4.497455in}{3.208257in}}%
\pgfpathlineto{\pgfqpoint{4.501964in}{3.211294in}}%
\pgfpathlineto{\pgfqpoint{4.510982in}{3.220403in}}%
\pgfpathlineto{\pgfqpoint{4.515491in}{3.231941in}}%
\pgfpathlineto{\pgfqpoint{4.520000in}{3.222224in}}%
\pgfpathlineto{\pgfqpoint{4.524509in}{3.231333in}}%
\pgfpathlineto{\pgfqpoint{4.529018in}{3.246515in}}%
\pgfpathlineto{\pgfqpoint{4.533527in}{3.236799in}}%
\pgfpathlineto{\pgfqpoint{4.538036in}{3.244693in}}%
\pgfpathlineto{\pgfqpoint{4.542545in}{3.248944in}}%
\pgfpathlineto{\pgfqpoint{4.547055in}{3.251373in}}%
\pgfpathlineto{\pgfqpoint{4.551564in}{3.261089in}}%
\pgfpathlineto{\pgfqpoint{4.556073in}{3.259875in}}%
\pgfpathlineto{\pgfqpoint{4.560582in}{3.257446in}}%
\pgfpathlineto{\pgfqpoint{4.569600in}{3.275057in}}%
\pgfpathlineto{\pgfqpoint{4.574109in}{3.278700in}}%
\pgfpathlineto{\pgfqpoint{4.578618in}{3.268984in}}%
\pgfpathlineto{\pgfqpoint{4.583127in}{3.282344in}}%
\pgfpathlineto{\pgfqpoint{4.587636in}{3.282344in}}%
\pgfpathlineto{\pgfqpoint{4.596655in}{3.300562in}}%
\pgfpathlineto{\pgfqpoint{4.601164in}{3.302991in}}%
\pgfpathlineto{\pgfqpoint{4.605673in}{3.298740in}}%
\pgfpathlineto{\pgfqpoint{4.610182in}{3.304205in}}%
\pgfpathlineto{\pgfqpoint{4.614691in}{3.321209in}}%
\pgfpathlineto{\pgfqpoint{4.619200in}{3.324245in}}%
\pgfpathlineto{\pgfqpoint{4.623709in}{3.323638in}}%
\pgfpathlineto{\pgfqpoint{4.628218in}{3.324245in}}%
\pgfpathlineto{\pgfqpoint{4.632727in}{3.326067in}}%
\pgfpathlineto{\pgfqpoint{4.637236in}{3.330318in}}%
\pgfpathlineto{\pgfqpoint{4.641745in}{3.338212in}}%
\pgfpathlineto{\pgfqpoint{4.646255in}{3.326674in}}%
\pgfpathlineto{\pgfqpoint{4.650764in}{3.338819in}}%
\pgfpathlineto{\pgfqpoint{4.655273in}{3.345499in}}%
\pgfpathlineto{\pgfqpoint{4.659782in}{3.343070in}}%
\pgfpathlineto{\pgfqpoint{4.664291in}{3.347928in}}%
\pgfpathlineto{\pgfqpoint{4.668800in}{3.357645in}}%
\pgfpathlineto{\pgfqpoint{4.673309in}{3.353394in}}%
\pgfpathlineto{\pgfqpoint{4.677818in}{3.363717in}}%
\pgfpathlineto{\pgfqpoint{4.682327in}{3.377077in}}%
\pgfpathlineto{\pgfqpoint{4.686836in}{3.369790in}}%
\pgfpathlineto{\pgfqpoint{4.691345in}{3.369790in}}%
\pgfpathlineto{\pgfqpoint{4.695855in}{3.363110in}}%
\pgfpathlineto{\pgfqpoint{4.704873in}{3.383757in}}%
\pgfpathlineto{\pgfqpoint{4.709382in}{3.385579in}}%
\pgfpathlineto{\pgfqpoint{4.718400in}{3.384972in}}%
\pgfpathlineto{\pgfqpoint{4.722909in}{3.391044in}}%
\pgfpathlineto{\pgfqpoint{4.727418in}{3.393473in}}%
\pgfpathlineto{\pgfqpoint{4.731927in}{3.408655in}}%
\pgfpathlineto{\pgfqpoint{4.736436in}{3.400153in}}%
\pgfpathlineto{\pgfqpoint{4.745455in}{3.407440in}}%
\pgfpathlineto{\pgfqpoint{4.749964in}{3.418371in}}%
\pgfpathlineto{\pgfqpoint{4.754473in}{3.420800in}}%
\pgfpathlineto{\pgfqpoint{4.758982in}{3.427480in}}%
\pgfpathlineto{\pgfqpoint{4.763491in}{3.424444in}}%
\pgfpathlineto{\pgfqpoint{4.768000in}{3.432338in}}%
\pgfpathlineto{\pgfqpoint{4.772509in}{3.436589in}}%
\pgfpathlineto{\pgfqpoint{4.781527in}{3.438411in}}%
\pgfpathlineto{\pgfqpoint{4.786036in}{3.445091in}}%
\pgfpathlineto{\pgfqpoint{4.790545in}{3.436589in}}%
\pgfpathlineto{\pgfqpoint{4.795055in}{3.452378in}}%
\pgfpathlineto{\pgfqpoint{4.799564in}{3.461487in}}%
\pgfpathlineto{\pgfqpoint{4.804073in}{3.462094in}}%
\pgfpathlineto{\pgfqpoint{4.808582in}{3.468774in}}%
\pgfpathlineto{\pgfqpoint{4.813091in}{3.467560in}}%
\pgfpathlineto{\pgfqpoint{4.817600in}{3.474240in}}%
\pgfpathlineto{\pgfqpoint{4.822109in}{3.469382in}}%
\pgfpathlineto{\pgfqpoint{4.826618in}{3.468774in}}%
\pgfpathlineto{\pgfqpoint{4.831127in}{3.473025in}}%
\pgfpathlineto{\pgfqpoint{4.835636in}{3.475454in}}%
\pgfpathlineto{\pgfqpoint{4.840145in}{3.493065in}}%
\pgfpathlineto{\pgfqpoint{4.844655in}{3.503996in}}%
\pgfpathlineto{\pgfqpoint{4.849164in}{3.495494in}}%
\pgfpathlineto{\pgfqpoint{4.853673in}{3.507032in}}%
\pgfpathlineto{\pgfqpoint{4.862691in}{3.511890in}}%
\pgfpathlineto{\pgfqpoint{4.867200in}{3.516141in}}%
\pgfpathlineto{\pgfqpoint{4.871709in}{3.523428in}}%
\pgfpathlineto{\pgfqpoint{4.876218in}{3.520392in}}%
\pgfpathlineto{\pgfqpoint{4.880727in}{3.527072in}}%
\pgfpathlineto{\pgfqpoint{4.885236in}{3.536181in}}%
\pgfpathlineto{\pgfqpoint{4.889745in}{3.534359in}}%
\pgfpathlineto{\pgfqpoint{4.894255in}{3.543468in}}%
\pgfpathlineto{\pgfqpoint{4.898764in}{3.542253in}}%
\pgfpathlineto{\pgfqpoint{4.903273in}{3.544075in}}%
\pgfpathlineto{\pgfqpoint{4.907782in}{3.548326in}}%
\pgfpathlineto{\pgfqpoint{4.912291in}{3.559864in}}%
\pgfpathlineto{\pgfqpoint{4.916800in}{3.561079in}}%
\pgfpathlineto{\pgfqpoint{4.921309in}{3.553791in}}%
\pgfpathlineto{\pgfqpoint{4.925818in}{3.559257in}}%
\pgfpathlineto{\pgfqpoint{4.930327in}{3.573224in}}%
\pgfpathlineto{\pgfqpoint{4.934836in}{3.565329in}}%
\pgfpathlineto{\pgfqpoint{4.939345in}{3.581726in}}%
\pgfpathlineto{\pgfqpoint{4.943855in}{3.583547in}}%
\pgfpathlineto{\pgfqpoint{4.948364in}{3.584155in}}%
\pgfpathlineto{\pgfqpoint{4.957382in}{3.598729in}}%
\pgfpathlineto{\pgfqpoint{4.961891in}{3.592656in}}%
\pgfpathlineto{\pgfqpoint{4.966400in}{3.592049in}}%
\pgfpathlineto{\pgfqpoint{4.970909in}{3.606624in}}%
\pgfpathlineto{\pgfqpoint{4.975418in}{3.609660in}}%
\pgfpathlineto{\pgfqpoint{4.979927in}{3.603587in}}%
\pgfpathlineto{\pgfqpoint{4.984436in}{3.617554in}}%
\pgfpathlineto{\pgfqpoint{4.988945in}{3.615733in}}%
\pgfpathlineto{\pgfqpoint{4.993455in}{3.626663in}}%
\pgfpathlineto{\pgfqpoint{4.997964in}{3.627878in}}%
\pgfpathlineto{\pgfqpoint{5.002473in}{3.633950in}}%
\pgfpathlineto{\pgfqpoint{5.006982in}{3.635772in}}%
\pgfpathlineto{\pgfqpoint{5.011491in}{3.641238in}}%
\pgfpathlineto{\pgfqpoint{5.016000in}{3.636987in}}%
\pgfpathlineto{\pgfqpoint{5.020509in}{3.651561in}}%
\pgfpathlineto{\pgfqpoint{5.025018in}{3.643667in}}%
\pgfpathlineto{\pgfqpoint{5.029527in}{3.652168in}}%
\pgfpathlineto{\pgfqpoint{5.034036in}{3.668565in}}%
\pgfpathlineto{\pgfqpoint{5.038545in}{3.667350in}}%
\pgfpathlineto{\pgfqpoint{5.043055in}{3.664314in}}%
\pgfpathlineto{\pgfqpoint{5.047564in}{3.684354in}}%
\pgfpathlineto{\pgfqpoint{5.052073in}{3.683746in}}%
\pgfpathlineto{\pgfqpoint{5.056582in}{3.684961in}}%
\pgfpathlineto{\pgfqpoint{5.061091in}{3.677674in}}%
\pgfpathlineto{\pgfqpoint{5.065600in}{3.692248in}}%
\pgfpathlineto{\pgfqpoint{5.070109in}{3.688604in}}%
\pgfpathlineto{\pgfqpoint{5.074618in}{3.700750in}}%
\pgfpathlineto{\pgfqpoint{5.083636in}{3.709251in}}%
\pgfpathlineto{\pgfqpoint{5.088145in}{3.712895in}}%
\pgfpathlineto{\pgfqpoint{5.092655in}{3.718968in}}%
\pgfpathlineto{\pgfqpoint{5.101673in}{3.725040in}}%
\pgfpathlineto{\pgfqpoint{5.106182in}{3.738400in}}%
\pgfpathlineto{\pgfqpoint{5.110691in}{3.740222in}}%
\pgfpathlineto{\pgfqpoint{5.115200in}{3.733542in}}%
\pgfpathlineto{\pgfqpoint{5.124218in}{3.765120in}}%
\pgfpathlineto{\pgfqpoint{5.128727in}{3.755404in}}%
\pgfpathlineto{\pgfqpoint{5.133236in}{3.759047in}}%
\pgfpathlineto{\pgfqpoint{5.137745in}{3.759654in}}%
\pgfpathlineto{\pgfqpoint{5.142255in}{3.769371in}}%
\pgfpathlineto{\pgfqpoint{5.146764in}{3.783338in}}%
\pgfpathlineto{\pgfqpoint{5.151273in}{3.776658in}}%
\pgfpathlineto{\pgfqpoint{5.155782in}{3.785767in}}%
\pgfpathlineto{\pgfqpoint{5.160291in}{3.777265in}}%
\pgfpathlineto{\pgfqpoint{5.164800in}{3.788196in}}%
\pgfpathlineto{\pgfqpoint{5.169309in}{3.789410in}}%
\pgfpathlineto{\pgfqpoint{5.173818in}{3.806414in}}%
\pgfpathlineto{\pgfqpoint{5.178327in}{3.803985in}}%
\pgfpathlineto{\pgfqpoint{5.187345in}{3.814916in}}%
\pgfpathlineto{\pgfqpoint{5.191855in}{3.827061in}}%
\pgfpathlineto{\pgfqpoint{5.196364in}{3.822203in}}%
\pgfpathlineto{\pgfqpoint{5.200873in}{3.825846in}}%
\pgfpathlineto{\pgfqpoint{5.205382in}{3.822203in}}%
\pgfpathlineto{\pgfqpoint{5.209891in}{3.827061in}}%
\pgfpathlineto{\pgfqpoint{5.214400in}{3.839813in}}%
\pgfpathlineto{\pgfqpoint{5.218909in}{3.836777in}}%
\pgfpathlineto{\pgfqpoint{5.223418in}{3.840421in}}%
\pgfpathlineto{\pgfqpoint{5.227927in}{3.841635in}}%
\pgfpathlineto{\pgfqpoint{5.232436in}{3.839813in}}%
\pgfpathlineto{\pgfqpoint{5.236945in}{3.851352in}}%
\pgfpathlineto{\pgfqpoint{5.245964in}{3.841635in}}%
\pgfpathlineto{\pgfqpoint{5.250473in}{3.848315in}}%
\pgfpathlineto{\pgfqpoint{5.254982in}{3.848922in}}%
\pgfpathlineto{\pgfqpoint{5.259491in}{3.857424in}}%
\pgfpathlineto{\pgfqpoint{5.264000in}{3.857424in}}%
\pgfpathlineto{\pgfqpoint{5.268509in}{3.870177in}}%
\pgfpathlineto{\pgfqpoint{5.273018in}{3.866533in}}%
\pgfpathlineto{\pgfqpoint{5.277527in}{3.879893in}}%
\pgfpathlineto{\pgfqpoint{5.282036in}{3.871999in}}%
\pgfpathlineto{\pgfqpoint{5.286545in}{3.866533in}}%
\pgfpathlineto{\pgfqpoint{5.291055in}{3.875035in}}%
\pgfpathlineto{\pgfqpoint{5.295564in}{3.872606in}}%
\pgfpathlineto{\pgfqpoint{5.300073in}{3.888395in}}%
\pgfpathlineto{\pgfqpoint{5.304582in}{3.883537in}}%
\pgfpathlineto{\pgfqpoint{5.309091in}{3.884751in}}%
\pgfpathlineto{\pgfqpoint{5.313600in}{3.880500in}}%
\pgfpathlineto{\pgfqpoint{5.318109in}{3.895682in}}%
\pgfpathlineto{\pgfqpoint{5.322618in}{3.890217in}}%
\pgfpathlineto{\pgfqpoint{5.327127in}{3.901755in}}%
\pgfpathlineto{\pgfqpoint{5.340655in}{3.913293in}}%
\pgfpathlineto{\pgfqpoint{5.345164in}{3.911471in}}%
\pgfpathlineto{\pgfqpoint{5.349673in}{3.901755in}}%
\pgfpathlineto{\pgfqpoint{5.354182in}{3.912685in}}%
\pgfpathlineto{\pgfqpoint{5.358691in}{3.919365in}}%
\pgfpathlineto{\pgfqpoint{5.363200in}{3.919973in}}%
\pgfpathlineto{\pgfqpoint{5.367709in}{3.924223in}}%
\pgfpathlineto{\pgfqpoint{5.372218in}{3.931511in}}%
\pgfpathlineto{\pgfqpoint{5.376727in}{3.926652in}}%
\pgfpathlineto{\pgfqpoint{5.385745in}{3.936369in}}%
\pgfpathlineto{\pgfqpoint{5.390255in}{3.946692in}}%
\pgfpathlineto{\pgfqpoint{5.394764in}{3.941227in}}%
\pgfpathlineto{\pgfqpoint{5.399273in}{3.957623in}}%
\pgfpathlineto{\pgfqpoint{5.403782in}{3.946692in}}%
\pgfpathlineto{\pgfqpoint{5.408291in}{3.954587in}}%
\pgfpathlineto{\pgfqpoint{5.412800in}{3.966125in}}%
\pgfpathlineto{\pgfqpoint{5.417309in}{3.958838in}}%
\pgfpathlineto{\pgfqpoint{5.421818in}{3.958838in}}%
\pgfpathlineto{\pgfqpoint{5.426327in}{3.967339in}}%
\pgfpathlineto{\pgfqpoint{5.430836in}{3.967947in}}%
\pgfpathlineto{\pgfqpoint{5.435345in}{3.961874in}}%
\pgfpathlineto{\pgfqpoint{5.439855in}{3.972197in}}%
\pgfpathlineto{\pgfqpoint{5.444364in}{3.987379in}}%
\pgfpathlineto{\pgfqpoint{5.448873in}{3.992237in}}%
\pgfpathlineto{\pgfqpoint{5.453382in}{3.984343in}}%
\pgfpathlineto{\pgfqpoint{5.457891in}{3.992237in}}%
\pgfpathlineto{\pgfqpoint{5.462400in}{3.992844in}}%
\pgfpathlineto{\pgfqpoint{5.466909in}{3.989808in}}%
\pgfpathlineto{\pgfqpoint{5.471418in}{4.001953in}}%
\pgfpathlineto{\pgfqpoint{5.475927in}{3.996488in}}%
\pgfpathlineto{\pgfqpoint{5.480436in}{4.015313in}}%
\pgfpathlineto{\pgfqpoint{5.484945in}{4.014099in}}%
\pgfpathlineto{\pgfqpoint{5.493964in}{4.020171in}}%
\pgfpathlineto{\pgfqpoint{5.498473in}{4.011062in}}%
\pgfpathlineto{\pgfqpoint{5.502982in}{4.017742in}}%
\pgfpathlineto{\pgfqpoint{5.507491in}{4.018350in}}%
\pgfpathlineto{\pgfqpoint{5.512000in}{4.027459in}}%
\pgfpathlineto{\pgfqpoint{5.521018in}{4.033531in}}%
\pgfpathlineto{\pgfqpoint{5.525527in}{4.027459in}}%
\pgfpathlineto{\pgfqpoint{5.534545in}{4.056000in}}%
\pgfpathlineto{\pgfqpoint{5.534545in}{4.056000in}}%
\pgfusepath{stroke}%
\end{pgfscope}%
\begin{pgfscope}%
\pgfsetrectcap%
\pgfsetmiterjoin%
\pgfsetlinewidth{0.803000pt}%
\definecolor{currentstroke}{rgb}{0.000000,0.000000,0.000000}%
\pgfsetstrokecolor{currentstroke}%
\pgfsetdash{}{0pt}%
\pgfpathmoveto{\pgfqpoint{0.800000in}{0.528000in}}%
\pgfpathlineto{\pgfqpoint{0.800000in}{4.224000in}}%
\pgfusepath{stroke}%
\end{pgfscope}%
\begin{pgfscope}%
\pgfsetrectcap%
\pgfsetmiterjoin%
\pgfsetlinewidth{0.803000pt}%
\definecolor{currentstroke}{rgb}{0.000000,0.000000,0.000000}%
\pgfsetstrokecolor{currentstroke}%
\pgfsetdash{}{0pt}%
\pgfpathmoveto{\pgfqpoint{5.760000in}{0.528000in}}%
\pgfpathlineto{\pgfqpoint{5.760000in}{4.224000in}}%
\pgfusepath{stroke}%
\end{pgfscope}%
\begin{pgfscope}%
\pgfsetrectcap%
\pgfsetmiterjoin%
\pgfsetlinewidth{0.803000pt}%
\definecolor{currentstroke}{rgb}{0.000000,0.000000,0.000000}%
\pgfsetstrokecolor{currentstroke}%
\pgfsetdash{}{0pt}%
\pgfpathmoveto{\pgfqpoint{0.800000in}{0.528000in}}%
\pgfpathlineto{\pgfqpoint{5.760000in}{0.528000in}}%
\pgfusepath{stroke}%
\end{pgfscope}%
\begin{pgfscope}%
\pgfsetrectcap%
\pgfsetmiterjoin%
\pgfsetlinewidth{0.803000pt}%
\definecolor{currentstroke}{rgb}{0.000000,0.000000,0.000000}%
\pgfsetstrokecolor{currentstroke}%
\pgfsetdash{}{0pt}%
\pgfpathmoveto{\pgfqpoint{0.800000in}{4.224000in}}%
\pgfpathlineto{\pgfqpoint{5.760000in}{4.224000in}}%
\pgfusepath{stroke}%
\end{pgfscope}%
\begin{pgfscope}%
\definecolor{textcolor}{rgb}{0.000000,0.000000,0.000000}%
\pgfsetstrokecolor{textcolor}%
\pgfsetfillcolor{textcolor}%
\pgftext[x=3.280000in,y=4.307333in,,base]{\color{textcolor}\ttfamily\fontsize{12.000000}{14.400000}\selectfont Merge Sort  Comparisons vs Input size}%
\end{pgfscope}%
\begin{pgfscope}%
\pgfsetbuttcap%
\pgfsetmiterjoin%
\definecolor{currentfill}{rgb}{1.000000,1.000000,1.000000}%
\pgfsetfillcolor{currentfill}%
\pgfsetfillopacity{0.800000}%
\pgfsetlinewidth{1.003750pt}%
\definecolor{currentstroke}{rgb}{0.800000,0.800000,0.800000}%
\pgfsetstrokecolor{currentstroke}%
\pgfsetstrokeopacity{0.800000}%
\pgfsetdash{}{0pt}%
\pgfpathmoveto{\pgfqpoint{0.897222in}{3.907336in}}%
\pgfpathlineto{\pgfqpoint{1.759758in}{3.907336in}}%
\pgfpathquadraticcurveto{\pgfqpoint{1.787535in}{3.907336in}}{\pgfqpoint{1.787535in}{3.935114in}}%
\pgfpathlineto{\pgfqpoint{1.787535in}{4.126778in}}%
\pgfpathquadraticcurveto{\pgfqpoint{1.787535in}{4.154556in}}{\pgfqpoint{1.759758in}{4.154556in}}%
\pgfpathlineto{\pgfqpoint{0.897222in}{4.154556in}}%
\pgfpathquadraticcurveto{\pgfqpoint{0.869444in}{4.154556in}}{\pgfqpoint{0.869444in}{4.126778in}}%
\pgfpathlineto{\pgfqpoint{0.869444in}{3.935114in}}%
\pgfpathquadraticcurveto{\pgfqpoint{0.869444in}{3.907336in}}{\pgfqpoint{0.897222in}{3.907336in}}%
\pgfpathlineto{\pgfqpoint{0.897222in}{3.907336in}}%
\pgfpathclose%
\pgfusepath{stroke,fill}%
\end{pgfscope}%
\begin{pgfscope}%
\pgfsetrectcap%
\pgfsetroundjoin%
\pgfsetlinewidth{1.505625pt}%
\definecolor{currentstroke}{rgb}{0.000000,1.000000,0.498039}%
\pgfsetstrokecolor{currentstroke}%
\pgfsetdash{}{0pt}%
\pgfpathmoveto{\pgfqpoint{0.925000in}{4.041342in}}%
\pgfpathlineto{\pgfqpoint{1.063889in}{4.041342in}}%
\pgfpathlineto{\pgfqpoint{1.202778in}{4.041342in}}%
\pgfusepath{stroke}%
\end{pgfscope}%
\begin{pgfscope}%
\definecolor{textcolor}{rgb}{0.000000,0.000000,0.000000}%
\pgfsetstrokecolor{textcolor}%
\pgfsetfillcolor{textcolor}%
\pgftext[x=1.313889in,y=3.992731in,left,base]{\color{textcolor}\ttfamily\fontsize{10.000000}{12.000000}\selectfont Merge}%
\end{pgfscope}%
\end{pgfpicture}%
\makeatother%
\endgroup%

%% Creator: Matplotlib, PGF backend
%%
%% To include the figure in your LaTeX document, write
%%   \input{<filename>.pgf}
%%
%% Make sure the required packages are loaded in your preamble
%%   \usepackage{pgf}
%%
%% Also ensure that all the required font packages are loaded; for instance,
%% the lmodern package is sometimes necessary when using math font.
%%   \usepackage{lmodern}
%%
%% Figures using additional raster images can only be included by \input if
%% they are in the same directory as the main LaTeX file. For loading figures
%% from other directories you can use the `import` package
%%   \usepackage{import}
%%
%% and then include the figures with
%%   \import{<path to file>}{<filename>.pgf}
%%
%% Matplotlib used the following preamble
%%   \usepackage{fontspec}
%%   \setmainfont{DejaVuSerif.ttf}[Path=\detokenize{/home/dbk/.local/lib/python3.10/site-packages/matplotlib/mpl-data/fonts/ttf/}]
%%   \setsansfont{DejaVuSans.ttf}[Path=\detokenize{/home/dbk/.local/lib/python3.10/site-packages/matplotlib/mpl-data/fonts/ttf/}]
%%   \setmonofont{DejaVuSansMono.ttf}[Path=\detokenize{/home/dbk/.local/lib/python3.10/site-packages/matplotlib/mpl-data/fonts/ttf/}]
%%
\begingroup%
\makeatletter%
\begin{pgfpicture}%
\pgfpathrectangle{\pgfpointorigin}{\pgfqpoint{6.400000in}{4.800000in}}%
\pgfusepath{use as bounding box, clip}%
\begin{pgfscope}%
\pgfsetbuttcap%
\pgfsetmiterjoin%
\definecolor{currentfill}{rgb}{1.000000,1.000000,1.000000}%
\pgfsetfillcolor{currentfill}%
\pgfsetlinewidth{0.000000pt}%
\definecolor{currentstroke}{rgb}{1.000000,1.000000,1.000000}%
\pgfsetstrokecolor{currentstroke}%
\pgfsetdash{}{0pt}%
\pgfpathmoveto{\pgfqpoint{0.000000in}{0.000000in}}%
\pgfpathlineto{\pgfqpoint{6.400000in}{0.000000in}}%
\pgfpathlineto{\pgfqpoint{6.400000in}{4.800000in}}%
\pgfpathlineto{\pgfqpoint{0.000000in}{4.800000in}}%
\pgfpathlineto{\pgfqpoint{0.000000in}{0.000000in}}%
\pgfpathclose%
\pgfusepath{fill}%
\end{pgfscope}%
\begin{pgfscope}%
\pgfsetbuttcap%
\pgfsetmiterjoin%
\definecolor{currentfill}{rgb}{1.000000,1.000000,1.000000}%
\pgfsetfillcolor{currentfill}%
\pgfsetlinewidth{0.000000pt}%
\definecolor{currentstroke}{rgb}{0.000000,0.000000,0.000000}%
\pgfsetstrokecolor{currentstroke}%
\pgfsetstrokeopacity{0.000000}%
\pgfsetdash{}{0pt}%
\pgfpathmoveto{\pgfqpoint{0.800000in}{0.528000in}}%
\pgfpathlineto{\pgfqpoint{5.760000in}{0.528000in}}%
\pgfpathlineto{\pgfqpoint{5.760000in}{4.224000in}}%
\pgfpathlineto{\pgfqpoint{0.800000in}{4.224000in}}%
\pgfpathlineto{\pgfqpoint{0.800000in}{0.528000in}}%
\pgfpathclose%
\pgfusepath{fill}%
\end{pgfscope}%
\begin{pgfscope}%
\pgfsetbuttcap%
\pgfsetroundjoin%
\definecolor{currentfill}{rgb}{0.000000,0.000000,0.000000}%
\pgfsetfillcolor{currentfill}%
\pgfsetlinewidth{0.803000pt}%
\definecolor{currentstroke}{rgb}{0.000000,0.000000,0.000000}%
\pgfsetstrokecolor{currentstroke}%
\pgfsetdash{}{0pt}%
\pgfsys@defobject{currentmarker}{\pgfqpoint{0.000000in}{-0.048611in}}{\pgfqpoint{0.000000in}{0.000000in}}{%
\pgfpathmoveto{\pgfqpoint{0.000000in}{0.000000in}}%
\pgfpathlineto{\pgfqpoint{0.000000in}{-0.048611in}}%
\pgfusepath{stroke,fill}%
}%
\begin{pgfscope}%
\pgfsys@transformshift{1.020945in}{0.528000in}%
\pgfsys@useobject{currentmarker}{}%
\end{pgfscope}%
\end{pgfscope}%
\begin{pgfscope}%
\definecolor{textcolor}{rgb}{0.000000,0.000000,0.000000}%
\pgfsetstrokecolor{textcolor}%
\pgfsetfillcolor{textcolor}%
\pgftext[x=1.020945in,y=0.430778in,,top]{\color{textcolor}\ttfamily\fontsize{10.000000}{12.000000}\selectfont 0}%
\end{pgfscope}%
\begin{pgfscope}%
\pgfsetbuttcap%
\pgfsetroundjoin%
\definecolor{currentfill}{rgb}{0.000000,0.000000,0.000000}%
\pgfsetfillcolor{currentfill}%
\pgfsetlinewidth{0.803000pt}%
\definecolor{currentstroke}{rgb}{0.000000,0.000000,0.000000}%
\pgfsetstrokecolor{currentstroke}%
\pgfsetdash{}{0pt}%
\pgfsys@defobject{currentmarker}{\pgfqpoint{0.000000in}{-0.048611in}}{\pgfqpoint{0.000000in}{0.000000in}}{%
\pgfpathmoveto{\pgfqpoint{0.000000in}{0.000000in}}%
\pgfpathlineto{\pgfqpoint{0.000000in}{-0.048611in}}%
\pgfusepath{stroke,fill}%
}%
\begin{pgfscope}%
\pgfsys@transformshift{1.922764in}{0.528000in}%
\pgfsys@useobject{currentmarker}{}%
\end{pgfscope}%
\end{pgfscope}%
\begin{pgfscope}%
\definecolor{textcolor}{rgb}{0.000000,0.000000,0.000000}%
\pgfsetstrokecolor{textcolor}%
\pgfsetfillcolor{textcolor}%
\pgftext[x=1.922764in,y=0.430778in,,top]{\color{textcolor}\ttfamily\fontsize{10.000000}{12.000000}\selectfont 200}%
\end{pgfscope}%
\begin{pgfscope}%
\pgfsetbuttcap%
\pgfsetroundjoin%
\definecolor{currentfill}{rgb}{0.000000,0.000000,0.000000}%
\pgfsetfillcolor{currentfill}%
\pgfsetlinewidth{0.803000pt}%
\definecolor{currentstroke}{rgb}{0.000000,0.000000,0.000000}%
\pgfsetstrokecolor{currentstroke}%
\pgfsetdash{}{0pt}%
\pgfsys@defobject{currentmarker}{\pgfqpoint{0.000000in}{-0.048611in}}{\pgfqpoint{0.000000in}{0.000000in}}{%
\pgfpathmoveto{\pgfqpoint{0.000000in}{0.000000in}}%
\pgfpathlineto{\pgfqpoint{0.000000in}{-0.048611in}}%
\pgfusepath{stroke,fill}%
}%
\begin{pgfscope}%
\pgfsys@transformshift{2.824582in}{0.528000in}%
\pgfsys@useobject{currentmarker}{}%
\end{pgfscope}%
\end{pgfscope}%
\begin{pgfscope}%
\definecolor{textcolor}{rgb}{0.000000,0.000000,0.000000}%
\pgfsetstrokecolor{textcolor}%
\pgfsetfillcolor{textcolor}%
\pgftext[x=2.824582in,y=0.430778in,,top]{\color{textcolor}\ttfamily\fontsize{10.000000}{12.000000}\selectfont 400}%
\end{pgfscope}%
\begin{pgfscope}%
\pgfsetbuttcap%
\pgfsetroundjoin%
\definecolor{currentfill}{rgb}{0.000000,0.000000,0.000000}%
\pgfsetfillcolor{currentfill}%
\pgfsetlinewidth{0.803000pt}%
\definecolor{currentstroke}{rgb}{0.000000,0.000000,0.000000}%
\pgfsetstrokecolor{currentstroke}%
\pgfsetdash{}{0pt}%
\pgfsys@defobject{currentmarker}{\pgfqpoint{0.000000in}{-0.048611in}}{\pgfqpoint{0.000000in}{0.000000in}}{%
\pgfpathmoveto{\pgfqpoint{0.000000in}{0.000000in}}%
\pgfpathlineto{\pgfqpoint{0.000000in}{-0.048611in}}%
\pgfusepath{stroke,fill}%
}%
\begin{pgfscope}%
\pgfsys@transformshift{3.726400in}{0.528000in}%
\pgfsys@useobject{currentmarker}{}%
\end{pgfscope}%
\end{pgfscope}%
\begin{pgfscope}%
\definecolor{textcolor}{rgb}{0.000000,0.000000,0.000000}%
\pgfsetstrokecolor{textcolor}%
\pgfsetfillcolor{textcolor}%
\pgftext[x=3.726400in,y=0.430778in,,top]{\color{textcolor}\ttfamily\fontsize{10.000000}{12.000000}\selectfont 600}%
\end{pgfscope}%
\begin{pgfscope}%
\pgfsetbuttcap%
\pgfsetroundjoin%
\definecolor{currentfill}{rgb}{0.000000,0.000000,0.000000}%
\pgfsetfillcolor{currentfill}%
\pgfsetlinewidth{0.803000pt}%
\definecolor{currentstroke}{rgb}{0.000000,0.000000,0.000000}%
\pgfsetstrokecolor{currentstroke}%
\pgfsetdash{}{0pt}%
\pgfsys@defobject{currentmarker}{\pgfqpoint{0.000000in}{-0.048611in}}{\pgfqpoint{0.000000in}{0.000000in}}{%
\pgfpathmoveto{\pgfqpoint{0.000000in}{0.000000in}}%
\pgfpathlineto{\pgfqpoint{0.000000in}{-0.048611in}}%
\pgfusepath{stroke,fill}%
}%
\begin{pgfscope}%
\pgfsys@transformshift{4.628218in}{0.528000in}%
\pgfsys@useobject{currentmarker}{}%
\end{pgfscope}%
\end{pgfscope}%
\begin{pgfscope}%
\definecolor{textcolor}{rgb}{0.000000,0.000000,0.000000}%
\pgfsetstrokecolor{textcolor}%
\pgfsetfillcolor{textcolor}%
\pgftext[x=4.628218in,y=0.430778in,,top]{\color{textcolor}\ttfamily\fontsize{10.000000}{12.000000}\selectfont 800}%
\end{pgfscope}%
\begin{pgfscope}%
\pgfsetbuttcap%
\pgfsetroundjoin%
\definecolor{currentfill}{rgb}{0.000000,0.000000,0.000000}%
\pgfsetfillcolor{currentfill}%
\pgfsetlinewidth{0.803000pt}%
\definecolor{currentstroke}{rgb}{0.000000,0.000000,0.000000}%
\pgfsetstrokecolor{currentstroke}%
\pgfsetdash{}{0pt}%
\pgfsys@defobject{currentmarker}{\pgfqpoint{0.000000in}{-0.048611in}}{\pgfqpoint{0.000000in}{0.000000in}}{%
\pgfpathmoveto{\pgfqpoint{0.000000in}{0.000000in}}%
\pgfpathlineto{\pgfqpoint{0.000000in}{-0.048611in}}%
\pgfusepath{stroke,fill}%
}%
\begin{pgfscope}%
\pgfsys@transformshift{5.530036in}{0.528000in}%
\pgfsys@useobject{currentmarker}{}%
\end{pgfscope}%
\end{pgfscope}%
\begin{pgfscope}%
\definecolor{textcolor}{rgb}{0.000000,0.000000,0.000000}%
\pgfsetstrokecolor{textcolor}%
\pgfsetfillcolor{textcolor}%
\pgftext[x=5.530036in,y=0.430778in,,top]{\color{textcolor}\ttfamily\fontsize{10.000000}{12.000000}\selectfont 1000}%
\end{pgfscope}%
\begin{pgfscope}%
\definecolor{textcolor}{rgb}{0.000000,0.000000,0.000000}%
\pgfsetstrokecolor{textcolor}%
\pgfsetfillcolor{textcolor}%
\pgftext[x=3.280000in,y=0.240063in,,top]{\color{textcolor}\ttfamily\fontsize{10.000000}{12.000000}\selectfont Size of Array}%
\end{pgfscope}%
\begin{pgfscope}%
\pgfsetbuttcap%
\pgfsetroundjoin%
\definecolor{currentfill}{rgb}{0.000000,0.000000,0.000000}%
\pgfsetfillcolor{currentfill}%
\pgfsetlinewidth{0.803000pt}%
\definecolor{currentstroke}{rgb}{0.000000,0.000000,0.000000}%
\pgfsetstrokecolor{currentstroke}%
\pgfsetdash{}{0pt}%
\pgfsys@defobject{currentmarker}{\pgfqpoint{-0.048611in}{0.000000in}}{\pgfqpoint{-0.000000in}{0.000000in}}{%
\pgfpathmoveto{\pgfqpoint{-0.000000in}{0.000000in}}%
\pgfpathlineto{\pgfqpoint{-0.048611in}{0.000000in}}%
\pgfusepath{stroke,fill}%
}%
\begin{pgfscope}%
\pgfsys@transformshift{0.800000in}{0.895149in}%
\pgfsys@useobject{currentmarker}{}%
\end{pgfscope}%
\end{pgfscope}%
\begin{pgfscope}%
\definecolor{textcolor}{rgb}{0.000000,0.000000,0.000000}%
\pgfsetstrokecolor{textcolor}%
\pgfsetfillcolor{textcolor}%
\pgftext[x=0.451923in, y=0.842014in, left, base]{\color{textcolor}\ttfamily\fontsize{10.000000}{12.000000}\selectfont 100}%
\end{pgfscope}%
\begin{pgfscope}%
\pgfsetbuttcap%
\pgfsetroundjoin%
\definecolor{currentfill}{rgb}{0.000000,0.000000,0.000000}%
\pgfsetfillcolor{currentfill}%
\pgfsetlinewidth{0.803000pt}%
\definecolor{currentstroke}{rgb}{0.000000,0.000000,0.000000}%
\pgfsetstrokecolor{currentstroke}%
\pgfsetdash{}{0pt}%
\pgfsys@defobject{currentmarker}{\pgfqpoint{-0.048611in}{0.000000in}}{\pgfqpoint{-0.000000in}{0.000000in}}{%
\pgfpathmoveto{\pgfqpoint{-0.000000in}{0.000000in}}%
\pgfpathlineto{\pgfqpoint{-0.048611in}{0.000000in}}%
\pgfusepath{stroke,fill}%
}%
\begin{pgfscope}%
\pgfsys@transformshift{0.800000in}{1.405787in}%
\pgfsys@useobject{currentmarker}{}%
\end{pgfscope}%
\end{pgfscope}%
\begin{pgfscope}%
\definecolor{textcolor}{rgb}{0.000000,0.000000,0.000000}%
\pgfsetstrokecolor{textcolor}%
\pgfsetfillcolor{textcolor}%
\pgftext[x=0.451923in, y=1.352653in, left, base]{\color{textcolor}\ttfamily\fontsize{10.000000}{12.000000}\selectfont 200}%
\end{pgfscope}%
\begin{pgfscope}%
\pgfsetbuttcap%
\pgfsetroundjoin%
\definecolor{currentfill}{rgb}{0.000000,0.000000,0.000000}%
\pgfsetfillcolor{currentfill}%
\pgfsetlinewidth{0.803000pt}%
\definecolor{currentstroke}{rgb}{0.000000,0.000000,0.000000}%
\pgfsetstrokecolor{currentstroke}%
\pgfsetdash{}{0pt}%
\pgfsys@defobject{currentmarker}{\pgfqpoint{-0.048611in}{0.000000in}}{\pgfqpoint{-0.000000in}{0.000000in}}{%
\pgfpathmoveto{\pgfqpoint{-0.000000in}{0.000000in}}%
\pgfpathlineto{\pgfqpoint{-0.048611in}{0.000000in}}%
\pgfusepath{stroke,fill}%
}%
\begin{pgfscope}%
\pgfsys@transformshift{0.800000in}{1.916426in}%
\pgfsys@useobject{currentmarker}{}%
\end{pgfscope}%
\end{pgfscope}%
\begin{pgfscope}%
\definecolor{textcolor}{rgb}{0.000000,0.000000,0.000000}%
\pgfsetstrokecolor{textcolor}%
\pgfsetfillcolor{textcolor}%
\pgftext[x=0.451923in, y=1.863291in, left, base]{\color{textcolor}\ttfamily\fontsize{10.000000}{12.000000}\selectfont 300}%
\end{pgfscope}%
\begin{pgfscope}%
\pgfsetbuttcap%
\pgfsetroundjoin%
\definecolor{currentfill}{rgb}{0.000000,0.000000,0.000000}%
\pgfsetfillcolor{currentfill}%
\pgfsetlinewidth{0.803000pt}%
\definecolor{currentstroke}{rgb}{0.000000,0.000000,0.000000}%
\pgfsetstrokecolor{currentstroke}%
\pgfsetdash{}{0pt}%
\pgfsys@defobject{currentmarker}{\pgfqpoint{-0.048611in}{0.000000in}}{\pgfqpoint{-0.000000in}{0.000000in}}{%
\pgfpathmoveto{\pgfqpoint{-0.000000in}{0.000000in}}%
\pgfpathlineto{\pgfqpoint{-0.048611in}{0.000000in}}%
\pgfusepath{stroke,fill}%
}%
\begin{pgfscope}%
\pgfsys@transformshift{0.800000in}{2.427064in}%
\pgfsys@useobject{currentmarker}{}%
\end{pgfscope}%
\end{pgfscope}%
\begin{pgfscope}%
\definecolor{textcolor}{rgb}{0.000000,0.000000,0.000000}%
\pgfsetstrokecolor{textcolor}%
\pgfsetfillcolor{textcolor}%
\pgftext[x=0.451923in, y=2.373929in, left, base]{\color{textcolor}\ttfamily\fontsize{10.000000}{12.000000}\selectfont 400}%
\end{pgfscope}%
\begin{pgfscope}%
\pgfsetbuttcap%
\pgfsetroundjoin%
\definecolor{currentfill}{rgb}{0.000000,0.000000,0.000000}%
\pgfsetfillcolor{currentfill}%
\pgfsetlinewidth{0.803000pt}%
\definecolor{currentstroke}{rgb}{0.000000,0.000000,0.000000}%
\pgfsetstrokecolor{currentstroke}%
\pgfsetdash{}{0pt}%
\pgfsys@defobject{currentmarker}{\pgfqpoint{-0.048611in}{0.000000in}}{\pgfqpoint{-0.000000in}{0.000000in}}{%
\pgfpathmoveto{\pgfqpoint{-0.000000in}{0.000000in}}%
\pgfpathlineto{\pgfqpoint{-0.048611in}{0.000000in}}%
\pgfusepath{stroke,fill}%
}%
\begin{pgfscope}%
\pgfsys@transformshift{0.800000in}{2.937702in}%
\pgfsys@useobject{currentmarker}{}%
\end{pgfscope}%
\end{pgfscope}%
\begin{pgfscope}%
\definecolor{textcolor}{rgb}{0.000000,0.000000,0.000000}%
\pgfsetstrokecolor{textcolor}%
\pgfsetfillcolor{textcolor}%
\pgftext[x=0.451923in, y=2.884568in, left, base]{\color{textcolor}\ttfamily\fontsize{10.000000}{12.000000}\selectfont 500}%
\end{pgfscope}%
\begin{pgfscope}%
\pgfsetbuttcap%
\pgfsetroundjoin%
\definecolor{currentfill}{rgb}{0.000000,0.000000,0.000000}%
\pgfsetfillcolor{currentfill}%
\pgfsetlinewidth{0.803000pt}%
\definecolor{currentstroke}{rgb}{0.000000,0.000000,0.000000}%
\pgfsetstrokecolor{currentstroke}%
\pgfsetdash{}{0pt}%
\pgfsys@defobject{currentmarker}{\pgfqpoint{-0.048611in}{0.000000in}}{\pgfqpoint{-0.000000in}{0.000000in}}{%
\pgfpathmoveto{\pgfqpoint{-0.000000in}{0.000000in}}%
\pgfpathlineto{\pgfqpoint{-0.048611in}{0.000000in}}%
\pgfusepath{stroke,fill}%
}%
\begin{pgfscope}%
\pgfsys@transformshift{0.800000in}{3.448340in}%
\pgfsys@useobject{currentmarker}{}%
\end{pgfscope}%
\end{pgfscope}%
\begin{pgfscope}%
\definecolor{textcolor}{rgb}{0.000000,0.000000,0.000000}%
\pgfsetstrokecolor{textcolor}%
\pgfsetfillcolor{textcolor}%
\pgftext[x=0.451923in, y=3.395206in, left, base]{\color{textcolor}\ttfamily\fontsize{10.000000}{12.000000}\selectfont 600}%
\end{pgfscope}%
\begin{pgfscope}%
\pgfsetbuttcap%
\pgfsetroundjoin%
\definecolor{currentfill}{rgb}{0.000000,0.000000,0.000000}%
\pgfsetfillcolor{currentfill}%
\pgfsetlinewidth{0.803000pt}%
\definecolor{currentstroke}{rgb}{0.000000,0.000000,0.000000}%
\pgfsetstrokecolor{currentstroke}%
\pgfsetdash{}{0pt}%
\pgfsys@defobject{currentmarker}{\pgfqpoint{-0.048611in}{0.000000in}}{\pgfqpoint{-0.000000in}{0.000000in}}{%
\pgfpathmoveto{\pgfqpoint{-0.000000in}{0.000000in}}%
\pgfpathlineto{\pgfqpoint{-0.048611in}{0.000000in}}%
\pgfusepath{stroke,fill}%
}%
\begin{pgfscope}%
\pgfsys@transformshift{0.800000in}{3.958979in}%
\pgfsys@useobject{currentmarker}{}%
\end{pgfscope}%
\end{pgfscope}%
\begin{pgfscope}%
\definecolor{textcolor}{rgb}{0.000000,0.000000,0.000000}%
\pgfsetstrokecolor{textcolor}%
\pgfsetfillcolor{textcolor}%
\pgftext[x=0.451923in, y=3.905844in, left, base]{\color{textcolor}\ttfamily\fontsize{10.000000}{12.000000}\selectfont 700}%
\end{pgfscope}%
\begin{pgfscope}%
\definecolor{textcolor}{rgb}{0.000000,0.000000,0.000000}%
\pgfsetstrokecolor{textcolor}%
\pgfsetfillcolor{textcolor}%
\pgftext[x=0.396368in,y=2.376000in,,bottom,rotate=90.000000]{\color{textcolor}\ttfamily\fontsize{10.000000}{12.000000}\selectfont Swaps}%
\end{pgfscope}%
\begin{pgfscope}%
\pgfpathrectangle{\pgfqpoint{0.800000in}{0.528000in}}{\pgfqpoint{4.960000in}{3.696000in}}%
\pgfusepath{clip}%
\pgfsetrectcap%
\pgfsetroundjoin%
\pgfsetlinewidth{1.505625pt}%
\definecolor{currentstroke}{rgb}{0.000000,1.000000,0.498039}%
\pgfsetstrokecolor{currentstroke}%
\pgfsetdash{}{0pt}%
\pgfpathmoveto{\pgfqpoint{1.025455in}{0.696000in}}%
\pgfpathlineto{\pgfqpoint{1.029964in}{0.721532in}}%
\pgfpathlineto{\pgfqpoint{1.034473in}{0.731745in}}%
\pgfpathlineto{\pgfqpoint{1.038982in}{0.706213in}}%
\pgfpathlineto{\pgfqpoint{1.043491in}{0.706213in}}%
\pgfpathlineto{\pgfqpoint{1.048000in}{0.741957in}}%
\pgfpathlineto{\pgfqpoint{1.052509in}{0.747064in}}%
\pgfpathlineto{\pgfqpoint{1.057018in}{0.706213in}}%
\pgfpathlineto{\pgfqpoint{1.061527in}{0.721532in}}%
\pgfpathlineto{\pgfqpoint{1.066036in}{0.731745in}}%
\pgfpathlineto{\pgfqpoint{1.070545in}{0.757277in}}%
\pgfpathlineto{\pgfqpoint{1.079564in}{0.757277in}}%
\pgfpathlineto{\pgfqpoint{1.084073in}{0.726638in}}%
\pgfpathlineto{\pgfqpoint{1.088582in}{0.782809in}}%
\pgfpathlineto{\pgfqpoint{1.093091in}{0.777702in}}%
\pgfpathlineto{\pgfqpoint{1.097600in}{0.747064in}}%
\pgfpathlineto{\pgfqpoint{1.102109in}{0.757277in}}%
\pgfpathlineto{\pgfqpoint{1.106618in}{0.782809in}}%
\pgfpathlineto{\pgfqpoint{1.115636in}{0.772596in}}%
\pgfpathlineto{\pgfqpoint{1.120145in}{0.803234in}}%
\pgfpathlineto{\pgfqpoint{1.124655in}{0.767489in}}%
\pgfpathlineto{\pgfqpoint{1.129164in}{0.772596in}}%
\pgfpathlineto{\pgfqpoint{1.133673in}{0.787915in}}%
\pgfpathlineto{\pgfqpoint{1.138182in}{0.813447in}}%
\pgfpathlineto{\pgfqpoint{1.142691in}{0.798128in}}%
\pgfpathlineto{\pgfqpoint{1.147200in}{0.803234in}}%
\pgfpathlineto{\pgfqpoint{1.151709in}{0.798128in}}%
\pgfpathlineto{\pgfqpoint{1.156218in}{0.803234in}}%
\pgfpathlineto{\pgfqpoint{1.165236in}{0.793021in}}%
\pgfpathlineto{\pgfqpoint{1.169745in}{0.818553in}}%
\pgfpathlineto{\pgfqpoint{1.174255in}{0.828766in}}%
\pgfpathlineto{\pgfqpoint{1.178764in}{0.808340in}}%
\pgfpathlineto{\pgfqpoint{1.183273in}{0.838979in}}%
\pgfpathlineto{\pgfqpoint{1.187782in}{0.762383in}}%
\pgfpathlineto{\pgfqpoint{1.192291in}{0.747064in}}%
\pgfpathlineto{\pgfqpoint{1.196800in}{0.844085in}}%
\pgfpathlineto{\pgfqpoint{1.205818in}{0.823660in}}%
\pgfpathlineto{\pgfqpoint{1.210327in}{0.798128in}}%
\pgfpathlineto{\pgfqpoint{1.214836in}{0.808340in}}%
\pgfpathlineto{\pgfqpoint{1.219345in}{0.844085in}}%
\pgfpathlineto{\pgfqpoint{1.223855in}{0.859404in}}%
\pgfpathlineto{\pgfqpoint{1.228364in}{0.849191in}}%
\pgfpathlineto{\pgfqpoint{1.232873in}{0.869617in}}%
\pgfpathlineto{\pgfqpoint{1.237382in}{0.838979in}}%
\pgfpathlineto{\pgfqpoint{1.241891in}{0.864511in}}%
\pgfpathlineto{\pgfqpoint{1.246400in}{0.879830in}}%
\pgfpathlineto{\pgfqpoint{1.250909in}{0.890043in}}%
\pgfpathlineto{\pgfqpoint{1.255418in}{0.874723in}}%
\pgfpathlineto{\pgfqpoint{1.259927in}{0.890043in}}%
\pgfpathlineto{\pgfqpoint{1.264436in}{0.859404in}}%
\pgfpathlineto{\pgfqpoint{1.268945in}{0.874723in}}%
\pgfpathlineto{\pgfqpoint{1.273455in}{0.884936in}}%
\pgfpathlineto{\pgfqpoint{1.277964in}{0.930894in}}%
\pgfpathlineto{\pgfqpoint{1.282473in}{0.915574in}}%
\pgfpathlineto{\pgfqpoint{1.291491in}{0.961532in}}%
\pgfpathlineto{\pgfqpoint{1.296000in}{0.905362in}}%
\pgfpathlineto{\pgfqpoint{1.300509in}{0.936000in}}%
\pgfpathlineto{\pgfqpoint{1.305018in}{0.987064in}}%
\pgfpathlineto{\pgfqpoint{1.309527in}{0.925787in}}%
\pgfpathlineto{\pgfqpoint{1.314036in}{0.946213in}}%
\pgfpathlineto{\pgfqpoint{1.318545in}{0.925787in}}%
\pgfpathlineto{\pgfqpoint{1.323055in}{0.936000in}}%
\pgfpathlineto{\pgfqpoint{1.327564in}{0.966638in}}%
\pgfpathlineto{\pgfqpoint{1.332073in}{0.961532in}}%
\pgfpathlineto{\pgfqpoint{1.336582in}{0.966638in}}%
\pgfpathlineto{\pgfqpoint{1.341091in}{0.930894in}}%
\pgfpathlineto{\pgfqpoint{1.345600in}{0.920681in}}%
\pgfpathlineto{\pgfqpoint{1.350109in}{0.976851in}}%
\pgfpathlineto{\pgfqpoint{1.354618in}{1.002383in}}%
\pgfpathlineto{\pgfqpoint{1.359127in}{0.961532in}}%
\pgfpathlineto{\pgfqpoint{1.363636in}{0.930894in}}%
\pgfpathlineto{\pgfqpoint{1.368145in}{1.002383in}}%
\pgfpathlineto{\pgfqpoint{1.372655in}{0.976851in}}%
\pgfpathlineto{\pgfqpoint{1.377164in}{0.976851in}}%
\pgfpathlineto{\pgfqpoint{1.381673in}{1.007489in}}%
\pgfpathlineto{\pgfqpoint{1.386182in}{1.022809in}}%
\pgfpathlineto{\pgfqpoint{1.390691in}{0.987064in}}%
\pgfpathlineto{\pgfqpoint{1.395200in}{1.007489in}}%
\pgfpathlineto{\pgfqpoint{1.399709in}{1.007489in}}%
\pgfpathlineto{\pgfqpoint{1.404218in}{0.997277in}}%
\pgfpathlineto{\pgfqpoint{1.408727in}{1.043234in}}%
\pgfpathlineto{\pgfqpoint{1.413236in}{1.017702in}}%
\pgfpathlineto{\pgfqpoint{1.417745in}{1.022809in}}%
\pgfpathlineto{\pgfqpoint{1.422255in}{1.002383in}}%
\pgfpathlineto{\pgfqpoint{1.426764in}{1.017702in}}%
\pgfpathlineto{\pgfqpoint{1.431273in}{1.022809in}}%
\pgfpathlineto{\pgfqpoint{1.435782in}{0.992170in}}%
\pgfpathlineto{\pgfqpoint{1.444800in}{1.038128in}}%
\pgfpathlineto{\pgfqpoint{1.449309in}{1.033021in}}%
\pgfpathlineto{\pgfqpoint{1.453818in}{1.007489in}}%
\pgfpathlineto{\pgfqpoint{1.458327in}{1.084085in}}%
\pgfpathlineto{\pgfqpoint{1.462836in}{1.043234in}}%
\pgfpathlineto{\pgfqpoint{1.467345in}{1.078979in}}%
\pgfpathlineto{\pgfqpoint{1.471855in}{1.022809in}}%
\pgfpathlineto{\pgfqpoint{1.476364in}{1.068766in}}%
\pgfpathlineto{\pgfqpoint{1.480873in}{1.073872in}}%
\pgfpathlineto{\pgfqpoint{1.485382in}{1.053447in}}%
\pgfpathlineto{\pgfqpoint{1.489891in}{1.017702in}}%
\pgfpathlineto{\pgfqpoint{1.498909in}{1.053447in}}%
\pgfpathlineto{\pgfqpoint{1.503418in}{1.053447in}}%
\pgfpathlineto{\pgfqpoint{1.507927in}{1.073872in}}%
\pgfpathlineto{\pgfqpoint{1.512436in}{1.063660in}}%
\pgfpathlineto{\pgfqpoint{1.516945in}{1.084085in}}%
\pgfpathlineto{\pgfqpoint{1.521455in}{1.078979in}}%
\pgfpathlineto{\pgfqpoint{1.525964in}{1.109617in}}%
\pgfpathlineto{\pgfqpoint{1.530473in}{1.058553in}}%
\pgfpathlineto{\pgfqpoint{1.539491in}{1.089191in}}%
\pgfpathlineto{\pgfqpoint{1.544000in}{1.038128in}}%
\pgfpathlineto{\pgfqpoint{1.548509in}{1.124936in}}%
\pgfpathlineto{\pgfqpoint{1.553018in}{1.165787in}}%
\pgfpathlineto{\pgfqpoint{1.557527in}{1.135149in}}%
\pgfpathlineto{\pgfqpoint{1.562036in}{1.084085in}}%
\pgfpathlineto{\pgfqpoint{1.566545in}{1.073872in}}%
\pgfpathlineto{\pgfqpoint{1.571055in}{1.104511in}}%
\pgfpathlineto{\pgfqpoint{1.575564in}{1.089191in}}%
\pgfpathlineto{\pgfqpoint{1.580073in}{1.124936in}}%
\pgfpathlineto{\pgfqpoint{1.584582in}{1.068766in}}%
\pgfpathlineto{\pgfqpoint{1.589091in}{1.150468in}}%
\pgfpathlineto{\pgfqpoint{1.593600in}{1.140255in}}%
\pgfpathlineto{\pgfqpoint{1.598109in}{1.170894in}}%
\pgfpathlineto{\pgfqpoint{1.602618in}{1.170894in}}%
\pgfpathlineto{\pgfqpoint{1.607127in}{1.089191in}}%
\pgfpathlineto{\pgfqpoint{1.611636in}{1.145362in}}%
\pgfpathlineto{\pgfqpoint{1.616145in}{1.104511in}}%
\pgfpathlineto{\pgfqpoint{1.620655in}{1.135149in}}%
\pgfpathlineto{\pgfqpoint{1.625164in}{1.135149in}}%
\pgfpathlineto{\pgfqpoint{1.629673in}{1.119830in}}%
\pgfpathlineto{\pgfqpoint{1.634182in}{1.155574in}}%
\pgfpathlineto{\pgfqpoint{1.638691in}{1.155574in}}%
\pgfpathlineto{\pgfqpoint{1.643200in}{1.150468in}}%
\pgfpathlineto{\pgfqpoint{1.647709in}{1.109617in}}%
\pgfpathlineto{\pgfqpoint{1.652218in}{1.150468in}}%
\pgfpathlineto{\pgfqpoint{1.656727in}{1.155574in}}%
\pgfpathlineto{\pgfqpoint{1.661236in}{1.186213in}}%
\pgfpathlineto{\pgfqpoint{1.665745in}{1.150468in}}%
\pgfpathlineto{\pgfqpoint{1.670255in}{1.145362in}}%
\pgfpathlineto{\pgfqpoint{1.674764in}{1.145362in}}%
\pgfpathlineto{\pgfqpoint{1.679273in}{1.181106in}}%
\pgfpathlineto{\pgfqpoint{1.683782in}{1.201532in}}%
\pgfpathlineto{\pgfqpoint{1.688291in}{1.176000in}}%
\pgfpathlineto{\pgfqpoint{1.692800in}{1.227064in}}%
\pgfpathlineto{\pgfqpoint{1.697309in}{1.206638in}}%
\pgfpathlineto{\pgfqpoint{1.701818in}{1.140255in}}%
\pgfpathlineto{\pgfqpoint{1.706327in}{1.196426in}}%
\pgfpathlineto{\pgfqpoint{1.710836in}{1.104511in}}%
\pgfpathlineto{\pgfqpoint{1.715345in}{1.216851in}}%
\pgfpathlineto{\pgfqpoint{1.719855in}{1.206638in}}%
\pgfpathlineto{\pgfqpoint{1.724364in}{1.165787in}}%
\pgfpathlineto{\pgfqpoint{1.728873in}{1.201532in}}%
\pgfpathlineto{\pgfqpoint{1.733382in}{1.196426in}}%
\pgfpathlineto{\pgfqpoint{1.737891in}{1.186213in}}%
\pgfpathlineto{\pgfqpoint{1.742400in}{1.252596in}}%
\pgfpathlineto{\pgfqpoint{1.746909in}{1.237277in}}%
\pgfpathlineto{\pgfqpoint{1.751418in}{1.206638in}}%
\pgfpathlineto{\pgfqpoint{1.755927in}{1.267915in}}%
\pgfpathlineto{\pgfqpoint{1.760436in}{1.232170in}}%
\pgfpathlineto{\pgfqpoint{1.764945in}{1.247489in}}%
\pgfpathlineto{\pgfqpoint{1.769455in}{1.298553in}}%
\pgfpathlineto{\pgfqpoint{1.773964in}{1.242383in}}%
\pgfpathlineto{\pgfqpoint{1.778473in}{1.308766in}}%
\pgfpathlineto{\pgfqpoint{1.782982in}{1.232170in}}%
\pgfpathlineto{\pgfqpoint{1.787491in}{1.221957in}}%
\pgfpathlineto{\pgfqpoint{1.792000in}{1.237277in}}%
\pgfpathlineto{\pgfqpoint{1.796509in}{1.247489in}}%
\pgfpathlineto{\pgfqpoint{1.801018in}{1.288340in}}%
\pgfpathlineto{\pgfqpoint{1.805527in}{1.318979in}}%
\pgfpathlineto{\pgfqpoint{1.810036in}{1.273021in}}%
\pgfpathlineto{\pgfqpoint{1.814545in}{1.324085in}}%
\pgfpathlineto{\pgfqpoint{1.823564in}{1.252596in}}%
\pgfpathlineto{\pgfqpoint{1.828073in}{1.318979in}}%
\pgfpathlineto{\pgfqpoint{1.832582in}{1.339404in}}%
\pgfpathlineto{\pgfqpoint{1.841600in}{1.293447in}}%
\pgfpathlineto{\pgfqpoint{1.846109in}{1.293447in}}%
\pgfpathlineto{\pgfqpoint{1.850618in}{1.318979in}}%
\pgfpathlineto{\pgfqpoint{1.855127in}{1.370043in}}%
\pgfpathlineto{\pgfqpoint{1.859636in}{1.334298in}}%
\pgfpathlineto{\pgfqpoint{1.864145in}{1.339404in}}%
\pgfpathlineto{\pgfqpoint{1.868655in}{1.339404in}}%
\pgfpathlineto{\pgfqpoint{1.873164in}{1.334298in}}%
\pgfpathlineto{\pgfqpoint{1.877673in}{1.344511in}}%
\pgfpathlineto{\pgfqpoint{1.882182in}{1.318979in}}%
\pgfpathlineto{\pgfqpoint{1.886691in}{1.364936in}}%
\pgfpathlineto{\pgfqpoint{1.891200in}{1.308766in}}%
\pgfpathlineto{\pgfqpoint{1.895709in}{1.380255in}}%
\pgfpathlineto{\pgfqpoint{1.900218in}{1.400681in}}%
\pgfpathlineto{\pgfqpoint{1.904727in}{1.364936in}}%
\pgfpathlineto{\pgfqpoint{1.909236in}{1.354723in}}%
\pgfpathlineto{\pgfqpoint{1.913745in}{1.339404in}}%
\pgfpathlineto{\pgfqpoint{1.918255in}{1.313872in}}%
\pgfpathlineto{\pgfqpoint{1.922764in}{1.395574in}}%
\pgfpathlineto{\pgfqpoint{1.927273in}{1.390468in}}%
\pgfpathlineto{\pgfqpoint{1.931782in}{1.339404in}}%
\pgfpathlineto{\pgfqpoint{1.945309in}{1.410894in}}%
\pgfpathlineto{\pgfqpoint{1.949818in}{1.431319in}}%
\pgfpathlineto{\pgfqpoint{1.954327in}{1.416000in}}%
\pgfpathlineto{\pgfqpoint{1.958836in}{1.431319in}}%
\pgfpathlineto{\pgfqpoint{1.963345in}{1.436426in}}%
\pgfpathlineto{\pgfqpoint{1.967855in}{1.436426in}}%
\pgfpathlineto{\pgfqpoint{1.972364in}{1.370043in}}%
\pgfpathlineto{\pgfqpoint{1.976873in}{1.385362in}}%
\pgfpathlineto{\pgfqpoint{1.981382in}{1.431319in}}%
\pgfpathlineto{\pgfqpoint{1.985891in}{1.380255in}}%
\pgfpathlineto{\pgfqpoint{1.990400in}{1.467064in}}%
\pgfpathlineto{\pgfqpoint{1.994909in}{1.436426in}}%
\pgfpathlineto{\pgfqpoint{1.999418in}{1.421106in}}%
\pgfpathlineto{\pgfqpoint{2.003927in}{1.502809in}}%
\pgfpathlineto{\pgfqpoint{2.008436in}{1.421106in}}%
\pgfpathlineto{\pgfqpoint{2.012945in}{1.477277in}}%
\pgfpathlineto{\pgfqpoint{2.017455in}{1.507915in}}%
\pgfpathlineto{\pgfqpoint{2.021964in}{1.467064in}}%
\pgfpathlineto{\pgfqpoint{2.026473in}{1.477277in}}%
\pgfpathlineto{\pgfqpoint{2.030982in}{1.482383in}}%
\pgfpathlineto{\pgfqpoint{2.035491in}{1.507915in}}%
\pgfpathlineto{\pgfqpoint{2.040000in}{1.472170in}}%
\pgfpathlineto{\pgfqpoint{2.044509in}{1.497702in}}%
\pgfpathlineto{\pgfqpoint{2.049018in}{1.533447in}}%
\pgfpathlineto{\pgfqpoint{2.053527in}{1.513021in}}%
\pgfpathlineto{\pgfqpoint{2.058036in}{1.502809in}}%
\pgfpathlineto{\pgfqpoint{2.062545in}{1.487489in}}%
\pgfpathlineto{\pgfqpoint{2.067055in}{1.533447in}}%
\pgfpathlineto{\pgfqpoint{2.076073in}{1.584511in}}%
\pgfpathlineto{\pgfqpoint{2.080582in}{1.523234in}}%
\pgfpathlineto{\pgfqpoint{2.085091in}{1.543660in}}%
\pgfpathlineto{\pgfqpoint{2.089600in}{1.548766in}}%
\pgfpathlineto{\pgfqpoint{2.094109in}{1.523234in}}%
\pgfpathlineto{\pgfqpoint{2.098618in}{1.513021in}}%
\pgfpathlineto{\pgfqpoint{2.103127in}{1.604936in}}%
\pgfpathlineto{\pgfqpoint{2.107636in}{1.513021in}}%
\pgfpathlineto{\pgfqpoint{2.112145in}{1.543660in}}%
\pgfpathlineto{\pgfqpoint{2.116655in}{1.436426in}}%
\pgfpathlineto{\pgfqpoint{2.121164in}{1.604936in}}%
\pgfpathlineto{\pgfqpoint{2.125673in}{1.492596in}}%
\pgfpathlineto{\pgfqpoint{2.134691in}{1.533447in}}%
\pgfpathlineto{\pgfqpoint{2.139200in}{1.625362in}}%
\pgfpathlineto{\pgfqpoint{2.148218in}{1.615149in}}%
\pgfpathlineto{\pgfqpoint{2.157236in}{1.553872in}}%
\pgfpathlineto{\pgfqpoint{2.161745in}{1.610043in}}%
\pgfpathlineto{\pgfqpoint{2.166255in}{1.497702in}}%
\pgfpathlineto{\pgfqpoint{2.170764in}{1.599830in}}%
\pgfpathlineto{\pgfqpoint{2.179782in}{1.610043in}}%
\pgfpathlineto{\pgfqpoint{2.184291in}{1.599830in}}%
\pgfpathlineto{\pgfqpoint{2.188800in}{1.558979in}}%
\pgfpathlineto{\pgfqpoint{2.193309in}{1.599830in}}%
\pgfpathlineto{\pgfqpoint{2.197818in}{1.543660in}}%
\pgfpathlineto{\pgfqpoint{2.206836in}{1.610043in}}%
\pgfpathlineto{\pgfqpoint{2.211345in}{1.615149in}}%
\pgfpathlineto{\pgfqpoint{2.215855in}{1.671319in}}%
\pgfpathlineto{\pgfqpoint{2.220364in}{1.666213in}}%
\pgfpathlineto{\pgfqpoint{2.224873in}{1.594723in}}%
\pgfpathlineto{\pgfqpoint{2.229382in}{1.635574in}}%
\pgfpathlineto{\pgfqpoint{2.233891in}{1.615149in}}%
\pgfpathlineto{\pgfqpoint{2.238400in}{1.620255in}}%
\pgfpathlineto{\pgfqpoint{2.242909in}{1.610043in}}%
\pgfpathlineto{\pgfqpoint{2.247418in}{1.615149in}}%
\pgfpathlineto{\pgfqpoint{2.251927in}{1.650894in}}%
\pgfpathlineto{\pgfqpoint{2.256436in}{1.615149in}}%
\pgfpathlineto{\pgfqpoint{2.260945in}{1.701957in}}%
\pgfpathlineto{\pgfqpoint{2.265455in}{1.620255in}}%
\pgfpathlineto{\pgfqpoint{2.269964in}{1.625362in}}%
\pgfpathlineto{\pgfqpoint{2.274473in}{1.635574in}}%
\pgfpathlineto{\pgfqpoint{2.278982in}{1.625362in}}%
\pgfpathlineto{\pgfqpoint{2.283491in}{1.681532in}}%
\pgfpathlineto{\pgfqpoint{2.288000in}{1.676426in}}%
\pgfpathlineto{\pgfqpoint{2.292509in}{1.681532in}}%
\pgfpathlineto{\pgfqpoint{2.297018in}{1.656000in}}%
\pgfpathlineto{\pgfqpoint{2.301527in}{1.681532in}}%
\pgfpathlineto{\pgfqpoint{2.306036in}{1.645787in}}%
\pgfpathlineto{\pgfqpoint{2.310545in}{1.732596in}}%
\pgfpathlineto{\pgfqpoint{2.315055in}{1.615149in}}%
\pgfpathlineto{\pgfqpoint{2.319564in}{1.666213in}}%
\pgfpathlineto{\pgfqpoint{2.328582in}{1.727489in}}%
\pgfpathlineto{\pgfqpoint{2.333091in}{1.722383in}}%
\pgfpathlineto{\pgfqpoint{2.337600in}{1.691745in}}%
\pgfpathlineto{\pgfqpoint{2.342109in}{1.717277in}}%
\pgfpathlineto{\pgfqpoint{2.346618in}{1.753021in}}%
\pgfpathlineto{\pgfqpoint{2.351127in}{1.742809in}}%
\pgfpathlineto{\pgfqpoint{2.355636in}{1.727489in}}%
\pgfpathlineto{\pgfqpoint{2.360145in}{1.768340in}}%
\pgfpathlineto{\pgfqpoint{2.364655in}{1.717277in}}%
\pgfpathlineto{\pgfqpoint{2.373673in}{1.753021in}}%
\pgfpathlineto{\pgfqpoint{2.378182in}{1.712170in}}%
\pgfpathlineto{\pgfqpoint{2.382691in}{1.753021in}}%
\pgfpathlineto{\pgfqpoint{2.387200in}{1.758128in}}%
\pgfpathlineto{\pgfqpoint{2.391709in}{1.707064in}}%
\pgfpathlineto{\pgfqpoint{2.396218in}{1.783660in}}%
\pgfpathlineto{\pgfqpoint{2.400727in}{1.778553in}}%
\pgfpathlineto{\pgfqpoint{2.405236in}{1.798979in}}%
\pgfpathlineto{\pgfqpoint{2.409745in}{1.753021in}}%
\pgfpathlineto{\pgfqpoint{2.414255in}{1.742809in}}%
\pgfpathlineto{\pgfqpoint{2.418764in}{1.722383in}}%
\pgfpathlineto{\pgfqpoint{2.432291in}{1.814298in}}%
\pgfpathlineto{\pgfqpoint{2.436800in}{1.798979in}}%
\pgfpathlineto{\pgfqpoint{2.441309in}{1.824511in}}%
\pgfpathlineto{\pgfqpoint{2.445818in}{1.778553in}}%
\pgfpathlineto{\pgfqpoint{2.450327in}{1.768340in}}%
\pgfpathlineto{\pgfqpoint{2.459345in}{1.758128in}}%
\pgfpathlineto{\pgfqpoint{2.468364in}{1.712170in}}%
\pgfpathlineto{\pgfqpoint{2.472873in}{1.814298in}}%
\pgfpathlineto{\pgfqpoint{2.477382in}{1.834723in}}%
\pgfpathlineto{\pgfqpoint{2.481891in}{1.814298in}}%
\pgfpathlineto{\pgfqpoint{2.486400in}{1.753021in}}%
\pgfpathlineto{\pgfqpoint{2.490909in}{1.819404in}}%
\pgfpathlineto{\pgfqpoint{2.495418in}{1.839830in}}%
\pgfpathlineto{\pgfqpoint{2.499927in}{1.850043in}}%
\pgfpathlineto{\pgfqpoint{2.504436in}{1.788766in}}%
\pgfpathlineto{\pgfqpoint{2.508945in}{1.809191in}}%
\pgfpathlineto{\pgfqpoint{2.517964in}{1.819404in}}%
\pgfpathlineto{\pgfqpoint{2.522473in}{1.860255in}}%
\pgfpathlineto{\pgfqpoint{2.526982in}{1.793872in}}%
\pgfpathlineto{\pgfqpoint{2.536000in}{1.844936in}}%
\pgfpathlineto{\pgfqpoint{2.540509in}{1.834723in}}%
\pgfpathlineto{\pgfqpoint{2.545018in}{1.860255in}}%
\pgfpathlineto{\pgfqpoint{2.549527in}{1.819404in}}%
\pgfpathlineto{\pgfqpoint{2.554036in}{1.839830in}}%
\pgfpathlineto{\pgfqpoint{2.558545in}{1.819404in}}%
\pgfpathlineto{\pgfqpoint{2.563055in}{1.850043in}}%
\pgfpathlineto{\pgfqpoint{2.567564in}{1.819404in}}%
\pgfpathlineto{\pgfqpoint{2.572073in}{1.916426in}}%
\pgfpathlineto{\pgfqpoint{2.576582in}{1.865362in}}%
\pgfpathlineto{\pgfqpoint{2.581091in}{1.896000in}}%
\pgfpathlineto{\pgfqpoint{2.585600in}{1.839830in}}%
\pgfpathlineto{\pgfqpoint{2.590109in}{1.824511in}}%
\pgfpathlineto{\pgfqpoint{2.603636in}{1.957277in}}%
\pgfpathlineto{\pgfqpoint{2.612655in}{1.804085in}}%
\pgfpathlineto{\pgfqpoint{2.617164in}{1.855149in}}%
\pgfpathlineto{\pgfqpoint{2.621673in}{1.850043in}}%
\pgfpathlineto{\pgfqpoint{2.626182in}{1.804085in}}%
\pgfpathlineto{\pgfqpoint{2.630691in}{1.926638in}}%
\pgfpathlineto{\pgfqpoint{2.635200in}{1.941957in}}%
\pgfpathlineto{\pgfqpoint{2.639709in}{1.896000in}}%
\pgfpathlineto{\pgfqpoint{2.644218in}{1.911319in}}%
\pgfpathlineto{\pgfqpoint{2.648727in}{1.916426in}}%
\pgfpathlineto{\pgfqpoint{2.653236in}{1.860255in}}%
\pgfpathlineto{\pgfqpoint{2.657745in}{1.952170in}}%
\pgfpathlineto{\pgfqpoint{2.662255in}{1.896000in}}%
\pgfpathlineto{\pgfqpoint{2.671273in}{1.916426in}}%
\pgfpathlineto{\pgfqpoint{2.675782in}{1.896000in}}%
\pgfpathlineto{\pgfqpoint{2.680291in}{1.926638in}}%
\pgfpathlineto{\pgfqpoint{2.684800in}{1.824511in}}%
\pgfpathlineto{\pgfqpoint{2.689309in}{1.906213in}}%
\pgfpathlineto{\pgfqpoint{2.693818in}{1.896000in}}%
\pgfpathlineto{\pgfqpoint{2.698327in}{1.906213in}}%
\pgfpathlineto{\pgfqpoint{2.702836in}{1.926638in}}%
\pgfpathlineto{\pgfqpoint{2.707345in}{1.901106in}}%
\pgfpathlineto{\pgfqpoint{2.711855in}{1.972596in}}%
\pgfpathlineto{\pgfqpoint{2.716364in}{1.941957in}}%
\pgfpathlineto{\pgfqpoint{2.720873in}{1.941957in}}%
\pgfpathlineto{\pgfqpoint{2.725382in}{1.957277in}}%
\pgfpathlineto{\pgfqpoint{2.729891in}{1.957277in}}%
\pgfpathlineto{\pgfqpoint{2.734400in}{1.962383in}}%
\pgfpathlineto{\pgfqpoint{2.738909in}{1.957277in}}%
\pgfpathlineto{\pgfqpoint{2.747927in}{2.023660in}}%
\pgfpathlineto{\pgfqpoint{2.752436in}{1.962383in}}%
\pgfpathlineto{\pgfqpoint{2.756945in}{1.967489in}}%
\pgfpathlineto{\pgfqpoint{2.761455in}{2.008340in}}%
\pgfpathlineto{\pgfqpoint{2.765964in}{1.962383in}}%
\pgfpathlineto{\pgfqpoint{2.770473in}{1.993021in}}%
\pgfpathlineto{\pgfqpoint{2.774982in}{1.885787in}}%
\pgfpathlineto{\pgfqpoint{2.779491in}{1.957277in}}%
\pgfpathlineto{\pgfqpoint{2.784000in}{1.977702in}}%
\pgfpathlineto{\pgfqpoint{2.788509in}{1.947064in}}%
\pgfpathlineto{\pgfqpoint{2.793018in}{2.033872in}}%
\pgfpathlineto{\pgfqpoint{2.797527in}{1.967489in}}%
\pgfpathlineto{\pgfqpoint{2.802036in}{2.044085in}}%
\pgfpathlineto{\pgfqpoint{2.806545in}{1.957277in}}%
\pgfpathlineto{\pgfqpoint{2.811055in}{2.038979in}}%
\pgfpathlineto{\pgfqpoint{2.815564in}{2.033872in}}%
\pgfpathlineto{\pgfqpoint{2.824582in}{1.952170in}}%
\pgfpathlineto{\pgfqpoint{2.829091in}{1.998128in}}%
\pgfpathlineto{\pgfqpoint{2.833600in}{2.069617in}}%
\pgfpathlineto{\pgfqpoint{2.838109in}{1.911319in}}%
\pgfpathlineto{\pgfqpoint{2.842618in}{2.054298in}}%
\pgfpathlineto{\pgfqpoint{2.847127in}{2.023660in}}%
\pgfpathlineto{\pgfqpoint{2.851636in}{2.018553in}}%
\pgfpathlineto{\pgfqpoint{2.856145in}{1.957277in}}%
\pgfpathlineto{\pgfqpoint{2.860655in}{2.100255in}}%
\pgfpathlineto{\pgfqpoint{2.865164in}{2.003234in}}%
\pgfpathlineto{\pgfqpoint{2.869673in}{2.013447in}}%
\pgfpathlineto{\pgfqpoint{2.874182in}{2.059404in}}%
\pgfpathlineto{\pgfqpoint{2.878691in}{2.054298in}}%
\pgfpathlineto{\pgfqpoint{2.883200in}{2.090043in}}%
\pgfpathlineto{\pgfqpoint{2.892218in}{1.993021in}}%
\pgfpathlineto{\pgfqpoint{2.896727in}{2.023660in}}%
\pgfpathlineto{\pgfqpoint{2.901236in}{2.110468in}}%
\pgfpathlineto{\pgfqpoint{2.905745in}{2.095149in}}%
\pgfpathlineto{\pgfqpoint{2.914764in}{1.957277in}}%
\pgfpathlineto{\pgfqpoint{2.919273in}{2.095149in}}%
\pgfpathlineto{\pgfqpoint{2.923782in}{1.972596in}}%
\pgfpathlineto{\pgfqpoint{2.928291in}{2.171745in}}%
\pgfpathlineto{\pgfqpoint{2.932800in}{2.023660in}}%
\pgfpathlineto{\pgfqpoint{2.937309in}{2.054298in}}%
\pgfpathlineto{\pgfqpoint{2.941818in}{2.115574in}}%
\pgfpathlineto{\pgfqpoint{2.946327in}{2.151319in}}%
\pgfpathlineto{\pgfqpoint{2.950836in}{2.100255in}}%
\pgfpathlineto{\pgfqpoint{2.955345in}{2.003234in}}%
\pgfpathlineto{\pgfqpoint{2.959855in}{2.125787in}}%
\pgfpathlineto{\pgfqpoint{2.964364in}{2.120681in}}%
\pgfpathlineto{\pgfqpoint{2.968873in}{2.120681in}}%
\pgfpathlineto{\pgfqpoint{2.973382in}{2.197277in}}%
\pgfpathlineto{\pgfqpoint{2.977891in}{2.171745in}}%
\pgfpathlineto{\pgfqpoint{2.982400in}{2.156426in}}%
\pgfpathlineto{\pgfqpoint{2.986909in}{2.136000in}}%
\pgfpathlineto{\pgfqpoint{2.991418in}{2.141106in}}%
\pgfpathlineto{\pgfqpoint{2.995927in}{2.105362in}}%
\pgfpathlineto{\pgfqpoint{3.000436in}{2.176851in}}%
\pgfpathlineto{\pgfqpoint{3.004945in}{2.222809in}}%
\pgfpathlineto{\pgfqpoint{3.009455in}{2.187064in}}%
\pgfpathlineto{\pgfqpoint{3.018473in}{2.100255in}}%
\pgfpathlineto{\pgfqpoint{3.022982in}{2.130894in}}%
\pgfpathlineto{\pgfqpoint{3.027491in}{2.125787in}}%
\pgfpathlineto{\pgfqpoint{3.032000in}{2.217702in}}%
\pgfpathlineto{\pgfqpoint{3.036509in}{2.136000in}}%
\pgfpathlineto{\pgfqpoint{3.041018in}{2.176851in}}%
\pgfpathlineto{\pgfqpoint{3.045527in}{2.248340in}}%
\pgfpathlineto{\pgfqpoint{3.050036in}{2.095149in}}%
\pgfpathlineto{\pgfqpoint{3.054545in}{2.187064in}}%
\pgfpathlineto{\pgfqpoint{3.059055in}{2.202383in}}%
\pgfpathlineto{\pgfqpoint{3.063564in}{2.166638in}}%
\pgfpathlineto{\pgfqpoint{3.068073in}{2.171745in}}%
\pgfpathlineto{\pgfqpoint{3.072582in}{2.166638in}}%
\pgfpathlineto{\pgfqpoint{3.077091in}{2.202383in}}%
\pgfpathlineto{\pgfqpoint{3.081600in}{2.212596in}}%
\pgfpathlineto{\pgfqpoint{3.086109in}{2.238128in}}%
\pgfpathlineto{\pgfqpoint{3.090618in}{2.278979in}}%
\pgfpathlineto{\pgfqpoint{3.095127in}{2.202383in}}%
\pgfpathlineto{\pgfqpoint{3.099636in}{2.253447in}}%
\pgfpathlineto{\pgfqpoint{3.104145in}{2.207489in}}%
\pgfpathlineto{\pgfqpoint{3.108655in}{2.243234in}}%
\pgfpathlineto{\pgfqpoint{3.113164in}{2.253447in}}%
\pgfpathlineto{\pgfqpoint{3.117673in}{2.187064in}}%
\pgfpathlineto{\pgfqpoint{3.122182in}{2.314723in}}%
\pgfpathlineto{\pgfqpoint{3.126691in}{2.340255in}}%
\pgfpathlineto{\pgfqpoint{3.131200in}{2.227915in}}%
\pgfpathlineto{\pgfqpoint{3.135709in}{2.187064in}}%
\pgfpathlineto{\pgfqpoint{3.140218in}{2.253447in}}%
\pgfpathlineto{\pgfqpoint{3.144727in}{2.340255in}}%
\pgfpathlineto{\pgfqpoint{3.153745in}{2.243234in}}%
\pgfpathlineto{\pgfqpoint{3.158255in}{2.258553in}}%
\pgfpathlineto{\pgfqpoint{3.162764in}{2.278979in}}%
\pgfpathlineto{\pgfqpoint{3.167273in}{2.268766in}}%
\pgfpathlineto{\pgfqpoint{3.171782in}{2.376000in}}%
\pgfpathlineto{\pgfqpoint{3.176291in}{2.324936in}}%
\pgfpathlineto{\pgfqpoint{3.180800in}{2.253447in}}%
\pgfpathlineto{\pgfqpoint{3.185309in}{2.345362in}}%
\pgfpathlineto{\pgfqpoint{3.189818in}{2.335149in}}%
\pgfpathlineto{\pgfqpoint{3.194327in}{2.304511in}}%
\pgfpathlineto{\pgfqpoint{3.198836in}{2.345362in}}%
\pgfpathlineto{\pgfqpoint{3.203345in}{2.258553in}}%
\pgfpathlineto{\pgfqpoint{3.207855in}{2.324936in}}%
\pgfpathlineto{\pgfqpoint{3.212364in}{2.314723in}}%
\pgfpathlineto{\pgfqpoint{3.216873in}{2.273872in}}%
\pgfpathlineto{\pgfqpoint{3.221382in}{2.340255in}}%
\pgfpathlineto{\pgfqpoint{3.225891in}{2.365787in}}%
\pgfpathlineto{\pgfqpoint{3.230400in}{2.345362in}}%
\pgfpathlineto{\pgfqpoint{3.234909in}{2.335149in}}%
\pgfpathlineto{\pgfqpoint{3.239418in}{2.268766in}}%
\pgfpathlineto{\pgfqpoint{3.243927in}{2.309617in}}%
\pgfpathlineto{\pgfqpoint{3.248436in}{2.381106in}}%
\pgfpathlineto{\pgfqpoint{3.252945in}{2.345362in}}%
\pgfpathlineto{\pgfqpoint{3.257455in}{2.370894in}}%
\pgfpathlineto{\pgfqpoint{3.261964in}{2.319830in}}%
\pgfpathlineto{\pgfqpoint{3.266473in}{2.314723in}}%
\pgfpathlineto{\pgfqpoint{3.270982in}{2.411745in}}%
\pgfpathlineto{\pgfqpoint{3.275491in}{2.381106in}}%
\pgfpathlineto{\pgfqpoint{3.280000in}{2.381106in}}%
\pgfpathlineto{\pgfqpoint{3.284509in}{2.427064in}}%
\pgfpathlineto{\pgfqpoint{3.289018in}{2.396426in}}%
\pgfpathlineto{\pgfqpoint{3.293527in}{2.345362in}}%
\pgfpathlineto{\pgfqpoint{3.298036in}{2.473021in}}%
\pgfpathlineto{\pgfqpoint{3.302545in}{2.360681in}}%
\pgfpathlineto{\pgfqpoint{3.307055in}{2.416851in}}%
\pgfpathlineto{\pgfqpoint{3.311564in}{2.457702in}}%
\pgfpathlineto{\pgfqpoint{3.316073in}{2.391319in}}%
\pgfpathlineto{\pgfqpoint{3.320582in}{2.365787in}}%
\pgfpathlineto{\pgfqpoint{3.325091in}{2.447489in}}%
\pgfpathlineto{\pgfqpoint{3.329600in}{2.462809in}}%
\pgfpathlineto{\pgfqpoint{3.334109in}{2.432170in}}%
\pgfpathlineto{\pgfqpoint{3.338618in}{2.452596in}}%
\pgfpathlineto{\pgfqpoint{3.343127in}{2.447489in}}%
\pgfpathlineto{\pgfqpoint{3.347636in}{2.381106in}}%
\pgfpathlineto{\pgfqpoint{3.352145in}{2.503660in}}%
\pgfpathlineto{\pgfqpoint{3.356655in}{2.498553in}}%
\pgfpathlineto{\pgfqpoint{3.361164in}{2.376000in}}%
\pgfpathlineto{\pgfqpoint{3.365673in}{2.427064in}}%
\pgfpathlineto{\pgfqpoint{3.370182in}{2.503660in}}%
\pgfpathlineto{\pgfqpoint{3.374691in}{2.427064in}}%
\pgfpathlineto{\pgfqpoint{3.379200in}{2.534298in}}%
\pgfpathlineto{\pgfqpoint{3.383709in}{2.365787in}}%
\pgfpathlineto{\pgfqpoint{3.388218in}{2.493447in}}%
\pgfpathlineto{\pgfqpoint{3.392727in}{2.473021in}}%
\pgfpathlineto{\pgfqpoint{3.397236in}{2.478128in}}%
\pgfpathlineto{\pgfqpoint{3.401745in}{2.411745in}}%
\pgfpathlineto{\pgfqpoint{3.406255in}{2.590468in}}%
\pgfpathlineto{\pgfqpoint{3.410764in}{2.508766in}}%
\pgfpathlineto{\pgfqpoint{3.415273in}{2.457702in}}%
\pgfpathlineto{\pgfqpoint{3.419782in}{2.437277in}}%
\pgfpathlineto{\pgfqpoint{3.424291in}{2.488340in}}%
\pgfpathlineto{\pgfqpoint{3.428800in}{2.641532in}}%
\pgfpathlineto{\pgfqpoint{3.433309in}{2.575149in}}%
\pgfpathlineto{\pgfqpoint{3.437818in}{2.488340in}}%
\pgfpathlineto{\pgfqpoint{3.442327in}{2.518979in}}%
\pgfpathlineto{\pgfqpoint{3.446836in}{2.595574in}}%
\pgfpathlineto{\pgfqpoint{3.451345in}{2.595574in}}%
\pgfpathlineto{\pgfqpoint{3.455855in}{2.508766in}}%
\pgfpathlineto{\pgfqpoint{3.460364in}{2.524085in}}%
\pgfpathlineto{\pgfqpoint{3.464873in}{2.534298in}}%
\pgfpathlineto{\pgfqpoint{3.469382in}{2.534298in}}%
\pgfpathlineto{\pgfqpoint{3.473891in}{2.626213in}}%
\pgfpathlineto{\pgfqpoint{3.478400in}{2.432170in}}%
\pgfpathlineto{\pgfqpoint{3.482909in}{2.575149in}}%
\pgfpathlineto{\pgfqpoint{3.487418in}{2.595574in}}%
\pgfpathlineto{\pgfqpoint{3.491927in}{2.508766in}}%
\pgfpathlineto{\pgfqpoint{3.496436in}{2.473021in}}%
\pgfpathlineto{\pgfqpoint{3.505455in}{2.564936in}}%
\pgfpathlineto{\pgfqpoint{3.509964in}{2.559830in}}%
\pgfpathlineto{\pgfqpoint{3.514473in}{2.616000in}}%
\pgfpathlineto{\pgfqpoint{3.518982in}{2.529191in}}%
\pgfpathlineto{\pgfqpoint{3.523491in}{2.544511in}}%
\pgfpathlineto{\pgfqpoint{3.528000in}{2.631319in}}%
\pgfpathlineto{\pgfqpoint{3.532509in}{2.564936in}}%
\pgfpathlineto{\pgfqpoint{3.537018in}{2.590468in}}%
\pgfpathlineto{\pgfqpoint{3.541527in}{2.636426in}}%
\pgfpathlineto{\pgfqpoint{3.546036in}{2.600681in}}%
\pgfpathlineto{\pgfqpoint{3.550545in}{2.554723in}}%
\pgfpathlineto{\pgfqpoint{3.555055in}{2.651745in}}%
\pgfpathlineto{\pgfqpoint{3.559564in}{2.570043in}}%
\pgfpathlineto{\pgfqpoint{3.564073in}{2.575149in}}%
\pgfpathlineto{\pgfqpoint{3.568582in}{2.636426in}}%
\pgfpathlineto{\pgfqpoint{3.573091in}{2.636426in}}%
\pgfpathlineto{\pgfqpoint{3.577600in}{2.595574in}}%
\pgfpathlineto{\pgfqpoint{3.582109in}{2.590468in}}%
\pgfpathlineto{\pgfqpoint{3.586618in}{2.697702in}}%
\pgfpathlineto{\pgfqpoint{3.595636in}{2.580255in}}%
\pgfpathlineto{\pgfqpoint{3.600145in}{2.641532in}}%
\pgfpathlineto{\pgfqpoint{3.604655in}{2.636426in}}%
\pgfpathlineto{\pgfqpoint{3.609164in}{2.672170in}}%
\pgfpathlineto{\pgfqpoint{3.613673in}{2.646638in}}%
\pgfpathlineto{\pgfqpoint{3.618182in}{2.656851in}}%
\pgfpathlineto{\pgfqpoint{3.622691in}{2.564936in}}%
\pgfpathlineto{\pgfqpoint{3.627200in}{2.718128in}}%
\pgfpathlineto{\pgfqpoint{3.631709in}{2.728340in}}%
\pgfpathlineto{\pgfqpoint{3.636218in}{2.631319in}}%
\pgfpathlineto{\pgfqpoint{3.640727in}{2.697702in}}%
\pgfpathlineto{\pgfqpoint{3.645236in}{2.580255in}}%
\pgfpathlineto{\pgfqpoint{3.649745in}{2.723234in}}%
\pgfpathlineto{\pgfqpoint{3.654255in}{2.713021in}}%
\pgfpathlineto{\pgfqpoint{3.658764in}{2.789617in}}%
\pgfpathlineto{\pgfqpoint{3.663273in}{2.677277in}}%
\pgfpathlineto{\pgfqpoint{3.667782in}{2.605787in}}%
\pgfpathlineto{\pgfqpoint{3.672291in}{2.764085in}}%
\pgfpathlineto{\pgfqpoint{3.676800in}{2.789617in}}%
\pgfpathlineto{\pgfqpoint{3.681309in}{2.672170in}}%
\pgfpathlineto{\pgfqpoint{3.685818in}{2.713021in}}%
\pgfpathlineto{\pgfqpoint{3.690327in}{2.636426in}}%
\pgfpathlineto{\pgfqpoint{3.694836in}{2.758979in}}%
\pgfpathlineto{\pgfqpoint{3.699345in}{2.738553in}}%
\pgfpathlineto{\pgfqpoint{3.703855in}{2.661957in}}%
\pgfpathlineto{\pgfqpoint{3.708364in}{2.718128in}}%
\pgfpathlineto{\pgfqpoint{3.712873in}{2.687489in}}%
\pgfpathlineto{\pgfqpoint{3.717382in}{2.672170in}}%
\pgfpathlineto{\pgfqpoint{3.721891in}{2.667064in}}%
\pgfpathlineto{\pgfqpoint{3.726400in}{2.774298in}}%
\pgfpathlineto{\pgfqpoint{3.730909in}{2.707915in}}%
\pgfpathlineto{\pgfqpoint{3.739927in}{2.789617in}}%
\pgfpathlineto{\pgfqpoint{3.744436in}{2.856000in}}%
\pgfpathlineto{\pgfqpoint{3.748945in}{2.728340in}}%
\pgfpathlineto{\pgfqpoint{3.753455in}{2.758979in}}%
\pgfpathlineto{\pgfqpoint{3.757964in}{2.815149in}}%
\pgfpathlineto{\pgfqpoint{3.762473in}{2.810043in}}%
\pgfpathlineto{\pgfqpoint{3.766982in}{2.845787in}}%
\pgfpathlineto{\pgfqpoint{3.771491in}{2.702809in}}%
\pgfpathlineto{\pgfqpoint{3.776000in}{2.697702in}}%
\pgfpathlineto{\pgfqpoint{3.780509in}{2.856000in}}%
\pgfpathlineto{\pgfqpoint{3.785018in}{2.896851in}}%
\pgfpathlineto{\pgfqpoint{3.789527in}{2.804936in}}%
\pgfpathlineto{\pgfqpoint{3.794036in}{2.804936in}}%
\pgfpathlineto{\pgfqpoint{3.798545in}{2.779404in}}%
\pgfpathlineto{\pgfqpoint{3.803055in}{2.784511in}}%
\pgfpathlineto{\pgfqpoint{3.807564in}{2.784511in}}%
\pgfpathlineto{\pgfqpoint{3.812073in}{2.804936in}}%
\pgfpathlineto{\pgfqpoint{3.816582in}{2.738553in}}%
\pgfpathlineto{\pgfqpoint{3.821091in}{2.856000in}}%
\pgfpathlineto{\pgfqpoint{3.825600in}{2.850894in}}%
\pgfpathlineto{\pgfqpoint{3.830109in}{2.794723in}}%
\pgfpathlineto{\pgfqpoint{3.834618in}{2.784511in}}%
\pgfpathlineto{\pgfqpoint{3.839127in}{2.947915in}}%
\pgfpathlineto{\pgfqpoint{3.843636in}{2.845787in}}%
\pgfpathlineto{\pgfqpoint{3.848145in}{2.912170in}}%
\pgfpathlineto{\pgfqpoint{3.852655in}{2.830468in}}%
\pgfpathlineto{\pgfqpoint{3.857164in}{2.871319in}}%
\pgfpathlineto{\pgfqpoint{3.861673in}{2.769191in}}%
\pgfpathlineto{\pgfqpoint{3.866182in}{2.815149in}}%
\pgfpathlineto{\pgfqpoint{3.870691in}{2.891745in}}%
\pgfpathlineto{\pgfqpoint{3.875200in}{2.789617in}}%
\pgfpathlineto{\pgfqpoint{3.879709in}{2.861106in}}%
\pgfpathlineto{\pgfqpoint{3.884218in}{2.901957in}}%
\pgfpathlineto{\pgfqpoint{3.888727in}{2.876426in}}%
\pgfpathlineto{\pgfqpoint{3.893236in}{2.835574in}}%
\pgfpathlineto{\pgfqpoint{3.897745in}{2.973447in}}%
\pgfpathlineto{\pgfqpoint{3.902255in}{2.850894in}}%
\pgfpathlineto{\pgfqpoint{3.906764in}{2.891745in}}%
\pgfpathlineto{\pgfqpoint{3.911273in}{2.891745in}}%
\pgfpathlineto{\pgfqpoint{3.915782in}{2.799830in}}%
\pgfpathlineto{\pgfqpoint{3.920291in}{2.891745in}}%
\pgfpathlineto{\pgfqpoint{3.924800in}{2.794723in}}%
\pgfpathlineto{\pgfqpoint{3.933818in}{2.825362in}}%
\pgfpathlineto{\pgfqpoint{3.938327in}{3.050043in}}%
\pgfpathlineto{\pgfqpoint{3.942836in}{3.039830in}}%
\pgfpathlineto{\pgfqpoint{3.947345in}{2.845787in}}%
\pgfpathlineto{\pgfqpoint{3.951855in}{2.861106in}}%
\pgfpathlineto{\pgfqpoint{3.956364in}{2.825362in}}%
\pgfpathlineto{\pgfqpoint{3.960873in}{2.876426in}}%
\pgfpathlineto{\pgfqpoint{3.965382in}{2.963234in}}%
\pgfpathlineto{\pgfqpoint{3.969891in}{2.953021in}}%
\pgfpathlineto{\pgfqpoint{3.974400in}{3.004085in}}%
\pgfpathlineto{\pgfqpoint{3.978909in}{2.866213in}}%
\pgfpathlineto{\pgfqpoint{3.983418in}{2.881532in}}%
\pgfpathlineto{\pgfqpoint{3.987927in}{2.988766in}}%
\pgfpathlineto{\pgfqpoint{3.992436in}{2.907064in}}%
\pgfpathlineto{\pgfqpoint{3.996945in}{3.009191in}}%
\pgfpathlineto{\pgfqpoint{4.001455in}{3.050043in}}%
\pgfpathlineto{\pgfqpoint{4.005964in}{2.978553in}}%
\pgfpathlineto{\pgfqpoint{4.010473in}{2.947915in}}%
\pgfpathlineto{\pgfqpoint{4.014982in}{2.983660in}}%
\pgfpathlineto{\pgfqpoint{4.019491in}{2.942809in}}%
\pgfpathlineto{\pgfqpoint{4.024000in}{2.998979in}}%
\pgfpathlineto{\pgfqpoint{4.028509in}{3.039830in}}%
\pgfpathlineto{\pgfqpoint{4.033018in}{2.973447in}}%
\pgfpathlineto{\pgfqpoint{4.037527in}{2.958128in}}%
\pgfpathlineto{\pgfqpoint{4.042036in}{2.901957in}}%
\pgfpathlineto{\pgfqpoint{4.046545in}{2.978553in}}%
\pgfpathlineto{\pgfqpoint{4.051055in}{3.116426in}}%
\pgfpathlineto{\pgfqpoint{4.055564in}{2.988766in}}%
\pgfpathlineto{\pgfqpoint{4.060073in}{3.080681in}}%
\pgfpathlineto{\pgfqpoint{4.064582in}{3.065362in}}%
\pgfpathlineto{\pgfqpoint{4.073600in}{2.998979in}}%
\pgfpathlineto{\pgfqpoint{4.078109in}{3.039830in}}%
\pgfpathlineto{\pgfqpoint{4.082618in}{3.065362in}}%
\pgfpathlineto{\pgfqpoint{4.087127in}{3.019404in}}%
\pgfpathlineto{\pgfqpoint{4.091636in}{3.060255in}}%
\pgfpathlineto{\pgfqpoint{4.096145in}{2.998979in}}%
\pgfpathlineto{\pgfqpoint{4.100655in}{3.106213in}}%
\pgfpathlineto{\pgfqpoint{4.105164in}{3.085787in}}%
\pgfpathlineto{\pgfqpoint{4.109673in}{3.014298in}}%
\pgfpathlineto{\pgfqpoint{4.114182in}{2.988766in}}%
\pgfpathlineto{\pgfqpoint{4.118691in}{3.014298in}}%
\pgfpathlineto{\pgfqpoint{4.123200in}{2.988766in}}%
\pgfpathlineto{\pgfqpoint{4.127709in}{3.014298in}}%
\pgfpathlineto{\pgfqpoint{4.132218in}{3.004085in}}%
\pgfpathlineto{\pgfqpoint{4.136727in}{3.019404in}}%
\pgfpathlineto{\pgfqpoint{4.141236in}{3.080681in}}%
\pgfpathlineto{\pgfqpoint{4.145745in}{2.998979in}}%
\pgfpathlineto{\pgfqpoint{4.150255in}{3.090894in}}%
\pgfpathlineto{\pgfqpoint{4.154764in}{3.131745in}}%
\pgfpathlineto{\pgfqpoint{4.159273in}{3.060255in}}%
\pgfpathlineto{\pgfqpoint{4.163782in}{3.131745in}}%
\pgfpathlineto{\pgfqpoint{4.168291in}{3.075574in}}%
\pgfpathlineto{\pgfqpoint{4.172800in}{3.101106in}}%
\pgfpathlineto{\pgfqpoint{4.177309in}{3.024511in}}%
\pgfpathlineto{\pgfqpoint{4.181818in}{3.085787in}}%
\pgfpathlineto{\pgfqpoint{4.186327in}{3.106213in}}%
\pgfpathlineto{\pgfqpoint{4.190836in}{3.167489in}}%
\pgfpathlineto{\pgfqpoint{4.195345in}{3.039830in}}%
\pgfpathlineto{\pgfqpoint{4.199855in}{3.096000in}}%
\pgfpathlineto{\pgfqpoint{4.204364in}{3.126638in}}%
\pgfpathlineto{\pgfqpoint{4.208873in}{3.101106in}}%
\pgfpathlineto{\pgfqpoint{4.213382in}{3.060255in}}%
\pgfpathlineto{\pgfqpoint{4.217891in}{2.973447in}}%
\pgfpathlineto{\pgfqpoint{4.222400in}{3.096000in}}%
\pgfpathlineto{\pgfqpoint{4.226909in}{3.162383in}}%
\pgfpathlineto{\pgfqpoint{4.231418in}{3.044936in}}%
\pgfpathlineto{\pgfqpoint{4.235927in}{3.090894in}}%
\pgfpathlineto{\pgfqpoint{4.240436in}{3.111319in}}%
\pgfpathlineto{\pgfqpoint{4.244945in}{3.121532in}}%
\pgfpathlineto{\pgfqpoint{4.249455in}{3.111319in}}%
\pgfpathlineto{\pgfqpoint{4.253964in}{3.080681in}}%
\pgfpathlineto{\pgfqpoint{4.258473in}{3.147064in}}%
\pgfpathlineto{\pgfqpoint{4.262982in}{3.131745in}}%
\pgfpathlineto{\pgfqpoint{4.267491in}{3.157277in}}%
\pgfpathlineto{\pgfqpoint{4.272000in}{3.116426in}}%
\pgfpathlineto{\pgfqpoint{4.276509in}{3.193021in}}%
\pgfpathlineto{\pgfqpoint{4.281018in}{3.167489in}}%
\pgfpathlineto{\pgfqpoint{4.285527in}{3.070468in}}%
\pgfpathlineto{\pgfqpoint{4.290036in}{3.111319in}}%
\pgfpathlineto{\pgfqpoint{4.294545in}{3.065362in}}%
\pgfpathlineto{\pgfqpoint{4.299055in}{2.993872in}}%
\pgfpathlineto{\pgfqpoint{4.303564in}{3.075574in}}%
\pgfpathlineto{\pgfqpoint{4.308073in}{3.187915in}}%
\pgfpathlineto{\pgfqpoint{4.312582in}{3.126638in}}%
\pgfpathlineto{\pgfqpoint{4.317091in}{3.249191in}}%
\pgfpathlineto{\pgfqpoint{4.321600in}{3.152170in}}%
\pgfpathlineto{\pgfqpoint{4.326109in}{3.157277in}}%
\pgfpathlineto{\pgfqpoint{4.330618in}{3.259404in}}%
\pgfpathlineto{\pgfqpoint{4.335127in}{3.152170in}}%
\pgfpathlineto{\pgfqpoint{4.339636in}{3.238979in}}%
\pgfpathlineto{\pgfqpoint{4.344145in}{3.121532in}}%
\pgfpathlineto{\pgfqpoint{4.348655in}{3.152170in}}%
\pgfpathlineto{\pgfqpoint{4.353164in}{3.213447in}}%
\pgfpathlineto{\pgfqpoint{4.357673in}{3.233872in}}%
\pgfpathlineto{\pgfqpoint{4.362182in}{3.152170in}}%
\pgfpathlineto{\pgfqpoint{4.366691in}{3.218553in}}%
\pgfpathlineto{\pgfqpoint{4.371200in}{3.315574in}}%
\pgfpathlineto{\pgfqpoint{4.375709in}{3.075574in}}%
\pgfpathlineto{\pgfqpoint{4.380218in}{3.203234in}}%
\pgfpathlineto{\pgfqpoint{4.384727in}{3.233872in}}%
\pgfpathlineto{\pgfqpoint{4.389236in}{3.198128in}}%
\pgfpathlineto{\pgfqpoint{4.393745in}{3.244085in}}%
\pgfpathlineto{\pgfqpoint{4.398255in}{3.254298in}}%
\pgfpathlineto{\pgfqpoint{4.402764in}{3.249191in}}%
\pgfpathlineto{\pgfqpoint{4.407273in}{3.228766in}}%
\pgfpathlineto{\pgfqpoint{4.411782in}{3.269617in}}%
\pgfpathlineto{\pgfqpoint{4.420800in}{3.300255in}}%
\pgfpathlineto{\pgfqpoint{4.425309in}{3.238979in}}%
\pgfpathlineto{\pgfqpoint{4.429818in}{3.162383in}}%
\pgfpathlineto{\pgfqpoint{4.434327in}{3.254298in}}%
\pgfpathlineto{\pgfqpoint{4.438836in}{3.433021in}}%
\pgfpathlineto{\pgfqpoint{4.443345in}{3.177702in}}%
\pgfpathlineto{\pgfqpoint{4.447855in}{3.193021in}}%
\pgfpathlineto{\pgfqpoint{4.452364in}{3.279830in}}%
\pgfpathlineto{\pgfqpoint{4.456873in}{3.279830in}}%
\pgfpathlineto{\pgfqpoint{4.461382in}{3.254298in}}%
\pgfpathlineto{\pgfqpoint{4.465891in}{3.244085in}}%
\pgfpathlineto{\pgfqpoint{4.470400in}{3.131745in}}%
\pgfpathlineto{\pgfqpoint{4.474909in}{3.228766in}}%
\pgfpathlineto{\pgfqpoint{4.479418in}{3.371745in}}%
\pgfpathlineto{\pgfqpoint{4.483927in}{3.269617in}}%
\pgfpathlineto{\pgfqpoint{4.488436in}{3.290043in}}%
\pgfpathlineto{\pgfqpoint{4.492945in}{3.330894in}}%
\pgfpathlineto{\pgfqpoint{4.501964in}{3.310468in}}%
\pgfpathlineto{\pgfqpoint{4.506473in}{3.213447in}}%
\pgfpathlineto{\pgfqpoint{4.510982in}{3.233872in}}%
\pgfpathlineto{\pgfqpoint{4.515491in}{3.305362in}}%
\pgfpathlineto{\pgfqpoint{4.520000in}{3.290043in}}%
\pgfpathlineto{\pgfqpoint{4.524509in}{3.249191in}}%
\pgfpathlineto{\pgfqpoint{4.529018in}{3.433021in}}%
\pgfpathlineto{\pgfqpoint{4.533527in}{3.336000in}}%
\pgfpathlineto{\pgfqpoint{4.538036in}{3.208340in}}%
\pgfpathlineto{\pgfqpoint{4.542545in}{3.254298in}}%
\pgfpathlineto{\pgfqpoint{4.547055in}{3.315574in}}%
\pgfpathlineto{\pgfqpoint{4.551564in}{3.284936in}}%
\pgfpathlineto{\pgfqpoint{4.556073in}{3.279830in}}%
\pgfpathlineto{\pgfqpoint{4.560582in}{3.315574in}}%
\pgfpathlineto{\pgfqpoint{4.565091in}{3.269617in}}%
\pgfpathlineto{\pgfqpoint{4.569600in}{3.320681in}}%
\pgfpathlineto{\pgfqpoint{4.574109in}{3.320681in}}%
\pgfpathlineto{\pgfqpoint{4.578618in}{3.295149in}}%
\pgfpathlineto{\pgfqpoint{4.583127in}{3.341106in}}%
\pgfpathlineto{\pgfqpoint{4.587636in}{3.325787in}}%
\pgfpathlineto{\pgfqpoint{4.592145in}{3.376851in}}%
\pgfpathlineto{\pgfqpoint{4.596655in}{3.346213in}}%
\pgfpathlineto{\pgfqpoint{4.601164in}{3.325787in}}%
\pgfpathlineto{\pgfqpoint{4.605673in}{3.366638in}}%
\pgfpathlineto{\pgfqpoint{4.610182in}{3.315574in}}%
\pgfpathlineto{\pgfqpoint{4.614691in}{3.351319in}}%
\pgfpathlineto{\pgfqpoint{4.619200in}{3.412596in}}%
\pgfpathlineto{\pgfqpoint{4.623709in}{3.341106in}}%
\pgfpathlineto{\pgfqpoint{4.628218in}{3.315574in}}%
\pgfpathlineto{\pgfqpoint{4.632727in}{3.392170in}}%
\pgfpathlineto{\pgfqpoint{4.637236in}{3.407489in}}%
\pgfpathlineto{\pgfqpoint{4.641745in}{3.407489in}}%
\pgfpathlineto{\pgfqpoint{4.646255in}{3.223660in}}%
\pgfpathlineto{\pgfqpoint{4.650764in}{3.366638in}}%
\pgfpathlineto{\pgfqpoint{4.655273in}{3.412596in}}%
\pgfpathlineto{\pgfqpoint{4.659782in}{3.376851in}}%
\pgfpathlineto{\pgfqpoint{4.664291in}{3.387064in}}%
\pgfpathlineto{\pgfqpoint{4.668800in}{3.371745in}}%
\pgfpathlineto{\pgfqpoint{4.673309in}{3.341106in}}%
\pgfpathlineto{\pgfqpoint{4.677818in}{3.387064in}}%
\pgfpathlineto{\pgfqpoint{4.682327in}{3.570894in}}%
\pgfpathlineto{\pgfqpoint{4.686836in}{3.412596in}}%
\pgfpathlineto{\pgfqpoint{4.691345in}{3.407489in}}%
\pgfpathlineto{\pgfqpoint{4.695855in}{3.376851in}}%
\pgfpathlineto{\pgfqpoint{4.700364in}{3.453447in}}%
\pgfpathlineto{\pgfqpoint{4.704873in}{3.504511in}}%
\pgfpathlineto{\pgfqpoint{4.709382in}{3.397277in}}%
\pgfpathlineto{\pgfqpoint{4.713891in}{3.346213in}}%
\pgfpathlineto{\pgfqpoint{4.718400in}{3.315574in}}%
\pgfpathlineto{\pgfqpoint{4.722909in}{3.524936in}}%
\pgfpathlineto{\pgfqpoint{4.727418in}{3.300255in}}%
\pgfpathlineto{\pgfqpoint{4.731927in}{3.397277in}}%
\pgfpathlineto{\pgfqpoint{4.736436in}{3.443234in}}%
\pgfpathlineto{\pgfqpoint{4.740945in}{3.422809in}}%
\pgfpathlineto{\pgfqpoint{4.745455in}{3.417702in}}%
\pgfpathlineto{\pgfqpoint{4.749964in}{3.463660in}}%
\pgfpathlineto{\pgfqpoint{4.754473in}{3.448340in}}%
\pgfpathlineto{\pgfqpoint{4.758982in}{3.407489in}}%
\pgfpathlineto{\pgfqpoint{4.763491in}{3.443234in}}%
\pgfpathlineto{\pgfqpoint{4.768000in}{3.448340in}}%
\pgfpathlineto{\pgfqpoint{4.772509in}{3.468766in}}%
\pgfpathlineto{\pgfqpoint{4.777018in}{3.381957in}}%
\pgfpathlineto{\pgfqpoint{4.781527in}{3.468766in}}%
\pgfpathlineto{\pgfqpoint{4.786036in}{3.453447in}}%
\pgfpathlineto{\pgfqpoint{4.790545in}{3.295149in}}%
\pgfpathlineto{\pgfqpoint{4.795055in}{3.417702in}}%
\pgfpathlineto{\pgfqpoint{4.799564in}{3.453447in}}%
\pgfpathlineto{\pgfqpoint{4.804073in}{3.402383in}}%
\pgfpathlineto{\pgfqpoint{4.808582in}{3.448340in}}%
\pgfpathlineto{\pgfqpoint{4.813091in}{3.453447in}}%
\pgfpathlineto{\pgfqpoint{4.817600in}{3.433021in}}%
\pgfpathlineto{\pgfqpoint{4.822109in}{3.397277in}}%
\pgfpathlineto{\pgfqpoint{4.826618in}{3.397277in}}%
\pgfpathlineto{\pgfqpoint{4.831127in}{3.433021in}}%
\pgfpathlineto{\pgfqpoint{4.835636in}{3.433021in}}%
\pgfpathlineto{\pgfqpoint{4.840145in}{3.484085in}}%
\pgfpathlineto{\pgfqpoint{4.844655in}{3.652596in}}%
\pgfpathlineto{\pgfqpoint{4.849164in}{3.509617in}}%
\pgfpathlineto{\pgfqpoint{4.853673in}{3.478979in}}%
\pgfpathlineto{\pgfqpoint{4.858182in}{3.524936in}}%
\pgfpathlineto{\pgfqpoint{4.867200in}{3.504511in}}%
\pgfpathlineto{\pgfqpoint{4.871709in}{3.473872in}}%
\pgfpathlineto{\pgfqpoint{4.876218in}{3.530043in}}%
\pgfpathlineto{\pgfqpoint{4.880727in}{3.484085in}}%
\pgfpathlineto{\pgfqpoint{4.885236in}{3.621957in}}%
\pgfpathlineto{\pgfqpoint{4.889745in}{3.397277in}}%
\pgfpathlineto{\pgfqpoint{4.894255in}{3.621957in}}%
\pgfpathlineto{\pgfqpoint{4.898764in}{3.463660in}}%
\pgfpathlineto{\pgfqpoint{4.903273in}{3.427915in}}%
\pgfpathlineto{\pgfqpoint{4.907782in}{3.530043in}}%
\pgfpathlineto{\pgfqpoint{4.912291in}{3.535149in}}%
\pgfpathlineto{\pgfqpoint{4.916800in}{3.514723in}}%
\pgfpathlineto{\pgfqpoint{4.921309in}{3.341106in}}%
\pgfpathlineto{\pgfqpoint{4.925818in}{3.453447in}}%
\pgfpathlineto{\pgfqpoint{4.930327in}{3.632170in}}%
\pgfpathlineto{\pgfqpoint{4.934836in}{3.504511in}}%
\pgfpathlineto{\pgfqpoint{4.939345in}{3.535149in}}%
\pgfpathlineto{\pgfqpoint{4.943855in}{3.504511in}}%
\pgfpathlineto{\pgfqpoint{4.948364in}{3.601532in}}%
\pgfpathlineto{\pgfqpoint{4.952873in}{3.637277in}}%
\pgfpathlineto{\pgfqpoint{4.957382in}{3.540255in}}%
\pgfpathlineto{\pgfqpoint{4.961891in}{3.576000in}}%
\pgfpathlineto{\pgfqpoint{4.966400in}{3.524936in}}%
\pgfpathlineto{\pgfqpoint{4.970909in}{3.606638in}}%
\pgfpathlineto{\pgfqpoint{4.975418in}{3.463660in}}%
\pgfpathlineto{\pgfqpoint{4.979927in}{3.484085in}}%
\pgfpathlineto{\pgfqpoint{4.984436in}{3.468766in}}%
\pgfpathlineto{\pgfqpoint{4.988945in}{3.550468in}}%
\pgfpathlineto{\pgfqpoint{4.993455in}{3.611745in}}%
\pgfpathlineto{\pgfqpoint{4.997964in}{3.550468in}}%
\pgfpathlineto{\pgfqpoint{5.002473in}{3.591319in}}%
\pgfpathlineto{\pgfqpoint{5.006982in}{3.433021in}}%
\pgfpathlineto{\pgfqpoint{5.011491in}{3.581106in}}%
\pgfpathlineto{\pgfqpoint{5.016000in}{3.611745in}}%
\pgfpathlineto{\pgfqpoint{5.025018in}{3.581106in}}%
\pgfpathlineto{\pgfqpoint{5.034036in}{3.627064in}}%
\pgfpathlineto{\pgfqpoint{5.038545in}{3.586213in}}%
\pgfpathlineto{\pgfqpoint{5.043055in}{3.463660in}}%
\pgfpathlineto{\pgfqpoint{5.047564in}{3.606638in}}%
\pgfpathlineto{\pgfqpoint{5.052073in}{3.627064in}}%
\pgfpathlineto{\pgfqpoint{5.056582in}{3.591319in}}%
\pgfpathlineto{\pgfqpoint{5.061091in}{3.494298in}}%
\pgfpathlineto{\pgfqpoint{5.065600in}{3.448340in}}%
\pgfpathlineto{\pgfqpoint{5.070109in}{3.555574in}}%
\pgfpathlineto{\pgfqpoint{5.074618in}{3.621957in}}%
\pgfpathlineto{\pgfqpoint{5.079127in}{3.739404in}}%
\pgfpathlineto{\pgfqpoint{5.088145in}{3.596426in}}%
\pgfpathlineto{\pgfqpoint{5.092655in}{3.494298in}}%
\pgfpathlineto{\pgfqpoint{5.097164in}{3.540255in}}%
\pgfpathlineto{\pgfqpoint{5.101673in}{3.688340in}}%
\pgfpathlineto{\pgfqpoint{5.106182in}{3.616851in}}%
\pgfpathlineto{\pgfqpoint{5.110691in}{3.713872in}}%
\pgfpathlineto{\pgfqpoint{5.115200in}{3.576000in}}%
\pgfpathlineto{\pgfqpoint{5.119709in}{3.637277in}}%
\pgfpathlineto{\pgfqpoint{5.124218in}{3.810894in}}%
\pgfpathlineto{\pgfqpoint{5.128727in}{3.616851in}}%
\pgfpathlineto{\pgfqpoint{5.133236in}{3.678128in}}%
\pgfpathlineto{\pgfqpoint{5.137745in}{3.586213in}}%
\pgfpathlineto{\pgfqpoint{5.142255in}{3.601532in}}%
\pgfpathlineto{\pgfqpoint{5.146764in}{3.693447in}}%
\pgfpathlineto{\pgfqpoint{5.151273in}{3.621957in}}%
\pgfpathlineto{\pgfqpoint{5.155782in}{3.708766in}}%
\pgfpathlineto{\pgfqpoint{5.160291in}{3.662809in}}%
\pgfpathlineto{\pgfqpoint{5.164800in}{3.816000in}}%
\pgfpathlineto{\pgfqpoint{5.169309in}{3.627064in}}%
\pgfpathlineto{\pgfqpoint{5.173818in}{3.698553in}}%
\pgfpathlineto{\pgfqpoint{5.178327in}{3.673021in}}%
\pgfpathlineto{\pgfqpoint{5.182836in}{3.606638in}}%
\pgfpathlineto{\pgfqpoint{5.187345in}{3.662809in}}%
\pgfpathlineto{\pgfqpoint{5.191855in}{3.734298in}}%
\pgfpathlineto{\pgfqpoint{5.200873in}{3.688340in}}%
\pgfpathlineto{\pgfqpoint{5.205382in}{3.627064in}}%
\pgfpathlineto{\pgfqpoint{5.209891in}{3.673021in}}%
\pgfpathlineto{\pgfqpoint{5.214400in}{3.754723in}}%
\pgfpathlineto{\pgfqpoint{5.218909in}{3.734298in}}%
\pgfpathlineto{\pgfqpoint{5.223418in}{3.744511in}}%
\pgfpathlineto{\pgfqpoint{5.227927in}{3.724085in}}%
\pgfpathlineto{\pgfqpoint{5.232436in}{3.810894in}}%
\pgfpathlineto{\pgfqpoint{5.236945in}{3.759830in}}%
\pgfpathlineto{\pgfqpoint{5.241455in}{3.800681in}}%
\pgfpathlineto{\pgfqpoint{5.245964in}{3.688340in}}%
\pgfpathlineto{\pgfqpoint{5.250473in}{3.688340in}}%
\pgfpathlineto{\pgfqpoint{5.254982in}{3.785362in}}%
\pgfpathlineto{\pgfqpoint{5.259491in}{3.683234in}}%
\pgfpathlineto{\pgfqpoint{5.264000in}{3.754723in}}%
\pgfpathlineto{\pgfqpoint{5.268509in}{3.856851in}}%
\pgfpathlineto{\pgfqpoint{5.273018in}{3.795574in}}%
\pgfpathlineto{\pgfqpoint{5.277527in}{3.790468in}}%
\pgfpathlineto{\pgfqpoint{5.282036in}{3.775149in}}%
\pgfpathlineto{\pgfqpoint{5.286545in}{3.627064in}}%
\pgfpathlineto{\pgfqpoint{5.291055in}{3.749617in}}%
\pgfpathlineto{\pgfqpoint{5.295564in}{3.693447in}}%
\pgfpathlineto{\pgfqpoint{5.300073in}{3.861957in}}%
\pgfpathlineto{\pgfqpoint{5.304582in}{3.693447in}}%
\pgfpathlineto{\pgfqpoint{5.309091in}{3.805787in}}%
\pgfpathlineto{\pgfqpoint{5.313600in}{3.754723in}}%
\pgfpathlineto{\pgfqpoint{5.318109in}{3.800681in}}%
\pgfpathlineto{\pgfqpoint{5.322618in}{3.724085in}}%
\pgfpathlineto{\pgfqpoint{5.327127in}{3.713872in}}%
\pgfpathlineto{\pgfqpoint{5.331636in}{3.841532in}}%
\pgfpathlineto{\pgfqpoint{5.336145in}{3.902809in}}%
\pgfpathlineto{\pgfqpoint{5.340655in}{3.764936in}}%
\pgfpathlineto{\pgfqpoint{5.345164in}{3.969191in}}%
\pgfpathlineto{\pgfqpoint{5.349673in}{3.606638in}}%
\pgfpathlineto{\pgfqpoint{5.358691in}{3.851745in}}%
\pgfpathlineto{\pgfqpoint{5.363200in}{3.821106in}}%
\pgfpathlineto{\pgfqpoint{5.367709in}{3.841532in}}%
\pgfpathlineto{\pgfqpoint{5.372218in}{3.913021in}}%
\pgfpathlineto{\pgfqpoint{5.376727in}{3.759830in}}%
\pgfpathlineto{\pgfqpoint{5.381236in}{3.902809in}}%
\pgfpathlineto{\pgfqpoint{5.385745in}{3.918128in}}%
\pgfpathlineto{\pgfqpoint{5.390255in}{3.913021in}}%
\pgfpathlineto{\pgfqpoint{5.394764in}{3.933447in}}%
\pgfpathlineto{\pgfqpoint{5.399273in}{3.969191in}}%
\pgfpathlineto{\pgfqpoint{5.403782in}{3.887489in}}%
\pgfpathlineto{\pgfqpoint{5.408291in}{3.872170in}}%
\pgfpathlineto{\pgfqpoint{5.412800in}{3.958979in}}%
\pgfpathlineto{\pgfqpoint{5.417309in}{3.856851in}}%
\pgfpathlineto{\pgfqpoint{5.421818in}{3.943660in}}%
\pgfpathlineto{\pgfqpoint{5.426327in}{3.805787in}}%
\pgfpathlineto{\pgfqpoint{5.430836in}{3.948766in}}%
\pgfpathlineto{\pgfqpoint{5.435345in}{3.969191in}}%
\pgfpathlineto{\pgfqpoint{5.444364in}{3.821106in}}%
\pgfpathlineto{\pgfqpoint{5.448873in}{3.826213in}}%
\pgfpathlineto{\pgfqpoint{5.453382in}{3.918128in}}%
\pgfpathlineto{\pgfqpoint{5.457891in}{3.953872in}}%
\pgfpathlineto{\pgfqpoint{5.462400in}{3.907915in}}%
\pgfpathlineto{\pgfqpoint{5.466909in}{3.928340in}}%
\pgfpathlineto{\pgfqpoint{5.471418in}{4.010043in}}%
\pgfpathlineto{\pgfqpoint{5.475927in}{3.841532in}}%
\pgfpathlineto{\pgfqpoint{5.484945in}{4.056000in}}%
\pgfpathlineto{\pgfqpoint{5.489455in}{4.004936in}}%
\pgfpathlineto{\pgfqpoint{5.493964in}{4.010043in}}%
\pgfpathlineto{\pgfqpoint{5.498473in}{3.958979in}}%
\pgfpathlineto{\pgfqpoint{5.502982in}{4.010043in}}%
\pgfpathlineto{\pgfqpoint{5.507491in}{3.892596in}}%
\pgfpathlineto{\pgfqpoint{5.512000in}{3.999830in}}%
\pgfpathlineto{\pgfqpoint{5.516509in}{3.994723in}}%
\pgfpathlineto{\pgfqpoint{5.521018in}{3.948766in}}%
\pgfpathlineto{\pgfqpoint{5.525527in}{3.851745in}}%
\pgfpathlineto{\pgfqpoint{5.530036in}{3.984511in}}%
\pgfpathlineto{\pgfqpoint{5.534545in}{3.958979in}}%
\pgfpathlineto{\pgfqpoint{5.534545in}{3.958979in}}%
\pgfusepath{stroke}%
\end{pgfscope}%
\begin{pgfscope}%
\pgfsetrectcap%
\pgfsetmiterjoin%
\pgfsetlinewidth{0.803000pt}%
\definecolor{currentstroke}{rgb}{0.000000,0.000000,0.000000}%
\pgfsetstrokecolor{currentstroke}%
\pgfsetdash{}{0pt}%
\pgfpathmoveto{\pgfqpoint{0.800000in}{0.528000in}}%
\pgfpathlineto{\pgfqpoint{0.800000in}{4.224000in}}%
\pgfusepath{stroke}%
\end{pgfscope}%
\begin{pgfscope}%
\pgfsetrectcap%
\pgfsetmiterjoin%
\pgfsetlinewidth{0.803000pt}%
\definecolor{currentstroke}{rgb}{0.000000,0.000000,0.000000}%
\pgfsetstrokecolor{currentstroke}%
\pgfsetdash{}{0pt}%
\pgfpathmoveto{\pgfqpoint{5.760000in}{0.528000in}}%
\pgfpathlineto{\pgfqpoint{5.760000in}{4.224000in}}%
\pgfusepath{stroke}%
\end{pgfscope}%
\begin{pgfscope}%
\pgfsetrectcap%
\pgfsetmiterjoin%
\pgfsetlinewidth{0.803000pt}%
\definecolor{currentstroke}{rgb}{0.000000,0.000000,0.000000}%
\pgfsetstrokecolor{currentstroke}%
\pgfsetdash{}{0pt}%
\pgfpathmoveto{\pgfqpoint{0.800000in}{0.528000in}}%
\pgfpathlineto{\pgfqpoint{5.760000in}{0.528000in}}%
\pgfusepath{stroke}%
\end{pgfscope}%
\begin{pgfscope}%
\pgfsetrectcap%
\pgfsetmiterjoin%
\pgfsetlinewidth{0.803000pt}%
\definecolor{currentstroke}{rgb}{0.000000,0.000000,0.000000}%
\pgfsetstrokecolor{currentstroke}%
\pgfsetdash{}{0pt}%
\pgfpathmoveto{\pgfqpoint{0.800000in}{4.224000in}}%
\pgfpathlineto{\pgfqpoint{5.760000in}{4.224000in}}%
\pgfusepath{stroke}%
\end{pgfscope}%
\begin{pgfscope}%
\definecolor{textcolor}{rgb}{0.000000,0.000000,0.000000}%
\pgfsetstrokecolor{textcolor}%
\pgfsetfillcolor{textcolor}%
\pgftext[x=3.280000in,y=4.307333in,,base]{\color{textcolor}\ttfamily\fontsize{12.000000}{14.400000}\selectfont Merge Sort  Swaps vs Input size}%
\end{pgfscope}%
\begin{pgfscope}%
\pgfsetbuttcap%
\pgfsetmiterjoin%
\definecolor{currentfill}{rgb}{1.000000,1.000000,1.000000}%
\pgfsetfillcolor{currentfill}%
\pgfsetfillopacity{0.800000}%
\pgfsetlinewidth{1.003750pt}%
\definecolor{currentstroke}{rgb}{0.800000,0.800000,0.800000}%
\pgfsetstrokecolor{currentstroke}%
\pgfsetstrokeopacity{0.800000}%
\pgfsetdash{}{0pt}%
\pgfpathmoveto{\pgfqpoint{0.897222in}{3.907336in}}%
\pgfpathlineto{\pgfqpoint{1.759758in}{3.907336in}}%
\pgfpathquadraticcurveto{\pgfqpoint{1.787535in}{3.907336in}}{\pgfqpoint{1.787535in}{3.935114in}}%
\pgfpathlineto{\pgfqpoint{1.787535in}{4.126778in}}%
\pgfpathquadraticcurveto{\pgfqpoint{1.787535in}{4.154556in}}{\pgfqpoint{1.759758in}{4.154556in}}%
\pgfpathlineto{\pgfqpoint{0.897222in}{4.154556in}}%
\pgfpathquadraticcurveto{\pgfqpoint{0.869444in}{4.154556in}}{\pgfqpoint{0.869444in}{4.126778in}}%
\pgfpathlineto{\pgfqpoint{0.869444in}{3.935114in}}%
\pgfpathquadraticcurveto{\pgfqpoint{0.869444in}{3.907336in}}{\pgfqpoint{0.897222in}{3.907336in}}%
\pgfpathlineto{\pgfqpoint{0.897222in}{3.907336in}}%
\pgfpathclose%
\pgfusepath{stroke,fill}%
\end{pgfscope}%
\begin{pgfscope}%
\pgfsetrectcap%
\pgfsetroundjoin%
\pgfsetlinewidth{1.505625pt}%
\definecolor{currentstroke}{rgb}{0.000000,1.000000,0.498039}%
\pgfsetstrokecolor{currentstroke}%
\pgfsetdash{}{0pt}%
\pgfpathmoveto{\pgfqpoint{0.925000in}{4.041342in}}%
\pgfpathlineto{\pgfqpoint{1.063889in}{4.041342in}}%
\pgfpathlineto{\pgfqpoint{1.202778in}{4.041342in}}%
\pgfusepath{stroke}%
\end{pgfscope}%
\begin{pgfscope}%
\definecolor{textcolor}{rgb}{0.000000,0.000000,0.000000}%
\pgfsetstrokecolor{textcolor}%
\pgfsetfillcolor{textcolor}%
\pgftext[x=1.313889in,y=3.992731in,left,base]{\color{textcolor}\ttfamily\fontsize{10.000000}{12.000000}\selectfont Merge}%
\end{pgfscope}%
\end{pgfpicture}%
\makeatother%
\endgroup%

%% Creator: Matplotlib, PGF backend
%%
%% To include the figure in your LaTeX document, write
%%   \input{<filename>.pgf}
%%
%% Make sure the required packages are loaded in your preamble
%%   \usepackage{pgf}
%%
%% Also ensure that all the required font packages are loaded; for instance,
%% the lmodern package is sometimes necessary when using math font.
%%   \usepackage{lmodern}
%%
%% Figures using additional raster images can only be included by \input if
%% they are in the same directory as the main LaTeX file. For loading figures
%% from other directories you can use the `import` package
%%   \usepackage{import}
%%
%% and then include the figures with
%%   \import{<path to file>}{<filename>.pgf}
%%
%% Matplotlib used the following preamble
%%   \usepackage{fontspec}
%%   \setmainfont{DejaVuSerif.ttf}[Path=\detokenize{/home/dbk/.local/lib/python3.10/site-packages/matplotlib/mpl-data/fonts/ttf/}]
%%   \setsansfont{DejaVuSans.ttf}[Path=\detokenize{/home/dbk/.local/lib/python3.10/site-packages/matplotlib/mpl-data/fonts/ttf/}]
%%   \setmonofont{DejaVuSansMono.ttf}[Path=\detokenize{/home/dbk/.local/lib/python3.10/site-packages/matplotlib/mpl-data/fonts/ttf/}]
%%
\begingroup%
\makeatletter%
\begin{pgfpicture}%
\pgfpathrectangle{\pgfpointorigin}{\pgfqpoint{6.400000in}{4.800000in}}%
\pgfusepath{use as bounding box, clip}%
\begin{pgfscope}%
\pgfsetbuttcap%
\pgfsetmiterjoin%
\definecolor{currentfill}{rgb}{1.000000,1.000000,1.000000}%
\pgfsetfillcolor{currentfill}%
\pgfsetlinewidth{0.000000pt}%
\definecolor{currentstroke}{rgb}{1.000000,1.000000,1.000000}%
\pgfsetstrokecolor{currentstroke}%
\pgfsetdash{}{0pt}%
\pgfpathmoveto{\pgfqpoint{0.000000in}{0.000000in}}%
\pgfpathlineto{\pgfqpoint{6.400000in}{0.000000in}}%
\pgfpathlineto{\pgfqpoint{6.400000in}{4.800000in}}%
\pgfpathlineto{\pgfqpoint{0.000000in}{4.800000in}}%
\pgfpathlineto{\pgfqpoint{0.000000in}{0.000000in}}%
\pgfpathclose%
\pgfusepath{fill}%
\end{pgfscope}%
\begin{pgfscope}%
\pgfsetbuttcap%
\pgfsetmiterjoin%
\definecolor{currentfill}{rgb}{1.000000,1.000000,1.000000}%
\pgfsetfillcolor{currentfill}%
\pgfsetlinewidth{0.000000pt}%
\definecolor{currentstroke}{rgb}{0.000000,0.000000,0.000000}%
\pgfsetstrokecolor{currentstroke}%
\pgfsetstrokeopacity{0.000000}%
\pgfsetdash{}{0pt}%
\pgfpathmoveto{\pgfqpoint{0.800000in}{0.528000in}}%
\pgfpathlineto{\pgfqpoint{5.760000in}{0.528000in}}%
\pgfpathlineto{\pgfqpoint{5.760000in}{4.224000in}}%
\pgfpathlineto{\pgfqpoint{0.800000in}{4.224000in}}%
\pgfpathlineto{\pgfqpoint{0.800000in}{0.528000in}}%
\pgfpathclose%
\pgfusepath{fill}%
\end{pgfscope}%
\begin{pgfscope}%
\pgfsetbuttcap%
\pgfsetroundjoin%
\definecolor{currentfill}{rgb}{0.000000,0.000000,0.000000}%
\pgfsetfillcolor{currentfill}%
\pgfsetlinewidth{0.803000pt}%
\definecolor{currentstroke}{rgb}{0.000000,0.000000,0.000000}%
\pgfsetstrokecolor{currentstroke}%
\pgfsetdash{}{0pt}%
\pgfsys@defobject{currentmarker}{\pgfqpoint{0.000000in}{-0.048611in}}{\pgfqpoint{0.000000in}{0.000000in}}{%
\pgfpathmoveto{\pgfqpoint{0.000000in}{0.000000in}}%
\pgfpathlineto{\pgfqpoint{0.000000in}{-0.048611in}}%
\pgfusepath{stroke,fill}%
}%
\begin{pgfscope}%
\pgfsys@transformshift{1.020945in}{0.528000in}%
\pgfsys@useobject{currentmarker}{}%
\end{pgfscope}%
\end{pgfscope}%
\begin{pgfscope}%
\definecolor{textcolor}{rgb}{0.000000,0.000000,0.000000}%
\pgfsetstrokecolor{textcolor}%
\pgfsetfillcolor{textcolor}%
\pgftext[x=1.020945in,y=0.430778in,,top]{\color{textcolor}\ttfamily\fontsize{10.000000}{12.000000}\selectfont 0}%
\end{pgfscope}%
\begin{pgfscope}%
\pgfsetbuttcap%
\pgfsetroundjoin%
\definecolor{currentfill}{rgb}{0.000000,0.000000,0.000000}%
\pgfsetfillcolor{currentfill}%
\pgfsetlinewidth{0.803000pt}%
\definecolor{currentstroke}{rgb}{0.000000,0.000000,0.000000}%
\pgfsetstrokecolor{currentstroke}%
\pgfsetdash{}{0pt}%
\pgfsys@defobject{currentmarker}{\pgfqpoint{0.000000in}{-0.048611in}}{\pgfqpoint{0.000000in}{0.000000in}}{%
\pgfpathmoveto{\pgfqpoint{0.000000in}{0.000000in}}%
\pgfpathlineto{\pgfqpoint{0.000000in}{-0.048611in}}%
\pgfusepath{stroke,fill}%
}%
\begin{pgfscope}%
\pgfsys@transformshift{1.922764in}{0.528000in}%
\pgfsys@useobject{currentmarker}{}%
\end{pgfscope}%
\end{pgfscope}%
\begin{pgfscope}%
\definecolor{textcolor}{rgb}{0.000000,0.000000,0.000000}%
\pgfsetstrokecolor{textcolor}%
\pgfsetfillcolor{textcolor}%
\pgftext[x=1.922764in,y=0.430778in,,top]{\color{textcolor}\ttfamily\fontsize{10.000000}{12.000000}\selectfont 200}%
\end{pgfscope}%
\begin{pgfscope}%
\pgfsetbuttcap%
\pgfsetroundjoin%
\definecolor{currentfill}{rgb}{0.000000,0.000000,0.000000}%
\pgfsetfillcolor{currentfill}%
\pgfsetlinewidth{0.803000pt}%
\definecolor{currentstroke}{rgb}{0.000000,0.000000,0.000000}%
\pgfsetstrokecolor{currentstroke}%
\pgfsetdash{}{0pt}%
\pgfsys@defobject{currentmarker}{\pgfqpoint{0.000000in}{-0.048611in}}{\pgfqpoint{0.000000in}{0.000000in}}{%
\pgfpathmoveto{\pgfqpoint{0.000000in}{0.000000in}}%
\pgfpathlineto{\pgfqpoint{0.000000in}{-0.048611in}}%
\pgfusepath{stroke,fill}%
}%
\begin{pgfscope}%
\pgfsys@transformshift{2.824582in}{0.528000in}%
\pgfsys@useobject{currentmarker}{}%
\end{pgfscope}%
\end{pgfscope}%
\begin{pgfscope}%
\definecolor{textcolor}{rgb}{0.000000,0.000000,0.000000}%
\pgfsetstrokecolor{textcolor}%
\pgfsetfillcolor{textcolor}%
\pgftext[x=2.824582in,y=0.430778in,,top]{\color{textcolor}\ttfamily\fontsize{10.000000}{12.000000}\selectfont 400}%
\end{pgfscope}%
\begin{pgfscope}%
\pgfsetbuttcap%
\pgfsetroundjoin%
\definecolor{currentfill}{rgb}{0.000000,0.000000,0.000000}%
\pgfsetfillcolor{currentfill}%
\pgfsetlinewidth{0.803000pt}%
\definecolor{currentstroke}{rgb}{0.000000,0.000000,0.000000}%
\pgfsetstrokecolor{currentstroke}%
\pgfsetdash{}{0pt}%
\pgfsys@defobject{currentmarker}{\pgfqpoint{0.000000in}{-0.048611in}}{\pgfqpoint{0.000000in}{0.000000in}}{%
\pgfpathmoveto{\pgfqpoint{0.000000in}{0.000000in}}%
\pgfpathlineto{\pgfqpoint{0.000000in}{-0.048611in}}%
\pgfusepath{stroke,fill}%
}%
\begin{pgfscope}%
\pgfsys@transformshift{3.726400in}{0.528000in}%
\pgfsys@useobject{currentmarker}{}%
\end{pgfscope}%
\end{pgfscope}%
\begin{pgfscope}%
\definecolor{textcolor}{rgb}{0.000000,0.000000,0.000000}%
\pgfsetstrokecolor{textcolor}%
\pgfsetfillcolor{textcolor}%
\pgftext[x=3.726400in,y=0.430778in,,top]{\color{textcolor}\ttfamily\fontsize{10.000000}{12.000000}\selectfont 600}%
\end{pgfscope}%
\begin{pgfscope}%
\pgfsetbuttcap%
\pgfsetroundjoin%
\definecolor{currentfill}{rgb}{0.000000,0.000000,0.000000}%
\pgfsetfillcolor{currentfill}%
\pgfsetlinewidth{0.803000pt}%
\definecolor{currentstroke}{rgb}{0.000000,0.000000,0.000000}%
\pgfsetstrokecolor{currentstroke}%
\pgfsetdash{}{0pt}%
\pgfsys@defobject{currentmarker}{\pgfqpoint{0.000000in}{-0.048611in}}{\pgfqpoint{0.000000in}{0.000000in}}{%
\pgfpathmoveto{\pgfqpoint{0.000000in}{0.000000in}}%
\pgfpathlineto{\pgfqpoint{0.000000in}{-0.048611in}}%
\pgfusepath{stroke,fill}%
}%
\begin{pgfscope}%
\pgfsys@transformshift{4.628218in}{0.528000in}%
\pgfsys@useobject{currentmarker}{}%
\end{pgfscope}%
\end{pgfscope}%
\begin{pgfscope}%
\definecolor{textcolor}{rgb}{0.000000,0.000000,0.000000}%
\pgfsetstrokecolor{textcolor}%
\pgfsetfillcolor{textcolor}%
\pgftext[x=4.628218in,y=0.430778in,,top]{\color{textcolor}\ttfamily\fontsize{10.000000}{12.000000}\selectfont 800}%
\end{pgfscope}%
\begin{pgfscope}%
\pgfsetbuttcap%
\pgfsetroundjoin%
\definecolor{currentfill}{rgb}{0.000000,0.000000,0.000000}%
\pgfsetfillcolor{currentfill}%
\pgfsetlinewidth{0.803000pt}%
\definecolor{currentstroke}{rgb}{0.000000,0.000000,0.000000}%
\pgfsetstrokecolor{currentstroke}%
\pgfsetdash{}{0pt}%
\pgfsys@defobject{currentmarker}{\pgfqpoint{0.000000in}{-0.048611in}}{\pgfqpoint{0.000000in}{0.000000in}}{%
\pgfpathmoveto{\pgfqpoint{0.000000in}{0.000000in}}%
\pgfpathlineto{\pgfqpoint{0.000000in}{-0.048611in}}%
\pgfusepath{stroke,fill}%
}%
\begin{pgfscope}%
\pgfsys@transformshift{5.530036in}{0.528000in}%
\pgfsys@useobject{currentmarker}{}%
\end{pgfscope}%
\end{pgfscope}%
\begin{pgfscope}%
\definecolor{textcolor}{rgb}{0.000000,0.000000,0.000000}%
\pgfsetstrokecolor{textcolor}%
\pgfsetfillcolor{textcolor}%
\pgftext[x=5.530036in,y=0.430778in,,top]{\color{textcolor}\ttfamily\fontsize{10.000000}{12.000000}\selectfont 1000}%
\end{pgfscope}%
\begin{pgfscope}%
\definecolor{textcolor}{rgb}{0.000000,0.000000,0.000000}%
\pgfsetstrokecolor{textcolor}%
\pgfsetfillcolor{textcolor}%
\pgftext[x=3.280000in,y=0.240063in,,top]{\color{textcolor}\ttfamily\fontsize{10.000000}{12.000000}\selectfont Size of Array}%
\end{pgfscope}%
\begin{pgfscope}%
\pgfsetbuttcap%
\pgfsetroundjoin%
\definecolor{currentfill}{rgb}{0.000000,0.000000,0.000000}%
\pgfsetfillcolor{currentfill}%
\pgfsetlinewidth{0.803000pt}%
\definecolor{currentstroke}{rgb}{0.000000,0.000000,0.000000}%
\pgfsetstrokecolor{currentstroke}%
\pgfsetdash{}{0pt}%
\pgfsys@defobject{currentmarker}{\pgfqpoint{-0.048611in}{0.000000in}}{\pgfqpoint{-0.000000in}{0.000000in}}{%
\pgfpathmoveto{\pgfqpoint{-0.000000in}{0.000000in}}%
\pgfpathlineto{\pgfqpoint{-0.048611in}{0.000000in}}%
\pgfusepath{stroke,fill}%
}%
\begin{pgfscope}%
\pgfsys@transformshift{0.800000in}{0.783360in}%
\pgfsys@useobject{currentmarker}{}%
\end{pgfscope}%
\end{pgfscope}%
\begin{pgfscope}%
\definecolor{textcolor}{rgb}{0.000000,0.000000,0.000000}%
\pgfsetstrokecolor{textcolor}%
\pgfsetfillcolor{textcolor}%
\pgftext[x=0.451923in, y=0.730225in, left, base]{\color{textcolor}\ttfamily\fontsize{10.000000}{12.000000}\selectfont 250}%
\end{pgfscope}%
\begin{pgfscope}%
\pgfsetbuttcap%
\pgfsetroundjoin%
\definecolor{currentfill}{rgb}{0.000000,0.000000,0.000000}%
\pgfsetfillcolor{currentfill}%
\pgfsetlinewidth{0.803000pt}%
\definecolor{currentstroke}{rgb}{0.000000,0.000000,0.000000}%
\pgfsetstrokecolor{currentstroke}%
\pgfsetdash{}{0pt}%
\pgfsys@defobject{currentmarker}{\pgfqpoint{-0.048611in}{0.000000in}}{\pgfqpoint{-0.000000in}{0.000000in}}{%
\pgfpathmoveto{\pgfqpoint{-0.000000in}{0.000000in}}%
\pgfpathlineto{\pgfqpoint{-0.048611in}{0.000000in}}%
\pgfusepath{stroke,fill}%
}%
\begin{pgfscope}%
\pgfsys@transformshift{0.800000in}{1.203360in}%
\pgfsys@useobject{currentmarker}{}%
\end{pgfscope}%
\end{pgfscope}%
\begin{pgfscope}%
\definecolor{textcolor}{rgb}{0.000000,0.000000,0.000000}%
\pgfsetstrokecolor{textcolor}%
\pgfsetfillcolor{textcolor}%
\pgftext[x=0.451923in, y=1.150225in, left, base]{\color{textcolor}\ttfamily\fontsize{10.000000}{12.000000}\selectfont 500}%
\end{pgfscope}%
\begin{pgfscope}%
\pgfsetbuttcap%
\pgfsetroundjoin%
\definecolor{currentfill}{rgb}{0.000000,0.000000,0.000000}%
\pgfsetfillcolor{currentfill}%
\pgfsetlinewidth{0.803000pt}%
\definecolor{currentstroke}{rgb}{0.000000,0.000000,0.000000}%
\pgfsetstrokecolor{currentstroke}%
\pgfsetdash{}{0pt}%
\pgfsys@defobject{currentmarker}{\pgfqpoint{-0.048611in}{0.000000in}}{\pgfqpoint{-0.000000in}{0.000000in}}{%
\pgfpathmoveto{\pgfqpoint{-0.000000in}{0.000000in}}%
\pgfpathlineto{\pgfqpoint{-0.048611in}{0.000000in}}%
\pgfusepath{stroke,fill}%
}%
\begin{pgfscope}%
\pgfsys@transformshift{0.800000in}{1.623360in}%
\pgfsys@useobject{currentmarker}{}%
\end{pgfscope}%
\end{pgfscope}%
\begin{pgfscope}%
\definecolor{textcolor}{rgb}{0.000000,0.000000,0.000000}%
\pgfsetstrokecolor{textcolor}%
\pgfsetfillcolor{textcolor}%
\pgftext[x=0.451923in, y=1.570225in, left, base]{\color{textcolor}\ttfamily\fontsize{10.000000}{12.000000}\selectfont 750}%
\end{pgfscope}%
\begin{pgfscope}%
\pgfsetbuttcap%
\pgfsetroundjoin%
\definecolor{currentfill}{rgb}{0.000000,0.000000,0.000000}%
\pgfsetfillcolor{currentfill}%
\pgfsetlinewidth{0.803000pt}%
\definecolor{currentstroke}{rgb}{0.000000,0.000000,0.000000}%
\pgfsetstrokecolor{currentstroke}%
\pgfsetdash{}{0pt}%
\pgfsys@defobject{currentmarker}{\pgfqpoint{-0.048611in}{0.000000in}}{\pgfqpoint{-0.000000in}{0.000000in}}{%
\pgfpathmoveto{\pgfqpoint{-0.000000in}{0.000000in}}%
\pgfpathlineto{\pgfqpoint{-0.048611in}{0.000000in}}%
\pgfusepath{stroke,fill}%
}%
\begin{pgfscope}%
\pgfsys@transformshift{0.800000in}{2.043360in}%
\pgfsys@useobject{currentmarker}{}%
\end{pgfscope}%
\end{pgfscope}%
\begin{pgfscope}%
\definecolor{textcolor}{rgb}{0.000000,0.000000,0.000000}%
\pgfsetstrokecolor{textcolor}%
\pgfsetfillcolor{textcolor}%
\pgftext[x=0.368305in, y=1.990225in, left, base]{\color{textcolor}\ttfamily\fontsize{10.000000}{12.000000}\selectfont 1000}%
\end{pgfscope}%
\begin{pgfscope}%
\pgfsetbuttcap%
\pgfsetroundjoin%
\definecolor{currentfill}{rgb}{0.000000,0.000000,0.000000}%
\pgfsetfillcolor{currentfill}%
\pgfsetlinewidth{0.803000pt}%
\definecolor{currentstroke}{rgb}{0.000000,0.000000,0.000000}%
\pgfsetstrokecolor{currentstroke}%
\pgfsetdash{}{0pt}%
\pgfsys@defobject{currentmarker}{\pgfqpoint{-0.048611in}{0.000000in}}{\pgfqpoint{-0.000000in}{0.000000in}}{%
\pgfpathmoveto{\pgfqpoint{-0.000000in}{0.000000in}}%
\pgfpathlineto{\pgfqpoint{-0.048611in}{0.000000in}}%
\pgfusepath{stroke,fill}%
}%
\begin{pgfscope}%
\pgfsys@transformshift{0.800000in}{2.463360in}%
\pgfsys@useobject{currentmarker}{}%
\end{pgfscope}%
\end{pgfscope}%
\begin{pgfscope}%
\definecolor{textcolor}{rgb}{0.000000,0.000000,0.000000}%
\pgfsetstrokecolor{textcolor}%
\pgfsetfillcolor{textcolor}%
\pgftext[x=0.368305in, y=2.410225in, left, base]{\color{textcolor}\ttfamily\fontsize{10.000000}{12.000000}\selectfont 1250}%
\end{pgfscope}%
\begin{pgfscope}%
\pgfsetbuttcap%
\pgfsetroundjoin%
\definecolor{currentfill}{rgb}{0.000000,0.000000,0.000000}%
\pgfsetfillcolor{currentfill}%
\pgfsetlinewidth{0.803000pt}%
\definecolor{currentstroke}{rgb}{0.000000,0.000000,0.000000}%
\pgfsetstrokecolor{currentstroke}%
\pgfsetdash{}{0pt}%
\pgfsys@defobject{currentmarker}{\pgfqpoint{-0.048611in}{0.000000in}}{\pgfqpoint{-0.000000in}{0.000000in}}{%
\pgfpathmoveto{\pgfqpoint{-0.000000in}{0.000000in}}%
\pgfpathlineto{\pgfqpoint{-0.048611in}{0.000000in}}%
\pgfusepath{stroke,fill}%
}%
\begin{pgfscope}%
\pgfsys@transformshift{0.800000in}{2.883360in}%
\pgfsys@useobject{currentmarker}{}%
\end{pgfscope}%
\end{pgfscope}%
\begin{pgfscope}%
\definecolor{textcolor}{rgb}{0.000000,0.000000,0.000000}%
\pgfsetstrokecolor{textcolor}%
\pgfsetfillcolor{textcolor}%
\pgftext[x=0.368305in, y=2.830225in, left, base]{\color{textcolor}\ttfamily\fontsize{10.000000}{12.000000}\selectfont 1500}%
\end{pgfscope}%
\begin{pgfscope}%
\pgfsetbuttcap%
\pgfsetroundjoin%
\definecolor{currentfill}{rgb}{0.000000,0.000000,0.000000}%
\pgfsetfillcolor{currentfill}%
\pgfsetlinewidth{0.803000pt}%
\definecolor{currentstroke}{rgb}{0.000000,0.000000,0.000000}%
\pgfsetstrokecolor{currentstroke}%
\pgfsetdash{}{0pt}%
\pgfsys@defobject{currentmarker}{\pgfqpoint{-0.048611in}{0.000000in}}{\pgfqpoint{-0.000000in}{0.000000in}}{%
\pgfpathmoveto{\pgfqpoint{-0.000000in}{0.000000in}}%
\pgfpathlineto{\pgfqpoint{-0.048611in}{0.000000in}}%
\pgfusepath{stroke,fill}%
}%
\begin{pgfscope}%
\pgfsys@transformshift{0.800000in}{3.303360in}%
\pgfsys@useobject{currentmarker}{}%
\end{pgfscope}%
\end{pgfscope}%
\begin{pgfscope}%
\definecolor{textcolor}{rgb}{0.000000,0.000000,0.000000}%
\pgfsetstrokecolor{textcolor}%
\pgfsetfillcolor{textcolor}%
\pgftext[x=0.368305in, y=3.250225in, left, base]{\color{textcolor}\ttfamily\fontsize{10.000000}{12.000000}\selectfont 1750}%
\end{pgfscope}%
\begin{pgfscope}%
\pgfsetbuttcap%
\pgfsetroundjoin%
\definecolor{currentfill}{rgb}{0.000000,0.000000,0.000000}%
\pgfsetfillcolor{currentfill}%
\pgfsetlinewidth{0.803000pt}%
\definecolor{currentstroke}{rgb}{0.000000,0.000000,0.000000}%
\pgfsetstrokecolor{currentstroke}%
\pgfsetdash{}{0pt}%
\pgfsys@defobject{currentmarker}{\pgfqpoint{-0.048611in}{0.000000in}}{\pgfqpoint{-0.000000in}{0.000000in}}{%
\pgfpathmoveto{\pgfqpoint{-0.000000in}{0.000000in}}%
\pgfpathlineto{\pgfqpoint{-0.048611in}{0.000000in}}%
\pgfusepath{stroke,fill}%
}%
\begin{pgfscope}%
\pgfsys@transformshift{0.800000in}{3.723360in}%
\pgfsys@useobject{currentmarker}{}%
\end{pgfscope}%
\end{pgfscope}%
\begin{pgfscope}%
\definecolor{textcolor}{rgb}{0.000000,0.000000,0.000000}%
\pgfsetstrokecolor{textcolor}%
\pgfsetfillcolor{textcolor}%
\pgftext[x=0.368305in, y=3.670225in, left, base]{\color{textcolor}\ttfamily\fontsize{10.000000}{12.000000}\selectfont 2000}%
\end{pgfscope}%
\begin{pgfscope}%
\pgfsetbuttcap%
\pgfsetroundjoin%
\definecolor{currentfill}{rgb}{0.000000,0.000000,0.000000}%
\pgfsetfillcolor{currentfill}%
\pgfsetlinewidth{0.803000pt}%
\definecolor{currentstroke}{rgb}{0.000000,0.000000,0.000000}%
\pgfsetstrokecolor{currentstroke}%
\pgfsetdash{}{0pt}%
\pgfsys@defobject{currentmarker}{\pgfqpoint{-0.048611in}{0.000000in}}{\pgfqpoint{-0.000000in}{0.000000in}}{%
\pgfpathmoveto{\pgfqpoint{-0.000000in}{0.000000in}}%
\pgfpathlineto{\pgfqpoint{-0.048611in}{0.000000in}}%
\pgfusepath{stroke,fill}%
}%
\begin{pgfscope}%
\pgfsys@transformshift{0.800000in}{4.143360in}%
\pgfsys@useobject{currentmarker}{}%
\end{pgfscope}%
\end{pgfscope}%
\begin{pgfscope}%
\definecolor{textcolor}{rgb}{0.000000,0.000000,0.000000}%
\pgfsetstrokecolor{textcolor}%
\pgfsetfillcolor{textcolor}%
\pgftext[x=0.368305in, y=4.090225in, left, base]{\color{textcolor}\ttfamily\fontsize{10.000000}{12.000000}\selectfont 2250}%
\end{pgfscope}%
\begin{pgfscope}%
\definecolor{textcolor}{rgb}{0.000000,0.000000,0.000000}%
\pgfsetstrokecolor{textcolor}%
\pgfsetfillcolor{textcolor}%
\pgftext[x=0.312750in,y=2.376000in,,bottom,rotate=90.000000]{\color{textcolor}\ttfamily\fontsize{10.000000}{12.000000}\selectfont Iterations}%
\end{pgfscope}%
\begin{pgfscope}%
\pgfpathrectangle{\pgfqpoint{0.800000in}{0.528000in}}{\pgfqpoint{4.960000in}{3.696000in}}%
\pgfusepath{clip}%
\pgfsetrectcap%
\pgfsetroundjoin%
\pgfsetlinewidth{1.505625pt}%
\definecolor{currentstroke}{rgb}{0.000000,1.000000,0.498039}%
\pgfsetstrokecolor{currentstroke}%
\pgfsetdash{}{0pt}%
\pgfpathmoveto{\pgfqpoint{1.025455in}{0.696000in}}%
\pgfpathlineto{\pgfqpoint{5.534545in}{4.056000in}}%
\pgfpathlineto{\pgfqpoint{5.534545in}{4.056000in}}%
\pgfusepath{stroke}%
\end{pgfscope}%
\begin{pgfscope}%
\pgfsetrectcap%
\pgfsetmiterjoin%
\pgfsetlinewidth{0.803000pt}%
\definecolor{currentstroke}{rgb}{0.000000,0.000000,0.000000}%
\pgfsetstrokecolor{currentstroke}%
\pgfsetdash{}{0pt}%
\pgfpathmoveto{\pgfqpoint{0.800000in}{0.528000in}}%
\pgfpathlineto{\pgfqpoint{0.800000in}{4.224000in}}%
\pgfusepath{stroke}%
\end{pgfscope}%
\begin{pgfscope}%
\pgfsetrectcap%
\pgfsetmiterjoin%
\pgfsetlinewidth{0.803000pt}%
\definecolor{currentstroke}{rgb}{0.000000,0.000000,0.000000}%
\pgfsetstrokecolor{currentstroke}%
\pgfsetdash{}{0pt}%
\pgfpathmoveto{\pgfqpoint{5.760000in}{0.528000in}}%
\pgfpathlineto{\pgfqpoint{5.760000in}{4.224000in}}%
\pgfusepath{stroke}%
\end{pgfscope}%
\begin{pgfscope}%
\pgfsetrectcap%
\pgfsetmiterjoin%
\pgfsetlinewidth{0.803000pt}%
\definecolor{currentstroke}{rgb}{0.000000,0.000000,0.000000}%
\pgfsetstrokecolor{currentstroke}%
\pgfsetdash{}{0pt}%
\pgfpathmoveto{\pgfqpoint{0.800000in}{0.528000in}}%
\pgfpathlineto{\pgfqpoint{5.760000in}{0.528000in}}%
\pgfusepath{stroke}%
\end{pgfscope}%
\begin{pgfscope}%
\pgfsetrectcap%
\pgfsetmiterjoin%
\pgfsetlinewidth{0.803000pt}%
\definecolor{currentstroke}{rgb}{0.000000,0.000000,0.000000}%
\pgfsetstrokecolor{currentstroke}%
\pgfsetdash{}{0pt}%
\pgfpathmoveto{\pgfqpoint{0.800000in}{4.224000in}}%
\pgfpathlineto{\pgfqpoint{5.760000in}{4.224000in}}%
\pgfusepath{stroke}%
\end{pgfscope}%
\begin{pgfscope}%
\definecolor{textcolor}{rgb}{0.000000,0.000000,0.000000}%
\pgfsetstrokecolor{textcolor}%
\pgfsetfillcolor{textcolor}%
\pgftext[x=3.280000in,y=4.307333in,,base]{\color{textcolor}\ttfamily\fontsize{12.000000}{14.400000}\selectfont Merge Sort  Iterations vs Input size}%
\end{pgfscope}%
\begin{pgfscope}%
\pgfsetbuttcap%
\pgfsetmiterjoin%
\definecolor{currentfill}{rgb}{1.000000,1.000000,1.000000}%
\pgfsetfillcolor{currentfill}%
\pgfsetfillopacity{0.800000}%
\pgfsetlinewidth{1.003750pt}%
\definecolor{currentstroke}{rgb}{0.800000,0.800000,0.800000}%
\pgfsetstrokecolor{currentstroke}%
\pgfsetstrokeopacity{0.800000}%
\pgfsetdash{}{0pt}%
\pgfpathmoveto{\pgfqpoint{0.897222in}{3.907336in}}%
\pgfpathlineto{\pgfqpoint{1.759758in}{3.907336in}}%
\pgfpathquadraticcurveto{\pgfqpoint{1.787535in}{3.907336in}}{\pgfqpoint{1.787535in}{3.935114in}}%
\pgfpathlineto{\pgfqpoint{1.787535in}{4.126778in}}%
\pgfpathquadraticcurveto{\pgfqpoint{1.787535in}{4.154556in}}{\pgfqpoint{1.759758in}{4.154556in}}%
\pgfpathlineto{\pgfqpoint{0.897222in}{4.154556in}}%
\pgfpathquadraticcurveto{\pgfqpoint{0.869444in}{4.154556in}}{\pgfqpoint{0.869444in}{4.126778in}}%
\pgfpathlineto{\pgfqpoint{0.869444in}{3.935114in}}%
\pgfpathquadraticcurveto{\pgfqpoint{0.869444in}{3.907336in}}{\pgfqpoint{0.897222in}{3.907336in}}%
\pgfpathlineto{\pgfqpoint{0.897222in}{3.907336in}}%
\pgfpathclose%
\pgfusepath{stroke,fill}%
\end{pgfscope}%
\begin{pgfscope}%
\pgfsetrectcap%
\pgfsetroundjoin%
\pgfsetlinewidth{1.505625pt}%
\definecolor{currentstroke}{rgb}{0.000000,1.000000,0.498039}%
\pgfsetstrokecolor{currentstroke}%
\pgfsetdash{}{0pt}%
\pgfpathmoveto{\pgfqpoint{0.925000in}{4.041342in}}%
\pgfpathlineto{\pgfqpoint{1.063889in}{4.041342in}}%
\pgfpathlineto{\pgfqpoint{1.202778in}{4.041342in}}%
\pgfusepath{stroke}%
\end{pgfscope}%
\begin{pgfscope}%
\definecolor{textcolor}{rgb}{0.000000,0.000000,0.000000}%
\pgfsetstrokecolor{textcolor}%
\pgfsetfillcolor{textcolor}%
\pgftext[x=1.313889in,y=3.992731in,left,base]{\color{textcolor}\ttfamily\fontsize{10.000000}{12.000000}\selectfont Merge}%
\end{pgfscope}%
\end{pgfpicture}%
\makeatother%
\endgroup%

\subsubsection*{Insights}
The time complexity is $O(n\log n)$ and its space complexity is $O(n)$.
\subsection{Quick Sort}
\subsubsection*{Principle}
In quick sort the array is divided into two arrays, one array
holds values that are smaller than the pivot and the other holds
values that are larger. The pivot can be either selected randomly
or it can be the rightmost or leftmost element of the array. The
resulting two arrays are again divided in the similar manner.
This continues until each subarray has only one element. At this
point the elements are already sorted, so we can just combine
them to get a sorted array. This method of sorting is fast and
very efficient but the performance largely depends on the
selection of the pivot
\subsubsection*{Code}
\begin{minted}{python}
def qsort(a, l, r):
    global t_comp
    global t_swap
    global t_itr
    t_comp += 1
    if l < r:
        p = part(a, l, r)
        qsort(a, l, p - 1)
        t_itr += 1
        qsort(a, p + 1, r)
        t_itr += 1


def part(a, l, r):
    global t_comp
    global t_swap
    global t_itr
    t_swap += 1
    # swap middle element with the left most
    (a[l], a[(l + r) // 2]) = (a[(l + r) // 2], a[l])
    # pivot_index
    p_i = l
    # pivot value
    p = a[p_i]
    while l < r:
        t_comp += 1
        while l < len(a) and a[l] <= p:
            t_comp += 1
            l = l + 1
        while a[r] > p:
            t_comp += 1
            r = r - 1
        # Check for overlap
        if l < r:
            t_comp += 1
            t_swap += 1
            # swap misplaced elements
            (a[l], a[r]) = (a[r], a[l])
    t_swap += 1
    # put pivot in the correct place
    (a[p_i], a[r]) = (a[r], a[p_i])
    return r
\end{minted}
\subsubsection*{Graphs}
%% Creator: Matplotlib, PGF backend
%%
%% To include the figure in your LaTeX document, write
%%   \input{<filename>.pgf}
%%
%% Make sure the required packages are loaded in your preamble
%%   \usepackage{pgf}
%%
%% Also ensure that all the required font packages are loaded; for instance,
%% the lmodern package is sometimes necessary when using math font.
%%   \usepackage{lmodern}
%%
%% Figures using additional raster images can only be included by \input if
%% they are in the same directory as the main LaTeX file. For loading figures
%% from other directories you can use the `import` package
%%   \usepackage{import}
%%
%% and then include the figures with
%%   \import{<path to file>}{<filename>.pgf}
%%
%% Matplotlib used the following preamble
%%   \usepackage{fontspec}
%%   \setmainfont{DejaVuSerif.ttf}[Path=\detokenize{/home/dbk/.local/lib/python3.10/site-packages/matplotlib/mpl-data/fonts/ttf/}]
%%   \setsansfont{DejaVuSans.ttf}[Path=\detokenize{/home/dbk/.local/lib/python3.10/site-packages/matplotlib/mpl-data/fonts/ttf/}]
%%   \setmonofont{DejaVuSansMono.ttf}[Path=\detokenize{/home/dbk/.local/lib/python3.10/site-packages/matplotlib/mpl-data/fonts/ttf/}]
%%
\begingroup%
\makeatletter%
\begin{pgfpicture}%
\pgfpathrectangle{\pgfpointorigin}{\pgfqpoint{6.400000in}{4.800000in}}%
\pgfusepath{use as bounding box, clip}%
\begin{pgfscope}%
\pgfsetbuttcap%
\pgfsetmiterjoin%
\definecolor{currentfill}{rgb}{1.000000,1.000000,1.000000}%
\pgfsetfillcolor{currentfill}%
\pgfsetlinewidth{0.000000pt}%
\definecolor{currentstroke}{rgb}{1.000000,1.000000,1.000000}%
\pgfsetstrokecolor{currentstroke}%
\pgfsetdash{}{0pt}%
\pgfpathmoveto{\pgfqpoint{0.000000in}{0.000000in}}%
\pgfpathlineto{\pgfqpoint{6.400000in}{0.000000in}}%
\pgfpathlineto{\pgfqpoint{6.400000in}{4.800000in}}%
\pgfpathlineto{\pgfqpoint{0.000000in}{4.800000in}}%
\pgfpathlineto{\pgfqpoint{0.000000in}{0.000000in}}%
\pgfpathclose%
\pgfusepath{fill}%
\end{pgfscope}%
\begin{pgfscope}%
\pgfsetbuttcap%
\pgfsetmiterjoin%
\definecolor{currentfill}{rgb}{1.000000,1.000000,1.000000}%
\pgfsetfillcolor{currentfill}%
\pgfsetlinewidth{0.000000pt}%
\definecolor{currentstroke}{rgb}{0.000000,0.000000,0.000000}%
\pgfsetstrokecolor{currentstroke}%
\pgfsetstrokeopacity{0.000000}%
\pgfsetdash{}{0pt}%
\pgfpathmoveto{\pgfqpoint{0.800000in}{0.528000in}}%
\pgfpathlineto{\pgfqpoint{5.760000in}{0.528000in}}%
\pgfpathlineto{\pgfqpoint{5.760000in}{4.224000in}}%
\pgfpathlineto{\pgfqpoint{0.800000in}{4.224000in}}%
\pgfpathlineto{\pgfqpoint{0.800000in}{0.528000in}}%
\pgfpathclose%
\pgfusepath{fill}%
\end{pgfscope}%
\begin{pgfscope}%
\pgfsetbuttcap%
\pgfsetroundjoin%
\definecolor{currentfill}{rgb}{0.000000,0.000000,0.000000}%
\pgfsetfillcolor{currentfill}%
\pgfsetlinewidth{0.803000pt}%
\definecolor{currentstroke}{rgb}{0.000000,0.000000,0.000000}%
\pgfsetstrokecolor{currentstroke}%
\pgfsetdash{}{0pt}%
\pgfsys@defobject{currentmarker}{\pgfqpoint{0.000000in}{-0.048611in}}{\pgfqpoint{0.000000in}{0.000000in}}{%
\pgfpathmoveto{\pgfqpoint{0.000000in}{0.000000in}}%
\pgfpathlineto{\pgfqpoint{0.000000in}{-0.048611in}}%
\pgfusepath{stroke,fill}%
}%
\begin{pgfscope}%
\pgfsys@transformshift{1.020945in}{0.528000in}%
\pgfsys@useobject{currentmarker}{}%
\end{pgfscope}%
\end{pgfscope}%
\begin{pgfscope}%
\definecolor{textcolor}{rgb}{0.000000,0.000000,0.000000}%
\pgfsetstrokecolor{textcolor}%
\pgfsetfillcolor{textcolor}%
\pgftext[x=1.020945in,y=0.430778in,,top]{\color{textcolor}\ttfamily\fontsize{10.000000}{12.000000}\selectfont 0}%
\end{pgfscope}%
\begin{pgfscope}%
\pgfsetbuttcap%
\pgfsetroundjoin%
\definecolor{currentfill}{rgb}{0.000000,0.000000,0.000000}%
\pgfsetfillcolor{currentfill}%
\pgfsetlinewidth{0.803000pt}%
\definecolor{currentstroke}{rgb}{0.000000,0.000000,0.000000}%
\pgfsetstrokecolor{currentstroke}%
\pgfsetdash{}{0pt}%
\pgfsys@defobject{currentmarker}{\pgfqpoint{0.000000in}{-0.048611in}}{\pgfqpoint{0.000000in}{0.000000in}}{%
\pgfpathmoveto{\pgfqpoint{0.000000in}{0.000000in}}%
\pgfpathlineto{\pgfqpoint{0.000000in}{-0.048611in}}%
\pgfusepath{stroke,fill}%
}%
\begin{pgfscope}%
\pgfsys@transformshift{1.922764in}{0.528000in}%
\pgfsys@useobject{currentmarker}{}%
\end{pgfscope}%
\end{pgfscope}%
\begin{pgfscope}%
\definecolor{textcolor}{rgb}{0.000000,0.000000,0.000000}%
\pgfsetstrokecolor{textcolor}%
\pgfsetfillcolor{textcolor}%
\pgftext[x=1.922764in,y=0.430778in,,top]{\color{textcolor}\ttfamily\fontsize{10.000000}{12.000000}\selectfont 200}%
\end{pgfscope}%
\begin{pgfscope}%
\pgfsetbuttcap%
\pgfsetroundjoin%
\definecolor{currentfill}{rgb}{0.000000,0.000000,0.000000}%
\pgfsetfillcolor{currentfill}%
\pgfsetlinewidth{0.803000pt}%
\definecolor{currentstroke}{rgb}{0.000000,0.000000,0.000000}%
\pgfsetstrokecolor{currentstroke}%
\pgfsetdash{}{0pt}%
\pgfsys@defobject{currentmarker}{\pgfqpoint{0.000000in}{-0.048611in}}{\pgfqpoint{0.000000in}{0.000000in}}{%
\pgfpathmoveto{\pgfqpoint{0.000000in}{0.000000in}}%
\pgfpathlineto{\pgfqpoint{0.000000in}{-0.048611in}}%
\pgfusepath{stroke,fill}%
}%
\begin{pgfscope}%
\pgfsys@transformshift{2.824582in}{0.528000in}%
\pgfsys@useobject{currentmarker}{}%
\end{pgfscope}%
\end{pgfscope}%
\begin{pgfscope}%
\definecolor{textcolor}{rgb}{0.000000,0.000000,0.000000}%
\pgfsetstrokecolor{textcolor}%
\pgfsetfillcolor{textcolor}%
\pgftext[x=2.824582in,y=0.430778in,,top]{\color{textcolor}\ttfamily\fontsize{10.000000}{12.000000}\selectfont 400}%
\end{pgfscope}%
\begin{pgfscope}%
\pgfsetbuttcap%
\pgfsetroundjoin%
\definecolor{currentfill}{rgb}{0.000000,0.000000,0.000000}%
\pgfsetfillcolor{currentfill}%
\pgfsetlinewidth{0.803000pt}%
\definecolor{currentstroke}{rgb}{0.000000,0.000000,0.000000}%
\pgfsetstrokecolor{currentstroke}%
\pgfsetdash{}{0pt}%
\pgfsys@defobject{currentmarker}{\pgfqpoint{0.000000in}{-0.048611in}}{\pgfqpoint{0.000000in}{0.000000in}}{%
\pgfpathmoveto{\pgfqpoint{0.000000in}{0.000000in}}%
\pgfpathlineto{\pgfqpoint{0.000000in}{-0.048611in}}%
\pgfusepath{stroke,fill}%
}%
\begin{pgfscope}%
\pgfsys@transformshift{3.726400in}{0.528000in}%
\pgfsys@useobject{currentmarker}{}%
\end{pgfscope}%
\end{pgfscope}%
\begin{pgfscope}%
\definecolor{textcolor}{rgb}{0.000000,0.000000,0.000000}%
\pgfsetstrokecolor{textcolor}%
\pgfsetfillcolor{textcolor}%
\pgftext[x=3.726400in,y=0.430778in,,top]{\color{textcolor}\ttfamily\fontsize{10.000000}{12.000000}\selectfont 600}%
\end{pgfscope}%
\begin{pgfscope}%
\pgfsetbuttcap%
\pgfsetroundjoin%
\definecolor{currentfill}{rgb}{0.000000,0.000000,0.000000}%
\pgfsetfillcolor{currentfill}%
\pgfsetlinewidth{0.803000pt}%
\definecolor{currentstroke}{rgb}{0.000000,0.000000,0.000000}%
\pgfsetstrokecolor{currentstroke}%
\pgfsetdash{}{0pt}%
\pgfsys@defobject{currentmarker}{\pgfqpoint{0.000000in}{-0.048611in}}{\pgfqpoint{0.000000in}{0.000000in}}{%
\pgfpathmoveto{\pgfqpoint{0.000000in}{0.000000in}}%
\pgfpathlineto{\pgfqpoint{0.000000in}{-0.048611in}}%
\pgfusepath{stroke,fill}%
}%
\begin{pgfscope}%
\pgfsys@transformshift{4.628218in}{0.528000in}%
\pgfsys@useobject{currentmarker}{}%
\end{pgfscope}%
\end{pgfscope}%
\begin{pgfscope}%
\definecolor{textcolor}{rgb}{0.000000,0.000000,0.000000}%
\pgfsetstrokecolor{textcolor}%
\pgfsetfillcolor{textcolor}%
\pgftext[x=4.628218in,y=0.430778in,,top]{\color{textcolor}\ttfamily\fontsize{10.000000}{12.000000}\selectfont 800}%
\end{pgfscope}%
\begin{pgfscope}%
\pgfsetbuttcap%
\pgfsetroundjoin%
\definecolor{currentfill}{rgb}{0.000000,0.000000,0.000000}%
\pgfsetfillcolor{currentfill}%
\pgfsetlinewidth{0.803000pt}%
\definecolor{currentstroke}{rgb}{0.000000,0.000000,0.000000}%
\pgfsetstrokecolor{currentstroke}%
\pgfsetdash{}{0pt}%
\pgfsys@defobject{currentmarker}{\pgfqpoint{0.000000in}{-0.048611in}}{\pgfqpoint{0.000000in}{0.000000in}}{%
\pgfpathmoveto{\pgfqpoint{0.000000in}{0.000000in}}%
\pgfpathlineto{\pgfqpoint{0.000000in}{-0.048611in}}%
\pgfusepath{stroke,fill}%
}%
\begin{pgfscope}%
\pgfsys@transformshift{5.530036in}{0.528000in}%
\pgfsys@useobject{currentmarker}{}%
\end{pgfscope}%
\end{pgfscope}%
\begin{pgfscope}%
\definecolor{textcolor}{rgb}{0.000000,0.000000,0.000000}%
\pgfsetstrokecolor{textcolor}%
\pgfsetfillcolor{textcolor}%
\pgftext[x=5.530036in,y=0.430778in,,top]{\color{textcolor}\ttfamily\fontsize{10.000000}{12.000000}\selectfont 1000}%
\end{pgfscope}%
\begin{pgfscope}%
\definecolor{textcolor}{rgb}{0.000000,0.000000,0.000000}%
\pgfsetstrokecolor{textcolor}%
\pgfsetfillcolor{textcolor}%
\pgftext[x=3.280000in,y=0.240063in,,top]{\color{textcolor}\ttfamily\fontsize{10.000000}{12.000000}\selectfont Size of Array}%
\end{pgfscope}%
\begin{pgfscope}%
\pgfsetbuttcap%
\pgfsetroundjoin%
\definecolor{currentfill}{rgb}{0.000000,0.000000,0.000000}%
\pgfsetfillcolor{currentfill}%
\pgfsetlinewidth{0.803000pt}%
\definecolor{currentstroke}{rgb}{0.000000,0.000000,0.000000}%
\pgfsetstrokecolor{currentstroke}%
\pgfsetdash{}{0pt}%
\pgfsys@defobject{currentmarker}{\pgfqpoint{-0.048611in}{0.000000in}}{\pgfqpoint{-0.000000in}{0.000000in}}{%
\pgfpathmoveto{\pgfqpoint{-0.000000in}{0.000000in}}%
\pgfpathlineto{\pgfqpoint{-0.048611in}{0.000000in}}%
\pgfusepath{stroke,fill}%
}%
\begin{pgfscope}%
\pgfsys@transformshift{0.800000in}{0.647031in}%
\pgfsys@useobject{currentmarker}{}%
\end{pgfscope}%
\end{pgfscope}%
\begin{pgfscope}%
\definecolor{textcolor}{rgb}{0.000000,0.000000,0.000000}%
\pgfsetstrokecolor{textcolor}%
\pgfsetfillcolor{textcolor}%
\pgftext[x=0.619160in, y=0.593896in, left, base]{\color{textcolor}\ttfamily\fontsize{10.000000}{12.000000}\selectfont 0}%
\end{pgfscope}%
\begin{pgfscope}%
\pgfsetbuttcap%
\pgfsetroundjoin%
\definecolor{currentfill}{rgb}{0.000000,0.000000,0.000000}%
\pgfsetfillcolor{currentfill}%
\pgfsetlinewidth{0.803000pt}%
\definecolor{currentstroke}{rgb}{0.000000,0.000000,0.000000}%
\pgfsetstrokecolor{currentstroke}%
\pgfsetdash{}{0pt}%
\pgfsys@defobject{currentmarker}{\pgfqpoint{-0.048611in}{0.000000in}}{\pgfqpoint{-0.000000in}{0.000000in}}{%
\pgfpathmoveto{\pgfqpoint{-0.000000in}{0.000000in}}%
\pgfpathlineto{\pgfqpoint{-0.048611in}{0.000000in}}%
\pgfusepath{stroke,fill}%
}%
\begin{pgfscope}%
\pgfsys@transformshift{0.800000in}{1.306366in}%
\pgfsys@useobject{currentmarker}{}%
\end{pgfscope}%
\end{pgfscope}%
\begin{pgfscope}%
\definecolor{textcolor}{rgb}{0.000000,0.000000,0.000000}%
\pgfsetstrokecolor{textcolor}%
\pgfsetfillcolor{textcolor}%
\pgftext[x=0.619160in, y=1.253232in, left, base]{\color{textcolor}\ttfamily\fontsize{10.000000}{12.000000}\selectfont 1}%
\end{pgfscope}%
\begin{pgfscope}%
\pgfsetbuttcap%
\pgfsetroundjoin%
\definecolor{currentfill}{rgb}{0.000000,0.000000,0.000000}%
\pgfsetfillcolor{currentfill}%
\pgfsetlinewidth{0.803000pt}%
\definecolor{currentstroke}{rgb}{0.000000,0.000000,0.000000}%
\pgfsetstrokecolor{currentstroke}%
\pgfsetdash{}{0pt}%
\pgfsys@defobject{currentmarker}{\pgfqpoint{-0.048611in}{0.000000in}}{\pgfqpoint{-0.000000in}{0.000000in}}{%
\pgfpathmoveto{\pgfqpoint{-0.000000in}{0.000000in}}%
\pgfpathlineto{\pgfqpoint{-0.048611in}{0.000000in}}%
\pgfusepath{stroke,fill}%
}%
\begin{pgfscope}%
\pgfsys@transformshift{0.800000in}{1.965702in}%
\pgfsys@useobject{currentmarker}{}%
\end{pgfscope}%
\end{pgfscope}%
\begin{pgfscope}%
\definecolor{textcolor}{rgb}{0.000000,0.000000,0.000000}%
\pgfsetstrokecolor{textcolor}%
\pgfsetfillcolor{textcolor}%
\pgftext[x=0.619160in, y=1.912567in, left, base]{\color{textcolor}\ttfamily\fontsize{10.000000}{12.000000}\selectfont 2}%
\end{pgfscope}%
\begin{pgfscope}%
\pgfsetbuttcap%
\pgfsetroundjoin%
\definecolor{currentfill}{rgb}{0.000000,0.000000,0.000000}%
\pgfsetfillcolor{currentfill}%
\pgfsetlinewidth{0.803000pt}%
\definecolor{currentstroke}{rgb}{0.000000,0.000000,0.000000}%
\pgfsetstrokecolor{currentstroke}%
\pgfsetdash{}{0pt}%
\pgfsys@defobject{currentmarker}{\pgfqpoint{-0.048611in}{0.000000in}}{\pgfqpoint{-0.000000in}{0.000000in}}{%
\pgfpathmoveto{\pgfqpoint{-0.000000in}{0.000000in}}%
\pgfpathlineto{\pgfqpoint{-0.048611in}{0.000000in}}%
\pgfusepath{stroke,fill}%
}%
\begin{pgfscope}%
\pgfsys@transformshift{0.800000in}{2.625037in}%
\pgfsys@useobject{currentmarker}{}%
\end{pgfscope}%
\end{pgfscope}%
\begin{pgfscope}%
\definecolor{textcolor}{rgb}{0.000000,0.000000,0.000000}%
\pgfsetstrokecolor{textcolor}%
\pgfsetfillcolor{textcolor}%
\pgftext[x=0.619160in, y=2.571903in, left, base]{\color{textcolor}\ttfamily\fontsize{10.000000}{12.000000}\selectfont 3}%
\end{pgfscope}%
\begin{pgfscope}%
\pgfsetbuttcap%
\pgfsetroundjoin%
\definecolor{currentfill}{rgb}{0.000000,0.000000,0.000000}%
\pgfsetfillcolor{currentfill}%
\pgfsetlinewidth{0.803000pt}%
\definecolor{currentstroke}{rgb}{0.000000,0.000000,0.000000}%
\pgfsetstrokecolor{currentstroke}%
\pgfsetdash{}{0pt}%
\pgfsys@defobject{currentmarker}{\pgfqpoint{-0.048611in}{0.000000in}}{\pgfqpoint{-0.000000in}{0.000000in}}{%
\pgfpathmoveto{\pgfqpoint{-0.000000in}{0.000000in}}%
\pgfpathlineto{\pgfqpoint{-0.048611in}{0.000000in}}%
\pgfusepath{stroke,fill}%
}%
\begin{pgfscope}%
\pgfsys@transformshift{0.800000in}{3.284373in}%
\pgfsys@useobject{currentmarker}{}%
\end{pgfscope}%
\end{pgfscope}%
\begin{pgfscope}%
\definecolor{textcolor}{rgb}{0.000000,0.000000,0.000000}%
\pgfsetstrokecolor{textcolor}%
\pgfsetfillcolor{textcolor}%
\pgftext[x=0.619160in, y=3.231238in, left, base]{\color{textcolor}\ttfamily\fontsize{10.000000}{12.000000}\selectfont 4}%
\end{pgfscope}%
\begin{pgfscope}%
\pgfsetbuttcap%
\pgfsetroundjoin%
\definecolor{currentfill}{rgb}{0.000000,0.000000,0.000000}%
\pgfsetfillcolor{currentfill}%
\pgfsetlinewidth{0.803000pt}%
\definecolor{currentstroke}{rgb}{0.000000,0.000000,0.000000}%
\pgfsetstrokecolor{currentstroke}%
\pgfsetdash{}{0pt}%
\pgfsys@defobject{currentmarker}{\pgfqpoint{-0.048611in}{0.000000in}}{\pgfqpoint{-0.000000in}{0.000000in}}{%
\pgfpathmoveto{\pgfqpoint{-0.000000in}{0.000000in}}%
\pgfpathlineto{\pgfqpoint{-0.048611in}{0.000000in}}%
\pgfusepath{stroke,fill}%
}%
\begin{pgfscope}%
\pgfsys@transformshift{0.800000in}{3.943708in}%
\pgfsys@useobject{currentmarker}{}%
\end{pgfscope}%
\end{pgfscope}%
\begin{pgfscope}%
\definecolor{textcolor}{rgb}{0.000000,0.000000,0.000000}%
\pgfsetstrokecolor{textcolor}%
\pgfsetfillcolor{textcolor}%
\pgftext[x=0.619160in, y=3.890574in, left, base]{\color{textcolor}\ttfamily\fontsize{10.000000}{12.000000}\selectfont 5}%
\end{pgfscope}%
\begin{pgfscope}%
\definecolor{textcolor}{rgb}{0.000000,0.000000,0.000000}%
\pgfsetstrokecolor{textcolor}%
\pgfsetfillcolor{textcolor}%
\pgftext[x=0.563604in,y=2.376000in,,bottom,rotate=90.000000]{\color{textcolor}\ttfamily\fontsize{10.000000}{12.000000}\selectfont Time}%
\end{pgfscope}%
\begin{pgfscope}%
\definecolor{textcolor}{rgb}{0.000000,0.000000,0.000000}%
\pgfsetstrokecolor{textcolor}%
\pgfsetfillcolor{textcolor}%
\pgftext[x=0.800000in,y=4.265667in,left,base]{\color{textcolor}\ttfamily\fontsize{10.000000}{12.000000}\selectfont 1e7}%
\end{pgfscope}%
\begin{pgfscope}%
\pgfpathrectangle{\pgfqpoint{0.800000in}{0.528000in}}{\pgfqpoint{4.960000in}{3.696000in}}%
\pgfusepath{clip}%
\pgfsetrectcap%
\pgfsetroundjoin%
\pgfsetlinewidth{1.505625pt}%
\definecolor{currentstroke}{rgb}{0.000000,1.000000,0.498039}%
\pgfsetstrokecolor{currentstroke}%
\pgfsetdash{}{0pt}%
\pgfpathmoveto{\pgfqpoint{1.025455in}{0.696547in}}%
\pgfpathlineto{\pgfqpoint{1.029964in}{0.696000in}}%
\pgfpathlineto{\pgfqpoint{1.034473in}{0.699490in}}%
\pgfpathlineto{\pgfqpoint{1.038982in}{0.739949in}}%
\pgfpathlineto{\pgfqpoint{1.043491in}{0.768001in}}%
\pgfpathlineto{\pgfqpoint{1.048000in}{0.711272in}}%
\pgfpathlineto{\pgfqpoint{1.052509in}{0.699836in}}%
\pgfpathlineto{\pgfqpoint{1.057018in}{0.704792in}}%
\pgfpathlineto{\pgfqpoint{1.061527in}{0.750197in}}%
\pgfpathlineto{\pgfqpoint{1.066036in}{0.706100in}}%
\pgfpathlineto{\pgfqpoint{1.070545in}{0.703852in}}%
\pgfpathlineto{\pgfqpoint{1.084073in}{0.710414in}}%
\pgfpathlineto{\pgfqpoint{1.088582in}{0.717286in}}%
\pgfpathlineto{\pgfqpoint{1.093091in}{0.711502in}}%
\pgfpathlineto{\pgfqpoint{1.097600in}{0.710913in}}%
\pgfpathlineto{\pgfqpoint{1.102109in}{0.709012in}}%
\pgfpathlineto{\pgfqpoint{1.106618in}{0.714408in}}%
\pgfpathlineto{\pgfqpoint{1.111127in}{0.730175in}}%
\pgfpathlineto{\pgfqpoint{1.115636in}{0.709412in}}%
\pgfpathlineto{\pgfqpoint{1.120145in}{0.712946in}}%
\pgfpathlineto{\pgfqpoint{1.124655in}{0.714450in}}%
\pgfpathlineto{\pgfqpoint{1.129164in}{0.714180in}}%
\pgfpathlineto{\pgfqpoint{1.133673in}{0.763104in}}%
\pgfpathlineto{\pgfqpoint{1.138182in}{0.714622in}}%
\pgfpathlineto{\pgfqpoint{1.147200in}{0.719238in}}%
\pgfpathlineto{\pgfqpoint{1.151709in}{0.722087in}}%
\pgfpathlineto{\pgfqpoint{1.156218in}{0.719582in}}%
\pgfpathlineto{\pgfqpoint{1.160727in}{0.724077in}}%
\pgfpathlineto{\pgfqpoint{1.165236in}{0.719590in}}%
\pgfpathlineto{\pgfqpoint{1.169745in}{0.773390in}}%
\pgfpathlineto{\pgfqpoint{1.174255in}{0.721085in}}%
\pgfpathlineto{\pgfqpoint{1.178764in}{0.721447in}}%
\pgfpathlineto{\pgfqpoint{1.183273in}{0.727667in}}%
\pgfpathlineto{\pgfqpoint{1.187782in}{0.772859in}}%
\pgfpathlineto{\pgfqpoint{1.192291in}{0.762203in}}%
\pgfpathlineto{\pgfqpoint{1.196800in}{0.725373in}}%
\pgfpathlineto{\pgfqpoint{1.201309in}{0.725659in}}%
\pgfpathlineto{\pgfqpoint{1.205818in}{0.742444in}}%
\pgfpathlineto{\pgfqpoint{1.210327in}{0.731026in}}%
\pgfpathlineto{\pgfqpoint{1.214836in}{0.729842in}}%
\pgfpathlineto{\pgfqpoint{1.219345in}{0.725593in}}%
\pgfpathlineto{\pgfqpoint{1.223855in}{0.814670in}}%
\pgfpathlineto{\pgfqpoint{1.228364in}{0.731367in}}%
\pgfpathlineto{\pgfqpoint{1.232873in}{0.733392in}}%
\pgfpathlineto{\pgfqpoint{1.237382in}{0.835902in}}%
\pgfpathlineto{\pgfqpoint{1.241891in}{0.835255in}}%
\pgfpathlineto{\pgfqpoint{1.246400in}{0.738252in}}%
\pgfpathlineto{\pgfqpoint{1.250909in}{0.736782in}}%
\pgfpathlineto{\pgfqpoint{1.255418in}{0.738389in}}%
\pgfpathlineto{\pgfqpoint{1.259927in}{0.736170in}}%
\pgfpathlineto{\pgfqpoint{1.264436in}{0.753186in}}%
\pgfpathlineto{\pgfqpoint{1.268945in}{0.733034in}}%
\pgfpathlineto{\pgfqpoint{1.273455in}{0.749161in}}%
\pgfpathlineto{\pgfqpoint{1.277964in}{0.745760in}}%
\pgfpathlineto{\pgfqpoint{1.282473in}{0.744393in}}%
\pgfpathlineto{\pgfqpoint{1.286982in}{0.752627in}}%
\pgfpathlineto{\pgfqpoint{1.291491in}{0.764257in}}%
\pgfpathlineto{\pgfqpoint{1.296000in}{0.755696in}}%
\pgfpathlineto{\pgfqpoint{1.300509in}{0.750599in}}%
\pgfpathlineto{\pgfqpoint{1.305018in}{0.740249in}}%
\pgfpathlineto{\pgfqpoint{1.309527in}{0.745549in}}%
\pgfpathlineto{\pgfqpoint{1.314036in}{0.746359in}}%
\pgfpathlineto{\pgfqpoint{1.318545in}{0.743562in}}%
\pgfpathlineto{\pgfqpoint{1.323055in}{0.747287in}}%
\pgfpathlineto{\pgfqpoint{1.327564in}{0.742862in}}%
\pgfpathlineto{\pgfqpoint{1.332073in}{0.762660in}}%
\pgfpathlineto{\pgfqpoint{1.336582in}{0.772264in}}%
\pgfpathlineto{\pgfqpoint{1.341091in}{0.858562in}}%
\pgfpathlineto{\pgfqpoint{1.350109in}{0.782813in}}%
\pgfpathlineto{\pgfqpoint{1.359127in}{1.001971in}}%
\pgfpathlineto{\pgfqpoint{1.363636in}{1.044409in}}%
\pgfpathlineto{\pgfqpoint{1.372655in}{0.829485in}}%
\pgfpathlineto{\pgfqpoint{1.377164in}{0.822535in}}%
\pgfpathlineto{\pgfqpoint{1.381673in}{0.867499in}}%
\pgfpathlineto{\pgfqpoint{1.386182in}{0.786400in}}%
\pgfpathlineto{\pgfqpoint{1.390691in}{0.980200in}}%
\pgfpathlineto{\pgfqpoint{1.395200in}{0.851471in}}%
\pgfpathlineto{\pgfqpoint{1.399709in}{0.960073in}}%
\pgfpathlineto{\pgfqpoint{1.404218in}{0.797659in}}%
\pgfpathlineto{\pgfqpoint{1.408727in}{0.829141in}}%
\pgfpathlineto{\pgfqpoint{1.413236in}{0.760090in}}%
\pgfpathlineto{\pgfqpoint{1.417745in}{0.773037in}}%
\pgfpathlineto{\pgfqpoint{1.422255in}{0.762339in}}%
\pgfpathlineto{\pgfqpoint{1.426764in}{0.772078in}}%
\pgfpathlineto{\pgfqpoint{1.431273in}{0.760745in}}%
\pgfpathlineto{\pgfqpoint{1.435782in}{0.766451in}}%
\pgfpathlineto{\pgfqpoint{1.440291in}{0.767365in}}%
\pgfpathlineto{\pgfqpoint{1.444800in}{0.769665in}}%
\pgfpathlineto{\pgfqpoint{1.449309in}{0.773770in}}%
\pgfpathlineto{\pgfqpoint{1.453818in}{0.769494in}}%
\pgfpathlineto{\pgfqpoint{1.458327in}{0.775317in}}%
\pgfpathlineto{\pgfqpoint{1.462836in}{0.773387in}}%
\pgfpathlineto{\pgfqpoint{1.467345in}{0.772762in}}%
\pgfpathlineto{\pgfqpoint{1.471855in}{0.780564in}}%
\pgfpathlineto{\pgfqpoint{1.476364in}{0.776246in}}%
\pgfpathlineto{\pgfqpoint{1.480873in}{0.776786in}}%
\pgfpathlineto{\pgfqpoint{1.489891in}{0.771263in}}%
\pgfpathlineto{\pgfqpoint{1.494400in}{0.771266in}}%
\pgfpathlineto{\pgfqpoint{1.498909in}{0.783136in}}%
\pgfpathlineto{\pgfqpoint{1.507927in}{0.972708in}}%
\pgfpathlineto{\pgfqpoint{1.512436in}{0.810769in}}%
\pgfpathlineto{\pgfqpoint{1.516945in}{0.786160in}}%
\pgfpathlineto{\pgfqpoint{1.521455in}{0.789055in}}%
\pgfpathlineto{\pgfqpoint{1.525964in}{0.782641in}}%
\pgfpathlineto{\pgfqpoint{1.530473in}{0.779954in}}%
\pgfpathlineto{\pgfqpoint{1.539491in}{0.800053in}}%
\pgfpathlineto{\pgfqpoint{1.544000in}{0.799467in}}%
\pgfpathlineto{\pgfqpoint{1.548509in}{0.857779in}}%
\pgfpathlineto{\pgfqpoint{1.553018in}{0.791781in}}%
\pgfpathlineto{\pgfqpoint{1.557527in}{0.786066in}}%
\pgfpathlineto{\pgfqpoint{1.562036in}{0.791318in}}%
\pgfpathlineto{\pgfqpoint{1.566545in}{0.791695in}}%
\pgfpathlineto{\pgfqpoint{1.571055in}{0.797917in}}%
\pgfpathlineto{\pgfqpoint{1.575564in}{0.797122in}}%
\pgfpathlineto{\pgfqpoint{1.580073in}{0.909927in}}%
\pgfpathlineto{\pgfqpoint{1.584582in}{0.836007in}}%
\pgfpathlineto{\pgfqpoint{1.589091in}{0.888073in}}%
\pgfpathlineto{\pgfqpoint{1.593600in}{0.865884in}}%
\pgfpathlineto{\pgfqpoint{1.598109in}{0.879387in}}%
\pgfpathlineto{\pgfqpoint{1.602618in}{0.832677in}}%
\pgfpathlineto{\pgfqpoint{1.607127in}{0.949958in}}%
\pgfpathlineto{\pgfqpoint{1.611636in}{0.967116in}}%
\pgfpathlineto{\pgfqpoint{1.616145in}{1.011029in}}%
\pgfpathlineto{\pgfqpoint{1.620655in}{0.965376in}}%
\pgfpathlineto{\pgfqpoint{1.625164in}{1.058572in}}%
\pgfpathlineto{\pgfqpoint{1.629673in}{0.810211in}}%
\pgfpathlineto{\pgfqpoint{1.634182in}{0.812193in}}%
\pgfpathlineto{\pgfqpoint{1.638691in}{0.812951in}}%
\pgfpathlineto{\pgfqpoint{1.643200in}{0.799362in}}%
\pgfpathlineto{\pgfqpoint{1.647709in}{0.798538in}}%
\pgfpathlineto{\pgfqpoint{1.652218in}{0.804652in}}%
\pgfpathlineto{\pgfqpoint{1.656727in}{0.816543in}}%
\pgfpathlineto{\pgfqpoint{1.661236in}{0.811136in}}%
\pgfpathlineto{\pgfqpoint{1.665745in}{0.953681in}}%
\pgfpathlineto{\pgfqpoint{1.670255in}{0.939773in}}%
\pgfpathlineto{\pgfqpoint{1.674764in}{0.853120in}}%
\pgfpathlineto{\pgfqpoint{1.679273in}{1.012813in}}%
\pgfpathlineto{\pgfqpoint{1.683782in}{0.804055in}}%
\pgfpathlineto{\pgfqpoint{1.688291in}{0.849718in}}%
\pgfpathlineto{\pgfqpoint{1.692800in}{0.825761in}}%
\pgfpathlineto{\pgfqpoint{1.697309in}{0.848176in}}%
\pgfpathlineto{\pgfqpoint{1.701818in}{0.849411in}}%
\pgfpathlineto{\pgfqpoint{1.706327in}{0.852876in}}%
\pgfpathlineto{\pgfqpoint{1.710836in}{0.906244in}}%
\pgfpathlineto{\pgfqpoint{1.715345in}{1.007057in}}%
\pgfpathlineto{\pgfqpoint{1.719855in}{0.818725in}}%
\pgfpathlineto{\pgfqpoint{1.724364in}{0.819957in}}%
\pgfpathlineto{\pgfqpoint{1.728873in}{0.872790in}}%
\pgfpathlineto{\pgfqpoint{1.733382in}{0.838751in}}%
\pgfpathlineto{\pgfqpoint{1.742400in}{0.858972in}}%
\pgfpathlineto{\pgfqpoint{1.746909in}{0.847681in}}%
\pgfpathlineto{\pgfqpoint{1.751418in}{1.027610in}}%
\pgfpathlineto{\pgfqpoint{1.755927in}{0.943741in}}%
\pgfpathlineto{\pgfqpoint{1.760436in}{0.933007in}}%
\pgfpathlineto{\pgfqpoint{1.764945in}{0.863667in}}%
\pgfpathlineto{\pgfqpoint{1.769455in}{0.874727in}}%
\pgfpathlineto{\pgfqpoint{1.773964in}{0.875276in}}%
\pgfpathlineto{\pgfqpoint{1.778473in}{0.880049in}}%
\pgfpathlineto{\pgfqpoint{1.782982in}{1.224530in}}%
\pgfpathlineto{\pgfqpoint{1.787491in}{0.925882in}}%
\pgfpathlineto{\pgfqpoint{1.792000in}{0.907496in}}%
\pgfpathlineto{\pgfqpoint{1.796509in}{1.053032in}}%
\pgfpathlineto{\pgfqpoint{1.801018in}{0.893159in}}%
\pgfpathlineto{\pgfqpoint{1.805527in}{1.043060in}}%
\pgfpathlineto{\pgfqpoint{1.810036in}{0.862761in}}%
\pgfpathlineto{\pgfqpoint{1.814545in}{0.874960in}}%
\pgfpathlineto{\pgfqpoint{1.819055in}{1.001381in}}%
\pgfpathlineto{\pgfqpoint{1.823564in}{0.909647in}}%
\pgfpathlineto{\pgfqpoint{1.828073in}{0.867248in}}%
\pgfpathlineto{\pgfqpoint{1.832582in}{0.983309in}}%
\pgfpathlineto{\pgfqpoint{1.837091in}{1.045042in}}%
\pgfpathlineto{\pgfqpoint{1.841600in}{0.924573in}}%
\pgfpathlineto{\pgfqpoint{1.846109in}{0.950586in}}%
\pgfpathlineto{\pgfqpoint{1.850618in}{1.140241in}}%
\pgfpathlineto{\pgfqpoint{1.855127in}{1.089737in}}%
\pgfpathlineto{\pgfqpoint{1.859636in}{0.871674in}}%
\pgfpathlineto{\pgfqpoint{1.864145in}{0.884225in}}%
\pgfpathlineto{\pgfqpoint{1.868655in}{0.877022in}}%
\pgfpathlineto{\pgfqpoint{1.873164in}{0.873113in}}%
\pgfpathlineto{\pgfqpoint{1.877673in}{0.884714in}}%
\pgfpathlineto{\pgfqpoint{1.882182in}{1.098970in}}%
\pgfpathlineto{\pgfqpoint{1.886691in}{0.887956in}}%
\pgfpathlineto{\pgfqpoint{1.891200in}{0.886923in}}%
\pgfpathlineto{\pgfqpoint{1.895709in}{0.875496in}}%
\pgfpathlineto{\pgfqpoint{1.900218in}{1.150452in}}%
\pgfpathlineto{\pgfqpoint{1.904727in}{0.987817in}}%
\pgfpathlineto{\pgfqpoint{1.909236in}{0.881514in}}%
\pgfpathlineto{\pgfqpoint{1.913745in}{0.889157in}}%
\pgfpathlineto{\pgfqpoint{1.918255in}{0.892531in}}%
\pgfpathlineto{\pgfqpoint{1.922764in}{0.884056in}}%
\pgfpathlineto{\pgfqpoint{1.927273in}{0.890012in}}%
\pgfpathlineto{\pgfqpoint{1.931782in}{0.902629in}}%
\pgfpathlineto{\pgfqpoint{1.936291in}{0.890868in}}%
\pgfpathlineto{\pgfqpoint{1.940800in}{0.903968in}}%
\pgfpathlineto{\pgfqpoint{1.945309in}{0.898215in}}%
\pgfpathlineto{\pgfqpoint{1.949818in}{0.895909in}}%
\pgfpathlineto{\pgfqpoint{1.954327in}{0.974404in}}%
\pgfpathlineto{\pgfqpoint{1.963345in}{0.935007in}}%
\pgfpathlineto{\pgfqpoint{1.967855in}{0.899759in}}%
\pgfpathlineto{\pgfqpoint{1.972364in}{0.925358in}}%
\pgfpathlineto{\pgfqpoint{1.976873in}{0.936634in}}%
\pgfpathlineto{\pgfqpoint{1.981382in}{0.919359in}}%
\pgfpathlineto{\pgfqpoint{1.985891in}{0.968325in}}%
\pgfpathlineto{\pgfqpoint{1.990400in}{0.900514in}}%
\pgfpathlineto{\pgfqpoint{1.994909in}{0.911408in}}%
\pgfpathlineto{\pgfqpoint{1.999418in}{0.909239in}}%
\pgfpathlineto{\pgfqpoint{2.003927in}{0.914823in}}%
\pgfpathlineto{\pgfqpoint{2.008436in}{1.009820in}}%
\pgfpathlineto{\pgfqpoint{2.012945in}{1.015971in}}%
\pgfpathlineto{\pgfqpoint{2.017455in}{1.012623in}}%
\pgfpathlineto{\pgfqpoint{2.021964in}{1.064888in}}%
\pgfpathlineto{\pgfqpoint{2.026473in}{0.918917in}}%
\pgfpathlineto{\pgfqpoint{2.030982in}{0.938460in}}%
\pgfpathlineto{\pgfqpoint{2.035491in}{0.975235in}}%
\pgfpathlineto{\pgfqpoint{2.040000in}{1.047010in}}%
\pgfpathlineto{\pgfqpoint{2.044509in}{0.996575in}}%
\pgfpathlineto{\pgfqpoint{2.049018in}{1.149836in}}%
\pgfpathlineto{\pgfqpoint{2.053527in}{1.072243in}}%
\pgfpathlineto{\pgfqpoint{2.058036in}{0.967331in}}%
\pgfpathlineto{\pgfqpoint{2.062545in}{0.951008in}}%
\pgfpathlineto{\pgfqpoint{2.067055in}{0.942189in}}%
\pgfpathlineto{\pgfqpoint{2.071564in}{0.925634in}}%
\pgfpathlineto{\pgfqpoint{2.076073in}{0.976520in}}%
\pgfpathlineto{\pgfqpoint{2.080582in}{0.944015in}}%
\pgfpathlineto{\pgfqpoint{2.085091in}{1.191834in}}%
\pgfpathlineto{\pgfqpoint{2.089600in}{0.930590in}}%
\pgfpathlineto{\pgfqpoint{2.094109in}{1.093335in}}%
\pgfpathlineto{\pgfqpoint{2.098618in}{0.959474in}}%
\pgfpathlineto{\pgfqpoint{2.103127in}{1.070676in}}%
\pgfpathlineto{\pgfqpoint{2.107636in}{1.150990in}}%
\pgfpathlineto{\pgfqpoint{2.112145in}{0.948515in}}%
\pgfpathlineto{\pgfqpoint{2.116655in}{0.986268in}}%
\pgfpathlineto{\pgfqpoint{2.121164in}{1.443981in}}%
\pgfpathlineto{\pgfqpoint{2.125673in}{1.125405in}}%
\pgfpathlineto{\pgfqpoint{2.130182in}{1.399037in}}%
\pgfpathlineto{\pgfqpoint{2.134691in}{1.260262in}}%
\pgfpathlineto{\pgfqpoint{2.139200in}{1.491491in}}%
\pgfpathlineto{\pgfqpoint{2.143709in}{0.979076in}}%
\pgfpathlineto{\pgfqpoint{2.148218in}{0.951808in}}%
\pgfpathlineto{\pgfqpoint{2.152727in}{0.937748in}}%
\pgfpathlineto{\pgfqpoint{2.157236in}{0.979501in}}%
\pgfpathlineto{\pgfqpoint{2.161745in}{0.998809in}}%
\pgfpathlineto{\pgfqpoint{2.166255in}{1.569751in}}%
\pgfpathlineto{\pgfqpoint{2.170764in}{1.023264in}}%
\pgfpathlineto{\pgfqpoint{2.175273in}{1.309947in}}%
\pgfpathlineto{\pgfqpoint{2.179782in}{1.519792in}}%
\pgfpathlineto{\pgfqpoint{2.184291in}{0.956932in}}%
\pgfpathlineto{\pgfqpoint{2.188800in}{1.450452in}}%
\pgfpathlineto{\pgfqpoint{2.193309in}{1.335417in}}%
\pgfpathlineto{\pgfqpoint{2.197818in}{1.598557in}}%
\pgfpathlineto{\pgfqpoint{2.206836in}{1.006413in}}%
\pgfpathlineto{\pgfqpoint{2.211345in}{1.298335in}}%
\pgfpathlineto{\pgfqpoint{2.215855in}{1.028105in}}%
\pgfpathlineto{\pgfqpoint{2.220364in}{1.032823in}}%
\pgfpathlineto{\pgfqpoint{2.224873in}{0.997317in}}%
\pgfpathlineto{\pgfqpoint{2.229382in}{1.130739in}}%
\pgfpathlineto{\pgfqpoint{2.238400in}{0.968419in}}%
\pgfpathlineto{\pgfqpoint{2.242909in}{1.110118in}}%
\pgfpathlineto{\pgfqpoint{2.247418in}{0.976300in}}%
\pgfpathlineto{\pgfqpoint{2.251927in}{0.990724in}}%
\pgfpathlineto{\pgfqpoint{2.256436in}{0.991898in}}%
\pgfpathlineto{\pgfqpoint{2.260945in}{1.237866in}}%
\pgfpathlineto{\pgfqpoint{2.265455in}{0.984135in}}%
\pgfpathlineto{\pgfqpoint{2.269964in}{1.272951in}}%
\pgfpathlineto{\pgfqpoint{2.274473in}{1.099294in}}%
\pgfpathlineto{\pgfqpoint{2.278982in}{1.005174in}}%
\pgfpathlineto{\pgfqpoint{2.283491in}{0.978930in}}%
\pgfpathlineto{\pgfqpoint{2.288000in}{1.122538in}}%
\pgfpathlineto{\pgfqpoint{2.292509in}{0.983105in}}%
\pgfpathlineto{\pgfqpoint{2.297018in}{1.020068in}}%
\pgfpathlineto{\pgfqpoint{2.301527in}{1.120196in}}%
\pgfpathlineto{\pgfqpoint{2.306036in}{1.137551in}}%
\pgfpathlineto{\pgfqpoint{2.310545in}{0.999860in}}%
\pgfpathlineto{\pgfqpoint{2.315055in}{1.168563in}}%
\pgfpathlineto{\pgfqpoint{2.319564in}{1.180910in}}%
\pgfpathlineto{\pgfqpoint{2.324073in}{0.991120in}}%
\pgfpathlineto{\pgfqpoint{2.328582in}{1.039039in}}%
\pgfpathlineto{\pgfqpoint{2.333091in}{1.021119in}}%
\pgfpathlineto{\pgfqpoint{2.337600in}{1.054128in}}%
\pgfpathlineto{\pgfqpoint{2.342109in}{1.149425in}}%
\pgfpathlineto{\pgfqpoint{2.346618in}{0.995957in}}%
\pgfpathlineto{\pgfqpoint{2.351127in}{1.006024in}}%
\pgfpathlineto{\pgfqpoint{2.355636in}{1.041882in}}%
\pgfpathlineto{\pgfqpoint{2.360145in}{1.019660in}}%
\pgfpathlineto{\pgfqpoint{2.364655in}{1.146599in}}%
\pgfpathlineto{\pgfqpoint{2.369164in}{1.023299in}}%
\pgfpathlineto{\pgfqpoint{2.373673in}{1.170985in}}%
\pgfpathlineto{\pgfqpoint{2.378182in}{1.170068in}}%
\pgfpathlineto{\pgfqpoint{2.382691in}{1.180095in}}%
\pgfpathlineto{\pgfqpoint{2.387200in}{1.183197in}}%
\pgfpathlineto{\pgfqpoint{2.391709in}{1.069256in}}%
\pgfpathlineto{\pgfqpoint{2.396218in}{1.048626in}}%
\pgfpathlineto{\pgfqpoint{2.400727in}{1.017997in}}%
\pgfpathlineto{\pgfqpoint{2.405236in}{1.148958in}}%
\pgfpathlineto{\pgfqpoint{2.409745in}{1.058277in}}%
\pgfpathlineto{\pgfqpoint{2.414255in}{1.027103in}}%
\pgfpathlineto{\pgfqpoint{2.418764in}{1.082490in}}%
\pgfpathlineto{\pgfqpoint{2.423273in}{1.027301in}}%
\pgfpathlineto{\pgfqpoint{2.427782in}{1.092772in}}%
\pgfpathlineto{\pgfqpoint{2.432291in}{1.347262in}}%
\pgfpathlineto{\pgfqpoint{2.436800in}{1.059768in}}%
\pgfpathlineto{\pgfqpoint{2.441309in}{1.237252in}}%
\pgfpathlineto{\pgfqpoint{2.445818in}{1.171757in}}%
\pgfpathlineto{\pgfqpoint{2.450327in}{1.345786in}}%
\pgfpathlineto{\pgfqpoint{2.454836in}{1.041731in}}%
\pgfpathlineto{\pgfqpoint{2.459345in}{1.118978in}}%
\pgfpathlineto{\pgfqpoint{2.463855in}{1.030407in}}%
\pgfpathlineto{\pgfqpoint{2.468364in}{1.215111in}}%
\pgfpathlineto{\pgfqpoint{2.472873in}{1.284553in}}%
\pgfpathlineto{\pgfqpoint{2.477382in}{1.133112in}}%
\pgfpathlineto{\pgfqpoint{2.481891in}{1.041107in}}%
\pgfpathlineto{\pgfqpoint{2.486400in}{1.222376in}}%
\pgfpathlineto{\pgfqpoint{2.490909in}{1.124391in}}%
\pgfpathlineto{\pgfqpoint{2.495418in}{1.073342in}}%
\pgfpathlineto{\pgfqpoint{2.499927in}{1.063389in}}%
\pgfpathlineto{\pgfqpoint{2.504436in}{1.080011in}}%
\pgfpathlineto{\pgfqpoint{2.508945in}{1.144531in}}%
\pgfpathlineto{\pgfqpoint{2.513455in}{1.054059in}}%
\pgfpathlineto{\pgfqpoint{2.522473in}{1.277520in}}%
\pgfpathlineto{\pgfqpoint{2.526982in}{1.054843in}}%
\pgfpathlineto{\pgfqpoint{2.531491in}{1.120773in}}%
\pgfpathlineto{\pgfqpoint{2.536000in}{1.048563in}}%
\pgfpathlineto{\pgfqpoint{2.540509in}{1.081884in}}%
\pgfpathlineto{\pgfqpoint{2.545018in}{1.202522in}}%
\pgfpathlineto{\pgfqpoint{2.549527in}{1.186640in}}%
\pgfpathlineto{\pgfqpoint{2.554036in}{1.060912in}}%
\pgfpathlineto{\pgfqpoint{2.558545in}{1.074776in}}%
\pgfpathlineto{\pgfqpoint{2.563055in}{1.041011in}}%
\pgfpathlineto{\pgfqpoint{2.567564in}{1.105745in}}%
\pgfpathlineto{\pgfqpoint{2.576582in}{1.066600in}}%
\pgfpathlineto{\pgfqpoint{2.581091in}{1.078439in}}%
\pgfpathlineto{\pgfqpoint{2.585600in}{1.273315in}}%
\pgfpathlineto{\pgfqpoint{2.590109in}{1.179079in}}%
\pgfpathlineto{\pgfqpoint{2.594618in}{1.058410in}}%
\pgfpathlineto{\pgfqpoint{2.599127in}{1.128188in}}%
\pgfpathlineto{\pgfqpoint{2.603636in}{1.102320in}}%
\pgfpathlineto{\pgfqpoint{2.608145in}{1.157807in}}%
\pgfpathlineto{\pgfqpoint{2.612655in}{1.099188in}}%
\pgfpathlineto{\pgfqpoint{2.617164in}{1.096133in}}%
\pgfpathlineto{\pgfqpoint{2.621673in}{1.309669in}}%
\pgfpathlineto{\pgfqpoint{2.626182in}{1.111613in}}%
\pgfpathlineto{\pgfqpoint{2.630691in}{1.171472in}}%
\pgfpathlineto{\pgfqpoint{2.635200in}{1.097093in}}%
\pgfpathlineto{\pgfqpoint{2.639709in}{1.098447in}}%
\pgfpathlineto{\pgfqpoint{2.644218in}{1.170821in}}%
\pgfpathlineto{\pgfqpoint{2.648727in}{1.163711in}}%
\pgfpathlineto{\pgfqpoint{2.653236in}{1.151653in}}%
\pgfpathlineto{\pgfqpoint{2.657745in}{1.105873in}}%
\pgfpathlineto{\pgfqpoint{2.662255in}{1.107686in}}%
\pgfpathlineto{\pgfqpoint{2.666764in}{1.066724in}}%
\pgfpathlineto{\pgfqpoint{2.671273in}{1.151061in}}%
\pgfpathlineto{\pgfqpoint{2.675782in}{1.109248in}}%
\pgfpathlineto{\pgfqpoint{2.680291in}{1.136889in}}%
\pgfpathlineto{\pgfqpoint{2.684800in}{1.194820in}}%
\pgfpathlineto{\pgfqpoint{2.689309in}{1.092250in}}%
\pgfpathlineto{\pgfqpoint{2.693818in}{1.090564in}}%
\pgfpathlineto{\pgfqpoint{2.698327in}{1.315892in}}%
\pgfpathlineto{\pgfqpoint{2.702836in}{1.399164in}}%
\pgfpathlineto{\pgfqpoint{2.707345in}{1.524449in}}%
\pgfpathlineto{\pgfqpoint{2.711855in}{1.181502in}}%
\pgfpathlineto{\pgfqpoint{2.716364in}{1.183097in}}%
\pgfpathlineto{\pgfqpoint{2.720873in}{1.366494in}}%
\pgfpathlineto{\pgfqpoint{2.725382in}{1.138545in}}%
\pgfpathlineto{\pgfqpoint{2.729891in}{1.137774in}}%
\pgfpathlineto{\pgfqpoint{2.734400in}{1.135577in}}%
\pgfpathlineto{\pgfqpoint{2.738909in}{1.124525in}}%
\pgfpathlineto{\pgfqpoint{2.743418in}{1.147876in}}%
\pgfpathlineto{\pgfqpoint{2.747927in}{1.141667in}}%
\pgfpathlineto{\pgfqpoint{2.752436in}{1.150907in}}%
\pgfpathlineto{\pgfqpoint{2.756945in}{1.262244in}}%
\pgfpathlineto{\pgfqpoint{2.761455in}{1.279307in}}%
\pgfpathlineto{\pgfqpoint{2.765964in}{1.181073in}}%
\pgfpathlineto{\pgfqpoint{2.770473in}{1.143973in}}%
\pgfpathlineto{\pgfqpoint{2.774982in}{1.195472in}}%
\pgfpathlineto{\pgfqpoint{2.779491in}{1.173928in}}%
\pgfpathlineto{\pgfqpoint{2.784000in}{1.239138in}}%
\pgfpathlineto{\pgfqpoint{2.788509in}{1.159332in}}%
\pgfpathlineto{\pgfqpoint{2.793018in}{1.215361in}}%
\pgfpathlineto{\pgfqpoint{2.797527in}{1.200016in}}%
\pgfpathlineto{\pgfqpoint{2.802036in}{1.271242in}}%
\pgfpathlineto{\pgfqpoint{2.806545in}{1.244553in}}%
\pgfpathlineto{\pgfqpoint{2.811055in}{1.277614in}}%
\pgfpathlineto{\pgfqpoint{2.815564in}{1.280565in}}%
\pgfpathlineto{\pgfqpoint{2.820073in}{1.171365in}}%
\pgfpathlineto{\pgfqpoint{2.824582in}{1.153360in}}%
\pgfpathlineto{\pgfqpoint{2.829091in}{1.374822in}}%
\pgfpathlineto{\pgfqpoint{2.833600in}{1.140810in}}%
\pgfpathlineto{\pgfqpoint{2.838109in}{1.169595in}}%
\pgfpathlineto{\pgfqpoint{2.842618in}{1.133068in}}%
\pgfpathlineto{\pgfqpoint{2.847127in}{1.165393in}}%
\pgfpathlineto{\pgfqpoint{2.851636in}{1.160455in}}%
\pgfpathlineto{\pgfqpoint{2.860655in}{1.229532in}}%
\pgfpathlineto{\pgfqpoint{2.865164in}{1.139212in}}%
\pgfpathlineto{\pgfqpoint{2.869673in}{1.166105in}}%
\pgfpathlineto{\pgfqpoint{2.874182in}{1.172326in}}%
\pgfpathlineto{\pgfqpoint{2.878691in}{1.204575in}}%
\pgfpathlineto{\pgfqpoint{2.887709in}{1.144560in}}%
\pgfpathlineto{\pgfqpoint{2.892218in}{1.725496in}}%
\pgfpathlineto{\pgfqpoint{2.896727in}{1.639577in}}%
\pgfpathlineto{\pgfqpoint{2.901236in}{1.311453in}}%
\pgfpathlineto{\pgfqpoint{2.905745in}{1.206525in}}%
\pgfpathlineto{\pgfqpoint{2.910255in}{1.199644in}}%
\pgfpathlineto{\pgfqpoint{2.914764in}{1.168982in}}%
\pgfpathlineto{\pgfqpoint{2.919273in}{1.583323in}}%
\pgfpathlineto{\pgfqpoint{2.923782in}{1.196148in}}%
\pgfpathlineto{\pgfqpoint{2.928291in}{1.185657in}}%
\pgfpathlineto{\pgfqpoint{2.932800in}{1.476046in}}%
\pgfpathlineto{\pgfqpoint{2.937309in}{1.302574in}}%
\pgfpathlineto{\pgfqpoint{2.941818in}{1.257921in}}%
\pgfpathlineto{\pgfqpoint{2.946327in}{1.300637in}}%
\pgfpathlineto{\pgfqpoint{2.950836in}{1.180012in}}%
\pgfpathlineto{\pgfqpoint{2.955345in}{1.155998in}}%
\pgfpathlineto{\pgfqpoint{2.959855in}{1.311774in}}%
\pgfpathlineto{\pgfqpoint{2.964364in}{1.216250in}}%
\pgfpathlineto{\pgfqpoint{2.968873in}{1.314173in}}%
\pgfpathlineto{\pgfqpoint{2.973382in}{1.246300in}}%
\pgfpathlineto{\pgfqpoint{2.977891in}{1.222537in}}%
\pgfpathlineto{\pgfqpoint{2.982400in}{1.296501in}}%
\pgfpathlineto{\pgfqpoint{2.986909in}{1.555005in}}%
\pgfpathlineto{\pgfqpoint{2.991418in}{1.452233in}}%
\pgfpathlineto{\pgfqpoint{2.995927in}{1.242911in}}%
\pgfpathlineto{\pgfqpoint{3.000436in}{1.232771in}}%
\pgfpathlineto{\pgfqpoint{3.004945in}{1.302022in}}%
\pgfpathlineto{\pgfqpoint{3.009455in}{1.289961in}}%
\pgfpathlineto{\pgfqpoint{3.013964in}{1.238089in}}%
\pgfpathlineto{\pgfqpoint{3.018473in}{1.230881in}}%
\pgfpathlineto{\pgfqpoint{3.022982in}{1.232664in}}%
\pgfpathlineto{\pgfqpoint{3.027491in}{1.304652in}}%
\pgfpathlineto{\pgfqpoint{3.032000in}{1.282515in}}%
\pgfpathlineto{\pgfqpoint{3.036509in}{1.173361in}}%
\pgfpathlineto{\pgfqpoint{3.041018in}{1.276915in}}%
\pgfpathlineto{\pgfqpoint{3.045527in}{1.325554in}}%
\pgfpathlineto{\pgfqpoint{3.050036in}{1.213243in}}%
\pgfpathlineto{\pgfqpoint{3.054545in}{1.343762in}}%
\pgfpathlineto{\pgfqpoint{3.059055in}{1.220452in}}%
\pgfpathlineto{\pgfqpoint{3.063564in}{1.223845in}}%
\pgfpathlineto{\pgfqpoint{3.068073in}{1.239652in}}%
\pgfpathlineto{\pgfqpoint{3.072582in}{1.225537in}}%
\pgfpathlineto{\pgfqpoint{3.077091in}{1.358702in}}%
\pgfpathlineto{\pgfqpoint{3.081600in}{1.326034in}}%
\pgfpathlineto{\pgfqpoint{3.086109in}{1.209192in}}%
\pgfpathlineto{\pgfqpoint{3.090618in}{1.212971in}}%
\pgfpathlineto{\pgfqpoint{3.095127in}{1.254727in}}%
\pgfpathlineto{\pgfqpoint{3.099636in}{1.375929in}}%
\pgfpathlineto{\pgfqpoint{3.104145in}{1.345252in}}%
\pgfpathlineto{\pgfqpoint{3.113164in}{1.232160in}}%
\pgfpathlineto{\pgfqpoint{3.117673in}{1.374527in}}%
\pgfpathlineto{\pgfqpoint{3.122182in}{1.347517in}}%
\pgfpathlineto{\pgfqpoint{3.126691in}{1.270205in}}%
\pgfpathlineto{\pgfqpoint{3.131200in}{1.374928in}}%
\pgfpathlineto{\pgfqpoint{3.135709in}{1.397840in}}%
\pgfpathlineto{\pgfqpoint{3.140218in}{1.233984in}}%
\pgfpathlineto{\pgfqpoint{3.144727in}{1.328303in}}%
\pgfpathlineto{\pgfqpoint{3.149236in}{1.385074in}}%
\pgfpathlineto{\pgfqpoint{3.153745in}{1.296014in}}%
\pgfpathlineto{\pgfqpoint{3.158255in}{1.368039in}}%
\pgfpathlineto{\pgfqpoint{3.162764in}{1.232987in}}%
\pgfpathlineto{\pgfqpoint{3.167273in}{1.279812in}}%
\pgfpathlineto{\pgfqpoint{3.171782in}{1.305797in}}%
\pgfpathlineto{\pgfqpoint{3.176291in}{1.437271in}}%
\pgfpathlineto{\pgfqpoint{3.180800in}{1.318412in}}%
\pgfpathlineto{\pgfqpoint{3.185309in}{1.285059in}}%
\pgfpathlineto{\pgfqpoint{3.189818in}{1.217121in}}%
\pgfpathlineto{\pgfqpoint{3.194327in}{1.371158in}}%
\pgfpathlineto{\pgfqpoint{3.198836in}{1.250239in}}%
\pgfpathlineto{\pgfqpoint{3.203345in}{1.433454in}}%
\pgfpathlineto{\pgfqpoint{3.207855in}{1.514284in}}%
\pgfpathlineto{\pgfqpoint{3.212364in}{1.455930in}}%
\pgfpathlineto{\pgfqpoint{3.216873in}{1.280534in}}%
\pgfpathlineto{\pgfqpoint{3.221382in}{1.584639in}}%
\pgfpathlineto{\pgfqpoint{3.225891in}{1.314654in}}%
\pgfpathlineto{\pgfqpoint{3.230400in}{1.372139in}}%
\pgfpathlineto{\pgfqpoint{3.234909in}{1.409785in}}%
\pgfpathlineto{\pgfqpoint{3.239418in}{1.274294in}}%
\pgfpathlineto{\pgfqpoint{3.243927in}{1.236281in}}%
\pgfpathlineto{\pgfqpoint{3.248436in}{1.364842in}}%
\pgfpathlineto{\pgfqpoint{3.252945in}{1.336288in}}%
\pgfpathlineto{\pgfqpoint{3.257455in}{1.283469in}}%
\pgfpathlineto{\pgfqpoint{3.261964in}{1.403501in}}%
\pgfpathlineto{\pgfqpoint{3.266473in}{1.288437in}}%
\pgfpathlineto{\pgfqpoint{3.270982in}{1.310085in}}%
\pgfpathlineto{\pgfqpoint{3.275491in}{1.341603in}}%
\pgfpathlineto{\pgfqpoint{3.280000in}{1.478608in}}%
\pgfpathlineto{\pgfqpoint{3.284509in}{1.290970in}}%
\pgfpathlineto{\pgfqpoint{3.289018in}{1.253982in}}%
\pgfpathlineto{\pgfqpoint{3.293527in}{1.419796in}}%
\pgfpathlineto{\pgfqpoint{3.298036in}{1.445762in}}%
\pgfpathlineto{\pgfqpoint{3.302545in}{1.328139in}}%
\pgfpathlineto{\pgfqpoint{3.311564in}{1.291594in}}%
\pgfpathlineto{\pgfqpoint{3.316073in}{1.420619in}}%
\pgfpathlineto{\pgfqpoint{3.320582in}{1.296467in}}%
\pgfpathlineto{\pgfqpoint{3.325091in}{1.625769in}}%
\pgfpathlineto{\pgfqpoint{3.329600in}{1.485964in}}%
\pgfpathlineto{\pgfqpoint{3.334109in}{1.287562in}}%
\pgfpathlineto{\pgfqpoint{3.338618in}{1.313044in}}%
\pgfpathlineto{\pgfqpoint{3.343127in}{1.411225in}}%
\pgfpathlineto{\pgfqpoint{3.347636in}{1.268522in}}%
\pgfpathlineto{\pgfqpoint{3.352145in}{1.332120in}}%
\pgfpathlineto{\pgfqpoint{3.356655in}{1.312617in}}%
\pgfpathlineto{\pgfqpoint{3.361164in}{1.258424in}}%
\pgfpathlineto{\pgfqpoint{3.365673in}{1.439912in}}%
\pgfpathlineto{\pgfqpoint{3.370182in}{1.295098in}}%
\pgfpathlineto{\pgfqpoint{3.374691in}{1.331003in}}%
\pgfpathlineto{\pgfqpoint{3.379200in}{1.324788in}}%
\pgfpathlineto{\pgfqpoint{3.383709in}{1.388272in}}%
\pgfpathlineto{\pgfqpoint{3.388218in}{1.538463in}}%
\pgfpathlineto{\pgfqpoint{3.392727in}{1.328822in}}%
\pgfpathlineto{\pgfqpoint{3.397236in}{1.296583in}}%
\pgfpathlineto{\pgfqpoint{3.401745in}{1.295644in}}%
\pgfpathlineto{\pgfqpoint{3.406255in}{1.976137in}}%
\pgfpathlineto{\pgfqpoint{3.410764in}{1.563758in}}%
\pgfpathlineto{\pgfqpoint{3.415273in}{1.514060in}}%
\pgfpathlineto{\pgfqpoint{3.424291in}{1.361597in}}%
\pgfpathlineto{\pgfqpoint{3.428800in}{1.328026in}}%
\pgfpathlineto{\pgfqpoint{3.433309in}{1.338019in}}%
\pgfpathlineto{\pgfqpoint{3.437818in}{1.327460in}}%
\pgfpathlineto{\pgfqpoint{3.442327in}{1.944538in}}%
\pgfpathlineto{\pgfqpoint{3.446836in}{1.361347in}}%
\pgfpathlineto{\pgfqpoint{3.451345in}{1.461243in}}%
\pgfpathlineto{\pgfqpoint{3.455855in}{1.398451in}}%
\pgfpathlineto{\pgfqpoint{3.460364in}{1.513130in}}%
\pgfpathlineto{\pgfqpoint{3.464873in}{1.385610in}}%
\pgfpathlineto{\pgfqpoint{3.469382in}{1.416803in}}%
\pgfpathlineto{\pgfqpoint{3.473891in}{1.504007in}}%
\pgfpathlineto{\pgfqpoint{3.478400in}{1.359787in}}%
\pgfpathlineto{\pgfqpoint{3.482909in}{1.514124in}}%
\pgfpathlineto{\pgfqpoint{3.487418in}{1.382009in}}%
\pgfpathlineto{\pgfqpoint{3.491927in}{1.301577in}}%
\pgfpathlineto{\pgfqpoint{3.496436in}{1.483470in}}%
\pgfpathlineto{\pgfqpoint{3.500945in}{1.354849in}}%
\pgfpathlineto{\pgfqpoint{3.505455in}{1.340043in}}%
\pgfpathlineto{\pgfqpoint{3.509964in}{1.482530in}}%
\pgfpathlineto{\pgfqpoint{3.514473in}{1.458442in}}%
\pgfpathlineto{\pgfqpoint{3.518982in}{1.535136in}}%
\pgfpathlineto{\pgfqpoint{3.523491in}{1.373088in}}%
\pgfpathlineto{\pgfqpoint{3.528000in}{1.330523in}}%
\pgfpathlineto{\pgfqpoint{3.537018in}{1.471137in}}%
\pgfpathlineto{\pgfqpoint{3.541527in}{1.342869in}}%
\pgfpathlineto{\pgfqpoint{3.546036in}{1.349823in}}%
\pgfpathlineto{\pgfqpoint{3.550545in}{1.363248in}}%
\pgfpathlineto{\pgfqpoint{3.555055in}{1.404982in}}%
\pgfpathlineto{\pgfqpoint{3.559564in}{1.348454in}}%
\pgfpathlineto{\pgfqpoint{3.564073in}{1.430311in}}%
\pgfpathlineto{\pgfqpoint{3.568582in}{1.438798in}}%
\pgfpathlineto{\pgfqpoint{3.573091in}{1.555245in}}%
\pgfpathlineto{\pgfqpoint{3.577600in}{1.446263in}}%
\pgfpathlineto{\pgfqpoint{3.582109in}{1.386214in}}%
\pgfpathlineto{\pgfqpoint{3.586618in}{1.360546in}}%
\pgfpathlineto{\pgfqpoint{3.591127in}{1.518068in}}%
\pgfpathlineto{\pgfqpoint{3.595636in}{1.434062in}}%
\pgfpathlineto{\pgfqpoint{3.600145in}{1.753849in}}%
\pgfpathlineto{\pgfqpoint{3.604655in}{1.440494in}}%
\pgfpathlineto{\pgfqpoint{3.609164in}{1.638562in}}%
\pgfpathlineto{\pgfqpoint{3.613673in}{1.398189in}}%
\pgfpathlineto{\pgfqpoint{3.618182in}{1.444761in}}%
\pgfpathlineto{\pgfqpoint{3.622691in}{1.532208in}}%
\pgfpathlineto{\pgfqpoint{3.627200in}{1.413511in}}%
\pgfpathlineto{\pgfqpoint{3.631709in}{1.606622in}}%
\pgfpathlineto{\pgfqpoint{3.636218in}{1.417832in}}%
\pgfpathlineto{\pgfqpoint{3.640727in}{1.399990in}}%
\pgfpathlineto{\pgfqpoint{3.645236in}{1.375886in}}%
\pgfpathlineto{\pgfqpoint{3.649745in}{1.435558in}}%
\pgfpathlineto{\pgfqpoint{3.654255in}{1.552445in}}%
\pgfpathlineto{\pgfqpoint{3.658764in}{1.925733in}}%
\pgfpathlineto{\pgfqpoint{3.663273in}{1.433181in}}%
\pgfpathlineto{\pgfqpoint{3.667782in}{1.385816in}}%
\pgfpathlineto{\pgfqpoint{3.672291in}{1.630005in}}%
\pgfpathlineto{\pgfqpoint{3.676800in}{1.506247in}}%
\pgfpathlineto{\pgfqpoint{3.681309in}{1.344217in}}%
\pgfpathlineto{\pgfqpoint{3.685818in}{1.514607in}}%
\pgfpathlineto{\pgfqpoint{3.690327in}{1.939598in}}%
\pgfpathlineto{\pgfqpoint{3.694836in}{1.486004in}}%
\pgfpathlineto{\pgfqpoint{3.699345in}{1.569439in}}%
\pgfpathlineto{\pgfqpoint{3.703855in}{1.531190in}}%
\pgfpathlineto{\pgfqpoint{3.708364in}{1.365107in}}%
\pgfpathlineto{\pgfqpoint{3.712873in}{1.461706in}}%
\pgfpathlineto{\pgfqpoint{3.717382in}{1.378974in}}%
\pgfpathlineto{\pgfqpoint{3.721891in}{1.486733in}}%
\pgfpathlineto{\pgfqpoint{3.726400in}{1.480369in}}%
\pgfpathlineto{\pgfqpoint{3.730909in}{1.464116in}}%
\pgfpathlineto{\pgfqpoint{3.735418in}{1.505697in}}%
\pgfpathlineto{\pgfqpoint{3.739927in}{1.475248in}}%
\pgfpathlineto{\pgfqpoint{3.744436in}{1.434453in}}%
\pgfpathlineto{\pgfqpoint{3.748945in}{1.468805in}}%
\pgfpathlineto{\pgfqpoint{3.753455in}{1.401593in}}%
\pgfpathlineto{\pgfqpoint{3.757964in}{1.406909in}}%
\pgfpathlineto{\pgfqpoint{3.762473in}{1.382550in}}%
\pgfpathlineto{\pgfqpoint{3.771491in}{1.617207in}}%
\pgfpathlineto{\pgfqpoint{3.776000in}{1.616941in}}%
\pgfpathlineto{\pgfqpoint{3.780509in}{1.455279in}}%
\pgfpathlineto{\pgfqpoint{3.785018in}{1.783924in}}%
\pgfpathlineto{\pgfqpoint{3.789527in}{2.672904in}}%
\pgfpathlineto{\pgfqpoint{3.794036in}{1.635569in}}%
\pgfpathlineto{\pgfqpoint{3.798545in}{1.831396in}}%
\pgfpathlineto{\pgfqpoint{3.803055in}{1.863967in}}%
\pgfpathlineto{\pgfqpoint{3.807564in}{1.829206in}}%
\pgfpathlineto{\pgfqpoint{3.812073in}{1.713099in}}%
\pgfpathlineto{\pgfqpoint{3.816582in}{2.270861in}}%
\pgfpathlineto{\pgfqpoint{3.821091in}{1.480223in}}%
\pgfpathlineto{\pgfqpoint{3.825600in}{1.855499in}}%
\pgfpathlineto{\pgfqpoint{3.830109in}{1.965378in}}%
\pgfpathlineto{\pgfqpoint{3.834618in}{1.418043in}}%
\pgfpathlineto{\pgfqpoint{3.839127in}{1.479277in}}%
\pgfpathlineto{\pgfqpoint{3.843636in}{2.182178in}}%
\pgfpathlineto{\pgfqpoint{3.848145in}{1.574551in}}%
\pgfpathlineto{\pgfqpoint{3.852655in}{1.502933in}}%
\pgfpathlineto{\pgfqpoint{3.857164in}{1.533772in}}%
\pgfpathlineto{\pgfqpoint{3.861673in}{1.516349in}}%
\pgfpathlineto{\pgfqpoint{3.866182in}{1.504381in}}%
\pgfpathlineto{\pgfqpoint{3.870691in}{1.954916in}}%
\pgfpathlineto{\pgfqpoint{3.875200in}{1.618491in}}%
\pgfpathlineto{\pgfqpoint{3.884218in}{1.460286in}}%
\pgfpathlineto{\pgfqpoint{3.893236in}{1.656763in}}%
\pgfpathlineto{\pgfqpoint{3.897745in}{1.450765in}}%
\pgfpathlineto{\pgfqpoint{3.902255in}{1.565779in}}%
\pgfpathlineto{\pgfqpoint{3.906764in}{1.777237in}}%
\pgfpathlineto{\pgfqpoint{3.915782in}{1.453834in}}%
\pgfpathlineto{\pgfqpoint{3.920291in}{1.605315in}}%
\pgfpathlineto{\pgfqpoint{3.924800in}{1.579610in}}%
\pgfpathlineto{\pgfqpoint{3.929309in}{1.467383in}}%
\pgfpathlineto{\pgfqpoint{3.933818in}{1.536150in}}%
\pgfpathlineto{\pgfqpoint{3.938327in}{1.667618in}}%
\pgfpathlineto{\pgfqpoint{3.942836in}{1.474566in}}%
\pgfpathlineto{\pgfqpoint{3.947345in}{1.700893in}}%
\pgfpathlineto{\pgfqpoint{3.951855in}{1.529636in}}%
\pgfpathlineto{\pgfqpoint{3.956364in}{1.477035in}}%
\pgfpathlineto{\pgfqpoint{3.960873in}{1.530189in}}%
\pgfpathlineto{\pgfqpoint{3.965382in}{1.706090in}}%
\pgfpathlineto{\pgfqpoint{3.969891in}{1.793895in}}%
\pgfpathlineto{\pgfqpoint{3.974400in}{1.488104in}}%
\pgfpathlineto{\pgfqpoint{3.978909in}{1.710511in}}%
\pgfpathlineto{\pgfqpoint{3.983418in}{1.593981in}}%
\pgfpathlineto{\pgfqpoint{3.987927in}{1.517031in}}%
\pgfpathlineto{\pgfqpoint{3.992436in}{1.651327in}}%
\pgfpathlineto{\pgfqpoint{3.996945in}{1.554743in}}%
\pgfpathlineto{\pgfqpoint{4.001455in}{1.544763in}}%
\pgfpathlineto{\pgfqpoint{4.005964in}{1.520971in}}%
\pgfpathlineto{\pgfqpoint{4.010473in}{1.572847in}}%
\pgfpathlineto{\pgfqpoint{4.014982in}{1.687719in}}%
\pgfpathlineto{\pgfqpoint{4.019491in}{1.509434in}}%
\pgfpathlineto{\pgfqpoint{4.024000in}{1.510804in}}%
\pgfpathlineto{\pgfqpoint{4.028509in}{1.567822in}}%
\pgfpathlineto{\pgfqpoint{4.033018in}{1.708761in}}%
\pgfpathlineto{\pgfqpoint{4.037527in}{2.111644in}}%
\pgfpathlineto{\pgfqpoint{4.042036in}{1.588779in}}%
\pgfpathlineto{\pgfqpoint{4.046545in}{1.612080in}}%
\pgfpathlineto{\pgfqpoint{4.051055in}{1.713253in}}%
\pgfpathlineto{\pgfqpoint{4.055564in}{1.520797in}}%
\pgfpathlineto{\pgfqpoint{4.060073in}{1.672326in}}%
\pgfpathlineto{\pgfqpoint{4.064582in}{1.745735in}}%
\pgfpathlineto{\pgfqpoint{4.069091in}{1.520499in}}%
\pgfpathlineto{\pgfqpoint{4.073600in}{1.718089in}}%
\pgfpathlineto{\pgfqpoint{4.078109in}{1.587202in}}%
\pgfpathlineto{\pgfqpoint{4.082618in}{1.638600in}}%
\pgfpathlineto{\pgfqpoint{4.087127in}{1.550453in}}%
\pgfpathlineto{\pgfqpoint{4.091636in}{1.652455in}}%
\pgfpathlineto{\pgfqpoint{4.096145in}{1.534003in}}%
\pgfpathlineto{\pgfqpoint{4.100655in}{1.550698in}}%
\pgfpathlineto{\pgfqpoint{4.109673in}{1.952328in}}%
\pgfpathlineto{\pgfqpoint{4.114182in}{2.374915in}}%
\pgfpathlineto{\pgfqpoint{4.118691in}{1.639596in}}%
\pgfpathlineto{\pgfqpoint{4.123200in}{1.716848in}}%
\pgfpathlineto{\pgfqpoint{4.132218in}{1.547859in}}%
\pgfpathlineto{\pgfqpoint{4.136727in}{1.676421in}}%
\pgfpathlineto{\pgfqpoint{4.141236in}{1.737425in}}%
\pgfpathlineto{\pgfqpoint{4.145745in}{1.504707in}}%
\pgfpathlineto{\pgfqpoint{4.150255in}{1.735336in}}%
\pgfpathlineto{\pgfqpoint{4.154764in}{1.685064in}}%
\pgfpathlineto{\pgfqpoint{4.159273in}{1.569688in}}%
\pgfpathlineto{\pgfqpoint{4.163782in}{1.598447in}}%
\pgfpathlineto{\pgfqpoint{4.168291in}{2.101575in}}%
\pgfpathlineto{\pgfqpoint{4.172800in}{1.568913in}}%
\pgfpathlineto{\pgfqpoint{4.177309in}{1.846002in}}%
\pgfpathlineto{\pgfqpoint{4.181818in}{1.578625in}}%
\pgfpathlineto{\pgfqpoint{4.186327in}{1.688674in}}%
\pgfpathlineto{\pgfqpoint{4.190836in}{1.631134in}}%
\pgfpathlineto{\pgfqpoint{4.195345in}{1.726002in}}%
\pgfpathlineto{\pgfqpoint{4.199855in}{1.700248in}}%
\pgfpathlineto{\pgfqpoint{4.208873in}{1.583932in}}%
\pgfpathlineto{\pgfqpoint{4.213382in}{1.745712in}}%
\pgfpathlineto{\pgfqpoint{4.217891in}{1.659991in}}%
\pgfpathlineto{\pgfqpoint{4.222400in}{2.092458in}}%
\pgfpathlineto{\pgfqpoint{4.226909in}{1.550977in}}%
\pgfpathlineto{\pgfqpoint{4.231418in}{1.653543in}}%
\pgfpathlineto{\pgfqpoint{4.235927in}{1.607597in}}%
\pgfpathlineto{\pgfqpoint{4.240436in}{1.691519in}}%
\pgfpathlineto{\pgfqpoint{4.244945in}{1.609193in}}%
\pgfpathlineto{\pgfqpoint{4.253964in}{1.680597in}}%
\pgfpathlineto{\pgfqpoint{4.258473in}{1.580442in}}%
\pgfpathlineto{\pgfqpoint{4.262982in}{1.736862in}}%
\pgfpathlineto{\pgfqpoint{4.267491in}{1.556268in}}%
\pgfpathlineto{\pgfqpoint{4.272000in}{1.737295in}}%
\pgfpathlineto{\pgfqpoint{4.276509in}{1.601827in}}%
\pgfpathlineto{\pgfqpoint{4.281018in}{1.568786in}}%
\pgfpathlineto{\pgfqpoint{4.290036in}{1.586616in}}%
\pgfpathlineto{\pgfqpoint{4.294545in}{1.586567in}}%
\pgfpathlineto{\pgfqpoint{4.299055in}{1.669895in}}%
\pgfpathlineto{\pgfqpoint{4.303564in}{1.617687in}}%
\pgfpathlineto{\pgfqpoint{4.308073in}{1.756341in}}%
\pgfpathlineto{\pgfqpoint{4.312582in}{1.657221in}}%
\pgfpathlineto{\pgfqpoint{4.317091in}{1.766455in}}%
\pgfpathlineto{\pgfqpoint{4.321600in}{1.607037in}}%
\pgfpathlineto{\pgfqpoint{4.330618in}{1.897000in}}%
\pgfpathlineto{\pgfqpoint{4.335127in}{1.895411in}}%
\pgfpathlineto{\pgfqpoint{4.339636in}{1.739486in}}%
\pgfpathlineto{\pgfqpoint{4.344145in}{1.792206in}}%
\pgfpathlineto{\pgfqpoint{4.348655in}{1.643502in}}%
\pgfpathlineto{\pgfqpoint{4.353164in}{1.730969in}}%
\pgfpathlineto{\pgfqpoint{4.357673in}{1.580671in}}%
\pgfpathlineto{\pgfqpoint{4.362182in}{1.718428in}}%
\pgfpathlineto{\pgfqpoint{4.366691in}{1.809240in}}%
\pgfpathlineto{\pgfqpoint{4.371200in}{1.648763in}}%
\pgfpathlineto{\pgfqpoint{4.375709in}{1.654043in}}%
\pgfpathlineto{\pgfqpoint{4.380218in}{1.778149in}}%
\pgfpathlineto{\pgfqpoint{4.384727in}{2.285422in}}%
\pgfpathlineto{\pgfqpoint{4.389236in}{1.811998in}}%
\pgfpathlineto{\pgfqpoint{4.393745in}{1.670147in}}%
\pgfpathlineto{\pgfqpoint{4.398255in}{1.730809in}}%
\pgfpathlineto{\pgfqpoint{4.402764in}{1.628325in}}%
\pgfpathlineto{\pgfqpoint{4.407273in}{1.737656in}}%
\pgfpathlineto{\pgfqpoint{4.411782in}{1.645876in}}%
\pgfpathlineto{\pgfqpoint{4.416291in}{1.760944in}}%
\pgfpathlineto{\pgfqpoint{4.420800in}{1.743592in}}%
\pgfpathlineto{\pgfqpoint{4.425309in}{1.659638in}}%
\pgfpathlineto{\pgfqpoint{4.429818in}{1.635792in}}%
\pgfpathlineto{\pgfqpoint{4.434327in}{1.891628in}}%
\pgfpathlineto{\pgfqpoint{4.438836in}{1.862394in}}%
\pgfpathlineto{\pgfqpoint{4.443345in}{1.694223in}}%
\pgfpathlineto{\pgfqpoint{4.447855in}{1.706948in}}%
\pgfpathlineto{\pgfqpoint{4.452364in}{1.832973in}}%
\pgfpathlineto{\pgfqpoint{4.456873in}{1.760266in}}%
\pgfpathlineto{\pgfqpoint{4.461382in}{1.721890in}}%
\pgfpathlineto{\pgfqpoint{4.470400in}{1.823632in}}%
\pgfpathlineto{\pgfqpoint{4.474909in}{1.621610in}}%
\pgfpathlineto{\pgfqpoint{4.483927in}{1.756281in}}%
\pgfpathlineto{\pgfqpoint{4.488436in}{1.745689in}}%
\pgfpathlineto{\pgfqpoint{4.492945in}{1.952161in}}%
\pgfpathlineto{\pgfqpoint{4.497455in}{1.636096in}}%
\pgfpathlineto{\pgfqpoint{4.506473in}{1.726513in}}%
\pgfpathlineto{\pgfqpoint{4.510982in}{1.642297in}}%
\pgfpathlineto{\pgfqpoint{4.515491in}{1.700674in}}%
\pgfpathlineto{\pgfqpoint{4.520000in}{1.733435in}}%
\pgfpathlineto{\pgfqpoint{4.524509in}{1.797868in}}%
\pgfpathlineto{\pgfqpoint{4.529018in}{1.813260in}}%
\pgfpathlineto{\pgfqpoint{4.533527in}{1.846965in}}%
\pgfpathlineto{\pgfqpoint{4.538036in}{1.868650in}}%
\pgfpathlineto{\pgfqpoint{4.542545in}{1.936602in}}%
\pgfpathlineto{\pgfqpoint{4.547055in}{1.850824in}}%
\pgfpathlineto{\pgfqpoint{4.551564in}{1.726742in}}%
\pgfpathlineto{\pgfqpoint{4.556073in}{1.655665in}}%
\pgfpathlineto{\pgfqpoint{4.560582in}{1.791941in}}%
\pgfpathlineto{\pgfqpoint{4.565091in}{2.045713in}}%
\pgfpathlineto{\pgfqpoint{4.569600in}{1.883254in}}%
\pgfpathlineto{\pgfqpoint{4.574109in}{1.850218in}}%
\pgfpathlineto{\pgfqpoint{4.578618in}{1.761367in}}%
\pgfpathlineto{\pgfqpoint{4.583127in}{1.628076in}}%
\pgfpathlineto{\pgfqpoint{4.587636in}{1.719013in}}%
\pgfpathlineto{\pgfqpoint{4.592145in}{1.664056in}}%
\pgfpathlineto{\pgfqpoint{4.596655in}{1.709011in}}%
\pgfpathlineto{\pgfqpoint{4.601164in}{1.641534in}}%
\pgfpathlineto{\pgfqpoint{4.605673in}{2.013041in}}%
\pgfpathlineto{\pgfqpoint{4.610182in}{2.139233in}}%
\pgfpathlineto{\pgfqpoint{4.614691in}{1.717236in}}%
\pgfpathlineto{\pgfqpoint{4.619200in}{2.059693in}}%
\pgfpathlineto{\pgfqpoint{4.623709in}{1.920768in}}%
\pgfpathlineto{\pgfqpoint{4.628218in}{3.341238in}}%
\pgfpathlineto{\pgfqpoint{4.632727in}{1.688806in}}%
\pgfpathlineto{\pgfqpoint{4.637236in}{1.698577in}}%
\pgfpathlineto{\pgfqpoint{4.641745in}{1.777315in}}%
\pgfpathlineto{\pgfqpoint{4.646255in}{1.789543in}}%
\pgfpathlineto{\pgfqpoint{4.650764in}{1.749260in}}%
\pgfpathlineto{\pgfqpoint{4.655273in}{2.050907in}}%
\pgfpathlineto{\pgfqpoint{4.659782in}{1.850827in}}%
\pgfpathlineto{\pgfqpoint{4.664291in}{2.567617in}}%
\pgfpathlineto{\pgfqpoint{4.668800in}{3.608569in}}%
\pgfpathlineto{\pgfqpoint{4.673309in}{1.888440in}}%
\pgfpathlineto{\pgfqpoint{4.677818in}{2.302975in}}%
\pgfpathlineto{\pgfqpoint{4.682327in}{1.672100in}}%
\pgfpathlineto{\pgfqpoint{4.686836in}{1.868939in}}%
\pgfpathlineto{\pgfqpoint{4.691345in}{1.698887in}}%
\pgfpathlineto{\pgfqpoint{4.695855in}{1.946175in}}%
\pgfpathlineto{\pgfqpoint{4.700364in}{1.781496in}}%
\pgfpathlineto{\pgfqpoint{4.704873in}{1.960678in}}%
\pgfpathlineto{\pgfqpoint{4.709382in}{3.439548in}}%
\pgfpathlineto{\pgfqpoint{4.713891in}{2.132794in}}%
\pgfpathlineto{\pgfqpoint{4.718400in}{2.038032in}}%
\pgfpathlineto{\pgfqpoint{4.722909in}{2.528376in}}%
\pgfpathlineto{\pgfqpoint{4.727418in}{1.969538in}}%
\pgfpathlineto{\pgfqpoint{4.731927in}{2.055888in}}%
\pgfpathlineto{\pgfqpoint{4.736436in}{2.002918in}}%
\pgfpathlineto{\pgfqpoint{4.740945in}{2.068213in}}%
\pgfpathlineto{\pgfqpoint{4.745455in}{2.500878in}}%
\pgfpathlineto{\pgfqpoint{4.749964in}{1.877912in}}%
\pgfpathlineto{\pgfqpoint{4.754473in}{2.433281in}}%
\pgfpathlineto{\pgfqpoint{4.758982in}{1.646731in}}%
\pgfpathlineto{\pgfqpoint{4.763491in}{1.733054in}}%
\pgfpathlineto{\pgfqpoint{4.772509in}{1.680647in}}%
\pgfpathlineto{\pgfqpoint{4.777018in}{1.667363in}}%
\pgfpathlineto{\pgfqpoint{4.781527in}{1.774144in}}%
\pgfpathlineto{\pgfqpoint{4.786036in}{1.667667in}}%
\pgfpathlineto{\pgfqpoint{4.790545in}{1.745306in}}%
\pgfpathlineto{\pgfqpoint{4.795055in}{1.723303in}}%
\pgfpathlineto{\pgfqpoint{4.799564in}{1.684735in}}%
\pgfpathlineto{\pgfqpoint{4.804073in}{2.945610in}}%
\pgfpathlineto{\pgfqpoint{4.808582in}{1.737933in}}%
\pgfpathlineto{\pgfqpoint{4.813091in}{1.719293in}}%
\pgfpathlineto{\pgfqpoint{4.817600in}{1.681052in}}%
\pgfpathlineto{\pgfqpoint{4.822109in}{1.821342in}}%
\pgfpathlineto{\pgfqpoint{4.826618in}{2.879372in}}%
\pgfpathlineto{\pgfqpoint{4.831127in}{1.813601in}}%
\pgfpathlineto{\pgfqpoint{4.835636in}{1.863270in}}%
\pgfpathlineto{\pgfqpoint{4.840145in}{1.877286in}}%
\pgfpathlineto{\pgfqpoint{4.844655in}{2.411906in}}%
\pgfpathlineto{\pgfqpoint{4.849164in}{1.884153in}}%
\pgfpathlineto{\pgfqpoint{4.853673in}{1.890967in}}%
\pgfpathlineto{\pgfqpoint{4.858182in}{1.949545in}}%
\pgfpathlineto{\pgfqpoint{4.862691in}{2.157696in}}%
\pgfpathlineto{\pgfqpoint{4.867200in}{3.013852in}}%
\pgfpathlineto{\pgfqpoint{4.876218in}{1.898273in}}%
\pgfpathlineto{\pgfqpoint{4.880727in}{1.845479in}}%
\pgfpathlineto{\pgfqpoint{4.885236in}{2.191759in}}%
\pgfpathlineto{\pgfqpoint{4.889745in}{1.744436in}}%
\pgfpathlineto{\pgfqpoint{4.894255in}{1.908183in}}%
\pgfpathlineto{\pgfqpoint{4.898764in}{1.858315in}}%
\pgfpathlineto{\pgfqpoint{4.903273in}{1.778085in}}%
\pgfpathlineto{\pgfqpoint{4.907782in}{1.856376in}}%
\pgfpathlineto{\pgfqpoint{4.912291in}{1.877350in}}%
\pgfpathlineto{\pgfqpoint{4.916800in}{2.013279in}}%
\pgfpathlineto{\pgfqpoint{4.921309in}{1.750726in}}%
\pgfpathlineto{\pgfqpoint{4.925818in}{1.771337in}}%
\pgfpathlineto{\pgfqpoint{4.930327in}{1.906369in}}%
\pgfpathlineto{\pgfqpoint{4.934836in}{1.766057in}}%
\pgfpathlineto{\pgfqpoint{4.939345in}{1.874867in}}%
\pgfpathlineto{\pgfqpoint{4.943855in}{1.890230in}}%
\pgfpathlineto{\pgfqpoint{4.948364in}{1.960676in}}%
\pgfpathlineto{\pgfqpoint{4.952873in}{1.785191in}}%
\pgfpathlineto{\pgfqpoint{4.957382in}{1.778345in}}%
\pgfpathlineto{\pgfqpoint{4.966400in}{2.338968in}}%
\pgfpathlineto{\pgfqpoint{4.970909in}{1.815548in}}%
\pgfpathlineto{\pgfqpoint{4.975418in}{1.799647in}}%
\pgfpathlineto{\pgfqpoint{4.979927in}{2.071158in}}%
\pgfpathlineto{\pgfqpoint{4.984436in}{1.807883in}}%
\pgfpathlineto{\pgfqpoint{4.988945in}{1.862487in}}%
\pgfpathlineto{\pgfqpoint{4.993455in}{1.877489in}}%
\pgfpathlineto{\pgfqpoint{4.997964in}{1.787446in}}%
\pgfpathlineto{\pgfqpoint{5.002473in}{1.968166in}}%
\pgfpathlineto{\pgfqpoint{5.006982in}{1.804960in}}%
\pgfpathlineto{\pgfqpoint{5.011491in}{1.880214in}}%
\pgfpathlineto{\pgfqpoint{5.016000in}{1.844083in}}%
\pgfpathlineto{\pgfqpoint{5.020509in}{2.140984in}}%
\pgfpathlineto{\pgfqpoint{5.025018in}{2.212976in}}%
\pgfpathlineto{\pgfqpoint{5.029527in}{1.752961in}}%
\pgfpathlineto{\pgfqpoint{5.034036in}{1.896433in}}%
\pgfpathlineto{\pgfqpoint{5.038545in}{1.913746in}}%
\pgfpathlineto{\pgfqpoint{5.043055in}{1.907368in}}%
\pgfpathlineto{\pgfqpoint{5.047564in}{1.904428in}}%
\pgfpathlineto{\pgfqpoint{5.052073in}{1.859532in}}%
\pgfpathlineto{\pgfqpoint{5.056582in}{2.173552in}}%
\pgfpathlineto{\pgfqpoint{5.061091in}{2.168989in}}%
\pgfpathlineto{\pgfqpoint{5.065600in}{1.928290in}}%
\pgfpathlineto{\pgfqpoint{5.070109in}{1.946721in}}%
\pgfpathlineto{\pgfqpoint{5.074618in}{2.153925in}}%
\pgfpathlineto{\pgfqpoint{5.079127in}{1.933276in}}%
\pgfpathlineto{\pgfqpoint{5.083636in}{2.362108in}}%
\pgfpathlineto{\pgfqpoint{5.088145in}{1.945045in}}%
\pgfpathlineto{\pgfqpoint{5.092655in}{1.916220in}}%
\pgfpathlineto{\pgfqpoint{5.097164in}{2.391229in}}%
\pgfpathlineto{\pgfqpoint{5.101673in}{2.194043in}}%
\pgfpathlineto{\pgfqpoint{5.106182in}{1.833279in}}%
\pgfpathlineto{\pgfqpoint{5.110691in}{1.819269in}}%
\pgfpathlineto{\pgfqpoint{5.115200in}{1.963213in}}%
\pgfpathlineto{\pgfqpoint{5.119709in}{1.868576in}}%
\pgfpathlineto{\pgfqpoint{5.124218in}{1.806522in}}%
\pgfpathlineto{\pgfqpoint{5.128727in}{2.141866in}}%
\pgfpathlineto{\pgfqpoint{5.133236in}{1.811166in}}%
\pgfpathlineto{\pgfqpoint{5.142255in}{1.935939in}}%
\pgfpathlineto{\pgfqpoint{5.146764in}{2.308577in}}%
\pgfpathlineto{\pgfqpoint{5.151273in}{2.278335in}}%
\pgfpathlineto{\pgfqpoint{5.155782in}{3.613591in}}%
\pgfpathlineto{\pgfqpoint{5.160291in}{2.405358in}}%
\pgfpathlineto{\pgfqpoint{5.164800in}{2.377921in}}%
\pgfpathlineto{\pgfqpoint{5.169309in}{2.004409in}}%
\pgfpathlineto{\pgfqpoint{5.173818in}{2.113869in}}%
\pgfpathlineto{\pgfqpoint{5.178327in}{2.253097in}}%
\pgfpathlineto{\pgfqpoint{5.182836in}{2.153815in}}%
\pgfpathlineto{\pgfqpoint{5.187345in}{2.118587in}}%
\pgfpathlineto{\pgfqpoint{5.191855in}{2.265879in}}%
\pgfpathlineto{\pgfqpoint{5.196364in}{2.355082in}}%
\pgfpathlineto{\pgfqpoint{5.200873in}{1.853988in}}%
\pgfpathlineto{\pgfqpoint{5.205382in}{2.021300in}}%
\pgfpathlineto{\pgfqpoint{5.209891in}{2.139705in}}%
\pgfpathlineto{\pgfqpoint{5.214400in}{2.030649in}}%
\pgfpathlineto{\pgfqpoint{5.218909in}{1.858827in}}%
\pgfpathlineto{\pgfqpoint{5.223418in}{2.185181in}}%
\pgfpathlineto{\pgfqpoint{5.227927in}{1.947320in}}%
\pgfpathlineto{\pgfqpoint{5.232436in}{2.096575in}}%
\pgfpathlineto{\pgfqpoint{5.236945in}{1.943032in}}%
\pgfpathlineto{\pgfqpoint{5.245964in}{2.826887in}}%
\pgfpathlineto{\pgfqpoint{5.250473in}{2.454176in}}%
\pgfpathlineto{\pgfqpoint{5.254982in}{1.890986in}}%
\pgfpathlineto{\pgfqpoint{5.259491in}{4.056000in}}%
\pgfpathlineto{\pgfqpoint{5.264000in}{2.439205in}}%
\pgfpathlineto{\pgfqpoint{5.268509in}{2.515106in}}%
\pgfpathlineto{\pgfqpoint{5.273018in}{1.908271in}}%
\pgfpathlineto{\pgfqpoint{5.277527in}{1.922504in}}%
\pgfpathlineto{\pgfqpoint{5.282036in}{2.400830in}}%
\pgfpathlineto{\pgfqpoint{5.286545in}{2.224674in}}%
\pgfpathlineto{\pgfqpoint{5.291055in}{2.459873in}}%
\pgfpathlineto{\pgfqpoint{5.295564in}{1.850299in}}%
\pgfpathlineto{\pgfqpoint{5.300073in}{1.866727in}}%
\pgfpathlineto{\pgfqpoint{5.304582in}{1.916590in}}%
\pgfpathlineto{\pgfqpoint{5.309091in}{2.227165in}}%
\pgfpathlineto{\pgfqpoint{5.313600in}{2.706246in}}%
\pgfpathlineto{\pgfqpoint{5.318109in}{2.052634in}}%
\pgfpathlineto{\pgfqpoint{5.322618in}{2.135386in}}%
\pgfpathlineto{\pgfqpoint{5.327127in}{2.063619in}}%
\pgfpathlineto{\pgfqpoint{5.331636in}{2.019758in}}%
\pgfpathlineto{\pgfqpoint{5.336145in}{2.094070in}}%
\pgfpathlineto{\pgfqpoint{5.340655in}{2.298212in}}%
\pgfpathlineto{\pgfqpoint{5.345164in}{2.274074in}}%
\pgfpathlineto{\pgfqpoint{5.349673in}{2.160422in}}%
\pgfpathlineto{\pgfqpoint{5.354182in}{2.097305in}}%
\pgfpathlineto{\pgfqpoint{5.358691in}{2.859724in}}%
\pgfpathlineto{\pgfqpoint{5.363200in}{1.906501in}}%
\pgfpathlineto{\pgfqpoint{5.367709in}{3.464417in}}%
\pgfpathlineto{\pgfqpoint{5.372218in}{2.190108in}}%
\pgfpathlineto{\pgfqpoint{5.376727in}{1.953229in}}%
\pgfpathlineto{\pgfqpoint{5.381236in}{1.965571in}}%
\pgfpathlineto{\pgfqpoint{5.385745in}{2.114387in}}%
\pgfpathlineto{\pgfqpoint{5.390255in}{1.885125in}}%
\pgfpathlineto{\pgfqpoint{5.394764in}{2.059344in}}%
\pgfpathlineto{\pgfqpoint{5.399273in}{1.940458in}}%
\pgfpathlineto{\pgfqpoint{5.403782in}{2.130857in}}%
\pgfpathlineto{\pgfqpoint{5.408291in}{2.037750in}}%
\pgfpathlineto{\pgfqpoint{5.412800in}{2.044553in}}%
\pgfpathlineto{\pgfqpoint{5.417309in}{2.414599in}}%
\pgfpathlineto{\pgfqpoint{5.421818in}{1.902863in}}%
\pgfpathlineto{\pgfqpoint{5.426327in}{1.951048in}}%
\pgfpathlineto{\pgfqpoint{5.430836in}{2.547924in}}%
\pgfpathlineto{\pgfqpoint{5.435345in}{1.970884in}}%
\pgfpathlineto{\pgfqpoint{5.439855in}{2.056904in}}%
\pgfpathlineto{\pgfqpoint{5.444364in}{1.941854in}}%
\pgfpathlineto{\pgfqpoint{5.448873in}{2.268030in}}%
\pgfpathlineto{\pgfqpoint{5.453382in}{2.222044in}}%
\pgfpathlineto{\pgfqpoint{5.457891in}{2.371110in}}%
\pgfpathlineto{\pgfqpoint{5.462400in}{1.941594in}}%
\pgfpathlineto{\pgfqpoint{5.466909in}{1.986615in}}%
\pgfpathlineto{\pgfqpoint{5.471418in}{2.085893in}}%
\pgfpathlineto{\pgfqpoint{5.475927in}{2.330503in}}%
\pgfpathlineto{\pgfqpoint{5.480436in}{1.946494in}}%
\pgfpathlineto{\pgfqpoint{5.484945in}{2.292637in}}%
\pgfpathlineto{\pgfqpoint{5.489455in}{2.255184in}}%
\pgfpathlineto{\pgfqpoint{5.493964in}{2.176015in}}%
\pgfpathlineto{\pgfqpoint{5.498473in}{2.003874in}}%
\pgfpathlineto{\pgfqpoint{5.502982in}{2.178017in}}%
\pgfpathlineto{\pgfqpoint{5.507491in}{1.972129in}}%
\pgfpathlineto{\pgfqpoint{5.512000in}{1.937322in}}%
\pgfpathlineto{\pgfqpoint{5.516509in}{1.949773in}}%
\pgfpathlineto{\pgfqpoint{5.521018in}{1.918222in}}%
\pgfpathlineto{\pgfqpoint{5.525527in}{1.979878in}}%
\pgfpathlineto{\pgfqpoint{5.530036in}{1.933340in}}%
\pgfpathlineto{\pgfqpoint{5.534545in}{2.461600in}}%
\pgfpathlineto{\pgfqpoint{5.534545in}{2.461600in}}%
\pgfusepath{stroke}%
\end{pgfscope}%
\begin{pgfscope}%
\pgfsetrectcap%
\pgfsetmiterjoin%
\pgfsetlinewidth{0.803000pt}%
\definecolor{currentstroke}{rgb}{0.000000,0.000000,0.000000}%
\pgfsetstrokecolor{currentstroke}%
\pgfsetdash{}{0pt}%
\pgfpathmoveto{\pgfqpoint{0.800000in}{0.528000in}}%
\pgfpathlineto{\pgfqpoint{0.800000in}{4.224000in}}%
\pgfusepath{stroke}%
\end{pgfscope}%
\begin{pgfscope}%
\pgfsetrectcap%
\pgfsetmiterjoin%
\pgfsetlinewidth{0.803000pt}%
\definecolor{currentstroke}{rgb}{0.000000,0.000000,0.000000}%
\pgfsetstrokecolor{currentstroke}%
\pgfsetdash{}{0pt}%
\pgfpathmoveto{\pgfqpoint{5.760000in}{0.528000in}}%
\pgfpathlineto{\pgfqpoint{5.760000in}{4.224000in}}%
\pgfusepath{stroke}%
\end{pgfscope}%
\begin{pgfscope}%
\pgfsetrectcap%
\pgfsetmiterjoin%
\pgfsetlinewidth{0.803000pt}%
\definecolor{currentstroke}{rgb}{0.000000,0.000000,0.000000}%
\pgfsetstrokecolor{currentstroke}%
\pgfsetdash{}{0pt}%
\pgfpathmoveto{\pgfqpoint{0.800000in}{0.528000in}}%
\pgfpathlineto{\pgfqpoint{5.760000in}{0.528000in}}%
\pgfusepath{stroke}%
\end{pgfscope}%
\begin{pgfscope}%
\pgfsetrectcap%
\pgfsetmiterjoin%
\pgfsetlinewidth{0.803000pt}%
\definecolor{currentstroke}{rgb}{0.000000,0.000000,0.000000}%
\pgfsetstrokecolor{currentstroke}%
\pgfsetdash{}{0pt}%
\pgfpathmoveto{\pgfqpoint{0.800000in}{4.224000in}}%
\pgfpathlineto{\pgfqpoint{5.760000in}{4.224000in}}%
\pgfusepath{stroke}%
\end{pgfscope}%
\begin{pgfscope}%
\definecolor{textcolor}{rgb}{0.000000,0.000000,0.000000}%
\pgfsetstrokecolor{textcolor}%
\pgfsetfillcolor{textcolor}%
\pgftext[x=3.280000in,y=4.307333in,,base]{\color{textcolor}\ttfamily\fontsize{12.000000}{14.400000}\selectfont Quick Sort Time vs Input size}%
\end{pgfscope}%
\begin{pgfscope}%
\pgfsetbuttcap%
\pgfsetmiterjoin%
\definecolor{currentfill}{rgb}{1.000000,1.000000,1.000000}%
\pgfsetfillcolor{currentfill}%
\pgfsetfillopacity{0.800000}%
\pgfsetlinewidth{1.003750pt}%
\definecolor{currentstroke}{rgb}{0.800000,0.800000,0.800000}%
\pgfsetstrokecolor{currentstroke}%
\pgfsetstrokeopacity{0.800000}%
\pgfsetdash{}{0pt}%
\pgfpathmoveto{\pgfqpoint{0.897222in}{3.908286in}}%
\pgfpathlineto{\pgfqpoint{1.759758in}{3.908286in}}%
\pgfpathquadraticcurveto{\pgfqpoint{1.787535in}{3.908286in}}{\pgfqpoint{1.787535in}{3.936063in}}%
\pgfpathlineto{\pgfqpoint{1.787535in}{4.126778in}}%
\pgfpathquadraticcurveto{\pgfqpoint{1.787535in}{4.154556in}}{\pgfqpoint{1.759758in}{4.154556in}}%
\pgfpathlineto{\pgfqpoint{0.897222in}{4.154556in}}%
\pgfpathquadraticcurveto{\pgfqpoint{0.869444in}{4.154556in}}{\pgfqpoint{0.869444in}{4.126778in}}%
\pgfpathlineto{\pgfqpoint{0.869444in}{3.936063in}}%
\pgfpathquadraticcurveto{\pgfqpoint{0.869444in}{3.908286in}}{\pgfqpoint{0.897222in}{3.908286in}}%
\pgfpathlineto{\pgfqpoint{0.897222in}{3.908286in}}%
\pgfpathclose%
\pgfusepath{stroke,fill}%
\end{pgfscope}%
\begin{pgfscope}%
\pgfsetrectcap%
\pgfsetroundjoin%
\pgfsetlinewidth{1.505625pt}%
\definecolor{currentstroke}{rgb}{0.000000,1.000000,0.498039}%
\pgfsetstrokecolor{currentstroke}%
\pgfsetdash{}{0pt}%
\pgfpathmoveto{\pgfqpoint{0.925000in}{4.041342in}}%
\pgfpathlineto{\pgfqpoint{1.063889in}{4.041342in}}%
\pgfpathlineto{\pgfqpoint{1.202778in}{4.041342in}}%
\pgfusepath{stroke}%
\end{pgfscope}%
\begin{pgfscope}%
\definecolor{textcolor}{rgb}{0.000000,0.000000,0.000000}%
\pgfsetstrokecolor{textcolor}%
\pgfsetfillcolor{textcolor}%
\pgftext[x=1.313889in,y=3.992731in,left,base]{\color{textcolor}\ttfamily\fontsize{10.000000}{12.000000}\selectfont Quick}%
\end{pgfscope}%
\end{pgfpicture}%
\makeatother%
\endgroup%

%% Creator: Matplotlib, PGF backend
%%
%% To include the figure in your LaTeX document, write
%%   \input{<filename>.pgf}
%%
%% Make sure the required packages are loaded in your preamble
%%   \usepackage{pgf}
%%
%% Also ensure that all the required font packages are loaded; for instance,
%% the lmodern package is sometimes necessary when using math font.
%%   \usepackage{lmodern}
%%
%% Figures using additional raster images can only be included by \input if
%% they are in the same directory as the main LaTeX file. For loading figures
%% from other directories you can use the `import` package
%%   \usepackage{import}
%%
%% and then include the figures with
%%   \import{<path to file>}{<filename>.pgf}
%%
%% Matplotlib used the following preamble
%%   \usepackage{fontspec}
%%   \setmainfont{DejaVuSerif.ttf}[Path=\detokenize{/home/dbk/.local/lib/python3.10/site-packages/matplotlib/mpl-data/fonts/ttf/}]
%%   \setsansfont{DejaVuSans.ttf}[Path=\detokenize{/home/dbk/.local/lib/python3.10/site-packages/matplotlib/mpl-data/fonts/ttf/}]
%%   \setmonofont{DejaVuSansMono.ttf}[Path=\detokenize{/home/dbk/.local/lib/python3.10/site-packages/matplotlib/mpl-data/fonts/ttf/}]
%%
\begingroup%
\makeatletter%
\begin{pgfpicture}%
\pgfpathrectangle{\pgfpointorigin}{\pgfqpoint{6.400000in}{4.800000in}}%
\pgfusepath{use as bounding box, clip}%
\begin{pgfscope}%
\pgfsetbuttcap%
\pgfsetmiterjoin%
\definecolor{currentfill}{rgb}{1.000000,1.000000,1.000000}%
\pgfsetfillcolor{currentfill}%
\pgfsetlinewidth{0.000000pt}%
\definecolor{currentstroke}{rgb}{1.000000,1.000000,1.000000}%
\pgfsetstrokecolor{currentstroke}%
\pgfsetdash{}{0pt}%
\pgfpathmoveto{\pgfqpoint{0.000000in}{0.000000in}}%
\pgfpathlineto{\pgfqpoint{6.400000in}{0.000000in}}%
\pgfpathlineto{\pgfqpoint{6.400000in}{4.800000in}}%
\pgfpathlineto{\pgfqpoint{0.000000in}{4.800000in}}%
\pgfpathlineto{\pgfqpoint{0.000000in}{0.000000in}}%
\pgfpathclose%
\pgfusepath{fill}%
\end{pgfscope}%
\begin{pgfscope}%
\pgfsetbuttcap%
\pgfsetmiterjoin%
\definecolor{currentfill}{rgb}{1.000000,1.000000,1.000000}%
\pgfsetfillcolor{currentfill}%
\pgfsetlinewidth{0.000000pt}%
\definecolor{currentstroke}{rgb}{0.000000,0.000000,0.000000}%
\pgfsetstrokecolor{currentstroke}%
\pgfsetstrokeopacity{0.000000}%
\pgfsetdash{}{0pt}%
\pgfpathmoveto{\pgfqpoint{0.800000in}{0.528000in}}%
\pgfpathlineto{\pgfqpoint{5.760000in}{0.528000in}}%
\pgfpathlineto{\pgfqpoint{5.760000in}{4.224000in}}%
\pgfpathlineto{\pgfqpoint{0.800000in}{4.224000in}}%
\pgfpathlineto{\pgfqpoint{0.800000in}{0.528000in}}%
\pgfpathclose%
\pgfusepath{fill}%
\end{pgfscope}%
\begin{pgfscope}%
\pgfsetbuttcap%
\pgfsetroundjoin%
\definecolor{currentfill}{rgb}{0.000000,0.000000,0.000000}%
\pgfsetfillcolor{currentfill}%
\pgfsetlinewidth{0.803000pt}%
\definecolor{currentstroke}{rgb}{0.000000,0.000000,0.000000}%
\pgfsetstrokecolor{currentstroke}%
\pgfsetdash{}{0pt}%
\pgfsys@defobject{currentmarker}{\pgfqpoint{0.000000in}{-0.048611in}}{\pgfqpoint{0.000000in}{0.000000in}}{%
\pgfpathmoveto{\pgfqpoint{0.000000in}{0.000000in}}%
\pgfpathlineto{\pgfqpoint{0.000000in}{-0.048611in}}%
\pgfusepath{stroke,fill}%
}%
\begin{pgfscope}%
\pgfsys@transformshift{1.020945in}{0.528000in}%
\pgfsys@useobject{currentmarker}{}%
\end{pgfscope}%
\end{pgfscope}%
\begin{pgfscope}%
\definecolor{textcolor}{rgb}{0.000000,0.000000,0.000000}%
\pgfsetstrokecolor{textcolor}%
\pgfsetfillcolor{textcolor}%
\pgftext[x=1.020945in,y=0.430778in,,top]{\color{textcolor}\ttfamily\fontsize{10.000000}{12.000000}\selectfont 0}%
\end{pgfscope}%
\begin{pgfscope}%
\pgfsetbuttcap%
\pgfsetroundjoin%
\definecolor{currentfill}{rgb}{0.000000,0.000000,0.000000}%
\pgfsetfillcolor{currentfill}%
\pgfsetlinewidth{0.803000pt}%
\definecolor{currentstroke}{rgb}{0.000000,0.000000,0.000000}%
\pgfsetstrokecolor{currentstroke}%
\pgfsetdash{}{0pt}%
\pgfsys@defobject{currentmarker}{\pgfqpoint{0.000000in}{-0.048611in}}{\pgfqpoint{0.000000in}{0.000000in}}{%
\pgfpathmoveto{\pgfqpoint{0.000000in}{0.000000in}}%
\pgfpathlineto{\pgfqpoint{0.000000in}{-0.048611in}}%
\pgfusepath{stroke,fill}%
}%
\begin{pgfscope}%
\pgfsys@transformshift{1.922764in}{0.528000in}%
\pgfsys@useobject{currentmarker}{}%
\end{pgfscope}%
\end{pgfscope}%
\begin{pgfscope}%
\definecolor{textcolor}{rgb}{0.000000,0.000000,0.000000}%
\pgfsetstrokecolor{textcolor}%
\pgfsetfillcolor{textcolor}%
\pgftext[x=1.922764in,y=0.430778in,,top]{\color{textcolor}\ttfamily\fontsize{10.000000}{12.000000}\selectfont 200}%
\end{pgfscope}%
\begin{pgfscope}%
\pgfsetbuttcap%
\pgfsetroundjoin%
\definecolor{currentfill}{rgb}{0.000000,0.000000,0.000000}%
\pgfsetfillcolor{currentfill}%
\pgfsetlinewidth{0.803000pt}%
\definecolor{currentstroke}{rgb}{0.000000,0.000000,0.000000}%
\pgfsetstrokecolor{currentstroke}%
\pgfsetdash{}{0pt}%
\pgfsys@defobject{currentmarker}{\pgfqpoint{0.000000in}{-0.048611in}}{\pgfqpoint{0.000000in}{0.000000in}}{%
\pgfpathmoveto{\pgfqpoint{0.000000in}{0.000000in}}%
\pgfpathlineto{\pgfqpoint{0.000000in}{-0.048611in}}%
\pgfusepath{stroke,fill}%
}%
\begin{pgfscope}%
\pgfsys@transformshift{2.824582in}{0.528000in}%
\pgfsys@useobject{currentmarker}{}%
\end{pgfscope}%
\end{pgfscope}%
\begin{pgfscope}%
\definecolor{textcolor}{rgb}{0.000000,0.000000,0.000000}%
\pgfsetstrokecolor{textcolor}%
\pgfsetfillcolor{textcolor}%
\pgftext[x=2.824582in,y=0.430778in,,top]{\color{textcolor}\ttfamily\fontsize{10.000000}{12.000000}\selectfont 400}%
\end{pgfscope}%
\begin{pgfscope}%
\pgfsetbuttcap%
\pgfsetroundjoin%
\definecolor{currentfill}{rgb}{0.000000,0.000000,0.000000}%
\pgfsetfillcolor{currentfill}%
\pgfsetlinewidth{0.803000pt}%
\definecolor{currentstroke}{rgb}{0.000000,0.000000,0.000000}%
\pgfsetstrokecolor{currentstroke}%
\pgfsetdash{}{0pt}%
\pgfsys@defobject{currentmarker}{\pgfqpoint{0.000000in}{-0.048611in}}{\pgfqpoint{0.000000in}{0.000000in}}{%
\pgfpathmoveto{\pgfqpoint{0.000000in}{0.000000in}}%
\pgfpathlineto{\pgfqpoint{0.000000in}{-0.048611in}}%
\pgfusepath{stroke,fill}%
}%
\begin{pgfscope}%
\pgfsys@transformshift{3.726400in}{0.528000in}%
\pgfsys@useobject{currentmarker}{}%
\end{pgfscope}%
\end{pgfscope}%
\begin{pgfscope}%
\definecolor{textcolor}{rgb}{0.000000,0.000000,0.000000}%
\pgfsetstrokecolor{textcolor}%
\pgfsetfillcolor{textcolor}%
\pgftext[x=3.726400in,y=0.430778in,,top]{\color{textcolor}\ttfamily\fontsize{10.000000}{12.000000}\selectfont 600}%
\end{pgfscope}%
\begin{pgfscope}%
\pgfsetbuttcap%
\pgfsetroundjoin%
\definecolor{currentfill}{rgb}{0.000000,0.000000,0.000000}%
\pgfsetfillcolor{currentfill}%
\pgfsetlinewidth{0.803000pt}%
\definecolor{currentstroke}{rgb}{0.000000,0.000000,0.000000}%
\pgfsetstrokecolor{currentstroke}%
\pgfsetdash{}{0pt}%
\pgfsys@defobject{currentmarker}{\pgfqpoint{0.000000in}{-0.048611in}}{\pgfqpoint{0.000000in}{0.000000in}}{%
\pgfpathmoveto{\pgfqpoint{0.000000in}{0.000000in}}%
\pgfpathlineto{\pgfqpoint{0.000000in}{-0.048611in}}%
\pgfusepath{stroke,fill}%
}%
\begin{pgfscope}%
\pgfsys@transformshift{4.628218in}{0.528000in}%
\pgfsys@useobject{currentmarker}{}%
\end{pgfscope}%
\end{pgfscope}%
\begin{pgfscope}%
\definecolor{textcolor}{rgb}{0.000000,0.000000,0.000000}%
\pgfsetstrokecolor{textcolor}%
\pgfsetfillcolor{textcolor}%
\pgftext[x=4.628218in,y=0.430778in,,top]{\color{textcolor}\ttfamily\fontsize{10.000000}{12.000000}\selectfont 800}%
\end{pgfscope}%
\begin{pgfscope}%
\pgfsetbuttcap%
\pgfsetroundjoin%
\definecolor{currentfill}{rgb}{0.000000,0.000000,0.000000}%
\pgfsetfillcolor{currentfill}%
\pgfsetlinewidth{0.803000pt}%
\definecolor{currentstroke}{rgb}{0.000000,0.000000,0.000000}%
\pgfsetstrokecolor{currentstroke}%
\pgfsetdash{}{0pt}%
\pgfsys@defobject{currentmarker}{\pgfqpoint{0.000000in}{-0.048611in}}{\pgfqpoint{0.000000in}{0.000000in}}{%
\pgfpathmoveto{\pgfqpoint{0.000000in}{0.000000in}}%
\pgfpathlineto{\pgfqpoint{0.000000in}{-0.048611in}}%
\pgfusepath{stroke,fill}%
}%
\begin{pgfscope}%
\pgfsys@transformshift{5.530036in}{0.528000in}%
\pgfsys@useobject{currentmarker}{}%
\end{pgfscope}%
\end{pgfscope}%
\begin{pgfscope}%
\definecolor{textcolor}{rgb}{0.000000,0.000000,0.000000}%
\pgfsetstrokecolor{textcolor}%
\pgfsetfillcolor{textcolor}%
\pgftext[x=5.530036in,y=0.430778in,,top]{\color{textcolor}\ttfamily\fontsize{10.000000}{12.000000}\selectfont 1000}%
\end{pgfscope}%
\begin{pgfscope}%
\definecolor{textcolor}{rgb}{0.000000,0.000000,0.000000}%
\pgfsetstrokecolor{textcolor}%
\pgfsetfillcolor{textcolor}%
\pgftext[x=3.280000in,y=0.240063in,,top]{\color{textcolor}\ttfamily\fontsize{10.000000}{12.000000}\selectfont Size of Array}%
\end{pgfscope}%
\begin{pgfscope}%
\pgfsetbuttcap%
\pgfsetroundjoin%
\definecolor{currentfill}{rgb}{0.000000,0.000000,0.000000}%
\pgfsetfillcolor{currentfill}%
\pgfsetlinewidth{0.803000pt}%
\definecolor{currentstroke}{rgb}{0.000000,0.000000,0.000000}%
\pgfsetstrokecolor{currentstroke}%
\pgfsetdash{}{0pt}%
\pgfsys@defobject{currentmarker}{\pgfqpoint{-0.048611in}{0.000000in}}{\pgfqpoint{-0.000000in}{0.000000in}}{%
\pgfpathmoveto{\pgfqpoint{-0.000000in}{0.000000in}}%
\pgfpathlineto{\pgfqpoint{-0.048611in}{0.000000in}}%
\pgfusepath{stroke,fill}%
}%
\begin{pgfscope}%
\pgfsys@transformshift{0.800000in}{1.028141in}%
\pgfsys@useobject{currentmarker}{}%
\end{pgfscope}%
\end{pgfscope}%
\begin{pgfscope}%
\definecolor{textcolor}{rgb}{0.000000,0.000000,0.000000}%
\pgfsetstrokecolor{textcolor}%
\pgfsetfillcolor{textcolor}%
\pgftext[x=0.368305in, y=0.975007in, left, base]{\color{textcolor}\ttfamily\fontsize{10.000000}{12.000000}\selectfont 2000}%
\end{pgfscope}%
\begin{pgfscope}%
\pgfsetbuttcap%
\pgfsetroundjoin%
\definecolor{currentfill}{rgb}{0.000000,0.000000,0.000000}%
\pgfsetfillcolor{currentfill}%
\pgfsetlinewidth{0.803000pt}%
\definecolor{currentstroke}{rgb}{0.000000,0.000000,0.000000}%
\pgfsetstrokecolor{currentstroke}%
\pgfsetdash{}{0pt}%
\pgfsys@defobject{currentmarker}{\pgfqpoint{-0.048611in}{0.000000in}}{\pgfqpoint{-0.000000in}{0.000000in}}{%
\pgfpathmoveto{\pgfqpoint{-0.000000in}{0.000000in}}%
\pgfpathlineto{\pgfqpoint{-0.048611in}{0.000000in}}%
\pgfusepath{stroke,fill}%
}%
\begin{pgfscope}%
\pgfsys@transformshift{0.800000in}{1.717231in}%
\pgfsys@useobject{currentmarker}{}%
\end{pgfscope}%
\end{pgfscope}%
\begin{pgfscope}%
\definecolor{textcolor}{rgb}{0.000000,0.000000,0.000000}%
\pgfsetstrokecolor{textcolor}%
\pgfsetfillcolor{textcolor}%
\pgftext[x=0.368305in, y=1.664096in, left, base]{\color{textcolor}\ttfamily\fontsize{10.000000}{12.000000}\selectfont 4000}%
\end{pgfscope}%
\begin{pgfscope}%
\pgfsetbuttcap%
\pgfsetroundjoin%
\definecolor{currentfill}{rgb}{0.000000,0.000000,0.000000}%
\pgfsetfillcolor{currentfill}%
\pgfsetlinewidth{0.803000pt}%
\definecolor{currentstroke}{rgb}{0.000000,0.000000,0.000000}%
\pgfsetstrokecolor{currentstroke}%
\pgfsetdash{}{0pt}%
\pgfsys@defobject{currentmarker}{\pgfqpoint{-0.048611in}{0.000000in}}{\pgfqpoint{-0.000000in}{0.000000in}}{%
\pgfpathmoveto{\pgfqpoint{-0.000000in}{0.000000in}}%
\pgfpathlineto{\pgfqpoint{-0.048611in}{0.000000in}}%
\pgfusepath{stroke,fill}%
}%
\begin{pgfscope}%
\pgfsys@transformshift{0.800000in}{2.406320in}%
\pgfsys@useobject{currentmarker}{}%
\end{pgfscope}%
\end{pgfscope}%
\begin{pgfscope}%
\definecolor{textcolor}{rgb}{0.000000,0.000000,0.000000}%
\pgfsetstrokecolor{textcolor}%
\pgfsetfillcolor{textcolor}%
\pgftext[x=0.368305in, y=2.353185in, left, base]{\color{textcolor}\ttfamily\fontsize{10.000000}{12.000000}\selectfont 6000}%
\end{pgfscope}%
\begin{pgfscope}%
\pgfsetbuttcap%
\pgfsetroundjoin%
\definecolor{currentfill}{rgb}{0.000000,0.000000,0.000000}%
\pgfsetfillcolor{currentfill}%
\pgfsetlinewidth{0.803000pt}%
\definecolor{currentstroke}{rgb}{0.000000,0.000000,0.000000}%
\pgfsetstrokecolor{currentstroke}%
\pgfsetdash{}{0pt}%
\pgfsys@defobject{currentmarker}{\pgfqpoint{-0.048611in}{0.000000in}}{\pgfqpoint{-0.000000in}{0.000000in}}{%
\pgfpathmoveto{\pgfqpoint{-0.000000in}{0.000000in}}%
\pgfpathlineto{\pgfqpoint{-0.048611in}{0.000000in}}%
\pgfusepath{stroke,fill}%
}%
\begin{pgfscope}%
\pgfsys@transformshift{0.800000in}{3.095409in}%
\pgfsys@useobject{currentmarker}{}%
\end{pgfscope}%
\end{pgfscope}%
\begin{pgfscope}%
\definecolor{textcolor}{rgb}{0.000000,0.000000,0.000000}%
\pgfsetstrokecolor{textcolor}%
\pgfsetfillcolor{textcolor}%
\pgftext[x=0.368305in, y=3.042275in, left, base]{\color{textcolor}\ttfamily\fontsize{10.000000}{12.000000}\selectfont 8000}%
\end{pgfscope}%
\begin{pgfscope}%
\pgfsetbuttcap%
\pgfsetroundjoin%
\definecolor{currentfill}{rgb}{0.000000,0.000000,0.000000}%
\pgfsetfillcolor{currentfill}%
\pgfsetlinewidth{0.803000pt}%
\definecolor{currentstroke}{rgb}{0.000000,0.000000,0.000000}%
\pgfsetstrokecolor{currentstroke}%
\pgfsetdash{}{0pt}%
\pgfsys@defobject{currentmarker}{\pgfqpoint{-0.048611in}{0.000000in}}{\pgfqpoint{-0.000000in}{0.000000in}}{%
\pgfpathmoveto{\pgfqpoint{-0.000000in}{0.000000in}}%
\pgfpathlineto{\pgfqpoint{-0.048611in}{0.000000in}}%
\pgfusepath{stroke,fill}%
}%
\begin{pgfscope}%
\pgfsys@transformshift{0.800000in}{3.784499in}%
\pgfsys@useobject{currentmarker}{}%
\end{pgfscope}%
\end{pgfscope}%
\begin{pgfscope}%
\definecolor{textcolor}{rgb}{0.000000,0.000000,0.000000}%
\pgfsetstrokecolor{textcolor}%
\pgfsetfillcolor{textcolor}%
\pgftext[x=0.284687in, y=3.731364in, left, base]{\color{textcolor}\ttfamily\fontsize{10.000000}{12.000000}\selectfont 10000}%
\end{pgfscope}%
\begin{pgfscope}%
\definecolor{textcolor}{rgb}{0.000000,0.000000,0.000000}%
\pgfsetstrokecolor{textcolor}%
\pgfsetfillcolor{textcolor}%
\pgftext[x=0.229131in,y=2.376000in,,bottom,rotate=90.000000]{\color{textcolor}\ttfamily\fontsize{10.000000}{12.000000}\selectfont Memory}%
\end{pgfscope}%
\begin{pgfscope}%
\pgfpathrectangle{\pgfqpoint{0.800000in}{0.528000in}}{\pgfqpoint{4.960000in}{3.696000in}}%
\pgfusepath{clip}%
\pgfsetrectcap%
\pgfsetroundjoin%
\pgfsetlinewidth{1.505625pt}%
\definecolor{currentstroke}{rgb}{0.000000,1.000000,0.498039}%
\pgfsetstrokecolor{currentstroke}%
\pgfsetdash{}{0pt}%
\pgfpathmoveto{\pgfqpoint{1.025455in}{1.773736in}}%
\pgfpathlineto{\pgfqpoint{1.029964in}{0.696000in}}%
\pgfpathlineto{\pgfqpoint{1.052509in}{0.709782in}}%
\pgfpathlineto{\pgfqpoint{1.061527in}{0.869651in}}%
\pgfpathlineto{\pgfqpoint{1.066036in}{0.727698in}}%
\pgfpathlineto{\pgfqpoint{1.070545in}{0.720807in}}%
\pgfpathlineto{\pgfqpoint{1.079564in}{0.726320in}}%
\pgfpathlineto{\pgfqpoint{1.084073in}{0.893080in}}%
\pgfpathlineto{\pgfqpoint{1.088582in}{0.888600in}}%
\pgfpathlineto{\pgfqpoint{1.093091in}{0.744236in}}%
\pgfpathlineto{\pgfqpoint{1.102109in}{0.749749in}}%
\pgfpathlineto{\pgfqpoint{1.106618in}{1.369929in}}%
\pgfpathlineto{\pgfqpoint{1.111127in}{0.755262in}}%
\pgfpathlineto{\pgfqpoint{1.354618in}{0.904105in}}%
\pgfpathlineto{\pgfqpoint{1.359127in}{1.061217in}}%
\pgfpathlineto{\pgfqpoint{1.363636in}{0.909618in}}%
\pgfpathlineto{\pgfqpoint{1.471855in}{0.975770in}}%
\pgfpathlineto{\pgfqpoint{1.476364in}{0.988174in}}%
\pgfpathlineto{\pgfqpoint{1.480873in}{0.990930in}}%
\pgfpathlineto{\pgfqpoint{1.485382in}{0.984039in}}%
\pgfpathlineto{\pgfqpoint{1.489891in}{0.986796in}}%
\pgfpathlineto{\pgfqpoint{1.494400in}{0.999199in}}%
\pgfpathlineto{\pgfqpoint{1.498909in}{0.992308in}}%
\pgfpathlineto{\pgfqpoint{1.503418in}{1.004712in}}%
\pgfpathlineto{\pgfqpoint{1.566545in}{1.043301in}}%
\pgfpathlineto{\pgfqpoint{1.571055in}{1.200413in}}%
\pgfpathlineto{\pgfqpoint{1.575564in}{1.048814in}}%
\pgfpathlineto{\pgfqpoint{1.616145in}{1.073621in}}%
\pgfpathlineto{\pgfqpoint{1.620655in}{1.385089in}}%
\pgfpathlineto{\pgfqpoint{1.625164in}{1.079134in}}%
\pgfpathlineto{\pgfqpoint{1.733382in}{1.145286in}}%
\pgfpathlineto{\pgfqpoint{1.737891in}{1.167337in}}%
\pgfpathlineto{\pgfqpoint{1.755927in}{1.216952in}}%
\pgfpathlineto{\pgfqpoint{1.760436in}{1.248650in}}%
\pgfpathlineto{\pgfqpoint{1.764945in}{1.251406in}}%
\pgfpathlineto{\pgfqpoint{1.769455in}{1.234868in}}%
\pgfpathlineto{\pgfqpoint{1.778473in}{1.259675in}}%
\pgfpathlineto{\pgfqpoint{1.782982in}{1.291373in}}%
\pgfpathlineto{\pgfqpoint{1.787491in}{1.294130in}}%
\pgfpathlineto{\pgfqpoint{1.792000in}{1.306533in}}%
\pgfpathlineto{\pgfqpoint{1.796509in}{1.289995in}}%
\pgfpathlineto{\pgfqpoint{1.801018in}{1.312046in}}%
\pgfpathlineto{\pgfqpoint{1.805527in}{1.285861in}}%
\pgfpathlineto{\pgfqpoint{1.810036in}{1.298264in}}%
\pgfpathlineto{\pgfqpoint{1.814545in}{1.301021in}}%
\pgfpathlineto{\pgfqpoint{1.828073in}{1.338231in}}%
\pgfpathlineto{\pgfqpoint{1.832582in}{1.321693in}}%
\pgfpathlineto{\pgfqpoint{1.837091in}{1.353391in}}%
\pgfpathlineto{\pgfqpoint{1.841600in}{1.346500in}}%
\pgfpathlineto{\pgfqpoint{1.846109in}{1.358904in}}%
\pgfpathlineto{\pgfqpoint{1.850618in}{1.361660in}}%
\pgfpathlineto{\pgfqpoint{1.855127in}{1.345122in}}%
\pgfpathlineto{\pgfqpoint{1.859636in}{1.357526in}}%
\pgfpathlineto{\pgfqpoint{1.868655in}{1.363039in}}%
\pgfpathlineto{\pgfqpoint{1.873164in}{1.356148in}}%
\pgfpathlineto{\pgfqpoint{1.877673in}{1.378199in}}%
\pgfpathlineto{\pgfqpoint{1.886691in}{1.403006in}}%
\pgfpathlineto{\pgfqpoint{1.891200in}{1.425057in}}%
\pgfpathlineto{\pgfqpoint{1.895709in}{1.389224in}}%
\pgfpathlineto{\pgfqpoint{1.904727in}{1.394737in}}%
\pgfpathlineto{\pgfqpoint{1.909236in}{1.426435in}}%
\pgfpathlineto{\pgfqpoint{1.918255in}{1.412653in}}%
\pgfpathlineto{\pgfqpoint{1.927273in}{1.476049in}}%
\pgfpathlineto{\pgfqpoint{1.931782in}{1.478806in}}%
\pgfpathlineto{\pgfqpoint{1.936291in}{1.442973in}}%
\pgfpathlineto{\pgfqpoint{1.940800in}{1.416788in}}%
\pgfpathlineto{\pgfqpoint{1.945309in}{1.438838in}}%
\pgfpathlineto{\pgfqpoint{1.949818in}{1.489831in}}%
\pgfpathlineto{\pgfqpoint{1.954327in}{1.463646in}}%
\pgfpathlineto{\pgfqpoint{1.958836in}{1.476049in}}%
\pgfpathlineto{\pgfqpoint{1.963345in}{1.517395in}}%
\pgfpathlineto{\pgfqpoint{1.967855in}{1.452620in}}%
\pgfpathlineto{\pgfqpoint{1.972364in}{1.484318in}}%
\pgfpathlineto{\pgfqpoint{1.976873in}{1.467780in}}%
\pgfpathlineto{\pgfqpoint{1.981382in}{1.528420in}}%
\pgfpathlineto{\pgfqpoint{1.985891in}{1.511882in}}%
\pgfpathlineto{\pgfqpoint{1.990400in}{1.466402in}}%
\pgfpathlineto{\pgfqpoint{1.994909in}{1.478806in}}%
\pgfpathlineto{\pgfqpoint{1.999418in}{1.481562in}}%
\pgfpathlineto{\pgfqpoint{2.003927in}{1.551849in}}%
\pgfpathlineto{\pgfqpoint{2.008436in}{1.516016in}}%
\pgfpathlineto{\pgfqpoint{2.012945in}{1.509126in}}%
\pgfpathlineto{\pgfqpoint{2.017455in}{1.531176in}}%
\pgfpathlineto{\pgfqpoint{2.021964in}{1.495344in}}%
\pgfpathlineto{\pgfqpoint{2.026473in}{1.517395in}}%
\pgfpathlineto{\pgfqpoint{2.030982in}{1.549093in}}%
\pgfpathlineto{\pgfqpoint{2.035491in}{1.522907in}}%
\pgfpathlineto{\pgfqpoint{2.040000in}{1.554605in}}%
\pgfpathlineto{\pgfqpoint{2.044509in}{1.538067in}}%
\pgfpathlineto{\pgfqpoint{2.049018in}{1.540824in}}%
\pgfpathlineto{\pgfqpoint{2.053527in}{1.553227in}}%
\pgfpathlineto{\pgfqpoint{2.058036in}{1.517395in}}%
\pgfpathlineto{\pgfqpoint{2.071564in}{1.583547in}}%
\pgfpathlineto{\pgfqpoint{2.076073in}{1.567009in}}%
\pgfpathlineto{\pgfqpoint{2.080582in}{1.579413in}}%
\pgfpathlineto{\pgfqpoint{2.085091in}{1.572522in}}%
\pgfpathlineto{\pgfqpoint{2.089600in}{1.594573in}}%
\pgfpathlineto{\pgfqpoint{2.094109in}{1.549093in}}%
\pgfpathlineto{\pgfqpoint{2.098618in}{1.590438in}}%
\pgfpathlineto{\pgfqpoint{2.103127in}{1.564253in}}%
\pgfpathlineto{\pgfqpoint{2.107636in}{1.557362in}}%
\pgfpathlineto{\pgfqpoint{2.112145in}{1.579413in}}%
\pgfpathlineto{\pgfqpoint{2.116655in}{1.572522in}}%
\pgfpathlineto{\pgfqpoint{2.121164in}{1.604220in}}%
\pgfpathlineto{\pgfqpoint{2.125673in}{1.568387in}}%
\pgfpathlineto{\pgfqpoint{2.130182in}{1.561496in}}%
\pgfpathlineto{\pgfqpoint{2.134691in}{1.622136in}}%
\pgfpathlineto{\pgfqpoint{2.139200in}{1.605598in}}%
\pgfpathlineto{\pgfqpoint{2.143709in}{1.598707in}}%
\pgfpathlineto{\pgfqpoint{2.148218in}{1.601463in}}%
\pgfpathlineto{\pgfqpoint{2.152727in}{1.652456in}}%
\pgfpathlineto{\pgfqpoint{2.157236in}{1.626271in}}%
\pgfpathlineto{\pgfqpoint{2.161745in}{1.715852in}}%
\pgfpathlineto{\pgfqpoint{2.166255in}{1.612489in}}%
\pgfpathlineto{\pgfqpoint{2.170764in}{1.692423in}}%
\pgfpathlineto{\pgfqpoint{2.175273in}{1.627649in}}%
\pgfpathlineto{\pgfqpoint{2.179782in}{1.649700in}}%
\pgfpathlineto{\pgfqpoint{2.184291in}{1.623514in}}%
\pgfpathlineto{\pgfqpoint{2.188800in}{1.713096in}}%
\pgfpathlineto{\pgfqpoint{2.193309in}{1.657969in}}%
\pgfpathlineto{\pgfqpoint{2.197818in}{1.660725in}}%
\pgfpathlineto{\pgfqpoint{2.202327in}{1.673129in}}%
\pgfpathlineto{\pgfqpoint{2.206836in}{1.666238in}}%
\pgfpathlineto{\pgfqpoint{2.211345in}{1.649700in}}%
\pgfpathlineto{\pgfqpoint{2.215855in}{1.662103in}}%
\pgfpathlineto{\pgfqpoint{2.220364in}{1.645565in}}%
\pgfpathlineto{\pgfqpoint{2.224873in}{1.648322in}}%
\pgfpathlineto{\pgfqpoint{2.229382in}{1.641431in}}%
\pgfpathlineto{\pgfqpoint{2.233891in}{1.673129in}}%
\pgfpathlineto{\pgfqpoint{2.238400in}{1.666238in}}%
\pgfpathlineto{\pgfqpoint{2.242909in}{1.707583in}}%
\pgfpathlineto{\pgfqpoint{2.247418in}{1.671751in}}%
\pgfpathlineto{\pgfqpoint{2.251927in}{1.713096in}}%
\pgfpathlineto{\pgfqpoint{2.256436in}{1.725500in}}%
\pgfpathlineto{\pgfqpoint{2.260945in}{1.651078in}}%
\pgfpathlineto{\pgfqpoint{2.265455in}{1.702071in}}%
\pgfpathlineto{\pgfqpoint{2.269964in}{1.772358in}}%
\pgfpathlineto{\pgfqpoint{2.274473in}{1.697936in}}%
\pgfpathlineto{\pgfqpoint{2.278982in}{1.739281in}}%
\pgfpathlineto{\pgfqpoint{2.283491in}{1.693801in}}%
\pgfpathlineto{\pgfqpoint{2.288000in}{1.735147in}}%
\pgfpathlineto{\pgfqpoint{2.292509in}{1.728256in}}%
\pgfpathlineto{\pgfqpoint{2.297018in}{1.711718in}}%
\pgfpathlineto{\pgfqpoint{2.301527in}{1.753063in}}%
\pgfpathlineto{\pgfqpoint{2.306036in}{1.726878in}}%
\pgfpathlineto{\pgfqpoint{2.310545in}{1.768223in}}%
\pgfpathlineto{\pgfqpoint{2.315055in}{1.770979in}}%
\pgfpathlineto{\pgfqpoint{2.319564in}{1.735147in}}%
\pgfpathlineto{\pgfqpoint{2.324073in}{1.728256in}}%
\pgfpathlineto{\pgfqpoint{2.328582in}{1.769601in}}%
\pgfpathlineto{\pgfqpoint{2.333091in}{1.772358in}}%
\pgfpathlineto{\pgfqpoint{2.337600in}{1.794409in}}%
\pgfpathlineto{\pgfqpoint{2.342109in}{1.748929in}}%
\pgfpathlineto{\pgfqpoint{2.346618in}{1.790274in}}%
\pgfpathlineto{\pgfqpoint{2.351127in}{1.793030in}}%
\pgfpathlineto{\pgfqpoint{2.355636in}{1.786139in}}%
\pgfpathlineto{\pgfqpoint{2.360145in}{1.827485in}}%
\pgfpathlineto{\pgfqpoint{2.364655in}{1.753063in}}%
\pgfpathlineto{\pgfqpoint{2.369164in}{1.775114in}}%
\pgfpathlineto{\pgfqpoint{2.373673in}{1.758576in}}%
\pgfpathlineto{\pgfqpoint{2.378182in}{1.809568in}}%
\pgfpathlineto{\pgfqpoint{2.382691in}{1.754441in}}%
\pgfpathlineto{\pgfqpoint{2.391709in}{1.856427in}}%
\pgfpathlineto{\pgfqpoint{2.396218in}{1.772358in}}%
\pgfpathlineto{\pgfqpoint{2.400727in}{1.794409in}}%
\pgfpathlineto{\pgfqpoint{2.405236in}{1.864696in}}%
\pgfpathlineto{\pgfqpoint{2.409745in}{1.819216in}}%
\pgfpathlineto{\pgfqpoint{2.414255in}{1.841267in}}%
\pgfpathlineto{\pgfqpoint{2.418764in}{1.815081in}}%
\pgfpathlineto{\pgfqpoint{2.423273in}{1.846779in}}%
\pgfpathlineto{\pgfqpoint{2.427782in}{1.868830in}}%
\pgfpathlineto{\pgfqpoint{2.432291in}{1.804056in}}%
\pgfpathlineto{\pgfqpoint{2.436800in}{1.835754in}}%
\pgfpathlineto{\pgfqpoint{2.445818in}{2.188568in}}%
\pgfpathlineto{\pgfqpoint{2.450327in}{1.882612in}}%
\pgfpathlineto{\pgfqpoint{2.454836in}{1.846779in}}%
\pgfpathlineto{\pgfqpoint{2.459345in}{1.868830in}}%
\pgfpathlineto{\pgfqpoint{2.463855in}{1.861939in}}%
\pgfpathlineto{\pgfqpoint{2.477382in}{1.870208in}}%
\pgfpathlineto{\pgfqpoint{2.481891in}{1.834376in}}%
\pgfpathlineto{\pgfqpoint{2.486400in}{1.856427in}}%
\pgfpathlineto{\pgfqpoint{2.490909in}{1.907419in}}%
\pgfpathlineto{\pgfqpoint{2.495418in}{1.919823in}}%
\pgfpathlineto{\pgfqpoint{2.499927in}{1.845401in}}%
\pgfpathlineto{\pgfqpoint{2.504436in}{1.877099in}}%
\pgfpathlineto{\pgfqpoint{2.513455in}{1.901906in}}%
\pgfpathlineto{\pgfqpoint{2.517964in}{1.875721in}}%
\pgfpathlineto{\pgfqpoint{2.522473in}{1.897772in}}%
\pgfpathlineto{\pgfqpoint{2.526982in}{1.910176in}}%
\pgfpathlineto{\pgfqpoint{2.531491in}{1.912932in}}%
\pgfpathlineto{\pgfqpoint{2.536000in}{1.886747in}}%
\pgfpathlineto{\pgfqpoint{2.545018in}{1.892259in}}%
\pgfpathlineto{\pgfqpoint{2.549527in}{1.991488in}}%
\pgfpathlineto{\pgfqpoint{2.554036in}{1.907419in}}%
\pgfpathlineto{\pgfqpoint{2.558545in}{1.900528in}}%
\pgfpathlineto{\pgfqpoint{2.563055in}{1.864696in}}%
\pgfpathlineto{\pgfqpoint{2.567564in}{1.963925in}}%
\pgfpathlineto{\pgfqpoint{2.572073in}{1.918445in}}%
\pgfpathlineto{\pgfqpoint{2.576582in}{1.892259in}}%
\pgfpathlineto{\pgfqpoint{2.581091in}{1.895016in}}%
\pgfpathlineto{\pgfqpoint{2.585600in}{1.984597in}}%
\pgfpathlineto{\pgfqpoint{2.590109in}{1.919823in}}%
\pgfpathlineto{\pgfqpoint{2.594618in}{1.951521in}}%
\pgfpathlineto{\pgfqpoint{2.599127in}{1.925336in}}%
\pgfpathlineto{\pgfqpoint{2.603636in}{1.918445in}}%
\pgfpathlineto{\pgfqpoint{2.608145in}{1.979084in}}%
\pgfpathlineto{\pgfqpoint{2.612655in}{1.943252in}}%
\pgfpathlineto{\pgfqpoint{2.617164in}{1.974950in}}%
\pgfpathlineto{\pgfqpoint{2.621673in}{1.977706in}}%
\pgfpathlineto{\pgfqpoint{2.626182in}{1.941874in}}%
\pgfpathlineto{\pgfqpoint{2.630691in}{1.983219in}}%
\pgfpathlineto{\pgfqpoint{2.635200in}{1.966681in}}%
\pgfpathlineto{\pgfqpoint{2.639709in}{1.998379in}}%
\pgfpathlineto{\pgfqpoint{2.644218in}{1.991488in}}%
\pgfpathlineto{\pgfqpoint{2.648727in}{1.965303in}}%
\pgfpathlineto{\pgfqpoint{2.653236in}{2.016295in}}%
\pgfpathlineto{\pgfqpoint{2.657745in}{1.999757in}}%
\pgfpathlineto{\pgfqpoint{2.662255in}{1.992866in}}%
\pgfpathlineto{\pgfqpoint{2.666764in}{1.976328in}}%
\pgfpathlineto{\pgfqpoint{2.671273in}{2.065910in}}%
\pgfpathlineto{\pgfqpoint{2.675782in}{2.001135in}}%
\pgfpathlineto{\pgfqpoint{2.680291in}{2.032833in}}%
\pgfpathlineto{\pgfqpoint{2.684800in}{1.977706in}}%
\pgfpathlineto{\pgfqpoint{2.689309in}{1.990110in}}%
\pgfpathlineto{\pgfqpoint{2.693818in}{1.992866in}}%
\pgfpathlineto{\pgfqpoint{2.698327in}{2.005270in}}%
\pgfpathlineto{\pgfqpoint{2.702836in}{2.046615in}}%
\pgfpathlineto{\pgfqpoint{2.707345in}{1.981841in}}%
\pgfpathlineto{\pgfqpoint{2.711855in}{2.023186in}}%
\pgfpathlineto{\pgfqpoint{2.720873in}{2.028699in}}%
\pgfpathlineto{\pgfqpoint{2.725382in}{2.021808in}}%
\pgfpathlineto{\pgfqpoint{2.729891in}{2.101742in}}%
\pgfpathlineto{\pgfqpoint{2.734400in}{2.017674in}}%
\pgfpathlineto{\pgfqpoint{2.738909in}{2.087961in}}%
\pgfpathlineto{\pgfqpoint{2.743418in}{2.003892in}}%
\pgfpathlineto{\pgfqpoint{2.747927in}{2.035590in}}%
\pgfpathlineto{\pgfqpoint{2.752436in}{2.057641in}}%
\pgfpathlineto{\pgfqpoint{2.756945in}{2.012161in}}%
\pgfpathlineto{\pgfqpoint{2.761455in}{2.063153in}}%
\pgfpathlineto{\pgfqpoint{2.765964in}{2.094852in}}%
\pgfpathlineto{\pgfqpoint{2.770473in}{2.068666in}}%
\pgfpathlineto{\pgfqpoint{2.774982in}{2.100364in}}%
\pgfpathlineto{\pgfqpoint{2.779491in}{2.064532in}}%
\pgfpathlineto{\pgfqpoint{2.784000in}{2.115524in}}%
\pgfpathlineto{\pgfqpoint{2.788509in}{2.079692in}}%
\pgfpathlineto{\pgfqpoint{2.793018in}{2.111390in}}%
\pgfpathlineto{\pgfqpoint{2.797527in}{2.133441in}}%
\pgfpathlineto{\pgfqpoint{2.802036in}{2.116902in}}%
\pgfpathlineto{\pgfqpoint{2.806545in}{2.052128in}}%
\pgfpathlineto{\pgfqpoint{2.811055in}{2.083826in}}%
\pgfpathlineto{\pgfqpoint{2.815564in}{2.086582in}}%
\pgfpathlineto{\pgfqpoint{2.820073in}{2.060397in}}%
\pgfpathlineto{\pgfqpoint{2.824582in}{2.130684in}}%
\pgfpathlineto{\pgfqpoint{2.829091in}{2.075557in}}%
\pgfpathlineto{\pgfqpoint{2.833600in}{2.097608in}}%
\pgfpathlineto{\pgfqpoint{2.838109in}{2.158248in}}%
\pgfpathlineto{\pgfqpoint{2.842618in}{2.093473in}}%
\pgfpathlineto{\pgfqpoint{2.847127in}{2.134819in}}%
\pgfpathlineto{\pgfqpoint{2.851636in}{2.098986in}}%
\pgfpathlineto{\pgfqpoint{2.856145in}{2.111390in}}%
\pgfpathlineto{\pgfqpoint{2.860655in}{2.094852in}}%
\pgfpathlineto{\pgfqpoint{2.865164in}{2.087961in}}%
\pgfpathlineto{\pgfqpoint{2.869673in}{2.100364in}}%
\pgfpathlineto{\pgfqpoint{2.874182in}{2.141710in}}%
\pgfpathlineto{\pgfqpoint{2.878691in}{2.115524in}}%
\pgfpathlineto{\pgfqpoint{2.883200in}{2.214753in}}%
\pgfpathlineto{\pgfqpoint{2.887709in}{2.149979in}}%
\pgfpathlineto{\pgfqpoint{2.892218in}{2.114146in}}%
\pgfpathlineto{\pgfqpoint{2.896727in}{2.184433in}}%
\pgfpathlineto{\pgfqpoint{2.901236in}{2.148600in}}%
\pgfpathlineto{\pgfqpoint{2.905745in}{2.218888in}}%
\pgfpathlineto{\pgfqpoint{2.910255in}{2.163760in}}%
\pgfpathlineto{\pgfqpoint{2.914764in}{2.176164in}}%
\pgfpathlineto{\pgfqpoint{2.919273in}{2.130684in}}%
\pgfpathlineto{\pgfqpoint{2.923782in}{2.239560in}}%
\pgfpathlineto{\pgfqpoint{2.928291in}{2.213375in}}%
\pgfpathlineto{\pgfqpoint{2.932800in}{2.177542in}}%
\pgfpathlineto{\pgfqpoint{2.937309in}{2.161004in}}%
\pgfpathlineto{\pgfqpoint{2.941818in}{2.183055in}}%
\pgfpathlineto{\pgfqpoint{2.946327in}{2.176164in}}%
\pgfpathlineto{\pgfqpoint{2.955345in}{2.181677in}}%
\pgfpathlineto{\pgfqpoint{2.959855in}{2.194080in}}%
\pgfpathlineto{\pgfqpoint{2.964364in}{2.196837in}}%
\pgfpathlineto{\pgfqpoint{2.968873in}{2.228535in}}%
\pgfpathlineto{\pgfqpoint{2.973382in}{2.173408in}}%
\pgfpathlineto{\pgfqpoint{2.977891in}{2.195459in}}%
\pgfpathlineto{\pgfqpoint{2.986909in}{2.200971in}}%
\pgfpathlineto{\pgfqpoint{2.991418in}{2.213375in}}%
\pgfpathlineto{\pgfqpoint{2.995927in}{2.235426in}}%
\pgfpathlineto{\pgfqpoint{3.000436in}{2.180299in}}%
\pgfpathlineto{\pgfqpoint{3.004945in}{2.221644in}}%
\pgfpathlineto{\pgfqpoint{3.009455in}{2.234048in}}%
\pgfpathlineto{\pgfqpoint{3.013964in}{2.236804in}}%
\pgfpathlineto{\pgfqpoint{3.018473in}{2.200971in}}%
\pgfpathlineto{\pgfqpoint{3.022982in}{2.242317in}}%
\pgfpathlineto{\pgfqpoint{3.027491in}{2.196837in}}%
\pgfpathlineto{\pgfqpoint{3.032000in}{2.228535in}}%
\pgfpathlineto{\pgfqpoint{3.036509in}{2.231291in}}%
\pgfpathlineto{\pgfqpoint{3.041018in}{2.214753in}}%
\pgfpathlineto{\pgfqpoint{3.045527in}{2.265746in}}%
\pgfpathlineto{\pgfqpoint{3.050036in}{2.210619in}}%
\pgfpathlineto{\pgfqpoint{3.063564in}{2.276771in}}%
\pgfpathlineto{\pgfqpoint{3.068073in}{2.327764in}}%
\pgfpathlineto{\pgfqpoint{3.072582in}{2.234048in}}%
\pgfpathlineto{\pgfqpoint{3.077091in}{2.275393in}}%
\pgfpathlineto{\pgfqpoint{3.081600in}{2.287797in}}%
\pgfpathlineto{\pgfqpoint{3.086109in}{2.290553in}}%
\pgfpathlineto{\pgfqpoint{3.090618in}{2.254720in}}%
\pgfpathlineto{\pgfqpoint{3.095127in}{2.315360in}}%
\pgfpathlineto{\pgfqpoint{3.099636in}{2.269880in}}%
\pgfpathlineto{\pgfqpoint{3.108655in}{2.275393in}}%
\pgfpathlineto{\pgfqpoint{3.113164in}{2.364975in}}%
\pgfpathlineto{\pgfqpoint{3.117673in}{2.309847in}}%
\pgfpathlineto{\pgfqpoint{3.122182in}{2.302957in}}%
\pgfpathlineto{\pgfqpoint{3.126691in}{2.344302in}}%
\pgfpathlineto{\pgfqpoint{3.135709in}{2.272637in}}%
\pgfpathlineto{\pgfqpoint{3.140218in}{2.285040in}}%
\pgfpathlineto{\pgfqpoint{3.144727in}{2.316738in}}%
\pgfpathlineto{\pgfqpoint{3.149236in}{2.309847in}}%
\pgfpathlineto{\pgfqpoint{3.153745in}{2.322251in}}%
\pgfpathlineto{\pgfqpoint{3.158255in}{2.325007in}}%
\pgfpathlineto{\pgfqpoint{3.162764in}{2.356705in}}%
\pgfpathlineto{\pgfqpoint{3.171782in}{2.323629in}}%
\pgfpathlineto{\pgfqpoint{3.176291in}{2.422858in}}%
\pgfpathlineto{\pgfqpoint{3.180800in}{2.396673in}}%
\pgfpathlineto{\pgfqpoint{3.189818in}{2.325007in}}%
\pgfpathlineto{\pgfqpoint{3.194327in}{2.327764in}}%
\pgfpathlineto{\pgfqpoint{3.198836in}{2.359462in}}%
\pgfpathlineto{\pgfqpoint{3.203345in}{2.381513in}}%
\pgfpathlineto{\pgfqpoint{3.207855in}{2.393916in}}%
\pgfpathlineto{\pgfqpoint{3.212364in}{2.377378in}}%
\pgfpathlineto{\pgfqpoint{3.216873in}{2.428371in}}%
\pgfpathlineto{\pgfqpoint{3.221382in}{2.373244in}}%
\pgfpathlineto{\pgfqpoint{3.225891in}{2.366353in}}%
\pgfpathlineto{\pgfqpoint{3.230400in}{2.340167in}}%
\pgfpathlineto{\pgfqpoint{3.234909in}{2.410454in}}%
\pgfpathlineto{\pgfqpoint{3.239418in}{2.422858in}}%
\pgfpathlineto{\pgfqpoint{3.243927in}{2.377378in}}%
\pgfpathlineto{\pgfqpoint{3.248436in}{2.351193in}}%
\pgfpathlineto{\pgfqpoint{3.257455in}{2.472473in}}%
\pgfpathlineto{\pgfqpoint{3.261964in}{2.398051in}}%
\pgfpathlineto{\pgfqpoint{3.266473in}{2.429749in}}%
\pgfpathlineto{\pgfqpoint{3.275491in}{2.454556in}}%
\pgfpathlineto{\pgfqpoint{3.280000in}{2.380135in}}%
\pgfpathlineto{\pgfqpoint{3.284509in}{2.450422in}}%
\pgfpathlineto{\pgfqpoint{3.289018in}{2.385647in}}%
\pgfpathlineto{\pgfqpoint{3.293527in}{2.475229in}}%
\pgfpathlineto{\pgfqpoint{3.298036in}{2.391160in}}%
\pgfpathlineto{\pgfqpoint{3.302545in}{2.413211in}}%
\pgfpathlineto{\pgfqpoint{3.307055in}{2.415967in}}%
\pgfpathlineto{\pgfqpoint{3.311564in}{2.389782in}}%
\pgfpathlineto{\pgfqpoint{3.316073in}{2.440774in}}%
\pgfpathlineto{\pgfqpoint{3.320582in}{2.433884in}}%
\pgfpathlineto{\pgfqpoint{3.325091in}{2.398051in}}%
\pgfpathlineto{\pgfqpoint{3.329600in}{2.449043in}}%
\pgfpathlineto{\pgfqpoint{3.334109in}{2.432505in}}%
\pgfpathlineto{\pgfqpoint{3.338618in}{2.444909in}}%
\pgfpathlineto{\pgfqpoint{3.347636in}{2.411833in}}%
\pgfpathlineto{\pgfqpoint{3.352145in}{2.453178in}}%
\pgfpathlineto{\pgfqpoint{3.361164in}{2.477985in}}%
\pgfpathlineto{\pgfqpoint{3.365673in}{2.500036in}}%
\pgfpathlineto{\pgfqpoint{3.370182in}{2.464203in}}%
\pgfpathlineto{\pgfqpoint{3.374691in}{2.418724in}}%
\pgfpathlineto{\pgfqpoint{3.379200in}{2.489011in}}%
\pgfpathlineto{\pgfqpoint{3.383709in}{2.491767in}}%
\pgfpathlineto{\pgfqpoint{3.388218in}{2.513818in}}%
\pgfpathlineto{\pgfqpoint{3.392727in}{2.497280in}}%
\pgfpathlineto{\pgfqpoint{3.397236in}{2.509683in}}%
\pgfpathlineto{\pgfqpoint{3.401745in}{2.444909in}}%
\pgfpathlineto{\pgfqpoint{3.406255in}{2.428371in}}%
\pgfpathlineto{\pgfqpoint{3.410764in}{2.517952in}}%
\pgfpathlineto{\pgfqpoint{3.415273in}{2.568945in}}%
\pgfpathlineto{\pgfqpoint{3.419782in}{2.436640in}}%
\pgfpathlineto{\pgfqpoint{3.424291in}{2.516574in}}%
\pgfpathlineto{\pgfqpoint{3.428800in}{2.490389in}}%
\pgfpathlineto{\pgfqpoint{3.433309in}{2.531734in}}%
\pgfpathlineto{\pgfqpoint{3.437818in}{2.544138in}}%
\pgfpathlineto{\pgfqpoint{3.442327in}{2.546894in}}%
\pgfpathlineto{\pgfqpoint{3.446836in}{2.511062in}}%
\pgfpathlineto{\pgfqpoint{3.451345in}{2.542760in}}%
\pgfpathlineto{\pgfqpoint{3.455855in}{2.535869in}}%
\pgfpathlineto{\pgfqpoint{3.460364in}{2.548272in}}%
\pgfpathlineto{\pgfqpoint{3.464873in}{2.551029in}}%
\pgfpathlineto{\pgfqpoint{3.469382in}{2.505549in}}%
\pgfpathlineto{\pgfqpoint{3.473891in}{2.546894in}}%
\pgfpathlineto{\pgfqpoint{3.478400in}{2.520709in}}%
\pgfpathlineto{\pgfqpoint{3.482909in}{2.523465in}}%
\pgfpathlineto{\pgfqpoint{3.487418in}{2.603400in}}%
\pgfpathlineto{\pgfqpoint{3.491927in}{2.528978in}}%
\pgfpathlineto{\pgfqpoint{3.496436in}{2.502792in}}%
\pgfpathlineto{\pgfqpoint{3.500945in}{2.573080in}}%
\pgfpathlineto{\pgfqpoint{3.505455in}{2.575836in}}%
\pgfpathlineto{\pgfqpoint{3.509964in}{2.549651in}}%
\pgfpathlineto{\pgfqpoint{3.514473in}{2.562054in}}%
\pgfpathlineto{\pgfqpoint{3.518982in}{2.613047in}}%
\pgfpathlineto{\pgfqpoint{3.528000in}{2.541381in}}%
\pgfpathlineto{\pgfqpoint{3.532509in}{2.573080in}}%
\pgfpathlineto{\pgfqpoint{3.537018in}{2.585483in}}%
\pgfpathlineto{\pgfqpoint{3.541527in}{2.540003in}}%
\pgfpathlineto{\pgfqpoint{3.546036in}{2.581349in}}%
\pgfpathlineto{\pgfqpoint{3.550545in}{2.526221in}}%
\pgfpathlineto{\pgfqpoint{3.559564in}{2.657148in}}%
\pgfpathlineto{\pgfqpoint{3.564073in}{2.592374in}}%
\pgfpathlineto{\pgfqpoint{3.568582in}{2.556541in}}%
\pgfpathlineto{\pgfqpoint{3.573091in}{2.588240in}}%
\pgfpathlineto{\pgfqpoint{3.577600in}{2.668174in}}%
\pgfpathlineto{\pgfqpoint{3.582109in}{2.593752in}}%
\pgfpathlineto{\pgfqpoint{3.586618in}{2.606156in}}%
\pgfpathlineto{\pgfqpoint{3.591127in}{2.570323in}}%
\pgfpathlineto{\pgfqpoint{3.595636in}{2.621316in}}%
\pgfpathlineto{\pgfqpoint{3.600145in}{2.691603in}}%
\pgfpathlineto{\pgfqpoint{3.604655in}{2.607534in}}%
\pgfpathlineto{\pgfqpoint{3.609164in}{2.658527in}}%
\pgfpathlineto{\pgfqpoint{3.613673in}{2.593752in}}%
\pgfpathlineto{\pgfqpoint{3.618182in}{2.606156in}}%
\pgfpathlineto{\pgfqpoint{3.622691in}{2.647501in}}%
\pgfpathlineto{\pgfqpoint{3.627200in}{2.621316in}}%
\pgfpathlineto{\pgfqpoint{3.631709in}{2.633719in}}%
\pgfpathlineto{\pgfqpoint{3.636218in}{2.607534in}}%
\pgfpathlineto{\pgfqpoint{3.640727in}{2.600643in}}%
\pgfpathlineto{\pgfqpoint{3.645236in}{2.632341in}}%
\pgfpathlineto{\pgfqpoint{3.649745in}{2.683334in}}%
\pgfpathlineto{\pgfqpoint{3.654255in}{2.705385in}}%
\pgfpathlineto{\pgfqpoint{3.658764in}{2.621316in}}%
\pgfpathlineto{\pgfqpoint{3.663273in}{2.672308in}}%
\pgfpathlineto{\pgfqpoint{3.667782in}{2.655770in}}%
\pgfpathlineto{\pgfqpoint{3.672291in}{2.706763in}}%
\pgfpathlineto{\pgfqpoint{3.676800in}{2.699872in}}%
\pgfpathlineto{\pgfqpoint{3.681309in}{2.635098in}}%
\pgfpathlineto{\pgfqpoint{3.685818in}{2.676443in}}%
\pgfpathlineto{\pgfqpoint{3.690327in}{2.650258in}}%
\pgfpathlineto{\pgfqpoint{3.694836in}{2.778428in}}%
\pgfpathlineto{\pgfqpoint{3.699345in}{2.742596in}}%
\pgfpathlineto{\pgfqpoint{3.703855in}{2.697116in}}%
\pgfpathlineto{\pgfqpoint{3.708364in}{2.670930in}}%
\pgfpathlineto{\pgfqpoint{3.712873in}{2.692981in}}%
\pgfpathlineto{\pgfqpoint{3.717382in}{2.676443in}}%
\pgfpathlineto{\pgfqpoint{3.721891in}{2.717788in}}%
\pgfpathlineto{\pgfqpoint{3.726400in}{2.701250in}}%
\pgfpathlineto{\pgfqpoint{3.730909in}{2.723301in}}%
\pgfpathlineto{\pgfqpoint{3.735418in}{2.687468in}}%
\pgfpathlineto{\pgfqpoint{3.739927in}{2.699872in}}%
\pgfpathlineto{\pgfqpoint{3.744436in}{2.750865in}}%
\pgfpathlineto{\pgfqpoint{3.748945in}{2.734326in}}%
\pgfpathlineto{\pgfqpoint{3.753455in}{2.679199in}}%
\pgfpathlineto{\pgfqpoint{3.757964in}{2.701250in}}%
\pgfpathlineto{\pgfqpoint{3.762473in}{2.684712in}}%
\pgfpathlineto{\pgfqpoint{3.771491in}{2.728814in}}%
\pgfpathlineto{\pgfqpoint{3.776000in}{2.683334in}}%
\pgfpathlineto{\pgfqpoint{3.780509in}{2.743974in}}%
\pgfpathlineto{\pgfqpoint{3.785018in}{2.727436in}}%
\pgfpathlineto{\pgfqpoint{3.789527in}{2.768781in}}%
\pgfpathlineto{\pgfqpoint{3.794036in}{2.761890in}}%
\pgfpathlineto{\pgfqpoint{3.798545in}{2.735705in}}%
\pgfpathlineto{\pgfqpoint{3.803055in}{2.738461in}}%
\pgfpathlineto{\pgfqpoint{3.807564in}{2.770159in}}%
\pgfpathlineto{\pgfqpoint{3.812073in}{2.743974in}}%
\pgfpathlineto{\pgfqpoint{3.821091in}{2.768781in}}%
\pgfpathlineto{\pgfqpoint{3.825600in}{2.742596in}}%
\pgfpathlineto{\pgfqpoint{3.830109in}{2.783941in}}%
\pgfpathlineto{\pgfqpoint{3.834618in}{2.786697in}}%
\pgfpathlineto{\pgfqpoint{3.839127in}{2.731570in}}%
\pgfpathlineto{\pgfqpoint{3.843636in}{2.840446in}}%
\pgfpathlineto{\pgfqpoint{3.848145in}{2.804614in}}%
\pgfpathlineto{\pgfqpoint{3.852655in}{2.788075in}}%
\pgfpathlineto{\pgfqpoint{3.857164in}{2.761890in}}%
\pgfpathlineto{\pgfqpoint{3.861673in}{2.841824in}}%
\pgfpathlineto{\pgfqpoint{3.866182in}{2.796345in}}%
\pgfpathlineto{\pgfqpoint{3.870691in}{2.828043in}}%
\pgfpathlineto{\pgfqpoint{3.875200in}{2.801857in}}%
\pgfpathlineto{\pgfqpoint{3.879709in}{2.823908in}}%
\pgfpathlineto{\pgfqpoint{3.884218in}{2.778428in}}%
\pgfpathlineto{\pgfqpoint{3.893236in}{2.783941in}}%
\pgfpathlineto{\pgfqpoint{3.897745in}{2.805992in}}%
\pgfpathlineto{\pgfqpoint{3.902255in}{2.779806in}}%
\pgfpathlineto{\pgfqpoint{3.906764in}{2.840446in}}%
\pgfpathlineto{\pgfqpoint{3.911273in}{2.794966in}}%
\pgfpathlineto{\pgfqpoint{3.915782in}{2.817017in}}%
\pgfpathlineto{\pgfqpoint{3.920291in}{2.810126in}}%
\pgfpathlineto{\pgfqpoint{3.924800in}{2.861119in}}%
\pgfpathlineto{\pgfqpoint{3.929309in}{2.805992in}}%
\pgfpathlineto{\pgfqpoint{3.938327in}{2.830799in}}%
\pgfpathlineto{\pgfqpoint{3.947345in}{2.797723in}}%
\pgfpathlineto{\pgfqpoint{3.951855in}{2.829421in}}%
\pgfpathlineto{\pgfqpoint{3.956364in}{2.822530in}}%
\pgfpathlineto{\pgfqpoint{3.960873in}{2.834934in}}%
\pgfpathlineto{\pgfqpoint{3.965382in}{2.837690in}}%
\pgfpathlineto{\pgfqpoint{3.969891in}{2.801857in}}%
\pgfpathlineto{\pgfqpoint{3.974400in}{2.949322in}}%
\pgfpathlineto{\pgfqpoint{3.983418in}{2.848715in}}%
\pgfpathlineto{\pgfqpoint{3.987927in}{2.851472in}}%
\pgfpathlineto{\pgfqpoint{3.992436in}{2.873523in}}%
\pgfpathlineto{\pgfqpoint{3.996945in}{2.847337in}}%
\pgfpathlineto{\pgfqpoint{4.001455in}{2.907977in}}%
\pgfpathlineto{\pgfqpoint{4.005964in}{2.910733in}}%
\pgfpathlineto{\pgfqpoint{4.010473in}{2.845959in}}%
\pgfpathlineto{\pgfqpoint{4.014982in}{2.848715in}}%
\pgfpathlineto{\pgfqpoint{4.019491in}{2.861119in}}%
\pgfpathlineto{\pgfqpoint{4.024000in}{2.921759in}}%
\pgfpathlineto{\pgfqpoint{4.028509in}{2.924515in}}%
\pgfpathlineto{\pgfqpoint{4.033018in}{2.859741in}}%
\pgfpathlineto{\pgfqpoint{4.037527in}{2.939675in}}%
\pgfpathlineto{\pgfqpoint{4.042036in}{2.923137in}}%
\pgfpathlineto{\pgfqpoint{4.046545in}{2.887304in}}%
\pgfpathlineto{\pgfqpoint{4.051055in}{2.909355in}}%
\pgfpathlineto{\pgfqpoint{4.055564in}{2.921759in}}%
\pgfpathlineto{\pgfqpoint{4.060073in}{3.011340in}}%
\pgfpathlineto{\pgfqpoint{4.064582in}{2.956213in}}%
\pgfpathlineto{\pgfqpoint{4.069091in}{2.949322in}}%
\pgfpathlineto{\pgfqpoint{4.073600in}{2.884548in}}%
\pgfpathlineto{\pgfqpoint{4.078109in}{2.945188in}}%
\pgfpathlineto{\pgfqpoint{4.082618in}{2.909355in}}%
\pgfpathlineto{\pgfqpoint{4.091636in}{2.934162in}}%
\pgfpathlineto{\pgfqpoint{4.096145in}{2.907977in}}%
\pgfpathlineto{\pgfqpoint{4.100655in}{2.930028in}}%
\pgfpathlineto{\pgfqpoint{4.105164in}{2.961726in}}%
\pgfpathlineto{\pgfqpoint{4.109673in}{2.935541in}}%
\pgfpathlineto{\pgfqpoint{4.114182in}{2.938297in}}%
\pgfpathlineto{\pgfqpoint{4.118691in}{2.931406in}}%
\pgfpathlineto{\pgfqpoint{4.123200in}{2.953457in}}%
\pgfpathlineto{\pgfqpoint{4.127709in}{2.936919in}}%
\pgfpathlineto{\pgfqpoint{4.132218in}{3.007206in}}%
\pgfpathlineto{\pgfqpoint{4.136727in}{2.932784in}}%
\pgfpathlineto{\pgfqpoint{4.141236in}{2.935541in}}%
\pgfpathlineto{\pgfqpoint{4.145745in}{2.928650in}}%
\pgfpathlineto{\pgfqpoint{4.150255in}{3.037526in}}%
\pgfpathlineto{\pgfqpoint{4.154764in}{2.934162in}}%
\pgfpathlineto{\pgfqpoint{4.163782in}{3.026500in}}%
\pgfpathlineto{\pgfqpoint{4.168291in}{2.981021in}}%
\pgfpathlineto{\pgfqpoint{4.172800in}{2.993424in}}%
\pgfpathlineto{\pgfqpoint{4.181818in}{2.960348in}}%
\pgfpathlineto{\pgfqpoint{4.186327in}{3.011340in}}%
\pgfpathlineto{\pgfqpoint{4.190836in}{3.023744in}}%
\pgfpathlineto{\pgfqpoint{4.195345in}{2.997559in}}%
\pgfpathlineto{\pgfqpoint{4.199855in}{2.981021in}}%
\pgfpathlineto{\pgfqpoint{4.204364in}{3.051308in}}%
\pgfpathlineto{\pgfqpoint{4.208873in}{2.996180in}}%
\pgfpathlineto{\pgfqpoint{4.213382in}{3.027879in}}%
\pgfpathlineto{\pgfqpoint{4.217891in}{3.011340in}}%
\pgfpathlineto{\pgfqpoint{4.222400in}{3.023744in}}%
\pgfpathlineto{\pgfqpoint{4.226909in}{3.045795in}}%
\pgfpathlineto{\pgfqpoint{4.231418in}{3.029257in}}%
\pgfpathlineto{\pgfqpoint{4.235927in}{3.080249in}}%
\pgfpathlineto{\pgfqpoint{4.240436in}{3.063711in}}%
\pgfpathlineto{\pgfqpoint{4.244945in}{3.008584in}}%
\pgfpathlineto{\pgfqpoint{4.249455in}{3.098166in}}%
\pgfpathlineto{\pgfqpoint{4.253964in}{3.023744in}}%
\pgfpathlineto{\pgfqpoint{4.258473in}{3.045795in}}%
\pgfpathlineto{\pgfqpoint{4.262982in}{3.106435in}}%
\pgfpathlineto{\pgfqpoint{4.267491in}{3.089897in}}%
\pgfpathlineto{\pgfqpoint{4.272000in}{3.034769in}}%
\pgfpathlineto{\pgfqpoint{4.276509in}{3.018231in}}%
\pgfpathlineto{\pgfqpoint{4.285527in}{3.004450in}}%
\pgfpathlineto{\pgfqpoint{4.290036in}{3.103678in}}%
\pgfpathlineto{\pgfqpoint{4.299055in}{3.051308in}}%
\pgfpathlineto{\pgfqpoint{4.303564in}{3.102300in}}%
\pgfpathlineto{\pgfqpoint{4.308073in}{3.076115in}}%
\pgfpathlineto{\pgfqpoint{4.312582in}{3.107813in}}%
\pgfpathlineto{\pgfqpoint{4.317091in}{3.071980in}}%
\pgfpathlineto{\pgfqpoint{4.321600in}{3.045795in}}%
\pgfpathlineto{\pgfqpoint{4.326109in}{3.145024in}}%
\pgfpathlineto{\pgfqpoint{4.330618in}{3.070602in}}%
\pgfpathlineto{\pgfqpoint{4.339636in}{3.162940in}}%
\pgfpathlineto{\pgfqpoint{4.344145in}{3.069224in}}%
\pgfpathlineto{\pgfqpoint{4.348655in}{3.100922in}}%
\pgfpathlineto{\pgfqpoint{4.353164in}{3.084384in}}%
\pgfpathlineto{\pgfqpoint{4.357673in}{3.077493in}}%
\pgfpathlineto{\pgfqpoint{4.362182in}{3.089897in}}%
\pgfpathlineto{\pgfqpoint{4.366691in}{3.140889in}}%
\pgfpathlineto{\pgfqpoint{4.371200in}{3.105057in}}%
\pgfpathlineto{\pgfqpoint{4.375709in}{3.194638in}}%
\pgfpathlineto{\pgfqpoint{4.380218in}{3.129864in}}%
\pgfpathlineto{\pgfqpoint{4.384727in}{3.161562in}}%
\pgfpathlineto{\pgfqpoint{4.389236in}{3.145024in}}%
\pgfpathlineto{\pgfqpoint{4.393745in}{3.157427in}}%
\pgfpathlineto{\pgfqpoint{4.398255in}{3.131242in}}%
\pgfpathlineto{\pgfqpoint{4.402764in}{3.133998in}}%
\pgfpathlineto{\pgfqpoint{4.407273in}{3.146402in}}%
\pgfpathlineto{\pgfqpoint{4.411782in}{3.129864in}}%
\pgfpathlineto{\pgfqpoint{4.416291in}{3.142267in}}%
\pgfpathlineto{\pgfqpoint{4.420800in}{3.125729in}}%
\pgfpathlineto{\pgfqpoint{4.425309in}{3.186369in}}%
\pgfpathlineto{\pgfqpoint{4.429818in}{3.140889in}}%
\pgfpathlineto{\pgfqpoint{4.434327in}{3.143646in}}%
\pgfpathlineto{\pgfqpoint{4.438836in}{3.175344in}}%
\pgfpathlineto{\pgfqpoint{4.443345in}{3.197395in}}%
\pgfpathlineto{\pgfqpoint{4.447855in}{3.171209in}}%
\pgfpathlineto{\pgfqpoint{4.452364in}{3.154671in}}%
\pgfpathlineto{\pgfqpoint{4.456873in}{3.176722in}}%
\pgfpathlineto{\pgfqpoint{4.461382in}{3.189126in}}%
\pgfpathlineto{\pgfqpoint{4.465891in}{3.191882in}}%
\pgfpathlineto{\pgfqpoint{4.470400in}{3.165696in}}%
\pgfpathlineto{\pgfqpoint{4.474909in}{3.168453in}}%
\pgfpathlineto{\pgfqpoint{4.479418in}{3.180856in}}%
\pgfpathlineto{\pgfqpoint{4.483927in}{3.183613in}}%
\pgfpathlineto{\pgfqpoint{4.488436in}{3.196016in}}%
\pgfpathlineto{\pgfqpoint{4.492945in}{3.189126in}}%
\pgfpathlineto{\pgfqpoint{4.497455in}{3.211176in}}%
\pgfpathlineto{\pgfqpoint{4.501964in}{3.184991in}}%
\pgfpathlineto{\pgfqpoint{4.506473in}{3.197395in}}%
\pgfpathlineto{\pgfqpoint{4.510982in}{3.190504in}}%
\pgfpathlineto{\pgfqpoint{4.515491in}{3.202907in}}%
\pgfpathlineto{\pgfqpoint{4.520000in}{3.176722in}}%
\pgfpathlineto{\pgfqpoint{4.524509in}{3.189126in}}%
\pgfpathlineto{\pgfqpoint{4.529018in}{3.249765in}}%
\pgfpathlineto{\pgfqpoint{4.533527in}{3.194638in}}%
\pgfpathlineto{\pgfqpoint{4.538036in}{3.255278in}}%
\pgfpathlineto{\pgfqpoint{4.542545in}{3.219445in}}%
\pgfpathlineto{\pgfqpoint{4.547055in}{3.280085in}}%
\pgfpathlineto{\pgfqpoint{4.551564in}{3.205664in}}%
\pgfpathlineto{\pgfqpoint{4.556073in}{3.208420in}}%
\pgfpathlineto{\pgfqpoint{4.560582in}{3.490947in}}%
\pgfpathlineto{\pgfqpoint{4.565091in}{3.262169in}}%
\pgfpathlineto{\pgfqpoint{4.569600in}{3.293867in}}%
\pgfpathlineto{\pgfqpoint{4.574109in}{3.238740in}}%
\pgfpathlineto{\pgfqpoint{4.587636in}{3.218067in}}%
\pgfpathlineto{\pgfqpoint{4.592145in}{3.249765in}}%
\pgfpathlineto{\pgfqpoint{4.605673in}{3.258034in}}%
\pgfpathlineto{\pgfqpoint{4.610182in}{3.299380in}}%
\pgfpathlineto{\pgfqpoint{4.614691in}{3.321431in}}%
\pgfpathlineto{\pgfqpoint{4.619200in}{3.314540in}}%
\pgfpathlineto{\pgfqpoint{4.623709in}{3.230471in}}%
\pgfpathlineto{\pgfqpoint{4.628218in}{3.242874in}}%
\pgfpathlineto{\pgfqpoint{4.632727in}{3.264925in}}%
\pgfpathlineto{\pgfqpoint{4.637236in}{3.238740in}}%
\pgfpathlineto{\pgfqpoint{4.641745in}{3.280085in}}%
\pgfpathlineto{\pgfqpoint{4.646255in}{3.282842in}}%
\pgfpathlineto{\pgfqpoint{4.650764in}{3.333834in}}%
\pgfpathlineto{\pgfqpoint{4.655273in}{3.269060in}}%
\pgfpathlineto{\pgfqpoint{4.659782in}{3.368289in}}%
\pgfpathlineto{\pgfqpoint{4.664291in}{3.264925in}}%
\pgfpathlineto{\pgfqpoint{4.668800in}{3.296623in}}%
\pgfpathlineto{\pgfqpoint{4.673309in}{3.318674in}}%
\pgfpathlineto{\pgfqpoint{4.677818in}{3.379314in}}%
\pgfpathlineto{\pgfqpoint{4.682327in}{3.304893in}}%
\pgfpathlineto{\pgfqpoint{4.686836in}{3.365532in}}%
\pgfpathlineto{\pgfqpoint{4.691345in}{3.291111in}}%
\pgfpathlineto{\pgfqpoint{4.695855in}{3.284220in}}%
\pgfpathlineto{\pgfqpoint{4.700364in}{3.296623in}}%
\pgfpathlineto{\pgfqpoint{4.704873in}{3.357263in}}%
\pgfpathlineto{\pgfqpoint{4.709382in}{3.350372in}}%
\pgfpathlineto{\pgfqpoint{4.713891in}{3.382071in}}%
\pgfpathlineto{\pgfqpoint{4.718400in}{3.365532in}}%
\pgfpathlineto{\pgfqpoint{4.727418in}{3.351751in}}%
\pgfpathlineto{\pgfqpoint{4.731927in}{3.315918in}}%
\pgfpathlineto{\pgfqpoint{4.736436in}{3.299380in}}%
\pgfpathlineto{\pgfqpoint{4.740945in}{3.302136in}}%
\pgfpathlineto{\pgfqpoint{4.745455in}{3.333834in}}%
\pgfpathlineto{\pgfqpoint{4.749964in}{3.404121in}}%
\pgfpathlineto{\pgfqpoint{4.754473in}{3.329700in}}%
\pgfpathlineto{\pgfqpoint{4.763491in}{3.335212in}}%
\pgfpathlineto{\pgfqpoint{4.772509in}{3.408256in}}%
\pgfpathlineto{\pgfqpoint{4.777018in}{3.343482in}}%
\pgfpathlineto{\pgfqpoint{4.781527in}{3.413769in}}%
\pgfpathlineto{\pgfqpoint{4.786036in}{3.348994in}}%
\pgfpathlineto{\pgfqpoint{4.790545in}{3.380692in}}%
\pgfpathlineto{\pgfqpoint{4.804073in}{3.388961in}}%
\pgfpathlineto{\pgfqpoint{4.808582in}{3.439954in}}%
\pgfpathlineto{\pgfqpoint{4.813091in}{3.404121in}}%
\pgfpathlineto{\pgfqpoint{4.817600in}{3.426172in}}%
\pgfpathlineto{\pgfqpoint{4.822109in}{3.399987in}}%
\pgfpathlineto{\pgfqpoint{4.826618in}{3.383449in}}%
\pgfpathlineto{\pgfqpoint{4.831127in}{3.415147in}}%
\pgfpathlineto{\pgfqpoint{4.835636in}{3.360020in}}%
\pgfpathlineto{\pgfqpoint{4.840145in}{3.439954in}}%
\pgfpathlineto{\pgfqpoint{4.844655in}{3.452358in}}%
\pgfpathlineto{\pgfqpoint{4.849164in}{3.416525in}}%
\pgfpathlineto{\pgfqpoint{4.853673in}{3.506107in}}%
\pgfpathlineto{\pgfqpoint{4.858182in}{3.364154in}}%
\pgfpathlineto{\pgfqpoint{4.867200in}{3.514376in}}%
\pgfpathlineto{\pgfqpoint{4.871709in}{3.439954in}}%
\pgfpathlineto{\pgfqpoint{4.876218in}{3.423416in}}%
\pgfpathlineto{\pgfqpoint{4.880727in}{3.426172in}}%
\pgfpathlineto{\pgfqpoint{4.885236in}{3.457870in}}%
\pgfpathlineto{\pgfqpoint{4.889745in}{3.412390in}}%
\pgfpathlineto{\pgfqpoint{4.894255in}{4.056000in}}%
\pgfpathlineto{\pgfqpoint{4.898764in}{3.475787in}}%
\pgfpathlineto{\pgfqpoint{4.903273in}{3.449601in}}%
\pgfpathlineto{\pgfqpoint{4.907782in}{3.548830in}}%
\pgfpathlineto{\pgfqpoint{4.912291in}{3.503350in}}%
\pgfpathlineto{\pgfqpoint{4.916800in}{3.544696in}}%
\pgfpathlineto{\pgfqpoint{4.921309in}{3.450979in}}%
\pgfpathlineto{\pgfqpoint{4.925818in}{3.463383in}}%
\pgfpathlineto{\pgfqpoint{4.930327in}{3.466139in}}%
\pgfpathlineto{\pgfqpoint{4.934836in}{3.459249in}}%
\pgfpathlineto{\pgfqpoint{4.939345in}{3.481299in}}%
\pgfpathlineto{\pgfqpoint{4.943855in}{3.512998in}}%
\pgfpathlineto{\pgfqpoint{4.952873in}{3.479921in}}%
\pgfpathlineto{\pgfqpoint{4.957382in}{3.473030in}}%
\pgfpathlineto{\pgfqpoint{4.961891in}{3.456492in}}%
\pgfpathlineto{\pgfqpoint{4.966400in}{3.517132in}}%
\pgfpathlineto{\pgfqpoint{4.970909in}{3.471652in}}%
\pgfpathlineto{\pgfqpoint{4.975418in}{3.484056in}}%
\pgfpathlineto{\pgfqpoint{4.979927in}{3.515754in}}%
\pgfpathlineto{\pgfqpoint{4.984436in}{3.537805in}}%
\pgfpathlineto{\pgfqpoint{4.988945in}{3.473030in}}%
\pgfpathlineto{\pgfqpoint{4.997964in}{3.517132in}}%
\pgfpathlineto{\pgfqpoint{5.002473in}{3.510241in}}%
\pgfpathlineto{\pgfqpoint{5.006982in}{3.512998in}}%
\pgfpathlineto{\pgfqpoint{5.011491in}{3.525401in}}%
\pgfpathlineto{\pgfqpoint{5.016000in}{3.557099in}}%
\pgfpathlineto{\pgfqpoint{5.020509in}{3.530914in}}%
\pgfpathlineto{\pgfqpoint{5.025018in}{3.581906in}}%
\pgfpathlineto{\pgfqpoint{5.029527in}{3.497838in}}%
\pgfpathlineto{\pgfqpoint{5.034036in}{3.548830in}}%
\pgfpathlineto{\pgfqpoint{5.038545in}{3.522645in}}%
\pgfpathlineto{\pgfqpoint{5.043055in}{3.544696in}}%
\pgfpathlineto{\pgfqpoint{5.047564in}{3.547452in}}%
\pgfpathlineto{\pgfqpoint{5.052073in}{3.540561in}}%
\pgfpathlineto{\pgfqpoint{5.056582in}{3.495081in}}%
\pgfpathlineto{\pgfqpoint{5.061091in}{3.526779in}}%
\pgfpathlineto{\pgfqpoint{5.065600in}{3.577772in}}%
\pgfpathlineto{\pgfqpoint{5.070109in}{3.551587in}}%
\pgfpathlineto{\pgfqpoint{5.074618in}{3.515754in}}%
\pgfpathlineto{\pgfqpoint{5.079127in}{3.614983in}}%
\pgfpathlineto{\pgfqpoint{5.083636in}{3.540561in}}%
\pgfpathlineto{\pgfqpoint{5.088145in}{3.552965in}}%
\pgfpathlineto{\pgfqpoint{5.092655in}{3.555721in}}%
\pgfpathlineto{\pgfqpoint{5.097164in}{3.539183in}}%
\pgfpathlineto{\pgfqpoint{5.101673in}{3.599823in}}%
\pgfpathlineto{\pgfqpoint{5.106182in}{3.583285in}}%
\pgfpathlineto{\pgfqpoint{5.110691in}{3.586041in}}%
\pgfpathlineto{\pgfqpoint{5.115200in}{3.627386in}}%
\pgfpathlineto{\pgfqpoint{5.119709in}{3.639790in}}%
\pgfpathlineto{\pgfqpoint{5.124218in}{3.613605in}}%
\pgfpathlineto{\pgfqpoint{5.128727in}{3.626008in}}%
\pgfpathlineto{\pgfqpoint{5.133236in}{3.561234in}}%
\pgfpathlineto{\pgfqpoint{5.137745in}{3.592932in}}%
\pgfpathlineto{\pgfqpoint{5.142255in}{3.576394in}}%
\pgfpathlineto{\pgfqpoint{5.146764in}{3.627386in}}%
\pgfpathlineto{\pgfqpoint{5.155782in}{3.594310in}}%
\pgfpathlineto{\pgfqpoint{5.160291in}{3.587419in}}%
\pgfpathlineto{\pgfqpoint{5.169309in}{3.689404in}}%
\pgfpathlineto{\pgfqpoint{5.173818in}{3.692161in}}%
\pgfpathlineto{\pgfqpoint{5.178327in}{3.656328in}}%
\pgfpathlineto{\pgfqpoint{5.182836in}{3.649437in}}%
\pgfpathlineto{\pgfqpoint{5.187345in}{3.661841in}}%
\pgfpathlineto{\pgfqpoint{5.191855in}{3.645303in}}%
\pgfpathlineto{\pgfqpoint{5.196364in}{3.648059in}}%
\pgfpathlineto{\pgfqpoint{5.200873in}{3.631521in}}%
\pgfpathlineto{\pgfqpoint{5.205382in}{3.653572in}}%
\pgfpathlineto{\pgfqpoint{5.209891in}{3.637034in}}%
\pgfpathlineto{\pgfqpoint{5.214400in}{3.736263in}}%
\pgfpathlineto{\pgfqpoint{5.218909in}{3.632899in}}%
\pgfpathlineto{\pgfqpoint{5.223418in}{3.664597in}}%
\pgfpathlineto{\pgfqpoint{5.227927in}{3.667354in}}%
\pgfpathlineto{\pgfqpoint{5.232436in}{3.689404in}}%
\pgfpathlineto{\pgfqpoint{5.236945in}{3.624630in}}%
\pgfpathlineto{\pgfqpoint{5.241455in}{3.694917in}}%
\pgfpathlineto{\pgfqpoint{5.245964in}{3.620495in}}%
\pgfpathlineto{\pgfqpoint{5.250473in}{3.661841in}}%
\pgfpathlineto{\pgfqpoint{5.254982in}{3.616361in}}%
\pgfpathlineto{\pgfqpoint{5.259491in}{3.705943in}}%
\pgfpathlineto{\pgfqpoint{5.264000in}{3.670110in}}%
\pgfpathlineto{\pgfqpoint{5.268509in}{3.663219in}}%
\pgfpathlineto{\pgfqpoint{5.277527in}{3.736263in}}%
\pgfpathlineto{\pgfqpoint{5.282036in}{3.700430in}}%
\pgfpathlineto{\pgfqpoint{5.286545in}{3.683892in}}%
\pgfpathlineto{\pgfqpoint{5.291055in}{3.686648in}}%
\pgfpathlineto{\pgfqpoint{5.295564in}{3.670110in}}%
\pgfpathlineto{\pgfqpoint{5.300073in}{3.701808in}}%
\pgfpathlineto{\pgfqpoint{5.304582in}{3.743153in}}%
\pgfpathlineto{\pgfqpoint{5.309091in}{3.688026in}}%
\pgfpathlineto{\pgfqpoint{5.313600in}{3.710077in}}%
\pgfpathlineto{\pgfqpoint{5.318109in}{3.703186in}}%
\pgfpathlineto{\pgfqpoint{5.322618in}{3.725237in}}%
\pgfpathlineto{\pgfqpoint{5.327127in}{3.756935in}}%
\pgfpathlineto{\pgfqpoint{5.331636in}{3.730750in}}%
\pgfpathlineto{\pgfqpoint{5.336145in}{3.694917in}}%
\pgfpathlineto{\pgfqpoint{5.340655in}{3.697674in}}%
\pgfpathlineto{\pgfqpoint{5.345164in}{3.739019in}}%
\pgfpathlineto{\pgfqpoint{5.349673in}{3.751422in}}%
\pgfpathlineto{\pgfqpoint{5.354182in}{3.783121in}}%
\pgfpathlineto{\pgfqpoint{5.358691in}{3.834113in}}%
\pgfpathlineto{\pgfqpoint{5.363200in}{3.788633in}}%
\pgfpathlineto{\pgfqpoint{5.367709in}{3.714212in}}%
\pgfpathlineto{\pgfqpoint{5.376727in}{3.825844in}}%
\pgfpathlineto{\pgfqpoint{5.381236in}{3.761070in}}%
\pgfpathlineto{\pgfqpoint{5.385745in}{3.754179in}}%
\pgfpathlineto{\pgfqpoint{5.390255in}{3.708699in}}%
\pgfpathlineto{\pgfqpoint{5.394764in}{3.807928in}}%
\pgfpathlineto{\pgfqpoint{5.399273in}{3.752801in}}%
\pgfpathlineto{\pgfqpoint{5.403782in}{3.774852in}}%
\pgfpathlineto{\pgfqpoint{5.408291in}{3.748666in}}%
\pgfpathlineto{\pgfqpoint{5.412800in}{3.712833in}}%
\pgfpathlineto{\pgfqpoint{5.417309in}{3.860299in}}%
\pgfpathlineto{\pgfqpoint{5.421818in}{3.737641in}}%
\pgfpathlineto{\pgfqpoint{5.426327in}{3.769339in}}%
\pgfpathlineto{\pgfqpoint{5.430836in}{3.752801in}}%
\pgfpathlineto{\pgfqpoint{5.435345in}{3.765204in}}%
\pgfpathlineto{\pgfqpoint{5.439855in}{3.796902in}}%
\pgfpathlineto{\pgfqpoint{5.444364in}{3.818953in}}%
\pgfpathlineto{\pgfqpoint{5.448873in}{3.754179in}}%
\pgfpathlineto{\pgfqpoint{5.453382in}{3.824466in}}%
\pgfpathlineto{\pgfqpoint{5.457891in}{3.846517in}}%
\pgfpathlineto{\pgfqpoint{5.462400in}{3.772095in}}%
\pgfpathlineto{\pgfqpoint{5.466909in}{3.880971in}}%
\pgfpathlineto{\pgfqpoint{5.471418in}{3.806550in}}%
\pgfpathlineto{\pgfqpoint{5.475927in}{3.857542in}}%
\pgfpathlineto{\pgfqpoint{5.480436in}{3.821710in}}%
\pgfpathlineto{\pgfqpoint{5.484945in}{3.843760in}}%
\pgfpathlineto{\pgfqpoint{5.489455in}{3.798281in}}%
\pgfpathlineto{\pgfqpoint{5.493964in}{3.868568in}}%
\pgfpathlineto{\pgfqpoint{5.498473in}{3.794146in}}%
\pgfpathlineto{\pgfqpoint{5.502982in}{3.874080in}}%
\pgfpathlineto{\pgfqpoint{5.507491in}{3.799659in}}%
\pgfpathlineto{\pgfqpoint{5.512000in}{3.831357in}}%
\pgfpathlineto{\pgfqpoint{5.516509in}{3.834113in}}%
\pgfpathlineto{\pgfqpoint{5.521018in}{3.778986in}}%
\pgfpathlineto{\pgfqpoint{5.525527in}{3.897509in}}%
\pgfpathlineto{\pgfqpoint{5.530036in}{3.803793in}}%
\pgfpathlineto{\pgfqpoint{5.534545in}{3.951258in}}%
\pgfpathlineto{\pgfqpoint{5.534545in}{3.951258in}}%
\pgfusepath{stroke}%
\end{pgfscope}%
\begin{pgfscope}%
\pgfsetrectcap%
\pgfsetmiterjoin%
\pgfsetlinewidth{0.803000pt}%
\definecolor{currentstroke}{rgb}{0.000000,0.000000,0.000000}%
\pgfsetstrokecolor{currentstroke}%
\pgfsetdash{}{0pt}%
\pgfpathmoveto{\pgfqpoint{0.800000in}{0.528000in}}%
\pgfpathlineto{\pgfqpoint{0.800000in}{4.224000in}}%
\pgfusepath{stroke}%
\end{pgfscope}%
\begin{pgfscope}%
\pgfsetrectcap%
\pgfsetmiterjoin%
\pgfsetlinewidth{0.803000pt}%
\definecolor{currentstroke}{rgb}{0.000000,0.000000,0.000000}%
\pgfsetstrokecolor{currentstroke}%
\pgfsetdash{}{0pt}%
\pgfpathmoveto{\pgfqpoint{5.760000in}{0.528000in}}%
\pgfpathlineto{\pgfqpoint{5.760000in}{4.224000in}}%
\pgfusepath{stroke}%
\end{pgfscope}%
\begin{pgfscope}%
\pgfsetrectcap%
\pgfsetmiterjoin%
\pgfsetlinewidth{0.803000pt}%
\definecolor{currentstroke}{rgb}{0.000000,0.000000,0.000000}%
\pgfsetstrokecolor{currentstroke}%
\pgfsetdash{}{0pt}%
\pgfpathmoveto{\pgfqpoint{0.800000in}{0.528000in}}%
\pgfpathlineto{\pgfqpoint{5.760000in}{0.528000in}}%
\pgfusepath{stroke}%
\end{pgfscope}%
\begin{pgfscope}%
\pgfsetrectcap%
\pgfsetmiterjoin%
\pgfsetlinewidth{0.803000pt}%
\definecolor{currentstroke}{rgb}{0.000000,0.000000,0.000000}%
\pgfsetstrokecolor{currentstroke}%
\pgfsetdash{}{0pt}%
\pgfpathmoveto{\pgfqpoint{0.800000in}{4.224000in}}%
\pgfpathlineto{\pgfqpoint{5.760000in}{4.224000in}}%
\pgfusepath{stroke}%
\end{pgfscope}%
\begin{pgfscope}%
\definecolor{textcolor}{rgb}{0.000000,0.000000,0.000000}%
\pgfsetstrokecolor{textcolor}%
\pgfsetfillcolor{textcolor}%
\pgftext[x=3.280000in,y=4.307333in,,base]{\color{textcolor}\ttfamily\fontsize{12.000000}{14.400000}\selectfont Quick Sort Memory vs Input size}%
\end{pgfscope}%
\begin{pgfscope}%
\pgfsetbuttcap%
\pgfsetmiterjoin%
\definecolor{currentfill}{rgb}{1.000000,1.000000,1.000000}%
\pgfsetfillcolor{currentfill}%
\pgfsetfillopacity{0.800000}%
\pgfsetlinewidth{1.003750pt}%
\definecolor{currentstroke}{rgb}{0.800000,0.800000,0.800000}%
\pgfsetstrokecolor{currentstroke}%
\pgfsetstrokeopacity{0.800000}%
\pgfsetdash{}{0pt}%
\pgfpathmoveto{\pgfqpoint{0.897222in}{3.908286in}}%
\pgfpathlineto{\pgfqpoint{1.759758in}{3.908286in}}%
\pgfpathquadraticcurveto{\pgfqpoint{1.787535in}{3.908286in}}{\pgfqpoint{1.787535in}{3.936063in}}%
\pgfpathlineto{\pgfqpoint{1.787535in}{4.126778in}}%
\pgfpathquadraticcurveto{\pgfqpoint{1.787535in}{4.154556in}}{\pgfqpoint{1.759758in}{4.154556in}}%
\pgfpathlineto{\pgfqpoint{0.897222in}{4.154556in}}%
\pgfpathquadraticcurveto{\pgfqpoint{0.869444in}{4.154556in}}{\pgfqpoint{0.869444in}{4.126778in}}%
\pgfpathlineto{\pgfqpoint{0.869444in}{3.936063in}}%
\pgfpathquadraticcurveto{\pgfqpoint{0.869444in}{3.908286in}}{\pgfqpoint{0.897222in}{3.908286in}}%
\pgfpathlineto{\pgfqpoint{0.897222in}{3.908286in}}%
\pgfpathclose%
\pgfusepath{stroke,fill}%
\end{pgfscope}%
\begin{pgfscope}%
\pgfsetrectcap%
\pgfsetroundjoin%
\pgfsetlinewidth{1.505625pt}%
\definecolor{currentstroke}{rgb}{0.000000,1.000000,0.498039}%
\pgfsetstrokecolor{currentstroke}%
\pgfsetdash{}{0pt}%
\pgfpathmoveto{\pgfqpoint{0.925000in}{4.041342in}}%
\pgfpathlineto{\pgfqpoint{1.063889in}{4.041342in}}%
\pgfpathlineto{\pgfqpoint{1.202778in}{4.041342in}}%
\pgfusepath{stroke}%
\end{pgfscope}%
\begin{pgfscope}%
\definecolor{textcolor}{rgb}{0.000000,0.000000,0.000000}%
\pgfsetstrokecolor{textcolor}%
\pgfsetfillcolor{textcolor}%
\pgftext[x=1.313889in,y=3.992731in,left,base]{\color{textcolor}\ttfamily\fontsize{10.000000}{12.000000}\selectfont Quick}%
\end{pgfscope}%
\end{pgfpicture}%
\makeatother%
\endgroup%

%% Creator: Matplotlib, PGF backend
%%
%% To include the figure in your LaTeX document, write
%%   \input{<filename>.pgf}
%%
%% Make sure the required packages are loaded in your preamble
%%   \usepackage{pgf}
%%
%% Also ensure that all the required font packages are loaded; for instance,
%% the lmodern package is sometimes necessary when using math font.
%%   \usepackage{lmodern}
%%
%% Figures using additional raster images can only be included by \input if
%% they are in the same directory as the main LaTeX file. For loading figures
%% from other directories you can use the `import` package
%%   \usepackage{import}
%%
%% and then include the figures with
%%   \import{<path to file>}{<filename>.pgf}
%%
%% Matplotlib used the following preamble
%%   \usepackage{fontspec}
%%   \setmainfont{DejaVuSerif.ttf}[Path=\detokenize{/home/dbk/.local/lib/python3.10/site-packages/matplotlib/mpl-data/fonts/ttf/}]
%%   \setsansfont{DejaVuSans.ttf}[Path=\detokenize{/home/dbk/.local/lib/python3.10/site-packages/matplotlib/mpl-data/fonts/ttf/}]
%%   \setmonofont{DejaVuSansMono.ttf}[Path=\detokenize{/home/dbk/.local/lib/python3.10/site-packages/matplotlib/mpl-data/fonts/ttf/}]
%%
\begingroup%
\makeatletter%
\begin{pgfpicture}%
\pgfpathrectangle{\pgfpointorigin}{\pgfqpoint{6.400000in}{4.800000in}}%
\pgfusepath{use as bounding box, clip}%
\begin{pgfscope}%
\pgfsetbuttcap%
\pgfsetmiterjoin%
\definecolor{currentfill}{rgb}{1.000000,1.000000,1.000000}%
\pgfsetfillcolor{currentfill}%
\pgfsetlinewidth{0.000000pt}%
\definecolor{currentstroke}{rgb}{1.000000,1.000000,1.000000}%
\pgfsetstrokecolor{currentstroke}%
\pgfsetdash{}{0pt}%
\pgfpathmoveto{\pgfqpoint{0.000000in}{0.000000in}}%
\pgfpathlineto{\pgfqpoint{6.400000in}{0.000000in}}%
\pgfpathlineto{\pgfqpoint{6.400000in}{4.800000in}}%
\pgfpathlineto{\pgfqpoint{0.000000in}{4.800000in}}%
\pgfpathlineto{\pgfqpoint{0.000000in}{0.000000in}}%
\pgfpathclose%
\pgfusepath{fill}%
\end{pgfscope}%
\begin{pgfscope}%
\pgfsetbuttcap%
\pgfsetmiterjoin%
\definecolor{currentfill}{rgb}{1.000000,1.000000,1.000000}%
\pgfsetfillcolor{currentfill}%
\pgfsetlinewidth{0.000000pt}%
\definecolor{currentstroke}{rgb}{0.000000,0.000000,0.000000}%
\pgfsetstrokecolor{currentstroke}%
\pgfsetstrokeopacity{0.000000}%
\pgfsetdash{}{0pt}%
\pgfpathmoveto{\pgfqpoint{0.800000in}{0.528000in}}%
\pgfpathlineto{\pgfqpoint{5.760000in}{0.528000in}}%
\pgfpathlineto{\pgfqpoint{5.760000in}{4.224000in}}%
\pgfpathlineto{\pgfqpoint{0.800000in}{4.224000in}}%
\pgfpathlineto{\pgfqpoint{0.800000in}{0.528000in}}%
\pgfpathclose%
\pgfusepath{fill}%
\end{pgfscope}%
\begin{pgfscope}%
\pgfsetbuttcap%
\pgfsetroundjoin%
\definecolor{currentfill}{rgb}{0.000000,0.000000,0.000000}%
\pgfsetfillcolor{currentfill}%
\pgfsetlinewidth{0.803000pt}%
\definecolor{currentstroke}{rgb}{0.000000,0.000000,0.000000}%
\pgfsetstrokecolor{currentstroke}%
\pgfsetdash{}{0pt}%
\pgfsys@defobject{currentmarker}{\pgfqpoint{0.000000in}{-0.048611in}}{\pgfqpoint{0.000000in}{0.000000in}}{%
\pgfpathmoveto{\pgfqpoint{0.000000in}{0.000000in}}%
\pgfpathlineto{\pgfqpoint{0.000000in}{-0.048611in}}%
\pgfusepath{stroke,fill}%
}%
\begin{pgfscope}%
\pgfsys@transformshift{1.020945in}{0.528000in}%
\pgfsys@useobject{currentmarker}{}%
\end{pgfscope}%
\end{pgfscope}%
\begin{pgfscope}%
\definecolor{textcolor}{rgb}{0.000000,0.000000,0.000000}%
\pgfsetstrokecolor{textcolor}%
\pgfsetfillcolor{textcolor}%
\pgftext[x=1.020945in,y=0.430778in,,top]{\color{textcolor}\ttfamily\fontsize{10.000000}{12.000000}\selectfont 0}%
\end{pgfscope}%
\begin{pgfscope}%
\pgfsetbuttcap%
\pgfsetroundjoin%
\definecolor{currentfill}{rgb}{0.000000,0.000000,0.000000}%
\pgfsetfillcolor{currentfill}%
\pgfsetlinewidth{0.803000pt}%
\definecolor{currentstroke}{rgb}{0.000000,0.000000,0.000000}%
\pgfsetstrokecolor{currentstroke}%
\pgfsetdash{}{0pt}%
\pgfsys@defobject{currentmarker}{\pgfqpoint{0.000000in}{-0.048611in}}{\pgfqpoint{0.000000in}{0.000000in}}{%
\pgfpathmoveto{\pgfqpoint{0.000000in}{0.000000in}}%
\pgfpathlineto{\pgfqpoint{0.000000in}{-0.048611in}}%
\pgfusepath{stroke,fill}%
}%
\begin{pgfscope}%
\pgfsys@transformshift{1.922764in}{0.528000in}%
\pgfsys@useobject{currentmarker}{}%
\end{pgfscope}%
\end{pgfscope}%
\begin{pgfscope}%
\definecolor{textcolor}{rgb}{0.000000,0.000000,0.000000}%
\pgfsetstrokecolor{textcolor}%
\pgfsetfillcolor{textcolor}%
\pgftext[x=1.922764in,y=0.430778in,,top]{\color{textcolor}\ttfamily\fontsize{10.000000}{12.000000}\selectfont 200}%
\end{pgfscope}%
\begin{pgfscope}%
\pgfsetbuttcap%
\pgfsetroundjoin%
\definecolor{currentfill}{rgb}{0.000000,0.000000,0.000000}%
\pgfsetfillcolor{currentfill}%
\pgfsetlinewidth{0.803000pt}%
\definecolor{currentstroke}{rgb}{0.000000,0.000000,0.000000}%
\pgfsetstrokecolor{currentstroke}%
\pgfsetdash{}{0pt}%
\pgfsys@defobject{currentmarker}{\pgfqpoint{0.000000in}{-0.048611in}}{\pgfqpoint{0.000000in}{0.000000in}}{%
\pgfpathmoveto{\pgfqpoint{0.000000in}{0.000000in}}%
\pgfpathlineto{\pgfqpoint{0.000000in}{-0.048611in}}%
\pgfusepath{stroke,fill}%
}%
\begin{pgfscope}%
\pgfsys@transformshift{2.824582in}{0.528000in}%
\pgfsys@useobject{currentmarker}{}%
\end{pgfscope}%
\end{pgfscope}%
\begin{pgfscope}%
\definecolor{textcolor}{rgb}{0.000000,0.000000,0.000000}%
\pgfsetstrokecolor{textcolor}%
\pgfsetfillcolor{textcolor}%
\pgftext[x=2.824582in,y=0.430778in,,top]{\color{textcolor}\ttfamily\fontsize{10.000000}{12.000000}\selectfont 400}%
\end{pgfscope}%
\begin{pgfscope}%
\pgfsetbuttcap%
\pgfsetroundjoin%
\definecolor{currentfill}{rgb}{0.000000,0.000000,0.000000}%
\pgfsetfillcolor{currentfill}%
\pgfsetlinewidth{0.803000pt}%
\definecolor{currentstroke}{rgb}{0.000000,0.000000,0.000000}%
\pgfsetstrokecolor{currentstroke}%
\pgfsetdash{}{0pt}%
\pgfsys@defobject{currentmarker}{\pgfqpoint{0.000000in}{-0.048611in}}{\pgfqpoint{0.000000in}{0.000000in}}{%
\pgfpathmoveto{\pgfqpoint{0.000000in}{0.000000in}}%
\pgfpathlineto{\pgfqpoint{0.000000in}{-0.048611in}}%
\pgfusepath{stroke,fill}%
}%
\begin{pgfscope}%
\pgfsys@transformshift{3.726400in}{0.528000in}%
\pgfsys@useobject{currentmarker}{}%
\end{pgfscope}%
\end{pgfscope}%
\begin{pgfscope}%
\definecolor{textcolor}{rgb}{0.000000,0.000000,0.000000}%
\pgfsetstrokecolor{textcolor}%
\pgfsetfillcolor{textcolor}%
\pgftext[x=3.726400in,y=0.430778in,,top]{\color{textcolor}\ttfamily\fontsize{10.000000}{12.000000}\selectfont 600}%
\end{pgfscope}%
\begin{pgfscope}%
\pgfsetbuttcap%
\pgfsetroundjoin%
\definecolor{currentfill}{rgb}{0.000000,0.000000,0.000000}%
\pgfsetfillcolor{currentfill}%
\pgfsetlinewidth{0.803000pt}%
\definecolor{currentstroke}{rgb}{0.000000,0.000000,0.000000}%
\pgfsetstrokecolor{currentstroke}%
\pgfsetdash{}{0pt}%
\pgfsys@defobject{currentmarker}{\pgfqpoint{0.000000in}{-0.048611in}}{\pgfqpoint{0.000000in}{0.000000in}}{%
\pgfpathmoveto{\pgfqpoint{0.000000in}{0.000000in}}%
\pgfpathlineto{\pgfqpoint{0.000000in}{-0.048611in}}%
\pgfusepath{stroke,fill}%
}%
\begin{pgfscope}%
\pgfsys@transformshift{4.628218in}{0.528000in}%
\pgfsys@useobject{currentmarker}{}%
\end{pgfscope}%
\end{pgfscope}%
\begin{pgfscope}%
\definecolor{textcolor}{rgb}{0.000000,0.000000,0.000000}%
\pgfsetstrokecolor{textcolor}%
\pgfsetfillcolor{textcolor}%
\pgftext[x=4.628218in,y=0.430778in,,top]{\color{textcolor}\ttfamily\fontsize{10.000000}{12.000000}\selectfont 800}%
\end{pgfscope}%
\begin{pgfscope}%
\pgfsetbuttcap%
\pgfsetroundjoin%
\definecolor{currentfill}{rgb}{0.000000,0.000000,0.000000}%
\pgfsetfillcolor{currentfill}%
\pgfsetlinewidth{0.803000pt}%
\definecolor{currentstroke}{rgb}{0.000000,0.000000,0.000000}%
\pgfsetstrokecolor{currentstroke}%
\pgfsetdash{}{0pt}%
\pgfsys@defobject{currentmarker}{\pgfqpoint{0.000000in}{-0.048611in}}{\pgfqpoint{0.000000in}{0.000000in}}{%
\pgfpathmoveto{\pgfqpoint{0.000000in}{0.000000in}}%
\pgfpathlineto{\pgfqpoint{0.000000in}{-0.048611in}}%
\pgfusepath{stroke,fill}%
}%
\begin{pgfscope}%
\pgfsys@transformshift{5.530036in}{0.528000in}%
\pgfsys@useobject{currentmarker}{}%
\end{pgfscope}%
\end{pgfscope}%
\begin{pgfscope}%
\definecolor{textcolor}{rgb}{0.000000,0.000000,0.000000}%
\pgfsetstrokecolor{textcolor}%
\pgfsetfillcolor{textcolor}%
\pgftext[x=5.530036in,y=0.430778in,,top]{\color{textcolor}\ttfamily\fontsize{10.000000}{12.000000}\selectfont 1000}%
\end{pgfscope}%
\begin{pgfscope}%
\definecolor{textcolor}{rgb}{0.000000,0.000000,0.000000}%
\pgfsetstrokecolor{textcolor}%
\pgfsetfillcolor{textcolor}%
\pgftext[x=3.280000in,y=0.240063in,,top]{\color{textcolor}\ttfamily\fontsize{10.000000}{12.000000}\selectfont Size of Array}%
\end{pgfscope}%
\begin{pgfscope}%
\pgfsetbuttcap%
\pgfsetroundjoin%
\definecolor{currentfill}{rgb}{0.000000,0.000000,0.000000}%
\pgfsetfillcolor{currentfill}%
\pgfsetlinewidth{0.803000pt}%
\definecolor{currentstroke}{rgb}{0.000000,0.000000,0.000000}%
\pgfsetstrokecolor{currentstroke}%
\pgfsetdash{}{0pt}%
\pgfsys@defobject{currentmarker}{\pgfqpoint{-0.048611in}{0.000000in}}{\pgfqpoint{-0.000000in}{0.000000in}}{%
\pgfpathmoveto{\pgfqpoint{-0.000000in}{0.000000in}}%
\pgfpathlineto{\pgfqpoint{-0.048611in}{0.000000in}}%
\pgfusepath{stroke,fill}%
}%
\begin{pgfscope}%
\pgfsys@transformshift{0.800000in}{0.955521in}%
\pgfsys@useobject{currentmarker}{}%
\end{pgfscope}%
\end{pgfscope}%
\begin{pgfscope}%
\definecolor{textcolor}{rgb}{0.000000,0.000000,0.000000}%
\pgfsetstrokecolor{textcolor}%
\pgfsetfillcolor{textcolor}%
\pgftext[x=0.368305in, y=0.902386in, left, base]{\color{textcolor}\ttfamily\fontsize{10.000000}{12.000000}\selectfont 2500}%
\end{pgfscope}%
\begin{pgfscope}%
\pgfsetbuttcap%
\pgfsetroundjoin%
\definecolor{currentfill}{rgb}{0.000000,0.000000,0.000000}%
\pgfsetfillcolor{currentfill}%
\pgfsetlinewidth{0.803000pt}%
\definecolor{currentstroke}{rgb}{0.000000,0.000000,0.000000}%
\pgfsetstrokecolor{currentstroke}%
\pgfsetdash{}{0pt}%
\pgfsys@defobject{currentmarker}{\pgfqpoint{-0.048611in}{0.000000in}}{\pgfqpoint{-0.000000in}{0.000000in}}{%
\pgfpathmoveto{\pgfqpoint{-0.000000in}{0.000000in}}%
\pgfpathlineto{\pgfqpoint{-0.048611in}{0.000000in}}%
\pgfusepath{stroke,fill}%
}%
\begin{pgfscope}%
\pgfsys@transformshift{0.800000in}{1.402662in}%
\pgfsys@useobject{currentmarker}{}%
\end{pgfscope}%
\end{pgfscope}%
\begin{pgfscope}%
\definecolor{textcolor}{rgb}{0.000000,0.000000,0.000000}%
\pgfsetstrokecolor{textcolor}%
\pgfsetfillcolor{textcolor}%
\pgftext[x=0.368305in, y=1.349528in, left, base]{\color{textcolor}\ttfamily\fontsize{10.000000}{12.000000}\selectfont 5000}%
\end{pgfscope}%
\begin{pgfscope}%
\pgfsetbuttcap%
\pgfsetroundjoin%
\definecolor{currentfill}{rgb}{0.000000,0.000000,0.000000}%
\pgfsetfillcolor{currentfill}%
\pgfsetlinewidth{0.803000pt}%
\definecolor{currentstroke}{rgb}{0.000000,0.000000,0.000000}%
\pgfsetstrokecolor{currentstroke}%
\pgfsetdash{}{0pt}%
\pgfsys@defobject{currentmarker}{\pgfqpoint{-0.048611in}{0.000000in}}{\pgfqpoint{-0.000000in}{0.000000in}}{%
\pgfpathmoveto{\pgfqpoint{-0.000000in}{0.000000in}}%
\pgfpathlineto{\pgfqpoint{-0.048611in}{0.000000in}}%
\pgfusepath{stroke,fill}%
}%
\begin{pgfscope}%
\pgfsys@transformshift{0.800000in}{1.849804in}%
\pgfsys@useobject{currentmarker}{}%
\end{pgfscope}%
\end{pgfscope}%
\begin{pgfscope}%
\definecolor{textcolor}{rgb}{0.000000,0.000000,0.000000}%
\pgfsetstrokecolor{textcolor}%
\pgfsetfillcolor{textcolor}%
\pgftext[x=0.368305in, y=1.796669in, left, base]{\color{textcolor}\ttfamily\fontsize{10.000000}{12.000000}\selectfont 7500}%
\end{pgfscope}%
\begin{pgfscope}%
\pgfsetbuttcap%
\pgfsetroundjoin%
\definecolor{currentfill}{rgb}{0.000000,0.000000,0.000000}%
\pgfsetfillcolor{currentfill}%
\pgfsetlinewidth{0.803000pt}%
\definecolor{currentstroke}{rgb}{0.000000,0.000000,0.000000}%
\pgfsetstrokecolor{currentstroke}%
\pgfsetdash{}{0pt}%
\pgfsys@defobject{currentmarker}{\pgfqpoint{-0.048611in}{0.000000in}}{\pgfqpoint{-0.000000in}{0.000000in}}{%
\pgfpathmoveto{\pgfqpoint{-0.000000in}{0.000000in}}%
\pgfpathlineto{\pgfqpoint{-0.048611in}{0.000000in}}%
\pgfusepath{stroke,fill}%
}%
\begin{pgfscope}%
\pgfsys@transformshift{0.800000in}{2.296945in}%
\pgfsys@useobject{currentmarker}{}%
\end{pgfscope}%
\end{pgfscope}%
\begin{pgfscope}%
\definecolor{textcolor}{rgb}{0.000000,0.000000,0.000000}%
\pgfsetstrokecolor{textcolor}%
\pgfsetfillcolor{textcolor}%
\pgftext[x=0.284687in, y=2.243811in, left, base]{\color{textcolor}\ttfamily\fontsize{10.000000}{12.000000}\selectfont 10000}%
\end{pgfscope}%
\begin{pgfscope}%
\pgfsetbuttcap%
\pgfsetroundjoin%
\definecolor{currentfill}{rgb}{0.000000,0.000000,0.000000}%
\pgfsetfillcolor{currentfill}%
\pgfsetlinewidth{0.803000pt}%
\definecolor{currentstroke}{rgb}{0.000000,0.000000,0.000000}%
\pgfsetstrokecolor{currentstroke}%
\pgfsetdash{}{0pt}%
\pgfsys@defobject{currentmarker}{\pgfqpoint{-0.048611in}{0.000000in}}{\pgfqpoint{-0.000000in}{0.000000in}}{%
\pgfpathmoveto{\pgfqpoint{-0.000000in}{0.000000in}}%
\pgfpathlineto{\pgfqpoint{-0.048611in}{0.000000in}}%
\pgfusepath{stroke,fill}%
}%
\begin{pgfscope}%
\pgfsys@transformshift{0.800000in}{2.744087in}%
\pgfsys@useobject{currentmarker}{}%
\end{pgfscope}%
\end{pgfscope}%
\begin{pgfscope}%
\definecolor{textcolor}{rgb}{0.000000,0.000000,0.000000}%
\pgfsetstrokecolor{textcolor}%
\pgfsetfillcolor{textcolor}%
\pgftext[x=0.284687in, y=2.690952in, left, base]{\color{textcolor}\ttfamily\fontsize{10.000000}{12.000000}\selectfont 12500}%
\end{pgfscope}%
\begin{pgfscope}%
\pgfsetbuttcap%
\pgfsetroundjoin%
\definecolor{currentfill}{rgb}{0.000000,0.000000,0.000000}%
\pgfsetfillcolor{currentfill}%
\pgfsetlinewidth{0.803000pt}%
\definecolor{currentstroke}{rgb}{0.000000,0.000000,0.000000}%
\pgfsetstrokecolor{currentstroke}%
\pgfsetdash{}{0pt}%
\pgfsys@defobject{currentmarker}{\pgfqpoint{-0.048611in}{0.000000in}}{\pgfqpoint{-0.000000in}{0.000000in}}{%
\pgfpathmoveto{\pgfqpoint{-0.000000in}{0.000000in}}%
\pgfpathlineto{\pgfqpoint{-0.048611in}{0.000000in}}%
\pgfusepath{stroke,fill}%
}%
\begin{pgfscope}%
\pgfsys@transformshift{0.800000in}{3.191228in}%
\pgfsys@useobject{currentmarker}{}%
\end{pgfscope}%
\end{pgfscope}%
\begin{pgfscope}%
\definecolor{textcolor}{rgb}{0.000000,0.000000,0.000000}%
\pgfsetstrokecolor{textcolor}%
\pgfsetfillcolor{textcolor}%
\pgftext[x=0.284687in, y=3.138094in, left, base]{\color{textcolor}\ttfamily\fontsize{10.000000}{12.000000}\selectfont 15000}%
\end{pgfscope}%
\begin{pgfscope}%
\pgfsetbuttcap%
\pgfsetroundjoin%
\definecolor{currentfill}{rgb}{0.000000,0.000000,0.000000}%
\pgfsetfillcolor{currentfill}%
\pgfsetlinewidth{0.803000pt}%
\definecolor{currentstroke}{rgb}{0.000000,0.000000,0.000000}%
\pgfsetstrokecolor{currentstroke}%
\pgfsetdash{}{0pt}%
\pgfsys@defobject{currentmarker}{\pgfqpoint{-0.048611in}{0.000000in}}{\pgfqpoint{-0.000000in}{0.000000in}}{%
\pgfpathmoveto{\pgfqpoint{-0.000000in}{0.000000in}}%
\pgfpathlineto{\pgfqpoint{-0.048611in}{0.000000in}}%
\pgfusepath{stroke,fill}%
}%
\begin{pgfscope}%
\pgfsys@transformshift{0.800000in}{3.638370in}%
\pgfsys@useobject{currentmarker}{}%
\end{pgfscope}%
\end{pgfscope}%
\begin{pgfscope}%
\definecolor{textcolor}{rgb}{0.000000,0.000000,0.000000}%
\pgfsetstrokecolor{textcolor}%
\pgfsetfillcolor{textcolor}%
\pgftext[x=0.284687in, y=3.585235in, left, base]{\color{textcolor}\ttfamily\fontsize{10.000000}{12.000000}\selectfont 17500}%
\end{pgfscope}%
\begin{pgfscope}%
\pgfsetbuttcap%
\pgfsetroundjoin%
\definecolor{currentfill}{rgb}{0.000000,0.000000,0.000000}%
\pgfsetfillcolor{currentfill}%
\pgfsetlinewidth{0.803000pt}%
\definecolor{currentstroke}{rgb}{0.000000,0.000000,0.000000}%
\pgfsetstrokecolor{currentstroke}%
\pgfsetdash{}{0pt}%
\pgfsys@defobject{currentmarker}{\pgfqpoint{-0.048611in}{0.000000in}}{\pgfqpoint{-0.000000in}{0.000000in}}{%
\pgfpathmoveto{\pgfqpoint{-0.000000in}{0.000000in}}%
\pgfpathlineto{\pgfqpoint{-0.048611in}{0.000000in}}%
\pgfusepath{stroke,fill}%
}%
\begin{pgfscope}%
\pgfsys@transformshift{0.800000in}{4.085511in}%
\pgfsys@useobject{currentmarker}{}%
\end{pgfscope}%
\end{pgfscope}%
\begin{pgfscope}%
\definecolor{textcolor}{rgb}{0.000000,0.000000,0.000000}%
\pgfsetstrokecolor{textcolor}%
\pgfsetfillcolor{textcolor}%
\pgftext[x=0.284687in, y=4.032377in, left, base]{\color{textcolor}\ttfamily\fontsize{10.000000}{12.000000}\selectfont 20000}%
\end{pgfscope}%
\begin{pgfscope}%
\definecolor{textcolor}{rgb}{0.000000,0.000000,0.000000}%
\pgfsetstrokecolor{textcolor}%
\pgfsetfillcolor{textcolor}%
\pgftext[x=0.229131in,y=2.376000in,,bottom,rotate=90.000000]{\color{textcolor}\ttfamily\fontsize{10.000000}{12.000000}\selectfont Comparisons}%
\end{pgfscope}%
\begin{pgfscope}%
\pgfpathrectangle{\pgfqpoint{0.800000in}{0.528000in}}{\pgfqpoint{4.960000in}{3.696000in}}%
\pgfusepath{clip}%
\pgfsetrectcap%
\pgfsetroundjoin%
\pgfsetlinewidth{1.505625pt}%
\definecolor{currentstroke}{rgb}{0.000000,1.000000,0.498039}%
\pgfsetstrokecolor{currentstroke}%
\pgfsetdash{}{0pt}%
\pgfpathmoveto{\pgfqpoint{1.025455in}{0.696000in}}%
\pgfpathlineto{\pgfqpoint{1.029964in}{0.697789in}}%
\pgfpathlineto{\pgfqpoint{1.034473in}{0.703333in}}%
\pgfpathlineto{\pgfqpoint{1.038982in}{0.713349in}}%
\pgfpathlineto{\pgfqpoint{1.043491in}{0.705301in}}%
\pgfpathlineto{\pgfqpoint{1.048000in}{0.712634in}}%
\pgfpathlineto{\pgfqpoint{1.052509in}{0.710666in}}%
\pgfpathlineto{\pgfqpoint{1.057018in}{0.717463in}}%
\pgfpathlineto{\pgfqpoint{1.061527in}{0.718178in}}%
\pgfpathlineto{\pgfqpoint{1.066036in}{0.728015in}}%
\pgfpathlineto{\pgfqpoint{1.070545in}{0.730698in}}%
\pgfpathlineto{\pgfqpoint{1.075055in}{0.736958in}}%
\pgfpathlineto{\pgfqpoint{1.079564in}{0.734633in}}%
\pgfpathlineto{\pgfqpoint{1.084073in}{0.743934in}}%
\pgfpathlineto{\pgfqpoint{1.088582in}{0.740356in}}%
\pgfpathlineto{\pgfqpoint{1.093091in}{0.747511in}}%
\pgfpathlineto{\pgfqpoint{1.097600in}{0.751624in}}%
\pgfpathlineto{\pgfqpoint{1.102109in}{0.736779in}}%
\pgfpathlineto{\pgfqpoint{1.106618in}{0.756811in}}%
\pgfpathlineto{\pgfqpoint{1.111127in}{0.742503in}}%
\pgfpathlineto{\pgfqpoint{1.115636in}{0.741787in}}%
\pgfpathlineto{\pgfqpoint{1.120145in}{0.753234in}}%
\pgfpathlineto{\pgfqpoint{1.124655in}{0.760210in}}%
\pgfpathlineto{\pgfqpoint{1.129164in}{0.759494in}}%
\pgfpathlineto{\pgfqpoint{1.133673in}{0.793298in}}%
\pgfpathlineto{\pgfqpoint{1.138182in}{0.757884in}}%
\pgfpathlineto{\pgfqpoint{1.142691in}{0.776843in}}%
\pgfpathlineto{\pgfqpoint{1.147200in}{0.768258in}}%
\pgfpathlineto{\pgfqpoint{1.151709in}{0.784713in}}%
\pgfpathlineto{\pgfqpoint{1.156218in}{0.776664in}}%
\pgfpathlineto{\pgfqpoint{1.160727in}{0.786323in}}%
\pgfpathlineto{\pgfqpoint{1.165236in}{0.779884in}}%
\pgfpathlineto{\pgfqpoint{1.169745in}{0.806176in}}%
\pgfpathlineto{\pgfqpoint{1.174255in}{0.786680in}}%
\pgfpathlineto{\pgfqpoint{1.178764in}{0.785607in}}%
\pgfpathlineto{\pgfqpoint{1.183273in}{0.798485in}}%
\pgfpathlineto{\pgfqpoint{1.187782in}{0.796875in}}%
\pgfpathlineto{\pgfqpoint{1.192291in}{0.806891in}}%
\pgfpathlineto{\pgfqpoint{1.196800in}{0.792046in}}%
\pgfpathlineto{\pgfqpoint{1.201309in}{0.786144in}}%
\pgfpathlineto{\pgfqpoint{1.205818in}{0.796339in}}%
\pgfpathlineto{\pgfqpoint{1.210327in}{0.802420in}}%
\pgfpathlineto{\pgfqpoint{1.214836in}{0.817444in}}%
\pgfpathlineto{\pgfqpoint{1.219345in}{0.797054in}}%
\pgfpathlineto{\pgfqpoint{1.223855in}{0.820305in}}%
\pgfpathlineto{\pgfqpoint{1.228364in}{0.817980in}}%
\pgfpathlineto{\pgfqpoint{1.232873in}{0.835329in}}%
\pgfpathlineto{\pgfqpoint{1.237382in}{0.812972in}}%
\pgfpathlineto{\pgfqpoint{1.241891in}{0.818517in}}%
\pgfpathlineto{\pgfqpoint{1.246400in}{0.834077in}}%
\pgfpathlineto{\pgfqpoint{1.250909in}{0.838191in}}%
\pgfpathlineto{\pgfqpoint{1.255418in}{0.844988in}}%
\pgfpathlineto{\pgfqpoint{1.259927in}{0.813330in}}%
\pgfpathlineto{\pgfqpoint{1.264436in}{0.836224in}}%
\pgfpathlineto{\pgfqpoint{1.268945in}{0.826565in}}%
\pgfpathlineto{\pgfqpoint{1.273455in}{0.840158in}}%
\pgfpathlineto{\pgfqpoint{1.277964in}{0.862694in}}%
\pgfpathlineto{\pgfqpoint{1.282473in}{0.869670in}}%
\pgfpathlineto{\pgfqpoint{1.286982in}{0.870743in}}%
\pgfpathlineto{\pgfqpoint{1.291491in}{0.875572in}}%
\pgfpathlineto{\pgfqpoint{1.296000in}{0.861263in}}%
\pgfpathlineto{\pgfqpoint{1.300509in}{0.900970in}}%
\pgfpathlineto{\pgfqpoint{1.305018in}{0.846597in}}%
\pgfpathlineto{\pgfqpoint{1.309527in}{0.865198in}}%
\pgfpathlineto{\pgfqpoint{1.314036in}{0.875751in}}%
\pgfpathlineto{\pgfqpoint{1.318545in}{0.855719in}}%
\pgfpathlineto{\pgfqpoint{1.323055in}{0.870743in}}%
\pgfpathlineto{\pgfqpoint{1.327564in}{0.854825in}}%
\pgfpathlineto{\pgfqpoint{1.332073in}{0.933879in}}%
\pgfpathlineto{\pgfqpoint{1.336582in}{0.920286in}}%
\pgfpathlineto{\pgfqpoint{1.341091in}{0.902043in}}%
\pgfpathlineto{\pgfqpoint{1.345600in}{0.876109in}}%
\pgfpathlineto{\pgfqpoint{1.354618in}{0.883621in}}%
\pgfpathlineto{\pgfqpoint{1.359127in}{0.946578in}}%
\pgfpathlineto{\pgfqpoint{1.363636in}{0.893279in}}%
\pgfpathlineto{\pgfqpoint{1.368145in}{0.901864in}}%
\pgfpathlineto{\pgfqpoint{1.372655in}{0.914563in}}%
\pgfpathlineto{\pgfqpoint{1.377164in}{0.901327in}}%
\pgfpathlineto{\pgfqpoint{1.381673in}{0.894531in}}%
\pgfpathlineto{\pgfqpoint{1.386182in}{0.903295in}}%
\pgfpathlineto{\pgfqpoint{1.390691in}{0.924042in}}%
\pgfpathlineto{\pgfqpoint{1.395200in}{0.918319in}}%
\pgfpathlineto{\pgfqpoint{1.404218in}{0.892921in}}%
\pgfpathlineto{\pgfqpoint{1.408727in}{0.937456in}}%
\pgfpathlineto{\pgfqpoint{1.413236in}{0.916530in}}%
\pgfpathlineto{\pgfqpoint{1.417745in}{0.924221in}}%
\pgfpathlineto{\pgfqpoint{1.422255in}{0.928692in}}%
\pgfpathlineto{\pgfqpoint{1.426764in}{0.967147in}}%
\pgfpathlineto{\pgfqpoint{1.431273in}{0.918855in}}%
\pgfpathlineto{\pgfqpoint{1.435782in}{0.939424in}}%
\pgfpathlineto{\pgfqpoint{1.440291in}{0.945684in}}%
\pgfpathlineto{\pgfqpoint{1.444800in}{0.927977in}}%
\pgfpathlineto{\pgfqpoint{1.449309in}{0.962675in}}%
\pgfpathlineto{\pgfqpoint{1.453818in}{0.951765in}}%
\pgfpathlineto{\pgfqpoint{1.458327in}{0.949440in}}%
\pgfpathlineto{\pgfqpoint{1.462836in}{0.972691in}}%
\pgfpathlineto{\pgfqpoint{1.467345in}{0.973407in}}%
\pgfpathlineto{\pgfqpoint{1.471855in}{0.985569in}}%
\pgfpathlineto{\pgfqpoint{1.476364in}{0.964643in}}%
\pgfpathlineto{\pgfqpoint{1.480873in}{0.979667in}}%
\pgfpathlineto{\pgfqpoint{1.489891in}{0.952480in}}%
\pgfpathlineto{\pgfqpoint{1.494400in}{0.952838in}}%
\pgfpathlineto{\pgfqpoint{1.498909in}{1.002739in}}%
\pgfpathlineto{\pgfqpoint{1.503418in}{1.002202in}}%
\pgfpathlineto{\pgfqpoint{1.507927in}{0.995227in}}%
\pgfpathlineto{\pgfqpoint{1.516945in}{0.992365in}}%
\pgfpathlineto{\pgfqpoint{1.521455in}{1.017048in}}%
\pgfpathlineto{\pgfqpoint{1.525964in}{0.976626in}}%
\pgfpathlineto{\pgfqpoint{1.530473in}{0.979488in}}%
\pgfpathlineto{\pgfqpoint{1.534982in}{1.023308in}}%
\pgfpathlineto{\pgfqpoint{1.539491in}{1.008284in}}%
\pgfpathlineto{\pgfqpoint{1.544000in}{1.007568in}}%
\pgfpathlineto{\pgfqpoint{1.548509in}{1.020088in}}%
\pgfpathlineto{\pgfqpoint{1.553018in}{1.016869in}}%
\pgfpathlineto{\pgfqpoint{1.557527in}{0.998447in}}%
\pgfpathlineto{\pgfqpoint{1.562036in}{1.030820in}}%
\pgfpathlineto{\pgfqpoint{1.571055in}{1.044055in}}%
\pgfpathlineto{\pgfqpoint{1.575564in}{1.034397in}}%
\pgfpathlineto{\pgfqpoint{1.580073in}{1.017405in}}%
\pgfpathlineto{\pgfqpoint{1.584582in}{1.020088in}}%
\pgfpathlineto{\pgfqpoint{1.593600in}{1.064266in}}%
\pgfpathlineto{\pgfqpoint{1.598109in}{1.061225in}}%
\pgfpathlineto{\pgfqpoint{1.602618in}{1.025096in}}%
\pgfpathlineto{\pgfqpoint{1.607127in}{1.078574in}}%
\pgfpathlineto{\pgfqpoint{1.611636in}{1.064981in}}%
\pgfpathlineto{\pgfqpoint{1.616145in}{1.047095in}}%
\pgfpathlineto{\pgfqpoint{1.620655in}{1.131874in}}%
\pgfpathlineto{\pgfqpoint{1.625164in}{1.041551in}}%
\pgfpathlineto{\pgfqpoint{1.629673in}{1.045128in}}%
\pgfpathlineto{\pgfqpoint{1.634182in}{1.065160in}}%
\pgfpathlineto{\pgfqpoint{1.638691in}{1.127223in}}%
\pgfpathlineto{\pgfqpoint{1.643200in}{1.049778in}}%
\pgfpathlineto{\pgfqpoint{1.647709in}{1.049778in}}%
\pgfpathlineto{\pgfqpoint{1.652218in}{1.071599in}}%
\pgfpathlineto{\pgfqpoint{1.656727in}{1.073030in}}%
\pgfpathlineto{\pgfqpoint{1.661236in}{1.070705in}}%
\pgfpathlineto{\pgfqpoint{1.665745in}{1.076249in}}%
\pgfpathlineto{\pgfqpoint{1.670255in}{1.059615in}}%
\pgfpathlineto{\pgfqpoint{1.674764in}{1.071062in}}%
\pgfpathlineto{\pgfqpoint{1.679273in}{1.089127in}}%
\pgfpathlineto{\pgfqpoint{1.683782in}{1.080363in}}%
\pgfpathlineto{\pgfqpoint{1.688291in}{1.081436in}}%
\pgfpathlineto{\pgfqpoint{1.692800in}{1.085907in}}%
\pgfpathlineto{\pgfqpoint{1.697309in}{1.105045in}}%
\pgfpathlineto{\pgfqpoint{1.701818in}{1.069810in}}%
\pgfpathlineto{\pgfqpoint{1.706327in}{1.140995in}}%
\pgfpathlineto{\pgfqpoint{1.710836in}{1.110768in}}%
\pgfpathlineto{\pgfqpoint{1.715345in}{1.105045in}}%
\pgfpathlineto{\pgfqpoint{1.719855in}{1.122573in}}%
\pgfpathlineto{\pgfqpoint{1.724364in}{1.107907in}}%
\pgfpathlineto{\pgfqpoint{1.728873in}{1.120963in}}%
\pgfpathlineto{\pgfqpoint{1.733382in}{1.117565in}}%
\pgfpathlineto{\pgfqpoint{1.737891in}{1.128654in}}%
\pgfpathlineto{\pgfqpoint{1.742400in}{1.166751in}}%
\pgfpathlineto{\pgfqpoint{1.746909in}{1.126508in}}%
\pgfpathlineto{\pgfqpoint{1.751418in}{1.105939in}}%
\pgfpathlineto{\pgfqpoint{1.755927in}{1.143857in}}%
\pgfpathlineto{\pgfqpoint{1.760436in}{1.123646in}}%
\pgfpathlineto{\pgfqpoint{1.764945in}{1.145288in}}%
\pgfpathlineto{\pgfqpoint{1.769455in}{1.186782in}}%
\pgfpathlineto{\pgfqpoint{1.773964in}{1.158165in}}%
\pgfpathlineto{\pgfqpoint{1.778473in}{1.172116in}}%
\pgfpathlineto{\pgfqpoint{1.782982in}{1.130443in}}%
\pgfpathlineto{\pgfqpoint{1.787491in}{1.190717in}}%
\pgfpathlineto{\pgfqpoint{1.792000in}{1.183921in}}%
\pgfpathlineto{\pgfqpoint{1.796509in}{1.165320in}}%
\pgfpathlineto{\pgfqpoint{1.801018in}{1.160848in}}%
\pgfpathlineto{\pgfqpoint{1.805527in}{1.166751in}}%
\pgfpathlineto{\pgfqpoint{1.810036in}{1.183205in}}%
\pgfpathlineto{\pgfqpoint{1.814545in}{1.177303in}}%
\pgfpathlineto{\pgfqpoint{1.819055in}{1.169076in}}%
\pgfpathlineto{\pgfqpoint{1.823564in}{1.211286in}}%
\pgfpathlineto{\pgfqpoint{1.828073in}{1.176588in}}%
\pgfpathlineto{\pgfqpoint{1.832582in}{1.179628in}}%
\pgfpathlineto{\pgfqpoint{1.837091in}{1.197872in}}%
\pgfpathlineto{\pgfqpoint{1.841600in}{1.168897in}}%
\pgfpathlineto{\pgfqpoint{1.850618in}{1.211107in}}%
\pgfpathlineto{\pgfqpoint{1.855127in}{1.208066in}}%
\pgfpathlineto{\pgfqpoint{1.859636in}{1.194473in}}%
\pgfpathlineto{\pgfqpoint{1.864145in}{1.213790in}}%
\pgfpathlineto{\pgfqpoint{1.868655in}{1.216473in}}%
\pgfpathlineto{\pgfqpoint{1.873164in}{1.192506in}}%
\pgfpathlineto{\pgfqpoint{1.877673in}{1.194652in}}%
\pgfpathlineto{\pgfqpoint{1.882182in}{1.234895in}}%
\pgfpathlineto{\pgfqpoint{1.891200in}{1.230245in}}%
\pgfpathlineto{\pgfqpoint{1.895709in}{1.196262in}}%
\pgfpathlineto{\pgfqpoint{1.900218in}{1.262975in}}%
\pgfpathlineto{\pgfqpoint{1.904727in}{1.276211in}}%
\pgfpathlineto{\pgfqpoint{1.909236in}{1.216294in}}%
\pgfpathlineto{\pgfqpoint{1.913745in}{1.238293in}}%
\pgfpathlineto{\pgfqpoint{1.918255in}{1.242943in}}%
\pgfpathlineto{\pgfqpoint{1.922764in}{1.225416in}}%
\pgfpathlineto{\pgfqpoint{1.927273in}{1.235968in}}%
\pgfpathlineto{\pgfqpoint{1.931782in}{1.269235in}}%
\pgfpathlineto{\pgfqpoint{1.936291in}{1.220586in}}%
\pgfpathlineto{\pgfqpoint{1.940800in}{1.274780in}}%
\pgfpathlineto{\pgfqpoint{1.945309in}{1.275317in}}%
\pgfpathlineto{\pgfqpoint{1.949818in}{1.255285in}}%
\pgfpathlineto{\pgfqpoint{1.954327in}{1.306438in}}%
\pgfpathlineto{\pgfqpoint{1.958836in}{1.248667in}}%
\pgfpathlineto{\pgfqpoint{1.963345in}{1.249919in}}%
\pgfpathlineto{\pgfqpoint{1.967855in}{1.268699in}}%
\pgfpathlineto{\pgfqpoint{1.972364in}{1.331835in}}%
\pgfpathlineto{\pgfqpoint{1.976873in}{1.271203in}}%
\pgfpathlineto{\pgfqpoint{1.981382in}{1.338632in}}%
\pgfpathlineto{\pgfqpoint{1.990400in}{1.268341in}}%
\pgfpathlineto{\pgfqpoint{1.994909in}{1.297495in}}%
\pgfpathlineto{\pgfqpoint{1.999418in}{1.265479in}}%
\pgfpathlineto{\pgfqpoint{2.003927in}{1.305543in}}%
\pgfpathlineto{\pgfqpoint{2.008436in}{1.289267in}}%
\pgfpathlineto{\pgfqpoint{2.017455in}{1.313771in}}%
\pgfpathlineto{\pgfqpoint{2.021964in}{1.330583in}}%
\pgfpathlineto{\pgfqpoint{2.026473in}{1.295885in}}%
\pgfpathlineto{\pgfqpoint{2.030982in}{1.374403in}}%
\pgfpathlineto{\pgfqpoint{2.035491in}{1.300893in}}%
\pgfpathlineto{\pgfqpoint{2.040000in}{1.288910in}}%
\pgfpathlineto{\pgfqpoint{2.044509in}{1.301072in}}%
\pgfpathlineto{\pgfqpoint{2.049018in}{1.412321in}}%
\pgfpathlineto{\pgfqpoint{2.053527in}{1.378517in}}%
\pgfpathlineto{\pgfqpoint{2.058036in}{1.312340in}}%
\pgfpathlineto{\pgfqpoint{2.062545in}{1.345428in}}%
\pgfpathlineto{\pgfqpoint{2.067055in}{1.355444in}}%
\pgfpathlineto{\pgfqpoint{2.071564in}{1.330047in}}%
\pgfpathlineto{\pgfqpoint{2.076073in}{1.377444in}}%
\pgfpathlineto{\pgfqpoint{2.080582in}{1.361704in}}%
\pgfpathlineto{\pgfqpoint{2.085091in}{1.332193in}}%
\pgfpathlineto{\pgfqpoint{2.089600in}{1.330762in}}%
\pgfpathlineto{\pgfqpoint{2.094109in}{1.321640in}}%
\pgfpathlineto{\pgfqpoint{2.098618in}{1.380842in}}%
\pgfpathlineto{\pgfqpoint{2.103127in}{1.373151in}}%
\pgfpathlineto{\pgfqpoint{2.107636in}{1.375655in}}%
\pgfpathlineto{\pgfqpoint{2.112145in}{1.392646in}}%
\pgfpathlineto{\pgfqpoint{2.116655in}{1.346323in}}%
\pgfpathlineto{\pgfqpoint{2.121164in}{1.405703in}}%
\pgfpathlineto{\pgfqpoint{2.125673in}{1.359021in}}%
\pgfpathlineto{\pgfqpoint{2.130182in}{1.376370in}}%
\pgfpathlineto{\pgfqpoint{2.134691in}{1.375118in}}%
\pgfpathlineto{\pgfqpoint{2.139200in}{1.400337in}}%
\pgfpathlineto{\pgfqpoint{2.143709in}{1.398549in}}%
\pgfpathlineto{\pgfqpoint{2.148218in}{1.388175in}}%
\pgfpathlineto{\pgfqpoint{2.152727in}{1.335949in}}%
\pgfpathlineto{\pgfqpoint{2.157236in}{1.457750in}}%
\pgfpathlineto{\pgfqpoint{2.161745in}{1.511944in}}%
\pgfpathlineto{\pgfqpoint{2.166255in}{1.449165in}}%
\pgfpathlineto{\pgfqpoint{2.175273in}{1.390858in}}%
\pgfpathlineto{\pgfqpoint{2.179782in}{1.411069in}}%
\pgfpathlineto{\pgfqpoint{2.184291in}{1.415898in}}%
\pgfpathlineto{\pgfqpoint{2.188800in}{1.450417in}}%
\pgfpathlineto{\pgfqpoint{2.193309in}{1.406597in}}%
\pgfpathlineto{\pgfqpoint{2.197818in}{1.412500in}}%
\pgfpathlineto{\pgfqpoint{2.202327in}{1.441295in}}%
\pgfpathlineto{\pgfqpoint{2.206836in}{1.496383in}}%
\pgfpathlineto{\pgfqpoint{2.211345in}{1.423946in}}%
\pgfpathlineto{\pgfqpoint{2.215855in}{1.437897in}}%
\pgfpathlineto{\pgfqpoint{2.220364in}{1.523927in}}%
\pgfpathlineto{\pgfqpoint{2.224873in}{1.405703in}}%
\pgfpathlineto{\pgfqpoint{2.229382in}{1.480107in}}%
\pgfpathlineto{\pgfqpoint{2.233891in}{1.529829in}}%
\pgfpathlineto{\pgfqpoint{2.238400in}{1.419117in}}%
\pgfpathlineto{\pgfqpoint{2.242909in}{1.549325in}}%
\pgfpathlineto{\pgfqpoint{2.247418in}{1.414288in}}%
\pgfpathlineto{\pgfqpoint{2.251927in}{1.435930in}}%
\pgfpathlineto{\pgfqpoint{2.256436in}{1.482790in}}%
\pgfpathlineto{\pgfqpoint{2.260945in}{1.500139in}}%
\pgfpathlineto{\pgfqpoint{2.265455in}{1.482790in}}%
\pgfpathlineto{\pgfqpoint{2.269964in}{1.523927in}}%
\pgfpathlineto{\pgfqpoint{2.274473in}{1.613177in}}%
\pgfpathlineto{\pgfqpoint{2.283491in}{1.435214in}}%
\pgfpathlineto{\pgfqpoint{2.288000in}{1.452385in}}%
\pgfpathlineto{\pgfqpoint{2.292509in}{1.492091in}}%
\pgfpathlineto{\pgfqpoint{2.297018in}{1.559520in}}%
\pgfpathlineto{\pgfqpoint{2.301527in}{1.479571in}}%
\pgfpathlineto{\pgfqpoint{2.306036in}{1.466693in}}%
\pgfpathlineto{\pgfqpoint{2.310545in}{1.496204in}}%
\pgfpathlineto{\pgfqpoint{2.315055in}{1.511944in}}%
\pgfpathlineto{\pgfqpoint{2.319564in}{1.522139in}}%
\pgfpathlineto{\pgfqpoint{2.324073in}{1.497456in}}%
\pgfpathlineto{\pgfqpoint{2.328582in}{1.610315in}}%
\pgfpathlineto{\pgfqpoint{2.333091in}{1.519456in}}%
\pgfpathlineto{\pgfqpoint{2.337600in}{1.536805in}}%
\pgfpathlineto{\pgfqpoint{2.342109in}{1.575080in}}%
\pgfpathlineto{\pgfqpoint{2.346618in}{1.525000in}}%
\pgfpathlineto{\pgfqpoint{2.351127in}{1.523391in}}%
\pgfpathlineto{\pgfqpoint{2.355636in}{1.670590in}}%
\pgfpathlineto{\pgfqpoint{2.360145in}{1.552008in}}%
\pgfpathlineto{\pgfqpoint{2.364655in}{1.609421in}}%
\pgfpathlineto{\pgfqpoint{2.369164in}{1.579373in}}%
\pgfpathlineto{\pgfqpoint{2.373673in}{1.568284in}}%
\pgfpathlineto{\pgfqpoint{2.378182in}{1.596722in}}%
\pgfpathlineto{\pgfqpoint{2.382691in}{1.538057in}}%
\pgfpathlineto{\pgfqpoint{2.387200in}{1.526073in}}%
\pgfpathlineto{\pgfqpoint{2.391709in}{1.578300in}}%
\pgfpathlineto{\pgfqpoint{2.396218in}{1.606201in}}%
\pgfpathlineto{\pgfqpoint{2.400727in}{1.550398in}}%
\pgfpathlineto{\pgfqpoint{2.405236in}{1.551292in}}%
\pgfpathlineto{\pgfqpoint{2.409745in}{1.626233in}}%
\pgfpathlineto{\pgfqpoint{2.414255in}{1.598332in}}%
\pgfpathlineto{\pgfqpoint{2.418764in}{1.590104in}}%
\pgfpathlineto{\pgfqpoint{2.423273in}{1.543423in}}%
\pgfpathlineto{\pgfqpoint{2.427782in}{1.634997in}}%
\pgfpathlineto{\pgfqpoint{2.432291in}{1.643761in}}%
\pgfpathlineto{\pgfqpoint{2.436800in}{1.577942in}}%
\pgfpathlineto{\pgfqpoint{2.441309in}{1.722100in}}%
\pgfpathlineto{\pgfqpoint{2.445818in}{1.707076in}}%
\pgfpathlineto{\pgfqpoint{2.450327in}{1.596722in}}%
\pgfpathlineto{\pgfqpoint{2.454836in}{1.589389in}}%
\pgfpathlineto{\pgfqpoint{2.459345in}{1.679890in}}%
\pgfpathlineto{\pgfqpoint{2.463855in}{1.555764in}}%
\pgfpathlineto{\pgfqpoint{2.468364in}{1.629989in}}%
\pgfpathlineto{\pgfqpoint{2.472873in}{1.586706in}}%
\pgfpathlineto{\pgfqpoint{2.477382in}{1.670947in}}%
\pgfpathlineto{\pgfqpoint{2.481891in}{1.597080in}}%
\pgfpathlineto{\pgfqpoint{2.486400in}{1.628201in}}%
\pgfpathlineto{\pgfqpoint{2.490909in}{1.680963in}}%
\pgfpathlineto{\pgfqpoint{2.495418in}{1.665760in}}%
\pgfpathlineto{\pgfqpoint{2.499927in}{1.646086in}}%
\pgfpathlineto{\pgfqpoint{2.504436in}{1.688654in}}%
\pgfpathlineto{\pgfqpoint{2.508945in}{1.578300in}}%
\pgfpathlineto{\pgfqpoint{2.513455in}{1.638395in}}%
\pgfpathlineto{\pgfqpoint{2.517964in}{1.626591in}}%
\pgfpathlineto{\pgfqpoint{2.522473in}{1.724425in}}%
\pgfpathlineto{\pgfqpoint{2.526982in}{1.643940in}}%
\pgfpathlineto{\pgfqpoint{2.531491in}{1.681321in}}%
\pgfpathlineto{\pgfqpoint{2.536000in}{1.628558in}}%
\pgfpathlineto{\pgfqpoint{2.540509in}{1.713694in}}%
\pgfpathlineto{\pgfqpoint{2.545018in}{1.662720in}}%
\pgfpathlineto{\pgfqpoint{2.549527in}{1.684898in}}%
\pgfpathlineto{\pgfqpoint{2.554036in}{1.634997in}}%
\pgfpathlineto{\pgfqpoint{2.558545in}{1.733726in}}%
\pgfpathlineto{\pgfqpoint{2.563055in}{1.597616in}}%
\pgfpathlineto{\pgfqpoint{2.567564in}{1.732295in}}%
\pgfpathlineto{\pgfqpoint{2.572073in}{1.661110in}}%
\pgfpathlineto{\pgfqpoint{2.576582in}{1.677028in}}%
\pgfpathlineto{\pgfqpoint{2.581091in}{1.700101in}}%
\pgfpathlineto{\pgfqpoint{2.585600in}{1.863218in}}%
\pgfpathlineto{\pgfqpoint{2.590109in}{1.830845in}}%
\pgfpathlineto{\pgfqpoint{2.594618in}{1.624981in}}%
\pgfpathlineto{\pgfqpoint{2.599127in}{1.712442in}}%
\pgfpathlineto{\pgfqpoint{2.603636in}{1.724604in}}%
\pgfpathlineto{\pgfqpoint{2.608145in}{1.685256in}}%
\pgfpathlineto{\pgfqpoint{2.612655in}{1.766636in}}%
\pgfpathlineto{\pgfqpoint{2.617164in}{1.725677in}}%
\pgfpathlineto{\pgfqpoint{2.621673in}{1.761270in}}%
\pgfpathlineto{\pgfqpoint{2.626182in}{1.685792in}}%
\pgfpathlineto{\pgfqpoint{2.630691in}{1.737840in}}%
\pgfpathlineto{\pgfqpoint{2.635200in}{1.680963in}}%
\pgfpathlineto{\pgfqpoint{2.639709in}{1.712084in}}%
\pgfpathlineto{\pgfqpoint{2.644218in}{1.778619in}}%
\pgfpathlineto{\pgfqpoint{2.648727in}{1.744457in}}%
\pgfpathlineto{\pgfqpoint{2.653236in}{1.731222in}}%
\pgfpathlineto{\pgfqpoint{2.657745in}{1.702605in}}%
\pgfpathlineto{\pgfqpoint{2.662255in}{1.775578in}}%
\pgfpathlineto{\pgfqpoint{2.666764in}{1.676850in}}%
\pgfpathlineto{\pgfqpoint{2.671273in}{1.837105in}}%
\pgfpathlineto{\pgfqpoint{2.675782in}{1.779871in}}%
\pgfpathlineto{\pgfqpoint{2.680291in}{1.788814in}}%
\pgfpathlineto{\pgfqpoint{2.684800in}{1.716377in}}%
\pgfpathlineto{\pgfqpoint{2.689309in}{1.717629in}}%
\pgfpathlineto{\pgfqpoint{2.693818in}{1.736945in}}%
\pgfpathlineto{\pgfqpoint{2.698327in}{1.713157in}}%
\pgfpathlineto{\pgfqpoint{2.702836in}{1.877527in}}%
\pgfpathlineto{\pgfqpoint{2.707345in}{1.856422in}}%
\pgfpathlineto{\pgfqpoint{2.711855in}{1.756441in}}%
\pgfpathlineto{\pgfqpoint{2.720873in}{1.808130in}}%
\pgfpathlineto{\pgfqpoint{2.725382in}{1.855170in}}%
\pgfpathlineto{\pgfqpoint{2.729891in}{1.813496in}}%
\pgfpathlineto{\pgfqpoint{2.734400in}{1.796683in}}%
\pgfpathlineto{\pgfqpoint{2.738909in}{1.805626in}}%
\pgfpathlineto{\pgfqpoint{2.743418in}{1.783448in}}%
\pgfpathlineto{\pgfqpoint{2.747927in}{1.924387in}}%
\pgfpathlineto{\pgfqpoint{2.752436in}{1.893445in}}%
\pgfpathlineto{\pgfqpoint{2.756945in}{1.840324in}}%
\pgfpathlineto{\pgfqpoint{2.761455in}{1.849625in}}%
\pgfpathlineto{\pgfqpoint{2.765964in}{1.951573in}}%
\pgfpathlineto{\pgfqpoint{2.770473in}{1.903282in}}%
\pgfpathlineto{\pgfqpoint{2.774982in}{1.891120in}}%
\pgfpathlineto{\pgfqpoint{2.779491in}{1.870730in}}%
\pgfpathlineto{\pgfqpoint{2.784000in}{1.859641in}}%
\pgfpathlineto{\pgfqpoint{2.788509in}{1.827983in}}%
\pgfpathlineto{\pgfqpoint{2.793018in}{1.865543in}}%
\pgfpathlineto{\pgfqpoint{2.797527in}{1.860893in}}%
\pgfpathlineto{\pgfqpoint{2.802036in}{1.956760in}}%
\pgfpathlineto{\pgfqpoint{2.806545in}{1.879673in}}%
\pgfpathlineto{\pgfqpoint{2.811055in}{1.885396in}}%
\pgfpathlineto{\pgfqpoint{2.815564in}{1.919916in}}%
\pgfpathlineto{\pgfqpoint{2.820073in}{1.882892in}}%
\pgfpathlineto{\pgfqpoint{2.824582in}{1.899884in}}%
\pgfpathlineto{\pgfqpoint{2.829091in}{1.850519in}}%
\pgfpathlineto{\pgfqpoint{2.833600in}{1.853202in}}%
\pgfpathlineto{\pgfqpoint{2.838109in}{1.863218in}}%
\pgfpathlineto{\pgfqpoint{2.842618in}{1.840682in}}%
\pgfpathlineto{\pgfqpoint{2.847127in}{1.848731in}}%
\pgfpathlineto{\pgfqpoint{2.851636in}{1.868584in}}%
\pgfpathlineto{\pgfqpoint{2.856145in}{1.832276in}}%
\pgfpathlineto{\pgfqpoint{2.860655in}{1.867690in}}%
\pgfpathlineto{\pgfqpoint{2.865164in}{1.874128in}}%
\pgfpathlineto{\pgfqpoint{2.869673in}{1.924924in}}%
\pgfpathlineto{\pgfqpoint{2.874182in}{1.886112in}}%
\pgfpathlineto{\pgfqpoint{2.878691in}{1.950500in}}%
\pgfpathlineto{\pgfqpoint{2.883200in}{1.902745in}}%
\pgfpathlineto{\pgfqpoint{2.887709in}{1.838178in}}%
\pgfpathlineto{\pgfqpoint{2.892218in}{1.981800in}}%
\pgfpathlineto{\pgfqpoint{2.896727in}{1.974109in}}%
\pgfpathlineto{\pgfqpoint{2.901236in}{1.913119in}}%
\pgfpathlineto{\pgfqpoint{2.905745in}{1.941200in}}%
\pgfpathlineto{\pgfqpoint{2.910255in}{1.943704in}}%
\pgfpathlineto{\pgfqpoint{2.914764in}{1.913119in}}%
\pgfpathlineto{\pgfqpoint{2.919273in}{1.905965in}}%
\pgfpathlineto{\pgfqpoint{2.923782in}{2.013637in}}%
\pgfpathlineto{\pgfqpoint{2.928291in}{2.010238in}}%
\pgfpathlineto{\pgfqpoint{2.932800in}{1.976077in}}%
\pgfpathlineto{\pgfqpoint{2.937309in}{1.963736in}}%
\pgfpathlineto{\pgfqpoint{2.941818in}{1.901315in}}%
\pgfpathlineto{\pgfqpoint{2.946327in}{1.915981in}}%
\pgfpathlineto{\pgfqpoint{2.950836in}{1.951573in}}%
\pgfpathlineto{\pgfqpoint{2.955345in}{1.859641in}}%
\pgfpathlineto{\pgfqpoint{2.959855in}{2.070871in}}%
\pgfpathlineto{\pgfqpoint{2.964364in}{1.896485in}}%
\pgfpathlineto{\pgfqpoint{2.968873in}{2.024010in}}%
\pgfpathlineto{\pgfqpoint{2.973382in}{1.903819in}}%
\pgfpathlineto{\pgfqpoint{2.977891in}{1.925997in}}%
\pgfpathlineto{\pgfqpoint{2.982400in}{2.089293in}}%
\pgfpathlineto{\pgfqpoint{2.991418in}{1.899884in}}%
\pgfpathlineto{\pgfqpoint{2.995927in}{1.999149in}}%
\pgfpathlineto{\pgfqpoint{3.000436in}{1.983946in}}%
\pgfpathlineto{\pgfqpoint{3.004945in}{2.095374in}}%
\pgfpathlineto{\pgfqpoint{3.009455in}{1.905786in}}%
\pgfpathlineto{\pgfqpoint{3.013964in}{2.100203in}}%
\pgfpathlineto{\pgfqpoint{3.022982in}{1.997361in}}%
\pgfpathlineto{\pgfqpoint{3.027491in}{1.932078in}}%
\pgfpathlineto{\pgfqpoint{3.032000in}{1.985735in}}%
\pgfpathlineto{\pgfqpoint{3.036509in}{1.942452in}}%
\pgfpathlineto{\pgfqpoint{3.041018in}{2.027230in}}%
\pgfpathlineto{\pgfqpoint{3.050036in}{1.992889in}}%
\pgfpathlineto{\pgfqpoint{3.054545in}{2.066220in}}%
\pgfpathlineto{\pgfqpoint{3.059055in}{2.028124in}}%
\pgfpathlineto{\pgfqpoint{3.063564in}{2.033132in}}%
\pgfpathlineto{\pgfqpoint{3.068073in}{2.145275in}}%
\pgfpathlineto{\pgfqpoint{3.072582in}{2.041896in}}%
\pgfpathlineto{\pgfqpoint{3.077091in}{2.053343in}}%
\pgfpathlineto{\pgfqpoint{3.081600in}{2.116300in}}%
\pgfpathlineto{\pgfqpoint{3.086109in}{2.052985in}}%
\pgfpathlineto{\pgfqpoint{3.090618in}{2.032595in}}%
\pgfpathlineto{\pgfqpoint{3.095127in}{2.126137in}}%
\pgfpathlineto{\pgfqpoint{3.099636in}{2.007555in}}%
\pgfpathlineto{\pgfqpoint{3.104145in}{2.095016in}}%
\pgfpathlineto{\pgfqpoint{3.108655in}{2.217712in}}%
\pgfpathlineto{\pgfqpoint{3.113164in}{2.090545in}}%
\pgfpathlineto{\pgfqpoint{3.117673in}{2.149389in}}%
\pgfpathlineto{\pgfqpoint{3.122182in}{2.052270in}}%
\pgfpathlineto{\pgfqpoint{3.126691in}{2.132934in}}%
\pgfpathlineto{\pgfqpoint{3.131200in}{2.081244in}}%
\pgfpathlineto{\pgfqpoint{3.135709in}{2.070871in}}%
\pgfpathlineto{\pgfqpoint{3.140218in}{2.049050in}}%
\pgfpathlineto{\pgfqpoint{3.144727in}{1.993962in}}%
\pgfpathlineto{\pgfqpoint{3.149236in}{2.124349in}}%
\pgfpathlineto{\pgfqpoint{3.153745in}{2.087862in}}%
\pgfpathlineto{\pgfqpoint{3.158255in}{2.206265in}}%
\pgfpathlineto{\pgfqpoint{3.162764in}{2.048156in}}%
\pgfpathlineto{\pgfqpoint{3.167273in}{2.128462in}}%
\pgfpathlineto{\pgfqpoint{3.176291in}{2.204298in}}%
\pgfpathlineto{\pgfqpoint{3.180800in}{2.179615in}}%
\pgfpathlineto{\pgfqpoint{3.185309in}{2.096626in}}%
\pgfpathlineto{\pgfqpoint{3.189818in}{2.042254in}}%
\pgfpathlineto{\pgfqpoint{3.194327in}{2.140088in}}%
\pgfpathlineto{\pgfqpoint{3.198836in}{2.113617in}}%
\pgfpathlineto{\pgfqpoint{3.203345in}{2.177111in}}%
\pgfpathlineto{\pgfqpoint{3.207855in}{2.191599in}}%
\pgfpathlineto{\pgfqpoint{3.216873in}{2.139909in}}%
\pgfpathlineto{\pgfqpoint{3.221382in}{2.147242in}}%
\pgfpathlineto{\pgfqpoint{3.225891in}{2.185875in}}%
\pgfpathlineto{\pgfqpoint{3.230400in}{2.174250in}}%
\pgfpathlineto{\pgfqpoint{3.234909in}{2.181225in}}%
\pgfpathlineto{\pgfqpoint{3.239418in}{2.205550in}}%
\pgfpathlineto{\pgfqpoint{3.243927in}{2.103065in}}%
\pgfpathlineto{\pgfqpoint{3.248436in}{2.194639in}}%
\pgfpathlineto{\pgfqpoint{3.252945in}{2.134722in}}%
\pgfpathlineto{\pgfqpoint{3.257455in}{2.175144in}}%
\pgfpathlineto{\pgfqpoint{3.261964in}{2.241500in}}%
\pgfpathlineto{\pgfqpoint{3.266473in}{2.243288in}}%
\pgfpathlineto{\pgfqpoint{3.270982in}{2.102528in}}%
\pgfpathlineto{\pgfqpoint{3.275491in}{2.243646in}}%
\pgfpathlineto{\pgfqpoint{3.280000in}{2.314652in}}%
\pgfpathlineto{\pgfqpoint{3.284509in}{2.269938in}}%
\pgfpathlineto{\pgfqpoint{3.289018in}{2.152608in}}%
\pgfpathlineto{\pgfqpoint{3.293527in}{2.224687in}}%
\pgfpathlineto{\pgfqpoint{3.298036in}{2.137226in}}%
\pgfpathlineto{\pgfqpoint{3.302545in}{2.236134in}}%
\pgfpathlineto{\pgfqpoint{3.307055in}{2.182656in}}%
\pgfpathlineto{\pgfqpoint{3.311564in}{2.181941in}}%
\pgfpathlineto{\pgfqpoint{3.316073in}{2.203940in}}%
\pgfpathlineto{\pgfqpoint{3.320582in}{2.176575in}}%
\pgfpathlineto{\pgfqpoint{3.325091in}{2.321091in}}%
\pgfpathlineto{\pgfqpoint{3.329600in}{2.295872in}}%
\pgfpathlineto{\pgfqpoint{3.334109in}{2.149568in}}%
\pgfpathlineto{\pgfqpoint{3.338618in}{2.245435in}}%
\pgfpathlineto{\pgfqpoint{3.343127in}{2.263678in}}%
\pgfpathlineto{\pgfqpoint{3.347636in}{2.201973in}}%
\pgfpathlineto{\pgfqpoint{3.352145in}{2.243288in}}%
\pgfpathlineto{\pgfqpoint{3.356655in}{2.201436in}}%
\pgfpathlineto{\pgfqpoint{3.361164in}{2.170315in}}%
\pgfpathlineto{\pgfqpoint{3.365673in}{2.334326in}}%
\pgfpathlineto{\pgfqpoint{3.370182in}{2.174071in}}%
\pgfpathlineto{\pgfqpoint{3.374691in}{2.202330in}}%
\pgfpathlineto{\pgfqpoint{3.379200in}{2.266897in}}%
\pgfpathlineto{\pgfqpoint{3.383709in}{2.252768in}}%
\pgfpathlineto{\pgfqpoint{3.388218in}{2.271905in}}%
\pgfpathlineto{\pgfqpoint{3.397236in}{2.178900in}}%
\pgfpathlineto{\pgfqpoint{3.401745in}{2.274588in}}%
\pgfpathlineto{\pgfqpoint{3.406255in}{2.175323in}}%
\pgfpathlineto{\pgfqpoint{3.410764in}{2.284247in}}%
\pgfpathlineto{\pgfqpoint{3.415273in}{2.284068in}}%
\pgfpathlineto{\pgfqpoint{3.419782in}{2.309465in}}%
\pgfpathlineto{\pgfqpoint{3.424291in}{2.295872in}}%
\pgfpathlineto{\pgfqpoint{3.428800in}{2.272800in}}%
\pgfpathlineto{\pgfqpoint{3.433309in}{2.329855in}}%
\pgfpathlineto{\pgfqpoint{3.437818in}{2.276019in}}%
\pgfpathlineto{\pgfqpoint{3.442327in}{2.385837in}}%
\pgfpathlineto{\pgfqpoint{3.446836in}{2.340050in}}%
\pgfpathlineto{\pgfqpoint{3.451345in}{2.338977in}}%
\pgfpathlineto{\pgfqpoint{3.455855in}{2.351497in}}%
\pgfpathlineto{\pgfqpoint{3.460364in}{2.386731in}}%
\pgfpathlineto{\pgfqpoint{3.464873in}{2.353643in}}%
\pgfpathlineto{\pgfqpoint{3.473891in}{2.354179in}}%
\pgfpathlineto{\pgfqpoint{3.478400in}{2.337725in}}%
\pgfpathlineto{\pgfqpoint{3.482909in}{2.454339in}}%
\pgfpathlineto{\pgfqpoint{3.487418in}{2.406763in}}%
\pgfpathlineto{\pgfqpoint{3.491927in}{2.280312in}}%
\pgfpathlineto{\pgfqpoint{3.496436in}{2.258491in}}%
\pgfpathlineto{\pgfqpoint{3.500945in}{2.298019in}}%
\pgfpathlineto{\pgfqpoint{3.505455in}{2.317693in}}%
\pgfpathlineto{\pgfqpoint{3.509964in}{2.406227in}}%
\pgfpathlineto{\pgfqpoint{3.514473in}{2.448616in}}%
\pgfpathlineto{\pgfqpoint{3.518982in}{2.513183in}}%
\pgfpathlineto{\pgfqpoint{3.523491in}{2.300165in}}%
\pgfpathlineto{\pgfqpoint{3.528000in}{2.318766in}}%
\pgfpathlineto{\pgfqpoint{3.532509in}{2.330570in}}%
\pgfpathlineto{\pgfqpoint{3.537018in}{2.369740in}}%
\pgfpathlineto{\pgfqpoint{3.541527in}{2.357757in}}%
\pgfpathlineto{\pgfqpoint{3.546036in}{2.333969in}}%
\pgfpathlineto{\pgfqpoint{3.550545in}{2.288181in}}%
\pgfpathlineto{\pgfqpoint{3.555055in}{2.405690in}}%
\pgfpathlineto{\pgfqpoint{3.559564in}{2.374927in}}%
\pgfpathlineto{\pgfqpoint{3.564073in}{2.450047in}}%
\pgfpathlineto{\pgfqpoint{3.573091in}{2.386195in}}%
\pgfpathlineto{\pgfqpoint{3.577600in}{2.533573in}}%
\pgfpathlineto{\pgfqpoint{3.582109in}{2.467932in}}%
\pgfpathlineto{\pgfqpoint{3.586618in}{2.419462in}}%
\pgfpathlineto{\pgfqpoint{3.591127in}{2.478485in}}%
\pgfpathlineto{\pgfqpoint{3.595636in}{2.474908in}}%
\pgfpathlineto{\pgfqpoint{3.600145in}{2.391382in}}%
\pgfpathlineto{\pgfqpoint{3.604655in}{2.487785in}}%
\pgfpathlineto{\pgfqpoint{3.609164in}{2.433950in}}%
\pgfpathlineto{\pgfqpoint{3.613673in}{2.514435in}}%
\pgfpathlineto{\pgfqpoint{3.618182in}{2.476160in}}%
\pgfpathlineto{\pgfqpoint{3.622691in}{2.400324in}}%
\pgfpathlineto{\pgfqpoint{3.627200in}{2.458989in}}%
\pgfpathlineto{\pgfqpoint{3.631709in}{2.570059in}}%
\pgfpathlineto{\pgfqpoint{3.636218in}{2.353285in}}%
\pgfpathlineto{\pgfqpoint{3.640727in}{2.493866in}}%
\pgfpathlineto{\pgfqpoint{3.645236in}{2.372781in}}%
\pgfpathlineto{\pgfqpoint{3.649745in}{2.554857in}}%
\pgfpathlineto{\pgfqpoint{3.654255in}{2.558613in}}%
\pgfpathlineto{\pgfqpoint{3.658764in}{2.473119in}}%
\pgfpathlineto{\pgfqpoint{3.663273in}{2.468648in}}%
\pgfpathlineto{\pgfqpoint{3.667782in}{2.443071in}}%
\pgfpathlineto{\pgfqpoint{3.672291in}{2.455591in}}%
\pgfpathlineto{\pgfqpoint{3.676800in}{2.425901in}}%
\pgfpathlineto{\pgfqpoint{3.681309in}{2.384585in}}%
\pgfpathlineto{\pgfqpoint{3.685818in}{2.503882in}}%
\pgfpathlineto{\pgfqpoint{3.690327in}{2.430015in}}%
\pgfpathlineto{\pgfqpoint{3.694836in}{2.498517in}}%
\pgfpathlineto{\pgfqpoint{3.699345in}{2.637846in}}%
\pgfpathlineto{\pgfqpoint{3.703855in}{2.488859in}}%
\pgfpathlineto{\pgfqpoint{3.708364in}{2.396390in}}%
\pgfpathlineto{\pgfqpoint{3.712873in}{2.478664in}}%
\pgfpathlineto{\pgfqpoint{3.717382in}{2.442714in}}%
\pgfpathlineto{\pgfqpoint{3.721891in}{2.572742in}}%
\pgfpathlineto{\pgfqpoint{3.726400in}{2.440925in}}%
\pgfpathlineto{\pgfqpoint{3.730909in}{2.539296in}}%
\pgfpathlineto{\pgfqpoint{3.735418in}{2.521410in}}%
\pgfpathlineto{\pgfqpoint{3.739927in}{2.533752in}}%
\pgfpathlineto{\pgfqpoint{3.744436in}{2.490468in}}%
\pgfpathlineto{\pgfqpoint{3.748945in}{2.667000in}}%
\pgfpathlineto{\pgfqpoint{3.753455in}{2.499232in}}%
\pgfpathlineto{\pgfqpoint{3.757964in}{2.490111in}}%
\pgfpathlineto{\pgfqpoint{3.762473in}{2.464892in}}%
\pgfpathlineto{\pgfqpoint{3.766982in}{2.532857in}}%
\pgfpathlineto{\pgfqpoint{3.771491in}{2.502809in}}%
\pgfpathlineto{\pgfqpoint{3.776000in}{2.425364in}}%
\pgfpathlineto{\pgfqpoint{3.780509in}{2.612448in}}%
\pgfpathlineto{\pgfqpoint{3.785018in}{2.528386in}}%
\pgfpathlineto{\pgfqpoint{3.789527in}{2.652691in}}%
\pgfpathlineto{\pgfqpoint{3.794036in}{2.487607in}}%
\pgfpathlineto{\pgfqpoint{3.798545in}{2.574889in}}%
\pgfpathlineto{\pgfqpoint{3.803055in}{2.497086in}}%
\pgfpathlineto{\pgfqpoint{3.807564in}{2.771631in}}%
\pgfpathlineto{\pgfqpoint{3.816582in}{2.533930in}}%
\pgfpathlineto{\pgfqpoint{3.821091in}{2.522126in}}%
\pgfpathlineto{\pgfqpoint{3.825600in}{2.681845in}}%
\pgfpathlineto{\pgfqpoint{3.830109in}{2.568450in}}%
\pgfpathlineto{\pgfqpoint{3.834618in}{2.565946in}}%
\pgfpathlineto{\pgfqpoint{3.839127in}{2.514256in}}%
\pgfpathlineto{\pgfqpoint{3.843636in}{2.669325in}}%
\pgfpathlineto{\pgfqpoint{3.848145in}{2.706527in}}%
\pgfpathlineto{\pgfqpoint{3.852655in}{2.561832in}}%
\pgfpathlineto{\pgfqpoint{3.857164in}{2.625326in}}%
\pgfpathlineto{\pgfqpoint{3.861673in}{2.745875in}}%
\pgfpathlineto{\pgfqpoint{3.866182in}{2.609944in}}%
\pgfpathlineto{\pgfqpoint{3.875200in}{2.639456in}}%
\pgfpathlineto{\pgfqpoint{3.879709in}{2.630692in}}%
\pgfpathlineto{\pgfqpoint{3.884218in}{2.608692in}}%
\pgfpathlineto{\pgfqpoint{3.888727in}{2.612091in}}%
\pgfpathlineto{\pgfqpoint{3.893236in}{2.593311in}}%
\pgfpathlineto{\pgfqpoint{3.897745in}{2.543410in}}%
\pgfpathlineto{\pgfqpoint{3.902255in}{2.708852in}}%
\pgfpathlineto{\pgfqpoint{3.906764in}{2.605294in}}%
\pgfpathlineto{\pgfqpoint{3.911273in}{2.631586in}}%
\pgfpathlineto{\pgfqpoint{3.915782in}{2.507281in}}%
\pgfpathlineto{\pgfqpoint{3.920291in}{2.710998in}}%
\pgfpathlineto{\pgfqpoint{3.924800in}{2.600286in}}%
\pgfpathlineto{\pgfqpoint{3.929309in}{2.600465in}}%
\pgfpathlineto{\pgfqpoint{3.933818in}{2.757501in}}%
\pgfpathlineto{\pgfqpoint{3.938327in}{2.561295in}}%
\pgfpathlineto{\pgfqpoint{3.942836in}{2.601717in}}%
\pgfpathlineto{\pgfqpoint{3.947345in}{2.684349in}}%
\pgfpathlineto{\pgfqpoint{3.951855in}{2.666642in}}%
\pgfpathlineto{\pgfqpoint{3.956364in}{2.655016in}}%
\pgfpathlineto{\pgfqpoint{3.960873in}{2.756428in}}%
\pgfpathlineto{\pgfqpoint{3.965382in}{2.651081in}}%
\pgfpathlineto{\pgfqpoint{3.969891in}{2.862132in}}%
\pgfpathlineto{\pgfqpoint{3.974400in}{2.793988in}}%
\pgfpathlineto{\pgfqpoint{3.978909in}{2.704023in}}%
\pgfpathlineto{\pgfqpoint{3.983418in}{2.724413in}}%
\pgfpathlineto{\pgfqpoint{3.987927in}{2.734965in}}%
\pgfpathlineto{\pgfqpoint{3.992436in}{2.631586in}}%
\pgfpathlineto{\pgfqpoint{3.996945in}{2.704560in}}%
\pgfpathlineto{\pgfqpoint{4.001455in}{2.837987in}}%
\pgfpathlineto{\pgfqpoint{4.005964in}{2.702413in}}%
\pgfpathlineto{\pgfqpoint{4.010473in}{2.683454in}}%
\pgfpathlineto{\pgfqpoint{4.014982in}{2.823499in}}%
\pgfpathlineto{\pgfqpoint{4.019491in}{2.710820in}}%
\pgfpathlineto{\pgfqpoint{4.024000in}{2.712429in}}%
\pgfpathlineto{\pgfqpoint{4.028509in}{2.667536in}}%
\pgfpathlineto{\pgfqpoint{4.033018in}{2.769127in}}%
\pgfpathlineto{\pgfqpoint{4.037527in}{2.835125in}}%
\pgfpathlineto{\pgfqpoint{4.042036in}{2.805971in}}%
\pgfpathlineto{\pgfqpoint{4.046545in}{2.789695in}}%
\pgfpathlineto{\pgfqpoint{4.051055in}{2.700267in}}%
\pgfpathlineto{\pgfqpoint{4.055564in}{2.795955in}}%
\pgfpathlineto{\pgfqpoint{4.060073in}{2.809191in}}%
\pgfpathlineto{\pgfqpoint{4.064582in}{2.832263in}}%
\pgfpathlineto{\pgfqpoint{4.069091in}{2.758217in}}%
\pgfpathlineto{\pgfqpoint{4.073600in}{2.852474in}}%
\pgfpathlineto{\pgfqpoint{4.078109in}{2.797923in}}%
\pgfpathlineto{\pgfqpoint{4.082618in}{2.805435in}}%
\pgfpathlineto{\pgfqpoint{4.087127in}{2.865531in}}%
\pgfpathlineto{\pgfqpoint{4.091636in}{2.858376in}}%
\pgfpathlineto{\pgfqpoint{4.096145in}{2.714218in}}%
\pgfpathlineto{\pgfqpoint{4.100655in}{2.887530in}}%
\pgfpathlineto{\pgfqpoint{4.105164in}{2.857303in}}%
\pgfpathlineto{\pgfqpoint{4.109673in}{2.766086in}}%
\pgfpathlineto{\pgfqpoint{4.114182in}{2.751062in}}%
\pgfpathlineto{\pgfqpoint{4.118691in}{2.745697in}}%
\pgfpathlineto{\pgfqpoint{4.123200in}{2.792378in}}%
\pgfpathlineto{\pgfqpoint{4.127709in}{2.704917in}}%
\pgfpathlineto{\pgfqpoint{4.132218in}{2.837629in}}%
\pgfpathlineto{\pgfqpoint{4.136727in}{2.789874in}}%
\pgfpathlineto{\pgfqpoint{4.141236in}{2.794524in}}%
\pgfpathlineto{\pgfqpoint{4.145745in}{2.755891in}}%
\pgfpathlineto{\pgfqpoint{4.150255in}{3.045639in}}%
\pgfpathlineto{\pgfqpoint{4.154764in}{2.893969in}}%
\pgfpathlineto{\pgfqpoint{4.159273in}{2.909529in}}%
\pgfpathlineto{\pgfqpoint{4.163782in}{2.891465in}}%
\pgfpathlineto{\pgfqpoint{4.168291in}{2.897904in}}%
\pgfpathlineto{\pgfqpoint{4.172800in}{2.924374in}}%
\pgfpathlineto{\pgfqpoint{4.177309in}{2.774135in}}%
\pgfpathlineto{\pgfqpoint{4.181818in}{2.902554in}}%
\pgfpathlineto{\pgfqpoint{4.186327in}{2.887888in}}%
\pgfpathlineto{\pgfqpoint{4.190836in}{3.036875in}}%
\pgfpathlineto{\pgfqpoint{4.199855in}{2.893611in}}%
\pgfpathlineto{\pgfqpoint{4.204364in}{3.005396in}}%
\pgfpathlineto{\pgfqpoint{4.208873in}{2.855157in}}%
\pgfpathlineto{\pgfqpoint{4.213382in}{3.116466in}}%
\pgfpathlineto{\pgfqpoint{4.217891in}{3.145262in}}%
\pgfpathlineto{\pgfqpoint{4.222400in}{2.912570in}}%
\pgfpathlineto{\pgfqpoint{4.226909in}{2.901123in}}%
\pgfpathlineto{\pgfqpoint{4.231418in}{2.968910in}}%
\pgfpathlineto{\pgfqpoint{4.235927in}{2.917578in}}%
\pgfpathlineto{\pgfqpoint{4.240436in}{2.948878in}}%
\pgfpathlineto{\pgfqpoint{4.244945in}{2.842637in}}%
\pgfpathlineto{\pgfqpoint{4.249455in}{2.891822in}}%
\pgfpathlineto{\pgfqpoint{4.253964in}{3.024534in}}%
\pgfpathlineto{\pgfqpoint{4.258473in}{2.881449in}}%
\pgfpathlineto{\pgfqpoint{4.262982in}{3.050647in}}%
\pgfpathlineto{\pgfqpoint{4.267491in}{2.846929in}}%
\pgfpathlineto{\pgfqpoint{4.272000in}{3.023997in}}%
\pgfpathlineto{\pgfqpoint{4.276509in}{2.935821in}}%
\pgfpathlineto{\pgfqpoint{4.281018in}{2.895578in}}%
\pgfpathlineto{\pgfqpoint{4.285527in}{2.903627in}}%
\pgfpathlineto{\pgfqpoint{4.290036in}{2.933317in}}%
\pgfpathlineto{\pgfqpoint{4.294545in}{2.908456in}}%
\pgfpathlineto{\pgfqpoint{4.299055in}{2.923301in}}%
\pgfpathlineto{\pgfqpoint{4.303564in}{3.060484in}}%
\pgfpathlineto{\pgfqpoint{4.308073in}{2.870538in}}%
\pgfpathlineto{\pgfqpoint{4.312582in}{2.829401in}}%
\pgfpathlineto{\pgfqpoint{4.317091in}{3.049574in}}%
\pgfpathlineto{\pgfqpoint{4.321600in}{2.925090in}}%
\pgfpathlineto{\pgfqpoint{4.326109in}{3.216268in}}%
\pgfpathlineto{\pgfqpoint{4.330618in}{2.921334in}}%
\pgfpathlineto{\pgfqpoint{4.335127in}{3.088386in}}%
\pgfpathlineto{\pgfqpoint{4.339636in}{3.181034in}}%
\pgfpathlineto{\pgfqpoint{4.344145in}{2.976600in}}%
\pgfpathlineto{\pgfqpoint{4.348655in}{2.918830in}}%
\pgfpathlineto{\pgfqpoint{4.353164in}{3.021851in}}%
\pgfpathlineto{\pgfqpoint{4.357673in}{2.941902in}}%
\pgfpathlineto{\pgfqpoint{4.362182in}{3.036160in}}%
\pgfpathlineto{\pgfqpoint{4.366691in}{3.079622in}}%
\pgfpathlineto{\pgfqpoint{4.371200in}{2.913822in}}%
\pgfpathlineto{\pgfqpoint{4.375709in}{3.098581in}}%
\pgfpathlineto{\pgfqpoint{4.380218in}{3.065313in}}%
\pgfpathlineto{\pgfqpoint{4.384727in}{3.134173in}}%
\pgfpathlineto{\pgfqpoint{4.389236in}{2.956390in}}%
\pgfpathlineto{\pgfqpoint{4.393745in}{3.050826in}}%
\pgfpathlineto{\pgfqpoint{4.398255in}{3.000567in}}%
\pgfpathlineto{\pgfqpoint{4.402764in}{3.074435in}}%
\pgfpathlineto{\pgfqpoint{4.407273in}{2.977495in}}%
\pgfpathlineto{\pgfqpoint{4.411782in}{3.026680in}}%
\pgfpathlineto{\pgfqpoint{4.416291in}{3.035087in}}%
\pgfpathlineto{\pgfqpoint{4.420800in}{3.028290in}}%
\pgfpathlineto{\pgfqpoint{4.425309in}{3.031688in}}%
\pgfpathlineto{\pgfqpoint{4.429818in}{3.025071in}}%
\pgfpathlineto{\pgfqpoint{4.434327in}{3.067281in}}%
\pgfpathlineto{\pgfqpoint{4.438836in}{3.198561in}}%
\pgfpathlineto{\pgfqpoint{4.443345in}{3.002892in}}%
\pgfpathlineto{\pgfqpoint{4.447855in}{3.144726in}}%
\pgfpathlineto{\pgfqpoint{4.452364in}{3.074793in}}%
\pgfpathlineto{\pgfqpoint{4.456873in}{3.082305in}}%
\pgfpathlineto{\pgfqpoint{4.461382in}{2.998958in}}%
\pgfpathlineto{\pgfqpoint{4.465891in}{3.182107in}}%
\pgfpathlineto{\pgfqpoint{4.470400in}{3.050468in}}%
\pgfpathlineto{\pgfqpoint{4.474909in}{3.090353in}}%
\pgfpathlineto{\pgfqpoint{4.479418in}{3.043672in}}%
\pgfpathlineto{\pgfqpoint{4.483927in}{3.083914in}}%
\pgfpathlineto{\pgfqpoint{4.488436in}{3.150270in}}%
\pgfpathlineto{\pgfqpoint{4.492945in}{3.243633in}}%
\pgfpathlineto{\pgfqpoint{4.497455in}{3.025786in}}%
\pgfpathlineto{\pgfqpoint{4.501964in}{3.161359in}}%
\pgfpathlineto{\pgfqpoint{4.506473in}{3.215732in}}%
\pgfpathlineto{\pgfqpoint{4.510982in}{3.061200in}}%
\pgfpathlineto{\pgfqpoint{4.515491in}{3.259194in}}%
\pgfpathlineto{\pgfqpoint{4.520000in}{2.996811in}}%
\pgfpathlineto{\pgfqpoint{4.524509in}{3.225390in}}%
\pgfpathlineto{\pgfqpoint{4.529018in}{3.139718in}}%
\pgfpathlineto{\pgfqpoint{4.533527in}{3.007364in}}%
\pgfpathlineto{\pgfqpoint{4.538036in}{3.150449in}}%
\pgfpathlineto{\pgfqpoint{4.547055in}{3.042956in}}%
\pgfpathlineto{\pgfqpoint{4.551564in}{3.132563in}}%
\pgfpathlineto{\pgfqpoint{4.556073in}{3.091963in}}%
\pgfpathlineto{\pgfqpoint{4.560582in}{3.237016in}}%
\pgfpathlineto{\pgfqpoint{4.565091in}{3.447172in}}%
\pgfpathlineto{\pgfqpoint{4.569600in}{3.203927in}}%
\pgfpathlineto{\pgfqpoint{4.574109in}{3.096434in}}%
\pgfpathlineto{\pgfqpoint{4.578618in}{3.105198in}}%
\pgfpathlineto{\pgfqpoint{4.583127in}{3.130417in}}%
\pgfpathlineto{\pgfqpoint{4.587636in}{3.106808in}}%
\pgfpathlineto{\pgfqpoint{4.592145in}{3.034192in}}%
\pgfpathlineto{\pgfqpoint{4.596655in}{3.216626in}}%
\pgfpathlineto{\pgfqpoint{4.601164in}{3.162790in}}%
\pgfpathlineto{\pgfqpoint{4.605673in}{3.405856in}}%
\pgfpathlineto{\pgfqpoint{4.610182in}{3.210008in}}%
\pgfpathlineto{\pgfqpoint{4.614691in}{3.219845in}}%
\pgfpathlineto{\pgfqpoint{4.619200in}{3.258478in}}%
\pgfpathlineto{\pgfqpoint{4.623709in}{3.247032in}}%
\pgfpathlineto{\pgfqpoint{4.628218in}{3.176383in}}%
\pgfpathlineto{\pgfqpoint{4.637236in}{3.216626in}}%
\pgfpathlineto{\pgfqpoint{4.646255in}{3.092142in}}%
\pgfpathlineto{\pgfqpoint{4.650764in}{3.309453in}}%
\pgfpathlineto{\pgfqpoint{4.655273in}{3.366508in}}%
\pgfpathlineto{\pgfqpoint{4.659782in}{3.333419in}}%
\pgfpathlineto{\pgfqpoint{4.664291in}{3.181749in}}%
\pgfpathlineto{\pgfqpoint{4.668800in}{3.266348in}}%
\pgfpathlineto{\pgfqpoint{4.673309in}{3.479009in}}%
\pgfpathlineto{\pgfqpoint{4.677818in}{3.602777in}}%
\pgfpathlineto{\pgfqpoint{4.682327in}{3.144368in}}%
\pgfpathlineto{\pgfqpoint{4.686836in}{3.580063in}}%
\pgfpathlineto{\pgfqpoint{4.691345in}{3.133637in}}%
\pgfpathlineto{\pgfqpoint{4.695855in}{3.247389in}}%
\pgfpathlineto{\pgfqpoint{4.700364in}{3.230219in}}%
\pgfpathlineto{\pgfqpoint{4.704873in}{3.387076in}}%
\pgfpathlineto{\pgfqpoint{4.709382in}{3.348086in}}%
\pgfpathlineto{\pgfqpoint{4.713891in}{3.371337in}}%
\pgfpathlineto{\pgfqpoint{4.718400in}{3.382605in}}%
\pgfpathlineto{\pgfqpoint{4.722909in}{3.364004in}}%
\pgfpathlineto{\pgfqpoint{4.727418in}{3.334314in}}%
\pgfpathlineto{\pgfqpoint{4.731927in}{3.349695in}}%
\pgfpathlineto{\pgfqpoint{4.736436in}{3.172985in}}%
\pgfpathlineto{\pgfqpoint{4.740945in}{3.284592in}}%
\pgfpathlineto{\pgfqpoint{4.745455in}{3.270104in}}%
\pgfpathlineto{\pgfqpoint{4.749964in}{3.274933in}}%
\pgfpathlineto{\pgfqpoint{4.754473in}{3.293355in}}%
\pgfpathlineto{\pgfqpoint{4.758982in}{3.185684in}}%
\pgfpathlineto{\pgfqpoint{4.763491in}{3.414263in}}%
\pgfpathlineto{\pgfqpoint{4.768000in}{3.325550in}}%
\pgfpathlineto{\pgfqpoint{4.772509in}{3.298363in}}%
\pgfpathlineto{\pgfqpoint{4.777018in}{3.194984in}}%
\pgfpathlineto{\pgfqpoint{4.781527in}{3.291567in}}%
\pgfpathlineto{\pgfqpoint{4.786036in}{3.203569in}}%
\pgfpathlineto{\pgfqpoint{4.790545in}{3.395304in}}%
\pgfpathlineto{\pgfqpoint{4.795055in}{3.350947in}}%
\pgfpathlineto{\pgfqpoint{4.799564in}{3.280299in}}%
\pgfpathlineto{\pgfqpoint{4.804073in}{3.459692in}}%
\pgfpathlineto{\pgfqpoint{4.808582in}{3.331989in}}%
\pgfpathlineto{\pgfqpoint{4.813091in}{3.366150in}}%
\pgfpathlineto{\pgfqpoint{4.817600in}{3.273324in}}%
\pgfpathlineto{\pgfqpoint{4.822109in}{3.361679in}}%
\pgfpathlineto{\pgfqpoint{4.826618in}{3.299615in}}%
\pgfpathlineto{\pgfqpoint{4.831127in}{3.321436in}}%
\pgfpathlineto{\pgfqpoint{4.835636in}{3.444132in}}%
\pgfpathlineto{\pgfqpoint{4.840145in}{3.410149in}}%
\pgfpathlineto{\pgfqpoint{4.844655in}{3.419986in}}%
\pgfpathlineto{\pgfqpoint{4.849164in}{3.411580in}}%
\pgfpathlineto{\pgfqpoint{4.853673in}{3.344330in}}%
\pgfpathlineto{\pgfqpoint{4.858182in}{3.311062in}}%
\pgfpathlineto{\pgfqpoint{4.862691in}{3.321257in}}%
\pgfpathlineto{\pgfqpoint{4.867200in}{3.419449in}}%
\pgfpathlineto{\pgfqpoint{4.871709in}{3.348443in}}%
\pgfpathlineto{\pgfqpoint{4.876218in}{3.435546in}}%
\pgfpathlineto{\pgfqpoint{4.880727in}{3.399417in}}%
\pgfpathlineto{\pgfqpoint{4.885236in}{3.383320in}}%
\pgfpathlineto{\pgfqpoint{4.889745in}{3.242202in}}%
\pgfpathlineto{\pgfqpoint{4.894255in}{3.679865in}}%
\pgfpathlineto{\pgfqpoint{4.898764in}{3.428571in}}%
\pgfpathlineto{\pgfqpoint{4.903273in}{3.425352in}}%
\pgfpathlineto{\pgfqpoint{4.907782in}{3.626923in}}%
\pgfpathlineto{\pgfqpoint{4.912291in}{3.351663in}}%
\pgfpathlineto{\pgfqpoint{4.916800in}{3.580420in}}%
\pgfpathlineto{\pgfqpoint{4.921309in}{3.352915in}}%
\pgfpathlineto{\pgfqpoint{4.925818in}{3.407287in}}%
\pgfpathlineto{\pgfqpoint{4.930327in}{3.414263in}}%
\pgfpathlineto{\pgfqpoint{4.934836in}{3.379564in}}%
\pgfpathlineto{\pgfqpoint{4.939345in}{3.511024in}}%
\pgfpathlineto{\pgfqpoint{4.943855in}{3.422490in}}%
\pgfpathlineto{\pgfqpoint{4.948364in}{3.620842in}}%
\pgfpathlineto{\pgfqpoint{4.952873in}{3.470424in}}%
\pgfpathlineto{\pgfqpoint{4.957382in}{3.449855in}}%
\pgfpathlineto{\pgfqpoint{4.961891in}{3.437693in}}%
\pgfpathlineto{\pgfqpoint{4.966400in}{3.757309in}}%
\pgfpathlineto{\pgfqpoint{4.970909in}{3.424994in}}%
\pgfpathlineto{\pgfqpoint{4.975418in}{3.379028in}}%
\pgfpathlineto{\pgfqpoint{4.979927in}{3.535348in}}%
\pgfpathlineto{\pgfqpoint{4.984436in}{3.536779in}}%
\pgfpathlineto{\pgfqpoint{4.993455in}{3.353451in}}%
\pgfpathlineto{\pgfqpoint{4.997964in}{3.487236in}}%
\pgfpathlineto{\pgfqpoint{5.002473in}{3.395661in}}%
\pgfpathlineto{\pgfqpoint{5.006982in}{3.470602in}}%
\pgfpathlineto{\pgfqpoint{5.011491in}{3.467562in}}%
\pgfpathlineto{\pgfqpoint{5.016000in}{3.594550in}}%
\pgfpathlineto{\pgfqpoint{5.020509in}{3.518536in}}%
\pgfpathlineto{\pgfqpoint{5.025018in}{3.419807in}}%
\pgfpathlineto{\pgfqpoint{5.029527in}{3.388328in}}%
\pgfpathlineto{\pgfqpoint{5.034036in}{3.389580in}}%
\pgfpathlineto{\pgfqpoint{5.038545in}{3.508878in}}%
\pgfpathlineto{\pgfqpoint{5.043055in}{3.463627in}}%
\pgfpathlineto{\pgfqpoint{5.047564in}{3.521755in}}%
\pgfpathlineto{\pgfqpoint{5.052073in}{3.473464in}}%
\pgfpathlineto{\pgfqpoint{5.056582in}{3.392621in}}%
\pgfpathlineto{\pgfqpoint{5.061091in}{3.437335in}}%
\pgfpathlineto{\pgfqpoint{5.065600in}{3.502439in}}%
\pgfpathlineto{\pgfqpoint{5.070109in}{3.596517in}}%
\pgfpathlineto{\pgfqpoint{5.074618in}{3.600095in}}%
\pgfpathlineto{\pgfqpoint{5.079127in}{3.588648in}}%
\pgfpathlineto{\pgfqpoint{5.083636in}{3.506910in}}%
\pgfpathlineto{\pgfqpoint{5.088145in}{3.626386in}}%
\pgfpathlineto{\pgfqpoint{5.092655in}{3.588469in}}%
\pgfpathlineto{\pgfqpoint{5.097164in}{3.594192in}}%
\pgfpathlineto{\pgfqpoint{5.101673in}{3.519251in}}%
\pgfpathlineto{\pgfqpoint{5.106182in}{3.563966in}}%
\pgfpathlineto{\pgfqpoint{5.110691in}{3.561998in}}%
\pgfpathlineto{\pgfqpoint{5.115200in}{3.736026in}}%
\pgfpathlineto{\pgfqpoint{5.119709in}{3.673426in}}%
\pgfpathlineto{\pgfqpoint{5.124218in}{3.493317in}}%
\pgfpathlineto{\pgfqpoint{5.128727in}{3.762139in}}%
\pgfpathlineto{\pgfqpoint{5.133236in}{3.551446in}}%
\pgfpathlineto{\pgfqpoint{5.137745in}{3.544291in}}%
\pgfpathlineto{\pgfqpoint{5.142255in}{3.577201in}}%
\pgfpathlineto{\pgfqpoint{5.146764in}{3.662873in}}%
\pgfpathlineto{\pgfqpoint{5.151273in}{3.632110in}}%
\pgfpathlineto{\pgfqpoint{5.160291in}{3.489919in}}%
\pgfpathlineto{\pgfqpoint{5.164800in}{3.725115in}}%
\pgfpathlineto{\pgfqpoint{5.169309in}{3.671816in}}%
\pgfpathlineto{\pgfqpoint{5.173818in}{3.644451in}}%
\pgfpathlineto{\pgfqpoint{5.178327in}{3.716709in}}%
\pgfpathlineto{\pgfqpoint{5.182836in}{3.679507in}}%
\pgfpathlineto{\pgfqpoint{5.187345in}{3.627102in}}%
\pgfpathlineto{\pgfqpoint{5.191855in}{3.751944in}}%
\pgfpathlineto{\pgfqpoint{5.196364in}{3.643914in}}%
\pgfpathlineto{\pgfqpoint{5.200873in}{3.592404in}}%
\pgfpathlineto{\pgfqpoint{5.205382in}{3.702579in}}%
\pgfpathlineto{\pgfqpoint{5.209891in}{3.776447in}}%
\pgfpathlineto{\pgfqpoint{5.218909in}{3.591867in}}%
\pgfpathlineto{\pgfqpoint{5.223418in}{3.687555in}}%
\pgfpathlineto{\pgfqpoint{5.227927in}{3.616013in}}%
\pgfpathlineto{\pgfqpoint{5.236945in}{3.688629in}}%
\pgfpathlineto{\pgfqpoint{5.241455in}{3.704010in}}%
\pgfpathlineto{\pgfqpoint{5.245964in}{3.629964in}}%
\pgfpathlineto{\pgfqpoint{5.250473in}{3.581315in}}%
\pgfpathlineto{\pgfqpoint{5.254982in}{3.696498in}}%
\pgfpathlineto{\pgfqpoint{5.259491in}{3.946003in}}%
\pgfpathlineto{\pgfqpoint{5.264000in}{3.712416in}}%
\pgfpathlineto{\pgfqpoint{5.268509in}{3.734595in}}%
\pgfpathlineto{\pgfqpoint{5.273018in}{3.599379in}}%
\pgfpathlineto{\pgfqpoint{5.282036in}{3.923467in}}%
\pgfpathlineto{\pgfqpoint{5.286545in}{3.689344in}}%
\pgfpathlineto{\pgfqpoint{5.291055in}{3.666271in}}%
\pgfpathlineto{\pgfqpoint{5.295564in}{3.650532in}}%
\pgfpathlineto{\pgfqpoint{5.300073in}{3.720107in}}%
\pgfpathlineto{\pgfqpoint{5.304582in}{3.811861in}}%
\pgfpathlineto{\pgfqpoint{5.309091in}{3.645703in}}%
\pgfpathlineto{\pgfqpoint{5.313600in}{3.868201in}}%
\pgfpathlineto{\pgfqpoint{5.318109in}{3.685051in}}%
\pgfpathlineto{\pgfqpoint{5.322618in}{4.008961in}}%
\pgfpathlineto{\pgfqpoint{5.327127in}{3.884119in}}%
\pgfpathlineto{\pgfqpoint{5.336145in}{3.556990in}}%
\pgfpathlineto{\pgfqpoint{5.340655in}{3.679686in}}%
\pgfpathlineto{\pgfqpoint{5.345164in}{3.765895in}}%
\pgfpathlineto{\pgfqpoint{5.349673in}{3.706156in}}%
\pgfpathlineto{\pgfqpoint{5.354182in}{3.925256in}}%
\pgfpathlineto{\pgfqpoint{5.363200in}{3.705620in}}%
\pgfpathlineto{\pgfqpoint{5.367709in}{3.790756in}}%
\pgfpathlineto{\pgfqpoint{5.372218in}{3.824560in}}%
\pgfpathlineto{\pgfqpoint{5.376727in}{3.908443in}}%
\pgfpathlineto{\pgfqpoint{5.381236in}{3.917744in}}%
\pgfpathlineto{\pgfqpoint{5.385745in}{3.732985in}}%
\pgfpathlineto{\pgfqpoint{5.394764in}{3.864623in}}%
\pgfpathlineto{\pgfqpoint{5.399273in}{3.656613in}}%
\pgfpathlineto{\pgfqpoint{5.403782in}{3.887159in}}%
\pgfpathlineto{\pgfqpoint{5.408291in}{3.750513in}}%
\pgfpathlineto{\pgfqpoint{5.412800in}{3.744253in}}%
\pgfpathlineto{\pgfqpoint{5.417309in}{3.901468in}}%
\pgfpathlineto{\pgfqpoint{5.421818in}{3.732448in}}%
\pgfpathlineto{\pgfqpoint{5.426327in}{3.849063in}}%
\pgfpathlineto{\pgfqpoint{5.430836in}{3.811324in}}%
\pgfpathlineto{\pgfqpoint{5.435345in}{3.671279in}}%
\pgfpathlineto{\pgfqpoint{5.439855in}{3.853713in}}%
\pgfpathlineto{\pgfqpoint{5.444364in}{3.823486in}}%
\pgfpathlineto{\pgfqpoint{5.448873in}{4.012180in}}%
\pgfpathlineto{\pgfqpoint{5.453382in}{4.056000in}}%
\pgfpathlineto{\pgfqpoint{5.457891in}{3.939385in}}%
\pgfpathlineto{\pgfqpoint{5.462400in}{3.741928in}}%
\pgfpathlineto{\pgfqpoint{5.466909in}{3.974620in}}%
\pgfpathlineto{\pgfqpoint{5.471418in}{3.853713in}}%
\pgfpathlineto{\pgfqpoint{5.475927in}{4.046699in}}%
\pgfpathlineto{\pgfqpoint{5.480436in}{3.802202in}}%
\pgfpathlineto{\pgfqpoint{5.484945in}{3.956913in}}%
\pgfpathlineto{\pgfqpoint{5.489455in}{3.901110in}}%
\pgfpathlineto{\pgfqpoint{5.493964in}{3.864981in}}%
\pgfpathlineto{\pgfqpoint{5.498473in}{3.996441in}}%
\pgfpathlineto{\pgfqpoint{5.502982in}{3.925077in}}%
\pgfpathlineto{\pgfqpoint{5.507491in}{3.951369in}}%
\pgfpathlineto{\pgfqpoint{5.512000in}{3.851567in}}%
\pgfpathlineto{\pgfqpoint{5.516509in}{3.938670in}}%
\pgfpathlineto{\pgfqpoint{5.521018in}{3.804706in}}%
\pgfpathlineto{\pgfqpoint{5.525527in}{3.931695in}}%
\pgfpathlineto{\pgfqpoint{5.530036in}{3.865160in}}%
\pgfpathlineto{\pgfqpoint{5.534545in}{4.027383in}}%
\pgfpathlineto{\pgfqpoint{5.534545in}{4.027383in}}%
\pgfusepath{stroke}%
\end{pgfscope}%
\begin{pgfscope}%
\pgfsetrectcap%
\pgfsetmiterjoin%
\pgfsetlinewidth{0.803000pt}%
\definecolor{currentstroke}{rgb}{0.000000,0.000000,0.000000}%
\pgfsetstrokecolor{currentstroke}%
\pgfsetdash{}{0pt}%
\pgfpathmoveto{\pgfqpoint{0.800000in}{0.528000in}}%
\pgfpathlineto{\pgfqpoint{0.800000in}{4.224000in}}%
\pgfusepath{stroke}%
\end{pgfscope}%
\begin{pgfscope}%
\pgfsetrectcap%
\pgfsetmiterjoin%
\pgfsetlinewidth{0.803000pt}%
\definecolor{currentstroke}{rgb}{0.000000,0.000000,0.000000}%
\pgfsetstrokecolor{currentstroke}%
\pgfsetdash{}{0pt}%
\pgfpathmoveto{\pgfqpoint{5.760000in}{0.528000in}}%
\pgfpathlineto{\pgfqpoint{5.760000in}{4.224000in}}%
\pgfusepath{stroke}%
\end{pgfscope}%
\begin{pgfscope}%
\pgfsetrectcap%
\pgfsetmiterjoin%
\pgfsetlinewidth{0.803000pt}%
\definecolor{currentstroke}{rgb}{0.000000,0.000000,0.000000}%
\pgfsetstrokecolor{currentstroke}%
\pgfsetdash{}{0pt}%
\pgfpathmoveto{\pgfqpoint{0.800000in}{0.528000in}}%
\pgfpathlineto{\pgfqpoint{5.760000in}{0.528000in}}%
\pgfusepath{stroke}%
\end{pgfscope}%
\begin{pgfscope}%
\pgfsetrectcap%
\pgfsetmiterjoin%
\pgfsetlinewidth{0.803000pt}%
\definecolor{currentstroke}{rgb}{0.000000,0.000000,0.000000}%
\pgfsetstrokecolor{currentstroke}%
\pgfsetdash{}{0pt}%
\pgfpathmoveto{\pgfqpoint{0.800000in}{4.224000in}}%
\pgfpathlineto{\pgfqpoint{5.760000in}{4.224000in}}%
\pgfusepath{stroke}%
\end{pgfscope}%
\begin{pgfscope}%
\definecolor{textcolor}{rgb}{0.000000,0.000000,0.000000}%
\pgfsetstrokecolor{textcolor}%
\pgfsetfillcolor{textcolor}%
\pgftext[x=3.280000in,y=4.307333in,,base]{\color{textcolor}\ttfamily\fontsize{12.000000}{14.400000}\selectfont Quick Sort  Comparisons vs Input size}%
\end{pgfscope}%
\begin{pgfscope}%
\pgfsetbuttcap%
\pgfsetmiterjoin%
\definecolor{currentfill}{rgb}{1.000000,1.000000,1.000000}%
\pgfsetfillcolor{currentfill}%
\pgfsetfillopacity{0.800000}%
\pgfsetlinewidth{1.003750pt}%
\definecolor{currentstroke}{rgb}{0.800000,0.800000,0.800000}%
\pgfsetstrokecolor{currentstroke}%
\pgfsetstrokeopacity{0.800000}%
\pgfsetdash{}{0pt}%
\pgfpathmoveto{\pgfqpoint{0.897222in}{3.908286in}}%
\pgfpathlineto{\pgfqpoint{1.759758in}{3.908286in}}%
\pgfpathquadraticcurveto{\pgfqpoint{1.787535in}{3.908286in}}{\pgfqpoint{1.787535in}{3.936063in}}%
\pgfpathlineto{\pgfqpoint{1.787535in}{4.126778in}}%
\pgfpathquadraticcurveto{\pgfqpoint{1.787535in}{4.154556in}}{\pgfqpoint{1.759758in}{4.154556in}}%
\pgfpathlineto{\pgfqpoint{0.897222in}{4.154556in}}%
\pgfpathquadraticcurveto{\pgfqpoint{0.869444in}{4.154556in}}{\pgfqpoint{0.869444in}{4.126778in}}%
\pgfpathlineto{\pgfqpoint{0.869444in}{3.936063in}}%
\pgfpathquadraticcurveto{\pgfqpoint{0.869444in}{3.908286in}}{\pgfqpoint{0.897222in}{3.908286in}}%
\pgfpathlineto{\pgfqpoint{0.897222in}{3.908286in}}%
\pgfpathclose%
\pgfusepath{stroke,fill}%
\end{pgfscope}%
\begin{pgfscope}%
\pgfsetrectcap%
\pgfsetroundjoin%
\pgfsetlinewidth{1.505625pt}%
\definecolor{currentstroke}{rgb}{0.000000,1.000000,0.498039}%
\pgfsetstrokecolor{currentstroke}%
\pgfsetdash{}{0pt}%
\pgfpathmoveto{\pgfqpoint{0.925000in}{4.041342in}}%
\pgfpathlineto{\pgfqpoint{1.063889in}{4.041342in}}%
\pgfpathlineto{\pgfqpoint{1.202778in}{4.041342in}}%
\pgfusepath{stroke}%
\end{pgfscope}%
\begin{pgfscope}%
\definecolor{textcolor}{rgb}{0.000000,0.000000,0.000000}%
\pgfsetstrokecolor{textcolor}%
\pgfsetfillcolor{textcolor}%
\pgftext[x=1.313889in,y=3.992731in,left,base]{\color{textcolor}\ttfamily\fontsize{10.000000}{12.000000}\selectfont Quick}%
\end{pgfscope}%
\end{pgfpicture}%
\makeatother%
\endgroup%

%% Creator: Matplotlib, PGF backend
%%
%% To include the figure in your LaTeX document, write
%%   \input{<filename>.pgf}
%%
%% Make sure the required packages are loaded in your preamble
%%   \usepackage{pgf}
%%
%% Also ensure that all the required font packages are loaded; for instance,
%% the lmodern package is sometimes necessary when using math font.
%%   \usepackage{lmodern}
%%
%% Figures using additional raster images can only be included by \input if
%% they are in the same directory as the main LaTeX file. For loading figures
%% from other directories you can use the `import` package
%%   \usepackage{import}
%%
%% and then include the figures with
%%   \import{<path to file>}{<filename>.pgf}
%%
%% Matplotlib used the following preamble
%%   \usepackage{fontspec}
%%   \setmainfont{DejaVuSerif.ttf}[Path=\detokenize{/home/dbk/.local/lib/python3.10/site-packages/matplotlib/mpl-data/fonts/ttf/}]
%%   \setsansfont{DejaVuSans.ttf}[Path=\detokenize{/home/dbk/.local/lib/python3.10/site-packages/matplotlib/mpl-data/fonts/ttf/}]
%%   \setmonofont{DejaVuSansMono.ttf}[Path=\detokenize{/home/dbk/.local/lib/python3.10/site-packages/matplotlib/mpl-data/fonts/ttf/}]
%%
\begingroup%
\makeatletter%
\begin{pgfpicture}%
\pgfpathrectangle{\pgfpointorigin}{\pgfqpoint{6.400000in}{4.800000in}}%
\pgfusepath{use as bounding box, clip}%
\begin{pgfscope}%
\pgfsetbuttcap%
\pgfsetmiterjoin%
\definecolor{currentfill}{rgb}{1.000000,1.000000,1.000000}%
\pgfsetfillcolor{currentfill}%
\pgfsetlinewidth{0.000000pt}%
\definecolor{currentstroke}{rgb}{1.000000,1.000000,1.000000}%
\pgfsetstrokecolor{currentstroke}%
\pgfsetdash{}{0pt}%
\pgfpathmoveto{\pgfqpoint{0.000000in}{0.000000in}}%
\pgfpathlineto{\pgfqpoint{6.400000in}{0.000000in}}%
\pgfpathlineto{\pgfqpoint{6.400000in}{4.800000in}}%
\pgfpathlineto{\pgfqpoint{0.000000in}{4.800000in}}%
\pgfpathlineto{\pgfqpoint{0.000000in}{0.000000in}}%
\pgfpathclose%
\pgfusepath{fill}%
\end{pgfscope}%
\begin{pgfscope}%
\pgfsetbuttcap%
\pgfsetmiterjoin%
\definecolor{currentfill}{rgb}{1.000000,1.000000,1.000000}%
\pgfsetfillcolor{currentfill}%
\pgfsetlinewidth{0.000000pt}%
\definecolor{currentstroke}{rgb}{0.000000,0.000000,0.000000}%
\pgfsetstrokecolor{currentstroke}%
\pgfsetstrokeopacity{0.000000}%
\pgfsetdash{}{0pt}%
\pgfpathmoveto{\pgfqpoint{0.800000in}{0.528000in}}%
\pgfpathlineto{\pgfqpoint{5.760000in}{0.528000in}}%
\pgfpathlineto{\pgfqpoint{5.760000in}{4.224000in}}%
\pgfpathlineto{\pgfqpoint{0.800000in}{4.224000in}}%
\pgfpathlineto{\pgfqpoint{0.800000in}{0.528000in}}%
\pgfpathclose%
\pgfusepath{fill}%
\end{pgfscope}%
\begin{pgfscope}%
\pgfsetbuttcap%
\pgfsetroundjoin%
\definecolor{currentfill}{rgb}{0.000000,0.000000,0.000000}%
\pgfsetfillcolor{currentfill}%
\pgfsetlinewidth{0.803000pt}%
\definecolor{currentstroke}{rgb}{0.000000,0.000000,0.000000}%
\pgfsetstrokecolor{currentstroke}%
\pgfsetdash{}{0pt}%
\pgfsys@defobject{currentmarker}{\pgfqpoint{0.000000in}{-0.048611in}}{\pgfqpoint{0.000000in}{0.000000in}}{%
\pgfpathmoveto{\pgfqpoint{0.000000in}{0.000000in}}%
\pgfpathlineto{\pgfqpoint{0.000000in}{-0.048611in}}%
\pgfusepath{stroke,fill}%
}%
\begin{pgfscope}%
\pgfsys@transformshift{1.020945in}{0.528000in}%
\pgfsys@useobject{currentmarker}{}%
\end{pgfscope}%
\end{pgfscope}%
\begin{pgfscope}%
\definecolor{textcolor}{rgb}{0.000000,0.000000,0.000000}%
\pgfsetstrokecolor{textcolor}%
\pgfsetfillcolor{textcolor}%
\pgftext[x=1.020945in,y=0.430778in,,top]{\color{textcolor}\ttfamily\fontsize{10.000000}{12.000000}\selectfont 0}%
\end{pgfscope}%
\begin{pgfscope}%
\pgfsetbuttcap%
\pgfsetroundjoin%
\definecolor{currentfill}{rgb}{0.000000,0.000000,0.000000}%
\pgfsetfillcolor{currentfill}%
\pgfsetlinewidth{0.803000pt}%
\definecolor{currentstroke}{rgb}{0.000000,0.000000,0.000000}%
\pgfsetstrokecolor{currentstroke}%
\pgfsetdash{}{0pt}%
\pgfsys@defobject{currentmarker}{\pgfqpoint{0.000000in}{-0.048611in}}{\pgfqpoint{0.000000in}{0.000000in}}{%
\pgfpathmoveto{\pgfqpoint{0.000000in}{0.000000in}}%
\pgfpathlineto{\pgfqpoint{0.000000in}{-0.048611in}}%
\pgfusepath{stroke,fill}%
}%
\begin{pgfscope}%
\pgfsys@transformshift{1.922764in}{0.528000in}%
\pgfsys@useobject{currentmarker}{}%
\end{pgfscope}%
\end{pgfscope}%
\begin{pgfscope}%
\definecolor{textcolor}{rgb}{0.000000,0.000000,0.000000}%
\pgfsetstrokecolor{textcolor}%
\pgfsetfillcolor{textcolor}%
\pgftext[x=1.922764in,y=0.430778in,,top]{\color{textcolor}\ttfamily\fontsize{10.000000}{12.000000}\selectfont 200}%
\end{pgfscope}%
\begin{pgfscope}%
\pgfsetbuttcap%
\pgfsetroundjoin%
\definecolor{currentfill}{rgb}{0.000000,0.000000,0.000000}%
\pgfsetfillcolor{currentfill}%
\pgfsetlinewidth{0.803000pt}%
\definecolor{currentstroke}{rgb}{0.000000,0.000000,0.000000}%
\pgfsetstrokecolor{currentstroke}%
\pgfsetdash{}{0pt}%
\pgfsys@defobject{currentmarker}{\pgfqpoint{0.000000in}{-0.048611in}}{\pgfqpoint{0.000000in}{0.000000in}}{%
\pgfpathmoveto{\pgfqpoint{0.000000in}{0.000000in}}%
\pgfpathlineto{\pgfqpoint{0.000000in}{-0.048611in}}%
\pgfusepath{stroke,fill}%
}%
\begin{pgfscope}%
\pgfsys@transformshift{2.824582in}{0.528000in}%
\pgfsys@useobject{currentmarker}{}%
\end{pgfscope}%
\end{pgfscope}%
\begin{pgfscope}%
\definecolor{textcolor}{rgb}{0.000000,0.000000,0.000000}%
\pgfsetstrokecolor{textcolor}%
\pgfsetfillcolor{textcolor}%
\pgftext[x=2.824582in,y=0.430778in,,top]{\color{textcolor}\ttfamily\fontsize{10.000000}{12.000000}\selectfont 400}%
\end{pgfscope}%
\begin{pgfscope}%
\pgfsetbuttcap%
\pgfsetroundjoin%
\definecolor{currentfill}{rgb}{0.000000,0.000000,0.000000}%
\pgfsetfillcolor{currentfill}%
\pgfsetlinewidth{0.803000pt}%
\definecolor{currentstroke}{rgb}{0.000000,0.000000,0.000000}%
\pgfsetstrokecolor{currentstroke}%
\pgfsetdash{}{0pt}%
\pgfsys@defobject{currentmarker}{\pgfqpoint{0.000000in}{-0.048611in}}{\pgfqpoint{0.000000in}{0.000000in}}{%
\pgfpathmoveto{\pgfqpoint{0.000000in}{0.000000in}}%
\pgfpathlineto{\pgfqpoint{0.000000in}{-0.048611in}}%
\pgfusepath{stroke,fill}%
}%
\begin{pgfscope}%
\pgfsys@transformshift{3.726400in}{0.528000in}%
\pgfsys@useobject{currentmarker}{}%
\end{pgfscope}%
\end{pgfscope}%
\begin{pgfscope}%
\definecolor{textcolor}{rgb}{0.000000,0.000000,0.000000}%
\pgfsetstrokecolor{textcolor}%
\pgfsetfillcolor{textcolor}%
\pgftext[x=3.726400in,y=0.430778in,,top]{\color{textcolor}\ttfamily\fontsize{10.000000}{12.000000}\selectfont 600}%
\end{pgfscope}%
\begin{pgfscope}%
\pgfsetbuttcap%
\pgfsetroundjoin%
\definecolor{currentfill}{rgb}{0.000000,0.000000,0.000000}%
\pgfsetfillcolor{currentfill}%
\pgfsetlinewidth{0.803000pt}%
\definecolor{currentstroke}{rgb}{0.000000,0.000000,0.000000}%
\pgfsetstrokecolor{currentstroke}%
\pgfsetdash{}{0pt}%
\pgfsys@defobject{currentmarker}{\pgfqpoint{0.000000in}{-0.048611in}}{\pgfqpoint{0.000000in}{0.000000in}}{%
\pgfpathmoveto{\pgfqpoint{0.000000in}{0.000000in}}%
\pgfpathlineto{\pgfqpoint{0.000000in}{-0.048611in}}%
\pgfusepath{stroke,fill}%
}%
\begin{pgfscope}%
\pgfsys@transformshift{4.628218in}{0.528000in}%
\pgfsys@useobject{currentmarker}{}%
\end{pgfscope}%
\end{pgfscope}%
\begin{pgfscope}%
\definecolor{textcolor}{rgb}{0.000000,0.000000,0.000000}%
\pgfsetstrokecolor{textcolor}%
\pgfsetfillcolor{textcolor}%
\pgftext[x=4.628218in,y=0.430778in,,top]{\color{textcolor}\ttfamily\fontsize{10.000000}{12.000000}\selectfont 800}%
\end{pgfscope}%
\begin{pgfscope}%
\pgfsetbuttcap%
\pgfsetroundjoin%
\definecolor{currentfill}{rgb}{0.000000,0.000000,0.000000}%
\pgfsetfillcolor{currentfill}%
\pgfsetlinewidth{0.803000pt}%
\definecolor{currentstroke}{rgb}{0.000000,0.000000,0.000000}%
\pgfsetstrokecolor{currentstroke}%
\pgfsetdash{}{0pt}%
\pgfsys@defobject{currentmarker}{\pgfqpoint{0.000000in}{-0.048611in}}{\pgfqpoint{0.000000in}{0.000000in}}{%
\pgfpathmoveto{\pgfqpoint{0.000000in}{0.000000in}}%
\pgfpathlineto{\pgfqpoint{0.000000in}{-0.048611in}}%
\pgfusepath{stroke,fill}%
}%
\begin{pgfscope}%
\pgfsys@transformshift{5.530036in}{0.528000in}%
\pgfsys@useobject{currentmarker}{}%
\end{pgfscope}%
\end{pgfscope}%
\begin{pgfscope}%
\definecolor{textcolor}{rgb}{0.000000,0.000000,0.000000}%
\pgfsetstrokecolor{textcolor}%
\pgfsetfillcolor{textcolor}%
\pgftext[x=5.530036in,y=0.430778in,,top]{\color{textcolor}\ttfamily\fontsize{10.000000}{12.000000}\selectfont 1000}%
\end{pgfscope}%
\begin{pgfscope}%
\definecolor{textcolor}{rgb}{0.000000,0.000000,0.000000}%
\pgfsetstrokecolor{textcolor}%
\pgfsetfillcolor{textcolor}%
\pgftext[x=3.280000in,y=0.240063in,,top]{\color{textcolor}\ttfamily\fontsize{10.000000}{12.000000}\selectfont Size of Array}%
\end{pgfscope}%
\begin{pgfscope}%
\pgfsetbuttcap%
\pgfsetroundjoin%
\definecolor{currentfill}{rgb}{0.000000,0.000000,0.000000}%
\pgfsetfillcolor{currentfill}%
\pgfsetlinewidth{0.803000pt}%
\definecolor{currentstroke}{rgb}{0.000000,0.000000,0.000000}%
\pgfsetstrokecolor{currentstroke}%
\pgfsetdash{}{0pt}%
\pgfsys@defobject{currentmarker}{\pgfqpoint{-0.048611in}{0.000000in}}{\pgfqpoint{-0.000000in}{0.000000in}}{%
\pgfpathmoveto{\pgfqpoint{-0.000000in}{0.000000in}}%
\pgfpathlineto{\pgfqpoint{-0.048611in}{0.000000in}}%
\pgfusepath{stroke,fill}%
}%
\begin{pgfscope}%
\pgfsys@transformshift{0.800000in}{0.988266in}%
\pgfsys@useobject{currentmarker}{}%
\end{pgfscope}%
\end{pgfscope}%
\begin{pgfscope}%
\definecolor{textcolor}{rgb}{0.000000,0.000000,0.000000}%
\pgfsetstrokecolor{textcolor}%
\pgfsetfillcolor{textcolor}%
\pgftext[x=0.451923in, y=0.935131in, left, base]{\color{textcolor}\ttfamily\fontsize{10.000000}{12.000000}\selectfont 500}%
\end{pgfscope}%
\begin{pgfscope}%
\pgfsetbuttcap%
\pgfsetroundjoin%
\definecolor{currentfill}{rgb}{0.000000,0.000000,0.000000}%
\pgfsetfillcolor{currentfill}%
\pgfsetlinewidth{0.803000pt}%
\definecolor{currentstroke}{rgb}{0.000000,0.000000,0.000000}%
\pgfsetstrokecolor{currentstroke}%
\pgfsetdash{}{0pt}%
\pgfsys@defobject{currentmarker}{\pgfqpoint{-0.048611in}{0.000000in}}{\pgfqpoint{-0.000000in}{0.000000in}}{%
\pgfpathmoveto{\pgfqpoint{-0.000000in}{0.000000in}}%
\pgfpathlineto{\pgfqpoint{-0.048611in}{0.000000in}}%
\pgfusepath{stroke,fill}%
}%
\begin{pgfscope}%
\pgfsys@transformshift{0.800000in}{1.517733in}%
\pgfsys@useobject{currentmarker}{}%
\end{pgfscope}%
\end{pgfscope}%
\begin{pgfscope}%
\definecolor{textcolor}{rgb}{0.000000,0.000000,0.000000}%
\pgfsetstrokecolor{textcolor}%
\pgfsetfillcolor{textcolor}%
\pgftext[x=0.368305in, y=1.464599in, left, base]{\color{textcolor}\ttfamily\fontsize{10.000000}{12.000000}\selectfont 1000}%
\end{pgfscope}%
\begin{pgfscope}%
\pgfsetbuttcap%
\pgfsetroundjoin%
\definecolor{currentfill}{rgb}{0.000000,0.000000,0.000000}%
\pgfsetfillcolor{currentfill}%
\pgfsetlinewidth{0.803000pt}%
\definecolor{currentstroke}{rgb}{0.000000,0.000000,0.000000}%
\pgfsetstrokecolor{currentstroke}%
\pgfsetdash{}{0pt}%
\pgfsys@defobject{currentmarker}{\pgfqpoint{-0.048611in}{0.000000in}}{\pgfqpoint{-0.000000in}{0.000000in}}{%
\pgfpathmoveto{\pgfqpoint{-0.000000in}{0.000000in}}%
\pgfpathlineto{\pgfqpoint{-0.048611in}{0.000000in}}%
\pgfusepath{stroke,fill}%
}%
\begin{pgfscope}%
\pgfsys@transformshift{0.800000in}{2.047201in}%
\pgfsys@useobject{currentmarker}{}%
\end{pgfscope}%
\end{pgfscope}%
\begin{pgfscope}%
\definecolor{textcolor}{rgb}{0.000000,0.000000,0.000000}%
\pgfsetstrokecolor{textcolor}%
\pgfsetfillcolor{textcolor}%
\pgftext[x=0.368305in, y=1.994066in, left, base]{\color{textcolor}\ttfamily\fontsize{10.000000}{12.000000}\selectfont 1500}%
\end{pgfscope}%
\begin{pgfscope}%
\pgfsetbuttcap%
\pgfsetroundjoin%
\definecolor{currentfill}{rgb}{0.000000,0.000000,0.000000}%
\pgfsetfillcolor{currentfill}%
\pgfsetlinewidth{0.803000pt}%
\definecolor{currentstroke}{rgb}{0.000000,0.000000,0.000000}%
\pgfsetstrokecolor{currentstroke}%
\pgfsetdash{}{0pt}%
\pgfsys@defobject{currentmarker}{\pgfqpoint{-0.048611in}{0.000000in}}{\pgfqpoint{-0.000000in}{0.000000in}}{%
\pgfpathmoveto{\pgfqpoint{-0.000000in}{0.000000in}}%
\pgfpathlineto{\pgfqpoint{-0.048611in}{0.000000in}}%
\pgfusepath{stroke,fill}%
}%
\begin{pgfscope}%
\pgfsys@transformshift{0.800000in}{2.576668in}%
\pgfsys@useobject{currentmarker}{}%
\end{pgfscope}%
\end{pgfscope}%
\begin{pgfscope}%
\definecolor{textcolor}{rgb}{0.000000,0.000000,0.000000}%
\pgfsetstrokecolor{textcolor}%
\pgfsetfillcolor{textcolor}%
\pgftext[x=0.368305in, y=2.523534in, left, base]{\color{textcolor}\ttfamily\fontsize{10.000000}{12.000000}\selectfont 2000}%
\end{pgfscope}%
\begin{pgfscope}%
\pgfsetbuttcap%
\pgfsetroundjoin%
\definecolor{currentfill}{rgb}{0.000000,0.000000,0.000000}%
\pgfsetfillcolor{currentfill}%
\pgfsetlinewidth{0.803000pt}%
\definecolor{currentstroke}{rgb}{0.000000,0.000000,0.000000}%
\pgfsetstrokecolor{currentstroke}%
\pgfsetdash{}{0pt}%
\pgfsys@defobject{currentmarker}{\pgfqpoint{-0.048611in}{0.000000in}}{\pgfqpoint{-0.000000in}{0.000000in}}{%
\pgfpathmoveto{\pgfqpoint{-0.000000in}{0.000000in}}%
\pgfpathlineto{\pgfqpoint{-0.048611in}{0.000000in}}%
\pgfusepath{stroke,fill}%
}%
\begin{pgfscope}%
\pgfsys@transformshift{0.800000in}{3.106136in}%
\pgfsys@useobject{currentmarker}{}%
\end{pgfscope}%
\end{pgfscope}%
\begin{pgfscope}%
\definecolor{textcolor}{rgb}{0.000000,0.000000,0.000000}%
\pgfsetstrokecolor{textcolor}%
\pgfsetfillcolor{textcolor}%
\pgftext[x=0.368305in, y=3.053001in, left, base]{\color{textcolor}\ttfamily\fontsize{10.000000}{12.000000}\selectfont 2500}%
\end{pgfscope}%
\begin{pgfscope}%
\pgfsetbuttcap%
\pgfsetroundjoin%
\definecolor{currentfill}{rgb}{0.000000,0.000000,0.000000}%
\pgfsetfillcolor{currentfill}%
\pgfsetlinewidth{0.803000pt}%
\definecolor{currentstroke}{rgb}{0.000000,0.000000,0.000000}%
\pgfsetstrokecolor{currentstroke}%
\pgfsetdash{}{0pt}%
\pgfsys@defobject{currentmarker}{\pgfqpoint{-0.048611in}{0.000000in}}{\pgfqpoint{-0.000000in}{0.000000in}}{%
\pgfpathmoveto{\pgfqpoint{-0.000000in}{0.000000in}}%
\pgfpathlineto{\pgfqpoint{-0.048611in}{0.000000in}}%
\pgfusepath{stroke,fill}%
}%
\begin{pgfscope}%
\pgfsys@transformshift{0.800000in}{3.635603in}%
\pgfsys@useobject{currentmarker}{}%
\end{pgfscope}%
\end{pgfscope}%
\begin{pgfscope}%
\definecolor{textcolor}{rgb}{0.000000,0.000000,0.000000}%
\pgfsetstrokecolor{textcolor}%
\pgfsetfillcolor{textcolor}%
\pgftext[x=0.368305in, y=3.582468in, left, base]{\color{textcolor}\ttfamily\fontsize{10.000000}{12.000000}\selectfont 3000}%
\end{pgfscope}%
\begin{pgfscope}%
\pgfsetbuttcap%
\pgfsetroundjoin%
\definecolor{currentfill}{rgb}{0.000000,0.000000,0.000000}%
\pgfsetfillcolor{currentfill}%
\pgfsetlinewidth{0.803000pt}%
\definecolor{currentstroke}{rgb}{0.000000,0.000000,0.000000}%
\pgfsetstrokecolor{currentstroke}%
\pgfsetdash{}{0pt}%
\pgfsys@defobject{currentmarker}{\pgfqpoint{-0.048611in}{0.000000in}}{\pgfqpoint{-0.000000in}{0.000000in}}{%
\pgfpathmoveto{\pgfqpoint{-0.000000in}{0.000000in}}%
\pgfpathlineto{\pgfqpoint{-0.048611in}{0.000000in}}%
\pgfusepath{stroke,fill}%
}%
\begin{pgfscope}%
\pgfsys@transformshift{0.800000in}{4.165070in}%
\pgfsys@useobject{currentmarker}{}%
\end{pgfscope}%
\end{pgfscope}%
\begin{pgfscope}%
\definecolor{textcolor}{rgb}{0.000000,0.000000,0.000000}%
\pgfsetstrokecolor{textcolor}%
\pgfsetfillcolor{textcolor}%
\pgftext[x=0.368305in, y=4.111936in, left, base]{\color{textcolor}\ttfamily\fontsize{10.000000}{12.000000}\selectfont 3500}%
\end{pgfscope}%
\begin{pgfscope}%
\definecolor{textcolor}{rgb}{0.000000,0.000000,0.000000}%
\pgfsetstrokecolor{textcolor}%
\pgfsetfillcolor{textcolor}%
\pgftext[x=0.312750in,y=2.376000in,,bottom,rotate=90.000000]{\color{textcolor}\ttfamily\fontsize{10.000000}{12.000000}\selectfont Swaps}%
\end{pgfscope}%
\begin{pgfscope}%
\pgfpathrectangle{\pgfqpoint{0.800000in}{0.528000in}}{\pgfqpoint{4.960000in}{3.696000in}}%
\pgfusepath{clip}%
\pgfsetrectcap%
\pgfsetroundjoin%
\pgfsetlinewidth{1.505625pt}%
\definecolor{currentstroke}{rgb}{0.000000,1.000000,0.498039}%
\pgfsetstrokecolor{currentstroke}%
\pgfsetdash{}{0pt}%
\pgfpathmoveto{\pgfqpoint{1.025455in}{0.696000in}}%
\pgfpathlineto{\pgfqpoint{1.029964in}{0.704471in}}%
\pgfpathlineto{\pgfqpoint{1.034473in}{0.720355in}}%
\pgfpathlineto{\pgfqpoint{1.038982in}{0.712943in}}%
\pgfpathlineto{\pgfqpoint{1.043491in}{0.719297in}}%
\pgfpathlineto{\pgfqpoint{1.048000in}{0.714002in}}%
\pgfpathlineto{\pgfqpoint{1.052509in}{0.719297in}}%
\pgfpathlineto{\pgfqpoint{1.057018in}{0.721414in}}%
\pgfpathlineto{\pgfqpoint{1.061527in}{0.707648in}}%
\pgfpathlineto{\pgfqpoint{1.066036in}{0.743652in}}%
\pgfpathlineto{\pgfqpoint{1.070545in}{0.723532in}}%
\pgfpathlineto{\pgfqpoint{1.075055in}{0.723532in}}%
\pgfpathlineto{\pgfqpoint{1.079564in}{0.725650in}}%
\pgfpathlineto{\pgfqpoint{1.084073in}{0.740475in}}%
\pgfpathlineto{\pgfqpoint{1.088582in}{0.740475in}}%
\pgfpathlineto{\pgfqpoint{1.093091in}{0.747888in}}%
\pgfpathlineto{\pgfqpoint{1.097600in}{0.742593in}}%
\pgfpathlineto{\pgfqpoint{1.102109in}{0.764831in}}%
\pgfpathlineto{\pgfqpoint{1.106618in}{0.747888in}}%
\pgfpathlineto{\pgfqpoint{1.111127in}{0.759536in}}%
\pgfpathlineto{\pgfqpoint{1.115636in}{0.761654in}}%
\pgfpathlineto{\pgfqpoint{1.120145in}{0.761654in}}%
\pgfpathlineto{\pgfqpoint{1.129164in}{0.774361in}}%
\pgfpathlineto{\pgfqpoint{1.133673in}{0.768008in}}%
\pgfpathlineto{\pgfqpoint{1.138182in}{0.776479in}}%
\pgfpathlineto{\pgfqpoint{1.142691in}{0.768008in}}%
\pgfpathlineto{\pgfqpoint{1.147200in}{0.800835in}}%
\pgfpathlineto{\pgfqpoint{1.151709in}{0.778597in}}%
\pgfpathlineto{\pgfqpoint{1.156218in}{0.789186in}}%
\pgfpathlineto{\pgfqpoint{1.160727in}{0.775420in}}%
\pgfpathlineto{\pgfqpoint{1.165236in}{0.784951in}}%
\pgfpathlineto{\pgfqpoint{1.169745in}{0.783892in}}%
\pgfpathlineto{\pgfqpoint{1.174255in}{0.791304in}}%
\pgfpathlineto{\pgfqpoint{1.178764in}{0.780715in}}%
\pgfpathlineto{\pgfqpoint{1.187782in}{0.810365in}}%
\pgfpathlineto{\pgfqpoint{1.192291in}{0.804011in}}%
\pgfpathlineto{\pgfqpoint{1.196800in}{0.820954in}}%
\pgfpathlineto{\pgfqpoint{1.201309in}{0.807188in}}%
\pgfpathlineto{\pgfqpoint{1.205818in}{0.816719in}}%
\pgfpathlineto{\pgfqpoint{1.210327in}{0.829426in}}%
\pgfpathlineto{\pgfqpoint{1.214836in}{0.826249in}}%
\pgfpathlineto{\pgfqpoint{1.219345in}{0.831544in}}%
\pgfpathlineto{\pgfqpoint{1.223855in}{0.823072in}}%
\pgfpathlineto{\pgfqpoint{1.228364in}{0.826249in}}%
\pgfpathlineto{\pgfqpoint{1.232873in}{0.826249in}}%
\pgfpathlineto{\pgfqpoint{1.241891in}{0.841074in}}%
\pgfpathlineto{\pgfqpoint{1.246400in}{0.838956in}}%
\pgfpathlineto{\pgfqpoint{1.250909in}{0.841074in}}%
\pgfpathlineto{\pgfqpoint{1.255418in}{0.832603in}}%
\pgfpathlineto{\pgfqpoint{1.259927in}{0.840015in}}%
\pgfpathlineto{\pgfqpoint{1.264436in}{0.851663in}}%
\pgfpathlineto{\pgfqpoint{1.268945in}{0.850604in}}%
\pgfpathlineto{\pgfqpoint{1.273455in}{0.855899in}}%
\pgfpathlineto{\pgfqpoint{1.277964in}{0.867547in}}%
\pgfpathlineto{\pgfqpoint{1.282473in}{0.850604in}}%
\pgfpathlineto{\pgfqpoint{1.286982in}{0.865430in}}%
\pgfpathlineto{\pgfqpoint{1.291491in}{0.861194in}}%
\pgfpathlineto{\pgfqpoint{1.296000in}{0.862253in}}%
\pgfpathlineto{\pgfqpoint{1.300509in}{0.867547in}}%
\pgfpathlineto{\pgfqpoint{1.305018in}{0.868606in}}%
\pgfpathlineto{\pgfqpoint{1.309527in}{0.886608in}}%
\pgfpathlineto{\pgfqpoint{1.314036in}{0.892962in}}%
\pgfpathlineto{\pgfqpoint{1.318545in}{0.907787in}}%
\pgfpathlineto{\pgfqpoint{1.323055in}{0.894021in}}%
\pgfpathlineto{\pgfqpoint{1.327564in}{0.888726in}}%
\pgfpathlineto{\pgfqpoint{1.332073in}{0.915199in}}%
\pgfpathlineto{\pgfqpoint{1.336582in}{0.907787in}}%
\pgfpathlineto{\pgfqpoint{1.341091in}{0.890844in}}%
\pgfpathlineto{\pgfqpoint{1.345600in}{0.915199in}}%
\pgfpathlineto{\pgfqpoint{1.350109in}{0.914141in}}%
\pgfpathlineto{\pgfqpoint{1.354618in}{0.904610in}}%
\pgfpathlineto{\pgfqpoint{1.359127in}{0.912023in}}%
\pgfpathlineto{\pgfqpoint{1.363636in}{0.913082in}}%
\pgfpathlineto{\pgfqpoint{1.368145in}{0.920494in}}%
\pgfpathlineto{\pgfqpoint{1.372655in}{0.912023in}}%
\pgfpathlineto{\pgfqpoint{1.377164in}{0.910964in}}%
\pgfpathlineto{\pgfqpoint{1.381673in}{0.927907in}}%
\pgfpathlineto{\pgfqpoint{1.386182in}{0.927907in}}%
\pgfpathlineto{\pgfqpoint{1.395200in}{0.933201in}}%
\pgfpathlineto{\pgfqpoint{1.399709in}{0.950144in}}%
\pgfpathlineto{\pgfqpoint{1.404218in}{0.936378in}}%
\pgfpathlineto{\pgfqpoint{1.408727in}{0.959675in}}%
\pgfpathlineto{\pgfqpoint{1.413236in}{0.966028in}}%
\pgfpathlineto{\pgfqpoint{1.417745in}{0.940614in}}%
\pgfpathlineto{\pgfqpoint{1.422255in}{0.959675in}}%
\pgfpathlineto{\pgfqpoint{1.426764in}{0.944850in}}%
\pgfpathlineto{\pgfqpoint{1.431273in}{0.957557in}}%
\pgfpathlineto{\pgfqpoint{1.435782in}{0.958616in}}%
\pgfpathlineto{\pgfqpoint{1.440291in}{0.949085in}}%
\pgfpathlineto{\pgfqpoint{1.444800in}{0.981912in}}%
\pgfpathlineto{\pgfqpoint{1.449309in}{0.986148in}}%
\pgfpathlineto{\pgfqpoint{1.453818in}{0.977677in}}%
\pgfpathlineto{\pgfqpoint{1.458327in}{0.990384in}}%
\pgfpathlineto{\pgfqpoint{1.467345in}{0.955439in}}%
\pgfpathlineto{\pgfqpoint{1.476364in}{0.999914in}}%
\pgfpathlineto{\pgfqpoint{1.480873in}{1.010504in}}%
\pgfpathlineto{\pgfqpoint{1.485382in}{1.004150in}}%
\pgfpathlineto{\pgfqpoint{1.489891in}{1.000973in}}%
\pgfpathlineto{\pgfqpoint{1.494400in}{1.014739in}}%
\pgfpathlineto{\pgfqpoint{1.498909in}{0.999914in}}%
\pgfpathlineto{\pgfqpoint{1.503418in}{1.023211in}}%
\pgfpathlineto{\pgfqpoint{1.507927in}{1.011563in}}%
\pgfpathlineto{\pgfqpoint{1.512436in}{1.029564in}}%
\pgfpathlineto{\pgfqpoint{1.521455in}{1.029564in}}%
\pgfpathlineto{\pgfqpoint{1.525964in}{1.041213in}}%
\pgfpathlineto{\pgfqpoint{1.530473in}{1.048625in}}%
\pgfpathlineto{\pgfqpoint{1.534982in}{1.049684in}}%
\pgfpathlineto{\pgfqpoint{1.539491in}{1.042272in}}%
\pgfpathlineto{\pgfqpoint{1.544000in}{1.031682in}}%
\pgfpathlineto{\pgfqpoint{1.548509in}{1.036977in}}%
\pgfpathlineto{\pgfqpoint{1.553018in}{1.038036in}}%
\pgfpathlineto{\pgfqpoint{1.557527in}{1.051802in}}%
\pgfpathlineto{\pgfqpoint{1.562036in}{1.034859in}}%
\pgfpathlineto{\pgfqpoint{1.566545in}{1.033800in}}%
\pgfpathlineto{\pgfqpoint{1.571055in}{1.054979in}}%
\pgfpathlineto{\pgfqpoint{1.575564in}{1.066627in}}%
\pgfpathlineto{\pgfqpoint{1.580073in}{1.060274in}}%
\pgfpathlineto{\pgfqpoint{1.584582in}{1.076158in}}%
\pgfpathlineto{\pgfqpoint{1.589091in}{1.068745in}}%
\pgfpathlineto{\pgfqpoint{1.593600in}{1.079334in}}%
\pgfpathlineto{\pgfqpoint{1.598109in}{1.076158in}}%
\pgfpathlineto{\pgfqpoint{1.602618in}{1.079334in}}%
\pgfpathlineto{\pgfqpoint{1.607127in}{1.065568in}}%
\pgfpathlineto{\pgfqpoint{1.611636in}{1.084629in}}%
\pgfpathlineto{\pgfqpoint{1.616145in}{1.083570in}}%
\pgfpathlineto{\pgfqpoint{1.620655in}{1.064509in}}%
\pgfpathlineto{\pgfqpoint{1.625164in}{1.079334in}}%
\pgfpathlineto{\pgfqpoint{1.629673in}{1.101572in}}%
\pgfpathlineto{\pgfqpoint{1.634182in}{1.106867in}}%
\pgfpathlineto{\pgfqpoint{1.638691in}{1.085688in}}%
\pgfpathlineto{\pgfqpoint{1.643200in}{1.092042in}}%
\pgfpathlineto{\pgfqpoint{1.647709in}{1.122751in}}%
\pgfpathlineto{\pgfqpoint{1.652218in}{1.111102in}}%
\pgfpathlineto{\pgfqpoint{1.661236in}{1.137576in}}%
\pgfpathlineto{\pgfqpoint{1.665745in}{1.148165in}}%
\pgfpathlineto{\pgfqpoint{1.670255in}{1.124869in}}%
\pgfpathlineto{\pgfqpoint{1.674764in}{1.122751in}}%
\pgfpathlineto{\pgfqpoint{1.679273in}{1.129104in}}%
\pgfpathlineto{\pgfqpoint{1.683782in}{1.114279in}}%
\pgfpathlineto{\pgfqpoint{1.688291in}{1.117456in}}%
\pgfpathlineto{\pgfqpoint{1.692800in}{1.126986in}}%
\pgfpathlineto{\pgfqpoint{1.697309in}{1.142870in}}%
\pgfpathlineto{\pgfqpoint{1.701818in}{1.168285in}}%
\pgfpathlineto{\pgfqpoint{1.706327in}{1.129104in}}%
\pgfpathlineto{\pgfqpoint{1.715345in}{1.154519in}}%
\pgfpathlineto{\pgfqpoint{1.719855in}{1.149224in}}%
\pgfpathlineto{\pgfqpoint{1.724364in}{1.184169in}}%
\pgfpathlineto{\pgfqpoint{1.728873in}{1.152401in}}%
\pgfpathlineto{\pgfqpoint{1.737891in}{1.188405in}}%
\pgfpathlineto{\pgfqpoint{1.742400in}{1.187346in}}%
\pgfpathlineto{\pgfqpoint{1.746909in}{1.178874in}}%
\pgfpathlineto{\pgfqpoint{1.751418in}{1.191581in}}%
\pgfpathlineto{\pgfqpoint{1.755927in}{1.165108in}}%
\pgfpathlineto{\pgfqpoint{1.760436in}{1.196876in}}%
\pgfpathlineto{\pgfqpoint{1.764945in}{1.196876in}}%
\pgfpathlineto{\pgfqpoint{1.769455in}{1.180992in}}%
\pgfpathlineto{\pgfqpoint{1.773964in}{1.200053in}}%
\pgfpathlineto{\pgfqpoint{1.778473in}{1.186287in}}%
\pgfpathlineto{\pgfqpoint{1.782982in}{1.200053in}}%
\pgfpathlineto{\pgfqpoint{1.787491in}{1.205348in}}%
\pgfpathlineto{\pgfqpoint{1.792000in}{1.192640in}}%
\pgfpathlineto{\pgfqpoint{1.796509in}{1.218055in}}%
\pgfpathlineto{\pgfqpoint{1.801018in}{1.218055in}}%
\pgfpathlineto{\pgfqpoint{1.805527in}{1.227585in}}%
\pgfpathlineto{\pgfqpoint{1.810036in}{1.231821in}}%
\pgfpathlineto{\pgfqpoint{1.814545in}{1.239234in}}%
\pgfpathlineto{\pgfqpoint{1.819055in}{1.237116in}}%
\pgfpathlineto{\pgfqpoint{1.823564in}{1.239234in}}%
\pgfpathlineto{\pgfqpoint{1.828073in}{1.245587in}}%
\pgfpathlineto{\pgfqpoint{1.832582in}{1.237116in}}%
\pgfpathlineto{\pgfqpoint{1.837091in}{1.261471in}}%
\pgfpathlineto{\pgfqpoint{1.841600in}{1.245587in}}%
\pgfpathlineto{\pgfqpoint{1.850618in}{1.250882in}}%
\pgfpathlineto{\pgfqpoint{1.855127in}{1.254059in}}%
\pgfpathlineto{\pgfqpoint{1.859636in}{1.248764in}}%
\pgfpathlineto{\pgfqpoint{1.864145in}{1.263589in}}%
\pgfpathlineto{\pgfqpoint{1.868655in}{1.251941in}}%
\pgfpathlineto{\pgfqpoint{1.873164in}{1.284768in}}%
\pgfpathlineto{\pgfqpoint{1.877673in}{1.296416in}}%
\pgfpathlineto{\pgfqpoint{1.882182in}{1.267825in}}%
\pgfpathlineto{\pgfqpoint{1.886691in}{1.263589in}}%
\pgfpathlineto{\pgfqpoint{1.891200in}{1.286886in}}%
\pgfpathlineto{\pgfqpoint{1.895709in}{1.282650in}}%
\pgfpathlineto{\pgfqpoint{1.900218in}{1.281591in}}%
\pgfpathlineto{\pgfqpoint{1.904727in}{1.273119in}}%
\pgfpathlineto{\pgfqpoint{1.909236in}{1.289003in}}%
\pgfpathlineto{\pgfqpoint{1.913745in}{1.300652in}}%
\pgfpathlineto{\pgfqpoint{1.918255in}{1.308064in}}%
\pgfpathlineto{\pgfqpoint{1.922764in}{1.293239in}}%
\pgfpathlineto{\pgfqpoint{1.927273in}{1.315477in}}%
\pgfpathlineto{\pgfqpoint{1.931782in}{1.321830in}}%
\pgfpathlineto{\pgfqpoint{1.936291in}{1.337714in}}%
\pgfpathlineto{\pgfqpoint{1.940800in}{1.318654in}}%
\pgfpathlineto{\pgfqpoint{1.945309in}{1.305946in}}%
\pgfpathlineto{\pgfqpoint{1.949818in}{1.326066in}}%
\pgfpathlineto{\pgfqpoint{1.954327in}{1.320772in}}%
\pgfpathlineto{\pgfqpoint{1.958836in}{1.333479in}}%
\pgfpathlineto{\pgfqpoint{1.963345in}{1.328184in}}%
\pgfpathlineto{\pgfqpoint{1.967855in}{1.326066in}}%
\pgfpathlineto{\pgfqpoint{1.972364in}{1.346186in}}%
\pgfpathlineto{\pgfqpoint{1.976873in}{1.353598in}}%
\pgfpathlineto{\pgfqpoint{1.981382in}{1.318654in}}%
\pgfpathlineto{\pgfqpoint{1.985891in}{1.350422in}}%
\pgfpathlineto{\pgfqpoint{1.994909in}{1.355716in}}%
\pgfpathlineto{\pgfqpoint{1.999418in}{1.344068in}}%
\pgfpathlineto{\pgfqpoint{2.003927in}{1.345127in}}%
\pgfpathlineto{\pgfqpoint{2.008436in}{1.357834in}}%
\pgfpathlineto{\pgfqpoint{2.012945in}{1.348304in}}%
\pgfpathlineto{\pgfqpoint{2.017455in}{1.346186in}}%
\pgfpathlineto{\pgfqpoint{2.021964in}{1.379013in}}%
\pgfpathlineto{\pgfqpoint{2.026473in}{1.379013in}}%
\pgfpathlineto{\pgfqpoint{2.030982in}{1.356775in}}%
\pgfpathlineto{\pgfqpoint{2.035491in}{1.400192in}}%
\pgfpathlineto{\pgfqpoint{2.040000in}{1.374777in}}%
\pgfpathlineto{\pgfqpoint{2.044509in}{1.415017in}}%
\pgfpathlineto{\pgfqpoint{2.049018in}{1.372659in}}%
\pgfpathlineto{\pgfqpoint{2.058036in}{1.393838in}}%
\pgfpathlineto{\pgfqpoint{2.062545in}{1.367365in}}%
\pgfpathlineto{\pgfqpoint{2.067055in}{1.400192in}}%
\pgfpathlineto{\pgfqpoint{2.071564in}{1.393838in}}%
\pgfpathlineto{\pgfqpoint{2.076073in}{1.408663in}}%
\pgfpathlineto{\pgfqpoint{2.080582in}{1.404427in}}%
\pgfpathlineto{\pgfqpoint{2.085091in}{1.404427in}}%
\pgfpathlineto{\pgfqpoint{2.089600in}{1.429842in}}%
\pgfpathlineto{\pgfqpoint{2.094109in}{1.440431in}}%
\pgfpathlineto{\pgfqpoint{2.098618in}{1.418194in}}%
\pgfpathlineto{\pgfqpoint{2.103127in}{1.412899in}}%
\pgfpathlineto{\pgfqpoint{2.107636in}{1.444667in}}%
\pgfpathlineto{\pgfqpoint{2.112145in}{1.424547in}}%
\pgfpathlineto{\pgfqpoint{2.116655in}{1.445726in}}%
\pgfpathlineto{\pgfqpoint{2.121164in}{1.451020in}}%
\pgfpathlineto{\pgfqpoint{2.125673in}{1.439372in}}%
\pgfpathlineto{\pgfqpoint{2.130182in}{1.437254in}}%
\pgfpathlineto{\pgfqpoint{2.134691in}{1.445726in}}%
\pgfpathlineto{\pgfqpoint{2.139200in}{1.428783in}}%
\pgfpathlineto{\pgfqpoint{2.148218in}{1.480671in}}%
\pgfpathlineto{\pgfqpoint{2.152727in}{1.453138in}}%
\pgfpathlineto{\pgfqpoint{2.157236in}{1.479612in}}%
\pgfpathlineto{\pgfqpoint{2.161745in}{1.452079in}}%
\pgfpathlineto{\pgfqpoint{2.166255in}{1.449962in}}%
\pgfpathlineto{\pgfqpoint{2.170764in}{1.487024in}}%
\pgfpathlineto{\pgfqpoint{2.175273in}{1.487024in}}%
\pgfpathlineto{\pgfqpoint{2.179782in}{1.473258in}}%
\pgfpathlineto{\pgfqpoint{2.184291in}{1.487024in}}%
\pgfpathlineto{\pgfqpoint{2.188800in}{1.482789in}}%
\pgfpathlineto{\pgfqpoint{2.193309in}{1.507144in}}%
\pgfpathlineto{\pgfqpoint{2.197818in}{1.488083in}}%
\pgfpathlineto{\pgfqpoint{2.206836in}{1.503967in}}%
\pgfpathlineto{\pgfqpoint{2.211345in}{1.491260in}}%
\pgfpathlineto{\pgfqpoint{2.215855in}{1.509262in}}%
\pgfpathlineto{\pgfqpoint{2.220364in}{1.482789in}}%
\pgfpathlineto{\pgfqpoint{2.224873in}{1.539971in}}%
\pgfpathlineto{\pgfqpoint{2.229382in}{1.497614in}}%
\pgfpathlineto{\pgfqpoint{2.233891in}{1.506085in}}%
\pgfpathlineto{\pgfqpoint{2.238400in}{1.492319in}}%
\pgfpathlineto{\pgfqpoint{2.242909in}{1.513498in}}%
\pgfpathlineto{\pgfqpoint{2.247418in}{1.562209in}}%
\pgfpathlineto{\pgfqpoint{2.251927in}{1.538912in}}%
\pgfpathlineto{\pgfqpoint{2.256436in}{1.550560in}}%
\pgfpathlineto{\pgfqpoint{2.260945in}{1.541030in}}%
\pgfpathlineto{\pgfqpoint{2.265455in}{1.553737in}}%
\pgfpathlineto{\pgfqpoint{2.269964in}{1.539971in}}%
\pgfpathlineto{\pgfqpoint{2.274473in}{1.543148in}}%
\pgfpathlineto{\pgfqpoint{2.283491in}{1.582328in}}%
\pgfpathlineto{\pgfqpoint{2.288000in}{1.582328in}}%
\pgfpathlineto{\pgfqpoint{2.292509in}{1.547384in}}%
\pgfpathlineto{\pgfqpoint{2.297018in}{1.561150in}}%
\pgfpathlineto{\pgfqpoint{2.301527in}{1.562209in}}%
\pgfpathlineto{\pgfqpoint{2.306036in}{1.595036in}}%
\pgfpathlineto{\pgfqpoint{2.310545in}{1.570680in}}%
\pgfpathlineto{\pgfqpoint{2.315055in}{1.584446in}}%
\pgfpathlineto{\pgfqpoint{2.319564in}{1.583387in}}%
\pgfpathlineto{\pgfqpoint{2.324073in}{1.571739in}}%
\pgfpathlineto{\pgfqpoint{2.328582in}{1.577034in}}%
\pgfpathlineto{\pgfqpoint{2.333091in}{1.600330in}}%
\pgfpathlineto{\pgfqpoint{2.342109in}{1.619391in}}%
\pgfpathlineto{\pgfqpoint{2.346618in}{1.607743in}}%
\pgfpathlineto{\pgfqpoint{2.351127in}{1.607743in}}%
\pgfpathlineto{\pgfqpoint{2.355636in}{1.569621in}}%
\pgfpathlineto{\pgfqpoint{2.360145in}{1.607743in}}%
\pgfpathlineto{\pgfqpoint{2.369164in}{1.608802in}}%
\pgfpathlineto{\pgfqpoint{2.373673in}{1.650100in}}%
\pgfpathlineto{\pgfqpoint{2.378182in}{1.629980in}}%
\pgfpathlineto{\pgfqpoint{2.382691in}{1.654336in}}%
\pgfpathlineto{\pgfqpoint{2.387200in}{1.643747in}}%
\pgfpathlineto{\pgfqpoint{2.391709in}{1.639511in}}%
\pgfpathlineto{\pgfqpoint{2.396218in}{1.617273in}}%
\pgfpathlineto{\pgfqpoint{2.400727in}{1.674456in}}%
\pgfpathlineto{\pgfqpoint{2.405236in}{1.636334in}}%
\pgfpathlineto{\pgfqpoint{2.409745in}{1.673397in}}%
\pgfpathlineto{\pgfqpoint{2.414255in}{1.634216in}}%
\pgfpathlineto{\pgfqpoint{2.418764in}{1.638452in}}%
\pgfpathlineto{\pgfqpoint{2.423273in}{1.676574in}}%
\pgfpathlineto{\pgfqpoint{2.427782in}{1.636334in}}%
\pgfpathlineto{\pgfqpoint{2.432291in}{1.686104in}}%
\pgfpathlineto{\pgfqpoint{2.441309in}{1.637393in}}%
\pgfpathlineto{\pgfqpoint{2.445818in}{1.662807in}}%
\pgfpathlineto{\pgfqpoint{2.450327in}{1.678691in}}%
\pgfpathlineto{\pgfqpoint{2.454836in}{1.667043in}}%
\pgfpathlineto{\pgfqpoint{2.459345in}{1.679750in}}%
\pgfpathlineto{\pgfqpoint{2.463855in}{1.681868in}}%
\pgfpathlineto{\pgfqpoint{2.468364in}{1.688222in}}%
\pgfpathlineto{\pgfqpoint{2.472873in}{1.674456in}}%
\pgfpathlineto{\pgfqpoint{2.481891in}{1.700929in}}%
\pgfpathlineto{\pgfqpoint{2.490909in}{1.704106in}}%
\pgfpathlineto{\pgfqpoint{2.495418in}{1.724226in}}%
\pgfpathlineto{\pgfqpoint{2.499927in}{1.718931in}}%
\pgfpathlineto{\pgfqpoint{2.504436in}{1.700929in}}%
\pgfpathlineto{\pgfqpoint{2.513455in}{1.749640in}}%
\pgfpathlineto{\pgfqpoint{2.517964in}{1.742228in}}%
\pgfpathlineto{\pgfqpoint{2.522473in}{1.719990in}}%
\pgfpathlineto{\pgfqpoint{2.526982in}{1.736933in}}%
\pgfpathlineto{\pgfqpoint{2.531491in}{1.730579in}}%
\pgfpathlineto{\pgfqpoint{2.536000in}{1.754935in}}%
\pgfpathlineto{\pgfqpoint{2.540509in}{1.760229in}}%
\pgfpathlineto{\pgfqpoint{2.545018in}{1.744345in}}%
\pgfpathlineto{\pgfqpoint{2.549527in}{1.755994in}}%
\pgfpathlineto{\pgfqpoint{2.554036in}{1.784585in}}%
\pgfpathlineto{\pgfqpoint{2.558545in}{1.741169in}}%
\pgfpathlineto{\pgfqpoint{2.563055in}{1.786703in}}%
\pgfpathlineto{\pgfqpoint{2.567564in}{1.734815in}}%
\pgfpathlineto{\pgfqpoint{2.572073in}{1.758112in}}%
\pgfpathlineto{\pgfqpoint{2.576582in}{1.767642in}}%
\pgfpathlineto{\pgfqpoint{2.581091in}{1.772937in}}%
\pgfpathlineto{\pgfqpoint{2.585600in}{1.773996in}}%
\pgfpathlineto{\pgfqpoint{2.590109in}{1.744345in}}%
\pgfpathlineto{\pgfqpoint{2.594618in}{1.804705in}}%
\pgfpathlineto{\pgfqpoint{2.599127in}{1.771878in}}%
\pgfpathlineto{\pgfqpoint{2.603636in}{1.794115in}}%
\pgfpathlineto{\pgfqpoint{2.608145in}{1.822707in}}%
\pgfpathlineto{\pgfqpoint{2.612655in}{1.784585in}}%
\pgfpathlineto{\pgfqpoint{2.617164in}{1.791997in}}%
\pgfpathlineto{\pgfqpoint{2.621673in}{1.766583in}}%
\pgfpathlineto{\pgfqpoint{2.626182in}{1.872477in}}%
\pgfpathlineto{\pgfqpoint{2.630691in}{1.823766in}}%
\pgfpathlineto{\pgfqpoint{2.635200in}{1.814235in}}%
\pgfpathlineto{\pgfqpoint{2.639709in}{1.830119in}}%
\pgfpathlineto{\pgfqpoint{2.644218in}{1.811058in}}%
\pgfpathlineto{\pgfqpoint{2.648727in}{1.842826in}}%
\pgfpathlineto{\pgfqpoint{2.653236in}{1.826942in}}%
\pgfpathlineto{\pgfqpoint{2.657745in}{1.822707in}}%
\pgfpathlineto{\pgfqpoint{2.662255in}{1.811058in}}%
\pgfpathlineto{\pgfqpoint{2.666764in}{1.807882in}}%
\pgfpathlineto{\pgfqpoint{2.671273in}{1.832237in}}%
\pgfpathlineto{\pgfqpoint{2.675782in}{1.824824in}}%
\pgfpathlineto{\pgfqpoint{2.680291in}{1.851298in}}%
\pgfpathlineto{\pgfqpoint{2.689309in}{1.878830in}}%
\pgfpathlineto{\pgfqpoint{2.693818in}{1.840708in}}%
\pgfpathlineto{\pgfqpoint{2.698327in}{1.897891in}}%
\pgfpathlineto{\pgfqpoint{2.702836in}{1.880948in}}%
\pgfpathlineto{\pgfqpoint{2.707345in}{1.848121in}}%
\pgfpathlineto{\pgfqpoint{2.711855in}{1.872477in}}%
\pgfpathlineto{\pgfqpoint{2.716364in}{1.873535in}}%
\pgfpathlineto{\pgfqpoint{2.720873in}{1.862946in}}%
\pgfpathlineto{\pgfqpoint{2.725382in}{1.870359in}}%
\pgfpathlineto{\pgfqpoint{2.729891in}{1.880948in}}%
\pgfpathlineto{\pgfqpoint{2.734400in}{1.885184in}}%
\pgfpathlineto{\pgfqpoint{2.738909in}{1.912716in}}%
\pgfpathlineto{\pgfqpoint{2.743418in}{1.887302in}}%
\pgfpathlineto{\pgfqpoint{2.747927in}{1.880948in}}%
\pgfpathlineto{\pgfqpoint{2.752436in}{1.852357in}}%
\pgfpathlineto{\pgfqpoint{2.756945in}{1.891537in}}%
\pgfpathlineto{\pgfqpoint{2.761455in}{1.904245in}}%
\pgfpathlineto{\pgfqpoint{2.765964in}{1.884125in}}%
\pgfpathlineto{\pgfqpoint{2.770473in}{1.871418in}}%
\pgfpathlineto{\pgfqpoint{2.774982in}{1.889419in}}%
\pgfpathlineto{\pgfqpoint{2.779491in}{1.922246in}}%
\pgfpathlineto{\pgfqpoint{2.784000in}{1.909539in}}%
\pgfpathlineto{\pgfqpoint{2.788509in}{1.931777in}}%
\pgfpathlineto{\pgfqpoint{2.793018in}{1.905303in}}%
\pgfpathlineto{\pgfqpoint{2.797527in}{1.920129in}}%
\pgfpathlineto{\pgfqpoint{2.802036in}{1.920129in}}%
\pgfpathlineto{\pgfqpoint{2.806545in}{1.941307in}}%
\pgfpathlineto{\pgfqpoint{2.811055in}{1.949779in}}%
\pgfpathlineto{\pgfqpoint{2.815564in}{1.929659in}}%
\pgfpathlineto{\pgfqpoint{2.820073in}{1.964604in}}%
\pgfpathlineto{\pgfqpoint{2.824582in}{1.949779in}}%
\pgfpathlineto{\pgfqpoint{2.829091in}{1.891537in}}%
\pgfpathlineto{\pgfqpoint{2.833600in}{1.992136in}}%
\pgfpathlineto{\pgfqpoint{2.838109in}{1.984724in}}%
\pgfpathlineto{\pgfqpoint{2.842618in}{1.981547in}}%
\pgfpathlineto{\pgfqpoint{2.847127in}{1.951897in}}%
\pgfpathlineto{\pgfqpoint{2.851636in}{2.011197in}}%
\pgfpathlineto{\pgfqpoint{2.856145in}{2.006961in}}%
\pgfpathlineto{\pgfqpoint{2.860655in}{1.978370in}}%
\pgfpathlineto{\pgfqpoint{2.865164in}{1.982606in}}%
\pgfpathlineto{\pgfqpoint{2.869673in}{1.969899in}}%
\pgfpathlineto{\pgfqpoint{2.874182in}{1.996372in}}%
\pgfpathlineto{\pgfqpoint{2.878691in}{2.000608in}}%
\pgfpathlineto{\pgfqpoint{2.883200in}{1.996372in}}%
\pgfpathlineto{\pgfqpoint{2.887709in}{1.998490in}}%
\pgfpathlineto{\pgfqpoint{2.892218in}{1.996372in}}%
\pgfpathlineto{\pgfqpoint{2.896727in}{1.996372in}}%
\pgfpathlineto{\pgfqpoint{2.901236in}{2.005902in}}%
\pgfpathlineto{\pgfqpoint{2.905745in}{2.004843in}}%
\pgfpathlineto{\pgfqpoint{2.910255in}{2.033435in}}%
\pgfpathlineto{\pgfqpoint{2.914764in}{2.027081in}}%
\pgfpathlineto{\pgfqpoint{2.919273in}{2.008020in}}%
\pgfpathlineto{\pgfqpoint{2.923782in}{1.994254in}}%
\pgfpathlineto{\pgfqpoint{2.928291in}{2.026022in}}%
\pgfpathlineto{\pgfqpoint{2.932800in}{2.014374in}}%
\pgfpathlineto{\pgfqpoint{2.937309in}{1.985783in}}%
\pgfpathlineto{\pgfqpoint{2.941818in}{2.067321in}}%
\pgfpathlineto{\pgfqpoint{2.946327in}{2.073674in}}%
\pgfpathlineto{\pgfqpoint{2.950836in}{2.032376in}}%
\pgfpathlineto{\pgfqpoint{2.955345in}{2.053554in}}%
\pgfpathlineto{\pgfqpoint{2.959855in}{2.055672in}}%
\pgfpathlineto{\pgfqpoint{2.968873in}{2.024963in}}%
\pgfpathlineto{\pgfqpoint{2.973382in}{2.099089in}}%
\pgfpathlineto{\pgfqpoint{2.977891in}{2.098030in}}%
\pgfpathlineto{\pgfqpoint{2.982400in}{2.078969in}}%
\pgfpathlineto{\pgfqpoint{2.986909in}{2.040847in}}%
\pgfpathlineto{\pgfqpoint{2.991418in}{2.120267in}}%
\pgfpathlineto{\pgfqpoint{2.995927in}{2.099089in}}%
\pgfpathlineto{\pgfqpoint{3.000436in}{2.064144in}}%
\pgfpathlineto{\pgfqpoint{3.004945in}{2.063085in}}%
\pgfpathlineto{\pgfqpoint{3.009455in}{2.077910in}}%
\pgfpathlineto{\pgfqpoint{3.018473in}{2.112855in}}%
\pgfpathlineto{\pgfqpoint{3.022982in}{2.094853in}}%
\pgfpathlineto{\pgfqpoint{3.027491in}{2.125562in}}%
\pgfpathlineto{\pgfqpoint{3.032000in}{2.132974in}}%
\pgfpathlineto{\pgfqpoint{3.036509in}{2.111796in}}%
\pgfpathlineto{\pgfqpoint{3.041018in}{2.125562in}}%
\pgfpathlineto{\pgfqpoint{3.045527in}{2.155212in}}%
\pgfpathlineto{\pgfqpoint{3.050036in}{2.129798in}}%
\pgfpathlineto{\pgfqpoint{3.054545in}{2.121326in}}%
\pgfpathlineto{\pgfqpoint{3.059055in}{2.120267in}}%
\pgfpathlineto{\pgfqpoint{3.063564in}{2.144623in}}%
\pgfpathlineto{\pgfqpoint{3.068073in}{2.083205in}}%
\pgfpathlineto{\pgfqpoint{3.072582in}{2.160507in}}%
\pgfpathlineto{\pgfqpoint{3.077091in}{2.109678in}}%
\pgfpathlineto{\pgfqpoint{3.081600in}{2.114973in}}%
\pgfpathlineto{\pgfqpoint{3.090618in}{2.154153in}}%
\pgfpathlineto{\pgfqpoint{3.095127in}{2.148858in}}%
\pgfpathlineto{\pgfqpoint{3.099636in}{2.179568in}}%
\pgfpathlineto{\pgfqpoint{3.104145in}{2.152035in}}%
\pgfpathlineto{\pgfqpoint{3.108655in}{2.146741in}}%
\pgfpathlineto{\pgfqpoint{3.113164in}{2.173214in}}%
\pgfpathlineto{\pgfqpoint{3.117673in}{2.153094in}}%
\pgfpathlineto{\pgfqpoint{3.122182in}{2.196511in}}%
\pgfpathlineto{\pgfqpoint{3.126691in}{2.165801in}}%
\pgfpathlineto{\pgfqpoint{3.131200in}{2.159448in}}%
\pgfpathlineto{\pgfqpoint{3.135709in}{2.167919in}}%
\pgfpathlineto{\pgfqpoint{3.140218in}{2.228279in}}%
\pgfpathlineto{\pgfqpoint{3.144727in}{2.165801in}}%
\pgfpathlineto{\pgfqpoint{3.149236in}{2.208159in}}%
\pgfpathlineto{\pgfqpoint{3.153745in}{2.219807in}}%
\pgfpathlineto{\pgfqpoint{3.158255in}{2.132974in}}%
\pgfpathlineto{\pgfqpoint{3.162764in}{2.203923in}}%
\pgfpathlineto{\pgfqpoint{3.167273in}{2.202864in}}%
\pgfpathlineto{\pgfqpoint{3.171782in}{2.219807in}}%
\pgfpathlineto{\pgfqpoint{3.176291in}{2.208159in}}%
\pgfpathlineto{\pgfqpoint{3.180800in}{2.200746in}}%
\pgfpathlineto{\pgfqpoint{3.185309in}{2.235691in}}%
\pgfpathlineto{\pgfqpoint{3.189818in}{2.255811in}}%
\pgfpathlineto{\pgfqpoint{3.194327in}{2.235691in}}%
\pgfpathlineto{\pgfqpoint{3.198836in}{2.255811in}}%
\pgfpathlineto{\pgfqpoint{3.203345in}{2.221925in}}%
\pgfpathlineto{\pgfqpoint{3.207855in}{2.204982in}}%
\pgfpathlineto{\pgfqpoint{3.212364in}{2.230396in}}%
\pgfpathlineto{\pgfqpoint{3.216873in}{2.244163in}}%
\pgfpathlineto{\pgfqpoint{3.221382in}{2.240986in}}%
\pgfpathlineto{\pgfqpoint{3.225891in}{2.248398in}}%
\pgfpathlineto{\pgfqpoint{3.230400in}{2.250516in}}%
\pgfpathlineto{\pgfqpoint{3.234909in}{2.191216in}}%
\pgfpathlineto{\pgfqpoint{3.239418in}{2.254752in}}%
\pgfpathlineto{\pgfqpoint{3.243927in}{2.266400in}}%
\pgfpathlineto{\pgfqpoint{3.248436in}{2.214512in}}%
\pgfpathlineto{\pgfqpoint{3.252945in}{2.269577in}}%
\pgfpathlineto{\pgfqpoint{3.257455in}{2.284402in}}%
\pgfpathlineto{\pgfqpoint{3.261964in}{2.294991in}}%
\pgfpathlineto{\pgfqpoint{3.266473in}{2.230396in}}%
\pgfpathlineto{\pgfqpoint{3.270982in}{2.301345in}}%
\pgfpathlineto{\pgfqpoint{3.275491in}{2.264282in}}%
\pgfpathlineto{\pgfqpoint{3.280000in}{2.293933in}}%
\pgfpathlineto{\pgfqpoint{3.284509in}{2.262165in}}%
\pgfpathlineto{\pgfqpoint{3.289018in}{2.312993in}}%
\pgfpathlineto{\pgfqpoint{3.293527in}{2.278049in}}%
\pgfpathlineto{\pgfqpoint{3.298036in}{2.325701in}}%
\pgfpathlineto{\pgfqpoint{3.302545in}{2.299227in}}%
\pgfpathlineto{\pgfqpoint{3.307055in}{2.332054in}}%
\pgfpathlineto{\pgfqpoint{3.311564in}{2.324642in}}%
\pgfpathlineto{\pgfqpoint{3.316073in}{2.308758in}}%
\pgfpathlineto{\pgfqpoint{3.320582in}{2.355351in}}%
\pgfpathlineto{\pgfqpoint{3.325091in}{2.345820in}}%
\pgfpathlineto{\pgfqpoint{3.329600in}{2.298168in}}%
\pgfpathlineto{\pgfqpoint{3.334109in}{2.366999in}}%
\pgfpathlineto{\pgfqpoint{3.338618in}{2.320406in}}%
\pgfpathlineto{\pgfqpoint{3.343127in}{2.300286in}}%
\pgfpathlineto{\pgfqpoint{3.347636in}{2.354292in}}%
\pgfpathlineto{\pgfqpoint{3.352145in}{2.379706in}}%
\pgfpathlineto{\pgfqpoint{3.356655in}{2.387119in}}%
\pgfpathlineto{\pgfqpoint{3.361164in}{2.374412in}}%
\pgfpathlineto{\pgfqpoint{3.365673in}{2.370176in}}%
\pgfpathlineto{\pgfqpoint{3.374691in}{2.336290in}}%
\pgfpathlineto{\pgfqpoint{3.383709in}{2.392413in}}%
\pgfpathlineto{\pgfqpoint{3.388218in}{2.344761in}}%
\pgfpathlineto{\pgfqpoint{3.392727in}{2.387119in}}%
\pgfpathlineto{\pgfqpoint{3.397236in}{2.394531in}}%
\pgfpathlineto{\pgfqpoint{3.406255in}{2.441124in}}%
\pgfpathlineto{\pgfqpoint{3.410764in}{2.399826in}}%
\pgfpathlineto{\pgfqpoint{3.415273in}{2.330995in}}%
\pgfpathlineto{\pgfqpoint{3.419782in}{2.372294in}}%
\pgfpathlineto{\pgfqpoint{3.424291in}{2.424182in}}%
\pgfpathlineto{\pgfqpoint{3.428800in}{2.399826in}}%
\pgfpathlineto{\pgfqpoint{3.433309in}{2.391355in}}%
\pgfpathlineto{\pgfqpoint{3.437818in}{2.453832in}}%
\pgfpathlineto{\pgfqpoint{3.442327in}{2.411474in}}%
\pgfpathlineto{\pgfqpoint{3.446836in}{2.423123in}}%
\pgfpathlineto{\pgfqpoint{3.455855in}{2.386060in}}%
\pgfpathlineto{\pgfqpoint{3.460364in}{2.426299in}}%
\pgfpathlineto{\pgfqpoint{3.464873in}{2.397708in}}%
\pgfpathlineto{\pgfqpoint{3.469382in}{2.446419in}}%
\pgfpathlineto{\pgfqpoint{3.473891in}{2.453832in}}%
\pgfpathlineto{\pgfqpoint{3.478400in}{2.418887in}}%
\pgfpathlineto{\pgfqpoint{3.482909in}{2.414651in}}%
\pgfpathlineto{\pgfqpoint{3.487418in}{2.414651in}}%
\pgfpathlineto{\pgfqpoint{3.491927in}{2.484541in}}%
\pgfpathlineto{\pgfqpoint{3.496436in}{2.473951in}}%
\pgfpathlineto{\pgfqpoint{3.500945in}{2.459126in}}%
\pgfpathlineto{\pgfqpoint{3.505455in}{2.465480in}}%
\pgfpathlineto{\pgfqpoint{3.509964in}{2.445360in}}%
\pgfpathlineto{\pgfqpoint{3.514473in}{2.464421in}}%
\pgfpathlineto{\pgfqpoint{3.518982in}{2.469716in}}%
\pgfpathlineto{\pgfqpoint{3.523491in}{2.527957in}}%
\pgfpathlineto{\pgfqpoint{3.528000in}{2.498307in}}%
\pgfpathlineto{\pgfqpoint{3.532509in}{2.485600in}}%
\pgfpathlineto{\pgfqpoint{3.537018in}{2.489835in}}%
\pgfpathlineto{\pgfqpoint{3.541527in}{2.480305in}}%
\pgfpathlineto{\pgfqpoint{3.546036in}{2.512073in}}%
\pgfpathlineto{\pgfqpoint{3.550545in}{2.525839in}}%
\pgfpathlineto{\pgfqpoint{3.555055in}{2.476069in}}%
\pgfpathlineto{\pgfqpoint{3.559564in}{2.526898in}}%
\pgfpathlineto{\pgfqpoint{3.564073in}{2.486659in}}%
\pgfpathlineto{\pgfqpoint{3.568582in}{2.525839in}}%
\pgfpathlineto{\pgfqpoint{3.573091in}{2.497248in}}%
\pgfpathlineto{\pgfqpoint{3.577600in}{2.503602in}}%
\pgfpathlineto{\pgfqpoint{3.582109in}{2.501484in}}%
\pgfpathlineto{\pgfqpoint{3.586618in}{2.488777in}}%
\pgfpathlineto{\pgfqpoint{3.591127in}{2.496189in}}%
\pgfpathlineto{\pgfqpoint{3.595636in}{2.519486in}}%
\pgfpathlineto{\pgfqpoint{3.600145in}{2.556548in}}%
\pgfpathlineto{\pgfqpoint{3.604655in}{2.496189in}}%
\pgfpathlineto{\pgfqpoint{3.609164in}{2.521604in}}%
\pgfpathlineto{\pgfqpoint{3.613673in}{2.520545in}}%
\pgfpathlineto{\pgfqpoint{3.618182in}{2.527957in}}%
\pgfpathlineto{\pgfqpoint{3.622691in}{2.558666in}}%
\pgfpathlineto{\pgfqpoint{3.627200in}{2.568197in}}%
\pgfpathlineto{\pgfqpoint{3.631709in}{2.555489in}}%
\pgfpathlineto{\pgfqpoint{3.636218in}{2.584081in}}%
\pgfpathlineto{\pgfqpoint{3.640727in}{2.567138in}}%
\pgfpathlineto{\pgfqpoint{3.645236in}{2.566079in}}%
\pgfpathlineto{\pgfqpoint{3.649745in}{2.557607in}}%
\pgfpathlineto{\pgfqpoint{3.654255in}{2.536429in}}%
\pgfpathlineto{\pgfqpoint{3.658764in}{2.615849in}}%
\pgfpathlineto{\pgfqpoint{3.663273in}{2.596788in}}%
\pgfpathlineto{\pgfqpoint{3.667782in}{2.609495in}}%
\pgfpathlineto{\pgfqpoint{3.672291in}{2.588316in}}%
\pgfpathlineto{\pgfqpoint{3.676800in}{2.606318in}}%
\pgfpathlineto{\pgfqpoint{3.681309in}{2.611613in}}%
\pgfpathlineto{\pgfqpoint{3.685818in}{2.620084in}}%
\pgfpathlineto{\pgfqpoint{3.690327in}{2.592552in}}%
\pgfpathlineto{\pgfqpoint{3.694836in}{2.599965in}}%
\pgfpathlineto{\pgfqpoint{3.699345in}{2.579845in}}%
\pgfpathlineto{\pgfqpoint{3.703855in}{2.607377in}}%
\pgfpathlineto{\pgfqpoint{3.708364in}{2.642322in}}%
\pgfpathlineto{\pgfqpoint{3.717382in}{2.657147in}}%
\pgfpathlineto{\pgfqpoint{3.721891in}{2.651853in}}%
\pgfpathlineto{\pgfqpoint{3.726400in}{2.653970in}}%
\pgfpathlineto{\pgfqpoint{3.730909in}{2.702681in}}%
\pgfpathlineto{\pgfqpoint{3.735418in}{2.635968in}}%
\pgfpathlineto{\pgfqpoint{3.739927in}{2.640204in}}%
\pgfpathlineto{\pgfqpoint{3.744436in}{2.668795in}}%
\pgfpathlineto{\pgfqpoint{3.748945in}{2.597847in}}%
\pgfpathlineto{\pgfqpoint{3.753455in}{2.678326in}}%
\pgfpathlineto{\pgfqpoint{3.757964in}{2.650794in}}%
\pgfpathlineto{\pgfqpoint{3.762473in}{2.659265in}}%
\pgfpathlineto{\pgfqpoint{3.766982in}{2.691033in}}%
\pgfpathlineto{\pgfqpoint{3.771491in}{2.694210in}}%
\pgfpathlineto{\pgfqpoint{3.776000in}{2.729155in}}%
\pgfpathlineto{\pgfqpoint{3.785018in}{2.693151in}}%
\pgfpathlineto{\pgfqpoint{3.789527in}{2.669854in}}%
\pgfpathlineto{\pgfqpoint{3.794036in}{2.748216in}}%
\pgfpathlineto{\pgfqpoint{3.798545in}{2.712212in}}%
\pgfpathlineto{\pgfqpoint{3.803055in}{2.715389in}}%
\pgfpathlineto{\pgfqpoint{3.807564in}{2.664560in}}%
\pgfpathlineto{\pgfqpoint{3.816582in}{2.691033in}}%
\pgfpathlineto{\pgfqpoint{3.821091in}{2.712212in}}%
\pgfpathlineto{\pgfqpoint{3.825600in}{2.692092in}}%
\pgfpathlineto{\pgfqpoint{3.830109in}{2.725978in}}%
\pgfpathlineto{\pgfqpoint{3.834618in}{2.732332in}}%
\pgfpathlineto{\pgfqpoint{3.839127in}{2.728096in}}%
\pgfpathlineto{\pgfqpoint{3.843636in}{2.694210in}}%
\pgfpathlineto{\pgfqpoint{3.848145in}{2.700564in}}%
\pgfpathlineto{\pgfqpoint{3.852655in}{2.755628in}}%
\pgfpathlineto{\pgfqpoint{3.857164in}{2.754569in}}%
\pgfpathlineto{\pgfqpoint{3.861673in}{2.728096in}}%
\pgfpathlineto{\pgfqpoint{3.866182in}{2.753510in}}%
\pgfpathlineto{\pgfqpoint{3.870691in}{2.722801in}}%
\pgfpathlineto{\pgfqpoint{3.875200in}{2.765159in}}%
\pgfpathlineto{\pgfqpoint{3.879709in}{2.759864in}}%
\pgfpathlineto{\pgfqpoint{3.884218in}{2.776807in}}%
\pgfpathlineto{\pgfqpoint{3.888727in}{2.761982in}}%
\pgfpathlineto{\pgfqpoint{3.893236in}{2.796927in}}%
\pgfpathlineto{\pgfqpoint{3.897745in}{2.793750in}}%
\pgfpathlineto{\pgfqpoint{3.902255in}{2.737626in}}%
\pgfpathlineto{\pgfqpoint{3.906764in}{2.783160in}}%
\pgfpathlineto{\pgfqpoint{3.911273in}{2.785278in}}%
\pgfpathlineto{\pgfqpoint{3.920291in}{2.845638in}}%
\pgfpathlineto{\pgfqpoint{3.924800in}{2.804339in}}%
\pgfpathlineto{\pgfqpoint{3.929309in}{2.781043in}}%
\pgfpathlineto{\pgfqpoint{3.933818in}{2.776807in}}%
\pgfpathlineto{\pgfqpoint{3.938327in}{2.838225in}}%
\pgfpathlineto{\pgfqpoint{3.942836in}{2.867875in}}%
\pgfpathlineto{\pgfqpoint{3.947345in}{2.779984in}}%
\pgfpathlineto{\pgfqpoint{3.951855in}{2.754569in}}%
\pgfpathlineto{\pgfqpoint{3.960873in}{2.787396in}}%
\pgfpathlineto{\pgfqpoint{3.965382in}{2.846697in}}%
\pgfpathlineto{\pgfqpoint{3.969891in}{2.771512in}}%
\pgfpathlineto{\pgfqpoint{3.974400in}{2.815987in}}%
\pgfpathlineto{\pgfqpoint{3.978909in}{2.831871in}}%
\pgfpathlineto{\pgfqpoint{3.983418in}{2.802221in}}%
\pgfpathlineto{\pgfqpoint{3.987927in}{2.811752in}}%
\pgfpathlineto{\pgfqpoint{3.996945in}{2.863639in}}%
\pgfpathlineto{\pgfqpoint{4.001455in}{2.854109in}}%
\pgfpathlineto{\pgfqpoint{4.005964in}{2.841402in}}%
\pgfpathlineto{\pgfqpoint{4.010473in}{2.856227in}}%
\pgfpathlineto{\pgfqpoint{4.014982in}{2.830812in}}%
\pgfpathlineto{\pgfqpoint{4.019491in}{2.832930in}}%
\pgfpathlineto{\pgfqpoint{4.024000in}{2.872111in}}%
\pgfpathlineto{\pgfqpoint{4.028509in}{2.873170in}}%
\pgfpathlineto{\pgfqpoint{4.033018in}{2.872111in}}%
\pgfpathlineto{\pgfqpoint{4.037527in}{2.855168in}}%
\pgfpathlineto{\pgfqpoint{4.042036in}{2.893290in}}%
\pgfpathlineto{\pgfqpoint{4.046545in}{2.857286in}}%
\pgfpathlineto{\pgfqpoint{4.051055in}{2.878465in}}%
\pgfpathlineto{\pgfqpoint{4.055564in}{2.845638in}}%
\pgfpathlineto{\pgfqpoint{4.060073in}{2.905997in}}%
\pgfpathlineto{\pgfqpoint{4.064582in}{2.849873in}}%
\pgfpathlineto{\pgfqpoint{4.069091in}{2.882700in}}%
\pgfpathlineto{\pgfqpoint{4.073600in}{2.871052in}}%
\pgfpathlineto{\pgfqpoint{4.078109in}{2.912350in}}%
\pgfpathlineto{\pgfqpoint{4.082618in}{2.879523in}}%
\pgfpathlineto{\pgfqpoint{4.087127in}{2.909174in}}%
\pgfpathlineto{\pgfqpoint{4.091636in}{2.886936in}}%
\pgfpathlineto{\pgfqpoint{4.096145in}{2.956826in}}%
\pgfpathlineto{\pgfqpoint{4.100655in}{2.919763in}}%
\pgfpathlineto{\pgfqpoint{4.105164in}{2.896466in}}%
\pgfpathlineto{\pgfqpoint{4.109673in}{2.891172in}}%
\pgfpathlineto{\pgfqpoint{4.114182in}{2.958944in}}%
\pgfpathlineto{\pgfqpoint{4.118691in}{2.973769in}}%
\pgfpathlineto{\pgfqpoint{4.123200in}{2.892231in}}%
\pgfpathlineto{\pgfqpoint{4.127709in}{2.974828in}}%
\pgfpathlineto{\pgfqpoint{4.132218in}{2.964238in}}%
\pgfpathlineto{\pgfqpoint{4.136727in}{2.931411in}}%
\pgfpathlineto{\pgfqpoint{4.141236in}{2.984358in}}%
\pgfpathlineto{\pgfqpoint{4.145745in}{2.940942in}}%
\pgfpathlineto{\pgfqpoint{4.150255in}{2.942001in}}%
\pgfpathlineto{\pgfqpoint{4.154764in}{2.962120in}}%
\pgfpathlineto{\pgfqpoint{4.159273in}{3.001301in}}%
\pgfpathlineto{\pgfqpoint{4.163782in}{2.979063in}}%
\pgfpathlineto{\pgfqpoint{4.168291in}{2.985417in}}%
\pgfpathlineto{\pgfqpoint{4.172800in}{2.928234in}}%
\pgfpathlineto{\pgfqpoint{4.177309in}{2.944118in}}%
\pgfpathlineto{\pgfqpoint{4.181818in}{2.978004in}}%
\pgfpathlineto{\pgfqpoint{4.186327in}{2.986476in}}%
\pgfpathlineto{\pgfqpoint{4.190836in}{2.944118in}}%
\pgfpathlineto{\pgfqpoint{4.195345in}{3.009772in}}%
\pgfpathlineto{\pgfqpoint{4.199855in}{3.007655in}}%
\pgfpathlineto{\pgfqpoint{4.204364in}{2.963179in}}%
\pgfpathlineto{\pgfqpoint{4.213382in}{3.015067in}}%
\pgfpathlineto{\pgfqpoint{4.217891in}{2.994947in}}%
\pgfpathlineto{\pgfqpoint{4.222400in}{3.026715in}}%
\pgfpathlineto{\pgfqpoint{4.226909in}{3.019303in}}%
\pgfpathlineto{\pgfqpoint{4.231418in}{2.991771in}}%
\pgfpathlineto{\pgfqpoint{4.235927in}{2.973769in}}%
\pgfpathlineto{\pgfqpoint{4.240436in}{3.045776in}}%
\pgfpathlineto{\pgfqpoint{4.244945in}{3.025656in}}%
\pgfpathlineto{\pgfqpoint{4.249455in}{3.034128in}}%
\pgfpathlineto{\pgfqpoint{4.253964in}{2.979063in}}%
\pgfpathlineto{\pgfqpoint{4.258473in}{3.073309in}}%
\pgfpathlineto{\pgfqpoint{4.262982in}{3.036246in}}%
\pgfpathlineto{\pgfqpoint{4.267491in}{3.110371in}}%
\pgfpathlineto{\pgfqpoint{4.272000in}{3.032010in}}%
\pgfpathlineto{\pgfqpoint{4.276509in}{3.004478in}}%
\pgfpathlineto{\pgfqpoint{4.281018in}{3.009772in}}%
\pgfpathlineto{\pgfqpoint{4.285527in}{3.102959in}}%
\pgfpathlineto{\pgfqpoint{4.290036in}{3.030951in}}%
\pgfpathlineto{\pgfqpoint{4.294545in}{3.079662in}}%
\pgfpathlineto{\pgfqpoint{4.299055in}{3.097664in}}%
\pgfpathlineto{\pgfqpoint{4.303564in}{3.061660in}}%
\pgfpathlineto{\pgfqpoint{4.312582in}{3.125196in}}%
\pgfpathlineto{\pgfqpoint{4.317091in}{3.074367in}}%
\pgfpathlineto{\pgfqpoint{4.321600in}{3.089193in}}%
\pgfpathlineto{\pgfqpoint{4.326109in}{3.026715in}}%
\pgfpathlineto{\pgfqpoint{4.330618in}{3.124137in}}%
\pgfpathlineto{\pgfqpoint{4.335127in}{3.042599in}}%
\pgfpathlineto{\pgfqpoint{4.339636in}{3.102959in}}%
\pgfpathlineto{\pgfqpoint{4.344145in}{3.082839in}}%
\pgfpathlineto{\pgfqpoint{4.348655in}{3.127314in}}%
\pgfpathlineto{\pgfqpoint{4.353164in}{3.136845in}}%
\pgfpathlineto{\pgfqpoint{4.357673in}{3.149552in}}%
\pgfpathlineto{\pgfqpoint{4.362182in}{3.090251in}}%
\pgfpathlineto{\pgfqpoint{4.371200in}{3.131550in}}%
\pgfpathlineto{\pgfqpoint{4.375709in}{3.105077in}}%
\pgfpathlineto{\pgfqpoint{4.384727in}{3.135786in}}%
\pgfpathlineto{\pgfqpoint{4.393745in}{3.164377in}}%
\pgfpathlineto{\pgfqpoint{4.398255in}{3.129432in}}%
\pgfpathlineto{\pgfqpoint{4.402764in}{3.117784in}}%
\pgfpathlineto{\pgfqpoint{4.407273in}{3.204616in}}%
\pgfpathlineto{\pgfqpoint{4.411782in}{3.168613in}}%
\pgfpathlineto{\pgfqpoint{4.416291in}{3.170731in}}%
\pgfpathlineto{\pgfqpoint{4.420800in}{3.176025in}}%
\pgfpathlineto{\pgfqpoint{4.425309in}{3.188732in}}%
\pgfpathlineto{\pgfqpoint{4.429818in}{3.206734in}}%
\pgfpathlineto{\pgfqpoint{4.434327in}{3.123078in}}%
\pgfpathlineto{\pgfqpoint{4.443345in}{3.155905in}}%
\pgfpathlineto{\pgfqpoint{4.447855in}{3.151670in}}%
\pgfpathlineto{\pgfqpoint{4.452364in}{3.210970in}}%
\pgfpathlineto{\pgfqpoint{4.456873in}{3.202499in}}%
\pgfpathlineto{\pgfqpoint{4.461382in}{3.201440in}}%
\pgfpathlineto{\pgfqpoint{4.465891in}{3.151670in}}%
\pgfpathlineto{\pgfqpoint{4.470400in}{3.236384in}}%
\pgfpathlineto{\pgfqpoint{4.474909in}{3.208852in}}%
\pgfpathlineto{\pgfqpoint{4.479418in}{3.205675in}}%
\pgfpathlineto{\pgfqpoint{4.483927in}{3.249092in}}%
\pgfpathlineto{\pgfqpoint{4.488436in}{3.216265in}}%
\pgfpathlineto{\pgfqpoint{4.492945in}{3.191909in}}%
\pgfpathlineto{\pgfqpoint{4.497455in}{3.268153in}}%
\pgfpathlineto{\pgfqpoint{4.501964in}{3.224736in}}%
\pgfpathlineto{\pgfqpoint{4.506473in}{3.238502in}}%
\pgfpathlineto{\pgfqpoint{4.510982in}{3.263917in}}%
\pgfpathlineto{\pgfqpoint{4.515491in}{3.195086in}}%
\pgfpathlineto{\pgfqpoint{4.520000in}{3.243797in}}%
\pgfpathlineto{\pgfqpoint{4.524509in}{3.240620in}}%
\pgfpathlineto{\pgfqpoint{4.533527in}{3.309451in}}%
\pgfpathlineto{\pgfqpoint{4.538036in}{3.233208in}}%
\pgfpathlineto{\pgfqpoint{4.542545in}{3.248033in}}%
\pgfpathlineto{\pgfqpoint{4.547055in}{3.323217in}}%
\pgfpathlineto{\pgfqpoint{4.551564in}{3.258622in}}%
\pgfpathlineto{\pgfqpoint{4.556073in}{3.270270in}}%
\pgfpathlineto{\pgfqpoint{4.560582in}{3.236384in}}%
\pgfpathlineto{\pgfqpoint{4.569600in}{3.308392in}}%
\pgfpathlineto{\pgfqpoint{4.574109in}{3.296744in}}%
\pgfpathlineto{\pgfqpoint{4.578618in}{3.305215in}}%
\pgfpathlineto{\pgfqpoint{4.583127in}{3.285095in}}%
\pgfpathlineto{\pgfqpoint{4.587636in}{3.311569in}}%
\pgfpathlineto{\pgfqpoint{4.592145in}{3.302038in}}%
\pgfpathlineto{\pgfqpoint{4.596655in}{3.285095in}}%
\pgfpathlineto{\pgfqpoint{4.601164in}{3.325335in}}%
\pgfpathlineto{\pgfqpoint{4.605673in}{3.278742in}}%
\pgfpathlineto{\pgfqpoint{4.610182in}{3.307333in}}%
\pgfpathlineto{\pgfqpoint{4.614691in}{3.267094in}}%
\pgfpathlineto{\pgfqpoint{4.619200in}{3.316864in}}%
\pgfpathlineto{\pgfqpoint{4.623709in}{3.295685in}}%
\pgfpathlineto{\pgfqpoint{4.628218in}{3.297803in}}%
\pgfpathlineto{\pgfqpoint{4.632727in}{3.334865in}}%
\pgfpathlineto{\pgfqpoint{4.641745in}{3.356044in}}%
\pgfpathlineto{\pgfqpoint{4.646255in}{3.401578in}}%
\pgfpathlineto{\pgfqpoint{4.650764in}{3.342278in}}%
\pgfpathlineto{\pgfqpoint{4.655273in}{3.317922in}}%
\pgfpathlineto{\pgfqpoint{4.659782in}{3.351808in}}%
\pgfpathlineto{\pgfqpoint{4.664291in}{3.333806in}}%
\pgfpathlineto{\pgfqpoint{4.668800in}{3.351808in}}%
\pgfpathlineto{\pgfqpoint{4.673309in}{3.308392in}}%
\pgfpathlineto{\pgfqpoint{4.677818in}{3.276624in}}%
\pgfpathlineto{\pgfqpoint{4.682327in}{3.435464in}}%
\pgfpathlineto{\pgfqpoint{4.686836in}{3.302038in}}%
\pgfpathlineto{\pgfqpoint{4.691345in}{3.402637in}}%
\pgfpathlineto{\pgfqpoint{4.695855in}{3.407932in}}%
\pgfpathlineto{\pgfqpoint{4.700364in}{3.372987in}}%
\pgfpathlineto{\pgfqpoint{4.704873in}{3.379341in}}%
\pgfpathlineto{\pgfqpoint{4.709382in}{3.327453in}}%
\pgfpathlineto{\pgfqpoint{4.713891in}{3.358162in}}%
\pgfpathlineto{\pgfqpoint{4.718400in}{3.400519in}}%
\pgfpathlineto{\pgfqpoint{4.722909in}{3.379341in}}%
\pgfpathlineto{\pgfqpoint{4.727418in}{3.418521in}}%
\pgfpathlineto{\pgfqpoint{4.731927in}{3.422757in}}%
\pgfpathlineto{\pgfqpoint{4.736436in}{3.431228in}}%
\pgfpathlineto{\pgfqpoint{4.740945in}{3.432287in}}%
\pgfpathlineto{\pgfqpoint{4.745455in}{3.397343in}}%
\pgfpathlineto{\pgfqpoint{4.749964in}{3.437582in}}%
\pgfpathlineto{\pgfqpoint{4.754473in}{3.432287in}}%
\pgfpathlineto{\pgfqpoint{4.758982in}{3.464055in}}%
\pgfpathlineto{\pgfqpoint{4.763491in}{3.443936in}}%
\pgfpathlineto{\pgfqpoint{4.768000in}{3.458761in}}%
\pgfpathlineto{\pgfqpoint{4.772509in}{3.437582in}}%
\pgfpathlineto{\pgfqpoint{4.777018in}{3.448171in}}%
\pgfpathlineto{\pgfqpoint{4.781527in}{3.539240in}}%
\pgfpathlineto{\pgfqpoint{4.786036in}{3.475704in}}%
\pgfpathlineto{\pgfqpoint{4.790545in}{3.432287in}}%
\pgfpathlineto{\pgfqpoint{4.799564in}{3.475704in}}%
\pgfpathlineto{\pgfqpoint{4.804073in}{3.468291in}}%
\pgfpathlineto{\pgfqpoint{4.808582in}{3.470409in}}%
\pgfpathlineto{\pgfqpoint{4.813091in}{3.425934in}}%
\pgfpathlineto{\pgfqpoint{4.817600in}{3.500059in}}%
\pgfpathlineto{\pgfqpoint{4.822109in}{3.452407in}}%
\pgfpathlineto{\pgfqpoint{4.826618in}{3.513825in}}%
\pgfpathlineto{\pgfqpoint{4.831127in}{3.540299in}}%
\pgfpathlineto{\pgfqpoint{4.835636in}{3.458761in}}%
\pgfpathlineto{\pgfqpoint{4.840145in}{3.518061in}}%
\pgfpathlineto{\pgfqpoint{4.844655in}{3.464055in}}%
\pgfpathlineto{\pgfqpoint{4.849164in}{3.519120in}}%
\pgfpathlineto{\pgfqpoint{4.853673in}{3.459820in}}%
\pgfpathlineto{\pgfqpoint{4.858182in}{3.545593in}}%
\pgfpathlineto{\pgfqpoint{4.862691in}{3.519120in}}%
\pgfpathlineto{\pgfqpoint{4.867200in}{3.550888in}}%
\pgfpathlineto{\pgfqpoint{4.871709in}{3.536063in}}%
\pgfpathlineto{\pgfqpoint{4.876218in}{3.601717in}}%
\pgfpathlineto{\pgfqpoint{4.880727in}{3.522297in}}%
\pgfpathlineto{\pgfqpoint{4.885236in}{3.513825in}}%
\pgfpathlineto{\pgfqpoint{4.889745in}{3.576303in}}%
\pgfpathlineto{\pgfqpoint{4.894255in}{3.483116in}}%
\pgfpathlineto{\pgfqpoint{4.898764in}{3.555124in}}%
\pgfpathlineto{\pgfqpoint{4.903273in}{3.567831in}}%
\pgfpathlineto{\pgfqpoint{4.907782in}{3.503236in}}%
\pgfpathlineto{\pgfqpoint{4.912291in}{3.543476in}}%
\pgfpathlineto{\pgfqpoint{4.916800in}{3.497941in}}%
\pgfpathlineto{\pgfqpoint{4.921309in}{3.608071in}}%
\pgfpathlineto{\pgfqpoint{4.925818in}{3.593246in}}%
\pgfpathlineto{\pgfqpoint{4.930327in}{3.527592in}}%
\pgfpathlineto{\pgfqpoint{4.939345in}{3.621837in}}%
\pgfpathlineto{\pgfqpoint{4.943855in}{3.601717in}}%
\pgfpathlineto{\pgfqpoint{4.948364in}{3.569949in}}%
\pgfpathlineto{\pgfqpoint{4.952873in}{3.622896in}}%
\pgfpathlineto{\pgfqpoint{4.957382in}{3.611247in}}%
\pgfpathlineto{\pgfqpoint{4.966400in}{3.537122in}}%
\pgfpathlineto{\pgfqpoint{4.970909in}{3.610188in}}%
\pgfpathlineto{\pgfqpoint{4.975418in}{3.617601in}}%
\pgfpathlineto{\pgfqpoint{4.979927in}{3.569949in}}%
\pgfpathlineto{\pgfqpoint{4.984436in}{3.648310in}}%
\pgfpathlineto{\pgfqpoint{4.988945in}{3.649369in}}%
\pgfpathlineto{\pgfqpoint{4.993455in}{3.690668in}}%
\pgfpathlineto{\pgfqpoint{4.997964in}{3.643015in}}%
\pgfpathlineto{\pgfqpoint{5.002473in}{3.636662in}}%
\pgfpathlineto{\pgfqpoint{5.006982in}{3.645133in}}%
\pgfpathlineto{\pgfqpoint{5.016000in}{3.641957in}}%
\pgfpathlineto{\pgfqpoint{5.020509in}{3.613365in}}%
\pgfpathlineto{\pgfqpoint{5.025018in}{3.657841in}}%
\pgfpathlineto{\pgfqpoint{5.029527in}{3.666312in}}%
\pgfpathlineto{\pgfqpoint{5.034036in}{3.725612in}}%
\pgfpathlineto{\pgfqpoint{5.038545in}{3.633485in}}%
\pgfpathlineto{\pgfqpoint{5.043055in}{3.672666in}}%
\pgfpathlineto{\pgfqpoint{5.047564in}{3.668430in}}%
\pgfpathlineto{\pgfqpoint{5.052073in}{3.645133in}}%
\pgfpathlineto{\pgfqpoint{5.056582in}{3.683255in}}%
\pgfpathlineto{\pgfqpoint{5.061091in}{3.634544in}}%
\pgfpathlineto{\pgfqpoint{5.065600in}{3.649369in}}%
\pgfpathlineto{\pgfqpoint{5.070109in}{3.684314in}}%
\pgfpathlineto{\pgfqpoint{5.074618in}{3.666312in}}%
\pgfpathlineto{\pgfqpoint{5.079127in}{3.694903in}}%
\pgfpathlineto{\pgfqpoint{5.083636in}{3.668430in}}%
\pgfpathlineto{\pgfqpoint{5.088145in}{3.701257in}}%
\pgfpathlineto{\pgfqpoint{5.092655in}{3.760557in}}%
\pgfpathlineto{\pgfqpoint{5.097164in}{3.710787in}}%
\pgfpathlineto{\pgfqpoint{5.101673in}{3.708669in}}%
\pgfpathlineto{\pgfqpoint{5.106182in}{3.690668in}}%
\pgfpathlineto{\pgfqpoint{5.110691in}{3.700198in}}%
\pgfpathlineto{\pgfqpoint{5.115200in}{3.655723in}}%
\pgfpathlineto{\pgfqpoint{5.119709in}{3.747850in}}%
\pgfpathlineto{\pgfqpoint{5.124218in}{3.757380in}}%
\pgfpathlineto{\pgfqpoint{5.128727in}{3.715023in}}%
\pgfpathlineto{\pgfqpoint{5.133236in}{3.736202in}}%
\pgfpathlineto{\pgfqpoint{5.137745in}{3.767970in}}%
\pgfpathlineto{\pgfqpoint{5.142255in}{3.729848in}}%
\pgfpathlineto{\pgfqpoint{5.146764in}{3.709728in}}%
\pgfpathlineto{\pgfqpoint{5.155782in}{3.760557in}}%
\pgfpathlineto{\pgfqpoint{5.160291in}{3.816681in}}%
\pgfpathlineto{\pgfqpoint{5.164800in}{3.723494in}}%
\pgfpathlineto{\pgfqpoint{5.169309in}{3.806091in}}%
\pgfpathlineto{\pgfqpoint{5.178327in}{3.723494in}}%
\pgfpathlineto{\pgfqpoint{5.182836in}{3.718200in}}%
\pgfpathlineto{\pgfqpoint{5.187345in}{3.803974in}}%
\pgfpathlineto{\pgfqpoint{5.191855in}{3.778559in}}%
\pgfpathlineto{\pgfqpoint{5.196364in}{3.727730in}}%
\pgfpathlineto{\pgfqpoint{5.200873in}{3.778559in}}%
\pgfpathlineto{\pgfqpoint{5.209891in}{3.761616in}}%
\pgfpathlineto{\pgfqpoint{5.214400in}{3.811386in}}%
\pgfpathlineto{\pgfqpoint{5.218909in}{3.818799in}}%
\pgfpathlineto{\pgfqpoint{5.227927in}{3.807150in}}%
\pgfpathlineto{\pgfqpoint{5.232436in}{3.784913in}}%
\pgfpathlineto{\pgfqpoint{5.236945in}{3.848449in}}%
\pgfpathlineto{\pgfqpoint{5.241455in}{3.776441in}}%
\pgfpathlineto{\pgfqpoint{5.250473in}{3.917280in}}%
\pgfpathlineto{\pgfqpoint{5.254982in}{3.877040in}}%
\pgfpathlineto{\pgfqpoint{5.259491in}{3.751027in}}%
\pgfpathlineto{\pgfqpoint{5.264000in}{3.826211in}}%
\pgfpathlineto{\pgfqpoint{5.268509in}{3.826211in}}%
\pgfpathlineto{\pgfqpoint{5.273018in}{3.882335in}}%
\pgfpathlineto{\pgfqpoint{5.277527in}{3.841036in}}%
\pgfpathlineto{\pgfqpoint{5.282036in}{3.828329in}}%
\pgfpathlineto{\pgfqpoint{5.286545in}{3.885512in}}%
\pgfpathlineto{\pgfqpoint{5.291055in}{3.838918in}}%
\pgfpathlineto{\pgfqpoint{5.295564in}{3.921515in}}%
\pgfpathlineto{\pgfqpoint{5.300073in}{3.856920in}}%
\pgfpathlineto{\pgfqpoint{5.304582in}{3.845272in}}%
\pgfpathlineto{\pgfqpoint{5.309091in}{3.893983in}}%
\pgfpathlineto{\pgfqpoint{5.318109in}{3.868569in}}%
\pgfpathlineto{\pgfqpoint{5.322618in}{3.826211in}}%
\pgfpathlineto{\pgfqpoint{5.331636in}{3.899278in}}%
\pgfpathlineto{\pgfqpoint{5.336145in}{3.972344in}}%
\pgfpathlineto{\pgfqpoint{5.340655in}{3.931046in}}%
\pgfpathlineto{\pgfqpoint{5.345164in}{3.878099in}}%
\pgfpathlineto{\pgfqpoint{5.349673in}{3.940576in}}%
\pgfpathlineto{\pgfqpoint{5.354182in}{3.897160in}}%
\pgfpathlineto{\pgfqpoint{5.358691in}{3.899278in}}%
\pgfpathlineto{\pgfqpoint{5.363200in}{3.952224in}}%
\pgfpathlineto{\pgfqpoint{5.367709in}{3.922574in}}%
\pgfpathlineto{\pgfqpoint{5.372218in}{3.940576in}}%
\pgfpathlineto{\pgfqpoint{5.376727in}{3.916221in}}%
\pgfpathlineto{\pgfqpoint{5.381236in}{3.860097in}}%
\pgfpathlineto{\pgfqpoint{5.385745in}{3.950107in}}%
\pgfpathlineto{\pgfqpoint{5.394764in}{3.915162in}}%
\pgfpathlineto{\pgfqpoint{5.399273in}{3.965991in}}%
\pgfpathlineto{\pgfqpoint{5.403782in}{3.925751in}}%
\pgfpathlineto{\pgfqpoint{5.408291in}{3.972344in}}%
\pgfpathlineto{\pgfqpoint{5.412800in}{3.982934in}}%
\pgfpathlineto{\pgfqpoint{5.417309in}{3.937399in}}%
\pgfpathlineto{\pgfqpoint{5.421818in}{3.932105in}}%
\pgfpathlineto{\pgfqpoint{5.426327in}{3.970226in}}%
\pgfpathlineto{\pgfqpoint{5.430836in}{4.028468in}}%
\pgfpathlineto{\pgfqpoint{5.435345in}{4.045411in}}%
\pgfpathlineto{\pgfqpoint{5.439855in}{3.960696in}}%
\pgfpathlineto{\pgfqpoint{5.444364in}{4.007289in}}%
\pgfpathlineto{\pgfqpoint{5.457891in}{3.919397in}}%
\pgfpathlineto{\pgfqpoint{5.462400in}{4.041175in}}%
\pgfpathlineto{\pgfqpoint{5.466909in}{4.004112in}}%
\pgfpathlineto{\pgfqpoint{5.471418in}{4.026350in}}%
\pgfpathlineto{\pgfqpoint{5.475927in}{3.965991in}}%
\pgfpathlineto{\pgfqpoint{5.480436in}{4.041175in}}%
\pgfpathlineto{\pgfqpoint{5.484945in}{4.000935in}}%
\pgfpathlineto{\pgfqpoint{5.489455in}{4.044352in}}%
\pgfpathlineto{\pgfqpoint{5.493964in}{4.025291in}}%
\pgfpathlineto{\pgfqpoint{5.498473in}{3.998818in}}%
\pgfpathlineto{\pgfqpoint{5.502982in}{3.995641in}}%
\pgfpathlineto{\pgfqpoint{5.507491in}{4.001994in}}%
\pgfpathlineto{\pgfqpoint{5.512000in}{4.056000in}}%
\pgfpathlineto{\pgfqpoint{5.516509in}{4.037998in}}%
\pgfpathlineto{\pgfqpoint{5.521018in}{4.033762in}}%
\pgfpathlineto{\pgfqpoint{5.525527in}{4.013643in}}%
\pgfpathlineto{\pgfqpoint{5.530036in}{4.031645in}}%
\pgfpathlineto{\pgfqpoint{5.534545in}{3.975521in}}%
\pgfpathlineto{\pgfqpoint{5.534545in}{3.975521in}}%
\pgfusepath{stroke}%
\end{pgfscope}%
\begin{pgfscope}%
\pgfsetrectcap%
\pgfsetmiterjoin%
\pgfsetlinewidth{0.803000pt}%
\definecolor{currentstroke}{rgb}{0.000000,0.000000,0.000000}%
\pgfsetstrokecolor{currentstroke}%
\pgfsetdash{}{0pt}%
\pgfpathmoveto{\pgfqpoint{0.800000in}{0.528000in}}%
\pgfpathlineto{\pgfqpoint{0.800000in}{4.224000in}}%
\pgfusepath{stroke}%
\end{pgfscope}%
\begin{pgfscope}%
\pgfsetrectcap%
\pgfsetmiterjoin%
\pgfsetlinewidth{0.803000pt}%
\definecolor{currentstroke}{rgb}{0.000000,0.000000,0.000000}%
\pgfsetstrokecolor{currentstroke}%
\pgfsetdash{}{0pt}%
\pgfpathmoveto{\pgfqpoint{5.760000in}{0.528000in}}%
\pgfpathlineto{\pgfqpoint{5.760000in}{4.224000in}}%
\pgfusepath{stroke}%
\end{pgfscope}%
\begin{pgfscope}%
\pgfsetrectcap%
\pgfsetmiterjoin%
\pgfsetlinewidth{0.803000pt}%
\definecolor{currentstroke}{rgb}{0.000000,0.000000,0.000000}%
\pgfsetstrokecolor{currentstroke}%
\pgfsetdash{}{0pt}%
\pgfpathmoveto{\pgfqpoint{0.800000in}{0.528000in}}%
\pgfpathlineto{\pgfqpoint{5.760000in}{0.528000in}}%
\pgfusepath{stroke}%
\end{pgfscope}%
\begin{pgfscope}%
\pgfsetrectcap%
\pgfsetmiterjoin%
\pgfsetlinewidth{0.803000pt}%
\definecolor{currentstroke}{rgb}{0.000000,0.000000,0.000000}%
\pgfsetstrokecolor{currentstroke}%
\pgfsetdash{}{0pt}%
\pgfpathmoveto{\pgfqpoint{0.800000in}{4.224000in}}%
\pgfpathlineto{\pgfqpoint{5.760000in}{4.224000in}}%
\pgfusepath{stroke}%
\end{pgfscope}%
\begin{pgfscope}%
\definecolor{textcolor}{rgb}{0.000000,0.000000,0.000000}%
\pgfsetstrokecolor{textcolor}%
\pgfsetfillcolor{textcolor}%
\pgftext[x=3.280000in,y=4.307333in,,base]{\color{textcolor}\ttfamily\fontsize{12.000000}{14.400000}\selectfont Quick Sort  Swaps vs Input size}%
\end{pgfscope}%
\begin{pgfscope}%
\pgfsetbuttcap%
\pgfsetmiterjoin%
\definecolor{currentfill}{rgb}{1.000000,1.000000,1.000000}%
\pgfsetfillcolor{currentfill}%
\pgfsetfillopacity{0.800000}%
\pgfsetlinewidth{1.003750pt}%
\definecolor{currentstroke}{rgb}{0.800000,0.800000,0.800000}%
\pgfsetstrokecolor{currentstroke}%
\pgfsetstrokeopacity{0.800000}%
\pgfsetdash{}{0pt}%
\pgfpathmoveto{\pgfqpoint{0.897222in}{3.908286in}}%
\pgfpathlineto{\pgfqpoint{1.759758in}{3.908286in}}%
\pgfpathquadraticcurveto{\pgfqpoint{1.787535in}{3.908286in}}{\pgfqpoint{1.787535in}{3.936063in}}%
\pgfpathlineto{\pgfqpoint{1.787535in}{4.126778in}}%
\pgfpathquadraticcurveto{\pgfqpoint{1.787535in}{4.154556in}}{\pgfqpoint{1.759758in}{4.154556in}}%
\pgfpathlineto{\pgfqpoint{0.897222in}{4.154556in}}%
\pgfpathquadraticcurveto{\pgfqpoint{0.869444in}{4.154556in}}{\pgfqpoint{0.869444in}{4.126778in}}%
\pgfpathlineto{\pgfqpoint{0.869444in}{3.936063in}}%
\pgfpathquadraticcurveto{\pgfqpoint{0.869444in}{3.908286in}}{\pgfqpoint{0.897222in}{3.908286in}}%
\pgfpathlineto{\pgfqpoint{0.897222in}{3.908286in}}%
\pgfpathclose%
\pgfusepath{stroke,fill}%
\end{pgfscope}%
\begin{pgfscope}%
\pgfsetrectcap%
\pgfsetroundjoin%
\pgfsetlinewidth{1.505625pt}%
\definecolor{currentstroke}{rgb}{0.000000,1.000000,0.498039}%
\pgfsetstrokecolor{currentstroke}%
\pgfsetdash{}{0pt}%
\pgfpathmoveto{\pgfqpoint{0.925000in}{4.041342in}}%
\pgfpathlineto{\pgfqpoint{1.063889in}{4.041342in}}%
\pgfpathlineto{\pgfqpoint{1.202778in}{4.041342in}}%
\pgfusepath{stroke}%
\end{pgfscope}%
\begin{pgfscope}%
\definecolor{textcolor}{rgb}{0.000000,0.000000,0.000000}%
\pgfsetstrokecolor{textcolor}%
\pgfsetfillcolor{textcolor}%
\pgftext[x=1.313889in,y=3.992731in,left,base]{\color{textcolor}\ttfamily\fontsize{10.000000}{12.000000}\selectfont Quick}%
\end{pgfscope}%
\end{pgfpicture}%
\makeatother%
\endgroup%

%% Creator: Matplotlib, PGF backend
%%
%% To include the figure in your LaTeX document, write
%%   \input{<filename>.pgf}
%%
%% Make sure the required packages are loaded in your preamble
%%   \usepackage{pgf}
%%
%% Also ensure that all the required font packages are loaded; for instance,
%% the lmodern package is sometimes necessary when using math font.
%%   \usepackage{lmodern}
%%
%% Figures using additional raster images can only be included by \input if
%% they are in the same directory as the main LaTeX file. For loading figures
%% from other directories you can use the `import` package
%%   \usepackage{import}
%%
%% and then include the figures with
%%   \import{<path to file>}{<filename>.pgf}
%%
%% Matplotlib used the following preamble
%%   \usepackage{fontspec}
%%   \setmainfont{DejaVuSerif.ttf}[Path=\detokenize{/home/dbk/.local/lib/python3.10/site-packages/matplotlib/mpl-data/fonts/ttf/}]
%%   \setsansfont{DejaVuSans.ttf}[Path=\detokenize{/home/dbk/.local/lib/python3.10/site-packages/matplotlib/mpl-data/fonts/ttf/}]
%%   \setmonofont{DejaVuSansMono.ttf}[Path=\detokenize{/home/dbk/.local/lib/python3.10/site-packages/matplotlib/mpl-data/fonts/ttf/}]
%%
\begingroup%
\makeatletter%
\begin{pgfpicture}%
\pgfpathrectangle{\pgfpointorigin}{\pgfqpoint{6.400000in}{4.800000in}}%
\pgfusepath{use as bounding box, clip}%
\begin{pgfscope}%
\pgfsetbuttcap%
\pgfsetmiterjoin%
\definecolor{currentfill}{rgb}{1.000000,1.000000,1.000000}%
\pgfsetfillcolor{currentfill}%
\pgfsetlinewidth{0.000000pt}%
\definecolor{currentstroke}{rgb}{1.000000,1.000000,1.000000}%
\pgfsetstrokecolor{currentstroke}%
\pgfsetdash{}{0pt}%
\pgfpathmoveto{\pgfqpoint{0.000000in}{0.000000in}}%
\pgfpathlineto{\pgfqpoint{6.400000in}{0.000000in}}%
\pgfpathlineto{\pgfqpoint{6.400000in}{4.800000in}}%
\pgfpathlineto{\pgfqpoint{0.000000in}{4.800000in}}%
\pgfpathlineto{\pgfqpoint{0.000000in}{0.000000in}}%
\pgfpathclose%
\pgfusepath{fill}%
\end{pgfscope}%
\begin{pgfscope}%
\pgfsetbuttcap%
\pgfsetmiterjoin%
\definecolor{currentfill}{rgb}{1.000000,1.000000,1.000000}%
\pgfsetfillcolor{currentfill}%
\pgfsetlinewidth{0.000000pt}%
\definecolor{currentstroke}{rgb}{0.000000,0.000000,0.000000}%
\pgfsetstrokecolor{currentstroke}%
\pgfsetstrokeopacity{0.000000}%
\pgfsetdash{}{0pt}%
\pgfpathmoveto{\pgfqpoint{0.800000in}{0.528000in}}%
\pgfpathlineto{\pgfqpoint{5.760000in}{0.528000in}}%
\pgfpathlineto{\pgfqpoint{5.760000in}{4.224000in}}%
\pgfpathlineto{\pgfqpoint{0.800000in}{4.224000in}}%
\pgfpathlineto{\pgfqpoint{0.800000in}{0.528000in}}%
\pgfpathclose%
\pgfusepath{fill}%
\end{pgfscope}%
\begin{pgfscope}%
\pgfsetbuttcap%
\pgfsetroundjoin%
\definecolor{currentfill}{rgb}{0.000000,0.000000,0.000000}%
\pgfsetfillcolor{currentfill}%
\pgfsetlinewidth{0.803000pt}%
\definecolor{currentstroke}{rgb}{0.000000,0.000000,0.000000}%
\pgfsetstrokecolor{currentstroke}%
\pgfsetdash{}{0pt}%
\pgfsys@defobject{currentmarker}{\pgfqpoint{0.000000in}{-0.048611in}}{\pgfqpoint{0.000000in}{0.000000in}}{%
\pgfpathmoveto{\pgfqpoint{0.000000in}{0.000000in}}%
\pgfpathlineto{\pgfqpoint{0.000000in}{-0.048611in}}%
\pgfusepath{stroke,fill}%
}%
\begin{pgfscope}%
\pgfsys@transformshift{1.020945in}{0.528000in}%
\pgfsys@useobject{currentmarker}{}%
\end{pgfscope}%
\end{pgfscope}%
\begin{pgfscope}%
\definecolor{textcolor}{rgb}{0.000000,0.000000,0.000000}%
\pgfsetstrokecolor{textcolor}%
\pgfsetfillcolor{textcolor}%
\pgftext[x=1.020945in,y=0.430778in,,top]{\color{textcolor}\ttfamily\fontsize{10.000000}{12.000000}\selectfont 0}%
\end{pgfscope}%
\begin{pgfscope}%
\pgfsetbuttcap%
\pgfsetroundjoin%
\definecolor{currentfill}{rgb}{0.000000,0.000000,0.000000}%
\pgfsetfillcolor{currentfill}%
\pgfsetlinewidth{0.803000pt}%
\definecolor{currentstroke}{rgb}{0.000000,0.000000,0.000000}%
\pgfsetstrokecolor{currentstroke}%
\pgfsetdash{}{0pt}%
\pgfsys@defobject{currentmarker}{\pgfqpoint{0.000000in}{-0.048611in}}{\pgfqpoint{0.000000in}{0.000000in}}{%
\pgfpathmoveto{\pgfqpoint{0.000000in}{0.000000in}}%
\pgfpathlineto{\pgfqpoint{0.000000in}{-0.048611in}}%
\pgfusepath{stroke,fill}%
}%
\begin{pgfscope}%
\pgfsys@transformshift{1.922764in}{0.528000in}%
\pgfsys@useobject{currentmarker}{}%
\end{pgfscope}%
\end{pgfscope}%
\begin{pgfscope}%
\definecolor{textcolor}{rgb}{0.000000,0.000000,0.000000}%
\pgfsetstrokecolor{textcolor}%
\pgfsetfillcolor{textcolor}%
\pgftext[x=1.922764in,y=0.430778in,,top]{\color{textcolor}\ttfamily\fontsize{10.000000}{12.000000}\selectfont 200}%
\end{pgfscope}%
\begin{pgfscope}%
\pgfsetbuttcap%
\pgfsetroundjoin%
\definecolor{currentfill}{rgb}{0.000000,0.000000,0.000000}%
\pgfsetfillcolor{currentfill}%
\pgfsetlinewidth{0.803000pt}%
\definecolor{currentstroke}{rgb}{0.000000,0.000000,0.000000}%
\pgfsetstrokecolor{currentstroke}%
\pgfsetdash{}{0pt}%
\pgfsys@defobject{currentmarker}{\pgfqpoint{0.000000in}{-0.048611in}}{\pgfqpoint{0.000000in}{0.000000in}}{%
\pgfpathmoveto{\pgfqpoint{0.000000in}{0.000000in}}%
\pgfpathlineto{\pgfqpoint{0.000000in}{-0.048611in}}%
\pgfusepath{stroke,fill}%
}%
\begin{pgfscope}%
\pgfsys@transformshift{2.824582in}{0.528000in}%
\pgfsys@useobject{currentmarker}{}%
\end{pgfscope}%
\end{pgfscope}%
\begin{pgfscope}%
\definecolor{textcolor}{rgb}{0.000000,0.000000,0.000000}%
\pgfsetstrokecolor{textcolor}%
\pgfsetfillcolor{textcolor}%
\pgftext[x=2.824582in,y=0.430778in,,top]{\color{textcolor}\ttfamily\fontsize{10.000000}{12.000000}\selectfont 400}%
\end{pgfscope}%
\begin{pgfscope}%
\pgfsetbuttcap%
\pgfsetroundjoin%
\definecolor{currentfill}{rgb}{0.000000,0.000000,0.000000}%
\pgfsetfillcolor{currentfill}%
\pgfsetlinewidth{0.803000pt}%
\definecolor{currentstroke}{rgb}{0.000000,0.000000,0.000000}%
\pgfsetstrokecolor{currentstroke}%
\pgfsetdash{}{0pt}%
\pgfsys@defobject{currentmarker}{\pgfqpoint{0.000000in}{-0.048611in}}{\pgfqpoint{0.000000in}{0.000000in}}{%
\pgfpathmoveto{\pgfqpoint{0.000000in}{0.000000in}}%
\pgfpathlineto{\pgfqpoint{0.000000in}{-0.048611in}}%
\pgfusepath{stroke,fill}%
}%
\begin{pgfscope}%
\pgfsys@transformshift{3.726400in}{0.528000in}%
\pgfsys@useobject{currentmarker}{}%
\end{pgfscope}%
\end{pgfscope}%
\begin{pgfscope}%
\definecolor{textcolor}{rgb}{0.000000,0.000000,0.000000}%
\pgfsetstrokecolor{textcolor}%
\pgfsetfillcolor{textcolor}%
\pgftext[x=3.726400in,y=0.430778in,,top]{\color{textcolor}\ttfamily\fontsize{10.000000}{12.000000}\selectfont 600}%
\end{pgfscope}%
\begin{pgfscope}%
\pgfsetbuttcap%
\pgfsetroundjoin%
\definecolor{currentfill}{rgb}{0.000000,0.000000,0.000000}%
\pgfsetfillcolor{currentfill}%
\pgfsetlinewidth{0.803000pt}%
\definecolor{currentstroke}{rgb}{0.000000,0.000000,0.000000}%
\pgfsetstrokecolor{currentstroke}%
\pgfsetdash{}{0pt}%
\pgfsys@defobject{currentmarker}{\pgfqpoint{0.000000in}{-0.048611in}}{\pgfqpoint{0.000000in}{0.000000in}}{%
\pgfpathmoveto{\pgfqpoint{0.000000in}{0.000000in}}%
\pgfpathlineto{\pgfqpoint{0.000000in}{-0.048611in}}%
\pgfusepath{stroke,fill}%
}%
\begin{pgfscope}%
\pgfsys@transformshift{4.628218in}{0.528000in}%
\pgfsys@useobject{currentmarker}{}%
\end{pgfscope}%
\end{pgfscope}%
\begin{pgfscope}%
\definecolor{textcolor}{rgb}{0.000000,0.000000,0.000000}%
\pgfsetstrokecolor{textcolor}%
\pgfsetfillcolor{textcolor}%
\pgftext[x=4.628218in,y=0.430778in,,top]{\color{textcolor}\ttfamily\fontsize{10.000000}{12.000000}\selectfont 800}%
\end{pgfscope}%
\begin{pgfscope}%
\pgfsetbuttcap%
\pgfsetroundjoin%
\definecolor{currentfill}{rgb}{0.000000,0.000000,0.000000}%
\pgfsetfillcolor{currentfill}%
\pgfsetlinewidth{0.803000pt}%
\definecolor{currentstroke}{rgb}{0.000000,0.000000,0.000000}%
\pgfsetstrokecolor{currentstroke}%
\pgfsetdash{}{0pt}%
\pgfsys@defobject{currentmarker}{\pgfqpoint{0.000000in}{-0.048611in}}{\pgfqpoint{0.000000in}{0.000000in}}{%
\pgfpathmoveto{\pgfqpoint{0.000000in}{0.000000in}}%
\pgfpathlineto{\pgfqpoint{0.000000in}{-0.048611in}}%
\pgfusepath{stroke,fill}%
}%
\begin{pgfscope}%
\pgfsys@transformshift{5.530036in}{0.528000in}%
\pgfsys@useobject{currentmarker}{}%
\end{pgfscope}%
\end{pgfscope}%
\begin{pgfscope}%
\definecolor{textcolor}{rgb}{0.000000,0.000000,0.000000}%
\pgfsetstrokecolor{textcolor}%
\pgfsetfillcolor{textcolor}%
\pgftext[x=5.530036in,y=0.430778in,,top]{\color{textcolor}\ttfamily\fontsize{10.000000}{12.000000}\selectfont 1000}%
\end{pgfscope}%
\begin{pgfscope}%
\definecolor{textcolor}{rgb}{0.000000,0.000000,0.000000}%
\pgfsetstrokecolor{textcolor}%
\pgfsetfillcolor{textcolor}%
\pgftext[x=3.280000in,y=0.240063in,,top]{\color{textcolor}\ttfamily\fontsize{10.000000}{12.000000}\selectfont Size of Array}%
\end{pgfscope}%
\begin{pgfscope}%
\pgfsetbuttcap%
\pgfsetroundjoin%
\definecolor{currentfill}{rgb}{0.000000,0.000000,0.000000}%
\pgfsetfillcolor{currentfill}%
\pgfsetlinewidth{0.803000pt}%
\definecolor{currentstroke}{rgb}{0.000000,0.000000,0.000000}%
\pgfsetstrokecolor{currentstroke}%
\pgfsetdash{}{0pt}%
\pgfsys@defobject{currentmarker}{\pgfqpoint{-0.048611in}{0.000000in}}{\pgfqpoint{-0.000000in}{0.000000in}}{%
\pgfpathmoveto{\pgfqpoint{-0.000000in}{0.000000in}}%
\pgfpathlineto{\pgfqpoint{-0.048611in}{0.000000in}}%
\pgfusepath{stroke,fill}%
}%
\begin{pgfscope}%
\pgfsys@transformshift{0.800000in}{0.870740in}%
\pgfsys@useobject{currentmarker}{}%
\end{pgfscope}%
\end{pgfscope}%
\begin{pgfscope}%
\definecolor{textcolor}{rgb}{0.000000,0.000000,0.000000}%
\pgfsetstrokecolor{textcolor}%
\pgfsetfillcolor{textcolor}%
\pgftext[x=0.451923in, y=0.817605in, left, base]{\color{textcolor}\ttfamily\fontsize{10.000000}{12.000000}\selectfont 200}%
\end{pgfscope}%
\begin{pgfscope}%
\pgfsetbuttcap%
\pgfsetroundjoin%
\definecolor{currentfill}{rgb}{0.000000,0.000000,0.000000}%
\pgfsetfillcolor{currentfill}%
\pgfsetlinewidth{0.803000pt}%
\definecolor{currentstroke}{rgb}{0.000000,0.000000,0.000000}%
\pgfsetstrokecolor{currentstroke}%
\pgfsetdash{}{0pt}%
\pgfsys@defobject{currentmarker}{\pgfqpoint{-0.048611in}{0.000000in}}{\pgfqpoint{-0.000000in}{0.000000in}}{%
\pgfpathmoveto{\pgfqpoint{-0.000000in}{0.000000in}}%
\pgfpathlineto{\pgfqpoint{-0.048611in}{0.000000in}}%
\pgfusepath{stroke,fill}%
}%
\begin{pgfscope}%
\pgfsys@transformshift{0.800000in}{1.369997in}%
\pgfsys@useobject{currentmarker}{}%
\end{pgfscope}%
\end{pgfscope}%
\begin{pgfscope}%
\definecolor{textcolor}{rgb}{0.000000,0.000000,0.000000}%
\pgfsetstrokecolor{textcolor}%
\pgfsetfillcolor{textcolor}%
\pgftext[x=0.451923in, y=1.316863in, left, base]{\color{textcolor}\ttfamily\fontsize{10.000000}{12.000000}\selectfont 400}%
\end{pgfscope}%
\begin{pgfscope}%
\pgfsetbuttcap%
\pgfsetroundjoin%
\definecolor{currentfill}{rgb}{0.000000,0.000000,0.000000}%
\pgfsetfillcolor{currentfill}%
\pgfsetlinewidth{0.803000pt}%
\definecolor{currentstroke}{rgb}{0.000000,0.000000,0.000000}%
\pgfsetstrokecolor{currentstroke}%
\pgfsetdash{}{0pt}%
\pgfsys@defobject{currentmarker}{\pgfqpoint{-0.048611in}{0.000000in}}{\pgfqpoint{-0.000000in}{0.000000in}}{%
\pgfpathmoveto{\pgfqpoint{-0.000000in}{0.000000in}}%
\pgfpathlineto{\pgfqpoint{-0.048611in}{0.000000in}}%
\pgfusepath{stroke,fill}%
}%
\begin{pgfscope}%
\pgfsys@transformshift{0.800000in}{1.869254in}%
\pgfsys@useobject{currentmarker}{}%
\end{pgfscope}%
\end{pgfscope}%
\begin{pgfscope}%
\definecolor{textcolor}{rgb}{0.000000,0.000000,0.000000}%
\pgfsetstrokecolor{textcolor}%
\pgfsetfillcolor{textcolor}%
\pgftext[x=0.451923in, y=1.816120in, left, base]{\color{textcolor}\ttfamily\fontsize{10.000000}{12.000000}\selectfont 600}%
\end{pgfscope}%
\begin{pgfscope}%
\pgfsetbuttcap%
\pgfsetroundjoin%
\definecolor{currentfill}{rgb}{0.000000,0.000000,0.000000}%
\pgfsetfillcolor{currentfill}%
\pgfsetlinewidth{0.803000pt}%
\definecolor{currentstroke}{rgb}{0.000000,0.000000,0.000000}%
\pgfsetstrokecolor{currentstroke}%
\pgfsetdash{}{0pt}%
\pgfsys@defobject{currentmarker}{\pgfqpoint{-0.048611in}{0.000000in}}{\pgfqpoint{-0.000000in}{0.000000in}}{%
\pgfpathmoveto{\pgfqpoint{-0.000000in}{0.000000in}}%
\pgfpathlineto{\pgfqpoint{-0.048611in}{0.000000in}}%
\pgfusepath{stroke,fill}%
}%
\begin{pgfscope}%
\pgfsys@transformshift{0.800000in}{2.368511in}%
\pgfsys@useobject{currentmarker}{}%
\end{pgfscope}%
\end{pgfscope}%
\begin{pgfscope}%
\definecolor{textcolor}{rgb}{0.000000,0.000000,0.000000}%
\pgfsetstrokecolor{textcolor}%
\pgfsetfillcolor{textcolor}%
\pgftext[x=0.451923in, y=2.315377in, left, base]{\color{textcolor}\ttfamily\fontsize{10.000000}{12.000000}\selectfont 800}%
\end{pgfscope}%
\begin{pgfscope}%
\pgfsetbuttcap%
\pgfsetroundjoin%
\definecolor{currentfill}{rgb}{0.000000,0.000000,0.000000}%
\pgfsetfillcolor{currentfill}%
\pgfsetlinewidth{0.803000pt}%
\definecolor{currentstroke}{rgb}{0.000000,0.000000,0.000000}%
\pgfsetstrokecolor{currentstroke}%
\pgfsetdash{}{0pt}%
\pgfsys@defobject{currentmarker}{\pgfqpoint{-0.048611in}{0.000000in}}{\pgfqpoint{-0.000000in}{0.000000in}}{%
\pgfpathmoveto{\pgfqpoint{-0.000000in}{0.000000in}}%
\pgfpathlineto{\pgfqpoint{-0.048611in}{0.000000in}}%
\pgfusepath{stroke,fill}%
}%
\begin{pgfscope}%
\pgfsys@transformshift{0.800000in}{2.867768in}%
\pgfsys@useobject{currentmarker}{}%
\end{pgfscope}%
\end{pgfscope}%
\begin{pgfscope}%
\definecolor{textcolor}{rgb}{0.000000,0.000000,0.000000}%
\pgfsetstrokecolor{textcolor}%
\pgfsetfillcolor{textcolor}%
\pgftext[x=0.368305in, y=2.814634in, left, base]{\color{textcolor}\ttfamily\fontsize{10.000000}{12.000000}\selectfont 1000}%
\end{pgfscope}%
\begin{pgfscope}%
\pgfsetbuttcap%
\pgfsetroundjoin%
\definecolor{currentfill}{rgb}{0.000000,0.000000,0.000000}%
\pgfsetfillcolor{currentfill}%
\pgfsetlinewidth{0.803000pt}%
\definecolor{currentstroke}{rgb}{0.000000,0.000000,0.000000}%
\pgfsetstrokecolor{currentstroke}%
\pgfsetdash{}{0pt}%
\pgfsys@defobject{currentmarker}{\pgfqpoint{-0.048611in}{0.000000in}}{\pgfqpoint{-0.000000in}{0.000000in}}{%
\pgfpathmoveto{\pgfqpoint{-0.000000in}{0.000000in}}%
\pgfpathlineto{\pgfqpoint{-0.048611in}{0.000000in}}%
\pgfusepath{stroke,fill}%
}%
\begin{pgfscope}%
\pgfsys@transformshift{0.800000in}{3.367025in}%
\pgfsys@useobject{currentmarker}{}%
\end{pgfscope}%
\end{pgfscope}%
\begin{pgfscope}%
\definecolor{textcolor}{rgb}{0.000000,0.000000,0.000000}%
\pgfsetstrokecolor{textcolor}%
\pgfsetfillcolor{textcolor}%
\pgftext[x=0.368305in, y=3.313891in, left, base]{\color{textcolor}\ttfamily\fontsize{10.000000}{12.000000}\selectfont 1200}%
\end{pgfscope}%
\begin{pgfscope}%
\pgfsetbuttcap%
\pgfsetroundjoin%
\definecolor{currentfill}{rgb}{0.000000,0.000000,0.000000}%
\pgfsetfillcolor{currentfill}%
\pgfsetlinewidth{0.803000pt}%
\definecolor{currentstroke}{rgb}{0.000000,0.000000,0.000000}%
\pgfsetstrokecolor{currentstroke}%
\pgfsetdash{}{0pt}%
\pgfsys@defobject{currentmarker}{\pgfqpoint{-0.048611in}{0.000000in}}{\pgfqpoint{-0.000000in}{0.000000in}}{%
\pgfpathmoveto{\pgfqpoint{-0.000000in}{0.000000in}}%
\pgfpathlineto{\pgfqpoint{-0.048611in}{0.000000in}}%
\pgfusepath{stroke,fill}%
}%
\begin{pgfscope}%
\pgfsys@transformshift{0.800000in}{3.866282in}%
\pgfsys@useobject{currentmarker}{}%
\end{pgfscope}%
\end{pgfscope}%
\begin{pgfscope}%
\definecolor{textcolor}{rgb}{0.000000,0.000000,0.000000}%
\pgfsetstrokecolor{textcolor}%
\pgfsetfillcolor{textcolor}%
\pgftext[x=0.368305in, y=3.813148in, left, base]{\color{textcolor}\ttfamily\fontsize{10.000000}{12.000000}\selectfont 1400}%
\end{pgfscope}%
\begin{pgfscope}%
\definecolor{textcolor}{rgb}{0.000000,0.000000,0.000000}%
\pgfsetstrokecolor{textcolor}%
\pgfsetfillcolor{textcolor}%
\pgftext[x=0.312750in,y=2.376000in,,bottom,rotate=90.000000]{\color{textcolor}\ttfamily\fontsize{10.000000}{12.000000}\selectfont Iterations}%
\end{pgfscope}%
\begin{pgfscope}%
\pgfpathrectangle{\pgfqpoint{0.800000in}{0.528000in}}{\pgfqpoint{4.960000in}{3.696000in}}%
\pgfusepath{clip}%
\pgfsetrectcap%
\pgfsetroundjoin%
\pgfsetlinewidth{1.505625pt}%
\definecolor{currentstroke}{rgb}{0.000000,1.000000,0.498039}%
\pgfsetstrokecolor{currentstroke}%
\pgfsetdash{}{0pt}%
\pgfpathmoveto{\pgfqpoint{1.025455in}{0.710978in}}%
\pgfpathlineto{\pgfqpoint{1.029964in}{0.700993in}}%
\pgfpathlineto{\pgfqpoint{1.034473in}{0.720963in}}%
\pgfpathlineto{\pgfqpoint{1.038982in}{0.705985in}}%
\pgfpathlineto{\pgfqpoint{1.048000in}{0.715970in}}%
\pgfpathlineto{\pgfqpoint{1.052509in}{0.725955in}}%
\pgfpathlineto{\pgfqpoint{1.057018in}{0.740933in}}%
\pgfpathlineto{\pgfqpoint{1.061527in}{0.696000in}}%
\pgfpathlineto{\pgfqpoint{1.066036in}{0.750918in}}%
\pgfpathlineto{\pgfqpoint{1.070545in}{0.725955in}}%
\pgfpathlineto{\pgfqpoint{1.075055in}{0.740933in}}%
\pgfpathlineto{\pgfqpoint{1.079564in}{0.745926in}}%
\pgfpathlineto{\pgfqpoint{1.084073in}{0.745926in}}%
\pgfpathlineto{\pgfqpoint{1.088582in}{0.735941in}}%
\pgfpathlineto{\pgfqpoint{1.093091in}{0.760903in}}%
\pgfpathlineto{\pgfqpoint{1.097600in}{0.760903in}}%
\pgfpathlineto{\pgfqpoint{1.102109in}{0.765896in}}%
\pgfpathlineto{\pgfqpoint{1.106618in}{0.760903in}}%
\pgfpathlineto{\pgfqpoint{1.111127in}{0.765896in}}%
\pgfpathlineto{\pgfqpoint{1.115636in}{0.760903in}}%
\pgfpathlineto{\pgfqpoint{1.120145in}{0.760903in}}%
\pgfpathlineto{\pgfqpoint{1.124655in}{0.775881in}}%
\pgfpathlineto{\pgfqpoint{1.129164in}{0.775881in}}%
\pgfpathlineto{\pgfqpoint{1.133673in}{0.795851in}}%
\pgfpathlineto{\pgfqpoint{1.138182in}{0.790859in}}%
\pgfpathlineto{\pgfqpoint{1.142691in}{0.790859in}}%
\pgfpathlineto{\pgfqpoint{1.147200in}{0.820814in}}%
\pgfpathlineto{\pgfqpoint{1.151709in}{0.780874in}}%
\pgfpathlineto{\pgfqpoint{1.156218in}{0.805837in}}%
\pgfpathlineto{\pgfqpoint{1.160727in}{0.805837in}}%
\pgfpathlineto{\pgfqpoint{1.169745in}{0.815822in}}%
\pgfpathlineto{\pgfqpoint{1.174255in}{0.815822in}}%
\pgfpathlineto{\pgfqpoint{1.178764in}{0.785866in}}%
\pgfpathlineto{\pgfqpoint{1.183273in}{0.815822in}}%
\pgfpathlineto{\pgfqpoint{1.187782in}{0.835792in}}%
\pgfpathlineto{\pgfqpoint{1.192291in}{0.830799in}}%
\pgfpathlineto{\pgfqpoint{1.196800in}{0.850770in}}%
\pgfpathlineto{\pgfqpoint{1.201309in}{0.835792in}}%
\pgfpathlineto{\pgfqpoint{1.205818in}{0.830799in}}%
\pgfpathlineto{\pgfqpoint{1.210327in}{0.875733in}}%
\pgfpathlineto{\pgfqpoint{1.214836in}{0.830799in}}%
\pgfpathlineto{\pgfqpoint{1.219345in}{0.845777in}}%
\pgfpathlineto{\pgfqpoint{1.228364in}{0.835792in}}%
\pgfpathlineto{\pgfqpoint{1.232873in}{0.850770in}}%
\pgfpathlineto{\pgfqpoint{1.237382in}{0.860755in}}%
\pgfpathlineto{\pgfqpoint{1.241891in}{0.855762in}}%
\pgfpathlineto{\pgfqpoint{1.246400in}{0.860755in}}%
\pgfpathlineto{\pgfqpoint{1.250909in}{0.870740in}}%
\pgfpathlineto{\pgfqpoint{1.255418in}{0.860755in}}%
\pgfpathlineto{\pgfqpoint{1.273455in}{0.880725in}}%
\pgfpathlineto{\pgfqpoint{1.277964in}{0.900695in}}%
\pgfpathlineto{\pgfqpoint{1.286982in}{0.880725in}}%
\pgfpathlineto{\pgfqpoint{1.291491in}{0.900695in}}%
\pgfpathlineto{\pgfqpoint{1.296000in}{0.875733in}}%
\pgfpathlineto{\pgfqpoint{1.300509in}{0.910681in}}%
\pgfpathlineto{\pgfqpoint{1.305018in}{0.905688in}}%
\pgfpathlineto{\pgfqpoint{1.309527in}{0.915673in}}%
\pgfpathlineto{\pgfqpoint{1.314036in}{0.930651in}}%
\pgfpathlineto{\pgfqpoint{1.318545in}{0.925658in}}%
\pgfpathlineto{\pgfqpoint{1.323055in}{0.930651in}}%
\pgfpathlineto{\pgfqpoint{1.327564in}{0.910681in}}%
\pgfpathlineto{\pgfqpoint{1.332073in}{0.970591in}}%
\pgfpathlineto{\pgfqpoint{1.336582in}{0.965599in}}%
\pgfpathlineto{\pgfqpoint{1.341091in}{0.935643in}}%
\pgfpathlineto{\pgfqpoint{1.345600in}{0.930651in}}%
\pgfpathlineto{\pgfqpoint{1.350109in}{0.940636in}}%
\pgfpathlineto{\pgfqpoint{1.354618in}{0.920666in}}%
\pgfpathlineto{\pgfqpoint{1.359127in}{0.980577in}}%
\pgfpathlineto{\pgfqpoint{1.363636in}{0.945629in}}%
\pgfpathlineto{\pgfqpoint{1.368145in}{0.920666in}}%
\pgfpathlineto{\pgfqpoint{1.372655in}{0.960606in}}%
\pgfpathlineto{\pgfqpoint{1.377164in}{0.950621in}}%
\pgfpathlineto{\pgfqpoint{1.381673in}{0.970591in}}%
\pgfpathlineto{\pgfqpoint{1.386182in}{0.970591in}}%
\pgfpathlineto{\pgfqpoint{1.390691in}{0.975584in}}%
\pgfpathlineto{\pgfqpoint{1.395200in}{0.970591in}}%
\pgfpathlineto{\pgfqpoint{1.399709in}{0.990562in}}%
\pgfpathlineto{\pgfqpoint{1.404218in}{0.970591in}}%
\pgfpathlineto{\pgfqpoint{1.408727in}{0.970591in}}%
\pgfpathlineto{\pgfqpoint{1.413236in}{0.985569in}}%
\pgfpathlineto{\pgfqpoint{1.417745in}{0.990562in}}%
\pgfpathlineto{\pgfqpoint{1.422255in}{0.970591in}}%
\pgfpathlineto{\pgfqpoint{1.426764in}{1.005539in}}%
\pgfpathlineto{\pgfqpoint{1.431273in}{1.005539in}}%
\pgfpathlineto{\pgfqpoint{1.435782in}{1.020517in}}%
\pgfpathlineto{\pgfqpoint{1.440291in}{0.995554in}}%
\pgfpathlineto{\pgfqpoint{1.444800in}{1.025510in}}%
\pgfpathlineto{\pgfqpoint{1.449309in}{1.020517in}}%
\pgfpathlineto{\pgfqpoint{1.453818in}{1.000547in}}%
\pgfpathlineto{\pgfqpoint{1.458327in}{1.035495in}}%
\pgfpathlineto{\pgfqpoint{1.462836in}{1.030502in}}%
\pgfpathlineto{\pgfqpoint{1.467345in}{1.005539in}}%
\pgfpathlineto{\pgfqpoint{1.471855in}{1.015525in}}%
\pgfpathlineto{\pgfqpoint{1.476364in}{1.050473in}}%
\pgfpathlineto{\pgfqpoint{1.480873in}{1.045480in}}%
\pgfpathlineto{\pgfqpoint{1.485382in}{1.030502in}}%
\pgfpathlineto{\pgfqpoint{1.489891in}{1.005539in}}%
\pgfpathlineto{\pgfqpoint{1.494400in}{1.050473in}}%
\pgfpathlineto{\pgfqpoint{1.498909in}{1.035495in}}%
\pgfpathlineto{\pgfqpoint{1.503418in}{1.060458in}}%
\pgfpathlineto{\pgfqpoint{1.507927in}{1.070443in}}%
\pgfpathlineto{\pgfqpoint{1.512436in}{1.095406in}}%
\pgfpathlineto{\pgfqpoint{1.516945in}{1.070443in}}%
\pgfpathlineto{\pgfqpoint{1.521455in}{1.065450in}}%
\pgfpathlineto{\pgfqpoint{1.525964in}{1.095406in}}%
\pgfpathlineto{\pgfqpoint{1.530473in}{1.100398in}}%
\pgfpathlineto{\pgfqpoint{1.534982in}{1.100398in}}%
\pgfpathlineto{\pgfqpoint{1.539491in}{1.090413in}}%
\pgfpathlineto{\pgfqpoint{1.544000in}{1.075435in}}%
\pgfpathlineto{\pgfqpoint{1.548509in}{1.095406in}}%
\pgfpathlineto{\pgfqpoint{1.553018in}{1.090413in}}%
\pgfpathlineto{\pgfqpoint{1.557527in}{1.075435in}}%
\pgfpathlineto{\pgfqpoint{1.562036in}{1.080428in}}%
\pgfpathlineto{\pgfqpoint{1.566545in}{1.100398in}}%
\pgfpathlineto{\pgfqpoint{1.571055in}{1.110383in}}%
\pgfpathlineto{\pgfqpoint{1.575564in}{1.105391in}}%
\pgfpathlineto{\pgfqpoint{1.580073in}{1.090413in}}%
\pgfpathlineto{\pgfqpoint{1.584582in}{1.110383in}}%
\pgfpathlineto{\pgfqpoint{1.589091in}{1.110383in}}%
\pgfpathlineto{\pgfqpoint{1.593600in}{1.180279in}}%
\pgfpathlineto{\pgfqpoint{1.598109in}{1.140339in}}%
\pgfpathlineto{\pgfqpoint{1.602618in}{1.110383in}}%
\pgfpathlineto{\pgfqpoint{1.607127in}{1.145331in}}%
\pgfpathlineto{\pgfqpoint{1.611636in}{1.140339in}}%
\pgfpathlineto{\pgfqpoint{1.616145in}{1.115376in}}%
\pgfpathlineto{\pgfqpoint{1.620655in}{1.140339in}}%
\pgfpathlineto{\pgfqpoint{1.629673in}{1.150324in}}%
\pgfpathlineto{\pgfqpoint{1.634182in}{1.145331in}}%
\pgfpathlineto{\pgfqpoint{1.638691in}{1.130354in}}%
\pgfpathlineto{\pgfqpoint{1.643200in}{1.165302in}}%
\pgfpathlineto{\pgfqpoint{1.647709in}{1.155316in}}%
\pgfpathlineto{\pgfqpoint{1.656727in}{1.175287in}}%
\pgfpathlineto{\pgfqpoint{1.661236in}{1.170294in}}%
\pgfpathlineto{\pgfqpoint{1.665745in}{1.200250in}}%
\pgfpathlineto{\pgfqpoint{1.670255in}{1.160309in}}%
\pgfpathlineto{\pgfqpoint{1.674764in}{1.175287in}}%
\pgfpathlineto{\pgfqpoint{1.679273in}{1.205242in}}%
\pgfpathlineto{\pgfqpoint{1.683782in}{1.140339in}}%
\pgfpathlineto{\pgfqpoint{1.688291in}{1.180279in}}%
\pgfpathlineto{\pgfqpoint{1.692800in}{1.175287in}}%
\pgfpathlineto{\pgfqpoint{1.697309in}{1.200250in}}%
\pgfpathlineto{\pgfqpoint{1.701818in}{1.205242in}}%
\pgfpathlineto{\pgfqpoint{1.706327in}{1.200250in}}%
\pgfpathlineto{\pgfqpoint{1.710836in}{1.200250in}}%
\pgfpathlineto{\pgfqpoint{1.715345in}{1.215227in}}%
\pgfpathlineto{\pgfqpoint{1.719855in}{1.215227in}}%
\pgfpathlineto{\pgfqpoint{1.724364in}{1.250175in}}%
\pgfpathlineto{\pgfqpoint{1.728873in}{1.225212in}}%
\pgfpathlineto{\pgfqpoint{1.733382in}{1.210235in}}%
\pgfpathlineto{\pgfqpoint{1.737891in}{1.225212in}}%
\pgfpathlineto{\pgfqpoint{1.742400in}{1.260160in}}%
\pgfpathlineto{\pgfqpoint{1.746909in}{1.235198in}}%
\pgfpathlineto{\pgfqpoint{1.751418in}{1.220220in}}%
\pgfpathlineto{\pgfqpoint{1.755927in}{1.250175in}}%
\pgfpathlineto{\pgfqpoint{1.760436in}{1.250175in}}%
\pgfpathlineto{\pgfqpoint{1.769455in}{1.280131in}}%
\pgfpathlineto{\pgfqpoint{1.773964in}{1.280131in}}%
\pgfpathlineto{\pgfqpoint{1.778473in}{1.260160in}}%
\pgfpathlineto{\pgfqpoint{1.782982in}{1.280131in}}%
\pgfpathlineto{\pgfqpoint{1.787491in}{1.255168in}}%
\pgfpathlineto{\pgfqpoint{1.792000in}{1.275138in}}%
\pgfpathlineto{\pgfqpoint{1.801018in}{1.275138in}}%
\pgfpathlineto{\pgfqpoint{1.805527in}{1.270146in}}%
\pgfpathlineto{\pgfqpoint{1.810036in}{1.260160in}}%
\pgfpathlineto{\pgfqpoint{1.814545in}{1.295108in}}%
\pgfpathlineto{\pgfqpoint{1.819055in}{1.315079in}}%
\pgfpathlineto{\pgfqpoint{1.823564in}{1.295108in}}%
\pgfpathlineto{\pgfqpoint{1.828073in}{1.300101in}}%
\pgfpathlineto{\pgfqpoint{1.832582in}{1.270146in}}%
\pgfpathlineto{\pgfqpoint{1.837091in}{1.310086in}}%
\pgfpathlineto{\pgfqpoint{1.841600in}{1.305094in}}%
\pgfpathlineto{\pgfqpoint{1.846109in}{1.320071in}}%
\pgfpathlineto{\pgfqpoint{1.850618in}{1.305094in}}%
\pgfpathlineto{\pgfqpoint{1.855127in}{1.300101in}}%
\pgfpathlineto{\pgfqpoint{1.859636in}{1.320071in}}%
\pgfpathlineto{\pgfqpoint{1.864145in}{1.325064in}}%
\pgfpathlineto{\pgfqpoint{1.868655in}{1.335049in}}%
\pgfpathlineto{\pgfqpoint{1.873164in}{1.355019in}}%
\pgfpathlineto{\pgfqpoint{1.877673in}{1.345034in}}%
\pgfpathlineto{\pgfqpoint{1.882182in}{1.350027in}}%
\pgfpathlineto{\pgfqpoint{1.886691in}{1.335049in}}%
\pgfpathlineto{\pgfqpoint{1.891200in}{1.345034in}}%
\pgfpathlineto{\pgfqpoint{1.895709in}{1.325064in}}%
\pgfpathlineto{\pgfqpoint{1.900218in}{1.379982in}}%
\pgfpathlineto{\pgfqpoint{1.904727in}{1.345034in}}%
\pgfpathlineto{\pgfqpoint{1.913745in}{1.355019in}}%
\pgfpathlineto{\pgfqpoint{1.918255in}{1.345034in}}%
\pgfpathlineto{\pgfqpoint{1.922764in}{1.374990in}}%
\pgfpathlineto{\pgfqpoint{1.927273in}{1.379982in}}%
\pgfpathlineto{\pgfqpoint{1.931782in}{1.424915in}}%
\pgfpathlineto{\pgfqpoint{1.936291in}{1.369997in}}%
\pgfpathlineto{\pgfqpoint{1.940800in}{1.394960in}}%
\pgfpathlineto{\pgfqpoint{1.954327in}{1.379982in}}%
\pgfpathlineto{\pgfqpoint{1.958836in}{1.389967in}}%
\pgfpathlineto{\pgfqpoint{1.963345in}{1.379982in}}%
\pgfpathlineto{\pgfqpoint{1.967855in}{1.384975in}}%
\pgfpathlineto{\pgfqpoint{1.972364in}{1.434900in}}%
\pgfpathlineto{\pgfqpoint{1.976873in}{1.404945in}}%
\pgfpathlineto{\pgfqpoint{1.981382in}{1.419923in}}%
\pgfpathlineto{\pgfqpoint{1.985891in}{1.424915in}}%
\pgfpathlineto{\pgfqpoint{1.990400in}{1.409938in}}%
\pgfpathlineto{\pgfqpoint{1.994909in}{1.409938in}}%
\pgfpathlineto{\pgfqpoint{1.999418in}{1.419923in}}%
\pgfpathlineto{\pgfqpoint{2.003927in}{1.404945in}}%
\pgfpathlineto{\pgfqpoint{2.008436in}{1.399952in}}%
\pgfpathlineto{\pgfqpoint{2.012945in}{1.409938in}}%
\pgfpathlineto{\pgfqpoint{2.017455in}{1.414930in}}%
\pgfpathlineto{\pgfqpoint{2.021964in}{1.459863in}}%
\pgfpathlineto{\pgfqpoint{2.026473in}{1.429908in}}%
\pgfpathlineto{\pgfqpoint{2.030982in}{1.449878in}}%
\pgfpathlineto{\pgfqpoint{2.035491in}{1.459863in}}%
\pgfpathlineto{\pgfqpoint{2.040000in}{1.444886in}}%
\pgfpathlineto{\pgfqpoint{2.044509in}{1.494811in}}%
\pgfpathlineto{\pgfqpoint{2.049018in}{1.449878in}}%
\pgfpathlineto{\pgfqpoint{2.053527in}{1.454871in}}%
\pgfpathlineto{\pgfqpoint{2.058036in}{1.424915in}}%
\pgfpathlineto{\pgfqpoint{2.062545in}{1.464856in}}%
\pgfpathlineto{\pgfqpoint{2.067055in}{1.459863in}}%
\pgfpathlineto{\pgfqpoint{2.071564in}{1.479834in}}%
\pgfpathlineto{\pgfqpoint{2.076073in}{1.514782in}}%
\pgfpathlineto{\pgfqpoint{2.080582in}{1.494811in}}%
\pgfpathlineto{\pgfqpoint{2.085091in}{1.449878in}}%
\pgfpathlineto{\pgfqpoint{2.089600in}{1.499804in}}%
\pgfpathlineto{\pgfqpoint{2.094109in}{1.459863in}}%
\pgfpathlineto{\pgfqpoint{2.103127in}{1.504796in}}%
\pgfpathlineto{\pgfqpoint{2.112145in}{1.514782in}}%
\pgfpathlineto{\pgfqpoint{2.116655in}{1.509789in}}%
\pgfpathlineto{\pgfqpoint{2.121164in}{1.519774in}}%
\pgfpathlineto{\pgfqpoint{2.125673in}{1.524767in}}%
\pgfpathlineto{\pgfqpoint{2.130182in}{1.489819in}}%
\pgfpathlineto{\pgfqpoint{2.134691in}{1.524767in}}%
\pgfpathlineto{\pgfqpoint{2.143709in}{1.469848in}}%
\pgfpathlineto{\pgfqpoint{2.148218in}{1.544737in}}%
\pgfpathlineto{\pgfqpoint{2.152727in}{1.499804in}}%
\pgfpathlineto{\pgfqpoint{2.157236in}{1.544737in}}%
\pgfpathlineto{\pgfqpoint{2.161745in}{1.544737in}}%
\pgfpathlineto{\pgfqpoint{2.166255in}{1.529759in}}%
\pgfpathlineto{\pgfqpoint{2.170764in}{1.524767in}}%
\pgfpathlineto{\pgfqpoint{2.175273in}{1.529759in}}%
\pgfpathlineto{\pgfqpoint{2.179782in}{1.554722in}}%
\pgfpathlineto{\pgfqpoint{2.184291in}{1.599655in}}%
\pgfpathlineto{\pgfqpoint{2.188800in}{1.574692in}}%
\pgfpathlineto{\pgfqpoint{2.193309in}{1.569700in}}%
\pgfpathlineto{\pgfqpoint{2.197818in}{1.574692in}}%
\pgfpathlineto{\pgfqpoint{2.202327in}{1.564707in}}%
\pgfpathlineto{\pgfqpoint{2.206836in}{1.584678in}}%
\pgfpathlineto{\pgfqpoint{2.211345in}{1.589670in}}%
\pgfpathlineto{\pgfqpoint{2.215855in}{1.564707in}}%
\pgfpathlineto{\pgfqpoint{2.220364in}{1.584678in}}%
\pgfpathlineto{\pgfqpoint{2.224873in}{1.584678in}}%
\pgfpathlineto{\pgfqpoint{2.229382in}{1.569700in}}%
\pgfpathlineto{\pgfqpoint{2.233891in}{1.609640in}}%
\pgfpathlineto{\pgfqpoint{2.238400in}{1.544737in}}%
\pgfpathlineto{\pgfqpoint{2.242909in}{1.624618in}}%
\pgfpathlineto{\pgfqpoint{2.247418in}{1.619626in}}%
\pgfpathlineto{\pgfqpoint{2.251927in}{1.584678in}}%
\pgfpathlineto{\pgfqpoint{2.256436in}{1.629611in}}%
\pgfpathlineto{\pgfqpoint{2.260945in}{1.644588in}}%
\pgfpathlineto{\pgfqpoint{2.265455in}{1.634603in}}%
\pgfpathlineto{\pgfqpoint{2.269964in}{1.614633in}}%
\pgfpathlineto{\pgfqpoint{2.274473in}{1.629611in}}%
\pgfpathlineto{\pgfqpoint{2.278982in}{1.604648in}}%
\pgfpathlineto{\pgfqpoint{2.283491in}{1.629611in}}%
\pgfpathlineto{\pgfqpoint{2.288000in}{1.669551in}}%
\pgfpathlineto{\pgfqpoint{2.292509in}{1.639596in}}%
\pgfpathlineto{\pgfqpoint{2.297018in}{1.639596in}}%
\pgfpathlineto{\pgfqpoint{2.301527in}{1.629611in}}%
\pgfpathlineto{\pgfqpoint{2.306036in}{1.659566in}}%
\pgfpathlineto{\pgfqpoint{2.310545in}{1.629611in}}%
\pgfpathlineto{\pgfqpoint{2.315055in}{1.659566in}}%
\pgfpathlineto{\pgfqpoint{2.319564in}{1.639596in}}%
\pgfpathlineto{\pgfqpoint{2.324073in}{1.609640in}}%
\pgfpathlineto{\pgfqpoint{2.328582in}{1.679536in}}%
\pgfpathlineto{\pgfqpoint{2.333091in}{1.659566in}}%
\pgfpathlineto{\pgfqpoint{2.337600in}{1.704499in}}%
\pgfpathlineto{\pgfqpoint{2.342109in}{1.709492in}}%
\pgfpathlineto{\pgfqpoint{2.346618in}{1.649581in}}%
\pgfpathlineto{\pgfqpoint{2.351127in}{1.719477in}}%
\pgfpathlineto{\pgfqpoint{2.355636in}{1.659566in}}%
\pgfpathlineto{\pgfqpoint{2.360145in}{1.684529in}}%
\pgfpathlineto{\pgfqpoint{2.364655in}{1.684529in}}%
\pgfpathlineto{\pgfqpoint{2.369164in}{1.694514in}}%
\pgfpathlineto{\pgfqpoint{2.373673in}{1.699507in}}%
\pgfpathlineto{\pgfqpoint{2.378182in}{1.719477in}}%
\pgfpathlineto{\pgfqpoint{2.382691in}{1.704499in}}%
\pgfpathlineto{\pgfqpoint{2.387200in}{1.734455in}}%
\pgfpathlineto{\pgfqpoint{2.391709in}{1.709492in}}%
\pgfpathlineto{\pgfqpoint{2.396218in}{1.714484in}}%
\pgfpathlineto{\pgfqpoint{2.400727in}{1.749432in}}%
\pgfpathlineto{\pgfqpoint{2.405236in}{1.724470in}}%
\pgfpathlineto{\pgfqpoint{2.409745in}{1.764410in}}%
\pgfpathlineto{\pgfqpoint{2.414255in}{1.654574in}}%
\pgfpathlineto{\pgfqpoint{2.423273in}{1.759418in}}%
\pgfpathlineto{\pgfqpoint{2.427782in}{1.734455in}}%
\pgfpathlineto{\pgfqpoint{2.432291in}{1.769403in}}%
\pgfpathlineto{\pgfqpoint{2.436800in}{1.729462in}}%
\pgfpathlineto{\pgfqpoint{2.441309in}{1.734455in}}%
\pgfpathlineto{\pgfqpoint{2.450327in}{1.769403in}}%
\pgfpathlineto{\pgfqpoint{2.454836in}{1.749432in}}%
\pgfpathlineto{\pgfqpoint{2.463855in}{1.769403in}}%
\pgfpathlineto{\pgfqpoint{2.468364in}{1.789373in}}%
\pgfpathlineto{\pgfqpoint{2.472873in}{1.744440in}}%
\pgfpathlineto{\pgfqpoint{2.481891in}{1.784380in}}%
\pgfpathlineto{\pgfqpoint{2.486400in}{1.784380in}}%
\pgfpathlineto{\pgfqpoint{2.490909in}{1.799358in}}%
\pgfpathlineto{\pgfqpoint{2.495418in}{1.804351in}}%
\pgfpathlineto{\pgfqpoint{2.499927in}{1.804351in}}%
\pgfpathlineto{\pgfqpoint{2.504436in}{1.799358in}}%
\pgfpathlineto{\pgfqpoint{2.508945in}{1.819328in}}%
\pgfpathlineto{\pgfqpoint{2.513455in}{1.779388in}}%
\pgfpathlineto{\pgfqpoint{2.517964in}{1.809343in}}%
\pgfpathlineto{\pgfqpoint{2.522473in}{1.784380in}}%
\pgfpathlineto{\pgfqpoint{2.526982in}{1.819328in}}%
\pgfpathlineto{\pgfqpoint{2.531491in}{1.809343in}}%
\pgfpathlineto{\pgfqpoint{2.536000in}{1.849284in}}%
\pgfpathlineto{\pgfqpoint{2.540509in}{1.869254in}}%
\pgfpathlineto{\pgfqpoint{2.545018in}{1.834306in}}%
\pgfpathlineto{\pgfqpoint{2.549527in}{1.844291in}}%
\pgfpathlineto{\pgfqpoint{2.554036in}{1.829314in}}%
\pgfpathlineto{\pgfqpoint{2.567564in}{1.859269in}}%
\pgfpathlineto{\pgfqpoint{2.572073in}{1.844291in}}%
\pgfpathlineto{\pgfqpoint{2.576582in}{1.819328in}}%
\pgfpathlineto{\pgfqpoint{2.581091in}{1.859269in}}%
\pgfpathlineto{\pgfqpoint{2.585600in}{1.869254in}}%
\pgfpathlineto{\pgfqpoint{2.590109in}{1.889224in}}%
\pgfpathlineto{\pgfqpoint{2.594618in}{1.864262in}}%
\pgfpathlineto{\pgfqpoint{2.599127in}{1.859269in}}%
\pgfpathlineto{\pgfqpoint{2.603636in}{1.889224in}}%
\pgfpathlineto{\pgfqpoint{2.612655in}{1.854276in}}%
\pgfpathlineto{\pgfqpoint{2.617164in}{1.909195in}}%
\pgfpathlineto{\pgfqpoint{2.621673in}{1.884232in}}%
\pgfpathlineto{\pgfqpoint{2.626182in}{1.929165in}}%
\pgfpathlineto{\pgfqpoint{2.630691in}{1.884232in}}%
\pgfpathlineto{\pgfqpoint{2.635200in}{1.874247in}}%
\pgfpathlineto{\pgfqpoint{2.639709in}{1.919180in}}%
\pgfpathlineto{\pgfqpoint{2.644218in}{1.869254in}}%
\pgfpathlineto{\pgfqpoint{2.648727in}{1.924172in}}%
\pgfpathlineto{\pgfqpoint{2.653236in}{1.914187in}}%
\pgfpathlineto{\pgfqpoint{2.657745in}{1.889224in}}%
\pgfpathlineto{\pgfqpoint{2.662255in}{1.914187in}}%
\pgfpathlineto{\pgfqpoint{2.666764in}{1.904202in}}%
\pgfpathlineto{\pgfqpoint{2.671273in}{1.929165in}}%
\pgfpathlineto{\pgfqpoint{2.675782in}{1.939150in}}%
\pgfpathlineto{\pgfqpoint{2.680291in}{1.944143in}}%
\pgfpathlineto{\pgfqpoint{2.689309in}{1.914187in}}%
\pgfpathlineto{\pgfqpoint{2.693818in}{1.914187in}}%
\pgfpathlineto{\pgfqpoint{2.698327in}{1.939150in}}%
\pgfpathlineto{\pgfqpoint{2.702836in}{2.004053in}}%
\pgfpathlineto{\pgfqpoint{2.707345in}{1.994068in}}%
\pgfpathlineto{\pgfqpoint{2.711855in}{1.944143in}}%
\pgfpathlineto{\pgfqpoint{2.716364in}{1.954128in}}%
\pgfpathlineto{\pgfqpoint{2.720873in}{1.969105in}}%
\pgfpathlineto{\pgfqpoint{2.725382in}{1.954128in}}%
\pgfpathlineto{\pgfqpoint{2.729891in}{1.954128in}}%
\pgfpathlineto{\pgfqpoint{2.734400in}{1.974098in}}%
\pgfpathlineto{\pgfqpoint{2.743418in}{1.984083in}}%
\pgfpathlineto{\pgfqpoint{2.747927in}{1.979091in}}%
\pgfpathlineto{\pgfqpoint{2.752436in}{1.979091in}}%
\pgfpathlineto{\pgfqpoint{2.756945in}{1.984083in}}%
\pgfpathlineto{\pgfqpoint{2.761455in}{1.934158in}}%
\pgfpathlineto{\pgfqpoint{2.765964in}{2.004053in}}%
\pgfpathlineto{\pgfqpoint{2.770473in}{1.999061in}}%
\pgfpathlineto{\pgfqpoint{2.774982in}{1.964113in}}%
\pgfpathlineto{\pgfqpoint{2.779491in}{2.034009in}}%
\pgfpathlineto{\pgfqpoint{2.784000in}{1.999061in}}%
\pgfpathlineto{\pgfqpoint{2.788509in}{2.014039in}}%
\pgfpathlineto{\pgfqpoint{2.793018in}{1.999061in}}%
\pgfpathlineto{\pgfqpoint{2.797527in}{2.043994in}}%
\pgfpathlineto{\pgfqpoint{2.802036in}{2.019031in}}%
\pgfpathlineto{\pgfqpoint{2.806545in}{2.019031in}}%
\pgfpathlineto{\pgfqpoint{2.811055in}{2.068957in}}%
\pgfpathlineto{\pgfqpoint{2.815564in}{2.024024in}}%
\pgfpathlineto{\pgfqpoint{2.820073in}{2.053979in}}%
\pgfpathlineto{\pgfqpoint{2.824582in}{2.063964in}}%
\pgfpathlineto{\pgfqpoint{2.829091in}{2.029016in}}%
\pgfpathlineto{\pgfqpoint{2.833600in}{2.048987in}}%
\pgfpathlineto{\pgfqpoint{2.838109in}{2.004053in}}%
\pgfpathlineto{\pgfqpoint{2.842618in}{2.053979in}}%
\pgfpathlineto{\pgfqpoint{2.847127in}{2.009046in}}%
\pgfpathlineto{\pgfqpoint{2.851636in}{2.043994in}}%
\pgfpathlineto{\pgfqpoint{2.856145in}{2.034009in}}%
\pgfpathlineto{\pgfqpoint{2.860655in}{2.014039in}}%
\pgfpathlineto{\pgfqpoint{2.865164in}{2.083935in}}%
\pgfpathlineto{\pgfqpoint{2.869673in}{2.093920in}}%
\pgfpathlineto{\pgfqpoint{2.874182in}{2.063964in}}%
\pgfpathlineto{\pgfqpoint{2.878691in}{2.103905in}}%
\pgfpathlineto{\pgfqpoint{2.883200in}{2.083935in}}%
\pgfpathlineto{\pgfqpoint{2.887709in}{2.039001in}}%
\pgfpathlineto{\pgfqpoint{2.892218in}{2.073949in}}%
\pgfpathlineto{\pgfqpoint{2.896727in}{2.063964in}}%
\pgfpathlineto{\pgfqpoint{2.901236in}{2.068957in}}%
\pgfpathlineto{\pgfqpoint{2.910255in}{2.098912in}}%
\pgfpathlineto{\pgfqpoint{2.914764in}{2.123875in}}%
\pgfpathlineto{\pgfqpoint{2.919273in}{2.068957in}}%
\pgfpathlineto{\pgfqpoint{2.923782in}{2.083935in}}%
\pgfpathlineto{\pgfqpoint{2.928291in}{2.113890in}}%
\pgfpathlineto{\pgfqpoint{2.937309in}{2.113890in}}%
\pgfpathlineto{\pgfqpoint{2.941818in}{2.128868in}}%
\pgfpathlineto{\pgfqpoint{2.946327in}{2.113890in}}%
\pgfpathlineto{\pgfqpoint{2.950836in}{2.133860in}}%
\pgfpathlineto{\pgfqpoint{2.955345in}{2.108897in}}%
\pgfpathlineto{\pgfqpoint{2.959855in}{2.153831in}}%
\pgfpathlineto{\pgfqpoint{2.964364in}{2.123875in}}%
\pgfpathlineto{\pgfqpoint{2.968873in}{2.108897in}}%
\pgfpathlineto{\pgfqpoint{2.973382in}{2.118883in}}%
\pgfpathlineto{\pgfqpoint{2.977891in}{2.158823in}}%
\pgfpathlineto{\pgfqpoint{2.982400in}{2.183786in}}%
\pgfpathlineto{\pgfqpoint{2.986909in}{2.138853in}}%
\pgfpathlineto{\pgfqpoint{2.991418in}{2.178793in}}%
\pgfpathlineto{\pgfqpoint{2.995927in}{2.158823in}}%
\pgfpathlineto{\pgfqpoint{3.000436in}{2.148838in}}%
\pgfpathlineto{\pgfqpoint{3.004945in}{2.168808in}}%
\pgfpathlineto{\pgfqpoint{3.009455in}{2.073949in}}%
\pgfpathlineto{\pgfqpoint{3.013964in}{2.213741in}}%
\pgfpathlineto{\pgfqpoint{3.018473in}{2.183786in}}%
\pgfpathlineto{\pgfqpoint{3.022982in}{2.123875in}}%
\pgfpathlineto{\pgfqpoint{3.027491in}{2.203756in}}%
\pgfpathlineto{\pgfqpoint{3.032000in}{2.178793in}}%
\pgfpathlineto{\pgfqpoint{3.036509in}{2.173801in}}%
\pgfpathlineto{\pgfqpoint{3.041018in}{2.223727in}}%
\pgfpathlineto{\pgfqpoint{3.045527in}{2.208749in}}%
\pgfpathlineto{\pgfqpoint{3.050036in}{2.213741in}}%
\pgfpathlineto{\pgfqpoint{3.054545in}{2.188779in}}%
\pgfpathlineto{\pgfqpoint{3.059055in}{2.223727in}}%
\pgfpathlineto{\pgfqpoint{3.063564in}{2.193771in}}%
\pgfpathlineto{\pgfqpoint{3.068073in}{2.208749in}}%
\pgfpathlineto{\pgfqpoint{3.072582in}{2.238704in}}%
\pgfpathlineto{\pgfqpoint{3.077091in}{2.158823in}}%
\pgfpathlineto{\pgfqpoint{3.081600in}{2.208749in}}%
\pgfpathlineto{\pgfqpoint{3.086109in}{2.198764in}}%
\pgfpathlineto{\pgfqpoint{3.090618in}{2.233712in}}%
\pgfpathlineto{\pgfqpoint{3.095127in}{2.238704in}}%
\pgfpathlineto{\pgfqpoint{3.099636in}{2.258675in}}%
\pgfpathlineto{\pgfqpoint{3.104145in}{2.248689in}}%
\pgfpathlineto{\pgfqpoint{3.108655in}{2.278645in}}%
\pgfpathlineto{\pgfqpoint{3.113164in}{2.248689in}}%
\pgfpathlineto{\pgfqpoint{3.117673in}{2.248689in}}%
\pgfpathlineto{\pgfqpoint{3.122182in}{2.253682in}}%
\pgfpathlineto{\pgfqpoint{3.126691in}{2.263667in}}%
\pgfpathlineto{\pgfqpoint{3.131200in}{2.248689in}}%
\pgfpathlineto{\pgfqpoint{3.135709in}{2.258675in}}%
\pgfpathlineto{\pgfqpoint{3.140218in}{2.303608in}}%
\pgfpathlineto{\pgfqpoint{3.144727in}{2.233712in}}%
\pgfpathlineto{\pgfqpoint{3.149236in}{2.323578in}}%
\pgfpathlineto{\pgfqpoint{3.153745in}{2.268660in}}%
\pgfpathlineto{\pgfqpoint{3.158255in}{2.228719in}}%
\pgfpathlineto{\pgfqpoint{3.162764in}{2.263667in}}%
\pgfpathlineto{\pgfqpoint{3.167273in}{2.288630in}}%
\pgfpathlineto{\pgfqpoint{3.171782in}{2.288630in}}%
\pgfpathlineto{\pgfqpoint{3.176291in}{2.328571in}}%
\pgfpathlineto{\pgfqpoint{3.180800in}{2.333563in}}%
\pgfpathlineto{\pgfqpoint{3.185309in}{2.348541in}}%
\pgfpathlineto{\pgfqpoint{3.189818in}{2.323578in}}%
\pgfpathlineto{\pgfqpoint{3.194327in}{2.328571in}}%
\pgfpathlineto{\pgfqpoint{3.198836in}{2.328571in}}%
\pgfpathlineto{\pgfqpoint{3.203345in}{2.343548in}}%
\pgfpathlineto{\pgfqpoint{3.212364in}{2.333563in}}%
\pgfpathlineto{\pgfqpoint{3.216873in}{2.298615in}}%
\pgfpathlineto{\pgfqpoint{3.221382in}{2.323578in}}%
\pgfpathlineto{\pgfqpoint{3.225891in}{2.308600in}}%
\pgfpathlineto{\pgfqpoint{3.230400in}{2.338556in}}%
\pgfpathlineto{\pgfqpoint{3.234909in}{2.293623in}}%
\pgfpathlineto{\pgfqpoint{3.239418in}{2.338556in}}%
\pgfpathlineto{\pgfqpoint{3.243927in}{2.303608in}}%
\pgfpathlineto{\pgfqpoint{3.248436in}{2.343548in}}%
\pgfpathlineto{\pgfqpoint{3.252945in}{2.348541in}}%
\pgfpathlineto{\pgfqpoint{3.257455in}{2.323578in}}%
\pgfpathlineto{\pgfqpoint{3.261964in}{2.398467in}}%
\pgfpathlineto{\pgfqpoint{3.266473in}{2.333563in}}%
\pgfpathlineto{\pgfqpoint{3.270982in}{2.358526in}}%
\pgfpathlineto{\pgfqpoint{3.275491in}{2.363519in}}%
\pgfpathlineto{\pgfqpoint{3.280000in}{2.403459in}}%
\pgfpathlineto{\pgfqpoint{3.284509in}{2.343548in}}%
\pgfpathlineto{\pgfqpoint{3.289018in}{2.413444in}}%
\pgfpathlineto{\pgfqpoint{3.293527in}{2.363519in}}%
\pgfpathlineto{\pgfqpoint{3.298036in}{2.368511in}}%
\pgfpathlineto{\pgfqpoint{3.302545in}{2.383489in}}%
\pgfpathlineto{\pgfqpoint{3.307055in}{2.403459in}}%
\pgfpathlineto{\pgfqpoint{3.311564in}{2.373504in}}%
\pgfpathlineto{\pgfqpoint{3.316073in}{2.358526in}}%
\pgfpathlineto{\pgfqpoint{3.320582in}{2.433415in}}%
\pgfpathlineto{\pgfqpoint{3.325091in}{2.463370in}}%
\pgfpathlineto{\pgfqpoint{3.329600in}{2.398467in}}%
\pgfpathlineto{\pgfqpoint{3.334109in}{2.458377in}}%
\pgfpathlineto{\pgfqpoint{3.338618in}{2.443400in}}%
\pgfpathlineto{\pgfqpoint{3.343127in}{2.363519in}}%
\pgfpathlineto{\pgfqpoint{3.352145in}{2.458377in}}%
\pgfpathlineto{\pgfqpoint{3.356655in}{2.428422in}}%
\pgfpathlineto{\pgfqpoint{3.361164in}{2.423429in}}%
\pgfpathlineto{\pgfqpoint{3.365673in}{2.458377in}}%
\pgfpathlineto{\pgfqpoint{3.370182in}{2.388481in}}%
\pgfpathlineto{\pgfqpoint{3.374691in}{2.428422in}}%
\pgfpathlineto{\pgfqpoint{3.379200in}{2.428422in}}%
\pgfpathlineto{\pgfqpoint{3.383709in}{2.453385in}}%
\pgfpathlineto{\pgfqpoint{3.388218in}{2.423429in}}%
\pgfpathlineto{\pgfqpoint{3.392727in}{2.468363in}}%
\pgfpathlineto{\pgfqpoint{3.397236in}{2.463370in}}%
\pgfpathlineto{\pgfqpoint{3.401745in}{2.468363in}}%
\pgfpathlineto{\pgfqpoint{3.406255in}{2.478348in}}%
\pgfpathlineto{\pgfqpoint{3.415273in}{2.428422in}}%
\pgfpathlineto{\pgfqpoint{3.419782in}{2.483340in}}%
\pgfpathlineto{\pgfqpoint{3.424291in}{2.428422in}}%
\pgfpathlineto{\pgfqpoint{3.428800in}{2.473355in}}%
\pgfpathlineto{\pgfqpoint{3.433309in}{2.453385in}}%
\pgfpathlineto{\pgfqpoint{3.437818in}{2.538259in}}%
\pgfpathlineto{\pgfqpoint{3.442327in}{2.513296in}}%
\pgfpathlineto{\pgfqpoint{3.446836in}{2.513296in}}%
\pgfpathlineto{\pgfqpoint{3.451345in}{2.498318in}}%
\pgfpathlineto{\pgfqpoint{3.455855in}{2.458377in}}%
\pgfpathlineto{\pgfqpoint{3.460364in}{2.508303in}}%
\pgfpathlineto{\pgfqpoint{3.464873in}{2.468363in}}%
\pgfpathlineto{\pgfqpoint{3.469382in}{2.568214in}}%
\pgfpathlineto{\pgfqpoint{3.473891in}{2.458377in}}%
\pgfpathlineto{\pgfqpoint{3.478400in}{2.503311in}}%
\pgfpathlineto{\pgfqpoint{3.482909in}{2.483340in}}%
\pgfpathlineto{\pgfqpoint{3.487418in}{2.528273in}}%
\pgfpathlineto{\pgfqpoint{3.491927in}{2.538259in}}%
\pgfpathlineto{\pgfqpoint{3.500945in}{2.518288in}}%
\pgfpathlineto{\pgfqpoint{3.505455in}{2.553236in}}%
\pgfpathlineto{\pgfqpoint{3.509964in}{2.538259in}}%
\pgfpathlineto{\pgfqpoint{3.514473in}{2.573207in}}%
\pgfpathlineto{\pgfqpoint{3.518982in}{2.573207in}}%
\pgfpathlineto{\pgfqpoint{3.523491in}{2.553236in}}%
\pgfpathlineto{\pgfqpoint{3.528000in}{2.518288in}}%
\pgfpathlineto{\pgfqpoint{3.532509in}{2.523281in}}%
\pgfpathlineto{\pgfqpoint{3.537018in}{2.573207in}}%
\pgfpathlineto{\pgfqpoint{3.541527in}{2.523281in}}%
\pgfpathlineto{\pgfqpoint{3.550545in}{2.568214in}}%
\pgfpathlineto{\pgfqpoint{3.555055in}{2.538259in}}%
\pgfpathlineto{\pgfqpoint{3.559564in}{2.608155in}}%
\pgfpathlineto{\pgfqpoint{3.564073in}{2.538259in}}%
\pgfpathlineto{\pgfqpoint{3.568582in}{2.633117in}}%
\pgfpathlineto{\pgfqpoint{3.573091in}{2.553236in}}%
\pgfpathlineto{\pgfqpoint{3.577600in}{2.618140in}}%
\pgfpathlineto{\pgfqpoint{3.582109in}{2.633117in}}%
\pgfpathlineto{\pgfqpoint{3.586618in}{2.593177in}}%
\pgfpathlineto{\pgfqpoint{3.591127in}{2.583192in}}%
\pgfpathlineto{\pgfqpoint{3.595636in}{2.623132in}}%
\pgfpathlineto{\pgfqpoint{3.600145in}{2.623132in}}%
\pgfpathlineto{\pgfqpoint{3.604655in}{2.613147in}}%
\pgfpathlineto{\pgfqpoint{3.613673in}{2.603162in}}%
\pgfpathlineto{\pgfqpoint{3.618182in}{2.603162in}}%
\pgfpathlineto{\pgfqpoint{3.622691in}{2.593177in}}%
\pgfpathlineto{\pgfqpoint{3.627200in}{2.613147in}}%
\pgfpathlineto{\pgfqpoint{3.631709in}{2.653088in}}%
\pgfpathlineto{\pgfqpoint{3.636218in}{2.638110in}}%
\pgfpathlineto{\pgfqpoint{3.640727in}{2.633117in}}%
\pgfpathlineto{\pgfqpoint{3.645236in}{2.643103in}}%
\pgfpathlineto{\pgfqpoint{3.649745in}{2.708006in}}%
\pgfpathlineto{\pgfqpoint{3.654255in}{2.608155in}}%
\pgfpathlineto{\pgfqpoint{3.658764in}{2.673058in}}%
\pgfpathlineto{\pgfqpoint{3.663273in}{2.648095in}}%
\pgfpathlineto{\pgfqpoint{3.667782in}{2.668065in}}%
\pgfpathlineto{\pgfqpoint{3.672291in}{2.598169in}}%
\pgfpathlineto{\pgfqpoint{3.676800in}{2.638110in}}%
\pgfpathlineto{\pgfqpoint{3.681309in}{2.633117in}}%
\pgfpathlineto{\pgfqpoint{3.685818in}{2.688036in}}%
\pgfpathlineto{\pgfqpoint{3.690327in}{2.618140in}}%
\pgfpathlineto{\pgfqpoint{3.694836in}{2.717991in}}%
\pgfpathlineto{\pgfqpoint{3.699345in}{2.678051in}}%
\pgfpathlineto{\pgfqpoint{3.703855in}{2.698021in}}%
\pgfpathlineto{\pgfqpoint{3.708364in}{2.638110in}}%
\pgfpathlineto{\pgfqpoint{3.712873in}{2.708006in}}%
\pgfpathlineto{\pgfqpoint{3.717382in}{2.717991in}}%
\pgfpathlineto{\pgfqpoint{3.721891in}{2.742954in}}%
\pgfpathlineto{\pgfqpoint{3.726400in}{2.688036in}}%
\pgfpathlineto{\pgfqpoint{3.730909in}{2.722984in}}%
\pgfpathlineto{\pgfqpoint{3.735418in}{2.717991in}}%
\pgfpathlineto{\pgfqpoint{3.739927in}{2.688036in}}%
\pgfpathlineto{\pgfqpoint{3.744436in}{2.693028in}}%
\pgfpathlineto{\pgfqpoint{3.748945in}{2.722984in}}%
\pgfpathlineto{\pgfqpoint{3.753455in}{2.732969in}}%
\pgfpathlineto{\pgfqpoint{3.762473in}{2.678051in}}%
\pgfpathlineto{\pgfqpoint{3.766982in}{2.747947in}}%
\pgfpathlineto{\pgfqpoint{3.771491in}{2.737961in}}%
\pgfpathlineto{\pgfqpoint{3.776000in}{2.732969in}}%
\pgfpathlineto{\pgfqpoint{3.780509in}{2.792880in}}%
\pgfpathlineto{\pgfqpoint{3.785018in}{2.742954in}}%
\pgfpathlineto{\pgfqpoint{3.789527in}{2.742954in}}%
\pgfpathlineto{\pgfqpoint{3.794036in}{2.752939in}}%
\pgfpathlineto{\pgfqpoint{3.798545in}{2.747947in}}%
\pgfpathlineto{\pgfqpoint{3.803055in}{2.722984in}}%
\pgfpathlineto{\pgfqpoint{3.812073in}{2.742954in}}%
\pgfpathlineto{\pgfqpoint{3.825600in}{2.787887in}}%
\pgfpathlineto{\pgfqpoint{3.830109in}{2.747947in}}%
\pgfpathlineto{\pgfqpoint{3.834618in}{2.812850in}}%
\pgfpathlineto{\pgfqpoint{3.839127in}{2.727976in}}%
\pgfpathlineto{\pgfqpoint{3.843636in}{2.792880in}}%
\pgfpathlineto{\pgfqpoint{3.848145in}{2.787887in}}%
\pgfpathlineto{\pgfqpoint{3.852655in}{2.812850in}}%
\pgfpathlineto{\pgfqpoint{3.857164in}{2.797872in}}%
\pgfpathlineto{\pgfqpoint{3.861673in}{2.857783in}}%
\pgfpathlineto{\pgfqpoint{3.870691in}{2.782895in}}%
\pgfpathlineto{\pgfqpoint{3.875200in}{2.852790in}}%
\pgfpathlineto{\pgfqpoint{3.879709in}{2.847798in}}%
\pgfpathlineto{\pgfqpoint{3.884218in}{2.817842in}}%
\pgfpathlineto{\pgfqpoint{3.888727in}{2.817842in}}%
\pgfpathlineto{\pgfqpoint{3.893236in}{2.797872in}}%
\pgfpathlineto{\pgfqpoint{3.897745in}{2.787887in}}%
\pgfpathlineto{\pgfqpoint{3.902255in}{2.807857in}}%
\pgfpathlineto{\pgfqpoint{3.906764in}{2.787887in}}%
\pgfpathlineto{\pgfqpoint{3.911273in}{2.822835in}}%
\pgfpathlineto{\pgfqpoint{3.915782in}{2.812850in}}%
\pgfpathlineto{\pgfqpoint{3.920291in}{2.857783in}}%
\pgfpathlineto{\pgfqpoint{3.929309in}{2.812850in}}%
\pgfpathlineto{\pgfqpoint{3.933818in}{2.842805in}}%
\pgfpathlineto{\pgfqpoint{3.938327in}{2.842805in}}%
\pgfpathlineto{\pgfqpoint{3.942836in}{2.827828in}}%
\pgfpathlineto{\pgfqpoint{3.947345in}{2.847798in}}%
\pgfpathlineto{\pgfqpoint{3.951855in}{2.782895in}}%
\pgfpathlineto{\pgfqpoint{3.956364in}{2.812850in}}%
\pgfpathlineto{\pgfqpoint{3.960873in}{2.857783in}}%
\pgfpathlineto{\pgfqpoint{3.965382in}{2.822835in}}%
\pgfpathlineto{\pgfqpoint{3.974400in}{2.897724in}}%
\pgfpathlineto{\pgfqpoint{3.983418in}{2.867768in}}%
\pgfpathlineto{\pgfqpoint{3.987927in}{2.877753in}}%
\pgfpathlineto{\pgfqpoint{3.992436in}{2.897724in}}%
\pgfpathlineto{\pgfqpoint{3.996945in}{2.932672in}}%
\pgfpathlineto{\pgfqpoint{4.001455in}{2.942657in}}%
\pgfpathlineto{\pgfqpoint{4.005964in}{2.887738in}}%
\pgfpathlineto{\pgfqpoint{4.010473in}{2.917694in}}%
\pgfpathlineto{\pgfqpoint{4.014982in}{2.927679in}}%
\pgfpathlineto{\pgfqpoint{4.019491in}{2.897724in}}%
\pgfpathlineto{\pgfqpoint{4.024000in}{2.972612in}}%
\pgfpathlineto{\pgfqpoint{4.028509in}{2.927679in}}%
\pgfpathlineto{\pgfqpoint{4.033018in}{2.927679in}}%
\pgfpathlineto{\pgfqpoint{4.037527in}{2.982597in}}%
\pgfpathlineto{\pgfqpoint{4.042036in}{2.947649in}}%
\pgfpathlineto{\pgfqpoint{4.046545in}{2.892731in}}%
\pgfpathlineto{\pgfqpoint{4.051055in}{2.912701in}}%
\pgfpathlineto{\pgfqpoint{4.055564in}{2.912701in}}%
\pgfpathlineto{\pgfqpoint{4.060073in}{2.992582in}}%
\pgfpathlineto{\pgfqpoint{4.069091in}{2.932672in}}%
\pgfpathlineto{\pgfqpoint{4.073600in}{2.952642in}}%
\pgfpathlineto{\pgfqpoint{4.078109in}{2.952642in}}%
\pgfpathlineto{\pgfqpoint{4.082618in}{2.972612in}}%
\pgfpathlineto{\pgfqpoint{4.087127in}{2.972612in}}%
\pgfpathlineto{\pgfqpoint{4.091636in}{2.922686in}}%
\pgfpathlineto{\pgfqpoint{4.096145in}{2.982597in}}%
\pgfpathlineto{\pgfqpoint{4.100655in}{2.952642in}}%
\pgfpathlineto{\pgfqpoint{4.105164in}{2.987590in}}%
\pgfpathlineto{\pgfqpoint{4.109673in}{2.952642in}}%
\pgfpathlineto{\pgfqpoint{4.114182in}{2.982597in}}%
\pgfpathlineto{\pgfqpoint{4.118691in}{3.022538in}}%
\pgfpathlineto{\pgfqpoint{4.123200in}{2.937664in}}%
\pgfpathlineto{\pgfqpoint{4.127709in}{2.982597in}}%
\pgfpathlineto{\pgfqpoint{4.132218in}{2.987590in}}%
\pgfpathlineto{\pgfqpoint{4.136727in}{3.032523in}}%
\pgfpathlineto{\pgfqpoint{4.141236in}{3.062478in}}%
\pgfpathlineto{\pgfqpoint{4.145745in}{2.967620in}}%
\pgfpathlineto{\pgfqpoint{4.150255in}{3.072464in}}%
\pgfpathlineto{\pgfqpoint{4.154764in}{3.017545in}}%
\pgfpathlineto{\pgfqpoint{4.159273in}{3.047501in}}%
\pgfpathlineto{\pgfqpoint{4.163782in}{3.022538in}}%
\pgfpathlineto{\pgfqpoint{4.168291in}{3.007560in}}%
\pgfpathlineto{\pgfqpoint{4.172800in}{3.032523in}}%
\pgfpathlineto{\pgfqpoint{4.177309in}{2.992582in}}%
\pgfpathlineto{\pgfqpoint{4.181818in}{3.052493in}}%
\pgfpathlineto{\pgfqpoint{4.186327in}{3.037516in}}%
\pgfpathlineto{\pgfqpoint{4.190836in}{3.067471in}}%
\pgfpathlineto{\pgfqpoint{4.195345in}{3.052493in}}%
\pgfpathlineto{\pgfqpoint{4.199855in}{3.042508in}}%
\pgfpathlineto{\pgfqpoint{4.204364in}{3.057486in}}%
\pgfpathlineto{\pgfqpoint{4.208873in}{3.067471in}}%
\pgfpathlineto{\pgfqpoint{4.213382in}{3.087441in}}%
\pgfpathlineto{\pgfqpoint{4.222400in}{3.032523in}}%
\pgfpathlineto{\pgfqpoint{4.226909in}{3.102419in}}%
\pgfpathlineto{\pgfqpoint{4.231418in}{3.067471in}}%
\pgfpathlineto{\pgfqpoint{4.235927in}{3.002568in}}%
\pgfpathlineto{\pgfqpoint{4.240436in}{3.132374in}}%
\pgfpathlineto{\pgfqpoint{4.244945in}{3.037516in}}%
\pgfpathlineto{\pgfqpoint{4.249455in}{3.072464in}}%
\pgfpathlineto{\pgfqpoint{4.253964in}{3.027530in}}%
\pgfpathlineto{\pgfqpoint{4.258473in}{3.117397in}}%
\pgfpathlineto{\pgfqpoint{4.262982in}{3.107412in}}%
\pgfpathlineto{\pgfqpoint{4.267491in}{3.062478in}}%
\pgfpathlineto{\pgfqpoint{4.272000in}{3.097426in}}%
\pgfpathlineto{\pgfqpoint{4.276509in}{3.037516in}}%
\pgfpathlineto{\pgfqpoint{4.281018in}{3.077456in}}%
\pgfpathlineto{\pgfqpoint{4.285527in}{3.127382in}}%
\pgfpathlineto{\pgfqpoint{4.290036in}{3.102419in}}%
\pgfpathlineto{\pgfqpoint{4.294545in}{3.162330in}}%
\pgfpathlineto{\pgfqpoint{4.299055in}{3.117397in}}%
\pgfpathlineto{\pgfqpoint{4.303564in}{3.092434in}}%
\pgfpathlineto{\pgfqpoint{4.308073in}{3.157337in}}%
\pgfpathlineto{\pgfqpoint{4.312582in}{3.142360in}}%
\pgfpathlineto{\pgfqpoint{4.317091in}{3.117397in}}%
\pgfpathlineto{\pgfqpoint{4.321600in}{3.127382in}}%
\pgfpathlineto{\pgfqpoint{4.326109in}{3.087441in}}%
\pgfpathlineto{\pgfqpoint{4.330618in}{3.202270in}}%
\pgfpathlineto{\pgfqpoint{4.335127in}{3.127382in}}%
\pgfpathlineto{\pgfqpoint{4.339636in}{3.167322in}}%
\pgfpathlineto{\pgfqpoint{4.344145in}{3.152345in}}%
\pgfpathlineto{\pgfqpoint{4.348655in}{3.182300in}}%
\pgfpathlineto{\pgfqpoint{4.353164in}{3.172315in}}%
\pgfpathlineto{\pgfqpoint{4.362182in}{3.192285in}}%
\pgfpathlineto{\pgfqpoint{4.366691in}{3.257189in}}%
\pgfpathlineto{\pgfqpoint{4.371200in}{3.172315in}}%
\pgfpathlineto{\pgfqpoint{4.375709in}{3.207263in}}%
\pgfpathlineto{\pgfqpoint{4.380218in}{3.172315in}}%
\pgfpathlineto{\pgfqpoint{4.384727in}{3.207263in}}%
\pgfpathlineto{\pgfqpoint{4.389236in}{3.162330in}}%
\pgfpathlineto{\pgfqpoint{4.393745in}{3.147352in}}%
\pgfpathlineto{\pgfqpoint{4.398255in}{3.212256in}}%
\pgfpathlineto{\pgfqpoint{4.402764in}{3.202270in}}%
\pgfpathlineto{\pgfqpoint{4.407273in}{3.172315in}}%
\pgfpathlineto{\pgfqpoint{4.411782in}{3.197278in}}%
\pgfpathlineto{\pgfqpoint{4.416291in}{3.237218in}}%
\pgfpathlineto{\pgfqpoint{4.420800in}{3.172315in}}%
\pgfpathlineto{\pgfqpoint{4.425309in}{3.202270in}}%
\pgfpathlineto{\pgfqpoint{4.429818in}{3.242211in}}%
\pgfpathlineto{\pgfqpoint{4.434327in}{3.137367in}}%
\pgfpathlineto{\pgfqpoint{4.438836in}{3.227233in}}%
\pgfpathlineto{\pgfqpoint{4.443345in}{3.202270in}}%
\pgfpathlineto{\pgfqpoint{4.447855in}{3.222241in}}%
\pgfpathlineto{\pgfqpoint{4.452364in}{3.197278in}}%
\pgfpathlineto{\pgfqpoint{4.456873in}{3.242211in}}%
\pgfpathlineto{\pgfqpoint{4.461382in}{3.237218in}}%
\pgfpathlineto{\pgfqpoint{4.465891in}{3.242211in}}%
\pgfpathlineto{\pgfqpoint{4.470400in}{3.277159in}}%
\pgfpathlineto{\pgfqpoint{4.474909in}{3.212256in}}%
\pgfpathlineto{\pgfqpoint{4.479418in}{3.247204in}}%
\pgfpathlineto{\pgfqpoint{4.483927in}{3.212256in}}%
\pgfpathlineto{\pgfqpoint{4.488436in}{3.262181in}}%
\pgfpathlineto{\pgfqpoint{4.492945in}{3.242211in}}%
\pgfpathlineto{\pgfqpoint{4.497455in}{3.292137in}}%
\pgfpathlineto{\pgfqpoint{4.501964in}{3.312107in}}%
\pgfpathlineto{\pgfqpoint{4.506473in}{3.247204in}}%
\pgfpathlineto{\pgfqpoint{4.510982in}{3.252196in}}%
\pgfpathlineto{\pgfqpoint{4.515491in}{3.222241in}}%
\pgfpathlineto{\pgfqpoint{4.520000in}{3.312107in}}%
\pgfpathlineto{\pgfqpoint{4.524509in}{3.332077in}}%
\pgfpathlineto{\pgfqpoint{4.529018in}{3.312107in}}%
\pgfpathlineto{\pgfqpoint{4.533527in}{3.272166in}}%
\pgfpathlineto{\pgfqpoint{4.538036in}{3.272166in}}%
\pgfpathlineto{\pgfqpoint{4.542545in}{3.262181in}}%
\pgfpathlineto{\pgfqpoint{4.551564in}{3.327085in}}%
\pgfpathlineto{\pgfqpoint{4.560582in}{3.292137in}}%
\pgfpathlineto{\pgfqpoint{4.565091in}{3.312107in}}%
\pgfpathlineto{\pgfqpoint{4.569600in}{3.357040in}}%
\pgfpathlineto{\pgfqpoint{4.574109in}{3.347055in}}%
\pgfpathlineto{\pgfqpoint{4.578618in}{3.382003in}}%
\pgfpathlineto{\pgfqpoint{4.583127in}{3.347055in}}%
\pgfpathlineto{\pgfqpoint{4.587636in}{3.337070in}}%
\pgfpathlineto{\pgfqpoint{4.592145in}{3.322092in}}%
\pgfpathlineto{\pgfqpoint{4.601164in}{3.372018in}}%
\pgfpathlineto{\pgfqpoint{4.605673in}{3.272166in}}%
\pgfpathlineto{\pgfqpoint{4.610182in}{3.302122in}}%
\pgfpathlineto{\pgfqpoint{4.614691in}{3.352048in}}%
\pgfpathlineto{\pgfqpoint{4.619200in}{3.362033in}}%
\pgfpathlineto{\pgfqpoint{4.623709in}{3.352048in}}%
\pgfpathlineto{\pgfqpoint{4.628218in}{3.362033in}}%
\pgfpathlineto{\pgfqpoint{4.632727in}{3.367025in}}%
\pgfpathlineto{\pgfqpoint{4.637236in}{3.416951in}}%
\pgfpathlineto{\pgfqpoint{4.641745in}{3.401973in}}%
\pgfpathlineto{\pgfqpoint{4.646255in}{3.441914in}}%
\pgfpathlineto{\pgfqpoint{4.650764in}{3.386996in}}%
\pgfpathlineto{\pgfqpoint{4.655273in}{3.377010in}}%
\pgfpathlineto{\pgfqpoint{4.659782in}{3.421944in}}%
\pgfpathlineto{\pgfqpoint{4.664291in}{3.362033in}}%
\pgfpathlineto{\pgfqpoint{4.668800in}{3.367025in}}%
\pgfpathlineto{\pgfqpoint{4.673309in}{3.357040in}}%
\pgfpathlineto{\pgfqpoint{4.682327in}{3.446906in}}%
\pgfpathlineto{\pgfqpoint{4.686836in}{3.436921in}}%
\pgfpathlineto{\pgfqpoint{4.691345in}{3.377010in}}%
\pgfpathlineto{\pgfqpoint{4.695855in}{3.406966in}}%
\pgfpathlineto{\pgfqpoint{4.700364in}{3.411958in}}%
\pgfpathlineto{\pgfqpoint{4.704873in}{3.401973in}}%
\pgfpathlineto{\pgfqpoint{4.709382in}{3.431929in}}%
\pgfpathlineto{\pgfqpoint{4.713891in}{3.411958in}}%
\pgfpathlineto{\pgfqpoint{4.722909in}{3.446906in}}%
\pgfpathlineto{\pgfqpoint{4.727418in}{3.456892in}}%
\pgfpathlineto{\pgfqpoint{4.731927in}{3.426936in}}%
\pgfpathlineto{\pgfqpoint{4.740945in}{3.436921in}}%
\pgfpathlineto{\pgfqpoint{4.745455in}{3.382003in}}%
\pgfpathlineto{\pgfqpoint{4.749964in}{3.451899in}}%
\pgfpathlineto{\pgfqpoint{4.754473in}{3.466877in}}%
\pgfpathlineto{\pgfqpoint{4.758982in}{3.461884in}}%
\pgfpathlineto{\pgfqpoint{4.763491in}{3.436921in}}%
\pgfpathlineto{\pgfqpoint{4.768000in}{3.491840in}}%
\pgfpathlineto{\pgfqpoint{4.772509in}{3.466877in}}%
\pgfpathlineto{\pgfqpoint{4.777018in}{3.491840in}}%
\pgfpathlineto{\pgfqpoint{4.781527in}{3.476862in}}%
\pgfpathlineto{\pgfqpoint{4.786036in}{3.466877in}}%
\pgfpathlineto{\pgfqpoint{4.790545in}{3.496832in}}%
\pgfpathlineto{\pgfqpoint{4.795055in}{3.476862in}}%
\pgfpathlineto{\pgfqpoint{4.799564in}{3.471869in}}%
\pgfpathlineto{\pgfqpoint{4.804073in}{3.531780in}}%
\pgfpathlineto{\pgfqpoint{4.808582in}{3.471869in}}%
\pgfpathlineto{\pgfqpoint{4.813091in}{3.436921in}}%
\pgfpathlineto{\pgfqpoint{4.817600in}{3.531780in}}%
\pgfpathlineto{\pgfqpoint{4.822109in}{3.421944in}}%
\pgfpathlineto{\pgfqpoint{4.831127in}{3.561736in}}%
\pgfpathlineto{\pgfqpoint{4.835636in}{3.551750in}}%
\pgfpathlineto{\pgfqpoint{4.840145in}{3.531780in}}%
\pgfpathlineto{\pgfqpoint{4.844655in}{3.556743in}}%
\pgfpathlineto{\pgfqpoint{4.849164in}{3.521795in}}%
\pgfpathlineto{\pgfqpoint{4.853673in}{3.496832in}}%
\pgfpathlineto{\pgfqpoint{4.858182in}{3.541765in}}%
\pgfpathlineto{\pgfqpoint{4.862691in}{3.511810in}}%
\pgfpathlineto{\pgfqpoint{4.867200in}{3.621646in}}%
\pgfpathlineto{\pgfqpoint{4.871709in}{3.586698in}}%
\pgfpathlineto{\pgfqpoint{4.876218in}{3.576713in}}%
\pgfpathlineto{\pgfqpoint{4.880727in}{3.536773in}}%
\pgfpathlineto{\pgfqpoint{4.885236in}{3.591691in}}%
\pgfpathlineto{\pgfqpoint{4.889745in}{3.601676in}}%
\pgfpathlineto{\pgfqpoint{4.894255in}{3.546758in}}%
\pgfpathlineto{\pgfqpoint{4.898764in}{3.581706in}}%
\pgfpathlineto{\pgfqpoint{4.903273in}{3.586698in}}%
\pgfpathlineto{\pgfqpoint{4.907782in}{3.611661in}}%
\pgfpathlineto{\pgfqpoint{4.912291in}{3.561736in}}%
\pgfpathlineto{\pgfqpoint{4.916800in}{3.566728in}}%
\pgfpathlineto{\pgfqpoint{4.921309in}{3.641617in}}%
\pgfpathlineto{\pgfqpoint{4.930327in}{3.581706in}}%
\pgfpathlineto{\pgfqpoint{4.934836in}{3.571721in}}%
\pgfpathlineto{\pgfqpoint{4.939345in}{3.611661in}}%
\pgfpathlineto{\pgfqpoint{4.943855in}{3.606669in}}%
\pgfpathlineto{\pgfqpoint{4.948364in}{3.641617in}}%
\pgfpathlineto{\pgfqpoint{4.952873in}{3.591691in}}%
\pgfpathlineto{\pgfqpoint{4.957382in}{3.616654in}}%
\pgfpathlineto{\pgfqpoint{4.961891in}{3.541765in}}%
\pgfpathlineto{\pgfqpoint{4.975418in}{3.611661in}}%
\pgfpathlineto{\pgfqpoint{4.979927in}{3.616654in}}%
\pgfpathlineto{\pgfqpoint{4.984436in}{3.606669in}}%
\pgfpathlineto{\pgfqpoint{4.988945in}{3.591691in}}%
\pgfpathlineto{\pgfqpoint{4.993455in}{3.681557in}}%
\pgfpathlineto{\pgfqpoint{4.997964in}{3.591691in}}%
\pgfpathlineto{\pgfqpoint{5.002473in}{3.671572in}}%
\pgfpathlineto{\pgfqpoint{5.006982in}{3.641617in}}%
\pgfpathlineto{\pgfqpoint{5.011491in}{3.626639in}}%
\pgfpathlineto{\pgfqpoint{5.016000in}{3.666579in}}%
\pgfpathlineto{\pgfqpoint{5.020509in}{3.676565in}}%
\pgfpathlineto{\pgfqpoint{5.025018in}{3.711513in}}%
\pgfpathlineto{\pgfqpoint{5.029527in}{3.656594in}}%
\pgfpathlineto{\pgfqpoint{5.034036in}{3.676565in}}%
\pgfpathlineto{\pgfqpoint{5.038545in}{3.626639in}}%
\pgfpathlineto{\pgfqpoint{5.043055in}{3.681557in}}%
\pgfpathlineto{\pgfqpoint{5.047564in}{3.686550in}}%
\pgfpathlineto{\pgfqpoint{5.052073in}{3.616654in}}%
\pgfpathlineto{\pgfqpoint{5.056582in}{3.656594in}}%
\pgfpathlineto{\pgfqpoint{5.061091in}{3.656594in}}%
\pgfpathlineto{\pgfqpoint{5.070109in}{3.741468in}}%
\pgfpathlineto{\pgfqpoint{5.074618in}{3.716505in}}%
\pgfpathlineto{\pgfqpoint{5.079127in}{3.671572in}}%
\pgfpathlineto{\pgfqpoint{5.083636in}{3.691542in}}%
\pgfpathlineto{\pgfqpoint{5.088145in}{3.696535in}}%
\pgfpathlineto{\pgfqpoint{5.092655in}{3.696535in}}%
\pgfpathlineto{\pgfqpoint{5.097164in}{3.701527in}}%
\pgfpathlineto{\pgfqpoint{5.101673in}{3.726490in}}%
\pgfpathlineto{\pgfqpoint{5.106182in}{3.731483in}}%
\pgfpathlineto{\pgfqpoint{5.115200in}{3.656594in}}%
\pgfpathlineto{\pgfqpoint{5.119709in}{3.756446in}}%
\pgfpathlineto{\pgfqpoint{5.124218in}{3.716505in}}%
\pgfpathlineto{\pgfqpoint{5.128727in}{3.746461in}}%
\pgfpathlineto{\pgfqpoint{5.133236in}{3.746461in}}%
\pgfpathlineto{\pgfqpoint{5.137745in}{3.761438in}}%
\pgfpathlineto{\pgfqpoint{5.142255in}{3.696535in}}%
\pgfpathlineto{\pgfqpoint{5.146764in}{3.781409in}}%
\pgfpathlineto{\pgfqpoint{5.151273in}{3.746461in}}%
\pgfpathlineto{\pgfqpoint{5.155782in}{3.771423in}}%
\pgfpathlineto{\pgfqpoint{5.160291in}{3.781409in}}%
\pgfpathlineto{\pgfqpoint{5.164800in}{3.696535in}}%
\pgfpathlineto{\pgfqpoint{5.169309in}{3.756446in}}%
\pgfpathlineto{\pgfqpoint{5.173818in}{3.751453in}}%
\pgfpathlineto{\pgfqpoint{5.178327in}{3.801379in}}%
\pgfpathlineto{\pgfqpoint{5.182836in}{3.791394in}}%
\pgfpathlineto{\pgfqpoint{5.187345in}{3.761438in}}%
\pgfpathlineto{\pgfqpoint{5.191855in}{3.811364in}}%
\pgfpathlineto{\pgfqpoint{5.196364in}{3.721498in}}%
\pgfpathlineto{\pgfqpoint{5.200873in}{3.726490in}}%
\pgfpathlineto{\pgfqpoint{5.205382in}{3.791394in}}%
\pgfpathlineto{\pgfqpoint{5.209891in}{3.766431in}}%
\pgfpathlineto{\pgfqpoint{5.214400in}{3.791394in}}%
\pgfpathlineto{\pgfqpoint{5.218909in}{3.831334in}}%
\pgfpathlineto{\pgfqpoint{5.223418in}{3.776416in}}%
\pgfpathlineto{\pgfqpoint{5.227927in}{3.766431in}}%
\pgfpathlineto{\pgfqpoint{5.232436in}{3.791394in}}%
\pgfpathlineto{\pgfqpoint{5.236945in}{3.826342in}}%
\pgfpathlineto{\pgfqpoint{5.241455in}{3.816357in}}%
\pgfpathlineto{\pgfqpoint{5.245964in}{3.821349in}}%
\pgfpathlineto{\pgfqpoint{5.250473in}{3.801379in}}%
\pgfpathlineto{\pgfqpoint{5.254982in}{3.811364in}}%
\pgfpathlineto{\pgfqpoint{5.259491in}{3.811364in}}%
\pgfpathlineto{\pgfqpoint{5.264000in}{3.776416in}}%
\pgfpathlineto{\pgfqpoint{5.268509in}{3.851305in}}%
\pgfpathlineto{\pgfqpoint{5.273018in}{3.836327in}}%
\pgfpathlineto{\pgfqpoint{5.277527in}{3.841319in}}%
\pgfpathlineto{\pgfqpoint{5.282036in}{3.871275in}}%
\pgfpathlineto{\pgfqpoint{5.286545in}{3.856297in}}%
\pgfpathlineto{\pgfqpoint{5.291055in}{3.781409in}}%
\pgfpathlineto{\pgfqpoint{5.295564in}{3.936178in}}%
\pgfpathlineto{\pgfqpoint{5.300073in}{3.931186in}}%
\pgfpathlineto{\pgfqpoint{5.304582in}{3.781409in}}%
\pgfpathlineto{\pgfqpoint{5.309091in}{3.896238in}}%
\pgfpathlineto{\pgfqpoint{5.313600in}{3.881260in}}%
\pgfpathlineto{\pgfqpoint{5.318109in}{3.831334in}}%
\pgfpathlineto{\pgfqpoint{5.322618in}{3.876267in}}%
\pgfpathlineto{\pgfqpoint{5.327127in}{3.861290in}}%
\pgfpathlineto{\pgfqpoint{5.331636in}{3.856297in}}%
\pgfpathlineto{\pgfqpoint{5.336145in}{3.926193in}}%
\pgfpathlineto{\pgfqpoint{5.340655in}{3.891245in}}%
\pgfpathlineto{\pgfqpoint{5.345164in}{3.931186in}}%
\pgfpathlineto{\pgfqpoint{5.349673in}{3.866282in}}%
\pgfpathlineto{\pgfqpoint{5.354182in}{3.926193in}}%
\pgfpathlineto{\pgfqpoint{5.358691in}{3.916208in}}%
\pgfpathlineto{\pgfqpoint{5.363200in}{3.891245in}}%
\pgfpathlineto{\pgfqpoint{5.367709in}{3.876267in}}%
\pgfpathlineto{\pgfqpoint{5.372218in}{3.916208in}}%
\pgfpathlineto{\pgfqpoint{5.376727in}{3.946163in}}%
\pgfpathlineto{\pgfqpoint{5.385745in}{3.866282in}}%
\pgfpathlineto{\pgfqpoint{5.390255in}{3.946163in}}%
\pgfpathlineto{\pgfqpoint{5.394764in}{3.921201in}}%
\pgfpathlineto{\pgfqpoint{5.399273in}{3.911215in}}%
\pgfpathlineto{\pgfqpoint{5.403782in}{3.951156in}}%
\pgfpathlineto{\pgfqpoint{5.408291in}{3.966134in}}%
\pgfpathlineto{\pgfqpoint{5.412800in}{3.941171in}}%
\pgfpathlineto{\pgfqpoint{5.417309in}{3.891245in}}%
\pgfpathlineto{\pgfqpoint{5.421818in}{3.921201in}}%
\pgfpathlineto{\pgfqpoint{5.430836in}{4.011067in}}%
\pgfpathlineto{\pgfqpoint{5.435345in}{4.001082in}}%
\pgfpathlineto{\pgfqpoint{5.439855in}{3.946163in}}%
\pgfpathlineto{\pgfqpoint{5.444364in}{3.941171in}}%
\pgfpathlineto{\pgfqpoint{5.448873in}{4.056000in}}%
\pgfpathlineto{\pgfqpoint{5.453382in}{3.976119in}}%
\pgfpathlineto{\pgfqpoint{5.457891in}{3.996089in}}%
\pgfpathlineto{\pgfqpoint{5.462400in}{3.936178in}}%
\pgfpathlineto{\pgfqpoint{5.466909in}{3.971126in}}%
\pgfpathlineto{\pgfqpoint{5.471418in}{4.031037in}}%
\pgfpathlineto{\pgfqpoint{5.475927in}{3.986104in}}%
\pgfpathlineto{\pgfqpoint{5.480436in}{4.026045in}}%
\pgfpathlineto{\pgfqpoint{5.484945in}{3.996089in}}%
\pgfpathlineto{\pgfqpoint{5.489455in}{4.021052in}}%
\pgfpathlineto{\pgfqpoint{5.493964in}{3.991097in}}%
\pgfpathlineto{\pgfqpoint{5.498473in}{3.991097in}}%
\pgfpathlineto{\pgfqpoint{5.502982in}{3.961141in}}%
\pgfpathlineto{\pgfqpoint{5.507491in}{3.991097in}}%
\pgfpathlineto{\pgfqpoint{5.512000in}{4.031037in}}%
\pgfpathlineto{\pgfqpoint{5.516509in}{4.026045in}}%
\pgfpathlineto{\pgfqpoint{5.521018in}{3.991097in}}%
\pgfpathlineto{\pgfqpoint{5.525527in}{3.971126in}}%
\pgfpathlineto{\pgfqpoint{5.530036in}{3.986104in}}%
\pgfpathlineto{\pgfqpoint{5.534545in}{3.971126in}}%
\pgfpathlineto{\pgfqpoint{5.534545in}{3.971126in}}%
\pgfusepath{stroke}%
\end{pgfscope}%
\begin{pgfscope}%
\pgfsetrectcap%
\pgfsetmiterjoin%
\pgfsetlinewidth{0.803000pt}%
\definecolor{currentstroke}{rgb}{0.000000,0.000000,0.000000}%
\pgfsetstrokecolor{currentstroke}%
\pgfsetdash{}{0pt}%
\pgfpathmoveto{\pgfqpoint{0.800000in}{0.528000in}}%
\pgfpathlineto{\pgfqpoint{0.800000in}{4.224000in}}%
\pgfusepath{stroke}%
\end{pgfscope}%
\begin{pgfscope}%
\pgfsetrectcap%
\pgfsetmiterjoin%
\pgfsetlinewidth{0.803000pt}%
\definecolor{currentstroke}{rgb}{0.000000,0.000000,0.000000}%
\pgfsetstrokecolor{currentstroke}%
\pgfsetdash{}{0pt}%
\pgfpathmoveto{\pgfqpoint{5.760000in}{0.528000in}}%
\pgfpathlineto{\pgfqpoint{5.760000in}{4.224000in}}%
\pgfusepath{stroke}%
\end{pgfscope}%
\begin{pgfscope}%
\pgfsetrectcap%
\pgfsetmiterjoin%
\pgfsetlinewidth{0.803000pt}%
\definecolor{currentstroke}{rgb}{0.000000,0.000000,0.000000}%
\pgfsetstrokecolor{currentstroke}%
\pgfsetdash{}{0pt}%
\pgfpathmoveto{\pgfqpoint{0.800000in}{0.528000in}}%
\pgfpathlineto{\pgfqpoint{5.760000in}{0.528000in}}%
\pgfusepath{stroke}%
\end{pgfscope}%
\begin{pgfscope}%
\pgfsetrectcap%
\pgfsetmiterjoin%
\pgfsetlinewidth{0.803000pt}%
\definecolor{currentstroke}{rgb}{0.000000,0.000000,0.000000}%
\pgfsetstrokecolor{currentstroke}%
\pgfsetdash{}{0pt}%
\pgfpathmoveto{\pgfqpoint{0.800000in}{4.224000in}}%
\pgfpathlineto{\pgfqpoint{5.760000in}{4.224000in}}%
\pgfusepath{stroke}%
\end{pgfscope}%
\begin{pgfscope}%
\definecolor{textcolor}{rgb}{0.000000,0.000000,0.000000}%
\pgfsetstrokecolor{textcolor}%
\pgfsetfillcolor{textcolor}%
\pgftext[x=3.280000in,y=4.307333in,,base]{\color{textcolor}\ttfamily\fontsize{12.000000}{14.400000}\selectfont Quick Sort Iterations vs Input size}%
\end{pgfscope}%
\begin{pgfscope}%
\pgfsetbuttcap%
\pgfsetmiterjoin%
\definecolor{currentfill}{rgb}{1.000000,1.000000,1.000000}%
\pgfsetfillcolor{currentfill}%
\pgfsetfillopacity{0.800000}%
\pgfsetlinewidth{1.003750pt}%
\definecolor{currentstroke}{rgb}{0.800000,0.800000,0.800000}%
\pgfsetstrokecolor{currentstroke}%
\pgfsetstrokeopacity{0.800000}%
\pgfsetdash{}{0pt}%
\pgfpathmoveto{\pgfqpoint{0.897222in}{3.908286in}}%
\pgfpathlineto{\pgfqpoint{1.759758in}{3.908286in}}%
\pgfpathquadraticcurveto{\pgfqpoint{1.787535in}{3.908286in}}{\pgfqpoint{1.787535in}{3.936063in}}%
\pgfpathlineto{\pgfqpoint{1.787535in}{4.126778in}}%
\pgfpathquadraticcurveto{\pgfqpoint{1.787535in}{4.154556in}}{\pgfqpoint{1.759758in}{4.154556in}}%
\pgfpathlineto{\pgfqpoint{0.897222in}{4.154556in}}%
\pgfpathquadraticcurveto{\pgfqpoint{0.869444in}{4.154556in}}{\pgfqpoint{0.869444in}{4.126778in}}%
\pgfpathlineto{\pgfqpoint{0.869444in}{3.936063in}}%
\pgfpathquadraticcurveto{\pgfqpoint{0.869444in}{3.908286in}}{\pgfqpoint{0.897222in}{3.908286in}}%
\pgfpathlineto{\pgfqpoint{0.897222in}{3.908286in}}%
\pgfpathclose%
\pgfusepath{stroke,fill}%
\end{pgfscope}%
\begin{pgfscope}%
\pgfsetrectcap%
\pgfsetroundjoin%
\pgfsetlinewidth{1.505625pt}%
\definecolor{currentstroke}{rgb}{0.000000,1.000000,0.498039}%
\pgfsetstrokecolor{currentstroke}%
\pgfsetdash{}{0pt}%
\pgfpathmoveto{\pgfqpoint{0.925000in}{4.041342in}}%
\pgfpathlineto{\pgfqpoint{1.063889in}{4.041342in}}%
\pgfpathlineto{\pgfqpoint{1.202778in}{4.041342in}}%
\pgfusepath{stroke}%
\end{pgfscope}%
\begin{pgfscope}%
\definecolor{textcolor}{rgb}{0.000000,0.000000,0.000000}%
\pgfsetstrokecolor{textcolor}%
\pgfsetfillcolor{textcolor}%
\pgftext[x=1.313889in,y=3.992731in,left,base]{\color{textcolor}\ttfamily\fontsize{10.000000}{12.000000}\selectfont Quick}%
\end{pgfscope}%
\end{pgfpicture}%
\makeatother%
\endgroup%

\subsubsection*{Insights}
The best and average case time complexity is $O(n \log n)$ and the worst case
complexity is $O(n2)$.
\section{Comparison}
%% Creator: Matplotlib, PGF backend
%%
%% To include the figure in your LaTeX document, write
%%   \input{<filename>.pgf}
%%
%% Make sure the required packages are loaded in your preamble
%%   \usepackage{pgf}
%%
%% Also ensure that all the required font packages are loaded; for instance,
%% the lmodern package is sometimes necessary when using math font.
%%   \usepackage{lmodern}
%%
%% Figures using additional raster images can only be included by \input if
%% they are in the same directory as the main LaTeX file. For loading figures
%% from other directories you can use the `import` package
%%   \usepackage{import}
%%
%% and then include the figures with
%%   \import{<path to file>}{<filename>.pgf}
%%
%% Matplotlib used the following preamble
%%   \usepackage{fontspec}
%%   \setmainfont{DejaVuSerif.ttf}[Path=\detokenize{/home/dbk/.local/lib/python3.10/site-packages/matplotlib/mpl-data/fonts/ttf/}]
%%   \setsansfont{DejaVuSans.ttf}[Path=\detokenize{/home/dbk/.local/lib/python3.10/site-packages/matplotlib/mpl-data/fonts/ttf/}]
%%   \setmonofont{DejaVuSansMono.ttf}[Path=\detokenize{/home/dbk/.local/lib/python3.10/site-packages/matplotlib/mpl-data/fonts/ttf/}]
%%
\begingroup%
\makeatletter%
\begin{pgfpicture}%
\pgfpathrectangle{\pgfpointorigin}{\pgfqpoint{6.400000in}{4.800000in}}%
\pgfusepath{use as bounding box, clip}%
\begin{pgfscope}%
\pgfsetbuttcap%
\pgfsetmiterjoin%
\definecolor{currentfill}{rgb}{1.000000,1.000000,1.000000}%
\pgfsetfillcolor{currentfill}%
\pgfsetlinewidth{0.000000pt}%
\definecolor{currentstroke}{rgb}{1.000000,1.000000,1.000000}%
\pgfsetstrokecolor{currentstroke}%
\pgfsetdash{}{0pt}%
\pgfpathmoveto{\pgfqpoint{0.000000in}{0.000000in}}%
\pgfpathlineto{\pgfqpoint{6.400000in}{0.000000in}}%
\pgfpathlineto{\pgfqpoint{6.400000in}{4.800000in}}%
\pgfpathlineto{\pgfqpoint{0.000000in}{4.800000in}}%
\pgfpathlineto{\pgfqpoint{0.000000in}{0.000000in}}%
\pgfpathclose%
\pgfusepath{fill}%
\end{pgfscope}%
\begin{pgfscope}%
\pgfsetbuttcap%
\pgfsetmiterjoin%
\definecolor{currentfill}{rgb}{1.000000,1.000000,1.000000}%
\pgfsetfillcolor{currentfill}%
\pgfsetlinewidth{0.000000pt}%
\definecolor{currentstroke}{rgb}{0.000000,0.000000,0.000000}%
\pgfsetstrokecolor{currentstroke}%
\pgfsetstrokeopacity{0.000000}%
\pgfsetdash{}{0pt}%
\pgfpathmoveto{\pgfqpoint{0.800000in}{0.528000in}}%
\pgfpathlineto{\pgfqpoint{5.760000in}{0.528000in}}%
\pgfpathlineto{\pgfqpoint{5.760000in}{4.224000in}}%
\pgfpathlineto{\pgfqpoint{0.800000in}{4.224000in}}%
\pgfpathlineto{\pgfqpoint{0.800000in}{0.528000in}}%
\pgfpathclose%
\pgfusepath{fill}%
\end{pgfscope}%
\begin{pgfscope}%
\pgfsetbuttcap%
\pgfsetroundjoin%
\definecolor{currentfill}{rgb}{0.000000,0.000000,0.000000}%
\pgfsetfillcolor{currentfill}%
\pgfsetlinewidth{0.803000pt}%
\definecolor{currentstroke}{rgb}{0.000000,0.000000,0.000000}%
\pgfsetstrokecolor{currentstroke}%
\pgfsetdash{}{0pt}%
\pgfsys@defobject{currentmarker}{\pgfqpoint{0.000000in}{-0.048611in}}{\pgfqpoint{0.000000in}{0.000000in}}{%
\pgfpathmoveto{\pgfqpoint{0.000000in}{0.000000in}}%
\pgfpathlineto{\pgfqpoint{0.000000in}{-0.048611in}}%
\pgfusepath{stroke,fill}%
}%
\begin{pgfscope}%
\pgfsys@transformshift{1.020945in}{0.528000in}%
\pgfsys@useobject{currentmarker}{}%
\end{pgfscope}%
\end{pgfscope}%
\begin{pgfscope}%
\definecolor{textcolor}{rgb}{0.000000,0.000000,0.000000}%
\pgfsetstrokecolor{textcolor}%
\pgfsetfillcolor{textcolor}%
\pgftext[x=1.020945in,y=0.430778in,,top]{\color{textcolor}\ttfamily\fontsize{10.000000}{12.000000}\selectfont 0}%
\end{pgfscope}%
\begin{pgfscope}%
\pgfsetbuttcap%
\pgfsetroundjoin%
\definecolor{currentfill}{rgb}{0.000000,0.000000,0.000000}%
\pgfsetfillcolor{currentfill}%
\pgfsetlinewidth{0.803000pt}%
\definecolor{currentstroke}{rgb}{0.000000,0.000000,0.000000}%
\pgfsetstrokecolor{currentstroke}%
\pgfsetdash{}{0pt}%
\pgfsys@defobject{currentmarker}{\pgfqpoint{0.000000in}{-0.048611in}}{\pgfqpoint{0.000000in}{0.000000in}}{%
\pgfpathmoveto{\pgfqpoint{0.000000in}{0.000000in}}%
\pgfpathlineto{\pgfqpoint{0.000000in}{-0.048611in}}%
\pgfusepath{stroke,fill}%
}%
\begin{pgfscope}%
\pgfsys@transformshift{1.922764in}{0.528000in}%
\pgfsys@useobject{currentmarker}{}%
\end{pgfscope}%
\end{pgfscope}%
\begin{pgfscope}%
\definecolor{textcolor}{rgb}{0.000000,0.000000,0.000000}%
\pgfsetstrokecolor{textcolor}%
\pgfsetfillcolor{textcolor}%
\pgftext[x=1.922764in,y=0.430778in,,top]{\color{textcolor}\ttfamily\fontsize{10.000000}{12.000000}\selectfont 200}%
\end{pgfscope}%
\begin{pgfscope}%
\pgfsetbuttcap%
\pgfsetroundjoin%
\definecolor{currentfill}{rgb}{0.000000,0.000000,0.000000}%
\pgfsetfillcolor{currentfill}%
\pgfsetlinewidth{0.803000pt}%
\definecolor{currentstroke}{rgb}{0.000000,0.000000,0.000000}%
\pgfsetstrokecolor{currentstroke}%
\pgfsetdash{}{0pt}%
\pgfsys@defobject{currentmarker}{\pgfqpoint{0.000000in}{-0.048611in}}{\pgfqpoint{0.000000in}{0.000000in}}{%
\pgfpathmoveto{\pgfqpoint{0.000000in}{0.000000in}}%
\pgfpathlineto{\pgfqpoint{0.000000in}{-0.048611in}}%
\pgfusepath{stroke,fill}%
}%
\begin{pgfscope}%
\pgfsys@transformshift{2.824582in}{0.528000in}%
\pgfsys@useobject{currentmarker}{}%
\end{pgfscope}%
\end{pgfscope}%
\begin{pgfscope}%
\definecolor{textcolor}{rgb}{0.000000,0.000000,0.000000}%
\pgfsetstrokecolor{textcolor}%
\pgfsetfillcolor{textcolor}%
\pgftext[x=2.824582in,y=0.430778in,,top]{\color{textcolor}\ttfamily\fontsize{10.000000}{12.000000}\selectfont 400}%
\end{pgfscope}%
\begin{pgfscope}%
\pgfsetbuttcap%
\pgfsetroundjoin%
\definecolor{currentfill}{rgb}{0.000000,0.000000,0.000000}%
\pgfsetfillcolor{currentfill}%
\pgfsetlinewidth{0.803000pt}%
\definecolor{currentstroke}{rgb}{0.000000,0.000000,0.000000}%
\pgfsetstrokecolor{currentstroke}%
\pgfsetdash{}{0pt}%
\pgfsys@defobject{currentmarker}{\pgfqpoint{0.000000in}{-0.048611in}}{\pgfqpoint{0.000000in}{0.000000in}}{%
\pgfpathmoveto{\pgfqpoint{0.000000in}{0.000000in}}%
\pgfpathlineto{\pgfqpoint{0.000000in}{-0.048611in}}%
\pgfusepath{stroke,fill}%
}%
\begin{pgfscope}%
\pgfsys@transformshift{3.726400in}{0.528000in}%
\pgfsys@useobject{currentmarker}{}%
\end{pgfscope}%
\end{pgfscope}%
\begin{pgfscope}%
\definecolor{textcolor}{rgb}{0.000000,0.000000,0.000000}%
\pgfsetstrokecolor{textcolor}%
\pgfsetfillcolor{textcolor}%
\pgftext[x=3.726400in,y=0.430778in,,top]{\color{textcolor}\ttfamily\fontsize{10.000000}{12.000000}\selectfont 600}%
\end{pgfscope}%
\begin{pgfscope}%
\pgfsetbuttcap%
\pgfsetroundjoin%
\definecolor{currentfill}{rgb}{0.000000,0.000000,0.000000}%
\pgfsetfillcolor{currentfill}%
\pgfsetlinewidth{0.803000pt}%
\definecolor{currentstroke}{rgb}{0.000000,0.000000,0.000000}%
\pgfsetstrokecolor{currentstroke}%
\pgfsetdash{}{0pt}%
\pgfsys@defobject{currentmarker}{\pgfqpoint{0.000000in}{-0.048611in}}{\pgfqpoint{0.000000in}{0.000000in}}{%
\pgfpathmoveto{\pgfqpoint{0.000000in}{0.000000in}}%
\pgfpathlineto{\pgfqpoint{0.000000in}{-0.048611in}}%
\pgfusepath{stroke,fill}%
}%
\begin{pgfscope}%
\pgfsys@transformshift{4.628218in}{0.528000in}%
\pgfsys@useobject{currentmarker}{}%
\end{pgfscope}%
\end{pgfscope}%
\begin{pgfscope}%
\definecolor{textcolor}{rgb}{0.000000,0.000000,0.000000}%
\pgfsetstrokecolor{textcolor}%
\pgfsetfillcolor{textcolor}%
\pgftext[x=4.628218in,y=0.430778in,,top]{\color{textcolor}\ttfamily\fontsize{10.000000}{12.000000}\selectfont 800}%
\end{pgfscope}%
\begin{pgfscope}%
\pgfsetbuttcap%
\pgfsetroundjoin%
\definecolor{currentfill}{rgb}{0.000000,0.000000,0.000000}%
\pgfsetfillcolor{currentfill}%
\pgfsetlinewidth{0.803000pt}%
\definecolor{currentstroke}{rgb}{0.000000,0.000000,0.000000}%
\pgfsetstrokecolor{currentstroke}%
\pgfsetdash{}{0pt}%
\pgfsys@defobject{currentmarker}{\pgfqpoint{0.000000in}{-0.048611in}}{\pgfqpoint{0.000000in}{0.000000in}}{%
\pgfpathmoveto{\pgfqpoint{0.000000in}{0.000000in}}%
\pgfpathlineto{\pgfqpoint{0.000000in}{-0.048611in}}%
\pgfusepath{stroke,fill}%
}%
\begin{pgfscope}%
\pgfsys@transformshift{5.530036in}{0.528000in}%
\pgfsys@useobject{currentmarker}{}%
\end{pgfscope}%
\end{pgfscope}%
\begin{pgfscope}%
\definecolor{textcolor}{rgb}{0.000000,0.000000,0.000000}%
\pgfsetstrokecolor{textcolor}%
\pgfsetfillcolor{textcolor}%
\pgftext[x=5.530036in,y=0.430778in,,top]{\color{textcolor}\ttfamily\fontsize{10.000000}{12.000000}\selectfont 1000}%
\end{pgfscope}%
\begin{pgfscope}%
\definecolor{textcolor}{rgb}{0.000000,0.000000,0.000000}%
\pgfsetstrokecolor{textcolor}%
\pgfsetfillcolor{textcolor}%
\pgftext[x=3.280000in,y=0.240063in,,top]{\color{textcolor}\ttfamily\fontsize{10.000000}{12.000000}\selectfont Size of Array}%
\end{pgfscope}%
\begin{pgfscope}%
\pgfsetbuttcap%
\pgfsetroundjoin%
\definecolor{currentfill}{rgb}{0.000000,0.000000,0.000000}%
\pgfsetfillcolor{currentfill}%
\pgfsetlinewidth{0.803000pt}%
\definecolor{currentstroke}{rgb}{0.000000,0.000000,0.000000}%
\pgfsetstrokecolor{currentstroke}%
\pgfsetdash{}{0pt}%
\pgfsys@defobject{currentmarker}{\pgfqpoint{-0.048611in}{0.000000in}}{\pgfqpoint{-0.000000in}{0.000000in}}{%
\pgfpathmoveto{\pgfqpoint{-0.000000in}{0.000000in}}%
\pgfpathlineto{\pgfqpoint{-0.048611in}{0.000000in}}%
\pgfusepath{stroke,fill}%
}%
\begin{pgfscope}%
\pgfsys@transformshift{0.800000in}{0.694603in}%
\pgfsys@useobject{currentmarker}{}%
\end{pgfscope}%
\end{pgfscope}%
\begin{pgfscope}%
\definecolor{textcolor}{rgb}{0.000000,0.000000,0.000000}%
\pgfsetstrokecolor{textcolor}%
\pgfsetfillcolor{textcolor}%
\pgftext[x=0.368305in, y=0.641468in, left, base]{\color{textcolor}\ttfamily\fontsize{10.000000}{12.000000}\selectfont 0.00}%
\end{pgfscope}%
\begin{pgfscope}%
\pgfsetbuttcap%
\pgfsetroundjoin%
\definecolor{currentfill}{rgb}{0.000000,0.000000,0.000000}%
\pgfsetfillcolor{currentfill}%
\pgfsetlinewidth{0.803000pt}%
\definecolor{currentstroke}{rgb}{0.000000,0.000000,0.000000}%
\pgfsetstrokecolor{currentstroke}%
\pgfsetdash{}{0pt}%
\pgfsys@defobject{currentmarker}{\pgfqpoint{-0.048611in}{0.000000in}}{\pgfqpoint{-0.000000in}{0.000000in}}{%
\pgfpathmoveto{\pgfqpoint{-0.000000in}{0.000000in}}%
\pgfpathlineto{\pgfqpoint{-0.048611in}{0.000000in}}%
\pgfusepath{stroke,fill}%
}%
\begin{pgfscope}%
\pgfsys@transformshift{0.800000in}{1.181274in}%
\pgfsys@useobject{currentmarker}{}%
\end{pgfscope}%
\end{pgfscope}%
\begin{pgfscope}%
\definecolor{textcolor}{rgb}{0.000000,0.000000,0.000000}%
\pgfsetstrokecolor{textcolor}%
\pgfsetfillcolor{textcolor}%
\pgftext[x=0.368305in, y=1.128140in, left, base]{\color{textcolor}\ttfamily\fontsize{10.000000}{12.000000}\selectfont 0.25}%
\end{pgfscope}%
\begin{pgfscope}%
\pgfsetbuttcap%
\pgfsetroundjoin%
\definecolor{currentfill}{rgb}{0.000000,0.000000,0.000000}%
\pgfsetfillcolor{currentfill}%
\pgfsetlinewidth{0.803000pt}%
\definecolor{currentstroke}{rgb}{0.000000,0.000000,0.000000}%
\pgfsetstrokecolor{currentstroke}%
\pgfsetdash{}{0pt}%
\pgfsys@defobject{currentmarker}{\pgfqpoint{-0.048611in}{0.000000in}}{\pgfqpoint{-0.000000in}{0.000000in}}{%
\pgfpathmoveto{\pgfqpoint{-0.000000in}{0.000000in}}%
\pgfpathlineto{\pgfqpoint{-0.048611in}{0.000000in}}%
\pgfusepath{stroke,fill}%
}%
\begin{pgfscope}%
\pgfsys@transformshift{0.800000in}{1.667946in}%
\pgfsys@useobject{currentmarker}{}%
\end{pgfscope}%
\end{pgfscope}%
\begin{pgfscope}%
\definecolor{textcolor}{rgb}{0.000000,0.000000,0.000000}%
\pgfsetstrokecolor{textcolor}%
\pgfsetfillcolor{textcolor}%
\pgftext[x=0.368305in, y=1.614811in, left, base]{\color{textcolor}\ttfamily\fontsize{10.000000}{12.000000}\selectfont 0.50}%
\end{pgfscope}%
\begin{pgfscope}%
\pgfsetbuttcap%
\pgfsetroundjoin%
\definecolor{currentfill}{rgb}{0.000000,0.000000,0.000000}%
\pgfsetfillcolor{currentfill}%
\pgfsetlinewidth{0.803000pt}%
\definecolor{currentstroke}{rgb}{0.000000,0.000000,0.000000}%
\pgfsetstrokecolor{currentstroke}%
\pgfsetdash{}{0pt}%
\pgfsys@defobject{currentmarker}{\pgfqpoint{-0.048611in}{0.000000in}}{\pgfqpoint{-0.000000in}{0.000000in}}{%
\pgfpathmoveto{\pgfqpoint{-0.000000in}{0.000000in}}%
\pgfpathlineto{\pgfqpoint{-0.048611in}{0.000000in}}%
\pgfusepath{stroke,fill}%
}%
\begin{pgfscope}%
\pgfsys@transformshift{0.800000in}{2.154617in}%
\pgfsys@useobject{currentmarker}{}%
\end{pgfscope}%
\end{pgfscope}%
\begin{pgfscope}%
\definecolor{textcolor}{rgb}{0.000000,0.000000,0.000000}%
\pgfsetstrokecolor{textcolor}%
\pgfsetfillcolor{textcolor}%
\pgftext[x=0.368305in, y=2.101483in, left, base]{\color{textcolor}\ttfamily\fontsize{10.000000}{12.000000}\selectfont 0.75}%
\end{pgfscope}%
\begin{pgfscope}%
\pgfsetbuttcap%
\pgfsetroundjoin%
\definecolor{currentfill}{rgb}{0.000000,0.000000,0.000000}%
\pgfsetfillcolor{currentfill}%
\pgfsetlinewidth{0.803000pt}%
\definecolor{currentstroke}{rgb}{0.000000,0.000000,0.000000}%
\pgfsetstrokecolor{currentstroke}%
\pgfsetdash{}{0pt}%
\pgfsys@defobject{currentmarker}{\pgfqpoint{-0.048611in}{0.000000in}}{\pgfqpoint{-0.000000in}{0.000000in}}{%
\pgfpathmoveto{\pgfqpoint{-0.000000in}{0.000000in}}%
\pgfpathlineto{\pgfqpoint{-0.048611in}{0.000000in}}%
\pgfusepath{stroke,fill}%
}%
\begin{pgfscope}%
\pgfsys@transformshift{0.800000in}{2.641289in}%
\pgfsys@useobject{currentmarker}{}%
\end{pgfscope}%
\end{pgfscope}%
\begin{pgfscope}%
\definecolor{textcolor}{rgb}{0.000000,0.000000,0.000000}%
\pgfsetstrokecolor{textcolor}%
\pgfsetfillcolor{textcolor}%
\pgftext[x=0.368305in, y=2.588154in, left, base]{\color{textcolor}\ttfamily\fontsize{10.000000}{12.000000}\selectfont 1.00}%
\end{pgfscope}%
\begin{pgfscope}%
\pgfsetbuttcap%
\pgfsetroundjoin%
\definecolor{currentfill}{rgb}{0.000000,0.000000,0.000000}%
\pgfsetfillcolor{currentfill}%
\pgfsetlinewidth{0.803000pt}%
\definecolor{currentstroke}{rgb}{0.000000,0.000000,0.000000}%
\pgfsetstrokecolor{currentstroke}%
\pgfsetdash{}{0pt}%
\pgfsys@defobject{currentmarker}{\pgfqpoint{-0.048611in}{0.000000in}}{\pgfqpoint{-0.000000in}{0.000000in}}{%
\pgfpathmoveto{\pgfqpoint{-0.000000in}{0.000000in}}%
\pgfpathlineto{\pgfqpoint{-0.048611in}{0.000000in}}%
\pgfusepath{stroke,fill}%
}%
\begin{pgfscope}%
\pgfsys@transformshift{0.800000in}{3.127960in}%
\pgfsys@useobject{currentmarker}{}%
\end{pgfscope}%
\end{pgfscope}%
\begin{pgfscope}%
\definecolor{textcolor}{rgb}{0.000000,0.000000,0.000000}%
\pgfsetstrokecolor{textcolor}%
\pgfsetfillcolor{textcolor}%
\pgftext[x=0.368305in, y=3.074826in, left, base]{\color{textcolor}\ttfamily\fontsize{10.000000}{12.000000}\selectfont 1.25}%
\end{pgfscope}%
\begin{pgfscope}%
\pgfsetbuttcap%
\pgfsetroundjoin%
\definecolor{currentfill}{rgb}{0.000000,0.000000,0.000000}%
\pgfsetfillcolor{currentfill}%
\pgfsetlinewidth{0.803000pt}%
\definecolor{currentstroke}{rgb}{0.000000,0.000000,0.000000}%
\pgfsetstrokecolor{currentstroke}%
\pgfsetdash{}{0pt}%
\pgfsys@defobject{currentmarker}{\pgfqpoint{-0.048611in}{0.000000in}}{\pgfqpoint{-0.000000in}{0.000000in}}{%
\pgfpathmoveto{\pgfqpoint{-0.000000in}{0.000000in}}%
\pgfpathlineto{\pgfqpoint{-0.048611in}{0.000000in}}%
\pgfusepath{stroke,fill}%
}%
\begin{pgfscope}%
\pgfsys@transformshift{0.800000in}{3.614632in}%
\pgfsys@useobject{currentmarker}{}%
\end{pgfscope}%
\end{pgfscope}%
\begin{pgfscope}%
\definecolor{textcolor}{rgb}{0.000000,0.000000,0.000000}%
\pgfsetstrokecolor{textcolor}%
\pgfsetfillcolor{textcolor}%
\pgftext[x=0.368305in, y=3.561497in, left, base]{\color{textcolor}\ttfamily\fontsize{10.000000}{12.000000}\selectfont 1.50}%
\end{pgfscope}%
\begin{pgfscope}%
\pgfsetbuttcap%
\pgfsetroundjoin%
\definecolor{currentfill}{rgb}{0.000000,0.000000,0.000000}%
\pgfsetfillcolor{currentfill}%
\pgfsetlinewidth{0.803000pt}%
\definecolor{currentstroke}{rgb}{0.000000,0.000000,0.000000}%
\pgfsetstrokecolor{currentstroke}%
\pgfsetdash{}{0pt}%
\pgfsys@defobject{currentmarker}{\pgfqpoint{-0.048611in}{0.000000in}}{\pgfqpoint{-0.000000in}{0.000000in}}{%
\pgfpathmoveto{\pgfqpoint{-0.000000in}{0.000000in}}%
\pgfpathlineto{\pgfqpoint{-0.048611in}{0.000000in}}%
\pgfusepath{stroke,fill}%
}%
\begin{pgfscope}%
\pgfsys@transformshift{0.800000in}{4.101303in}%
\pgfsys@useobject{currentmarker}{}%
\end{pgfscope}%
\end{pgfscope}%
\begin{pgfscope}%
\definecolor{textcolor}{rgb}{0.000000,0.000000,0.000000}%
\pgfsetstrokecolor{textcolor}%
\pgfsetfillcolor{textcolor}%
\pgftext[x=0.368305in, y=4.048169in, left, base]{\color{textcolor}\ttfamily\fontsize{10.000000}{12.000000}\selectfont 1.75}%
\end{pgfscope}%
\begin{pgfscope}%
\definecolor{textcolor}{rgb}{0.000000,0.000000,0.000000}%
\pgfsetstrokecolor{textcolor}%
\pgfsetfillcolor{textcolor}%
\pgftext[x=0.312750in,y=2.376000in,,bottom,rotate=90.000000]{\color{textcolor}\ttfamily\fontsize{10.000000}{12.000000}\selectfont Time}%
\end{pgfscope}%
\begin{pgfscope}%
\definecolor{textcolor}{rgb}{0.000000,0.000000,0.000000}%
\pgfsetstrokecolor{textcolor}%
\pgfsetfillcolor{textcolor}%
\pgftext[x=0.800000in,y=4.265667in,left,base]{\color{textcolor}\ttfamily\fontsize{10.000000}{12.000000}\selectfont 1e9}%
\end{pgfscope}%
\begin{pgfscope}%
\pgfpathrectangle{\pgfqpoint{0.800000in}{0.528000in}}{\pgfqpoint{4.960000in}{3.696000in}}%
\pgfusepath{clip}%
\pgfsetrectcap%
\pgfsetroundjoin%
\pgfsetlinewidth{1.505625pt}%
\definecolor{currentstroke}{rgb}{1.000000,0.000000,0.000000}%
\pgfsetstrokecolor{currentstroke}%
\pgfsetdash{}{0pt}%
\pgfpathmoveto{\pgfqpoint{1.025455in}{0.705994in}}%
\pgfpathlineto{\pgfqpoint{1.029964in}{0.707555in}}%
\pgfpathlineto{\pgfqpoint{1.034473in}{0.711520in}}%
\pgfpathlineto{\pgfqpoint{1.038982in}{0.710393in}}%
\pgfpathlineto{\pgfqpoint{1.048000in}{0.710155in}}%
\pgfpathlineto{\pgfqpoint{1.052509in}{0.708055in}}%
\pgfpathlineto{\pgfqpoint{1.057018in}{0.708065in}}%
\pgfpathlineto{\pgfqpoint{1.061527in}{0.710139in}}%
\pgfpathlineto{\pgfqpoint{1.075055in}{0.710185in}}%
\pgfpathlineto{\pgfqpoint{1.079564in}{0.712372in}}%
\pgfpathlineto{\pgfqpoint{1.084073in}{0.710427in}}%
\pgfpathlineto{\pgfqpoint{1.093091in}{0.711727in}}%
\pgfpathlineto{\pgfqpoint{1.097600in}{0.714095in}}%
\pgfpathlineto{\pgfqpoint{1.102109in}{0.711154in}}%
\pgfpathlineto{\pgfqpoint{1.115636in}{0.715457in}}%
\pgfpathlineto{\pgfqpoint{1.120145in}{0.713994in}}%
\pgfpathlineto{\pgfqpoint{1.124655in}{0.714078in}}%
\pgfpathlineto{\pgfqpoint{1.129164in}{0.712918in}}%
\pgfpathlineto{\pgfqpoint{1.138182in}{0.718356in}}%
\pgfpathlineto{\pgfqpoint{1.142691in}{0.714331in}}%
\pgfpathlineto{\pgfqpoint{1.147200in}{0.718646in}}%
\pgfpathlineto{\pgfqpoint{1.151709in}{0.714883in}}%
\pgfpathlineto{\pgfqpoint{1.156218in}{0.718932in}}%
\pgfpathlineto{\pgfqpoint{1.160727in}{0.716608in}}%
\pgfpathlineto{\pgfqpoint{1.165236in}{0.718834in}}%
\pgfpathlineto{\pgfqpoint{1.169745in}{0.717317in}}%
\pgfpathlineto{\pgfqpoint{1.174255in}{0.718930in}}%
\pgfpathlineto{\pgfqpoint{1.178764in}{0.716561in}}%
\pgfpathlineto{\pgfqpoint{1.183273in}{0.718876in}}%
\pgfpathlineto{\pgfqpoint{1.187782in}{0.718163in}}%
\pgfpathlineto{\pgfqpoint{1.192291in}{0.720827in}}%
\pgfpathlineto{\pgfqpoint{1.196800in}{0.718497in}}%
\pgfpathlineto{\pgfqpoint{1.210327in}{0.721208in}}%
\pgfpathlineto{\pgfqpoint{1.223855in}{0.721005in}}%
\pgfpathlineto{\pgfqpoint{1.228364in}{0.734285in}}%
\pgfpathlineto{\pgfqpoint{1.232873in}{0.722072in}}%
\pgfpathlineto{\pgfqpoint{1.237382in}{0.723613in}}%
\pgfpathlineto{\pgfqpoint{1.246400in}{0.742208in}}%
\pgfpathlineto{\pgfqpoint{1.250909in}{0.727134in}}%
\pgfpathlineto{\pgfqpoint{1.255418in}{0.729291in}}%
\pgfpathlineto{\pgfqpoint{1.259927in}{0.725436in}}%
\pgfpathlineto{\pgfqpoint{1.264436in}{0.731073in}}%
\pgfpathlineto{\pgfqpoint{1.268945in}{0.733817in}}%
\pgfpathlineto{\pgfqpoint{1.273455in}{0.729445in}}%
\pgfpathlineto{\pgfqpoint{1.282473in}{0.729435in}}%
\pgfpathlineto{\pgfqpoint{1.291491in}{0.732971in}}%
\pgfpathlineto{\pgfqpoint{1.296000in}{0.731895in}}%
\pgfpathlineto{\pgfqpoint{1.300509in}{0.728986in}}%
\pgfpathlineto{\pgfqpoint{1.323055in}{0.730689in}}%
\pgfpathlineto{\pgfqpoint{1.327564in}{0.729674in}}%
\pgfpathlineto{\pgfqpoint{1.332073in}{0.732503in}}%
\pgfpathlineto{\pgfqpoint{1.336582in}{0.732688in}}%
\pgfpathlineto{\pgfqpoint{1.341091in}{0.753260in}}%
\pgfpathlineto{\pgfqpoint{1.345600in}{0.767060in}}%
\pgfpathlineto{\pgfqpoint{1.354618in}{0.752852in}}%
\pgfpathlineto{\pgfqpoint{1.359127in}{0.773272in}}%
\pgfpathlineto{\pgfqpoint{1.363636in}{0.783665in}}%
\pgfpathlineto{\pgfqpoint{1.368145in}{0.797576in}}%
\pgfpathlineto{\pgfqpoint{1.377164in}{0.763613in}}%
\pgfpathlineto{\pgfqpoint{1.381673in}{0.754399in}}%
\pgfpathlineto{\pgfqpoint{1.386182in}{0.761047in}}%
\pgfpathlineto{\pgfqpoint{1.390691in}{0.773839in}}%
\pgfpathlineto{\pgfqpoint{1.395200in}{0.795379in}}%
\pgfpathlineto{\pgfqpoint{1.399709in}{0.786600in}}%
\pgfpathlineto{\pgfqpoint{1.404218in}{0.783350in}}%
\pgfpathlineto{\pgfqpoint{1.408727in}{0.742536in}}%
\pgfpathlineto{\pgfqpoint{1.413236in}{0.750133in}}%
\pgfpathlineto{\pgfqpoint{1.417745in}{0.747654in}}%
\pgfpathlineto{\pgfqpoint{1.422255in}{0.741305in}}%
\pgfpathlineto{\pgfqpoint{1.431273in}{0.739814in}}%
\pgfpathlineto{\pgfqpoint{1.440291in}{0.743238in}}%
\pgfpathlineto{\pgfqpoint{1.453818in}{0.744351in}}%
\pgfpathlineto{\pgfqpoint{1.458327in}{0.744797in}}%
\pgfpathlineto{\pgfqpoint{1.462836in}{0.743825in}}%
\pgfpathlineto{\pgfqpoint{1.471855in}{0.745115in}}%
\pgfpathlineto{\pgfqpoint{1.476364in}{0.749041in}}%
\pgfpathlineto{\pgfqpoint{1.480873in}{0.749052in}}%
\pgfpathlineto{\pgfqpoint{1.485382in}{0.747618in}}%
\pgfpathlineto{\pgfqpoint{1.489891in}{0.748720in}}%
\pgfpathlineto{\pgfqpoint{1.494400in}{0.746384in}}%
\pgfpathlineto{\pgfqpoint{1.498909in}{0.751382in}}%
\pgfpathlineto{\pgfqpoint{1.503418in}{0.762120in}}%
\pgfpathlineto{\pgfqpoint{1.512436in}{0.850655in}}%
\pgfpathlineto{\pgfqpoint{1.516945in}{0.758987in}}%
\pgfpathlineto{\pgfqpoint{1.521455in}{0.750456in}}%
\pgfpathlineto{\pgfqpoint{1.525964in}{0.752490in}}%
\pgfpathlineto{\pgfqpoint{1.534982in}{0.752339in}}%
\pgfpathlineto{\pgfqpoint{1.539491in}{0.753856in}}%
\pgfpathlineto{\pgfqpoint{1.548509in}{0.768781in}}%
\pgfpathlineto{\pgfqpoint{1.553018in}{0.773555in}}%
\pgfpathlineto{\pgfqpoint{1.557527in}{0.762596in}}%
\pgfpathlineto{\pgfqpoint{1.562036in}{0.755849in}}%
\pgfpathlineto{\pgfqpoint{1.566545in}{0.758293in}}%
\pgfpathlineto{\pgfqpoint{1.571055in}{0.764089in}}%
\pgfpathlineto{\pgfqpoint{1.575564in}{0.767676in}}%
\pgfpathlineto{\pgfqpoint{1.580073in}{0.764969in}}%
\pgfpathlineto{\pgfqpoint{1.584582in}{0.766709in}}%
\pgfpathlineto{\pgfqpoint{1.589091in}{0.778980in}}%
\pgfpathlineto{\pgfqpoint{1.593600in}{0.826640in}}%
\pgfpathlineto{\pgfqpoint{1.598109in}{0.775100in}}%
\pgfpathlineto{\pgfqpoint{1.602618in}{0.830574in}}%
\pgfpathlineto{\pgfqpoint{1.607127in}{0.791318in}}%
\pgfpathlineto{\pgfqpoint{1.611636in}{0.813538in}}%
\pgfpathlineto{\pgfqpoint{1.616145in}{0.809893in}}%
\pgfpathlineto{\pgfqpoint{1.620655in}{0.866072in}}%
\pgfpathlineto{\pgfqpoint{1.625164in}{0.845864in}}%
\pgfpathlineto{\pgfqpoint{1.629673in}{0.858828in}}%
\pgfpathlineto{\pgfqpoint{1.634182in}{0.788241in}}%
\pgfpathlineto{\pgfqpoint{1.638691in}{0.766726in}}%
\pgfpathlineto{\pgfqpoint{1.643200in}{0.766909in}}%
\pgfpathlineto{\pgfqpoint{1.665745in}{0.776621in}}%
\pgfpathlineto{\pgfqpoint{1.670255in}{0.822181in}}%
\pgfpathlineto{\pgfqpoint{1.674764in}{0.854702in}}%
\pgfpathlineto{\pgfqpoint{1.679273in}{0.830752in}}%
\pgfpathlineto{\pgfqpoint{1.683782in}{0.832404in}}%
\pgfpathlineto{\pgfqpoint{1.688291in}{0.773467in}}%
\pgfpathlineto{\pgfqpoint{1.697309in}{0.783879in}}%
\pgfpathlineto{\pgfqpoint{1.701818in}{0.784847in}}%
\pgfpathlineto{\pgfqpoint{1.706327in}{0.801576in}}%
\pgfpathlineto{\pgfqpoint{1.710836in}{0.796071in}}%
\pgfpathlineto{\pgfqpoint{1.715345in}{0.810637in}}%
\pgfpathlineto{\pgfqpoint{1.719855in}{0.856389in}}%
\pgfpathlineto{\pgfqpoint{1.724364in}{0.779520in}}%
\pgfpathlineto{\pgfqpoint{1.728873in}{0.784032in}}%
\pgfpathlineto{\pgfqpoint{1.733382in}{0.810281in}}%
\pgfpathlineto{\pgfqpoint{1.737891in}{0.780403in}}%
\pgfpathlineto{\pgfqpoint{1.742400in}{0.790057in}}%
\pgfpathlineto{\pgfqpoint{1.746909in}{0.785346in}}%
\pgfpathlineto{\pgfqpoint{1.751418in}{0.789673in}}%
\pgfpathlineto{\pgfqpoint{1.755927in}{0.870721in}}%
\pgfpathlineto{\pgfqpoint{1.760436in}{0.798362in}}%
\pgfpathlineto{\pgfqpoint{1.764945in}{0.801110in}}%
\pgfpathlineto{\pgfqpoint{1.769455in}{0.805866in}}%
\pgfpathlineto{\pgfqpoint{1.773964in}{0.799178in}}%
\pgfpathlineto{\pgfqpoint{1.778473in}{0.802463in}}%
\pgfpathlineto{\pgfqpoint{1.787491in}{0.872765in}}%
\pgfpathlineto{\pgfqpoint{1.792000in}{0.842261in}}%
\pgfpathlineto{\pgfqpoint{1.796509in}{0.821171in}}%
\pgfpathlineto{\pgfqpoint{1.801018in}{0.837169in}}%
\pgfpathlineto{\pgfqpoint{1.805527in}{0.824915in}}%
\pgfpathlineto{\pgfqpoint{1.810036in}{0.799689in}}%
\pgfpathlineto{\pgfqpoint{1.814545in}{0.794706in}}%
\pgfpathlineto{\pgfqpoint{1.819055in}{0.807277in}}%
\pgfpathlineto{\pgfqpoint{1.823564in}{0.895925in}}%
\pgfpathlineto{\pgfqpoint{1.828073in}{0.805394in}}%
\pgfpathlineto{\pgfqpoint{1.832582in}{0.808247in}}%
\pgfpathlineto{\pgfqpoint{1.837091in}{0.838458in}}%
\pgfpathlineto{\pgfqpoint{1.841600in}{0.822728in}}%
\pgfpathlineto{\pgfqpoint{1.846109in}{0.797293in}}%
\pgfpathlineto{\pgfqpoint{1.850618in}{0.820534in}}%
\pgfpathlineto{\pgfqpoint{1.855127in}{0.927870in}}%
\pgfpathlineto{\pgfqpoint{1.859636in}{0.832995in}}%
\pgfpathlineto{\pgfqpoint{1.864145in}{0.800223in}}%
\pgfpathlineto{\pgfqpoint{1.873164in}{0.802273in}}%
\pgfpathlineto{\pgfqpoint{1.877673in}{0.801554in}}%
\pgfpathlineto{\pgfqpoint{1.882182in}{0.833188in}}%
\pgfpathlineto{\pgfqpoint{1.886691in}{0.910693in}}%
\pgfpathlineto{\pgfqpoint{1.891200in}{0.803430in}}%
\pgfpathlineto{\pgfqpoint{1.895709in}{0.803865in}}%
\pgfpathlineto{\pgfqpoint{1.900218in}{0.806322in}}%
\pgfpathlineto{\pgfqpoint{1.904727in}{0.958370in}}%
\pgfpathlineto{\pgfqpoint{1.909236in}{0.842010in}}%
\pgfpathlineto{\pgfqpoint{1.913745in}{0.815112in}}%
\pgfpathlineto{\pgfqpoint{1.922764in}{0.810026in}}%
\pgfpathlineto{\pgfqpoint{1.927273in}{0.812727in}}%
\pgfpathlineto{\pgfqpoint{1.931782in}{0.814104in}}%
\pgfpathlineto{\pgfqpoint{1.936291in}{0.818258in}}%
\pgfpathlineto{\pgfqpoint{1.940800in}{0.812877in}}%
\pgfpathlineto{\pgfqpoint{1.945309in}{0.817801in}}%
\pgfpathlineto{\pgfqpoint{1.949818in}{0.820098in}}%
\pgfpathlineto{\pgfqpoint{1.954327in}{0.825705in}}%
\pgfpathlineto{\pgfqpoint{1.958836in}{0.876068in}}%
\pgfpathlineto{\pgfqpoint{1.963345in}{0.827997in}}%
\pgfpathlineto{\pgfqpoint{1.967855in}{0.823430in}}%
\pgfpathlineto{\pgfqpoint{1.972364in}{0.827172in}}%
\pgfpathlineto{\pgfqpoint{1.981382in}{0.856431in}}%
\pgfpathlineto{\pgfqpoint{1.985891in}{0.839168in}}%
\pgfpathlineto{\pgfqpoint{1.999418in}{0.825998in}}%
\pgfpathlineto{\pgfqpoint{2.003927in}{0.833279in}}%
\pgfpathlineto{\pgfqpoint{2.008436in}{0.849199in}}%
\pgfpathlineto{\pgfqpoint{2.012945in}{0.828193in}}%
\pgfpathlineto{\pgfqpoint{2.017455in}{0.861094in}}%
\pgfpathlineto{\pgfqpoint{2.021964in}{0.830845in}}%
\pgfpathlineto{\pgfqpoint{2.026473in}{0.912420in}}%
\pgfpathlineto{\pgfqpoint{2.030982in}{0.875481in}}%
\pgfpathlineto{\pgfqpoint{2.040000in}{0.849006in}}%
\pgfpathlineto{\pgfqpoint{2.044509in}{0.957651in}}%
\pgfpathlineto{\pgfqpoint{2.049018in}{0.873008in}}%
\pgfpathlineto{\pgfqpoint{2.053527in}{0.925650in}}%
\pgfpathlineto{\pgfqpoint{2.058036in}{0.862240in}}%
\pgfpathlineto{\pgfqpoint{2.062545in}{0.880662in}}%
\pgfpathlineto{\pgfqpoint{2.067055in}{0.851273in}}%
\pgfpathlineto{\pgfqpoint{2.071564in}{0.848341in}}%
\pgfpathlineto{\pgfqpoint{2.076073in}{0.851044in}}%
\pgfpathlineto{\pgfqpoint{2.080582in}{0.869317in}}%
\pgfpathlineto{\pgfqpoint{2.085091in}{0.851134in}}%
\pgfpathlineto{\pgfqpoint{2.089600in}{0.863288in}}%
\pgfpathlineto{\pgfqpoint{2.094109in}{0.856273in}}%
\pgfpathlineto{\pgfqpoint{2.098618in}{0.861389in}}%
\pgfpathlineto{\pgfqpoint{2.103127in}{0.874215in}}%
\pgfpathlineto{\pgfqpoint{2.107636in}{0.861082in}}%
\pgfpathlineto{\pgfqpoint{2.112145in}{0.898191in}}%
\pgfpathlineto{\pgfqpoint{2.116655in}{0.876466in}}%
\pgfpathlineto{\pgfqpoint{2.121164in}{0.986317in}}%
\pgfpathlineto{\pgfqpoint{2.125673in}{0.962070in}}%
\pgfpathlineto{\pgfqpoint{2.130182in}{0.907472in}}%
\pgfpathlineto{\pgfqpoint{2.134691in}{1.036358in}}%
\pgfpathlineto{\pgfqpoint{2.139200in}{0.978150in}}%
\pgfpathlineto{\pgfqpoint{2.143709in}{0.989095in}}%
\pgfpathlineto{\pgfqpoint{2.148218in}{0.909070in}}%
\pgfpathlineto{\pgfqpoint{2.152727in}{0.852492in}}%
\pgfpathlineto{\pgfqpoint{2.157236in}{0.861814in}}%
\pgfpathlineto{\pgfqpoint{2.161745in}{0.889912in}}%
\pgfpathlineto{\pgfqpoint{2.166255in}{0.892735in}}%
\pgfpathlineto{\pgfqpoint{2.170764in}{1.060671in}}%
\pgfpathlineto{\pgfqpoint{2.175273in}{0.914758in}}%
\pgfpathlineto{\pgfqpoint{2.179782in}{1.038833in}}%
\pgfpathlineto{\pgfqpoint{2.184291in}{1.031878in}}%
\pgfpathlineto{\pgfqpoint{2.188800in}{0.885877in}}%
\pgfpathlineto{\pgfqpoint{2.193309in}{1.126212in}}%
\pgfpathlineto{\pgfqpoint{2.197818in}{1.075332in}}%
\pgfpathlineto{\pgfqpoint{2.202327in}{1.145579in}}%
\pgfpathlineto{\pgfqpoint{2.206836in}{0.985362in}}%
\pgfpathlineto{\pgfqpoint{2.211345in}{0.956137in}}%
\pgfpathlineto{\pgfqpoint{2.215855in}{0.990710in}}%
\pgfpathlineto{\pgfqpoint{2.220364in}{0.932110in}}%
\pgfpathlineto{\pgfqpoint{2.224873in}{0.903590in}}%
\pgfpathlineto{\pgfqpoint{2.229382in}{0.924119in}}%
\pgfpathlineto{\pgfqpoint{2.233891in}{0.968180in}}%
\pgfpathlineto{\pgfqpoint{2.238400in}{0.922464in}}%
\pgfpathlineto{\pgfqpoint{2.242909in}{0.906275in}}%
\pgfpathlineto{\pgfqpoint{2.247418in}{0.900151in}}%
\pgfpathlineto{\pgfqpoint{2.251927in}{0.908924in}}%
\pgfpathlineto{\pgfqpoint{2.256436in}{0.901233in}}%
\pgfpathlineto{\pgfqpoint{2.260945in}{0.926456in}}%
\pgfpathlineto{\pgfqpoint{2.265455in}{0.913629in}}%
\pgfpathlineto{\pgfqpoint{2.269964in}{0.916236in}}%
\pgfpathlineto{\pgfqpoint{2.274473in}{1.050382in}}%
\pgfpathlineto{\pgfqpoint{2.278982in}{0.925577in}}%
\pgfpathlineto{\pgfqpoint{2.283491in}{0.914606in}}%
\pgfpathlineto{\pgfqpoint{2.288000in}{0.920059in}}%
\pgfpathlineto{\pgfqpoint{2.292509in}{0.913103in}}%
\pgfpathlineto{\pgfqpoint{2.297018in}{0.922520in}}%
\pgfpathlineto{\pgfqpoint{2.301527in}{0.923658in}}%
\pgfpathlineto{\pgfqpoint{2.306036in}{0.917199in}}%
\pgfpathlineto{\pgfqpoint{2.310545in}{0.916421in}}%
\pgfpathlineto{\pgfqpoint{2.315055in}{0.918229in}}%
\pgfpathlineto{\pgfqpoint{2.319564in}{1.046793in}}%
\pgfpathlineto{\pgfqpoint{2.324073in}{0.923298in}}%
\pgfpathlineto{\pgfqpoint{2.337600in}{0.935935in}}%
\pgfpathlineto{\pgfqpoint{2.342109in}{0.934151in}}%
\pgfpathlineto{\pgfqpoint{2.346618in}{0.929902in}}%
\pgfpathlineto{\pgfqpoint{2.351127in}{0.927960in}}%
\pgfpathlineto{\pgfqpoint{2.355636in}{0.932416in}}%
\pgfpathlineto{\pgfqpoint{2.360145in}{0.946059in}}%
\pgfpathlineto{\pgfqpoint{2.364655in}{0.940450in}}%
\pgfpathlineto{\pgfqpoint{2.369164in}{0.937365in}}%
\pgfpathlineto{\pgfqpoint{2.373673in}{0.949653in}}%
\pgfpathlineto{\pgfqpoint{2.378182in}{0.942898in}}%
\pgfpathlineto{\pgfqpoint{2.382691in}{0.955757in}}%
\pgfpathlineto{\pgfqpoint{2.391709in}{0.938252in}}%
\pgfpathlineto{\pgfqpoint{2.396218in}{0.945012in}}%
\pgfpathlineto{\pgfqpoint{2.400727in}{0.947472in}}%
\pgfpathlineto{\pgfqpoint{2.405236in}{0.957189in}}%
\pgfpathlineto{\pgfqpoint{2.409745in}{0.959516in}}%
\pgfpathlineto{\pgfqpoint{2.414255in}{1.003627in}}%
\pgfpathlineto{\pgfqpoint{2.418764in}{0.954570in}}%
\pgfpathlineto{\pgfqpoint{2.423273in}{0.955781in}}%
\pgfpathlineto{\pgfqpoint{2.427782in}{0.950705in}}%
\pgfpathlineto{\pgfqpoint{2.432291in}{0.962822in}}%
\pgfpathlineto{\pgfqpoint{2.436800in}{0.962648in}}%
\pgfpathlineto{\pgfqpoint{2.441309in}{0.966135in}}%
\pgfpathlineto{\pgfqpoint{2.445818in}{0.986072in}}%
\pgfpathlineto{\pgfqpoint{2.450327in}{0.971723in}}%
\pgfpathlineto{\pgfqpoint{2.454836in}{0.970543in}}%
\pgfpathlineto{\pgfqpoint{2.459345in}{0.985215in}}%
\pgfpathlineto{\pgfqpoint{2.463855in}{0.976193in}}%
\pgfpathlineto{\pgfqpoint{2.468364in}{0.987366in}}%
\pgfpathlineto{\pgfqpoint{2.472873in}{0.978105in}}%
\pgfpathlineto{\pgfqpoint{2.477382in}{0.994987in}}%
\pgfpathlineto{\pgfqpoint{2.481891in}{0.969229in}}%
\pgfpathlineto{\pgfqpoint{2.486400in}{0.976417in}}%
\pgfpathlineto{\pgfqpoint{2.490909in}{1.010175in}}%
\pgfpathlineto{\pgfqpoint{2.495418in}{0.984513in}}%
\pgfpathlineto{\pgfqpoint{2.499927in}{1.009714in}}%
\pgfpathlineto{\pgfqpoint{2.504436in}{0.992617in}}%
\pgfpathlineto{\pgfqpoint{2.508945in}{0.987249in}}%
\pgfpathlineto{\pgfqpoint{2.513455in}{0.988371in}}%
\pgfpathlineto{\pgfqpoint{2.517964in}{0.984354in}}%
\pgfpathlineto{\pgfqpoint{2.522473in}{0.994095in}}%
\pgfpathlineto{\pgfqpoint{2.526982in}{0.977814in}}%
\pgfpathlineto{\pgfqpoint{2.536000in}{1.008026in}}%
\pgfpathlineto{\pgfqpoint{2.540509in}{0.993748in}}%
\pgfpathlineto{\pgfqpoint{2.545018in}{1.012513in}}%
\pgfpathlineto{\pgfqpoint{2.549527in}{0.992111in}}%
\pgfpathlineto{\pgfqpoint{2.554036in}{1.005261in}}%
\pgfpathlineto{\pgfqpoint{2.558545in}{1.001581in}}%
\pgfpathlineto{\pgfqpoint{2.563055in}{1.004043in}}%
\pgfpathlineto{\pgfqpoint{2.567564in}{0.997169in}}%
\pgfpathlineto{\pgfqpoint{2.572073in}{0.993282in}}%
\pgfpathlineto{\pgfqpoint{2.576582in}{1.006830in}}%
\pgfpathlineto{\pgfqpoint{2.581091in}{1.003006in}}%
\pgfpathlineto{\pgfqpoint{2.585600in}{1.005990in}}%
\pgfpathlineto{\pgfqpoint{2.590109in}{1.001941in}}%
\pgfpathlineto{\pgfqpoint{2.594618in}{1.027261in}}%
\pgfpathlineto{\pgfqpoint{2.599127in}{1.008024in}}%
\pgfpathlineto{\pgfqpoint{2.603636in}{1.008151in}}%
\pgfpathlineto{\pgfqpoint{2.608145in}{1.019620in}}%
\pgfpathlineto{\pgfqpoint{2.612655in}{1.015602in}}%
\pgfpathlineto{\pgfqpoint{2.617164in}{1.020862in}}%
\pgfpathlineto{\pgfqpoint{2.621673in}{1.014710in}}%
\pgfpathlineto{\pgfqpoint{2.626182in}{1.022604in}}%
\pgfpathlineto{\pgfqpoint{2.630691in}{1.026940in}}%
\pgfpathlineto{\pgfqpoint{2.635200in}{1.026524in}}%
\pgfpathlineto{\pgfqpoint{2.639709in}{1.023132in}}%
\pgfpathlineto{\pgfqpoint{2.644218in}{1.026069in}}%
\pgfpathlineto{\pgfqpoint{2.648727in}{1.036436in}}%
\pgfpathlineto{\pgfqpoint{2.653236in}{1.027977in}}%
\pgfpathlineto{\pgfqpoint{2.657745in}{1.028597in}}%
\pgfpathlineto{\pgfqpoint{2.662255in}{1.036686in}}%
\pgfpathlineto{\pgfqpoint{2.666764in}{1.027234in}}%
\pgfpathlineto{\pgfqpoint{2.671273in}{1.029046in}}%
\pgfpathlineto{\pgfqpoint{2.675782in}{1.035163in}}%
\pgfpathlineto{\pgfqpoint{2.680291in}{1.056415in}}%
\pgfpathlineto{\pgfqpoint{2.684800in}{1.034682in}}%
\pgfpathlineto{\pgfqpoint{2.689309in}{1.061053in}}%
\pgfpathlineto{\pgfqpoint{2.693818in}{1.048057in}}%
\pgfpathlineto{\pgfqpoint{2.698327in}{1.052375in}}%
\pgfpathlineto{\pgfqpoint{2.702836in}{1.060225in}}%
\pgfpathlineto{\pgfqpoint{2.707345in}{1.062214in}}%
\pgfpathlineto{\pgfqpoint{2.711855in}{1.060604in}}%
\pgfpathlineto{\pgfqpoint{2.716364in}{1.062372in}}%
\pgfpathlineto{\pgfqpoint{2.720873in}{1.059697in}}%
\pgfpathlineto{\pgfqpoint{2.725382in}{1.094738in}}%
\pgfpathlineto{\pgfqpoint{2.729891in}{1.061281in}}%
\pgfpathlineto{\pgfqpoint{2.734400in}{1.074846in}}%
\pgfpathlineto{\pgfqpoint{2.738909in}{1.065960in}}%
\pgfpathlineto{\pgfqpoint{2.743418in}{1.065750in}}%
\pgfpathlineto{\pgfqpoint{2.747927in}{1.069877in}}%
\pgfpathlineto{\pgfqpoint{2.752436in}{1.086435in}}%
\pgfpathlineto{\pgfqpoint{2.756945in}{1.072525in}}%
\pgfpathlineto{\pgfqpoint{2.765964in}{1.078865in}}%
\pgfpathlineto{\pgfqpoint{2.770473in}{1.071973in}}%
\pgfpathlineto{\pgfqpoint{2.774982in}{1.077059in}}%
\pgfpathlineto{\pgfqpoint{2.779491in}{1.072141in}}%
\pgfpathlineto{\pgfqpoint{2.788509in}{1.088140in}}%
\pgfpathlineto{\pgfqpoint{2.793018in}{1.081336in}}%
\pgfpathlineto{\pgfqpoint{2.797527in}{1.080787in}}%
\pgfpathlineto{\pgfqpoint{2.802036in}{1.082479in}}%
\pgfpathlineto{\pgfqpoint{2.806545in}{1.098781in}}%
\pgfpathlineto{\pgfqpoint{2.811055in}{1.100344in}}%
\pgfpathlineto{\pgfqpoint{2.815564in}{1.090902in}}%
\pgfpathlineto{\pgfqpoint{2.820073in}{1.111929in}}%
\pgfpathlineto{\pgfqpoint{2.824582in}{1.093176in}}%
\pgfpathlineto{\pgfqpoint{2.829091in}{1.086454in}}%
\pgfpathlineto{\pgfqpoint{2.833600in}{1.099420in}}%
\pgfpathlineto{\pgfqpoint{2.838109in}{1.102592in}}%
\pgfpathlineto{\pgfqpoint{2.842618in}{1.123138in}}%
\pgfpathlineto{\pgfqpoint{2.847127in}{1.110264in}}%
\pgfpathlineto{\pgfqpoint{2.851636in}{1.107280in}}%
\pgfpathlineto{\pgfqpoint{2.856145in}{1.099442in}}%
\pgfpathlineto{\pgfqpoint{2.860655in}{1.105621in}}%
\pgfpathlineto{\pgfqpoint{2.865164in}{1.102098in}}%
\pgfpathlineto{\pgfqpoint{2.869673in}{1.109005in}}%
\pgfpathlineto{\pgfqpoint{2.874182in}{1.134101in}}%
\pgfpathlineto{\pgfqpoint{2.878691in}{1.209663in}}%
\pgfpathlineto{\pgfqpoint{2.883200in}{1.143630in}}%
\pgfpathlineto{\pgfqpoint{2.887709in}{1.122621in}}%
\pgfpathlineto{\pgfqpoint{2.892218in}{1.144824in}}%
\pgfpathlineto{\pgfqpoint{2.896727in}{1.149384in}}%
\pgfpathlineto{\pgfqpoint{2.905745in}{1.176139in}}%
\pgfpathlineto{\pgfqpoint{2.910255in}{1.168975in}}%
\pgfpathlineto{\pgfqpoint{2.919273in}{1.200422in}}%
\pgfpathlineto{\pgfqpoint{2.923782in}{1.190944in}}%
\pgfpathlineto{\pgfqpoint{2.928291in}{1.152713in}}%
\pgfpathlineto{\pgfqpoint{2.932800in}{1.165464in}}%
\pgfpathlineto{\pgfqpoint{2.937309in}{1.115101in}}%
\pgfpathlineto{\pgfqpoint{2.941818in}{1.145425in}}%
\pgfpathlineto{\pgfqpoint{2.946327in}{1.166683in}}%
\pgfpathlineto{\pgfqpoint{2.950836in}{1.142835in}}%
\pgfpathlineto{\pgfqpoint{2.955345in}{1.141037in}}%
\pgfpathlineto{\pgfqpoint{2.959855in}{1.151174in}}%
\pgfpathlineto{\pgfqpoint{2.964364in}{1.153953in}}%
\pgfpathlineto{\pgfqpoint{2.973382in}{1.156629in}}%
\pgfpathlineto{\pgfqpoint{2.982400in}{1.172012in}}%
\pgfpathlineto{\pgfqpoint{2.986909in}{1.165971in}}%
\pgfpathlineto{\pgfqpoint{2.991418in}{1.165479in}}%
\pgfpathlineto{\pgfqpoint{2.995927in}{1.185058in}}%
\pgfpathlineto{\pgfqpoint{3.000436in}{1.153747in}}%
\pgfpathlineto{\pgfqpoint{3.004945in}{1.172161in}}%
\pgfpathlineto{\pgfqpoint{3.009455in}{1.181808in}}%
\pgfpathlineto{\pgfqpoint{3.013964in}{1.176617in}}%
\pgfpathlineto{\pgfqpoint{3.018473in}{1.185406in}}%
\pgfpathlineto{\pgfqpoint{3.022982in}{1.201992in}}%
\pgfpathlineto{\pgfqpoint{3.027491in}{1.160575in}}%
\pgfpathlineto{\pgfqpoint{3.032000in}{1.186866in}}%
\pgfpathlineto{\pgfqpoint{3.036509in}{1.180296in}}%
\pgfpathlineto{\pgfqpoint{3.041018in}{1.195118in}}%
\pgfpathlineto{\pgfqpoint{3.045527in}{1.229483in}}%
\pgfpathlineto{\pgfqpoint{3.050036in}{1.192339in}}%
\pgfpathlineto{\pgfqpoint{3.054545in}{1.187284in}}%
\pgfpathlineto{\pgfqpoint{3.059055in}{1.200659in}}%
\pgfpathlineto{\pgfqpoint{3.063564in}{1.229111in}}%
\pgfpathlineto{\pgfqpoint{3.068073in}{1.232889in}}%
\pgfpathlineto{\pgfqpoint{3.072582in}{1.213747in}}%
\pgfpathlineto{\pgfqpoint{3.077091in}{1.209556in}}%
\pgfpathlineto{\pgfqpoint{3.081600in}{1.222580in}}%
\pgfpathlineto{\pgfqpoint{3.086109in}{1.214608in}}%
\pgfpathlineto{\pgfqpoint{3.090618in}{1.233536in}}%
\pgfpathlineto{\pgfqpoint{3.099636in}{1.196972in}}%
\pgfpathlineto{\pgfqpoint{3.104145in}{1.191033in}}%
\pgfpathlineto{\pgfqpoint{3.108655in}{1.247290in}}%
\pgfpathlineto{\pgfqpoint{3.113164in}{1.225417in}}%
\pgfpathlineto{\pgfqpoint{3.117673in}{1.234284in}}%
\pgfpathlineto{\pgfqpoint{3.122182in}{1.220262in}}%
\pgfpathlineto{\pgfqpoint{3.126691in}{1.221157in}}%
\pgfpathlineto{\pgfqpoint{3.131200in}{1.265854in}}%
\pgfpathlineto{\pgfqpoint{3.135709in}{1.282263in}}%
\pgfpathlineto{\pgfqpoint{3.140218in}{1.232643in}}%
\pgfpathlineto{\pgfqpoint{3.144727in}{1.229098in}}%
\pgfpathlineto{\pgfqpoint{3.149236in}{1.228173in}}%
\pgfpathlineto{\pgfqpoint{3.153745in}{1.247734in}}%
\pgfpathlineto{\pgfqpoint{3.158255in}{1.236540in}}%
\pgfpathlineto{\pgfqpoint{3.162764in}{1.271432in}}%
\pgfpathlineto{\pgfqpoint{3.167273in}{1.250698in}}%
\pgfpathlineto{\pgfqpoint{3.171782in}{1.259698in}}%
\pgfpathlineto{\pgfqpoint{3.176291in}{1.239012in}}%
\pgfpathlineto{\pgfqpoint{3.180800in}{1.244327in}}%
\pgfpathlineto{\pgfqpoint{3.185309in}{1.238092in}}%
\pgfpathlineto{\pgfqpoint{3.189818in}{1.270530in}}%
\pgfpathlineto{\pgfqpoint{3.194327in}{1.263404in}}%
\pgfpathlineto{\pgfqpoint{3.198836in}{1.252537in}}%
\pgfpathlineto{\pgfqpoint{3.207855in}{1.238731in}}%
\pgfpathlineto{\pgfqpoint{3.212364in}{1.283756in}}%
\pgfpathlineto{\pgfqpoint{3.216873in}{1.266584in}}%
\pgfpathlineto{\pgfqpoint{3.221382in}{1.290459in}}%
\pgfpathlineto{\pgfqpoint{3.225891in}{1.256415in}}%
\pgfpathlineto{\pgfqpoint{3.239418in}{1.293790in}}%
\pgfpathlineto{\pgfqpoint{3.243927in}{1.274283in}}%
\pgfpathlineto{\pgfqpoint{3.248436in}{1.302147in}}%
\pgfpathlineto{\pgfqpoint{3.252945in}{1.294423in}}%
\pgfpathlineto{\pgfqpoint{3.257455in}{1.278009in}}%
\pgfpathlineto{\pgfqpoint{3.261964in}{1.315639in}}%
\pgfpathlineto{\pgfqpoint{3.266473in}{1.294370in}}%
\pgfpathlineto{\pgfqpoint{3.270982in}{1.326436in}}%
\pgfpathlineto{\pgfqpoint{3.275491in}{1.294781in}}%
\pgfpathlineto{\pgfqpoint{3.280000in}{1.290503in}}%
\pgfpathlineto{\pgfqpoint{3.284509in}{1.308364in}}%
\pgfpathlineto{\pgfqpoint{3.289018in}{1.341638in}}%
\pgfpathlineto{\pgfqpoint{3.293527in}{1.326872in}}%
\pgfpathlineto{\pgfqpoint{3.298036in}{1.303879in}}%
\pgfpathlineto{\pgfqpoint{3.302545in}{1.341652in}}%
\pgfpathlineto{\pgfqpoint{3.307055in}{1.318092in}}%
\pgfpathlineto{\pgfqpoint{3.311564in}{1.347345in}}%
\pgfpathlineto{\pgfqpoint{3.316073in}{1.309714in}}%
\pgfpathlineto{\pgfqpoint{3.320582in}{1.390707in}}%
\pgfpathlineto{\pgfqpoint{3.325091in}{1.333222in}}%
\pgfpathlineto{\pgfqpoint{3.329600in}{1.312573in}}%
\pgfpathlineto{\pgfqpoint{3.334109in}{1.365845in}}%
\pgfpathlineto{\pgfqpoint{3.338618in}{1.323129in}}%
\pgfpathlineto{\pgfqpoint{3.343127in}{1.308928in}}%
\pgfpathlineto{\pgfqpoint{3.347636in}{1.338940in}}%
\pgfpathlineto{\pgfqpoint{3.352145in}{1.336307in}}%
\pgfpathlineto{\pgfqpoint{3.356655in}{1.350349in}}%
\pgfpathlineto{\pgfqpoint{3.361164in}{1.333266in}}%
\pgfpathlineto{\pgfqpoint{3.365673in}{1.352742in}}%
\pgfpathlineto{\pgfqpoint{3.370182in}{1.344412in}}%
\pgfpathlineto{\pgfqpoint{3.374691in}{1.322671in}}%
\pgfpathlineto{\pgfqpoint{3.379200in}{1.376451in}}%
\pgfpathlineto{\pgfqpoint{3.383709in}{1.346181in}}%
\pgfpathlineto{\pgfqpoint{3.388218in}{1.347940in}}%
\pgfpathlineto{\pgfqpoint{3.392727in}{1.364769in}}%
\pgfpathlineto{\pgfqpoint{3.397236in}{1.369092in}}%
\pgfpathlineto{\pgfqpoint{3.401745in}{1.375750in}}%
\pgfpathlineto{\pgfqpoint{3.406255in}{1.355619in}}%
\pgfpathlineto{\pgfqpoint{3.410764in}{1.356172in}}%
\pgfpathlineto{\pgfqpoint{3.415273in}{1.404859in}}%
\pgfpathlineto{\pgfqpoint{3.419782in}{1.361069in}}%
\pgfpathlineto{\pgfqpoint{3.424291in}{1.370843in}}%
\pgfpathlineto{\pgfqpoint{3.428800in}{1.400851in}}%
\pgfpathlineto{\pgfqpoint{3.433309in}{1.415350in}}%
\pgfpathlineto{\pgfqpoint{3.437818in}{1.393097in}}%
\pgfpathlineto{\pgfqpoint{3.442327in}{1.404488in}}%
\pgfpathlineto{\pgfqpoint{3.446836in}{1.380319in}}%
\pgfpathlineto{\pgfqpoint{3.451345in}{1.447423in}}%
\pgfpathlineto{\pgfqpoint{3.460364in}{1.390150in}}%
\pgfpathlineto{\pgfqpoint{3.464873in}{1.408457in}}%
\pgfpathlineto{\pgfqpoint{3.469382in}{1.385420in}}%
\pgfpathlineto{\pgfqpoint{3.473891in}{1.416010in}}%
\pgfpathlineto{\pgfqpoint{3.478400in}{1.420027in}}%
\pgfpathlineto{\pgfqpoint{3.482909in}{1.417615in}}%
\pgfpathlineto{\pgfqpoint{3.487418in}{1.411093in}}%
\pgfpathlineto{\pgfqpoint{3.491927in}{1.466148in}}%
\pgfpathlineto{\pgfqpoint{3.496436in}{1.420473in}}%
\pgfpathlineto{\pgfqpoint{3.500945in}{1.400172in}}%
\pgfpathlineto{\pgfqpoint{3.505455in}{1.449901in}}%
\pgfpathlineto{\pgfqpoint{3.509964in}{1.425013in}}%
\pgfpathlineto{\pgfqpoint{3.514473in}{1.428378in}}%
\pgfpathlineto{\pgfqpoint{3.518982in}{1.458186in}}%
\pgfpathlineto{\pgfqpoint{3.523491in}{1.447353in}}%
\pgfpathlineto{\pgfqpoint{3.528000in}{1.445050in}}%
\pgfpathlineto{\pgfqpoint{3.532509in}{1.459378in}}%
\pgfpathlineto{\pgfqpoint{3.537018in}{1.464745in}}%
\pgfpathlineto{\pgfqpoint{3.546036in}{1.449691in}}%
\pgfpathlineto{\pgfqpoint{3.550545in}{1.446047in}}%
\pgfpathlineto{\pgfqpoint{3.559564in}{1.505388in}}%
\pgfpathlineto{\pgfqpoint{3.564073in}{1.455004in}}%
\pgfpathlineto{\pgfqpoint{3.568582in}{1.463322in}}%
\pgfpathlineto{\pgfqpoint{3.573091in}{1.464978in}}%
\pgfpathlineto{\pgfqpoint{3.577600in}{1.446570in}}%
\pgfpathlineto{\pgfqpoint{3.582109in}{1.495016in}}%
\pgfpathlineto{\pgfqpoint{3.586618in}{1.465569in}}%
\pgfpathlineto{\pgfqpoint{3.591127in}{1.486718in}}%
\pgfpathlineto{\pgfqpoint{3.595636in}{1.475017in}}%
\pgfpathlineto{\pgfqpoint{3.600145in}{1.489751in}}%
\pgfpathlineto{\pgfqpoint{3.604655in}{1.508726in}}%
\pgfpathlineto{\pgfqpoint{3.609164in}{1.498751in}}%
\pgfpathlineto{\pgfqpoint{3.613673in}{1.485201in}}%
\pgfpathlineto{\pgfqpoint{3.618182in}{1.491444in}}%
\pgfpathlineto{\pgfqpoint{3.622691in}{1.478545in}}%
\pgfpathlineto{\pgfqpoint{3.631709in}{1.496065in}}%
\pgfpathlineto{\pgfqpoint{3.636218in}{1.530190in}}%
\pgfpathlineto{\pgfqpoint{3.640727in}{1.501716in}}%
\pgfpathlineto{\pgfqpoint{3.645236in}{1.528850in}}%
\pgfpathlineto{\pgfqpoint{3.649745in}{1.489639in}}%
\pgfpathlineto{\pgfqpoint{3.654255in}{1.500316in}}%
\pgfpathlineto{\pgfqpoint{3.658764in}{1.552154in}}%
\pgfpathlineto{\pgfqpoint{3.663273in}{1.488033in}}%
\pgfpathlineto{\pgfqpoint{3.667782in}{1.504007in}}%
\pgfpathlineto{\pgfqpoint{3.676800in}{1.546490in}}%
\pgfpathlineto{\pgfqpoint{3.681309in}{1.559006in}}%
\pgfpathlineto{\pgfqpoint{3.685818in}{1.548226in}}%
\pgfpathlineto{\pgfqpoint{3.690327in}{1.530236in}}%
\pgfpathlineto{\pgfqpoint{3.694836in}{1.537538in}}%
\pgfpathlineto{\pgfqpoint{3.699345in}{1.581317in}}%
\pgfpathlineto{\pgfqpoint{3.703855in}{1.555555in}}%
\pgfpathlineto{\pgfqpoint{3.708364in}{1.511215in}}%
\pgfpathlineto{\pgfqpoint{3.717382in}{1.573896in}}%
\pgfpathlineto{\pgfqpoint{3.721891in}{1.563504in}}%
\pgfpathlineto{\pgfqpoint{3.726400in}{1.572684in}}%
\pgfpathlineto{\pgfqpoint{3.730909in}{1.602428in}}%
\pgfpathlineto{\pgfqpoint{3.735418in}{1.536454in}}%
\pgfpathlineto{\pgfqpoint{3.739927in}{1.563339in}}%
\pgfpathlineto{\pgfqpoint{3.744436in}{1.563531in}}%
\pgfpathlineto{\pgfqpoint{3.748945in}{1.537504in}}%
\pgfpathlineto{\pgfqpoint{3.753455in}{1.553492in}}%
\pgfpathlineto{\pgfqpoint{3.757964in}{1.582015in}}%
\pgfpathlineto{\pgfqpoint{3.762473in}{1.590348in}}%
\pgfpathlineto{\pgfqpoint{3.766982in}{1.593146in}}%
\pgfpathlineto{\pgfqpoint{3.771491in}{1.577619in}}%
\pgfpathlineto{\pgfqpoint{3.776000in}{1.591896in}}%
\pgfpathlineto{\pgfqpoint{3.780509in}{1.587774in}}%
\pgfpathlineto{\pgfqpoint{3.785018in}{1.571611in}}%
\pgfpathlineto{\pgfqpoint{3.789527in}{1.686971in}}%
\pgfpathlineto{\pgfqpoint{3.794036in}{1.749753in}}%
\pgfpathlineto{\pgfqpoint{3.798545in}{1.906250in}}%
\pgfpathlineto{\pgfqpoint{3.803055in}{2.295794in}}%
\pgfpathlineto{\pgfqpoint{3.807564in}{1.707089in}}%
\pgfpathlineto{\pgfqpoint{3.812073in}{1.748807in}}%
\pgfpathlineto{\pgfqpoint{3.816582in}{2.014522in}}%
\pgfpathlineto{\pgfqpoint{3.821091in}{1.640972in}}%
\pgfpathlineto{\pgfqpoint{3.825600in}{1.565568in}}%
\pgfpathlineto{\pgfqpoint{3.830109in}{1.572390in}}%
\pgfpathlineto{\pgfqpoint{3.834618in}{1.595664in}}%
\pgfpathlineto{\pgfqpoint{3.839127in}{1.589558in}}%
\pgfpathlineto{\pgfqpoint{3.843636in}{1.964280in}}%
\pgfpathlineto{\pgfqpoint{3.848145in}{1.741023in}}%
\pgfpathlineto{\pgfqpoint{3.852655in}{1.656809in}}%
\pgfpathlineto{\pgfqpoint{3.861673in}{1.604329in}}%
\pgfpathlineto{\pgfqpoint{3.866182in}{1.655687in}}%
\pgfpathlineto{\pgfqpoint{3.870691in}{1.628140in}}%
\pgfpathlineto{\pgfqpoint{3.879709in}{1.623559in}}%
\pgfpathlineto{\pgfqpoint{3.884218in}{1.614742in}}%
\pgfpathlineto{\pgfqpoint{3.888727in}{1.682960in}}%
\pgfpathlineto{\pgfqpoint{3.893236in}{1.656275in}}%
\pgfpathlineto{\pgfqpoint{3.897745in}{1.659909in}}%
\pgfpathlineto{\pgfqpoint{3.902255in}{1.641732in}}%
\pgfpathlineto{\pgfqpoint{3.906764in}{1.680809in}}%
\pgfpathlineto{\pgfqpoint{3.911273in}{1.656338in}}%
\pgfpathlineto{\pgfqpoint{3.915782in}{1.708269in}}%
\pgfpathlineto{\pgfqpoint{3.920291in}{1.716526in}}%
\pgfpathlineto{\pgfqpoint{3.924800in}{1.652975in}}%
\pgfpathlineto{\pgfqpoint{3.929309in}{1.668636in}}%
\pgfpathlineto{\pgfqpoint{3.933818in}{1.679220in}}%
\pgfpathlineto{\pgfqpoint{3.938327in}{1.683715in}}%
\pgfpathlineto{\pgfqpoint{3.947345in}{1.653501in}}%
\pgfpathlineto{\pgfqpoint{3.951855in}{1.651075in}}%
\pgfpathlineto{\pgfqpoint{3.956364in}{1.676111in}}%
\pgfpathlineto{\pgfqpoint{3.960873in}{1.674195in}}%
\pgfpathlineto{\pgfqpoint{3.965382in}{1.702046in}}%
\pgfpathlineto{\pgfqpoint{3.969891in}{1.708379in}}%
\pgfpathlineto{\pgfqpoint{3.974400in}{1.690177in}}%
\pgfpathlineto{\pgfqpoint{3.978909in}{1.726709in}}%
\pgfpathlineto{\pgfqpoint{3.983418in}{1.703022in}}%
\pgfpathlineto{\pgfqpoint{3.987927in}{1.736996in}}%
\pgfpathlineto{\pgfqpoint{3.992436in}{1.730998in}}%
\pgfpathlineto{\pgfqpoint{3.996945in}{1.702246in}}%
\pgfpathlineto{\pgfqpoint{4.001455in}{1.731436in}}%
\pgfpathlineto{\pgfqpoint{4.005964in}{1.700132in}}%
\pgfpathlineto{\pgfqpoint{4.010473in}{1.683762in}}%
\pgfpathlineto{\pgfqpoint{4.014982in}{1.681198in}}%
\pgfpathlineto{\pgfqpoint{4.019491in}{1.711284in}}%
\pgfpathlineto{\pgfqpoint{4.024000in}{1.703028in}}%
\pgfpathlineto{\pgfqpoint{4.028509in}{1.706301in}}%
\pgfpathlineto{\pgfqpoint{4.033018in}{1.701620in}}%
\pgfpathlineto{\pgfqpoint{4.037527in}{1.732087in}}%
\pgfpathlineto{\pgfqpoint{4.042036in}{1.726973in}}%
\pgfpathlineto{\pgfqpoint{4.046545in}{1.739329in}}%
\pgfpathlineto{\pgfqpoint{4.051055in}{1.746236in}}%
\pgfpathlineto{\pgfqpoint{4.055564in}{1.739048in}}%
\pgfpathlineto{\pgfqpoint{4.060073in}{1.727352in}}%
\pgfpathlineto{\pgfqpoint{4.064582in}{1.744746in}}%
\pgfpathlineto{\pgfqpoint{4.069091in}{1.754033in}}%
\pgfpathlineto{\pgfqpoint{4.073600in}{1.730655in}}%
\pgfpathlineto{\pgfqpoint{4.078109in}{1.781966in}}%
\pgfpathlineto{\pgfqpoint{4.082618in}{1.800132in}}%
\pgfpathlineto{\pgfqpoint{4.087127in}{1.741259in}}%
\pgfpathlineto{\pgfqpoint{4.091636in}{1.782580in}}%
\pgfpathlineto{\pgfqpoint{4.096145in}{1.753011in}}%
\pgfpathlineto{\pgfqpoint{4.100655in}{1.790731in}}%
\pgfpathlineto{\pgfqpoint{4.105164in}{1.783057in}}%
\pgfpathlineto{\pgfqpoint{4.109673in}{1.792212in}}%
\pgfpathlineto{\pgfqpoint{4.114182in}{1.782980in}}%
\pgfpathlineto{\pgfqpoint{4.118691in}{1.833512in}}%
\pgfpathlineto{\pgfqpoint{4.123200in}{1.782218in}}%
\pgfpathlineto{\pgfqpoint{4.127709in}{1.821688in}}%
\pgfpathlineto{\pgfqpoint{4.132218in}{1.744512in}}%
\pgfpathlineto{\pgfqpoint{4.136727in}{1.844742in}}%
\pgfpathlineto{\pgfqpoint{4.141236in}{1.789375in}}%
\pgfpathlineto{\pgfqpoint{4.145745in}{1.859887in}}%
\pgfpathlineto{\pgfqpoint{4.150255in}{1.767380in}}%
\pgfpathlineto{\pgfqpoint{4.154764in}{1.775115in}}%
\pgfpathlineto{\pgfqpoint{4.159273in}{1.850493in}}%
\pgfpathlineto{\pgfqpoint{4.163782in}{1.800315in}}%
\pgfpathlineto{\pgfqpoint{4.168291in}{1.805510in}}%
\pgfpathlineto{\pgfqpoint{4.172800in}{1.825686in}}%
\pgfpathlineto{\pgfqpoint{4.177309in}{1.796156in}}%
\pgfpathlineto{\pgfqpoint{4.181818in}{1.817041in}}%
\pgfpathlineto{\pgfqpoint{4.195345in}{1.847894in}}%
\pgfpathlineto{\pgfqpoint{4.199855in}{1.789469in}}%
\pgfpathlineto{\pgfqpoint{4.204364in}{1.825686in}}%
\pgfpathlineto{\pgfqpoint{4.208873in}{1.813424in}}%
\pgfpathlineto{\pgfqpoint{4.213382in}{1.837837in}}%
\pgfpathlineto{\pgfqpoint{4.217891in}{1.851772in}}%
\pgfpathlineto{\pgfqpoint{4.222400in}{1.909396in}}%
\pgfpathlineto{\pgfqpoint{4.226909in}{1.834609in}}%
\pgfpathlineto{\pgfqpoint{4.231418in}{1.852979in}}%
\pgfpathlineto{\pgfqpoint{4.235927in}{1.825989in}}%
\pgfpathlineto{\pgfqpoint{4.240436in}{1.878438in}}%
\pgfpathlineto{\pgfqpoint{4.244945in}{1.864554in}}%
\pgfpathlineto{\pgfqpoint{4.249455in}{1.880879in}}%
\pgfpathlineto{\pgfqpoint{4.253964in}{1.948563in}}%
\pgfpathlineto{\pgfqpoint{4.258473in}{1.884616in}}%
\pgfpathlineto{\pgfqpoint{4.262982in}{1.848260in}}%
\pgfpathlineto{\pgfqpoint{4.267491in}{1.863563in}}%
\pgfpathlineto{\pgfqpoint{4.272000in}{1.866456in}}%
\pgfpathlineto{\pgfqpoint{4.276509in}{1.925413in}}%
\pgfpathlineto{\pgfqpoint{4.281018in}{1.858115in}}%
\pgfpathlineto{\pgfqpoint{4.285527in}{1.896239in}}%
\pgfpathlineto{\pgfqpoint{4.294545in}{1.880568in}}%
\pgfpathlineto{\pgfqpoint{4.299055in}{1.893002in}}%
\pgfpathlineto{\pgfqpoint{4.303564in}{1.937097in}}%
\pgfpathlineto{\pgfqpoint{4.308073in}{1.964407in}}%
\pgfpathlineto{\pgfqpoint{4.312582in}{1.876227in}}%
\pgfpathlineto{\pgfqpoint{4.321600in}{1.931382in}}%
\pgfpathlineto{\pgfqpoint{4.326109in}{1.963078in}}%
\pgfpathlineto{\pgfqpoint{4.330618in}{1.909977in}}%
\pgfpathlineto{\pgfqpoint{4.335127in}{1.905949in}}%
\pgfpathlineto{\pgfqpoint{4.339636in}{1.927864in}}%
\pgfpathlineto{\pgfqpoint{4.344145in}{1.871671in}}%
\pgfpathlineto{\pgfqpoint{4.348655in}{2.015506in}}%
\pgfpathlineto{\pgfqpoint{4.353164in}{1.943898in}}%
\pgfpathlineto{\pgfqpoint{4.357673in}{1.919150in}}%
\pgfpathlineto{\pgfqpoint{4.366691in}{1.927801in}}%
\pgfpathlineto{\pgfqpoint{4.371200in}{1.941709in}}%
\pgfpathlineto{\pgfqpoint{4.375709in}{1.930146in}}%
\pgfpathlineto{\pgfqpoint{4.380218in}{1.968521in}}%
\pgfpathlineto{\pgfqpoint{4.384727in}{1.943424in}}%
\pgfpathlineto{\pgfqpoint{4.389236in}{1.959850in}}%
\pgfpathlineto{\pgfqpoint{4.393745in}{1.926545in}}%
\pgfpathlineto{\pgfqpoint{4.398255in}{1.935135in}}%
\pgfpathlineto{\pgfqpoint{4.402764in}{1.947817in}}%
\pgfpathlineto{\pgfqpoint{4.407273in}{1.986200in}}%
\pgfpathlineto{\pgfqpoint{4.411782in}{2.006377in}}%
\pgfpathlineto{\pgfqpoint{4.416291in}{1.972313in}}%
\pgfpathlineto{\pgfqpoint{4.420800in}{2.040015in}}%
\pgfpathlineto{\pgfqpoint{4.425309in}{1.987108in}}%
\pgfpathlineto{\pgfqpoint{4.429818in}{1.976077in}}%
\pgfpathlineto{\pgfqpoint{4.434327in}{1.980099in}}%
\pgfpathlineto{\pgfqpoint{4.438836in}{1.966729in}}%
\pgfpathlineto{\pgfqpoint{4.443345in}{1.971804in}}%
\pgfpathlineto{\pgfqpoint{4.447855in}{1.992119in}}%
\pgfpathlineto{\pgfqpoint{4.452364in}{1.998834in}}%
\pgfpathlineto{\pgfqpoint{4.456873in}{2.000685in}}%
\pgfpathlineto{\pgfqpoint{4.461382in}{1.990872in}}%
\pgfpathlineto{\pgfqpoint{4.465891in}{2.014750in}}%
\pgfpathlineto{\pgfqpoint{4.470400in}{1.992441in}}%
\pgfpathlineto{\pgfqpoint{4.474909in}{2.003186in}}%
\pgfpathlineto{\pgfqpoint{4.479418in}{2.040404in}}%
\pgfpathlineto{\pgfqpoint{4.483927in}{2.001773in}}%
\pgfpathlineto{\pgfqpoint{4.488436in}{2.023702in}}%
\pgfpathlineto{\pgfqpoint{4.492945in}{2.030245in}}%
\pgfpathlineto{\pgfqpoint{4.497455in}{2.049293in}}%
\pgfpathlineto{\pgfqpoint{4.501964in}{2.029476in}}%
\pgfpathlineto{\pgfqpoint{4.506473in}{2.037398in}}%
\pgfpathlineto{\pgfqpoint{4.510982in}{2.081790in}}%
\pgfpathlineto{\pgfqpoint{4.515491in}{2.027353in}}%
\pgfpathlineto{\pgfqpoint{4.520000in}{2.044664in}}%
\pgfpathlineto{\pgfqpoint{4.524509in}{2.047365in}}%
\pgfpathlineto{\pgfqpoint{4.529018in}{2.116284in}}%
\pgfpathlineto{\pgfqpoint{4.533527in}{2.076056in}}%
\pgfpathlineto{\pgfqpoint{4.538036in}{2.064602in}}%
\pgfpathlineto{\pgfqpoint{4.542545in}{2.095268in}}%
\pgfpathlineto{\pgfqpoint{4.547055in}{2.059462in}}%
\pgfpathlineto{\pgfqpoint{4.551564in}{2.073685in}}%
\pgfpathlineto{\pgfqpoint{4.556073in}{2.108167in}}%
\pgfpathlineto{\pgfqpoint{4.560582in}{2.050772in}}%
\pgfpathlineto{\pgfqpoint{4.565091in}{2.090560in}}%
\pgfpathlineto{\pgfqpoint{4.569600in}{2.061038in}}%
\pgfpathlineto{\pgfqpoint{4.574109in}{2.093193in}}%
\pgfpathlineto{\pgfqpoint{4.578618in}{2.073213in}}%
\pgfpathlineto{\pgfqpoint{4.587636in}{2.178125in}}%
\pgfpathlineto{\pgfqpoint{4.592145in}{2.208738in}}%
\pgfpathlineto{\pgfqpoint{4.596655in}{2.231341in}}%
\pgfpathlineto{\pgfqpoint{4.601164in}{2.175106in}}%
\pgfpathlineto{\pgfqpoint{4.605673in}{2.218659in}}%
\pgfpathlineto{\pgfqpoint{4.610182in}{2.632338in}}%
\pgfpathlineto{\pgfqpoint{4.614691in}{2.176535in}}%
\pgfpathlineto{\pgfqpoint{4.619200in}{2.135027in}}%
\pgfpathlineto{\pgfqpoint{4.623709in}{2.193625in}}%
\pgfpathlineto{\pgfqpoint{4.628218in}{2.224630in}}%
\pgfpathlineto{\pgfqpoint{4.632727in}{2.291972in}}%
\pgfpathlineto{\pgfqpoint{4.637236in}{2.130263in}}%
\pgfpathlineto{\pgfqpoint{4.641745in}{2.277501in}}%
\pgfpathlineto{\pgfqpoint{4.646255in}{2.731787in}}%
\pgfpathlineto{\pgfqpoint{4.655273in}{2.206544in}}%
\pgfpathlineto{\pgfqpoint{4.659782in}{2.259383in}}%
\pgfpathlineto{\pgfqpoint{4.664291in}{2.210077in}}%
\pgfpathlineto{\pgfqpoint{4.668800in}{2.972980in}}%
\pgfpathlineto{\pgfqpoint{4.673309in}{2.698765in}}%
\pgfpathlineto{\pgfqpoint{4.677818in}{2.312429in}}%
\pgfpathlineto{\pgfqpoint{4.682327in}{2.174088in}}%
\pgfpathlineto{\pgfqpoint{4.691345in}{2.352457in}}%
\pgfpathlineto{\pgfqpoint{4.695855in}{2.339267in}}%
\pgfpathlineto{\pgfqpoint{4.700364in}{2.399046in}}%
\pgfpathlineto{\pgfqpoint{4.704873in}{2.425784in}}%
\pgfpathlineto{\pgfqpoint{4.709382in}{2.544096in}}%
\pgfpathlineto{\pgfqpoint{4.713891in}{2.833558in}}%
\pgfpathlineto{\pgfqpoint{4.718400in}{2.226012in}}%
\pgfpathlineto{\pgfqpoint{4.722909in}{2.422844in}}%
\pgfpathlineto{\pgfqpoint{4.727418in}{2.402245in}}%
\pgfpathlineto{\pgfqpoint{4.731927in}{2.333944in}}%
\pgfpathlineto{\pgfqpoint{4.736436in}{2.451704in}}%
\pgfpathlineto{\pgfqpoint{4.740945in}{2.808971in}}%
\pgfpathlineto{\pgfqpoint{4.745455in}{2.318643in}}%
\pgfpathlineto{\pgfqpoint{4.749964in}{2.534552in}}%
\pgfpathlineto{\pgfqpoint{4.754473in}{2.345442in}}%
\pgfpathlineto{\pgfqpoint{4.758982in}{2.490381in}}%
\pgfpathlineto{\pgfqpoint{4.763491in}{2.077575in}}%
\pgfpathlineto{\pgfqpoint{4.768000in}{2.067374in}}%
\pgfpathlineto{\pgfqpoint{4.772509in}{2.087914in}}%
\pgfpathlineto{\pgfqpoint{4.777018in}{2.064193in}}%
\pgfpathlineto{\pgfqpoint{4.781527in}{2.160042in}}%
\pgfpathlineto{\pgfqpoint{4.786036in}{2.142133in}}%
\pgfpathlineto{\pgfqpoint{4.790545in}{2.075594in}}%
\pgfpathlineto{\pgfqpoint{4.795055in}{2.113506in}}%
\pgfpathlineto{\pgfqpoint{4.804073in}{2.219978in}}%
\pgfpathlineto{\pgfqpoint{4.808582in}{3.200935in}}%
\pgfpathlineto{\pgfqpoint{4.813091in}{2.955781in}}%
\pgfpathlineto{\pgfqpoint{4.817600in}{2.112107in}}%
\pgfpathlineto{\pgfqpoint{4.822109in}{2.082748in}}%
\pgfpathlineto{\pgfqpoint{4.826618in}{2.159353in}}%
\pgfpathlineto{\pgfqpoint{4.831127in}{2.775151in}}%
\pgfpathlineto{\pgfqpoint{4.835636in}{2.280601in}}%
\pgfpathlineto{\pgfqpoint{4.840145in}{2.146657in}}%
\pgfpathlineto{\pgfqpoint{4.844655in}{2.253423in}}%
\pgfpathlineto{\pgfqpoint{4.849164in}{3.060828in}}%
\pgfpathlineto{\pgfqpoint{4.853673in}{2.350302in}}%
\pgfpathlineto{\pgfqpoint{4.858182in}{2.358419in}}%
\pgfpathlineto{\pgfqpoint{4.862691in}{2.346876in}}%
\pgfpathlineto{\pgfqpoint{4.867200in}{2.594748in}}%
\pgfpathlineto{\pgfqpoint{4.871709in}{3.627543in}}%
\pgfpathlineto{\pgfqpoint{4.876218in}{2.764710in}}%
\pgfpathlineto{\pgfqpoint{4.880727in}{2.615652in}}%
\pgfpathlineto{\pgfqpoint{4.885236in}{2.560811in}}%
\pgfpathlineto{\pgfqpoint{4.889745in}{2.307512in}}%
\pgfpathlineto{\pgfqpoint{4.894255in}{2.372542in}}%
\pgfpathlineto{\pgfqpoint{4.898764in}{2.328575in}}%
\pgfpathlineto{\pgfqpoint{4.903273in}{2.410564in}}%
\pgfpathlineto{\pgfqpoint{4.907782in}{2.379501in}}%
\pgfpathlineto{\pgfqpoint{4.912291in}{2.332544in}}%
\pgfpathlineto{\pgfqpoint{4.916800in}{2.397565in}}%
\pgfpathlineto{\pgfqpoint{4.921309in}{2.423984in}}%
\pgfpathlineto{\pgfqpoint{4.925818in}{2.379439in}}%
\pgfpathlineto{\pgfqpoint{4.930327in}{2.388904in}}%
\pgfpathlineto{\pgfqpoint{4.934836in}{2.395143in}}%
\pgfpathlineto{\pgfqpoint{4.939345in}{2.371538in}}%
\pgfpathlineto{\pgfqpoint{4.943855in}{2.551291in}}%
\pgfpathlineto{\pgfqpoint{4.948364in}{2.571971in}}%
\pgfpathlineto{\pgfqpoint{4.952873in}{2.402050in}}%
\pgfpathlineto{\pgfqpoint{4.957382in}{2.412526in}}%
\pgfpathlineto{\pgfqpoint{4.961891in}{2.445435in}}%
\pgfpathlineto{\pgfqpoint{4.966400in}{2.436680in}}%
\pgfpathlineto{\pgfqpoint{4.970909in}{2.468950in}}%
\pgfpathlineto{\pgfqpoint{4.975418in}{2.403910in}}%
\pgfpathlineto{\pgfqpoint{4.979927in}{2.405273in}}%
\pgfpathlineto{\pgfqpoint{4.984436in}{2.459984in}}%
\pgfpathlineto{\pgfqpoint{4.988945in}{2.380656in}}%
\pgfpathlineto{\pgfqpoint{4.993455in}{2.436702in}}%
\pgfpathlineto{\pgfqpoint{5.002473in}{2.463303in}}%
\pgfpathlineto{\pgfqpoint{5.006982in}{2.455106in}}%
\pgfpathlineto{\pgfqpoint{5.011491in}{2.406741in}}%
\pgfpathlineto{\pgfqpoint{5.016000in}{2.458959in}}%
\pgfpathlineto{\pgfqpoint{5.020509in}{2.392371in}}%
\pgfpathlineto{\pgfqpoint{5.025018in}{2.411274in}}%
\pgfpathlineto{\pgfqpoint{5.029527in}{2.443762in}}%
\pgfpathlineto{\pgfqpoint{5.034036in}{2.439915in}}%
\pgfpathlineto{\pgfqpoint{5.038545in}{2.472118in}}%
\pgfpathlineto{\pgfqpoint{5.043055in}{2.563051in}}%
\pgfpathlineto{\pgfqpoint{5.047564in}{2.536415in}}%
\pgfpathlineto{\pgfqpoint{5.052073in}{2.448746in}}%
\pgfpathlineto{\pgfqpoint{5.056582in}{2.639829in}}%
\pgfpathlineto{\pgfqpoint{5.065600in}{2.503605in}}%
\pgfpathlineto{\pgfqpoint{5.070109in}{2.490181in}}%
\pgfpathlineto{\pgfqpoint{5.074618in}{2.543001in}}%
\pgfpathlineto{\pgfqpoint{5.079127in}{2.448363in}}%
\pgfpathlineto{\pgfqpoint{5.083636in}{2.607401in}}%
\pgfpathlineto{\pgfqpoint{5.088145in}{2.482535in}}%
\pgfpathlineto{\pgfqpoint{5.092655in}{2.508883in}}%
\pgfpathlineto{\pgfqpoint{5.097164in}{2.586154in}}%
\pgfpathlineto{\pgfqpoint{5.101673in}{2.511636in}}%
\pgfpathlineto{\pgfqpoint{5.106182in}{2.503647in}}%
\pgfpathlineto{\pgfqpoint{5.110691in}{2.552922in}}%
\pgfpathlineto{\pgfqpoint{5.115200in}{2.561920in}}%
\pgfpathlineto{\pgfqpoint{5.119709in}{2.539316in}}%
\pgfpathlineto{\pgfqpoint{5.124218in}{2.612519in}}%
\pgfpathlineto{\pgfqpoint{5.128727in}{2.558216in}}%
\pgfpathlineto{\pgfqpoint{5.133236in}{2.520169in}}%
\pgfpathlineto{\pgfqpoint{5.137745in}{2.578526in}}%
\pgfpathlineto{\pgfqpoint{5.142255in}{2.584518in}}%
\pgfpathlineto{\pgfqpoint{5.146764in}{2.587685in}}%
\pgfpathlineto{\pgfqpoint{5.151273in}{3.686178in}}%
\pgfpathlineto{\pgfqpoint{5.155782in}{3.078717in}}%
\pgfpathlineto{\pgfqpoint{5.160291in}{4.056000in}}%
\pgfpathlineto{\pgfqpoint{5.164800in}{3.075599in}}%
\pgfpathlineto{\pgfqpoint{5.169309in}{2.915784in}}%
\pgfpathlineto{\pgfqpoint{5.173818in}{2.669904in}}%
\pgfpathlineto{\pgfqpoint{5.178327in}{2.586558in}}%
\pgfpathlineto{\pgfqpoint{5.182836in}{2.613573in}}%
\pgfpathlineto{\pgfqpoint{5.187345in}{2.632916in}}%
\pgfpathlineto{\pgfqpoint{5.191855in}{2.684035in}}%
\pgfpathlineto{\pgfqpoint{5.196364in}{2.646620in}}%
\pgfpathlineto{\pgfqpoint{5.200873in}{2.645683in}}%
\pgfpathlineto{\pgfqpoint{5.205382in}{2.657372in}}%
\pgfpathlineto{\pgfqpoint{5.209891in}{2.594713in}}%
\pgfpathlineto{\pgfqpoint{5.218909in}{2.677316in}}%
\pgfpathlineto{\pgfqpoint{5.223418in}{2.682656in}}%
\pgfpathlineto{\pgfqpoint{5.227927in}{2.653118in}}%
\pgfpathlineto{\pgfqpoint{5.232436in}{2.652762in}}%
\pgfpathlineto{\pgfqpoint{5.241455in}{2.752057in}}%
\pgfpathlineto{\pgfqpoint{5.245964in}{2.689401in}}%
\pgfpathlineto{\pgfqpoint{5.250473in}{3.558727in}}%
\pgfpathlineto{\pgfqpoint{5.254982in}{2.764875in}}%
\pgfpathlineto{\pgfqpoint{5.259491in}{3.886136in}}%
\pgfpathlineto{\pgfqpoint{5.264000in}{3.258392in}}%
\pgfpathlineto{\pgfqpoint{5.273018in}{2.854119in}}%
\pgfpathlineto{\pgfqpoint{5.277527in}{2.709495in}}%
\pgfpathlineto{\pgfqpoint{5.282036in}{2.749507in}}%
\pgfpathlineto{\pgfqpoint{5.286545in}{2.682466in}}%
\pgfpathlineto{\pgfqpoint{5.291055in}{2.807601in}}%
\pgfpathlineto{\pgfqpoint{5.295564in}{2.693341in}}%
\pgfpathlineto{\pgfqpoint{5.300073in}{2.680888in}}%
\pgfpathlineto{\pgfqpoint{5.304582in}{2.720960in}}%
\pgfpathlineto{\pgfqpoint{5.309091in}{2.778631in}}%
\pgfpathlineto{\pgfqpoint{5.313600in}{2.669324in}}%
\pgfpathlineto{\pgfqpoint{5.318109in}{2.692847in}}%
\pgfpathlineto{\pgfqpoint{5.322618in}{2.680315in}}%
\pgfpathlineto{\pgfqpoint{5.327127in}{2.754862in}}%
\pgfpathlineto{\pgfqpoint{5.331636in}{2.737223in}}%
\pgfpathlineto{\pgfqpoint{5.336145in}{2.736727in}}%
\pgfpathlineto{\pgfqpoint{5.340655in}{2.734245in}}%
\pgfpathlineto{\pgfqpoint{5.345164in}{2.752231in}}%
\pgfpathlineto{\pgfqpoint{5.349673in}{2.652086in}}%
\pgfpathlineto{\pgfqpoint{5.354182in}{2.724862in}}%
\pgfpathlineto{\pgfqpoint{5.358691in}{2.831239in}}%
\pgfpathlineto{\pgfqpoint{5.363200in}{2.808736in}}%
\pgfpathlineto{\pgfqpoint{5.367709in}{2.796951in}}%
\pgfpathlineto{\pgfqpoint{5.372218in}{2.888464in}}%
\pgfpathlineto{\pgfqpoint{5.376727in}{2.727071in}}%
\pgfpathlineto{\pgfqpoint{5.381236in}{2.750306in}}%
\pgfpathlineto{\pgfqpoint{5.385745in}{2.764704in}}%
\pgfpathlineto{\pgfqpoint{5.390255in}{2.758905in}}%
\pgfpathlineto{\pgfqpoint{5.394764in}{2.803202in}}%
\pgfpathlineto{\pgfqpoint{5.399273in}{2.770703in}}%
\pgfpathlineto{\pgfqpoint{5.403782in}{2.819247in}}%
\pgfpathlineto{\pgfqpoint{5.408291in}{2.808466in}}%
\pgfpathlineto{\pgfqpoint{5.412800in}{2.699353in}}%
\pgfpathlineto{\pgfqpoint{5.421818in}{2.795142in}}%
\pgfpathlineto{\pgfqpoint{5.430836in}{2.917114in}}%
\pgfpathlineto{\pgfqpoint{5.435345in}{2.825283in}}%
\pgfpathlineto{\pgfqpoint{5.439855in}{2.778815in}}%
\pgfpathlineto{\pgfqpoint{5.444364in}{2.864164in}}%
\pgfpathlineto{\pgfqpoint{5.448873in}{2.798202in}}%
\pgfpathlineto{\pgfqpoint{5.453382in}{2.827114in}}%
\pgfpathlineto{\pgfqpoint{5.457891in}{2.846741in}}%
\pgfpathlineto{\pgfqpoint{5.462400in}{2.822478in}}%
\pgfpathlineto{\pgfqpoint{5.466909in}{2.865931in}}%
\pgfpathlineto{\pgfqpoint{5.471418in}{2.836793in}}%
\pgfpathlineto{\pgfqpoint{5.475927in}{2.827073in}}%
\pgfpathlineto{\pgfqpoint{5.480436in}{2.965651in}}%
\pgfpathlineto{\pgfqpoint{5.484945in}{2.839820in}}%
\pgfpathlineto{\pgfqpoint{5.489455in}{2.855647in}}%
\pgfpathlineto{\pgfqpoint{5.493964in}{2.917394in}}%
\pgfpathlineto{\pgfqpoint{5.498473in}{2.809700in}}%
\pgfpathlineto{\pgfqpoint{5.502982in}{2.921462in}}%
\pgfpathlineto{\pgfqpoint{5.507491in}{2.847293in}}%
\pgfpathlineto{\pgfqpoint{5.512000in}{2.866493in}}%
\pgfpathlineto{\pgfqpoint{5.516509in}{2.880357in}}%
\pgfpathlineto{\pgfqpoint{5.521018in}{2.961537in}}%
\pgfpathlineto{\pgfqpoint{5.525527in}{2.962523in}}%
\pgfpathlineto{\pgfqpoint{5.530036in}{2.931375in}}%
\pgfpathlineto{\pgfqpoint{5.534545in}{2.970248in}}%
\pgfpathlineto{\pgfqpoint{5.534545in}{2.970248in}}%
\pgfusepath{stroke}%
\end{pgfscope}%
\begin{pgfscope}%
\pgfpathrectangle{\pgfqpoint{0.800000in}{0.528000in}}{\pgfqpoint{4.960000in}{3.696000in}}%
\pgfusepath{clip}%
\pgfsetrectcap%
\pgfsetroundjoin%
\pgfsetlinewidth{1.505625pt}%
\definecolor{currentstroke}{rgb}{0.486275,0.988235,0.000000}%
\pgfsetstrokecolor{currentstroke}%
\pgfsetdash{}{0pt}%
\pgfpathmoveto{\pgfqpoint{1.025455in}{0.702897in}}%
\pgfpathlineto{\pgfqpoint{1.043491in}{0.703416in}}%
\pgfpathlineto{\pgfqpoint{1.052509in}{0.706296in}}%
\pgfpathlineto{\pgfqpoint{1.061527in}{0.705114in}}%
\pgfpathlineto{\pgfqpoint{1.066036in}{0.706657in}}%
\pgfpathlineto{\pgfqpoint{1.075055in}{0.704950in}}%
\pgfpathlineto{\pgfqpoint{1.084073in}{0.707556in}}%
\pgfpathlineto{\pgfqpoint{1.088582in}{0.707497in}}%
\pgfpathlineto{\pgfqpoint{1.093091in}{0.705589in}}%
\pgfpathlineto{\pgfqpoint{1.102109in}{0.709238in}}%
\pgfpathlineto{\pgfqpoint{1.106618in}{0.708726in}}%
\pgfpathlineto{\pgfqpoint{1.111127in}{0.706121in}}%
\pgfpathlineto{\pgfqpoint{1.120145in}{0.709154in}}%
\pgfpathlineto{\pgfqpoint{1.124655in}{0.708030in}}%
\pgfpathlineto{\pgfqpoint{1.129164in}{0.709301in}}%
\pgfpathlineto{\pgfqpoint{1.133673in}{0.707691in}}%
\pgfpathlineto{\pgfqpoint{1.151709in}{0.708807in}}%
\pgfpathlineto{\pgfqpoint{1.156218in}{0.710134in}}%
\pgfpathlineto{\pgfqpoint{1.160727in}{0.708665in}}%
\pgfpathlineto{\pgfqpoint{1.165236in}{0.710460in}}%
\pgfpathlineto{\pgfqpoint{1.169745in}{0.709087in}}%
\pgfpathlineto{\pgfqpoint{1.178764in}{0.712336in}}%
\pgfpathlineto{\pgfqpoint{1.187782in}{0.710954in}}%
\pgfpathlineto{\pgfqpoint{1.192291in}{0.712938in}}%
\pgfpathlineto{\pgfqpoint{1.196800in}{0.713239in}}%
\pgfpathlineto{\pgfqpoint{1.201309in}{0.715175in}}%
\pgfpathlineto{\pgfqpoint{1.205818in}{0.715389in}}%
\pgfpathlineto{\pgfqpoint{1.219345in}{0.711458in}}%
\pgfpathlineto{\pgfqpoint{1.223855in}{0.728568in}}%
\pgfpathlineto{\pgfqpoint{1.228364in}{0.712633in}}%
\pgfpathlineto{\pgfqpoint{1.232873in}{0.713568in}}%
\pgfpathlineto{\pgfqpoint{1.237382in}{0.715852in}}%
\pgfpathlineto{\pgfqpoint{1.241891in}{0.739612in}}%
\pgfpathlineto{\pgfqpoint{1.246400in}{0.714990in}}%
\pgfpathlineto{\pgfqpoint{1.250909in}{0.714024in}}%
\pgfpathlineto{\pgfqpoint{1.259927in}{0.717645in}}%
\pgfpathlineto{\pgfqpoint{1.264436in}{0.715333in}}%
\pgfpathlineto{\pgfqpoint{1.273455in}{0.716679in}}%
\pgfpathlineto{\pgfqpoint{1.277964in}{0.717981in}}%
\pgfpathlineto{\pgfqpoint{1.282473in}{0.716155in}}%
\pgfpathlineto{\pgfqpoint{1.286982in}{0.717651in}}%
\pgfpathlineto{\pgfqpoint{1.291491in}{0.716623in}}%
\pgfpathlineto{\pgfqpoint{1.296000in}{0.717999in}}%
\pgfpathlineto{\pgfqpoint{1.300509in}{0.715780in}}%
\pgfpathlineto{\pgfqpoint{1.305018in}{0.717258in}}%
\pgfpathlineto{\pgfqpoint{1.323055in}{0.717786in}}%
\pgfpathlineto{\pgfqpoint{1.332073in}{0.718269in}}%
\pgfpathlineto{\pgfqpoint{1.336582in}{0.720743in}}%
\pgfpathlineto{\pgfqpoint{1.341091in}{0.738666in}}%
\pgfpathlineto{\pgfqpoint{1.350109in}{0.746514in}}%
\pgfpathlineto{\pgfqpoint{1.354618in}{0.754835in}}%
\pgfpathlineto{\pgfqpoint{1.359127in}{0.751614in}}%
\pgfpathlineto{\pgfqpoint{1.363636in}{0.759075in}}%
\pgfpathlineto{\pgfqpoint{1.368145in}{0.763344in}}%
\pgfpathlineto{\pgfqpoint{1.372655in}{0.736998in}}%
\pgfpathlineto{\pgfqpoint{1.377164in}{0.736562in}}%
\pgfpathlineto{\pgfqpoint{1.381673in}{0.734846in}}%
\pgfpathlineto{\pgfqpoint{1.386182in}{0.736697in}}%
\pgfpathlineto{\pgfqpoint{1.395200in}{0.764225in}}%
\pgfpathlineto{\pgfqpoint{1.399709in}{0.757156in}}%
\pgfpathlineto{\pgfqpoint{1.404218in}{0.724358in}}%
\pgfpathlineto{\pgfqpoint{1.408727in}{0.724050in}}%
\pgfpathlineto{\pgfqpoint{1.413236in}{0.730903in}}%
\pgfpathlineto{\pgfqpoint{1.417745in}{0.724034in}}%
\pgfpathlineto{\pgfqpoint{1.489891in}{0.729447in}}%
\pgfpathlineto{\pgfqpoint{1.494400in}{0.731441in}}%
\pgfpathlineto{\pgfqpoint{1.498909in}{0.730195in}}%
\pgfpathlineto{\pgfqpoint{1.503418in}{0.735012in}}%
\pgfpathlineto{\pgfqpoint{1.507927in}{0.784212in}}%
\pgfpathlineto{\pgfqpoint{1.512436in}{0.779146in}}%
\pgfpathlineto{\pgfqpoint{1.516945in}{0.730916in}}%
\pgfpathlineto{\pgfqpoint{1.530473in}{0.732032in}}%
\pgfpathlineto{\pgfqpoint{1.534982in}{0.731968in}}%
\pgfpathlineto{\pgfqpoint{1.539491in}{0.737116in}}%
\pgfpathlineto{\pgfqpoint{1.544000in}{0.736491in}}%
\pgfpathlineto{\pgfqpoint{1.548509in}{0.738135in}}%
\pgfpathlineto{\pgfqpoint{1.553018in}{0.738207in}}%
\pgfpathlineto{\pgfqpoint{1.557527in}{0.739572in}}%
\pgfpathlineto{\pgfqpoint{1.562036in}{0.734560in}}%
\pgfpathlineto{\pgfqpoint{1.566545in}{0.737747in}}%
\pgfpathlineto{\pgfqpoint{1.571055in}{0.743432in}}%
\pgfpathlineto{\pgfqpoint{1.575564in}{0.739329in}}%
\pgfpathlineto{\pgfqpoint{1.580073in}{0.740536in}}%
\pgfpathlineto{\pgfqpoint{1.584582in}{0.745508in}}%
\pgfpathlineto{\pgfqpoint{1.589091in}{0.763527in}}%
\pgfpathlineto{\pgfqpoint{1.593600in}{0.761085in}}%
\pgfpathlineto{\pgfqpoint{1.602618in}{0.751377in}}%
\pgfpathlineto{\pgfqpoint{1.607127in}{0.761887in}}%
\pgfpathlineto{\pgfqpoint{1.611636in}{0.779664in}}%
\pgfpathlineto{\pgfqpoint{1.616145in}{0.783224in}}%
\pgfpathlineto{\pgfqpoint{1.620655in}{0.793949in}}%
\pgfpathlineto{\pgfqpoint{1.629673in}{0.752514in}}%
\pgfpathlineto{\pgfqpoint{1.634182in}{0.749613in}}%
\pgfpathlineto{\pgfqpoint{1.638691in}{0.740571in}}%
\pgfpathlineto{\pgfqpoint{1.643200in}{0.743251in}}%
\pgfpathlineto{\pgfqpoint{1.647709in}{0.741016in}}%
\pgfpathlineto{\pgfqpoint{1.652218in}{0.743336in}}%
\pgfpathlineto{\pgfqpoint{1.656727in}{0.743590in}}%
\pgfpathlineto{\pgfqpoint{1.661236in}{0.745156in}}%
\pgfpathlineto{\pgfqpoint{1.665745in}{0.745480in}}%
\pgfpathlineto{\pgfqpoint{1.670255in}{0.781793in}}%
\pgfpathlineto{\pgfqpoint{1.674764in}{0.768773in}}%
\pgfpathlineto{\pgfqpoint{1.679273in}{0.799230in}}%
\pgfpathlineto{\pgfqpoint{1.683782in}{0.744805in}}%
\pgfpathlineto{\pgfqpoint{1.688291in}{0.751696in}}%
\pgfpathlineto{\pgfqpoint{1.692800in}{0.751853in}}%
\pgfpathlineto{\pgfqpoint{1.697309in}{0.753542in}}%
\pgfpathlineto{\pgfqpoint{1.701818in}{0.765027in}}%
\pgfpathlineto{\pgfqpoint{1.706327in}{0.766286in}}%
\pgfpathlineto{\pgfqpoint{1.710836in}{0.763715in}}%
\pgfpathlineto{\pgfqpoint{1.715345in}{0.770980in}}%
\pgfpathlineto{\pgfqpoint{1.719855in}{0.760357in}}%
\pgfpathlineto{\pgfqpoint{1.724364in}{0.754086in}}%
\pgfpathlineto{\pgfqpoint{1.728873in}{0.753176in}}%
\pgfpathlineto{\pgfqpoint{1.733382in}{0.750915in}}%
\pgfpathlineto{\pgfqpoint{1.737891in}{0.753385in}}%
\pgfpathlineto{\pgfqpoint{1.742400in}{0.753129in}}%
\pgfpathlineto{\pgfqpoint{1.746909in}{0.754979in}}%
\pgfpathlineto{\pgfqpoint{1.751418in}{0.800518in}}%
\pgfpathlineto{\pgfqpoint{1.755927in}{0.790232in}}%
\pgfpathlineto{\pgfqpoint{1.760436in}{0.773530in}}%
\pgfpathlineto{\pgfqpoint{1.764945in}{0.762188in}}%
\pgfpathlineto{\pgfqpoint{1.769455in}{0.760201in}}%
\pgfpathlineto{\pgfqpoint{1.773964in}{0.761358in}}%
\pgfpathlineto{\pgfqpoint{1.778473in}{0.764017in}}%
\pgfpathlineto{\pgfqpoint{1.782982in}{0.830131in}}%
\pgfpathlineto{\pgfqpoint{1.787491in}{0.774486in}}%
\pgfpathlineto{\pgfqpoint{1.792000in}{0.801998in}}%
\pgfpathlineto{\pgfqpoint{1.796509in}{0.795532in}}%
\pgfpathlineto{\pgfqpoint{1.801018in}{0.770171in}}%
\pgfpathlineto{\pgfqpoint{1.805527in}{0.802047in}}%
\pgfpathlineto{\pgfqpoint{1.810036in}{0.766231in}}%
\pgfpathlineto{\pgfqpoint{1.814545in}{0.763156in}}%
\pgfpathlineto{\pgfqpoint{1.819055in}{0.784776in}}%
\pgfpathlineto{\pgfqpoint{1.823564in}{0.771730in}}%
\pgfpathlineto{\pgfqpoint{1.828073in}{0.765751in}}%
\pgfpathlineto{\pgfqpoint{1.832582in}{0.764274in}}%
\pgfpathlineto{\pgfqpoint{1.837091in}{0.773151in}}%
\pgfpathlineto{\pgfqpoint{1.841600in}{0.768143in}}%
\pgfpathlineto{\pgfqpoint{1.846109in}{0.765264in}}%
\pgfpathlineto{\pgfqpoint{1.850618in}{0.802807in}}%
\pgfpathlineto{\pgfqpoint{1.855127in}{0.807841in}}%
\pgfpathlineto{\pgfqpoint{1.859636in}{0.779684in}}%
\pgfpathlineto{\pgfqpoint{1.864145in}{0.764083in}}%
\pgfpathlineto{\pgfqpoint{1.877673in}{0.767633in}}%
\pgfpathlineto{\pgfqpoint{1.882182in}{0.813604in}}%
\pgfpathlineto{\pgfqpoint{1.886691in}{0.770675in}}%
\pgfpathlineto{\pgfqpoint{1.891200in}{0.767799in}}%
\pgfpathlineto{\pgfqpoint{1.895709in}{0.769651in}}%
\pgfpathlineto{\pgfqpoint{1.900218in}{0.791327in}}%
\pgfpathlineto{\pgfqpoint{1.904727in}{0.831237in}}%
\pgfpathlineto{\pgfqpoint{1.909236in}{0.779608in}}%
\pgfpathlineto{\pgfqpoint{1.913745in}{0.773955in}}%
\pgfpathlineto{\pgfqpoint{1.918255in}{0.770942in}}%
\pgfpathlineto{\pgfqpoint{1.922764in}{0.772819in}}%
\pgfpathlineto{\pgfqpoint{1.927273in}{0.772994in}}%
\pgfpathlineto{\pgfqpoint{1.931782in}{0.774310in}}%
\pgfpathlineto{\pgfqpoint{1.940800in}{0.774641in}}%
\pgfpathlineto{\pgfqpoint{1.945309in}{0.776318in}}%
\pgfpathlineto{\pgfqpoint{1.949818in}{0.779875in}}%
\pgfpathlineto{\pgfqpoint{1.954327in}{0.808566in}}%
\pgfpathlineto{\pgfqpoint{1.958836in}{0.784011in}}%
\pgfpathlineto{\pgfqpoint{1.963345in}{0.786553in}}%
\pgfpathlineto{\pgfqpoint{1.967855in}{0.778320in}}%
\pgfpathlineto{\pgfqpoint{1.972364in}{0.787362in}}%
\pgfpathlineto{\pgfqpoint{1.976873in}{0.810836in}}%
\pgfpathlineto{\pgfqpoint{1.981382in}{0.781069in}}%
\pgfpathlineto{\pgfqpoint{1.985891in}{0.798475in}}%
\pgfpathlineto{\pgfqpoint{1.990400in}{0.806869in}}%
\pgfpathlineto{\pgfqpoint{1.994909in}{0.783363in}}%
\pgfpathlineto{\pgfqpoint{1.999418in}{0.782153in}}%
\pgfpathlineto{\pgfqpoint{2.003927in}{0.784619in}}%
\pgfpathlineto{\pgfqpoint{2.008436in}{0.799031in}}%
\pgfpathlineto{\pgfqpoint{2.012945in}{0.787821in}}%
\pgfpathlineto{\pgfqpoint{2.017455in}{0.794053in}}%
\pgfpathlineto{\pgfqpoint{2.021964in}{0.791877in}}%
\pgfpathlineto{\pgfqpoint{2.030982in}{0.802881in}}%
\pgfpathlineto{\pgfqpoint{2.035491in}{0.793077in}}%
\pgfpathlineto{\pgfqpoint{2.044509in}{0.808522in}}%
\pgfpathlineto{\pgfqpoint{2.049018in}{0.814279in}}%
\pgfpathlineto{\pgfqpoint{2.053527in}{0.856387in}}%
\pgfpathlineto{\pgfqpoint{2.058036in}{0.801823in}}%
\pgfpathlineto{\pgfqpoint{2.062545in}{0.800978in}}%
\pgfpathlineto{\pgfqpoint{2.067055in}{0.801505in}}%
\pgfpathlineto{\pgfqpoint{2.071564in}{0.799215in}}%
\pgfpathlineto{\pgfqpoint{2.076073in}{0.799082in}}%
\pgfpathlineto{\pgfqpoint{2.080582in}{0.804143in}}%
\pgfpathlineto{\pgfqpoint{2.085091in}{0.837534in}}%
\pgfpathlineto{\pgfqpoint{2.089600in}{0.798221in}}%
\pgfpathlineto{\pgfqpoint{2.094109in}{0.824039in}}%
\pgfpathlineto{\pgfqpoint{2.098618in}{0.801141in}}%
\pgfpathlineto{\pgfqpoint{2.103127in}{0.819208in}}%
\pgfpathlineto{\pgfqpoint{2.107636in}{0.894869in}}%
\pgfpathlineto{\pgfqpoint{2.112145in}{0.809683in}}%
\pgfpathlineto{\pgfqpoint{2.116655in}{0.831780in}}%
\pgfpathlineto{\pgfqpoint{2.121164in}{0.869236in}}%
\pgfpathlineto{\pgfqpoint{2.125673in}{0.833487in}}%
\pgfpathlineto{\pgfqpoint{2.130182in}{0.875646in}}%
\pgfpathlineto{\pgfqpoint{2.134691in}{0.976513in}}%
\pgfpathlineto{\pgfqpoint{2.139200in}{0.919333in}}%
\pgfpathlineto{\pgfqpoint{2.143709in}{0.838616in}}%
\pgfpathlineto{\pgfqpoint{2.148218in}{0.824965in}}%
\pgfpathlineto{\pgfqpoint{2.152727in}{0.805190in}}%
\pgfpathlineto{\pgfqpoint{2.157236in}{0.821591in}}%
\pgfpathlineto{\pgfqpoint{2.161745in}{0.829818in}}%
\pgfpathlineto{\pgfqpoint{2.166255in}{0.885502in}}%
\pgfpathlineto{\pgfqpoint{2.170764in}{0.842878in}}%
\pgfpathlineto{\pgfqpoint{2.175273in}{0.931441in}}%
\pgfpathlineto{\pgfqpoint{2.184291in}{0.831626in}}%
\pgfpathlineto{\pgfqpoint{2.188800in}{0.935128in}}%
\pgfpathlineto{\pgfqpoint{2.193309in}{0.912606in}}%
\pgfpathlineto{\pgfqpoint{2.197818in}{0.977687in}}%
\pgfpathlineto{\pgfqpoint{2.206836in}{0.840158in}}%
\pgfpathlineto{\pgfqpoint{2.211345in}{0.908466in}}%
\pgfpathlineto{\pgfqpoint{2.215855in}{0.835099in}}%
\pgfpathlineto{\pgfqpoint{2.220364in}{0.849735in}}%
\pgfpathlineto{\pgfqpoint{2.224873in}{0.852291in}}%
\pgfpathlineto{\pgfqpoint{2.229382in}{0.863833in}}%
\pgfpathlineto{\pgfqpoint{2.233891in}{0.839959in}}%
\pgfpathlineto{\pgfqpoint{2.238400in}{0.839331in}}%
\pgfpathlineto{\pgfqpoint{2.242909in}{0.842266in}}%
\pgfpathlineto{\pgfqpoint{2.247418in}{0.837803in}}%
\pgfpathlineto{\pgfqpoint{2.251927in}{0.836335in}}%
\pgfpathlineto{\pgfqpoint{2.260945in}{0.837701in}}%
\pgfpathlineto{\pgfqpoint{2.265455in}{0.847079in}}%
\pgfpathlineto{\pgfqpoint{2.269964in}{0.863330in}}%
\pgfpathlineto{\pgfqpoint{2.274473in}{0.845609in}}%
\pgfpathlineto{\pgfqpoint{2.278982in}{0.868701in}}%
\pgfpathlineto{\pgfqpoint{2.283491in}{0.839241in}}%
\pgfpathlineto{\pgfqpoint{2.288000in}{0.839222in}}%
\pgfpathlineto{\pgfqpoint{2.292509in}{0.847364in}}%
\pgfpathlineto{\pgfqpoint{2.297018in}{0.852606in}}%
\pgfpathlineto{\pgfqpoint{2.301527in}{0.843117in}}%
\pgfpathlineto{\pgfqpoint{2.306036in}{0.851505in}}%
\pgfpathlineto{\pgfqpoint{2.310545in}{0.852264in}}%
\pgfpathlineto{\pgfqpoint{2.315055in}{0.844948in}}%
\pgfpathlineto{\pgfqpoint{2.319564in}{0.918950in}}%
\pgfpathlineto{\pgfqpoint{2.324073in}{0.852398in}}%
\pgfpathlineto{\pgfqpoint{2.328582in}{0.855629in}}%
\pgfpathlineto{\pgfqpoint{2.333091in}{0.856079in}}%
\pgfpathlineto{\pgfqpoint{2.337600in}{0.849787in}}%
\pgfpathlineto{\pgfqpoint{2.342109in}{0.849203in}}%
\pgfpathlineto{\pgfqpoint{2.346618in}{0.852238in}}%
\pgfpathlineto{\pgfqpoint{2.351127in}{0.858685in}}%
\pgfpathlineto{\pgfqpoint{2.355636in}{0.856639in}}%
\pgfpathlineto{\pgfqpoint{2.360145in}{0.863274in}}%
\pgfpathlineto{\pgfqpoint{2.364655in}{0.859266in}}%
\pgfpathlineto{\pgfqpoint{2.369164in}{0.867341in}}%
\pgfpathlineto{\pgfqpoint{2.373673in}{0.861327in}}%
\pgfpathlineto{\pgfqpoint{2.382691in}{0.865873in}}%
\pgfpathlineto{\pgfqpoint{2.387200in}{0.862398in}}%
\pgfpathlineto{\pgfqpoint{2.391709in}{0.866728in}}%
\pgfpathlineto{\pgfqpoint{2.396218in}{0.882188in}}%
\pgfpathlineto{\pgfqpoint{2.400727in}{0.891370in}}%
\pgfpathlineto{\pgfqpoint{2.405236in}{0.870323in}}%
\pgfpathlineto{\pgfqpoint{2.409745in}{0.868745in}}%
\pgfpathlineto{\pgfqpoint{2.414255in}{0.871150in}}%
\pgfpathlineto{\pgfqpoint{2.418764in}{0.865621in}}%
\pgfpathlineto{\pgfqpoint{2.423273in}{0.871156in}}%
\pgfpathlineto{\pgfqpoint{2.427782in}{0.886889in}}%
\pgfpathlineto{\pgfqpoint{2.432291in}{0.862549in}}%
\pgfpathlineto{\pgfqpoint{2.436800in}{0.874798in}}%
\pgfpathlineto{\pgfqpoint{2.441309in}{0.877436in}}%
\pgfpathlineto{\pgfqpoint{2.450327in}{0.886361in}}%
\pgfpathlineto{\pgfqpoint{2.454836in}{0.884447in}}%
\pgfpathlineto{\pgfqpoint{2.459345in}{0.898000in}}%
\pgfpathlineto{\pgfqpoint{2.463855in}{0.880754in}}%
\pgfpathlineto{\pgfqpoint{2.468364in}{0.885111in}}%
\pgfpathlineto{\pgfqpoint{2.472873in}{0.874688in}}%
\pgfpathlineto{\pgfqpoint{2.477382in}{0.903714in}}%
\pgfpathlineto{\pgfqpoint{2.481891in}{0.910741in}}%
\pgfpathlineto{\pgfqpoint{2.486400in}{0.890153in}}%
\pgfpathlineto{\pgfqpoint{2.490909in}{0.890395in}}%
\pgfpathlineto{\pgfqpoint{2.499927in}{0.894831in}}%
\pgfpathlineto{\pgfqpoint{2.504436in}{0.887065in}}%
\pgfpathlineto{\pgfqpoint{2.508945in}{0.891593in}}%
\pgfpathlineto{\pgfqpoint{2.513455in}{0.883996in}}%
\pgfpathlineto{\pgfqpoint{2.517964in}{0.892866in}}%
\pgfpathlineto{\pgfqpoint{2.522473in}{0.889704in}}%
\pgfpathlineto{\pgfqpoint{2.526982in}{0.911434in}}%
\pgfpathlineto{\pgfqpoint{2.531491in}{0.895381in}}%
\pgfpathlineto{\pgfqpoint{2.536000in}{0.895820in}}%
\pgfpathlineto{\pgfqpoint{2.540509in}{0.891578in}}%
\pgfpathlineto{\pgfqpoint{2.545018in}{0.896488in}}%
\pgfpathlineto{\pgfqpoint{2.549527in}{0.896128in}}%
\pgfpathlineto{\pgfqpoint{2.554036in}{0.898678in}}%
\pgfpathlineto{\pgfqpoint{2.558545in}{0.898159in}}%
\pgfpathlineto{\pgfqpoint{2.563055in}{0.914747in}}%
\pgfpathlineto{\pgfqpoint{2.567564in}{0.898962in}}%
\pgfpathlineto{\pgfqpoint{2.572073in}{0.904544in}}%
\pgfpathlineto{\pgfqpoint{2.576582in}{0.905711in}}%
\pgfpathlineto{\pgfqpoint{2.581091in}{0.902065in}}%
\pgfpathlineto{\pgfqpoint{2.585600in}{0.901756in}}%
\pgfpathlineto{\pgfqpoint{2.590109in}{0.923700in}}%
\pgfpathlineto{\pgfqpoint{2.594618in}{0.908317in}}%
\pgfpathlineto{\pgfqpoint{2.599127in}{0.904656in}}%
\pgfpathlineto{\pgfqpoint{2.603636in}{0.910042in}}%
\pgfpathlineto{\pgfqpoint{2.608145in}{0.911033in}}%
\pgfpathlineto{\pgfqpoint{2.612655in}{0.909883in}}%
\pgfpathlineto{\pgfqpoint{2.617164in}{0.907232in}}%
\pgfpathlineto{\pgfqpoint{2.621673in}{0.909821in}}%
\pgfpathlineto{\pgfqpoint{2.626182in}{0.915129in}}%
\pgfpathlineto{\pgfqpoint{2.630691in}{0.916735in}}%
\pgfpathlineto{\pgfqpoint{2.639709in}{0.913965in}}%
\pgfpathlineto{\pgfqpoint{2.644218in}{0.916537in}}%
\pgfpathlineto{\pgfqpoint{2.648727in}{0.920562in}}%
\pgfpathlineto{\pgfqpoint{2.653236in}{0.913554in}}%
\pgfpathlineto{\pgfqpoint{2.657745in}{0.922831in}}%
\pgfpathlineto{\pgfqpoint{2.662255in}{0.926539in}}%
\pgfpathlineto{\pgfqpoint{2.666764in}{0.920400in}}%
\pgfpathlineto{\pgfqpoint{2.671273in}{0.952672in}}%
\pgfpathlineto{\pgfqpoint{2.675782in}{0.927754in}}%
\pgfpathlineto{\pgfqpoint{2.680291in}{0.933795in}}%
\pgfpathlineto{\pgfqpoint{2.684800in}{0.933275in}}%
\pgfpathlineto{\pgfqpoint{2.689309in}{0.929670in}}%
\pgfpathlineto{\pgfqpoint{2.693818in}{0.936005in}}%
\pgfpathlineto{\pgfqpoint{2.698327in}{0.926343in}}%
\pgfpathlineto{\pgfqpoint{2.702836in}{0.931131in}}%
\pgfpathlineto{\pgfqpoint{2.707345in}{0.939090in}}%
\pgfpathlineto{\pgfqpoint{2.711855in}{0.951687in}}%
\pgfpathlineto{\pgfqpoint{2.716364in}{0.936988in}}%
\pgfpathlineto{\pgfqpoint{2.725382in}{0.938710in}}%
\pgfpathlineto{\pgfqpoint{2.729891in}{0.942311in}}%
\pgfpathlineto{\pgfqpoint{2.734400in}{0.961279in}}%
\pgfpathlineto{\pgfqpoint{2.738909in}{0.939186in}}%
\pgfpathlineto{\pgfqpoint{2.743418in}{0.941874in}}%
\pgfpathlineto{\pgfqpoint{2.752436in}{0.941996in}}%
\pgfpathlineto{\pgfqpoint{2.756945in}{0.947380in}}%
\pgfpathlineto{\pgfqpoint{2.770473in}{0.944309in}}%
\pgfpathlineto{\pgfqpoint{2.774982in}{0.950440in}}%
\pgfpathlineto{\pgfqpoint{2.779491in}{0.952291in}}%
\pgfpathlineto{\pgfqpoint{2.784000in}{0.949008in}}%
\pgfpathlineto{\pgfqpoint{2.788509in}{0.947914in}}%
\pgfpathlineto{\pgfqpoint{2.793018in}{0.951162in}}%
\pgfpathlineto{\pgfqpoint{2.797527in}{0.942709in}}%
\pgfpathlineto{\pgfqpoint{2.806545in}{0.954915in}}%
\pgfpathlineto{\pgfqpoint{2.811055in}{0.955575in}}%
\pgfpathlineto{\pgfqpoint{2.815564in}{0.959609in}}%
\pgfpathlineto{\pgfqpoint{2.820073in}{0.968018in}}%
\pgfpathlineto{\pgfqpoint{2.824582in}{0.954253in}}%
\pgfpathlineto{\pgfqpoint{2.829091in}{0.982698in}}%
\pgfpathlineto{\pgfqpoint{2.833600in}{0.971980in}}%
\pgfpathlineto{\pgfqpoint{2.838109in}{0.957360in}}%
\pgfpathlineto{\pgfqpoint{2.842618in}{0.964416in}}%
\pgfpathlineto{\pgfqpoint{2.847127in}{0.966987in}}%
\pgfpathlineto{\pgfqpoint{2.851636in}{0.962780in}}%
\pgfpathlineto{\pgfqpoint{2.856145in}{0.972923in}}%
\pgfpathlineto{\pgfqpoint{2.860655in}{0.973311in}}%
\pgfpathlineto{\pgfqpoint{2.865164in}{0.963369in}}%
\pgfpathlineto{\pgfqpoint{2.869673in}{0.962101in}}%
\pgfpathlineto{\pgfqpoint{2.874182in}{0.981945in}}%
\pgfpathlineto{\pgfqpoint{2.878691in}{0.991532in}}%
\pgfpathlineto{\pgfqpoint{2.883200in}{1.028604in}}%
\pgfpathlineto{\pgfqpoint{2.887709in}{0.997229in}}%
\pgfpathlineto{\pgfqpoint{2.892218in}{0.984485in}}%
\pgfpathlineto{\pgfqpoint{2.896727in}{0.992894in}}%
\pgfpathlineto{\pgfqpoint{2.901236in}{0.986071in}}%
\pgfpathlineto{\pgfqpoint{2.905745in}{0.982977in}}%
\pgfpathlineto{\pgfqpoint{2.910255in}{0.997051in}}%
\pgfpathlineto{\pgfqpoint{2.914764in}{0.999795in}}%
\pgfpathlineto{\pgfqpoint{2.919273in}{0.985358in}}%
\pgfpathlineto{\pgfqpoint{2.928291in}{0.989937in}}%
\pgfpathlineto{\pgfqpoint{2.932800in}{0.983891in}}%
\pgfpathlineto{\pgfqpoint{2.937309in}{0.987712in}}%
\pgfpathlineto{\pgfqpoint{2.941818in}{1.001735in}}%
\pgfpathlineto{\pgfqpoint{2.946327in}{1.000606in}}%
\pgfpathlineto{\pgfqpoint{2.950836in}{0.988165in}}%
\pgfpathlineto{\pgfqpoint{2.959855in}{0.990647in}}%
\pgfpathlineto{\pgfqpoint{2.964364in}{1.005607in}}%
\pgfpathlineto{\pgfqpoint{2.968873in}{1.051789in}}%
\pgfpathlineto{\pgfqpoint{2.973382in}{0.992941in}}%
\pgfpathlineto{\pgfqpoint{2.977891in}{0.995405in}}%
\pgfpathlineto{\pgfqpoint{2.982400in}{1.082462in}}%
\pgfpathlineto{\pgfqpoint{2.986909in}{0.998497in}}%
\pgfpathlineto{\pgfqpoint{2.991418in}{0.993815in}}%
\pgfpathlineto{\pgfqpoint{2.995927in}{1.008351in}}%
\pgfpathlineto{\pgfqpoint{3.000436in}{0.992963in}}%
\pgfpathlineto{\pgfqpoint{3.004945in}{1.002094in}}%
\pgfpathlineto{\pgfqpoint{3.009455in}{0.997811in}}%
\pgfpathlineto{\pgfqpoint{3.013964in}{1.019601in}}%
\pgfpathlineto{\pgfqpoint{3.018473in}{1.002068in}}%
\pgfpathlineto{\pgfqpoint{3.022982in}{1.008016in}}%
\pgfpathlineto{\pgfqpoint{3.027491in}{1.010872in}}%
\pgfpathlineto{\pgfqpoint{3.032000in}{1.011212in}}%
\pgfpathlineto{\pgfqpoint{3.036509in}{1.004846in}}%
\pgfpathlineto{\pgfqpoint{3.041018in}{1.017773in}}%
\pgfpathlineto{\pgfqpoint{3.045527in}{1.014780in}}%
\pgfpathlineto{\pgfqpoint{3.054545in}{1.015140in}}%
\pgfpathlineto{\pgfqpoint{3.059055in}{1.027138in}}%
\pgfpathlineto{\pgfqpoint{3.068073in}{1.034766in}}%
\pgfpathlineto{\pgfqpoint{3.072582in}{1.028919in}}%
\pgfpathlineto{\pgfqpoint{3.077091in}{1.027391in}}%
\pgfpathlineto{\pgfqpoint{3.081600in}{1.058213in}}%
\pgfpathlineto{\pgfqpoint{3.086109in}{1.041211in}}%
\pgfpathlineto{\pgfqpoint{3.090618in}{1.019240in}}%
\pgfpathlineto{\pgfqpoint{3.095127in}{1.033045in}}%
\pgfpathlineto{\pgfqpoint{3.099636in}{1.019926in}}%
\pgfpathlineto{\pgfqpoint{3.104145in}{1.040511in}}%
\pgfpathlineto{\pgfqpoint{3.113164in}{1.032146in}}%
\pgfpathlineto{\pgfqpoint{3.117673in}{1.039522in}}%
\pgfpathlineto{\pgfqpoint{3.122182in}{1.038276in}}%
\pgfpathlineto{\pgfqpoint{3.126691in}{1.031463in}}%
\pgfpathlineto{\pgfqpoint{3.131200in}{1.051430in}}%
\pgfpathlineto{\pgfqpoint{3.135709in}{1.044454in}}%
\pgfpathlineto{\pgfqpoint{3.144727in}{1.058757in}}%
\pgfpathlineto{\pgfqpoint{3.153745in}{1.067537in}}%
\pgfpathlineto{\pgfqpoint{3.158255in}{1.047712in}}%
\pgfpathlineto{\pgfqpoint{3.162764in}{1.051458in}}%
\pgfpathlineto{\pgfqpoint{3.167273in}{1.048201in}}%
\pgfpathlineto{\pgfqpoint{3.171782in}{1.053966in}}%
\pgfpathlineto{\pgfqpoint{3.176291in}{1.065099in}}%
\pgfpathlineto{\pgfqpoint{3.180800in}{1.048571in}}%
\pgfpathlineto{\pgfqpoint{3.185309in}{1.046141in}}%
\pgfpathlineto{\pgfqpoint{3.189818in}{1.052362in}}%
\pgfpathlineto{\pgfqpoint{3.194327in}{1.063955in}}%
\pgfpathlineto{\pgfqpoint{3.203345in}{1.054940in}}%
\pgfpathlineto{\pgfqpoint{3.207855in}{1.055863in}}%
\pgfpathlineto{\pgfqpoint{3.212364in}{1.068133in}}%
\pgfpathlineto{\pgfqpoint{3.216873in}{1.063757in}}%
\pgfpathlineto{\pgfqpoint{3.221382in}{1.065079in}}%
\pgfpathlineto{\pgfqpoint{3.230400in}{1.086836in}}%
\pgfpathlineto{\pgfqpoint{3.234909in}{1.069245in}}%
\pgfpathlineto{\pgfqpoint{3.239418in}{1.073992in}}%
\pgfpathlineto{\pgfqpoint{3.243927in}{1.082051in}}%
\pgfpathlineto{\pgfqpoint{3.248436in}{1.080956in}}%
\pgfpathlineto{\pgfqpoint{3.252945in}{1.104548in}}%
\pgfpathlineto{\pgfqpoint{3.257455in}{1.067995in}}%
\pgfpathlineto{\pgfqpoint{3.261964in}{1.077305in}}%
\pgfpathlineto{\pgfqpoint{3.266473in}{1.078648in}}%
\pgfpathlineto{\pgfqpoint{3.270982in}{1.090147in}}%
\pgfpathlineto{\pgfqpoint{3.275491in}{1.111933in}}%
\pgfpathlineto{\pgfqpoint{3.280000in}{1.084796in}}%
\pgfpathlineto{\pgfqpoint{3.284509in}{1.081497in}}%
\pgfpathlineto{\pgfqpoint{3.289018in}{1.081376in}}%
\pgfpathlineto{\pgfqpoint{3.293527in}{1.107258in}}%
\pgfpathlineto{\pgfqpoint{3.298036in}{1.090774in}}%
\pgfpathlineto{\pgfqpoint{3.302545in}{1.092833in}}%
\pgfpathlineto{\pgfqpoint{3.307055in}{1.089294in}}%
\pgfpathlineto{\pgfqpoint{3.311564in}{1.125530in}}%
\pgfpathlineto{\pgfqpoint{3.316073in}{1.093425in}}%
\pgfpathlineto{\pgfqpoint{3.320582in}{1.095988in}}%
\pgfpathlineto{\pgfqpoint{3.325091in}{1.159897in}}%
\pgfpathlineto{\pgfqpoint{3.329600in}{1.094730in}}%
\pgfpathlineto{\pgfqpoint{3.334109in}{1.108255in}}%
\pgfpathlineto{\pgfqpoint{3.338618in}{1.136252in}}%
\pgfpathlineto{\pgfqpoint{3.343127in}{1.111693in}}%
\pgfpathlineto{\pgfqpoint{3.347636in}{1.101600in}}%
\pgfpathlineto{\pgfqpoint{3.352145in}{1.104186in}}%
\pgfpathlineto{\pgfqpoint{3.356655in}{1.145671in}}%
\pgfpathlineto{\pgfqpoint{3.361164in}{1.111533in}}%
\pgfpathlineto{\pgfqpoint{3.365673in}{1.103728in}}%
\pgfpathlineto{\pgfqpoint{3.370182in}{1.131782in}}%
\pgfpathlineto{\pgfqpoint{3.374691in}{1.108462in}}%
\pgfpathlineto{\pgfqpoint{3.379200in}{1.108151in}}%
\pgfpathlineto{\pgfqpoint{3.383709in}{1.143746in}}%
\pgfpathlineto{\pgfqpoint{3.392727in}{1.115657in}}%
\pgfpathlineto{\pgfqpoint{3.397236in}{1.117636in}}%
\pgfpathlineto{\pgfqpoint{3.401745in}{1.111204in}}%
\pgfpathlineto{\pgfqpoint{3.406255in}{1.133038in}}%
\pgfpathlineto{\pgfqpoint{3.410764in}{1.139591in}}%
\pgfpathlineto{\pgfqpoint{3.415273in}{1.125396in}}%
\pgfpathlineto{\pgfqpoint{3.419782in}{1.160952in}}%
\pgfpathlineto{\pgfqpoint{3.424291in}{1.124162in}}%
\pgfpathlineto{\pgfqpoint{3.428800in}{1.131123in}}%
\pgfpathlineto{\pgfqpoint{3.433309in}{1.163235in}}%
\pgfpathlineto{\pgfqpoint{3.437818in}{1.132235in}}%
\pgfpathlineto{\pgfqpoint{3.442327in}{1.156806in}}%
\pgfpathlineto{\pgfqpoint{3.446836in}{1.138957in}}%
\pgfpathlineto{\pgfqpoint{3.451345in}{1.138709in}}%
\pgfpathlineto{\pgfqpoint{3.455855in}{1.177412in}}%
\pgfpathlineto{\pgfqpoint{3.460364in}{1.134597in}}%
\pgfpathlineto{\pgfqpoint{3.464873in}{1.149637in}}%
\pgfpathlineto{\pgfqpoint{3.469382in}{1.136178in}}%
\pgfpathlineto{\pgfqpoint{3.478400in}{1.170757in}}%
\pgfpathlineto{\pgfqpoint{3.482909in}{1.148726in}}%
\pgfpathlineto{\pgfqpoint{3.487418in}{1.154660in}}%
\pgfpathlineto{\pgfqpoint{3.491927in}{1.151686in}}%
\pgfpathlineto{\pgfqpoint{3.496436in}{1.170364in}}%
\pgfpathlineto{\pgfqpoint{3.500945in}{1.150884in}}%
\pgfpathlineto{\pgfqpoint{3.505455in}{1.168796in}}%
\pgfpathlineto{\pgfqpoint{3.509964in}{1.156216in}}%
\pgfpathlineto{\pgfqpoint{3.514473in}{1.158693in}}%
\pgfpathlineto{\pgfqpoint{3.518982in}{1.163455in}}%
\pgfpathlineto{\pgfqpoint{3.523491in}{1.188244in}}%
\pgfpathlineto{\pgfqpoint{3.528000in}{1.162909in}}%
\pgfpathlineto{\pgfqpoint{3.532509in}{1.162528in}}%
\pgfpathlineto{\pgfqpoint{3.537018in}{1.173446in}}%
\pgfpathlineto{\pgfqpoint{3.541527in}{1.201345in}}%
\pgfpathlineto{\pgfqpoint{3.546036in}{1.175180in}}%
\pgfpathlineto{\pgfqpoint{3.550545in}{1.172131in}}%
\pgfpathlineto{\pgfqpoint{3.555055in}{1.178600in}}%
\pgfpathlineto{\pgfqpoint{3.559564in}{1.187034in}}%
\pgfpathlineto{\pgfqpoint{3.564073in}{1.211928in}}%
\pgfpathlineto{\pgfqpoint{3.568582in}{1.163922in}}%
\pgfpathlineto{\pgfqpoint{3.573091in}{1.199582in}}%
\pgfpathlineto{\pgfqpoint{3.577600in}{1.180643in}}%
\pgfpathlineto{\pgfqpoint{3.582109in}{1.175459in}}%
\pgfpathlineto{\pgfqpoint{3.586618in}{1.188536in}}%
\pgfpathlineto{\pgfqpoint{3.591127in}{1.213241in}}%
\pgfpathlineto{\pgfqpoint{3.595636in}{1.190151in}}%
\pgfpathlineto{\pgfqpoint{3.600145in}{1.190755in}}%
\pgfpathlineto{\pgfqpoint{3.609164in}{1.200789in}}%
\pgfpathlineto{\pgfqpoint{3.613673in}{1.218754in}}%
\pgfpathlineto{\pgfqpoint{3.618182in}{1.227362in}}%
\pgfpathlineto{\pgfqpoint{3.622691in}{1.197518in}}%
\pgfpathlineto{\pgfqpoint{3.627200in}{1.191157in}}%
\pgfpathlineto{\pgfqpoint{3.631709in}{1.238881in}}%
\pgfpathlineto{\pgfqpoint{3.636218in}{1.190623in}}%
\pgfpathlineto{\pgfqpoint{3.640727in}{1.216787in}}%
\pgfpathlineto{\pgfqpoint{3.645236in}{1.215088in}}%
\pgfpathlineto{\pgfqpoint{3.649745in}{1.206991in}}%
\pgfpathlineto{\pgfqpoint{3.654255in}{1.228349in}}%
\pgfpathlineto{\pgfqpoint{3.658764in}{1.203938in}}%
\pgfpathlineto{\pgfqpoint{3.663273in}{1.216387in}}%
\pgfpathlineto{\pgfqpoint{3.667782in}{1.245601in}}%
\pgfpathlineto{\pgfqpoint{3.672291in}{1.208355in}}%
\pgfpathlineto{\pgfqpoint{3.676800in}{1.228422in}}%
\pgfpathlineto{\pgfqpoint{3.681309in}{1.236563in}}%
\pgfpathlineto{\pgfqpoint{3.685818in}{1.279281in}}%
\pgfpathlineto{\pgfqpoint{3.690327in}{1.229019in}}%
\pgfpathlineto{\pgfqpoint{3.694836in}{1.228374in}}%
\pgfpathlineto{\pgfqpoint{3.699345in}{1.233735in}}%
\pgfpathlineto{\pgfqpoint{3.703855in}{1.226441in}}%
\pgfpathlineto{\pgfqpoint{3.712873in}{1.245360in}}%
\pgfpathlineto{\pgfqpoint{3.717382in}{1.228418in}}%
\pgfpathlineto{\pgfqpoint{3.721891in}{1.235207in}}%
\pgfpathlineto{\pgfqpoint{3.726400in}{1.246610in}}%
\pgfpathlineto{\pgfqpoint{3.730909in}{1.261549in}}%
\pgfpathlineto{\pgfqpoint{3.735418in}{1.234099in}}%
\pgfpathlineto{\pgfqpoint{3.739927in}{1.248475in}}%
\pgfpathlineto{\pgfqpoint{3.744436in}{1.272229in}}%
\pgfpathlineto{\pgfqpoint{3.748945in}{1.285509in}}%
\pgfpathlineto{\pgfqpoint{3.753455in}{1.231786in}}%
\pgfpathlineto{\pgfqpoint{3.757964in}{1.293259in}}%
\pgfpathlineto{\pgfqpoint{3.762473in}{1.241932in}}%
\pgfpathlineto{\pgfqpoint{3.766982in}{1.273442in}}%
\pgfpathlineto{\pgfqpoint{3.771491in}{1.250266in}}%
\pgfpathlineto{\pgfqpoint{3.776000in}{1.236626in}}%
\pgfpathlineto{\pgfqpoint{3.780509in}{1.235072in}}%
\pgfpathlineto{\pgfqpoint{3.785018in}{1.395627in}}%
\pgfpathlineto{\pgfqpoint{3.789527in}{1.611840in}}%
\pgfpathlineto{\pgfqpoint{3.794036in}{1.355495in}}%
\pgfpathlineto{\pgfqpoint{3.798545in}{1.725434in}}%
\pgfpathlineto{\pgfqpoint{3.803055in}{1.783119in}}%
\pgfpathlineto{\pgfqpoint{3.807564in}{1.505057in}}%
\pgfpathlineto{\pgfqpoint{3.812073in}{1.553118in}}%
\pgfpathlineto{\pgfqpoint{3.816582in}{1.295637in}}%
\pgfpathlineto{\pgfqpoint{3.821091in}{1.341723in}}%
\pgfpathlineto{\pgfqpoint{3.825600in}{1.244776in}}%
\pgfpathlineto{\pgfqpoint{3.830109in}{1.366133in}}%
\pgfpathlineto{\pgfqpoint{3.834618in}{1.276406in}}%
\pgfpathlineto{\pgfqpoint{3.839127in}{1.436565in}}%
\pgfpathlineto{\pgfqpoint{3.848145in}{1.249052in}}%
\pgfpathlineto{\pgfqpoint{3.861673in}{1.319785in}}%
\pgfpathlineto{\pgfqpoint{3.866182in}{1.279605in}}%
\pgfpathlineto{\pgfqpoint{3.870691in}{1.293299in}}%
\pgfpathlineto{\pgfqpoint{3.875200in}{1.276318in}}%
\pgfpathlineto{\pgfqpoint{3.879709in}{1.274118in}}%
\pgfpathlineto{\pgfqpoint{3.884218in}{1.310498in}}%
\pgfpathlineto{\pgfqpoint{3.888727in}{1.272939in}}%
\pgfpathlineto{\pgfqpoint{3.893236in}{1.282457in}}%
\pgfpathlineto{\pgfqpoint{3.897745in}{1.297006in}}%
\pgfpathlineto{\pgfqpoint{3.902255in}{1.295711in}}%
\pgfpathlineto{\pgfqpoint{3.906764in}{1.311257in}}%
\pgfpathlineto{\pgfqpoint{3.911273in}{1.307269in}}%
\pgfpathlineto{\pgfqpoint{3.915782in}{1.332875in}}%
\pgfpathlineto{\pgfqpoint{3.920291in}{1.332185in}}%
\pgfpathlineto{\pgfqpoint{3.924800in}{1.316185in}}%
\pgfpathlineto{\pgfqpoint{3.929309in}{1.330006in}}%
\pgfpathlineto{\pgfqpoint{3.933818in}{1.329491in}}%
\pgfpathlineto{\pgfqpoint{3.938327in}{1.308687in}}%
\pgfpathlineto{\pgfqpoint{3.942836in}{1.326793in}}%
\pgfpathlineto{\pgfqpoint{3.947345in}{1.301956in}}%
\pgfpathlineto{\pgfqpoint{3.951855in}{1.290439in}}%
\pgfpathlineto{\pgfqpoint{3.960873in}{1.321930in}}%
\pgfpathlineto{\pgfqpoint{3.965382in}{1.316432in}}%
\pgfpathlineto{\pgfqpoint{3.969891in}{1.395927in}}%
\pgfpathlineto{\pgfqpoint{3.974400in}{1.320420in}}%
\pgfpathlineto{\pgfqpoint{3.978909in}{1.330968in}}%
\pgfpathlineto{\pgfqpoint{3.983418in}{1.344498in}}%
\pgfpathlineto{\pgfqpoint{3.987927in}{1.335177in}}%
\pgfpathlineto{\pgfqpoint{3.992436in}{1.356512in}}%
\pgfpathlineto{\pgfqpoint{3.996945in}{1.335244in}}%
\pgfpathlineto{\pgfqpoint{4.001455in}{1.335345in}}%
\pgfpathlineto{\pgfqpoint{4.005964in}{1.324087in}}%
\pgfpathlineto{\pgfqpoint{4.010473in}{1.356943in}}%
\pgfpathlineto{\pgfqpoint{4.014982in}{1.331062in}}%
\pgfpathlineto{\pgfqpoint{4.019491in}{1.329766in}}%
\pgfpathlineto{\pgfqpoint{4.024000in}{1.339079in}}%
\pgfpathlineto{\pgfqpoint{4.028509in}{1.328279in}}%
\pgfpathlineto{\pgfqpoint{4.033018in}{1.348817in}}%
\pgfpathlineto{\pgfqpoint{4.037527in}{1.346003in}}%
\pgfpathlineto{\pgfqpoint{4.042036in}{1.353818in}}%
\pgfpathlineto{\pgfqpoint{4.046545in}{1.334685in}}%
\pgfpathlineto{\pgfqpoint{4.051055in}{1.344208in}}%
\pgfpathlineto{\pgfqpoint{4.055564in}{1.357507in}}%
\pgfpathlineto{\pgfqpoint{4.060073in}{1.414078in}}%
\pgfpathlineto{\pgfqpoint{4.064582in}{1.364508in}}%
\pgfpathlineto{\pgfqpoint{4.069091in}{1.366318in}}%
\pgfpathlineto{\pgfqpoint{4.073600in}{1.355086in}}%
\pgfpathlineto{\pgfqpoint{4.078109in}{1.354647in}}%
\pgfpathlineto{\pgfqpoint{4.082618in}{1.384226in}}%
\pgfpathlineto{\pgfqpoint{4.087127in}{1.373672in}}%
\pgfpathlineto{\pgfqpoint{4.091636in}{1.401501in}}%
\pgfpathlineto{\pgfqpoint{4.096145in}{1.397328in}}%
\pgfpathlineto{\pgfqpoint{4.100655in}{1.370800in}}%
\pgfpathlineto{\pgfqpoint{4.105164in}{1.356822in}}%
\pgfpathlineto{\pgfqpoint{4.109673in}{1.358167in}}%
\pgfpathlineto{\pgfqpoint{4.114182in}{1.371925in}}%
\pgfpathlineto{\pgfqpoint{4.118691in}{1.367999in}}%
\pgfpathlineto{\pgfqpoint{4.123200in}{1.359337in}}%
\pgfpathlineto{\pgfqpoint{4.127709in}{1.378557in}}%
\pgfpathlineto{\pgfqpoint{4.132218in}{1.388548in}}%
\pgfpathlineto{\pgfqpoint{4.136727in}{1.373285in}}%
\pgfpathlineto{\pgfqpoint{4.145745in}{1.383096in}}%
\pgfpathlineto{\pgfqpoint{4.150255in}{1.458200in}}%
\pgfpathlineto{\pgfqpoint{4.159273in}{1.383478in}}%
\pgfpathlineto{\pgfqpoint{4.163782in}{1.389105in}}%
\pgfpathlineto{\pgfqpoint{4.168291in}{1.424208in}}%
\pgfpathlineto{\pgfqpoint{4.172800in}{1.409406in}}%
\pgfpathlineto{\pgfqpoint{4.177309in}{1.404383in}}%
\pgfpathlineto{\pgfqpoint{4.181818in}{1.403031in}}%
\pgfpathlineto{\pgfqpoint{4.186327in}{1.409313in}}%
\pgfpathlineto{\pgfqpoint{4.190836in}{1.409958in}}%
\pgfpathlineto{\pgfqpoint{4.195345in}{1.392651in}}%
\pgfpathlineto{\pgfqpoint{4.199855in}{1.385477in}}%
\pgfpathlineto{\pgfqpoint{4.204364in}{1.425820in}}%
\pgfpathlineto{\pgfqpoint{4.208873in}{1.412123in}}%
\pgfpathlineto{\pgfqpoint{4.213382in}{1.432409in}}%
\pgfpathlineto{\pgfqpoint{4.217891in}{1.434799in}}%
\pgfpathlineto{\pgfqpoint{4.222400in}{1.407294in}}%
\pgfpathlineto{\pgfqpoint{4.226909in}{1.433072in}}%
\pgfpathlineto{\pgfqpoint{4.231418in}{1.434812in}}%
\pgfpathlineto{\pgfqpoint{4.235927in}{1.401934in}}%
\pgfpathlineto{\pgfqpoint{4.240436in}{1.439336in}}%
\pgfpathlineto{\pgfqpoint{4.244945in}{1.411767in}}%
\pgfpathlineto{\pgfqpoint{4.249455in}{1.413321in}}%
\pgfpathlineto{\pgfqpoint{4.253964in}{1.429609in}}%
\pgfpathlineto{\pgfqpoint{4.258473in}{1.460588in}}%
\pgfpathlineto{\pgfqpoint{4.267491in}{1.409011in}}%
\pgfpathlineto{\pgfqpoint{4.272000in}{1.471856in}}%
\pgfpathlineto{\pgfqpoint{4.276509in}{1.433672in}}%
\pgfpathlineto{\pgfqpoint{4.281018in}{1.434521in}}%
\pgfpathlineto{\pgfqpoint{4.285527in}{1.463313in}}%
\pgfpathlineto{\pgfqpoint{4.290036in}{1.461134in}}%
\pgfpathlineto{\pgfqpoint{4.294545in}{1.448368in}}%
\pgfpathlineto{\pgfqpoint{4.299055in}{1.460484in}}%
\pgfpathlineto{\pgfqpoint{4.303564in}{1.427858in}}%
\pgfpathlineto{\pgfqpoint{4.308073in}{1.466765in}}%
\pgfpathlineto{\pgfqpoint{4.312582in}{1.465046in}}%
\pgfpathlineto{\pgfqpoint{4.317091in}{1.434178in}}%
\pgfpathlineto{\pgfqpoint{4.321600in}{1.456273in}}%
\pgfpathlineto{\pgfqpoint{4.326109in}{1.446134in}}%
\pgfpathlineto{\pgfqpoint{4.330618in}{1.477280in}}%
\pgfpathlineto{\pgfqpoint{4.335127in}{1.448913in}}%
\pgfpathlineto{\pgfqpoint{4.339636in}{1.494328in}}%
\pgfpathlineto{\pgfqpoint{4.344145in}{1.463350in}}%
\pgfpathlineto{\pgfqpoint{4.348655in}{1.453594in}}%
\pgfpathlineto{\pgfqpoint{4.353164in}{1.457496in}}%
\pgfpathlineto{\pgfqpoint{4.357673in}{1.456761in}}%
\pgfpathlineto{\pgfqpoint{4.362182in}{1.485298in}}%
\pgfpathlineto{\pgfqpoint{4.366691in}{1.468158in}}%
\pgfpathlineto{\pgfqpoint{4.371200in}{1.472355in}}%
\pgfpathlineto{\pgfqpoint{4.375709in}{1.486397in}}%
\pgfpathlineto{\pgfqpoint{4.380218in}{1.484021in}}%
\pgfpathlineto{\pgfqpoint{4.384727in}{1.474442in}}%
\pgfpathlineto{\pgfqpoint{4.389236in}{1.515789in}}%
\pgfpathlineto{\pgfqpoint{4.393745in}{1.481472in}}%
\pgfpathlineto{\pgfqpoint{4.398255in}{1.484156in}}%
\pgfpathlineto{\pgfqpoint{4.402764in}{1.532175in}}%
\pgfpathlineto{\pgfqpoint{4.411782in}{1.499242in}}%
\pgfpathlineto{\pgfqpoint{4.416291in}{1.488058in}}%
\pgfpathlineto{\pgfqpoint{4.420800in}{1.516999in}}%
\pgfpathlineto{\pgfqpoint{4.425309in}{1.479750in}}%
\pgfpathlineto{\pgfqpoint{4.429818in}{1.502530in}}%
\pgfpathlineto{\pgfqpoint{4.434327in}{1.494473in}}%
\pgfpathlineto{\pgfqpoint{4.438836in}{1.539744in}}%
\pgfpathlineto{\pgfqpoint{4.443345in}{1.496810in}}%
\pgfpathlineto{\pgfqpoint{4.447855in}{1.489321in}}%
\pgfpathlineto{\pgfqpoint{4.452364in}{1.509142in}}%
\pgfpathlineto{\pgfqpoint{4.456873in}{1.503993in}}%
\pgfpathlineto{\pgfqpoint{4.461382in}{1.491214in}}%
\pgfpathlineto{\pgfqpoint{4.465891in}{1.519219in}}%
\pgfpathlineto{\pgfqpoint{4.470400in}{1.520064in}}%
\pgfpathlineto{\pgfqpoint{4.474909in}{1.497463in}}%
\pgfpathlineto{\pgfqpoint{4.479418in}{1.505681in}}%
\pgfpathlineto{\pgfqpoint{4.483927in}{1.522901in}}%
\pgfpathlineto{\pgfqpoint{4.488436in}{1.511971in}}%
\pgfpathlineto{\pgfqpoint{4.492945in}{1.525795in}}%
\pgfpathlineto{\pgfqpoint{4.497455in}{1.514327in}}%
\pgfpathlineto{\pgfqpoint{4.501964in}{1.516179in}}%
\pgfpathlineto{\pgfqpoint{4.506473in}{1.510807in}}%
\pgfpathlineto{\pgfqpoint{4.510982in}{1.524841in}}%
\pgfpathlineto{\pgfqpoint{4.515491in}{1.530179in}}%
\pgfpathlineto{\pgfqpoint{4.520000in}{1.538966in}}%
\pgfpathlineto{\pgfqpoint{4.524509in}{1.594272in}}%
\pgfpathlineto{\pgfqpoint{4.529018in}{1.578986in}}%
\pgfpathlineto{\pgfqpoint{4.533527in}{1.532371in}}%
\pgfpathlineto{\pgfqpoint{4.538036in}{1.555538in}}%
\pgfpathlineto{\pgfqpoint{4.547055in}{1.564520in}}%
\pgfpathlineto{\pgfqpoint{4.551564in}{1.562860in}}%
\pgfpathlineto{\pgfqpoint{4.556073in}{1.557162in}}%
\pgfpathlineto{\pgfqpoint{4.560582in}{1.573223in}}%
\pgfpathlineto{\pgfqpoint{4.565091in}{1.550139in}}%
\pgfpathlineto{\pgfqpoint{4.569600in}{1.544155in}}%
\pgfpathlineto{\pgfqpoint{4.574109in}{1.552373in}}%
\pgfpathlineto{\pgfqpoint{4.578618in}{1.549819in}}%
\pgfpathlineto{\pgfqpoint{4.583127in}{1.628279in}}%
\pgfpathlineto{\pgfqpoint{4.587636in}{1.589075in}}%
\pgfpathlineto{\pgfqpoint{4.592145in}{1.627087in}}%
\pgfpathlineto{\pgfqpoint{4.596655in}{1.601349in}}%
\pgfpathlineto{\pgfqpoint{4.601164in}{1.667382in}}%
\pgfpathlineto{\pgfqpoint{4.605673in}{1.617173in}}%
\pgfpathlineto{\pgfqpoint{4.610182in}{1.588625in}}%
\pgfpathlineto{\pgfqpoint{4.614691in}{1.633419in}}%
\pgfpathlineto{\pgfqpoint{4.619200in}{1.571095in}}%
\pgfpathlineto{\pgfqpoint{4.623709in}{1.631667in}}%
\pgfpathlineto{\pgfqpoint{4.628218in}{1.624174in}}%
\pgfpathlineto{\pgfqpoint{4.632727in}{1.583946in}}%
\pgfpathlineto{\pgfqpoint{4.637236in}{1.579721in}}%
\pgfpathlineto{\pgfqpoint{4.641745in}{1.628143in}}%
\pgfpathlineto{\pgfqpoint{4.646255in}{1.722900in}}%
\pgfpathlineto{\pgfqpoint{4.650764in}{1.651854in}}%
\pgfpathlineto{\pgfqpoint{4.655273in}{1.679621in}}%
\pgfpathlineto{\pgfqpoint{4.659782in}{1.674564in}}%
\pgfpathlineto{\pgfqpoint{4.664291in}{1.995225in}}%
\pgfpathlineto{\pgfqpoint{4.668800in}{1.831127in}}%
\pgfpathlineto{\pgfqpoint{4.673309in}{2.068419in}}%
\pgfpathlineto{\pgfqpoint{4.677818in}{1.742475in}}%
\pgfpathlineto{\pgfqpoint{4.682327in}{1.589814in}}%
\pgfpathlineto{\pgfqpoint{4.686836in}{1.673419in}}%
\pgfpathlineto{\pgfqpoint{4.691345in}{1.672239in}}%
\pgfpathlineto{\pgfqpoint{4.695855in}{1.809951in}}%
\pgfpathlineto{\pgfqpoint{4.700364in}{1.645048in}}%
\pgfpathlineto{\pgfqpoint{4.704873in}{1.811875in}}%
\pgfpathlineto{\pgfqpoint{4.709382in}{2.272519in}}%
\pgfpathlineto{\pgfqpoint{4.713891in}{1.710722in}}%
\pgfpathlineto{\pgfqpoint{4.718400in}{1.845206in}}%
\pgfpathlineto{\pgfqpoint{4.722909in}{1.804145in}}%
\pgfpathlineto{\pgfqpoint{4.727418in}{1.729919in}}%
\pgfpathlineto{\pgfqpoint{4.731927in}{1.743808in}}%
\pgfpathlineto{\pgfqpoint{4.736436in}{1.685397in}}%
\pgfpathlineto{\pgfqpoint{4.740945in}{1.701462in}}%
\pgfpathlineto{\pgfqpoint{4.745455in}{1.696760in}}%
\pgfpathlineto{\pgfqpoint{4.749964in}{1.863471in}}%
\pgfpathlineto{\pgfqpoint{4.754473in}{1.788761in}}%
\pgfpathlineto{\pgfqpoint{4.758982in}{1.532655in}}%
\pgfpathlineto{\pgfqpoint{4.763491in}{2.154066in}}%
\pgfpathlineto{\pgfqpoint{4.768000in}{1.535092in}}%
\pgfpathlineto{\pgfqpoint{4.772509in}{1.551812in}}%
\pgfpathlineto{\pgfqpoint{4.777018in}{1.572874in}}%
\pgfpathlineto{\pgfqpoint{4.781527in}{1.585510in}}%
\pgfpathlineto{\pgfqpoint{4.786036in}{1.578920in}}%
\pgfpathlineto{\pgfqpoint{4.790545in}{1.549345in}}%
\pgfpathlineto{\pgfqpoint{4.795055in}{1.556594in}}%
\pgfpathlineto{\pgfqpoint{4.799564in}{1.567196in}}%
\pgfpathlineto{\pgfqpoint{4.804073in}{2.395411in}}%
\pgfpathlineto{\pgfqpoint{4.808582in}{2.132067in}}%
\pgfpathlineto{\pgfqpoint{4.813091in}{1.676059in}}%
\pgfpathlineto{\pgfqpoint{4.817600in}{1.562915in}}%
\pgfpathlineto{\pgfqpoint{4.822109in}{1.636295in}}%
\pgfpathlineto{\pgfqpoint{4.826618in}{1.635577in}}%
\pgfpathlineto{\pgfqpoint{4.831127in}{1.561481in}}%
\pgfpathlineto{\pgfqpoint{4.835636in}{1.712340in}}%
\pgfpathlineto{\pgfqpoint{4.840145in}{1.655620in}}%
\pgfpathlineto{\pgfqpoint{4.844655in}{1.667128in}}%
\pgfpathlineto{\pgfqpoint{4.849164in}{2.185699in}}%
\pgfpathlineto{\pgfqpoint{4.853673in}{1.692295in}}%
\pgfpathlineto{\pgfqpoint{4.858182in}{1.707723in}}%
\pgfpathlineto{\pgfqpoint{4.862691in}{1.730139in}}%
\pgfpathlineto{\pgfqpoint{4.867200in}{2.032958in}}%
\pgfpathlineto{\pgfqpoint{4.871709in}{1.783255in}}%
\pgfpathlineto{\pgfqpoint{4.876218in}{1.710996in}}%
\pgfpathlineto{\pgfqpoint{4.885236in}{1.870093in}}%
\pgfpathlineto{\pgfqpoint{4.889745in}{1.694214in}}%
\pgfpathlineto{\pgfqpoint{4.894255in}{1.717293in}}%
\pgfpathlineto{\pgfqpoint{4.898764in}{1.702734in}}%
\pgfpathlineto{\pgfqpoint{4.903273in}{1.716229in}}%
\pgfpathlineto{\pgfqpoint{4.907782in}{1.702618in}}%
\pgfpathlineto{\pgfqpoint{4.912291in}{1.744049in}}%
\pgfpathlineto{\pgfqpoint{4.916800in}{1.702360in}}%
\pgfpathlineto{\pgfqpoint{4.921309in}{1.730170in}}%
\pgfpathlineto{\pgfqpoint{4.925818in}{1.708672in}}%
\pgfpathlineto{\pgfqpoint{4.930327in}{1.745722in}}%
\pgfpathlineto{\pgfqpoint{4.934836in}{1.757030in}}%
\pgfpathlineto{\pgfqpoint{4.939345in}{1.863605in}}%
\pgfpathlineto{\pgfqpoint{4.943855in}{1.773182in}}%
\pgfpathlineto{\pgfqpoint{4.948364in}{1.756074in}}%
\pgfpathlineto{\pgfqpoint{4.952873in}{1.802570in}}%
\pgfpathlineto{\pgfqpoint{4.957382in}{1.762749in}}%
\pgfpathlineto{\pgfqpoint{4.961891in}{1.767326in}}%
\pgfpathlineto{\pgfqpoint{4.966400in}{1.750259in}}%
\pgfpathlineto{\pgfqpoint{4.970909in}{1.744026in}}%
\pgfpathlineto{\pgfqpoint{4.975418in}{1.729366in}}%
\pgfpathlineto{\pgfqpoint{4.979927in}{1.794608in}}%
\pgfpathlineto{\pgfqpoint{4.984436in}{1.752728in}}%
\pgfpathlineto{\pgfqpoint{4.988945in}{1.814897in}}%
\pgfpathlineto{\pgfqpoint{4.993455in}{1.761543in}}%
\pgfpathlineto{\pgfqpoint{4.997964in}{1.744051in}}%
\pgfpathlineto{\pgfqpoint{5.002473in}{1.785016in}}%
\pgfpathlineto{\pgfqpoint{5.006982in}{1.752885in}}%
\pgfpathlineto{\pgfqpoint{5.011491in}{1.780527in}}%
\pgfpathlineto{\pgfqpoint{5.016000in}{1.739278in}}%
\pgfpathlineto{\pgfqpoint{5.020509in}{1.768167in}}%
\pgfpathlineto{\pgfqpoint{5.025018in}{1.743386in}}%
\pgfpathlineto{\pgfqpoint{5.029527in}{1.797244in}}%
\pgfpathlineto{\pgfqpoint{5.034036in}{1.767088in}}%
\pgfpathlineto{\pgfqpoint{5.038545in}{1.868946in}}%
\pgfpathlineto{\pgfqpoint{5.043055in}{1.900557in}}%
\pgfpathlineto{\pgfqpoint{5.047564in}{1.801084in}}%
\pgfpathlineto{\pgfqpoint{5.052073in}{1.827968in}}%
\pgfpathlineto{\pgfqpoint{5.056582in}{1.905465in}}%
\pgfpathlineto{\pgfqpoint{5.061091in}{1.803572in}}%
\pgfpathlineto{\pgfqpoint{5.070109in}{1.781448in}}%
\pgfpathlineto{\pgfqpoint{5.074618in}{1.800881in}}%
\pgfpathlineto{\pgfqpoint{5.079127in}{1.804509in}}%
\pgfpathlineto{\pgfqpoint{5.083636in}{1.812420in}}%
\pgfpathlineto{\pgfqpoint{5.088145in}{1.835256in}}%
\pgfpathlineto{\pgfqpoint{5.092655in}{1.819851in}}%
\pgfpathlineto{\pgfqpoint{5.097164in}{1.818005in}}%
\pgfpathlineto{\pgfqpoint{5.101673in}{1.803461in}}%
\pgfpathlineto{\pgfqpoint{5.106182in}{1.843899in}}%
\pgfpathlineto{\pgfqpoint{5.110691in}{1.834720in}}%
\pgfpathlineto{\pgfqpoint{5.119709in}{1.799619in}}%
\pgfpathlineto{\pgfqpoint{5.124218in}{1.874134in}}%
\pgfpathlineto{\pgfqpoint{5.128727in}{1.800092in}}%
\pgfpathlineto{\pgfqpoint{5.133236in}{1.803402in}}%
\pgfpathlineto{\pgfqpoint{5.142255in}{1.847842in}}%
\pgfpathlineto{\pgfqpoint{5.146764in}{2.203246in}}%
\pgfpathlineto{\pgfqpoint{5.151273in}{2.407714in}}%
\pgfpathlineto{\pgfqpoint{5.155782in}{2.198188in}}%
\pgfpathlineto{\pgfqpoint{5.160291in}{2.643475in}}%
\pgfpathlineto{\pgfqpoint{5.164800in}{2.036764in}}%
\pgfpathlineto{\pgfqpoint{5.169309in}{1.982060in}}%
\pgfpathlineto{\pgfqpoint{5.173818in}{1.878220in}}%
\pgfpathlineto{\pgfqpoint{5.178327in}{2.045005in}}%
\pgfpathlineto{\pgfqpoint{5.182836in}{1.875107in}}%
\pgfpathlineto{\pgfqpoint{5.187345in}{1.898451in}}%
\pgfpathlineto{\pgfqpoint{5.191855in}{1.913628in}}%
\pgfpathlineto{\pgfqpoint{5.196364in}{1.840658in}}%
\pgfpathlineto{\pgfqpoint{5.200873in}{1.891493in}}%
\pgfpathlineto{\pgfqpoint{5.205382in}{1.863198in}}%
\pgfpathlineto{\pgfqpoint{5.209891in}{1.863986in}}%
\pgfpathlineto{\pgfqpoint{5.214400in}{1.894211in}}%
\pgfpathlineto{\pgfqpoint{5.218909in}{1.863257in}}%
\pgfpathlineto{\pgfqpoint{5.223418in}{1.956670in}}%
\pgfpathlineto{\pgfqpoint{5.227927in}{1.934855in}}%
\pgfpathlineto{\pgfqpoint{5.232436in}{1.921728in}}%
\pgfpathlineto{\pgfqpoint{5.236945in}{1.875118in}}%
\pgfpathlineto{\pgfqpoint{5.241455in}{1.865959in}}%
\pgfpathlineto{\pgfqpoint{5.245964in}{2.173128in}}%
\pgfpathlineto{\pgfqpoint{5.250473in}{2.575701in}}%
\pgfpathlineto{\pgfqpoint{5.254982in}{2.358230in}}%
\pgfpathlineto{\pgfqpoint{5.259491in}{1.965306in}}%
\pgfpathlineto{\pgfqpoint{5.268509in}{2.241158in}}%
\pgfpathlineto{\pgfqpoint{5.273018in}{1.907988in}}%
\pgfpathlineto{\pgfqpoint{5.282036in}{1.946124in}}%
\pgfpathlineto{\pgfqpoint{5.291055in}{1.846783in}}%
\pgfpathlineto{\pgfqpoint{5.295564in}{1.921443in}}%
\pgfpathlineto{\pgfqpoint{5.300073in}{1.911918in}}%
\pgfpathlineto{\pgfqpoint{5.304582in}{1.892749in}}%
\pgfpathlineto{\pgfqpoint{5.313600in}{1.941530in}}%
\pgfpathlineto{\pgfqpoint{5.318109in}{1.896609in}}%
\pgfpathlineto{\pgfqpoint{5.322618in}{1.923432in}}%
\pgfpathlineto{\pgfqpoint{5.327127in}{1.891222in}}%
\pgfpathlineto{\pgfqpoint{5.331636in}{1.952334in}}%
\pgfpathlineto{\pgfqpoint{5.336145in}{1.986732in}}%
\pgfpathlineto{\pgfqpoint{5.340655in}{1.885066in}}%
\pgfpathlineto{\pgfqpoint{5.349673in}{1.947587in}}%
\pgfpathlineto{\pgfqpoint{5.354182in}{1.932677in}}%
\pgfpathlineto{\pgfqpoint{5.358691in}{1.925895in}}%
\pgfpathlineto{\pgfqpoint{5.363200in}{1.916812in}}%
\pgfpathlineto{\pgfqpoint{5.367709in}{1.935022in}}%
\pgfpathlineto{\pgfqpoint{5.372218in}{1.924077in}}%
\pgfpathlineto{\pgfqpoint{5.376727in}{1.944951in}}%
\pgfpathlineto{\pgfqpoint{5.381236in}{1.928360in}}%
\pgfpathlineto{\pgfqpoint{5.385745in}{1.927500in}}%
\pgfpathlineto{\pgfqpoint{5.390255in}{2.004353in}}%
\pgfpathlineto{\pgfqpoint{5.394764in}{1.993368in}}%
\pgfpathlineto{\pgfqpoint{5.399273in}{1.956701in}}%
\pgfpathlineto{\pgfqpoint{5.403782in}{1.956724in}}%
\pgfpathlineto{\pgfqpoint{5.408291in}{1.979639in}}%
\pgfpathlineto{\pgfqpoint{5.412800in}{1.955718in}}%
\pgfpathlineto{\pgfqpoint{5.417309in}{1.967713in}}%
\pgfpathlineto{\pgfqpoint{5.421818in}{1.965122in}}%
\pgfpathlineto{\pgfqpoint{5.426327in}{1.982353in}}%
\pgfpathlineto{\pgfqpoint{5.430836in}{1.957691in}}%
\pgfpathlineto{\pgfqpoint{5.435345in}{1.994354in}}%
\pgfpathlineto{\pgfqpoint{5.439855in}{1.992903in}}%
\pgfpathlineto{\pgfqpoint{5.444364in}{1.959797in}}%
\pgfpathlineto{\pgfqpoint{5.448873in}{2.026700in}}%
\pgfpathlineto{\pgfqpoint{5.453382in}{1.970584in}}%
\pgfpathlineto{\pgfqpoint{5.457891in}{2.093954in}}%
\pgfpathlineto{\pgfqpoint{5.462400in}{2.046637in}}%
\pgfpathlineto{\pgfqpoint{5.466909in}{1.987397in}}%
\pgfpathlineto{\pgfqpoint{5.471418in}{2.009182in}}%
\pgfpathlineto{\pgfqpoint{5.475927in}{1.966126in}}%
\pgfpathlineto{\pgfqpoint{5.480436in}{2.108572in}}%
\pgfpathlineto{\pgfqpoint{5.484945in}{2.038640in}}%
\pgfpathlineto{\pgfqpoint{5.489455in}{1.992623in}}%
\pgfpathlineto{\pgfqpoint{5.493964in}{2.042104in}}%
\pgfpathlineto{\pgfqpoint{5.498473in}{1.982394in}}%
\pgfpathlineto{\pgfqpoint{5.502982in}{2.025739in}}%
\pgfpathlineto{\pgfqpoint{5.507491in}{2.011266in}}%
\pgfpathlineto{\pgfqpoint{5.512000in}{2.027096in}}%
\pgfpathlineto{\pgfqpoint{5.516509in}{2.021556in}}%
\pgfpathlineto{\pgfqpoint{5.521018in}{2.029855in}}%
\pgfpathlineto{\pgfqpoint{5.525527in}{2.111842in}}%
\pgfpathlineto{\pgfqpoint{5.530036in}{1.981415in}}%
\pgfpathlineto{\pgfqpoint{5.534545in}{2.016216in}}%
\pgfpathlineto{\pgfqpoint{5.534545in}{2.016216in}}%
\pgfusepath{stroke}%
\end{pgfscope}%
\begin{pgfscope}%
\pgfpathrectangle{\pgfqpoint{0.800000in}{0.528000in}}{\pgfqpoint{4.960000in}{3.696000in}}%
\pgfusepath{clip}%
\pgfsetrectcap%
\pgfsetroundjoin%
\pgfsetlinewidth{1.505625pt}%
\definecolor{currentstroke}{rgb}{0.000000,1.000000,0.498039}%
\pgfsetstrokecolor{currentstroke}%
\pgfsetdash{}{0pt}%
\pgfpathmoveto{\pgfqpoint{1.025455in}{0.698065in}}%
\pgfpathlineto{\pgfqpoint{1.034473in}{0.700117in}}%
\pgfpathlineto{\pgfqpoint{1.043491in}{0.699819in}}%
\pgfpathlineto{\pgfqpoint{1.048000in}{0.701393in}}%
\pgfpathlineto{\pgfqpoint{1.057018in}{0.700091in}}%
\pgfpathlineto{\pgfqpoint{1.070545in}{0.701844in}}%
\pgfpathlineto{\pgfqpoint{1.075055in}{0.700714in}}%
\pgfpathlineto{\pgfqpoint{1.093091in}{0.701639in}}%
\pgfpathlineto{\pgfqpoint{1.097600in}{0.700571in}}%
\pgfpathlineto{\pgfqpoint{1.115636in}{0.702775in}}%
\pgfpathlineto{\pgfqpoint{1.120145in}{0.702551in}}%
\pgfpathlineto{\pgfqpoint{1.124655in}{0.704066in}}%
\pgfpathlineto{\pgfqpoint{1.129164in}{0.702147in}}%
\pgfpathlineto{\pgfqpoint{1.133673in}{0.701435in}}%
\pgfpathlineto{\pgfqpoint{1.138182in}{0.703298in}}%
\pgfpathlineto{\pgfqpoint{1.142691in}{0.702053in}}%
\pgfpathlineto{\pgfqpoint{1.147200in}{0.704453in}}%
\pgfpathlineto{\pgfqpoint{1.151709in}{0.701849in}}%
\pgfpathlineto{\pgfqpoint{1.156218in}{0.704186in}}%
\pgfpathlineto{\pgfqpoint{1.160727in}{0.703023in}}%
\pgfpathlineto{\pgfqpoint{1.174255in}{0.705892in}}%
\pgfpathlineto{\pgfqpoint{1.178764in}{0.703461in}}%
\pgfpathlineto{\pgfqpoint{1.183273in}{0.703749in}}%
\pgfpathlineto{\pgfqpoint{1.187782in}{0.702693in}}%
\pgfpathlineto{\pgfqpoint{1.192291in}{0.703055in}}%
\pgfpathlineto{\pgfqpoint{1.196800in}{0.704744in}}%
\pgfpathlineto{\pgfqpoint{1.201309in}{0.703482in}}%
\pgfpathlineto{\pgfqpoint{1.210327in}{0.704108in}}%
\pgfpathlineto{\pgfqpoint{1.214836in}{0.706857in}}%
\pgfpathlineto{\pgfqpoint{1.219345in}{0.707638in}}%
\pgfpathlineto{\pgfqpoint{1.223855in}{0.705239in}}%
\pgfpathlineto{\pgfqpoint{1.228364in}{0.704382in}}%
\pgfpathlineto{\pgfqpoint{1.232873in}{0.705476in}}%
\pgfpathlineto{\pgfqpoint{1.237382in}{0.710662in}}%
\pgfpathlineto{\pgfqpoint{1.241891in}{0.711621in}}%
\pgfpathlineto{\pgfqpoint{1.246400in}{0.707230in}}%
\pgfpathlineto{\pgfqpoint{1.250909in}{0.705482in}}%
\pgfpathlineto{\pgfqpoint{1.255418in}{0.707363in}}%
\pgfpathlineto{\pgfqpoint{1.259927in}{0.706075in}}%
\pgfpathlineto{\pgfqpoint{1.264436in}{0.707685in}}%
\pgfpathlineto{\pgfqpoint{1.273455in}{0.706048in}}%
\pgfpathlineto{\pgfqpoint{1.277964in}{0.708881in}}%
\pgfpathlineto{\pgfqpoint{1.282473in}{0.706809in}}%
\pgfpathlineto{\pgfqpoint{1.291491in}{0.706885in}}%
\pgfpathlineto{\pgfqpoint{1.296000in}{0.709033in}}%
\pgfpathlineto{\pgfqpoint{1.305018in}{0.705843in}}%
\pgfpathlineto{\pgfqpoint{1.323055in}{0.708413in}}%
\pgfpathlineto{\pgfqpoint{1.327564in}{0.706722in}}%
\pgfpathlineto{\pgfqpoint{1.332073in}{0.709041in}}%
\pgfpathlineto{\pgfqpoint{1.336582in}{0.709854in}}%
\pgfpathlineto{\pgfqpoint{1.341091in}{0.720126in}}%
\pgfpathlineto{\pgfqpoint{1.345600in}{0.719584in}}%
\pgfpathlineto{\pgfqpoint{1.350109in}{0.720864in}}%
\pgfpathlineto{\pgfqpoint{1.354618in}{0.720384in}}%
\pgfpathlineto{\pgfqpoint{1.359127in}{0.724665in}}%
\pgfpathlineto{\pgfqpoint{1.363636in}{0.723348in}}%
\pgfpathlineto{\pgfqpoint{1.368145in}{0.729773in}}%
\pgfpathlineto{\pgfqpoint{1.372655in}{0.716180in}}%
\pgfpathlineto{\pgfqpoint{1.381673in}{0.711939in}}%
\pgfpathlineto{\pgfqpoint{1.386182in}{0.720690in}}%
\pgfpathlineto{\pgfqpoint{1.390691in}{0.716732in}}%
\pgfpathlineto{\pgfqpoint{1.395200in}{0.734752in}}%
\pgfpathlineto{\pgfqpoint{1.399709in}{0.727174in}}%
\pgfpathlineto{\pgfqpoint{1.404218in}{0.717069in}}%
\pgfpathlineto{\pgfqpoint{1.408727in}{0.711734in}}%
\pgfpathlineto{\pgfqpoint{1.413236in}{0.714011in}}%
\pgfpathlineto{\pgfqpoint{1.417745in}{0.711951in}}%
\pgfpathlineto{\pgfqpoint{1.422255in}{0.712740in}}%
\pgfpathlineto{\pgfqpoint{1.435782in}{0.711671in}}%
\pgfpathlineto{\pgfqpoint{1.453818in}{0.713826in}}%
\pgfpathlineto{\pgfqpoint{1.467345in}{0.712884in}}%
\pgfpathlineto{\pgfqpoint{1.471855in}{0.713297in}}%
\pgfpathlineto{\pgfqpoint{1.476364in}{0.716221in}}%
\pgfpathlineto{\pgfqpoint{1.480873in}{0.713910in}}%
\pgfpathlineto{\pgfqpoint{1.494400in}{0.714104in}}%
\pgfpathlineto{\pgfqpoint{1.498909in}{0.716273in}}%
\pgfpathlineto{\pgfqpoint{1.503418in}{0.720955in}}%
\pgfpathlineto{\pgfqpoint{1.507927in}{0.738111in}}%
\pgfpathlineto{\pgfqpoint{1.512436in}{0.748212in}}%
\pgfpathlineto{\pgfqpoint{1.516945in}{0.714786in}}%
\pgfpathlineto{\pgfqpoint{1.525964in}{0.715634in}}%
\pgfpathlineto{\pgfqpoint{1.539491in}{0.717225in}}%
\pgfpathlineto{\pgfqpoint{1.544000in}{0.720919in}}%
\pgfpathlineto{\pgfqpoint{1.548509in}{0.720183in}}%
\pgfpathlineto{\pgfqpoint{1.553018in}{0.722525in}}%
\pgfpathlineto{\pgfqpoint{1.557527in}{0.721161in}}%
\pgfpathlineto{\pgfqpoint{1.562036in}{0.718103in}}%
\pgfpathlineto{\pgfqpoint{1.566545in}{0.717936in}}%
\pgfpathlineto{\pgfqpoint{1.571055in}{0.720152in}}%
\pgfpathlineto{\pgfqpoint{1.575564in}{0.720653in}}%
\pgfpathlineto{\pgfqpoint{1.584582in}{0.725172in}}%
\pgfpathlineto{\pgfqpoint{1.589091in}{0.732522in}}%
\pgfpathlineto{\pgfqpoint{1.593600in}{0.751824in}}%
\pgfpathlineto{\pgfqpoint{1.598109in}{0.726075in}}%
\pgfpathlineto{\pgfqpoint{1.602618in}{0.731972in}}%
\pgfpathlineto{\pgfqpoint{1.607127in}{0.732014in}}%
\pgfpathlineto{\pgfqpoint{1.611636in}{0.741944in}}%
\pgfpathlineto{\pgfqpoint{1.616145in}{0.742509in}}%
\pgfpathlineto{\pgfqpoint{1.620655in}{0.760280in}}%
\pgfpathlineto{\pgfqpoint{1.625164in}{0.732192in}}%
\pgfpathlineto{\pgfqpoint{1.629673in}{0.727393in}}%
\pgfpathlineto{\pgfqpoint{1.634182in}{0.735692in}}%
\pgfpathlineto{\pgfqpoint{1.638691in}{0.721757in}}%
\pgfpathlineto{\pgfqpoint{1.652218in}{0.722670in}}%
\pgfpathlineto{\pgfqpoint{1.656727in}{0.725463in}}%
\pgfpathlineto{\pgfqpoint{1.661236in}{0.723809in}}%
\pgfpathlineto{\pgfqpoint{1.665745in}{0.724595in}}%
\pgfpathlineto{\pgfqpoint{1.670255in}{0.752162in}}%
\pgfpathlineto{\pgfqpoint{1.674764in}{0.754267in}}%
\pgfpathlineto{\pgfqpoint{1.679273in}{0.759249in}}%
\pgfpathlineto{\pgfqpoint{1.683782in}{0.726503in}}%
\pgfpathlineto{\pgfqpoint{1.688291in}{0.723018in}}%
\pgfpathlineto{\pgfqpoint{1.697309in}{0.728098in}}%
\pgfpathlineto{\pgfqpoint{1.701818in}{0.737937in}}%
\pgfpathlineto{\pgfqpoint{1.706327in}{0.737815in}}%
\pgfpathlineto{\pgfqpoint{1.710836in}{0.728085in}}%
\pgfpathlineto{\pgfqpoint{1.715345in}{0.735771in}}%
\pgfpathlineto{\pgfqpoint{1.719855in}{0.757524in}}%
\pgfpathlineto{\pgfqpoint{1.724364in}{0.733684in}}%
\pgfpathlineto{\pgfqpoint{1.728873in}{0.728779in}}%
\pgfpathlineto{\pgfqpoint{1.733382in}{0.726996in}}%
\pgfpathlineto{\pgfqpoint{1.737891in}{0.731700in}}%
\pgfpathlineto{\pgfqpoint{1.742400in}{0.731508in}}%
\pgfpathlineto{\pgfqpoint{1.746909in}{0.729341in}}%
\pgfpathlineto{\pgfqpoint{1.751418in}{0.757011in}}%
\pgfpathlineto{\pgfqpoint{1.755927in}{0.735919in}}%
\pgfpathlineto{\pgfqpoint{1.760436in}{0.733601in}}%
\pgfpathlineto{\pgfqpoint{1.764945in}{0.736693in}}%
\pgfpathlineto{\pgfqpoint{1.769455in}{0.731048in}}%
\pgfpathlineto{\pgfqpoint{1.773964in}{0.732316in}}%
\pgfpathlineto{\pgfqpoint{1.778473in}{0.731443in}}%
\pgfpathlineto{\pgfqpoint{1.782982in}{0.780598in}}%
\pgfpathlineto{\pgfqpoint{1.787491in}{0.752116in}}%
\pgfpathlineto{\pgfqpoint{1.792000in}{0.756757in}}%
\pgfpathlineto{\pgfqpoint{1.796509in}{0.742495in}}%
\pgfpathlineto{\pgfqpoint{1.801018in}{0.743146in}}%
\pgfpathlineto{\pgfqpoint{1.805527in}{0.751517in}}%
\pgfpathlineto{\pgfqpoint{1.810036in}{0.732897in}}%
\pgfpathlineto{\pgfqpoint{1.814545in}{0.734556in}}%
\pgfpathlineto{\pgfqpoint{1.819055in}{0.741175in}}%
\pgfpathlineto{\pgfqpoint{1.823564in}{0.737813in}}%
\pgfpathlineto{\pgfqpoint{1.828073in}{0.739229in}}%
\pgfpathlineto{\pgfqpoint{1.832582in}{0.735829in}}%
\pgfpathlineto{\pgfqpoint{1.837091in}{0.748846in}}%
\pgfpathlineto{\pgfqpoint{1.841600in}{0.734673in}}%
\pgfpathlineto{\pgfqpoint{1.846109in}{0.732058in}}%
\pgfpathlineto{\pgfqpoint{1.850618in}{0.745355in}}%
\pgfpathlineto{\pgfqpoint{1.855127in}{0.773280in}}%
\pgfpathlineto{\pgfqpoint{1.859636in}{0.757322in}}%
\pgfpathlineto{\pgfqpoint{1.864145in}{0.735095in}}%
\pgfpathlineto{\pgfqpoint{1.868655in}{0.735037in}}%
\pgfpathlineto{\pgfqpoint{1.873164in}{0.733636in}}%
\pgfpathlineto{\pgfqpoint{1.877673in}{0.734046in}}%
\pgfpathlineto{\pgfqpoint{1.886691in}{0.751919in}}%
\pgfpathlineto{\pgfqpoint{1.891200in}{0.733996in}}%
\pgfpathlineto{\pgfqpoint{1.895709in}{0.735594in}}%
\pgfpathlineto{\pgfqpoint{1.900218in}{0.741543in}}%
\pgfpathlineto{\pgfqpoint{1.904727in}{0.781517in}}%
\pgfpathlineto{\pgfqpoint{1.909236in}{0.741148in}}%
\pgfpathlineto{\pgfqpoint{1.913745in}{0.741667in}}%
\pgfpathlineto{\pgfqpoint{1.922764in}{0.738037in}}%
\pgfpathlineto{\pgfqpoint{1.927273in}{0.741066in}}%
\pgfpathlineto{\pgfqpoint{1.931782in}{0.740375in}}%
\pgfpathlineto{\pgfqpoint{1.936291in}{0.740959in}}%
\pgfpathlineto{\pgfqpoint{1.940800in}{0.738615in}}%
\pgfpathlineto{\pgfqpoint{1.945309in}{0.742554in}}%
\pgfpathlineto{\pgfqpoint{1.949818in}{0.742851in}}%
\pgfpathlineto{\pgfqpoint{1.954327in}{0.752935in}}%
\pgfpathlineto{\pgfqpoint{1.958836in}{0.751460in}}%
\pgfpathlineto{\pgfqpoint{1.963345in}{0.744820in}}%
\pgfpathlineto{\pgfqpoint{1.967855in}{0.744406in}}%
\pgfpathlineto{\pgfqpoint{1.976873in}{0.748485in}}%
\pgfpathlineto{\pgfqpoint{1.985891in}{0.745038in}}%
\pgfpathlineto{\pgfqpoint{1.990400in}{0.761745in}}%
\pgfpathlineto{\pgfqpoint{1.994909in}{0.745437in}}%
\pgfpathlineto{\pgfqpoint{1.999418in}{0.742073in}}%
\pgfpathlineto{\pgfqpoint{2.003927in}{0.744586in}}%
\pgfpathlineto{\pgfqpoint{2.008436in}{0.744928in}}%
\pgfpathlineto{\pgfqpoint{2.012945in}{0.750877in}}%
\pgfpathlineto{\pgfqpoint{2.017455in}{0.750119in}}%
\pgfpathlineto{\pgfqpoint{2.021964in}{0.744468in}}%
\pgfpathlineto{\pgfqpoint{2.026473in}{0.750878in}}%
\pgfpathlineto{\pgfqpoint{2.030982in}{0.751381in}}%
\pgfpathlineto{\pgfqpoint{2.035491in}{0.758920in}}%
\pgfpathlineto{\pgfqpoint{2.040000in}{0.754599in}}%
\pgfpathlineto{\pgfqpoint{2.044509in}{0.768592in}}%
\pgfpathlineto{\pgfqpoint{2.049018in}{0.758576in}}%
\pgfpathlineto{\pgfqpoint{2.053527in}{0.779951in}}%
\pgfpathlineto{\pgfqpoint{2.058036in}{0.751976in}}%
\pgfpathlineto{\pgfqpoint{2.062545in}{0.755853in}}%
\pgfpathlineto{\pgfqpoint{2.067055in}{0.755186in}}%
\pgfpathlineto{\pgfqpoint{2.071564in}{0.752049in}}%
\pgfpathlineto{\pgfqpoint{2.076073in}{0.757899in}}%
\pgfpathlineto{\pgfqpoint{2.080582in}{0.793163in}}%
\pgfpathlineto{\pgfqpoint{2.085091in}{0.795794in}}%
\pgfpathlineto{\pgfqpoint{2.089600in}{0.751771in}}%
\pgfpathlineto{\pgfqpoint{2.098618in}{0.753592in}}%
\pgfpathlineto{\pgfqpoint{2.103127in}{0.764237in}}%
\pgfpathlineto{\pgfqpoint{2.107636in}{0.756501in}}%
\pgfpathlineto{\pgfqpoint{2.112145in}{0.763139in}}%
\pgfpathlineto{\pgfqpoint{2.116655in}{0.774121in}}%
\pgfpathlineto{\pgfqpoint{2.121164in}{0.807922in}}%
\pgfpathlineto{\pgfqpoint{2.125673in}{0.773082in}}%
\pgfpathlineto{\pgfqpoint{2.130182in}{0.763535in}}%
\pgfpathlineto{\pgfqpoint{2.134691in}{0.829317in}}%
\pgfpathlineto{\pgfqpoint{2.139200in}{0.798251in}}%
\pgfpathlineto{\pgfqpoint{2.143709in}{0.779431in}}%
\pgfpathlineto{\pgfqpoint{2.148218in}{0.802820in}}%
\pgfpathlineto{\pgfqpoint{2.152727in}{0.754690in}}%
\pgfpathlineto{\pgfqpoint{2.157236in}{0.757887in}}%
\pgfpathlineto{\pgfqpoint{2.161745in}{0.802149in}}%
\pgfpathlineto{\pgfqpoint{2.166255in}{0.765652in}}%
\pgfpathlineto{\pgfqpoint{2.170764in}{0.787087in}}%
\pgfpathlineto{\pgfqpoint{2.175273in}{0.784345in}}%
\pgfpathlineto{\pgfqpoint{2.179782in}{0.819408in}}%
\pgfpathlineto{\pgfqpoint{2.184291in}{0.778869in}}%
\pgfpathlineto{\pgfqpoint{2.188800in}{0.767894in}}%
\pgfpathlineto{\pgfqpoint{2.193309in}{0.833818in}}%
\pgfpathlineto{\pgfqpoint{2.197818in}{0.829599in}}%
\pgfpathlineto{\pgfqpoint{2.202327in}{0.830963in}}%
\pgfpathlineto{\pgfqpoint{2.206836in}{0.788062in}}%
\pgfpathlineto{\pgfqpoint{2.211345in}{0.813159in}}%
\pgfpathlineto{\pgfqpoint{2.215855in}{0.777554in}}%
\pgfpathlineto{\pgfqpoint{2.220364in}{0.766314in}}%
\pgfpathlineto{\pgfqpoint{2.224873in}{0.770443in}}%
\pgfpathlineto{\pgfqpoint{2.229382in}{0.794864in}}%
\pgfpathlineto{\pgfqpoint{2.233891in}{0.768095in}}%
\pgfpathlineto{\pgfqpoint{2.238400in}{0.792740in}}%
\pgfpathlineto{\pgfqpoint{2.242909in}{0.777911in}}%
\pgfpathlineto{\pgfqpoint{2.247418in}{0.773719in}}%
\pgfpathlineto{\pgfqpoint{2.251927in}{0.774888in}}%
\pgfpathlineto{\pgfqpoint{2.256436in}{0.770460in}}%
\pgfpathlineto{\pgfqpoint{2.260945in}{0.771891in}}%
\pgfpathlineto{\pgfqpoint{2.265455in}{0.778509in}}%
\pgfpathlineto{\pgfqpoint{2.269964in}{0.776613in}}%
\pgfpathlineto{\pgfqpoint{2.274473in}{0.781117in}}%
\pgfpathlineto{\pgfqpoint{2.278982in}{0.780158in}}%
\pgfpathlineto{\pgfqpoint{2.283491in}{0.788346in}}%
\pgfpathlineto{\pgfqpoint{2.288000in}{0.774394in}}%
\pgfpathlineto{\pgfqpoint{2.292509in}{0.774268in}}%
\pgfpathlineto{\pgfqpoint{2.297018in}{0.779626in}}%
\pgfpathlineto{\pgfqpoint{2.301527in}{0.776955in}}%
\pgfpathlineto{\pgfqpoint{2.306036in}{0.792141in}}%
\pgfpathlineto{\pgfqpoint{2.310545in}{0.783751in}}%
\pgfpathlineto{\pgfqpoint{2.315055in}{0.783496in}}%
\pgfpathlineto{\pgfqpoint{2.319564in}{0.881908in}}%
\pgfpathlineto{\pgfqpoint{2.324073in}{0.783108in}}%
\pgfpathlineto{\pgfqpoint{2.328582in}{0.789318in}}%
\pgfpathlineto{\pgfqpoint{2.333091in}{0.782064in}}%
\pgfpathlineto{\pgfqpoint{2.337600in}{0.788446in}}%
\pgfpathlineto{\pgfqpoint{2.342109in}{0.784461in}}%
\pgfpathlineto{\pgfqpoint{2.346618in}{0.784286in}}%
\pgfpathlineto{\pgfqpoint{2.351127in}{0.776153in}}%
\pgfpathlineto{\pgfqpoint{2.355636in}{0.780037in}}%
\pgfpathlineto{\pgfqpoint{2.360145in}{0.788925in}}%
\pgfpathlineto{\pgfqpoint{2.364655in}{0.788536in}}%
\pgfpathlineto{\pgfqpoint{2.369164in}{0.782727in}}%
\pgfpathlineto{\pgfqpoint{2.373673in}{0.793112in}}%
\pgfpathlineto{\pgfqpoint{2.378182in}{0.800186in}}%
\pgfpathlineto{\pgfqpoint{2.382691in}{0.788344in}}%
\pgfpathlineto{\pgfqpoint{2.387200in}{0.790268in}}%
\pgfpathlineto{\pgfqpoint{2.396218in}{0.782983in}}%
\pgfpathlineto{\pgfqpoint{2.400727in}{0.795948in}}%
\pgfpathlineto{\pgfqpoint{2.405236in}{0.798847in}}%
\pgfpathlineto{\pgfqpoint{2.409745in}{0.791394in}}%
\pgfpathlineto{\pgfqpoint{2.414255in}{0.812794in}}%
\pgfpathlineto{\pgfqpoint{2.418764in}{0.792304in}}%
\pgfpathlineto{\pgfqpoint{2.423273in}{0.788528in}}%
\pgfpathlineto{\pgfqpoint{2.432291in}{0.809609in}}%
\pgfpathlineto{\pgfqpoint{2.436800in}{0.811658in}}%
\pgfpathlineto{\pgfqpoint{2.441309in}{0.791618in}}%
\pgfpathlineto{\pgfqpoint{2.445818in}{0.821995in}}%
\pgfpathlineto{\pgfqpoint{2.450327in}{0.798108in}}%
\pgfpathlineto{\pgfqpoint{2.454836in}{0.812665in}}%
\pgfpathlineto{\pgfqpoint{2.459345in}{0.798255in}}%
\pgfpathlineto{\pgfqpoint{2.463855in}{0.807429in}}%
\pgfpathlineto{\pgfqpoint{2.468364in}{0.791927in}}%
\pgfpathlineto{\pgfqpoint{2.472873in}{0.804705in}}%
\pgfpathlineto{\pgfqpoint{2.477382in}{0.803945in}}%
\pgfpathlineto{\pgfqpoint{2.481891in}{0.796522in}}%
\pgfpathlineto{\pgfqpoint{2.486400in}{0.811565in}}%
\pgfpathlineto{\pgfqpoint{2.490909in}{0.804132in}}%
\pgfpathlineto{\pgfqpoint{2.495418in}{0.810332in}}%
\pgfpathlineto{\pgfqpoint{2.499927in}{0.800970in}}%
\pgfpathlineto{\pgfqpoint{2.504436in}{0.826645in}}%
\pgfpathlineto{\pgfqpoint{2.508945in}{0.798710in}}%
\pgfpathlineto{\pgfqpoint{2.513455in}{0.804190in}}%
\pgfpathlineto{\pgfqpoint{2.517964in}{0.811773in}}%
\pgfpathlineto{\pgfqpoint{2.522473in}{0.807082in}}%
\pgfpathlineto{\pgfqpoint{2.526982in}{0.805921in}}%
\pgfpathlineto{\pgfqpoint{2.531491in}{0.809307in}}%
\pgfpathlineto{\pgfqpoint{2.536000in}{0.816466in}}%
\pgfpathlineto{\pgfqpoint{2.540509in}{0.833182in}}%
\pgfpathlineto{\pgfqpoint{2.545018in}{0.814150in}}%
\pgfpathlineto{\pgfqpoint{2.549527in}{0.816453in}}%
\pgfpathlineto{\pgfqpoint{2.554036in}{0.831738in}}%
\pgfpathlineto{\pgfqpoint{2.558545in}{0.804550in}}%
\pgfpathlineto{\pgfqpoint{2.563055in}{0.798671in}}%
\pgfpathlineto{\pgfqpoint{2.567564in}{0.813081in}}%
\pgfpathlineto{\pgfqpoint{2.572073in}{0.813731in}}%
\pgfpathlineto{\pgfqpoint{2.576582in}{0.817099in}}%
\pgfpathlineto{\pgfqpoint{2.581091in}{0.812641in}}%
\pgfpathlineto{\pgfqpoint{2.585600in}{0.819406in}}%
\pgfpathlineto{\pgfqpoint{2.590109in}{0.809225in}}%
\pgfpathlineto{\pgfqpoint{2.594618in}{0.813178in}}%
\pgfpathlineto{\pgfqpoint{2.599127in}{0.809678in}}%
\pgfpathlineto{\pgfqpoint{2.603636in}{0.814815in}}%
\pgfpathlineto{\pgfqpoint{2.608145in}{0.823407in}}%
\pgfpathlineto{\pgfqpoint{2.612655in}{0.825175in}}%
\pgfpathlineto{\pgfqpoint{2.621673in}{0.816198in}}%
\pgfpathlineto{\pgfqpoint{2.626182in}{0.824377in}}%
\pgfpathlineto{\pgfqpoint{2.630691in}{0.822080in}}%
\pgfpathlineto{\pgfqpoint{2.635200in}{0.818099in}}%
\pgfpathlineto{\pgfqpoint{2.639709in}{0.819348in}}%
\pgfpathlineto{\pgfqpoint{2.644218in}{0.814338in}}%
\pgfpathlineto{\pgfqpoint{2.648727in}{0.821840in}}%
\pgfpathlineto{\pgfqpoint{2.657745in}{0.829340in}}%
\pgfpathlineto{\pgfqpoint{2.662255in}{0.818503in}}%
\pgfpathlineto{\pgfqpoint{2.666764in}{0.825852in}}%
\pgfpathlineto{\pgfqpoint{2.671273in}{0.828021in}}%
\pgfpathlineto{\pgfqpoint{2.675782in}{0.862427in}}%
\pgfpathlineto{\pgfqpoint{2.684800in}{0.822207in}}%
\pgfpathlineto{\pgfqpoint{2.689309in}{0.830117in}}%
\pgfpathlineto{\pgfqpoint{2.693818in}{0.830322in}}%
\pgfpathlineto{\pgfqpoint{2.698327in}{0.838676in}}%
\pgfpathlineto{\pgfqpoint{2.702836in}{0.826321in}}%
\pgfpathlineto{\pgfqpoint{2.707345in}{0.837319in}}%
\pgfpathlineto{\pgfqpoint{2.711855in}{0.831872in}}%
\pgfpathlineto{\pgfqpoint{2.716364in}{0.820643in}}%
\pgfpathlineto{\pgfqpoint{2.720873in}{0.836315in}}%
\pgfpathlineto{\pgfqpoint{2.725382in}{0.837704in}}%
\pgfpathlineto{\pgfqpoint{2.729891in}{0.847322in}}%
\pgfpathlineto{\pgfqpoint{2.734400in}{0.844858in}}%
\pgfpathlineto{\pgfqpoint{2.738909in}{0.837233in}}%
\pgfpathlineto{\pgfqpoint{2.743418in}{0.838210in}}%
\pgfpathlineto{\pgfqpoint{2.747927in}{0.851894in}}%
\pgfpathlineto{\pgfqpoint{2.752436in}{0.843094in}}%
\pgfpathlineto{\pgfqpoint{2.756945in}{0.850151in}}%
\pgfpathlineto{\pgfqpoint{2.761455in}{0.849937in}}%
\pgfpathlineto{\pgfqpoint{2.765964in}{0.836398in}}%
\pgfpathlineto{\pgfqpoint{2.770473in}{0.838122in}}%
\pgfpathlineto{\pgfqpoint{2.774982in}{0.835336in}}%
\pgfpathlineto{\pgfqpoint{2.779491in}{0.834748in}}%
\pgfpathlineto{\pgfqpoint{2.784000in}{0.832039in}}%
\pgfpathlineto{\pgfqpoint{2.788509in}{0.838029in}}%
\pgfpathlineto{\pgfqpoint{2.793018in}{0.837982in}}%
\pgfpathlineto{\pgfqpoint{2.797527in}{0.841697in}}%
\pgfpathlineto{\pgfqpoint{2.802036in}{0.856712in}}%
\pgfpathlineto{\pgfqpoint{2.806545in}{0.845138in}}%
\pgfpathlineto{\pgfqpoint{2.811055in}{0.844145in}}%
\pgfpathlineto{\pgfqpoint{2.815564in}{0.849094in}}%
\pgfpathlineto{\pgfqpoint{2.820073in}{0.872481in}}%
\pgfpathlineto{\pgfqpoint{2.824582in}{0.850110in}}%
\pgfpathlineto{\pgfqpoint{2.829091in}{0.843425in}}%
\pgfpathlineto{\pgfqpoint{2.838109in}{0.857422in}}%
\pgfpathlineto{\pgfqpoint{2.842618in}{0.852324in}}%
\pgfpathlineto{\pgfqpoint{2.847127in}{0.859259in}}%
\pgfpathlineto{\pgfqpoint{2.851636in}{0.856112in}}%
\pgfpathlineto{\pgfqpoint{2.856145in}{0.842279in}}%
\pgfpathlineto{\pgfqpoint{2.860655in}{0.859329in}}%
\pgfpathlineto{\pgfqpoint{2.865164in}{0.852158in}}%
\pgfpathlineto{\pgfqpoint{2.874182in}{0.855843in}}%
\pgfpathlineto{\pgfqpoint{2.878691in}{0.873274in}}%
\pgfpathlineto{\pgfqpoint{2.883200in}{0.858510in}}%
\pgfpathlineto{\pgfqpoint{2.887709in}{0.861253in}}%
\pgfpathlineto{\pgfqpoint{2.892218in}{0.861766in}}%
\pgfpathlineto{\pgfqpoint{2.896727in}{0.870600in}}%
\pgfpathlineto{\pgfqpoint{2.901236in}{0.872113in}}%
\pgfpathlineto{\pgfqpoint{2.905745in}{0.880924in}}%
\pgfpathlineto{\pgfqpoint{2.910255in}{0.863452in}}%
\pgfpathlineto{\pgfqpoint{2.914764in}{0.863565in}}%
\pgfpathlineto{\pgfqpoint{2.919273in}{0.870581in}}%
\pgfpathlineto{\pgfqpoint{2.923782in}{0.868112in}}%
\pgfpathlineto{\pgfqpoint{2.928291in}{0.871016in}}%
\pgfpathlineto{\pgfqpoint{2.932800in}{0.861292in}}%
\pgfpathlineto{\pgfqpoint{2.937309in}{0.857439in}}%
\pgfpathlineto{\pgfqpoint{2.941818in}{0.868118in}}%
\pgfpathlineto{\pgfqpoint{2.946327in}{0.864837in}}%
\pgfpathlineto{\pgfqpoint{2.950836in}{0.858467in}}%
\pgfpathlineto{\pgfqpoint{2.955345in}{0.863300in}}%
\pgfpathlineto{\pgfqpoint{2.959855in}{0.866074in}}%
\pgfpathlineto{\pgfqpoint{2.964364in}{0.870510in}}%
\pgfpathlineto{\pgfqpoint{2.968873in}{0.879093in}}%
\pgfpathlineto{\pgfqpoint{2.973382in}{0.881235in}}%
\pgfpathlineto{\pgfqpoint{2.977891in}{0.871508in}}%
\pgfpathlineto{\pgfqpoint{2.982400in}{0.883531in}}%
\pgfpathlineto{\pgfqpoint{2.986909in}{0.859646in}}%
\pgfpathlineto{\pgfqpoint{2.991418in}{0.875679in}}%
\pgfpathlineto{\pgfqpoint{2.995927in}{0.907649in}}%
\pgfpathlineto{\pgfqpoint{3.000436in}{0.861788in}}%
\pgfpathlineto{\pgfqpoint{3.004945in}{0.870718in}}%
\pgfpathlineto{\pgfqpoint{3.009455in}{0.886867in}}%
\pgfpathlineto{\pgfqpoint{3.013964in}{0.875135in}}%
\pgfpathlineto{\pgfqpoint{3.018473in}{0.888484in}}%
\pgfpathlineto{\pgfqpoint{3.022982in}{0.889325in}}%
\pgfpathlineto{\pgfqpoint{3.027491in}{0.869910in}}%
\pgfpathlineto{\pgfqpoint{3.032000in}{0.887136in}}%
\pgfpathlineto{\pgfqpoint{3.036509in}{0.885927in}}%
\pgfpathlineto{\pgfqpoint{3.041018in}{0.882990in}}%
\pgfpathlineto{\pgfqpoint{3.045527in}{0.887832in}}%
\pgfpathlineto{\pgfqpoint{3.050036in}{0.890274in}}%
\pgfpathlineto{\pgfqpoint{3.054545in}{0.880906in}}%
\pgfpathlineto{\pgfqpoint{3.059055in}{0.884636in}}%
\pgfpathlineto{\pgfqpoint{3.063564in}{0.908120in}}%
\pgfpathlineto{\pgfqpoint{3.068073in}{0.899737in}}%
\pgfpathlineto{\pgfqpoint{3.072582in}{0.903838in}}%
\pgfpathlineto{\pgfqpoint{3.077091in}{0.892346in}}%
\pgfpathlineto{\pgfqpoint{3.081600in}{0.915982in}}%
\pgfpathlineto{\pgfqpoint{3.086109in}{0.890898in}}%
\pgfpathlineto{\pgfqpoint{3.090618in}{0.902813in}}%
\pgfpathlineto{\pgfqpoint{3.095127in}{0.908885in}}%
\pgfpathlineto{\pgfqpoint{3.104145in}{0.888999in}}%
\pgfpathlineto{\pgfqpoint{3.108655in}{0.903329in}}%
\pgfpathlineto{\pgfqpoint{3.113164in}{0.910831in}}%
\pgfpathlineto{\pgfqpoint{3.117673in}{0.898475in}}%
\pgfpathlineto{\pgfqpoint{3.122182in}{0.893442in}}%
\pgfpathlineto{\pgfqpoint{3.126691in}{0.882318in}}%
\pgfpathlineto{\pgfqpoint{3.131200in}{0.918050in}}%
\pgfpathlineto{\pgfqpoint{3.135709in}{0.892586in}}%
\pgfpathlineto{\pgfqpoint{3.140218in}{0.905046in}}%
\pgfpathlineto{\pgfqpoint{3.144727in}{0.892576in}}%
\pgfpathlineto{\pgfqpoint{3.149236in}{0.906737in}}%
\pgfpathlineto{\pgfqpoint{3.153745in}{0.901536in}}%
\pgfpathlineto{\pgfqpoint{3.158255in}{0.901978in}}%
\pgfpathlineto{\pgfqpoint{3.162764in}{0.910915in}}%
\pgfpathlineto{\pgfqpoint{3.167273in}{0.902992in}}%
\pgfpathlineto{\pgfqpoint{3.176291in}{0.910861in}}%
\pgfpathlineto{\pgfqpoint{3.180800in}{0.906213in}}%
\pgfpathlineto{\pgfqpoint{3.185309in}{0.906514in}}%
\pgfpathlineto{\pgfqpoint{3.189818in}{0.914216in}}%
\pgfpathlineto{\pgfqpoint{3.194327in}{0.909442in}}%
\pgfpathlineto{\pgfqpoint{3.198836in}{0.921742in}}%
\pgfpathlineto{\pgfqpoint{3.203345in}{0.913220in}}%
\pgfpathlineto{\pgfqpoint{3.207855in}{0.910068in}}%
\pgfpathlineto{\pgfqpoint{3.212364in}{0.912045in}}%
\pgfpathlineto{\pgfqpoint{3.216873in}{0.909622in}}%
\pgfpathlineto{\pgfqpoint{3.221382in}{0.920371in}}%
\pgfpathlineto{\pgfqpoint{3.225891in}{0.904364in}}%
\pgfpathlineto{\pgfqpoint{3.230400in}{0.903045in}}%
\pgfpathlineto{\pgfqpoint{3.234909in}{0.917627in}}%
\pgfpathlineto{\pgfqpoint{3.239418in}{0.922369in}}%
\pgfpathlineto{\pgfqpoint{3.243927in}{0.912801in}}%
\pgfpathlineto{\pgfqpoint{3.248436in}{0.925310in}}%
\pgfpathlineto{\pgfqpoint{3.252945in}{0.926533in}}%
\pgfpathlineto{\pgfqpoint{3.257455in}{0.925964in}}%
\pgfpathlineto{\pgfqpoint{3.261964in}{0.937053in}}%
\pgfpathlineto{\pgfqpoint{3.266473in}{0.924695in}}%
\pgfpathlineto{\pgfqpoint{3.270982in}{0.920649in}}%
\pgfpathlineto{\pgfqpoint{3.275491in}{0.929246in}}%
\pgfpathlineto{\pgfqpoint{3.280000in}{0.969237in}}%
\pgfpathlineto{\pgfqpoint{3.284509in}{0.914162in}}%
\pgfpathlineto{\pgfqpoint{3.289018in}{0.952518in}}%
\pgfpathlineto{\pgfqpoint{3.293527in}{0.933915in}}%
\pgfpathlineto{\pgfqpoint{3.298036in}{0.932905in}}%
\pgfpathlineto{\pgfqpoint{3.302545in}{0.942687in}}%
\pgfpathlineto{\pgfqpoint{3.307055in}{0.941997in}}%
\pgfpathlineto{\pgfqpoint{3.311564in}{0.938417in}}%
\pgfpathlineto{\pgfqpoint{3.316073in}{0.936980in}}%
\pgfpathlineto{\pgfqpoint{3.320582in}{0.954356in}}%
\pgfpathlineto{\pgfqpoint{3.325091in}{0.946321in}}%
\pgfpathlineto{\pgfqpoint{3.329600in}{0.930881in}}%
\pgfpathlineto{\pgfqpoint{3.334109in}{0.946574in}}%
\pgfpathlineto{\pgfqpoint{3.338618in}{0.935990in}}%
\pgfpathlineto{\pgfqpoint{3.343127in}{0.930499in}}%
\pgfpathlineto{\pgfqpoint{3.347636in}{0.959344in}}%
\pgfpathlineto{\pgfqpoint{3.352145in}{0.936765in}}%
\pgfpathlineto{\pgfqpoint{3.356655in}{0.943833in}}%
\pgfpathlineto{\pgfqpoint{3.361164in}{0.943099in}}%
\pgfpathlineto{\pgfqpoint{3.365673in}{0.937189in}}%
\pgfpathlineto{\pgfqpoint{3.370182in}{0.946267in}}%
\pgfpathlineto{\pgfqpoint{3.374691in}{0.929958in}}%
\pgfpathlineto{\pgfqpoint{3.379200in}{0.956105in}}%
\pgfpathlineto{\pgfqpoint{3.383709in}{0.935072in}}%
\pgfpathlineto{\pgfqpoint{3.388218in}{0.936177in}}%
\pgfpathlineto{\pgfqpoint{3.392727in}{0.940704in}}%
\pgfpathlineto{\pgfqpoint{3.401745in}{0.964188in}}%
\pgfpathlineto{\pgfqpoint{3.406255in}{0.961081in}}%
\pgfpathlineto{\pgfqpoint{3.410764in}{0.945545in}}%
\pgfpathlineto{\pgfqpoint{3.415273in}{0.950335in}}%
\pgfpathlineto{\pgfqpoint{3.419782in}{0.951109in}}%
\pgfpathlineto{\pgfqpoint{3.424291in}{0.958589in}}%
\pgfpathlineto{\pgfqpoint{3.433309in}{0.962463in}}%
\pgfpathlineto{\pgfqpoint{3.437818in}{0.986249in}}%
\pgfpathlineto{\pgfqpoint{3.442327in}{0.962610in}}%
\pgfpathlineto{\pgfqpoint{3.446836in}{0.952430in}}%
\pgfpathlineto{\pgfqpoint{3.451345in}{0.957679in}}%
\pgfpathlineto{\pgfqpoint{3.455855in}{0.964794in}}%
\pgfpathlineto{\pgfqpoint{3.460364in}{0.969487in}}%
\pgfpathlineto{\pgfqpoint{3.464873in}{0.953067in}}%
\pgfpathlineto{\pgfqpoint{3.469382in}{0.984837in}}%
\pgfpathlineto{\pgfqpoint{3.473891in}{0.969021in}}%
\pgfpathlineto{\pgfqpoint{3.478400in}{0.967342in}}%
\pgfpathlineto{\pgfqpoint{3.482909in}{0.955677in}}%
\pgfpathlineto{\pgfqpoint{3.487418in}{0.974348in}}%
\pgfpathlineto{\pgfqpoint{3.491927in}{0.985465in}}%
\pgfpathlineto{\pgfqpoint{3.496436in}{0.976851in}}%
\pgfpathlineto{\pgfqpoint{3.500945in}{0.953995in}}%
\pgfpathlineto{\pgfqpoint{3.505455in}{0.979197in}}%
\pgfpathlineto{\pgfqpoint{3.509964in}{0.990196in}}%
\pgfpathlineto{\pgfqpoint{3.514473in}{0.973045in}}%
\pgfpathlineto{\pgfqpoint{3.518982in}{0.971174in}}%
\pgfpathlineto{\pgfqpoint{3.523491in}{0.981578in}}%
\pgfpathlineto{\pgfqpoint{3.528000in}{0.968706in}}%
\pgfpathlineto{\pgfqpoint{3.532509in}{0.983846in}}%
\pgfpathlineto{\pgfqpoint{3.537018in}{0.975221in}}%
\pgfpathlineto{\pgfqpoint{3.541527in}{0.986515in}}%
\pgfpathlineto{\pgfqpoint{3.546036in}{0.986203in}}%
\pgfpathlineto{\pgfqpoint{3.550545in}{0.980718in}}%
\pgfpathlineto{\pgfqpoint{3.555055in}{0.971654in}}%
\pgfpathlineto{\pgfqpoint{3.559564in}{0.986528in}}%
\pgfpathlineto{\pgfqpoint{3.564073in}{0.982193in}}%
\pgfpathlineto{\pgfqpoint{3.568582in}{0.991208in}}%
\pgfpathlineto{\pgfqpoint{3.573091in}{0.990964in}}%
\pgfpathlineto{\pgfqpoint{3.577600in}{0.987960in}}%
\pgfpathlineto{\pgfqpoint{3.582109in}{0.994294in}}%
\pgfpathlineto{\pgfqpoint{3.586618in}{1.015135in}}%
\pgfpathlineto{\pgfqpoint{3.591127in}{0.987382in}}%
\pgfpathlineto{\pgfqpoint{3.595636in}{0.981479in}}%
\pgfpathlineto{\pgfqpoint{3.600145in}{0.993139in}}%
\pgfpathlineto{\pgfqpoint{3.604655in}{0.999245in}}%
\pgfpathlineto{\pgfqpoint{3.609164in}{0.992943in}}%
\pgfpathlineto{\pgfqpoint{3.613673in}{1.004776in}}%
\pgfpathlineto{\pgfqpoint{3.622691in}{1.009337in}}%
\pgfpathlineto{\pgfqpoint{3.627200in}{1.036617in}}%
\pgfpathlineto{\pgfqpoint{3.636218in}{0.993411in}}%
\pgfpathlineto{\pgfqpoint{3.640727in}{1.037286in}}%
\pgfpathlineto{\pgfqpoint{3.645236in}{1.035990in}}%
\pgfpathlineto{\pgfqpoint{3.649745in}{1.031630in}}%
\pgfpathlineto{\pgfqpoint{3.654255in}{0.994669in}}%
\pgfpathlineto{\pgfqpoint{3.663273in}{1.038039in}}%
\pgfpathlineto{\pgfqpoint{3.667782in}{1.008967in}}%
\pgfpathlineto{\pgfqpoint{3.672291in}{1.014644in}}%
\pgfpathlineto{\pgfqpoint{3.676800in}{1.027073in}}%
\pgfpathlineto{\pgfqpoint{3.681309in}{1.011614in}}%
\pgfpathlineto{\pgfqpoint{3.685818in}{1.033294in}}%
\pgfpathlineto{\pgfqpoint{3.690327in}{1.022004in}}%
\pgfpathlineto{\pgfqpoint{3.694836in}{1.017758in}}%
\pgfpathlineto{\pgfqpoint{3.699345in}{1.018005in}}%
\pgfpathlineto{\pgfqpoint{3.703855in}{1.010440in}}%
\pgfpathlineto{\pgfqpoint{3.708364in}{1.038926in}}%
\pgfpathlineto{\pgfqpoint{3.712873in}{1.011173in}}%
\pgfpathlineto{\pgfqpoint{3.717382in}{1.033926in}}%
\pgfpathlineto{\pgfqpoint{3.721891in}{1.072975in}}%
\pgfpathlineto{\pgfqpoint{3.726400in}{1.045181in}}%
\pgfpathlineto{\pgfqpoint{3.730909in}{1.045977in}}%
\pgfpathlineto{\pgfqpoint{3.735418in}{1.019935in}}%
\pgfpathlineto{\pgfqpoint{3.739927in}{1.023446in}}%
\pgfpathlineto{\pgfqpoint{3.744436in}{1.029002in}}%
\pgfpathlineto{\pgfqpoint{3.748945in}{1.025123in}}%
\pgfpathlineto{\pgfqpoint{3.753455in}{1.037870in}}%
\pgfpathlineto{\pgfqpoint{3.757964in}{1.020994in}}%
\pgfpathlineto{\pgfqpoint{3.762473in}{1.024241in}}%
\pgfpathlineto{\pgfqpoint{3.766982in}{1.030381in}}%
\pgfpathlineto{\pgfqpoint{3.771491in}{1.033409in}}%
\pgfpathlineto{\pgfqpoint{3.776000in}{1.022599in}}%
\pgfpathlineto{\pgfqpoint{3.780509in}{1.021920in}}%
\pgfpathlineto{\pgfqpoint{3.785018in}{1.046732in}}%
\pgfpathlineto{\pgfqpoint{3.789527in}{1.126568in}}%
\pgfpathlineto{\pgfqpoint{3.794036in}{1.103146in}}%
\pgfpathlineto{\pgfqpoint{3.798545in}{1.243943in}}%
\pgfpathlineto{\pgfqpoint{3.803055in}{1.456832in}}%
\pgfpathlineto{\pgfqpoint{3.807564in}{1.218952in}}%
\pgfpathlineto{\pgfqpoint{3.812073in}{1.495004in}}%
\pgfpathlineto{\pgfqpoint{3.816582in}{1.179379in}}%
\pgfpathlineto{\pgfqpoint{3.821091in}{1.038222in}}%
\pgfpathlineto{\pgfqpoint{3.825600in}{1.031652in}}%
\pgfpathlineto{\pgfqpoint{3.830109in}{1.089543in}}%
\pgfpathlineto{\pgfqpoint{3.834618in}{1.023611in}}%
\pgfpathlineto{\pgfqpoint{3.839127in}{1.028211in}}%
\pgfpathlineto{\pgfqpoint{3.843636in}{1.138138in}}%
\pgfpathlineto{\pgfqpoint{3.848145in}{1.041518in}}%
\pgfpathlineto{\pgfqpoint{3.852655in}{1.082667in}}%
\pgfpathlineto{\pgfqpoint{3.857164in}{1.072651in}}%
\pgfpathlineto{\pgfqpoint{3.866182in}{1.039161in}}%
\pgfpathlineto{\pgfqpoint{3.870691in}{1.047140in}}%
\pgfpathlineto{\pgfqpoint{3.875200in}{1.058337in}}%
\pgfpathlineto{\pgfqpoint{3.879709in}{1.087183in}}%
\pgfpathlineto{\pgfqpoint{3.884218in}{1.106875in}}%
\pgfpathlineto{\pgfqpoint{3.888727in}{1.056549in}}%
\pgfpathlineto{\pgfqpoint{3.893236in}{1.047629in}}%
\pgfpathlineto{\pgfqpoint{3.897745in}{1.046072in}}%
\pgfpathlineto{\pgfqpoint{3.902255in}{1.037084in}}%
\pgfpathlineto{\pgfqpoint{3.906764in}{1.081006in}}%
\pgfpathlineto{\pgfqpoint{3.911273in}{1.088955in}}%
\pgfpathlineto{\pgfqpoint{3.915782in}{1.062601in}}%
\pgfpathlineto{\pgfqpoint{3.920291in}{1.116248in}}%
\pgfpathlineto{\pgfqpoint{3.924800in}{1.052196in}}%
\pgfpathlineto{\pgfqpoint{3.929309in}{1.053617in}}%
\pgfpathlineto{\pgfqpoint{3.938327in}{1.070880in}}%
\pgfpathlineto{\pgfqpoint{3.942836in}{1.070922in}}%
\pgfpathlineto{\pgfqpoint{3.947345in}{1.132448in}}%
\pgfpathlineto{\pgfqpoint{3.951855in}{1.082156in}}%
\pgfpathlineto{\pgfqpoint{3.956364in}{1.067436in}}%
\pgfpathlineto{\pgfqpoint{3.965382in}{1.051879in}}%
\pgfpathlineto{\pgfqpoint{3.969891in}{1.085632in}}%
\pgfpathlineto{\pgfqpoint{3.974400in}{1.076892in}}%
\pgfpathlineto{\pgfqpoint{3.978909in}{1.083755in}}%
\pgfpathlineto{\pgfqpoint{3.983418in}{1.076436in}}%
\pgfpathlineto{\pgfqpoint{3.987927in}{1.080755in}}%
\pgfpathlineto{\pgfqpoint{3.992436in}{1.065442in}}%
\pgfpathlineto{\pgfqpoint{3.996945in}{1.088494in}}%
\pgfpathlineto{\pgfqpoint{4.005964in}{1.064874in}}%
\pgfpathlineto{\pgfqpoint{4.010473in}{1.081516in}}%
\pgfpathlineto{\pgfqpoint{4.014982in}{1.071916in}}%
\pgfpathlineto{\pgfqpoint{4.019491in}{1.093661in}}%
\pgfpathlineto{\pgfqpoint{4.024000in}{1.077355in}}%
\pgfpathlineto{\pgfqpoint{4.028509in}{1.078173in}}%
\pgfpathlineto{\pgfqpoint{4.033018in}{1.109953in}}%
\pgfpathlineto{\pgfqpoint{4.037527in}{1.100191in}}%
\pgfpathlineto{\pgfqpoint{4.042036in}{1.096920in}}%
\pgfpathlineto{\pgfqpoint{4.046545in}{1.079538in}}%
\pgfpathlineto{\pgfqpoint{4.051055in}{1.093999in}}%
\pgfpathlineto{\pgfqpoint{4.055564in}{1.100138in}}%
\pgfpathlineto{\pgfqpoint{4.060073in}{1.096933in}}%
\pgfpathlineto{\pgfqpoint{4.064582in}{1.075587in}}%
\pgfpathlineto{\pgfqpoint{4.069091in}{1.111508in}}%
\pgfpathlineto{\pgfqpoint{4.073600in}{1.103132in}}%
\pgfpathlineto{\pgfqpoint{4.078109in}{1.085898in}}%
\pgfpathlineto{\pgfqpoint{4.082618in}{1.089611in}}%
\pgfpathlineto{\pgfqpoint{4.087127in}{1.106243in}}%
\pgfpathlineto{\pgfqpoint{4.091636in}{1.091355in}}%
\pgfpathlineto{\pgfqpoint{4.096145in}{1.098545in}}%
\pgfpathlineto{\pgfqpoint{4.100655in}{1.109550in}}%
\pgfpathlineto{\pgfqpoint{4.105164in}{1.123551in}}%
\pgfpathlineto{\pgfqpoint{4.109673in}{1.089954in}}%
\pgfpathlineto{\pgfqpoint{4.114182in}{1.103393in}}%
\pgfpathlineto{\pgfqpoint{4.118691in}{1.111893in}}%
\pgfpathlineto{\pgfqpoint{4.123200in}{1.112622in}}%
\pgfpathlineto{\pgfqpoint{4.127709in}{1.124073in}}%
\pgfpathlineto{\pgfqpoint{4.132218in}{1.092953in}}%
\pgfpathlineto{\pgfqpoint{4.136727in}{1.097785in}}%
\pgfpathlineto{\pgfqpoint{4.141236in}{1.116587in}}%
\pgfpathlineto{\pgfqpoint{4.145745in}{1.127065in}}%
\pgfpathlineto{\pgfqpoint{4.150255in}{1.113026in}}%
\pgfpathlineto{\pgfqpoint{4.154764in}{1.108934in}}%
\pgfpathlineto{\pgfqpoint{4.159273in}{1.166909in}}%
\pgfpathlineto{\pgfqpoint{4.168291in}{1.128638in}}%
\pgfpathlineto{\pgfqpoint{4.172800in}{1.120105in}}%
\pgfpathlineto{\pgfqpoint{4.177309in}{1.130730in}}%
\pgfpathlineto{\pgfqpoint{4.181818in}{1.114135in}}%
\pgfpathlineto{\pgfqpoint{4.186327in}{1.136481in}}%
\pgfpathlineto{\pgfqpoint{4.190836in}{1.142693in}}%
\pgfpathlineto{\pgfqpoint{4.195345in}{1.157251in}}%
\pgfpathlineto{\pgfqpoint{4.199855in}{1.132391in}}%
\pgfpathlineto{\pgfqpoint{4.204364in}{1.136346in}}%
\pgfpathlineto{\pgfqpoint{4.208873in}{1.114234in}}%
\pgfpathlineto{\pgfqpoint{4.213382in}{1.134462in}}%
\pgfpathlineto{\pgfqpoint{4.217891in}{1.168285in}}%
\pgfpathlineto{\pgfqpoint{4.222400in}{1.142848in}}%
\pgfpathlineto{\pgfqpoint{4.226909in}{1.126896in}}%
\pgfpathlineto{\pgfqpoint{4.231418in}{1.167610in}}%
\pgfpathlineto{\pgfqpoint{4.235927in}{1.144410in}}%
\pgfpathlineto{\pgfqpoint{4.240436in}{1.148643in}}%
\pgfpathlineto{\pgfqpoint{4.244945in}{1.130835in}}%
\pgfpathlineto{\pgfqpoint{4.249455in}{1.139160in}}%
\pgfpathlineto{\pgfqpoint{4.253964in}{1.163175in}}%
\pgfpathlineto{\pgfqpoint{4.258473in}{1.141705in}}%
\pgfpathlineto{\pgfqpoint{4.262982in}{1.158906in}}%
\pgfpathlineto{\pgfqpoint{4.267491in}{1.137144in}}%
\pgfpathlineto{\pgfqpoint{4.272000in}{1.150545in}}%
\pgfpathlineto{\pgfqpoint{4.276509in}{1.178855in}}%
\pgfpathlineto{\pgfqpoint{4.281018in}{1.129956in}}%
\pgfpathlineto{\pgfqpoint{4.285527in}{1.153262in}}%
\pgfpathlineto{\pgfqpoint{4.294545in}{1.165362in}}%
\pgfpathlineto{\pgfqpoint{4.299055in}{1.175348in}}%
\pgfpathlineto{\pgfqpoint{4.303564in}{1.167923in}}%
\pgfpathlineto{\pgfqpoint{4.308073in}{1.191383in}}%
\pgfpathlineto{\pgfqpoint{4.312582in}{1.143633in}}%
\pgfpathlineto{\pgfqpoint{4.317091in}{1.160591in}}%
\pgfpathlineto{\pgfqpoint{4.321600in}{1.153518in}}%
\pgfpathlineto{\pgfqpoint{4.326109in}{1.173078in}}%
\pgfpathlineto{\pgfqpoint{4.330618in}{1.161259in}}%
\pgfpathlineto{\pgfqpoint{4.335127in}{1.156684in}}%
\pgfpathlineto{\pgfqpoint{4.339636in}{1.167777in}}%
\pgfpathlineto{\pgfqpoint{4.344145in}{1.200157in}}%
\pgfpathlineto{\pgfqpoint{4.348655in}{1.162169in}}%
\pgfpathlineto{\pgfqpoint{4.353164in}{1.200738in}}%
\pgfpathlineto{\pgfqpoint{4.357673in}{1.165316in}}%
\pgfpathlineto{\pgfqpoint{4.362182in}{1.145033in}}%
\pgfpathlineto{\pgfqpoint{4.366691in}{1.169837in}}%
\pgfpathlineto{\pgfqpoint{4.371200in}{1.162208in}}%
\pgfpathlineto{\pgfqpoint{4.375709in}{1.182161in}}%
\pgfpathlineto{\pgfqpoint{4.380218in}{1.171183in}}%
\pgfpathlineto{\pgfqpoint{4.384727in}{1.180291in}}%
\pgfpathlineto{\pgfqpoint{4.389236in}{1.176606in}}%
\pgfpathlineto{\pgfqpoint{4.393745in}{1.201624in}}%
\pgfpathlineto{\pgfqpoint{4.398255in}{1.176077in}}%
\pgfpathlineto{\pgfqpoint{4.402764in}{1.178548in}}%
\pgfpathlineto{\pgfqpoint{4.407273in}{1.205167in}}%
\pgfpathlineto{\pgfqpoint{4.411782in}{1.189035in}}%
\pgfpathlineto{\pgfqpoint{4.416291in}{1.182240in}}%
\pgfpathlineto{\pgfqpoint{4.420800in}{1.215306in}}%
\pgfpathlineto{\pgfqpoint{4.425309in}{1.166606in}}%
\pgfpathlineto{\pgfqpoint{4.429818in}{1.182301in}}%
\pgfpathlineto{\pgfqpoint{4.434327in}{1.221378in}}%
\pgfpathlineto{\pgfqpoint{4.438836in}{1.175815in}}%
\pgfpathlineto{\pgfqpoint{4.443345in}{1.176386in}}%
\pgfpathlineto{\pgfqpoint{4.447855in}{1.185812in}}%
\pgfpathlineto{\pgfqpoint{4.452364in}{1.224771in}}%
\pgfpathlineto{\pgfqpoint{4.456873in}{1.181931in}}%
\pgfpathlineto{\pgfqpoint{4.461382in}{1.212502in}}%
\pgfpathlineto{\pgfqpoint{4.465891in}{1.172934in}}%
\pgfpathlineto{\pgfqpoint{4.470400in}{1.203539in}}%
\pgfpathlineto{\pgfqpoint{4.474909in}{1.178467in}}%
\pgfpathlineto{\pgfqpoint{4.479418in}{1.213121in}}%
\pgfpathlineto{\pgfqpoint{4.483927in}{1.213612in}}%
\pgfpathlineto{\pgfqpoint{4.488436in}{1.196946in}}%
\pgfpathlineto{\pgfqpoint{4.492945in}{1.222331in}}%
\pgfpathlineto{\pgfqpoint{4.497455in}{1.207329in}}%
\pgfpathlineto{\pgfqpoint{4.501964in}{1.203635in}}%
\pgfpathlineto{\pgfqpoint{4.506473in}{1.217945in}}%
\pgfpathlineto{\pgfqpoint{4.510982in}{1.213963in}}%
\pgfpathlineto{\pgfqpoint{4.515491in}{1.193110in}}%
\pgfpathlineto{\pgfqpoint{4.520000in}{1.224346in}}%
\pgfpathlineto{\pgfqpoint{4.524509in}{1.192859in}}%
\pgfpathlineto{\pgfqpoint{4.529018in}{1.289344in}}%
\pgfpathlineto{\pgfqpoint{4.533527in}{1.207994in}}%
\pgfpathlineto{\pgfqpoint{4.538036in}{1.207270in}}%
\pgfpathlineto{\pgfqpoint{4.542545in}{1.259508in}}%
\pgfpathlineto{\pgfqpoint{4.551564in}{1.217425in}}%
\pgfpathlineto{\pgfqpoint{4.556073in}{1.208645in}}%
\pgfpathlineto{\pgfqpoint{4.560582in}{1.211206in}}%
\pgfpathlineto{\pgfqpoint{4.565091in}{1.237301in}}%
\pgfpathlineto{\pgfqpoint{4.574109in}{1.227685in}}%
\pgfpathlineto{\pgfqpoint{4.578618in}{1.217258in}}%
\pgfpathlineto{\pgfqpoint{4.583127in}{1.230432in}}%
\pgfpathlineto{\pgfqpoint{4.587636in}{1.257789in}}%
\pgfpathlineto{\pgfqpoint{4.592145in}{1.237241in}}%
\pgfpathlineto{\pgfqpoint{4.596655in}{1.273767in}}%
\pgfpathlineto{\pgfqpoint{4.601164in}{1.230421in}}%
\pgfpathlineto{\pgfqpoint{4.605673in}{1.260903in}}%
\pgfpathlineto{\pgfqpoint{4.610182in}{1.238221in}}%
\pgfpathlineto{\pgfqpoint{4.614691in}{1.246696in}}%
\pgfpathlineto{\pgfqpoint{4.619200in}{1.266237in}}%
\pgfpathlineto{\pgfqpoint{4.623709in}{1.239850in}}%
\pgfpathlineto{\pgfqpoint{4.628218in}{1.242257in}}%
\pgfpathlineto{\pgfqpoint{4.632727in}{1.239712in}}%
\pgfpathlineto{\pgfqpoint{4.637236in}{1.203360in}}%
\pgfpathlineto{\pgfqpoint{4.641745in}{1.310418in}}%
\pgfpathlineto{\pgfqpoint{4.646255in}{1.348813in}}%
\pgfpathlineto{\pgfqpoint{4.650764in}{1.237189in}}%
\pgfpathlineto{\pgfqpoint{4.655273in}{1.285857in}}%
\pgfpathlineto{\pgfqpoint{4.659782in}{1.310691in}}%
\pgfpathlineto{\pgfqpoint{4.664291in}{1.358175in}}%
\pgfpathlineto{\pgfqpoint{4.668800in}{1.266373in}}%
\pgfpathlineto{\pgfqpoint{4.673309in}{1.534395in}}%
\pgfpathlineto{\pgfqpoint{4.677818in}{1.298134in}}%
\pgfpathlineto{\pgfqpoint{4.686836in}{1.282173in}}%
\pgfpathlineto{\pgfqpoint{4.691345in}{1.283476in}}%
\pgfpathlineto{\pgfqpoint{4.695855in}{1.276171in}}%
\pgfpathlineto{\pgfqpoint{4.700364in}{1.276728in}}%
\pgfpathlineto{\pgfqpoint{4.704873in}{1.356079in}}%
\pgfpathlineto{\pgfqpoint{4.709382in}{1.606452in}}%
\pgfpathlineto{\pgfqpoint{4.713891in}{1.304072in}}%
\pgfpathlineto{\pgfqpoint{4.718400in}{1.397683in}}%
\pgfpathlineto{\pgfqpoint{4.722909in}{1.349347in}}%
\pgfpathlineto{\pgfqpoint{4.727418in}{1.361934in}}%
\pgfpathlineto{\pgfqpoint{4.731927in}{1.335734in}}%
\pgfpathlineto{\pgfqpoint{4.736436in}{1.413328in}}%
\pgfpathlineto{\pgfqpoint{4.740945in}{1.323823in}}%
\pgfpathlineto{\pgfqpoint{4.745455in}{1.294369in}}%
\pgfpathlineto{\pgfqpoint{4.749964in}{1.283122in}}%
\pgfpathlineto{\pgfqpoint{4.754473in}{1.260566in}}%
\pgfpathlineto{\pgfqpoint{4.758982in}{1.210487in}}%
\pgfpathlineto{\pgfqpoint{4.763491in}{1.248356in}}%
\pgfpathlineto{\pgfqpoint{4.768000in}{1.229331in}}%
\pgfpathlineto{\pgfqpoint{4.772509in}{1.234580in}}%
\pgfpathlineto{\pgfqpoint{4.777018in}{1.249839in}}%
\pgfpathlineto{\pgfqpoint{4.781527in}{1.243840in}}%
\pgfpathlineto{\pgfqpoint{4.786036in}{1.220069in}}%
\pgfpathlineto{\pgfqpoint{4.790545in}{1.222902in}}%
\pgfpathlineto{\pgfqpoint{4.795055in}{1.243137in}}%
\pgfpathlineto{\pgfqpoint{4.799564in}{1.224119in}}%
\pgfpathlineto{\pgfqpoint{4.804073in}{1.296662in}}%
\pgfpathlineto{\pgfqpoint{4.808582in}{1.482509in}}%
\pgfpathlineto{\pgfqpoint{4.813091in}{1.537592in}}%
\pgfpathlineto{\pgfqpoint{4.817600in}{1.225002in}}%
\pgfpathlineto{\pgfqpoint{4.822109in}{1.233983in}}%
\pgfpathlineto{\pgfqpoint{4.826618in}{1.253780in}}%
\pgfpathlineto{\pgfqpoint{4.831127in}{1.708332in}}%
\pgfpathlineto{\pgfqpoint{4.835636in}{1.351340in}}%
\pgfpathlineto{\pgfqpoint{4.840145in}{1.243738in}}%
\pgfpathlineto{\pgfqpoint{4.844655in}{1.347559in}}%
\pgfpathlineto{\pgfqpoint{4.849164in}{1.992337in}}%
\pgfpathlineto{\pgfqpoint{4.853673in}{1.330058in}}%
\pgfpathlineto{\pgfqpoint{4.858182in}{1.311449in}}%
\pgfpathlineto{\pgfqpoint{4.862691in}{1.324107in}}%
\pgfpathlineto{\pgfqpoint{4.867200in}{1.418403in}}%
\pgfpathlineto{\pgfqpoint{4.871709in}{1.537248in}}%
\pgfpathlineto{\pgfqpoint{4.876218in}{1.399164in}}%
\pgfpathlineto{\pgfqpoint{4.880727in}{1.380937in}}%
\pgfpathlineto{\pgfqpoint{4.885236in}{1.447986in}}%
\pgfpathlineto{\pgfqpoint{4.889745in}{1.322328in}}%
\pgfpathlineto{\pgfqpoint{4.898764in}{1.337167in}}%
\pgfpathlineto{\pgfqpoint{4.903273in}{1.347570in}}%
\pgfpathlineto{\pgfqpoint{4.907782in}{1.376560in}}%
\pgfpathlineto{\pgfqpoint{4.912291in}{1.357796in}}%
\pgfpathlineto{\pgfqpoint{4.916800in}{1.321278in}}%
\pgfpathlineto{\pgfqpoint{4.921309in}{1.333839in}}%
\pgfpathlineto{\pgfqpoint{4.925818in}{1.338268in}}%
\pgfpathlineto{\pgfqpoint{4.934836in}{1.362555in}}%
\pgfpathlineto{\pgfqpoint{4.939345in}{1.355629in}}%
\pgfpathlineto{\pgfqpoint{4.943855in}{1.382092in}}%
\pgfpathlineto{\pgfqpoint{4.948364in}{1.357149in}}%
\pgfpathlineto{\pgfqpoint{4.952873in}{1.362409in}}%
\pgfpathlineto{\pgfqpoint{4.957382in}{1.344363in}}%
\pgfpathlineto{\pgfqpoint{4.961891in}{1.366574in}}%
\pgfpathlineto{\pgfqpoint{4.966400in}{1.339435in}}%
\pgfpathlineto{\pgfqpoint{4.970909in}{1.392513in}}%
\pgfpathlineto{\pgfqpoint{4.975418in}{1.384871in}}%
\pgfpathlineto{\pgfqpoint{4.979927in}{1.320700in}}%
\pgfpathlineto{\pgfqpoint{4.984436in}{1.372143in}}%
\pgfpathlineto{\pgfqpoint{4.988945in}{1.344620in}}%
\pgfpathlineto{\pgfqpoint{4.993455in}{1.376836in}}%
\pgfpathlineto{\pgfqpoint{4.997964in}{1.368152in}}%
\pgfpathlineto{\pgfqpoint{5.002473in}{1.383542in}}%
\pgfpathlineto{\pgfqpoint{5.006982in}{1.404576in}}%
\pgfpathlineto{\pgfqpoint{5.011491in}{1.347453in}}%
\pgfpathlineto{\pgfqpoint{5.016000in}{1.376241in}}%
\pgfpathlineto{\pgfqpoint{5.025018in}{1.353719in}}%
\pgfpathlineto{\pgfqpoint{5.029527in}{1.373559in}}%
\pgfpathlineto{\pgfqpoint{5.034036in}{1.398620in}}%
\pgfpathlineto{\pgfqpoint{5.038545in}{1.508431in}}%
\pgfpathlineto{\pgfqpoint{5.043055in}{1.419768in}}%
\pgfpathlineto{\pgfqpoint{5.047564in}{1.376277in}}%
\pgfpathlineto{\pgfqpoint{5.052073in}{1.514869in}}%
\pgfpathlineto{\pgfqpoint{5.056582in}{1.384484in}}%
\pgfpathlineto{\pgfqpoint{5.065600in}{1.403471in}}%
\pgfpathlineto{\pgfqpoint{5.070109in}{1.398500in}}%
\pgfpathlineto{\pgfqpoint{5.074618in}{1.400992in}}%
\pgfpathlineto{\pgfqpoint{5.079127in}{1.397675in}}%
\pgfpathlineto{\pgfqpoint{5.083636in}{1.367494in}}%
\pgfpathlineto{\pgfqpoint{5.088145in}{1.374200in}}%
\pgfpathlineto{\pgfqpoint{5.092655in}{1.388787in}}%
\pgfpathlineto{\pgfqpoint{5.097164in}{1.387276in}}%
\pgfpathlineto{\pgfqpoint{5.101673in}{1.403638in}}%
\pgfpathlineto{\pgfqpoint{5.106182in}{1.373622in}}%
\pgfpathlineto{\pgfqpoint{5.110691in}{1.376453in}}%
\pgfpathlineto{\pgfqpoint{5.115200in}{1.500176in}}%
\pgfpathlineto{\pgfqpoint{5.119709in}{1.355786in}}%
\pgfpathlineto{\pgfqpoint{5.124218in}{1.510835in}}%
\pgfpathlineto{\pgfqpoint{5.128727in}{1.371560in}}%
\pgfpathlineto{\pgfqpoint{5.133236in}{1.375616in}}%
\pgfpathlineto{\pgfqpoint{5.137745in}{1.404872in}}%
\pgfpathlineto{\pgfqpoint{5.142255in}{1.409763in}}%
\pgfpathlineto{\pgfqpoint{5.146764in}{1.492550in}}%
\pgfpathlineto{\pgfqpoint{5.151273in}{2.160008in}}%
\pgfpathlineto{\pgfqpoint{5.155782in}{1.811493in}}%
\pgfpathlineto{\pgfqpoint{5.160291in}{2.055033in}}%
\pgfpathlineto{\pgfqpoint{5.164800in}{1.631569in}}%
\pgfpathlineto{\pgfqpoint{5.169309in}{1.450465in}}%
\pgfpathlineto{\pgfqpoint{5.173818in}{1.489064in}}%
\pgfpathlineto{\pgfqpoint{5.178327in}{1.461260in}}%
\pgfpathlineto{\pgfqpoint{5.187345in}{1.394223in}}%
\pgfpathlineto{\pgfqpoint{5.191855in}{1.443888in}}%
\pgfpathlineto{\pgfqpoint{5.196364in}{1.438686in}}%
\pgfpathlineto{\pgfqpoint{5.200873in}{1.450730in}}%
\pgfpathlineto{\pgfqpoint{5.205382in}{1.480719in}}%
\pgfpathlineto{\pgfqpoint{5.214400in}{1.434928in}}%
\pgfpathlineto{\pgfqpoint{5.218909in}{1.461477in}}%
\pgfpathlineto{\pgfqpoint{5.223418in}{1.418419in}}%
\pgfpathlineto{\pgfqpoint{5.227927in}{1.439736in}}%
\pgfpathlineto{\pgfqpoint{5.232436in}{1.471903in}}%
\pgfpathlineto{\pgfqpoint{5.241455in}{1.450320in}}%
\pgfpathlineto{\pgfqpoint{5.245964in}{1.447682in}}%
\pgfpathlineto{\pgfqpoint{5.250473in}{1.892995in}}%
\pgfpathlineto{\pgfqpoint{5.254982in}{1.442774in}}%
\pgfpathlineto{\pgfqpoint{5.259491in}{1.878738in}}%
\pgfpathlineto{\pgfqpoint{5.264000in}{1.728197in}}%
\pgfpathlineto{\pgfqpoint{5.268509in}{1.809776in}}%
\pgfpathlineto{\pgfqpoint{5.273018in}{1.472871in}}%
\pgfpathlineto{\pgfqpoint{5.277527in}{1.452782in}}%
\pgfpathlineto{\pgfqpoint{5.282036in}{1.509770in}}%
\pgfpathlineto{\pgfqpoint{5.286545in}{1.454767in}}%
\pgfpathlineto{\pgfqpoint{5.291055in}{1.518656in}}%
\pgfpathlineto{\pgfqpoint{5.295564in}{1.470719in}}%
\pgfpathlineto{\pgfqpoint{5.300073in}{1.466128in}}%
\pgfpathlineto{\pgfqpoint{5.304582in}{1.501172in}}%
\pgfpathlineto{\pgfqpoint{5.309091in}{1.452809in}}%
\pgfpathlineto{\pgfqpoint{5.313600in}{1.472019in}}%
\pgfpathlineto{\pgfqpoint{5.318109in}{1.463682in}}%
\pgfpathlineto{\pgfqpoint{5.322618in}{1.470668in}}%
\pgfpathlineto{\pgfqpoint{5.327127in}{1.475192in}}%
\pgfpathlineto{\pgfqpoint{5.331636in}{1.473749in}}%
\pgfpathlineto{\pgfqpoint{5.336145in}{1.533164in}}%
\pgfpathlineto{\pgfqpoint{5.340655in}{1.452424in}}%
\pgfpathlineto{\pgfqpoint{5.345164in}{1.473795in}}%
\pgfpathlineto{\pgfqpoint{5.349673in}{1.454928in}}%
\pgfpathlineto{\pgfqpoint{5.354182in}{1.488892in}}%
\pgfpathlineto{\pgfqpoint{5.358691in}{1.511956in}}%
\pgfpathlineto{\pgfqpoint{5.363200in}{1.488128in}}%
\pgfpathlineto{\pgfqpoint{5.367709in}{1.487831in}}%
\pgfpathlineto{\pgfqpoint{5.372218in}{1.458155in}}%
\pgfpathlineto{\pgfqpoint{5.376727in}{1.509445in}}%
\pgfpathlineto{\pgfqpoint{5.381236in}{1.478477in}}%
\pgfpathlineto{\pgfqpoint{5.385745in}{1.479157in}}%
\pgfpathlineto{\pgfqpoint{5.390255in}{1.432176in}}%
\pgfpathlineto{\pgfqpoint{5.394764in}{1.485028in}}%
\pgfpathlineto{\pgfqpoint{5.399273in}{1.486802in}}%
\pgfpathlineto{\pgfqpoint{5.403782in}{1.573947in}}%
\pgfpathlineto{\pgfqpoint{5.408291in}{1.486210in}}%
\pgfpathlineto{\pgfqpoint{5.412800in}{1.455436in}}%
\pgfpathlineto{\pgfqpoint{5.417309in}{1.517920in}}%
\pgfpathlineto{\pgfqpoint{5.421818in}{1.502740in}}%
\pgfpathlineto{\pgfqpoint{5.426327in}{1.531073in}}%
\pgfpathlineto{\pgfqpoint{5.430836in}{1.573051in}}%
\pgfpathlineto{\pgfqpoint{5.435345in}{1.502244in}}%
\pgfpathlineto{\pgfqpoint{5.444364in}{1.487533in}}%
\pgfpathlineto{\pgfqpoint{5.448873in}{1.521274in}}%
\pgfpathlineto{\pgfqpoint{5.453382in}{1.488489in}}%
\pgfpathlineto{\pgfqpoint{5.457891in}{1.480757in}}%
\pgfpathlineto{\pgfqpoint{5.462400in}{1.534537in}}%
\pgfpathlineto{\pgfqpoint{5.466909in}{1.551471in}}%
\pgfpathlineto{\pgfqpoint{5.471418in}{1.534043in}}%
\pgfpathlineto{\pgfqpoint{5.475927in}{1.492120in}}%
\pgfpathlineto{\pgfqpoint{5.480436in}{1.503837in}}%
\pgfpathlineto{\pgfqpoint{5.484945in}{1.496471in}}%
\pgfpathlineto{\pgfqpoint{5.489455in}{1.547101in}}%
\pgfpathlineto{\pgfqpoint{5.493964in}{1.491933in}}%
\pgfpathlineto{\pgfqpoint{5.498473in}{1.471887in}}%
\pgfpathlineto{\pgfqpoint{5.502982in}{1.600771in}}%
\pgfpathlineto{\pgfqpoint{5.507491in}{1.564784in}}%
\pgfpathlineto{\pgfqpoint{5.512000in}{1.574922in}}%
\pgfpathlineto{\pgfqpoint{5.516509in}{1.512957in}}%
\pgfpathlineto{\pgfqpoint{5.521018in}{1.561806in}}%
\pgfpathlineto{\pgfqpoint{5.525527in}{1.502672in}}%
\pgfpathlineto{\pgfqpoint{5.530036in}{1.560320in}}%
\pgfpathlineto{\pgfqpoint{5.534545in}{1.597216in}}%
\pgfpathlineto{\pgfqpoint{5.534545in}{1.597216in}}%
\pgfusepath{stroke}%
\end{pgfscope}%
\begin{pgfscope}%
\pgfpathrectangle{\pgfqpoint{0.800000in}{0.528000in}}{\pgfqpoint{4.960000in}{3.696000in}}%
\pgfusepath{clip}%
\pgfsetrectcap%
\pgfsetroundjoin%
\pgfsetlinewidth{1.505625pt}%
\definecolor{currentstroke}{rgb}{0.000000,1.000000,1.000000}%
\pgfsetstrokecolor{currentstroke}%
\pgfsetdash{}{0pt}%
\pgfpathmoveto{\pgfqpoint{1.025455in}{0.696000in}}%
\pgfpathlineto{\pgfqpoint{1.038982in}{0.696075in}}%
\pgfpathlineto{\pgfqpoint{1.043491in}{0.697457in}}%
\pgfpathlineto{\pgfqpoint{1.075055in}{0.696188in}}%
\pgfpathlineto{\pgfqpoint{1.079564in}{0.697236in}}%
\pgfpathlineto{\pgfqpoint{1.084073in}{0.696240in}}%
\pgfpathlineto{\pgfqpoint{1.088582in}{0.697836in}}%
\pgfpathlineto{\pgfqpoint{1.093091in}{0.696247in}}%
\pgfpathlineto{\pgfqpoint{1.097600in}{0.697374in}}%
\pgfpathlineto{\pgfqpoint{1.102109in}{0.696271in}}%
\pgfpathlineto{\pgfqpoint{1.178764in}{0.696662in}}%
\pgfpathlineto{\pgfqpoint{1.183273in}{0.698609in}}%
\pgfpathlineto{\pgfqpoint{1.187782in}{0.696643in}}%
\pgfpathlineto{\pgfqpoint{1.219345in}{0.696766in}}%
\pgfpathlineto{\pgfqpoint{1.223855in}{0.698944in}}%
\pgfpathlineto{\pgfqpoint{1.228364in}{0.696812in}}%
\pgfpathlineto{\pgfqpoint{1.237382in}{0.697011in}}%
\pgfpathlineto{\pgfqpoint{1.241891in}{0.699456in}}%
\pgfpathlineto{\pgfqpoint{1.246400in}{0.697347in}}%
\pgfpathlineto{\pgfqpoint{1.255418in}{0.696957in}}%
\pgfpathlineto{\pgfqpoint{1.268945in}{0.697012in}}%
\pgfpathlineto{\pgfqpoint{1.341091in}{0.697833in}}%
\pgfpathlineto{\pgfqpoint{1.345600in}{0.699931in}}%
\pgfpathlineto{\pgfqpoint{1.350109in}{0.697912in}}%
\pgfpathlineto{\pgfqpoint{1.354618in}{0.701020in}}%
\pgfpathlineto{\pgfqpoint{1.359127in}{0.700729in}}%
\pgfpathlineto{\pgfqpoint{1.368145in}{0.702189in}}%
\pgfpathlineto{\pgfqpoint{1.372655in}{0.699161in}}%
\pgfpathlineto{\pgfqpoint{1.377164in}{0.698963in}}%
\pgfpathlineto{\pgfqpoint{1.390691in}{0.703373in}}%
\pgfpathlineto{\pgfqpoint{1.395200in}{0.701114in}}%
\pgfpathlineto{\pgfqpoint{1.399709in}{0.701918in}}%
\pgfpathlineto{\pgfqpoint{1.404218in}{0.698490in}}%
\pgfpathlineto{\pgfqpoint{1.408727in}{0.699229in}}%
\pgfpathlineto{\pgfqpoint{1.413236in}{0.697698in}}%
\pgfpathlineto{\pgfqpoint{1.422255in}{0.697639in}}%
\pgfpathlineto{\pgfqpoint{1.503418in}{0.698247in}}%
\pgfpathlineto{\pgfqpoint{1.507927in}{0.702913in}}%
\pgfpathlineto{\pgfqpoint{1.512436in}{0.698851in}}%
\pgfpathlineto{\pgfqpoint{1.521455in}{0.698101in}}%
\pgfpathlineto{\pgfqpoint{1.571055in}{0.698266in}}%
\pgfpathlineto{\pgfqpoint{1.575564in}{0.700773in}}%
\pgfpathlineto{\pgfqpoint{1.580073in}{0.700768in}}%
\pgfpathlineto{\pgfqpoint{1.584582in}{0.702335in}}%
\pgfpathlineto{\pgfqpoint{1.589091in}{0.701964in}}%
\pgfpathlineto{\pgfqpoint{1.593600in}{0.699867in}}%
\pgfpathlineto{\pgfqpoint{1.598109in}{0.701839in}}%
\pgfpathlineto{\pgfqpoint{1.611636in}{0.702546in}}%
\pgfpathlineto{\pgfqpoint{1.616145in}{0.704788in}}%
\pgfpathlineto{\pgfqpoint{1.620655in}{0.703506in}}%
\pgfpathlineto{\pgfqpoint{1.625164in}{0.705654in}}%
\pgfpathlineto{\pgfqpoint{1.629673in}{0.698969in}}%
\pgfpathlineto{\pgfqpoint{1.643200in}{0.698591in}}%
\pgfpathlineto{\pgfqpoint{1.661236in}{0.698954in}}%
\pgfpathlineto{\pgfqpoint{1.665745in}{0.701633in}}%
\pgfpathlineto{\pgfqpoint{1.670255in}{0.702266in}}%
\pgfpathlineto{\pgfqpoint{1.674764in}{0.700100in}}%
\pgfpathlineto{\pgfqpoint{1.679273in}{0.704029in}}%
\pgfpathlineto{\pgfqpoint{1.683782in}{0.698798in}}%
\pgfpathlineto{\pgfqpoint{1.701818in}{0.701770in}}%
\pgfpathlineto{\pgfqpoint{1.706327in}{0.699123in}}%
\pgfpathlineto{\pgfqpoint{1.710836in}{0.703857in}}%
\pgfpathlineto{\pgfqpoint{1.715345in}{0.702867in}}%
\pgfpathlineto{\pgfqpoint{1.719855in}{0.698943in}}%
\pgfpathlineto{\pgfqpoint{1.746909in}{0.699057in}}%
\pgfpathlineto{\pgfqpoint{1.751418in}{0.703385in}}%
\pgfpathlineto{\pgfqpoint{1.760436in}{0.699448in}}%
\pgfpathlineto{\pgfqpoint{1.764945in}{0.701430in}}%
\pgfpathlineto{\pgfqpoint{1.769455in}{0.701933in}}%
\pgfpathlineto{\pgfqpoint{1.773964in}{0.699990in}}%
\pgfpathlineto{\pgfqpoint{1.778473in}{0.702052in}}%
\pgfpathlineto{\pgfqpoint{1.782982in}{0.706431in}}%
\pgfpathlineto{\pgfqpoint{1.787491in}{0.701346in}}%
\pgfpathlineto{\pgfqpoint{1.792000in}{0.700571in}}%
\pgfpathlineto{\pgfqpoint{1.796509in}{0.704014in}}%
\pgfpathlineto{\pgfqpoint{1.801018in}{0.701346in}}%
\pgfpathlineto{\pgfqpoint{1.805527in}{0.703653in}}%
\pgfpathlineto{\pgfqpoint{1.810036in}{0.699296in}}%
\pgfpathlineto{\pgfqpoint{1.814545in}{0.699399in}}%
\pgfpathlineto{\pgfqpoint{1.819055in}{0.706199in}}%
\pgfpathlineto{\pgfqpoint{1.828073in}{0.699568in}}%
\pgfpathlineto{\pgfqpoint{1.837091in}{0.700449in}}%
\pgfpathlineto{\pgfqpoint{1.841600in}{0.699499in}}%
\pgfpathlineto{\pgfqpoint{1.850618in}{0.704710in}}%
\pgfpathlineto{\pgfqpoint{1.859636in}{0.699562in}}%
\pgfpathlineto{\pgfqpoint{1.877673in}{0.699687in}}%
\pgfpathlineto{\pgfqpoint{1.882182in}{0.704879in}}%
\pgfpathlineto{\pgfqpoint{1.886691in}{0.699809in}}%
\pgfpathlineto{\pgfqpoint{1.900218in}{0.700082in}}%
\pgfpathlineto{\pgfqpoint{1.904727in}{0.701100in}}%
\pgfpathlineto{\pgfqpoint{1.909236in}{0.699791in}}%
\pgfpathlineto{\pgfqpoint{1.949818in}{0.699957in}}%
\pgfpathlineto{\pgfqpoint{1.954327in}{0.701745in}}%
\pgfpathlineto{\pgfqpoint{1.958836in}{0.701821in}}%
\pgfpathlineto{\pgfqpoint{1.967855in}{0.700063in}}%
\pgfpathlineto{\pgfqpoint{1.972364in}{0.700048in}}%
\pgfpathlineto{\pgfqpoint{1.976873in}{0.702563in}}%
\pgfpathlineto{\pgfqpoint{1.981382in}{0.700154in}}%
\pgfpathlineto{\pgfqpoint{1.985891in}{0.701568in}}%
\pgfpathlineto{\pgfqpoint{1.990400in}{0.700278in}}%
\pgfpathlineto{\pgfqpoint{1.999418in}{0.700247in}}%
\pgfpathlineto{\pgfqpoint{2.003927in}{0.701447in}}%
\pgfpathlineto{\pgfqpoint{2.012945in}{0.700308in}}%
\pgfpathlineto{\pgfqpoint{2.017455in}{0.702290in}}%
\pgfpathlineto{\pgfqpoint{2.021964in}{0.700501in}}%
\pgfpathlineto{\pgfqpoint{2.026473in}{0.705068in}}%
\pgfpathlineto{\pgfqpoint{2.030982in}{0.701151in}}%
\pgfpathlineto{\pgfqpoint{2.035491in}{0.703426in}}%
\pgfpathlineto{\pgfqpoint{2.040000in}{0.702948in}}%
\pgfpathlineto{\pgfqpoint{2.044509in}{0.704484in}}%
\pgfpathlineto{\pgfqpoint{2.049018in}{0.700430in}}%
\pgfpathlineto{\pgfqpoint{2.053527in}{0.702121in}}%
\pgfpathlineto{\pgfqpoint{2.089600in}{0.700633in}}%
\pgfpathlineto{\pgfqpoint{2.094109in}{0.702328in}}%
\pgfpathlineto{\pgfqpoint{2.098618in}{0.700776in}}%
\pgfpathlineto{\pgfqpoint{2.107636in}{0.702729in}}%
\pgfpathlineto{\pgfqpoint{2.112145in}{0.701568in}}%
\pgfpathlineto{\pgfqpoint{2.116655in}{0.701937in}}%
\pgfpathlineto{\pgfqpoint{2.121164in}{0.707878in}}%
\pgfpathlineto{\pgfqpoint{2.125673in}{0.705633in}}%
\pgfpathlineto{\pgfqpoint{2.130182in}{0.705125in}}%
\pgfpathlineto{\pgfqpoint{2.139200in}{0.713215in}}%
\pgfpathlineto{\pgfqpoint{2.143709in}{0.701600in}}%
\pgfpathlineto{\pgfqpoint{2.152727in}{0.700916in}}%
\pgfpathlineto{\pgfqpoint{2.157236in}{0.702903in}}%
\pgfpathlineto{\pgfqpoint{2.161745in}{0.701797in}}%
\pgfpathlineto{\pgfqpoint{2.166255in}{0.711874in}}%
\pgfpathlineto{\pgfqpoint{2.170764in}{0.703762in}}%
\pgfpathlineto{\pgfqpoint{2.175273in}{0.712153in}}%
\pgfpathlineto{\pgfqpoint{2.179782in}{0.709577in}}%
\pgfpathlineto{\pgfqpoint{2.184291in}{0.702026in}}%
\pgfpathlineto{\pgfqpoint{2.188800in}{0.716900in}}%
\pgfpathlineto{\pgfqpoint{2.193309in}{0.707438in}}%
\pgfpathlineto{\pgfqpoint{2.197818in}{0.712129in}}%
\pgfpathlineto{\pgfqpoint{2.202327in}{0.709690in}}%
\pgfpathlineto{\pgfqpoint{2.206836in}{0.703432in}}%
\pgfpathlineto{\pgfqpoint{2.211345in}{0.705888in}}%
\pgfpathlineto{\pgfqpoint{2.220364in}{0.704982in}}%
\pgfpathlineto{\pgfqpoint{2.224873in}{0.702779in}}%
\pgfpathlineto{\pgfqpoint{2.229382in}{0.707227in}}%
\pgfpathlineto{\pgfqpoint{2.233891in}{0.702481in}}%
\pgfpathlineto{\pgfqpoint{2.238400in}{0.701483in}}%
\pgfpathlineto{\pgfqpoint{2.260945in}{0.702379in}}%
\pgfpathlineto{\pgfqpoint{2.269964in}{0.705775in}}%
\pgfpathlineto{\pgfqpoint{2.274473in}{0.701517in}}%
\pgfpathlineto{\pgfqpoint{2.283491in}{0.702132in}}%
\pgfpathlineto{\pgfqpoint{2.292509in}{0.703093in}}%
\pgfpathlineto{\pgfqpoint{2.297018in}{0.701723in}}%
\pgfpathlineto{\pgfqpoint{2.301527in}{0.705420in}}%
\pgfpathlineto{\pgfqpoint{2.306036in}{0.701816in}}%
\pgfpathlineto{\pgfqpoint{2.315055in}{0.703050in}}%
\pgfpathlineto{\pgfqpoint{2.319564in}{0.705220in}}%
\pgfpathlineto{\pgfqpoint{2.324073in}{0.702110in}}%
\pgfpathlineto{\pgfqpoint{2.351127in}{0.701947in}}%
\pgfpathlineto{\pgfqpoint{2.355636in}{0.703587in}}%
\pgfpathlineto{\pgfqpoint{2.360145in}{0.702108in}}%
\pgfpathlineto{\pgfqpoint{2.364655in}{0.702619in}}%
\pgfpathlineto{\pgfqpoint{2.369164in}{0.705171in}}%
\pgfpathlineto{\pgfqpoint{2.373673in}{0.702164in}}%
\pgfpathlineto{\pgfqpoint{2.378182in}{0.704186in}}%
\pgfpathlineto{\pgfqpoint{2.382691in}{0.702637in}}%
\pgfpathlineto{\pgfqpoint{2.387200in}{0.711408in}}%
\pgfpathlineto{\pgfqpoint{2.391709in}{0.704159in}}%
\pgfpathlineto{\pgfqpoint{2.396218in}{0.702302in}}%
\pgfpathlineto{\pgfqpoint{2.409745in}{0.704176in}}%
\pgfpathlineto{\pgfqpoint{2.414255in}{0.702947in}}%
\pgfpathlineto{\pgfqpoint{2.432291in}{0.703095in}}%
\pgfpathlineto{\pgfqpoint{2.436800in}{0.705613in}}%
\pgfpathlineto{\pgfqpoint{2.441309in}{0.702582in}}%
\pgfpathlineto{\pgfqpoint{2.445818in}{0.704337in}}%
\pgfpathlineto{\pgfqpoint{2.450327in}{0.711928in}}%
\pgfpathlineto{\pgfqpoint{2.454836in}{0.702930in}}%
\pgfpathlineto{\pgfqpoint{2.459345in}{0.707451in}}%
\pgfpathlineto{\pgfqpoint{2.463855in}{0.702776in}}%
\pgfpathlineto{\pgfqpoint{2.468364in}{0.702887in}}%
\pgfpathlineto{\pgfqpoint{2.472873in}{0.705498in}}%
\pgfpathlineto{\pgfqpoint{2.477382in}{0.703342in}}%
\pgfpathlineto{\pgfqpoint{2.481891in}{0.702640in}}%
\pgfpathlineto{\pgfqpoint{2.486400in}{0.707158in}}%
\pgfpathlineto{\pgfqpoint{2.490909in}{0.707391in}}%
\pgfpathlineto{\pgfqpoint{2.495418in}{0.702832in}}%
\pgfpathlineto{\pgfqpoint{2.499927in}{0.703052in}}%
\pgfpathlineto{\pgfqpoint{2.504436in}{0.705593in}}%
\pgfpathlineto{\pgfqpoint{2.508945in}{0.702907in}}%
\pgfpathlineto{\pgfqpoint{2.513455in}{0.702826in}}%
\pgfpathlineto{\pgfqpoint{2.517964in}{0.704817in}}%
\pgfpathlineto{\pgfqpoint{2.522473in}{0.702856in}}%
\pgfpathlineto{\pgfqpoint{2.526982in}{0.703987in}}%
\pgfpathlineto{\pgfqpoint{2.531491in}{0.702788in}}%
\pgfpathlineto{\pgfqpoint{2.540509in}{0.703647in}}%
\pgfpathlineto{\pgfqpoint{2.554036in}{0.702997in}}%
\pgfpathlineto{\pgfqpoint{2.558545in}{0.708808in}}%
\pgfpathlineto{\pgfqpoint{2.563055in}{0.703147in}}%
\pgfpathlineto{\pgfqpoint{2.567564in}{0.703832in}}%
\pgfpathlineto{\pgfqpoint{2.572073in}{0.703004in}}%
\pgfpathlineto{\pgfqpoint{2.599127in}{0.707201in}}%
\pgfpathlineto{\pgfqpoint{2.603636in}{0.704002in}}%
\pgfpathlineto{\pgfqpoint{2.608145in}{0.707271in}}%
\pgfpathlineto{\pgfqpoint{2.612655in}{0.708773in}}%
\pgfpathlineto{\pgfqpoint{2.617164in}{0.703430in}}%
\pgfpathlineto{\pgfqpoint{2.626182in}{0.707024in}}%
\pgfpathlineto{\pgfqpoint{2.630691in}{0.704311in}}%
\pgfpathlineto{\pgfqpoint{2.635200in}{0.703669in}}%
\pgfpathlineto{\pgfqpoint{2.639709in}{0.704746in}}%
\pgfpathlineto{\pgfqpoint{2.644218in}{0.703312in}}%
\pgfpathlineto{\pgfqpoint{2.648727in}{0.704898in}}%
\pgfpathlineto{\pgfqpoint{2.653236in}{0.708381in}}%
\pgfpathlineto{\pgfqpoint{2.657745in}{0.703757in}}%
\pgfpathlineto{\pgfqpoint{2.680291in}{0.705072in}}%
\pgfpathlineto{\pgfqpoint{2.684800in}{0.707307in}}%
\pgfpathlineto{\pgfqpoint{2.689309in}{0.705319in}}%
\pgfpathlineto{\pgfqpoint{2.707345in}{0.704093in}}%
\pgfpathlineto{\pgfqpoint{2.711855in}{0.709345in}}%
\pgfpathlineto{\pgfqpoint{2.716364in}{0.707708in}}%
\pgfpathlineto{\pgfqpoint{2.720873in}{0.708093in}}%
\pgfpathlineto{\pgfqpoint{2.725382in}{0.705090in}}%
\pgfpathlineto{\pgfqpoint{2.729891in}{0.704276in}}%
\pgfpathlineto{\pgfqpoint{2.734400in}{0.712878in}}%
\pgfpathlineto{\pgfqpoint{2.738909in}{0.704765in}}%
\pgfpathlineto{\pgfqpoint{2.743418in}{0.703839in}}%
\pgfpathlineto{\pgfqpoint{2.747927in}{0.707438in}}%
\pgfpathlineto{\pgfqpoint{2.752436in}{0.704599in}}%
\pgfpathlineto{\pgfqpoint{2.756945in}{0.713933in}}%
\pgfpathlineto{\pgfqpoint{2.761455in}{0.704314in}}%
\pgfpathlineto{\pgfqpoint{2.770473in}{0.704583in}}%
\pgfpathlineto{\pgfqpoint{2.774982in}{0.705278in}}%
\pgfpathlineto{\pgfqpoint{2.779491in}{0.704073in}}%
\pgfpathlineto{\pgfqpoint{2.793018in}{0.704239in}}%
\pgfpathlineto{\pgfqpoint{2.797527in}{0.713740in}}%
\pgfpathlineto{\pgfqpoint{2.802036in}{0.704600in}}%
\pgfpathlineto{\pgfqpoint{2.806545in}{0.704354in}}%
\pgfpathlineto{\pgfqpoint{2.811055in}{0.706570in}}%
\pgfpathlineto{\pgfqpoint{2.815564in}{0.704704in}}%
\pgfpathlineto{\pgfqpoint{2.820073in}{0.704436in}}%
\pgfpathlineto{\pgfqpoint{2.824582in}{0.707051in}}%
\pgfpathlineto{\pgfqpoint{2.829091in}{0.705444in}}%
\pgfpathlineto{\pgfqpoint{2.838109in}{0.705684in}}%
\pgfpathlineto{\pgfqpoint{2.842618in}{0.707085in}}%
\pgfpathlineto{\pgfqpoint{2.847127in}{0.704663in}}%
\pgfpathlineto{\pgfqpoint{2.851636in}{0.709382in}}%
\pgfpathlineto{\pgfqpoint{2.856145in}{0.704491in}}%
\pgfpathlineto{\pgfqpoint{2.860655in}{0.704766in}}%
\pgfpathlineto{\pgfqpoint{2.865164in}{0.706507in}}%
\pgfpathlineto{\pgfqpoint{2.878691in}{0.704576in}}%
\pgfpathlineto{\pgfqpoint{2.892218in}{0.705850in}}%
\pgfpathlineto{\pgfqpoint{2.910255in}{0.705677in}}%
\pgfpathlineto{\pgfqpoint{2.914764in}{0.707745in}}%
\pgfpathlineto{\pgfqpoint{2.919273in}{0.705583in}}%
\pgfpathlineto{\pgfqpoint{2.923782in}{0.705366in}}%
\pgfpathlineto{\pgfqpoint{2.928291in}{0.706518in}}%
\pgfpathlineto{\pgfqpoint{2.932800in}{0.705420in}}%
\pgfpathlineto{\pgfqpoint{2.937309in}{0.707444in}}%
\pgfpathlineto{\pgfqpoint{2.941818in}{0.705006in}}%
\pgfpathlineto{\pgfqpoint{2.946327in}{0.705058in}}%
\pgfpathlineto{\pgfqpoint{2.950836in}{0.710745in}}%
\pgfpathlineto{\pgfqpoint{2.955345in}{0.705712in}}%
\pgfpathlineto{\pgfqpoint{2.959855in}{0.705518in}}%
\pgfpathlineto{\pgfqpoint{2.964364in}{0.707837in}}%
\pgfpathlineto{\pgfqpoint{2.968873in}{0.704996in}}%
\pgfpathlineto{\pgfqpoint{2.973382in}{0.706362in}}%
\pgfpathlineto{\pgfqpoint{2.986909in}{0.705729in}}%
\pgfpathlineto{\pgfqpoint{2.991418in}{0.713252in}}%
\pgfpathlineto{\pgfqpoint{2.995927in}{0.706403in}}%
\pgfpathlineto{\pgfqpoint{3.000436in}{0.706427in}}%
\pgfpathlineto{\pgfqpoint{3.004945in}{0.710077in}}%
\pgfpathlineto{\pgfqpoint{3.009455in}{0.706201in}}%
\pgfpathlineto{\pgfqpoint{3.018473in}{0.705162in}}%
\pgfpathlineto{\pgfqpoint{3.036509in}{0.706944in}}%
\pgfpathlineto{\pgfqpoint{3.041018in}{0.705836in}}%
\pgfpathlineto{\pgfqpoint{3.045527in}{0.706271in}}%
\pgfpathlineto{\pgfqpoint{3.050036in}{0.709064in}}%
\pgfpathlineto{\pgfqpoint{3.054545in}{0.706265in}}%
\pgfpathlineto{\pgfqpoint{3.059055in}{0.705983in}}%
\pgfpathlineto{\pgfqpoint{3.063564in}{0.708814in}}%
\pgfpathlineto{\pgfqpoint{3.068073in}{0.706564in}}%
\pgfpathlineto{\pgfqpoint{3.072582in}{0.705731in}}%
\pgfpathlineto{\pgfqpoint{3.077091in}{0.710127in}}%
\pgfpathlineto{\pgfqpoint{3.081600in}{0.706370in}}%
\pgfpathlineto{\pgfqpoint{3.126691in}{0.707383in}}%
\pgfpathlineto{\pgfqpoint{3.131200in}{0.713372in}}%
\pgfpathlineto{\pgfqpoint{3.135709in}{0.708769in}}%
\pgfpathlineto{\pgfqpoint{3.140218in}{0.706586in}}%
\pgfpathlineto{\pgfqpoint{3.149236in}{0.706646in}}%
\pgfpathlineto{\pgfqpoint{3.153745in}{0.707446in}}%
\pgfpathlineto{\pgfqpoint{3.158255in}{0.712973in}}%
\pgfpathlineto{\pgfqpoint{3.162764in}{0.706087in}}%
\pgfpathlineto{\pgfqpoint{3.171782in}{0.707706in}}%
\pgfpathlineto{\pgfqpoint{3.176291in}{0.710903in}}%
\pgfpathlineto{\pgfqpoint{3.180800in}{0.706934in}}%
\pgfpathlineto{\pgfqpoint{3.185309in}{0.706254in}}%
\pgfpathlineto{\pgfqpoint{3.189818in}{0.707025in}}%
\pgfpathlineto{\pgfqpoint{3.194327in}{0.706432in}}%
\pgfpathlineto{\pgfqpoint{3.198836in}{0.714277in}}%
\pgfpathlineto{\pgfqpoint{3.203345in}{0.707372in}}%
\pgfpathlineto{\pgfqpoint{3.207855in}{0.708408in}}%
\pgfpathlineto{\pgfqpoint{3.212364in}{0.706727in}}%
\pgfpathlineto{\pgfqpoint{3.221382in}{0.708536in}}%
\pgfpathlineto{\pgfqpoint{3.225891in}{0.706833in}}%
\pgfpathlineto{\pgfqpoint{3.230400in}{0.707976in}}%
\pgfpathlineto{\pgfqpoint{3.234909in}{0.706730in}}%
\pgfpathlineto{\pgfqpoint{3.248436in}{0.708420in}}%
\pgfpathlineto{\pgfqpoint{3.252945in}{0.710504in}}%
\pgfpathlineto{\pgfqpoint{3.261964in}{0.706648in}}%
\pgfpathlineto{\pgfqpoint{3.266473in}{0.710738in}}%
\pgfpathlineto{\pgfqpoint{3.270982in}{0.707238in}}%
\pgfpathlineto{\pgfqpoint{3.275491in}{0.707001in}}%
\pgfpathlineto{\pgfqpoint{3.280000in}{0.708405in}}%
\pgfpathlineto{\pgfqpoint{3.284509in}{0.711037in}}%
\pgfpathlineto{\pgfqpoint{3.289018in}{0.707538in}}%
\pgfpathlineto{\pgfqpoint{3.293527in}{0.713101in}}%
\pgfpathlineto{\pgfqpoint{3.298036in}{0.707981in}}%
\pgfpathlineto{\pgfqpoint{3.302545in}{0.711893in}}%
\pgfpathlineto{\pgfqpoint{3.307055in}{0.707975in}}%
\pgfpathlineto{\pgfqpoint{3.311564in}{0.709183in}}%
\pgfpathlineto{\pgfqpoint{3.316073in}{0.715352in}}%
\pgfpathlineto{\pgfqpoint{3.320582in}{0.706962in}}%
\pgfpathlineto{\pgfqpoint{3.325091in}{0.708817in}}%
\pgfpathlineto{\pgfqpoint{3.329600in}{0.714238in}}%
\pgfpathlineto{\pgfqpoint{3.334109in}{0.707669in}}%
\pgfpathlineto{\pgfqpoint{3.338618in}{0.711938in}}%
\pgfpathlineto{\pgfqpoint{3.343127in}{0.707744in}}%
\pgfpathlineto{\pgfqpoint{3.347636in}{0.707982in}}%
\pgfpathlineto{\pgfqpoint{3.352145in}{0.711589in}}%
\pgfpathlineto{\pgfqpoint{3.356655in}{0.708166in}}%
\pgfpathlineto{\pgfqpoint{3.361164in}{0.711183in}}%
\pgfpathlineto{\pgfqpoint{3.365673in}{0.709411in}}%
\pgfpathlineto{\pgfqpoint{3.370182in}{0.710434in}}%
\pgfpathlineto{\pgfqpoint{3.379200in}{0.707769in}}%
\pgfpathlineto{\pgfqpoint{3.388218in}{0.707371in}}%
\pgfpathlineto{\pgfqpoint{3.392727in}{0.709655in}}%
\pgfpathlineto{\pgfqpoint{3.397236in}{0.708524in}}%
\pgfpathlineto{\pgfqpoint{3.401745in}{0.710990in}}%
\pgfpathlineto{\pgfqpoint{3.406255in}{0.707450in}}%
\pgfpathlineto{\pgfqpoint{3.410764in}{0.709843in}}%
\pgfpathlineto{\pgfqpoint{3.419782in}{0.707966in}}%
\pgfpathlineto{\pgfqpoint{3.428800in}{0.708282in}}%
\pgfpathlineto{\pgfqpoint{3.433309in}{0.708412in}}%
\pgfpathlineto{\pgfqpoint{3.437818in}{0.719591in}}%
\pgfpathlineto{\pgfqpoint{3.442327in}{0.711367in}}%
\pgfpathlineto{\pgfqpoint{3.446836in}{0.712465in}}%
\pgfpathlineto{\pgfqpoint{3.451345in}{0.708129in}}%
\pgfpathlineto{\pgfqpoint{3.464873in}{0.709707in}}%
\pgfpathlineto{\pgfqpoint{3.473891in}{0.719970in}}%
\pgfpathlineto{\pgfqpoint{3.478400in}{0.708463in}}%
\pgfpathlineto{\pgfqpoint{3.482909in}{0.712773in}}%
\pgfpathlineto{\pgfqpoint{3.487418in}{0.712281in}}%
\pgfpathlineto{\pgfqpoint{3.496436in}{0.708728in}}%
\pgfpathlineto{\pgfqpoint{3.500945in}{0.711542in}}%
\pgfpathlineto{\pgfqpoint{3.505455in}{0.712275in}}%
\pgfpathlineto{\pgfqpoint{3.509964in}{0.708099in}}%
\pgfpathlineto{\pgfqpoint{3.514473in}{0.714069in}}%
\pgfpathlineto{\pgfqpoint{3.518982in}{0.708143in}}%
\pgfpathlineto{\pgfqpoint{3.528000in}{0.709508in}}%
\pgfpathlineto{\pgfqpoint{3.541527in}{0.708479in}}%
\pgfpathlineto{\pgfqpoint{3.546036in}{0.711276in}}%
\pgfpathlineto{\pgfqpoint{3.550545in}{0.719567in}}%
\pgfpathlineto{\pgfqpoint{3.555055in}{0.708238in}}%
\pgfpathlineto{\pgfqpoint{3.559564in}{0.711432in}}%
\pgfpathlineto{\pgfqpoint{3.564073in}{0.709907in}}%
\pgfpathlineto{\pgfqpoint{3.582109in}{0.708893in}}%
\pgfpathlineto{\pgfqpoint{3.586618in}{0.710350in}}%
\pgfpathlineto{\pgfqpoint{3.591127in}{0.708762in}}%
\pgfpathlineto{\pgfqpoint{3.595636in}{0.709726in}}%
\pgfpathlineto{\pgfqpoint{3.609164in}{0.709031in}}%
\pgfpathlineto{\pgfqpoint{3.622691in}{0.709650in}}%
\pgfpathlineto{\pgfqpoint{3.627200in}{0.708539in}}%
\pgfpathlineto{\pgfqpoint{3.631709in}{0.711781in}}%
\pgfpathlineto{\pgfqpoint{3.636218in}{0.709445in}}%
\pgfpathlineto{\pgfqpoint{3.640727in}{0.709289in}}%
\pgfpathlineto{\pgfqpoint{3.649745in}{0.714283in}}%
\pgfpathlineto{\pgfqpoint{3.654255in}{0.708761in}}%
\pgfpathlineto{\pgfqpoint{3.658764in}{0.712735in}}%
\pgfpathlineto{\pgfqpoint{3.663273in}{0.710156in}}%
\pgfpathlineto{\pgfqpoint{3.667782in}{0.709147in}}%
\pgfpathlineto{\pgfqpoint{3.672291in}{0.710012in}}%
\pgfpathlineto{\pgfqpoint{3.676800in}{0.709253in}}%
\pgfpathlineto{\pgfqpoint{3.681309in}{0.713082in}}%
\pgfpathlineto{\pgfqpoint{3.685818in}{0.714496in}}%
\pgfpathlineto{\pgfqpoint{3.690327in}{0.710030in}}%
\pgfpathlineto{\pgfqpoint{3.699345in}{0.711117in}}%
\pgfpathlineto{\pgfqpoint{3.703855in}{0.718270in}}%
\pgfpathlineto{\pgfqpoint{3.708364in}{0.709728in}}%
\pgfpathlineto{\pgfqpoint{3.717382in}{0.710810in}}%
\pgfpathlineto{\pgfqpoint{3.721891in}{0.709673in}}%
\pgfpathlineto{\pgfqpoint{3.739927in}{0.711090in}}%
\pgfpathlineto{\pgfqpoint{3.744436in}{0.709201in}}%
\pgfpathlineto{\pgfqpoint{3.748945in}{0.713173in}}%
\pgfpathlineto{\pgfqpoint{3.753455in}{0.710174in}}%
\pgfpathlineto{\pgfqpoint{3.757964in}{0.713190in}}%
\pgfpathlineto{\pgfqpoint{3.762473in}{0.713169in}}%
\pgfpathlineto{\pgfqpoint{3.766982in}{0.710770in}}%
\pgfpathlineto{\pgfqpoint{3.780509in}{0.709852in}}%
\pgfpathlineto{\pgfqpoint{3.785018in}{0.712386in}}%
\pgfpathlineto{\pgfqpoint{3.789527in}{0.739062in}}%
\pgfpathlineto{\pgfqpoint{3.794036in}{0.716641in}}%
\pgfpathlineto{\pgfqpoint{3.798545in}{0.728056in}}%
\pgfpathlineto{\pgfqpoint{3.803055in}{0.725399in}}%
\pgfpathlineto{\pgfqpoint{3.807564in}{0.711471in}}%
\pgfpathlineto{\pgfqpoint{3.812073in}{0.724594in}}%
\pgfpathlineto{\pgfqpoint{3.821091in}{0.711740in}}%
\pgfpathlineto{\pgfqpoint{3.825600in}{0.710041in}}%
\pgfpathlineto{\pgfqpoint{3.830109in}{0.715946in}}%
\pgfpathlineto{\pgfqpoint{3.834618in}{0.710446in}}%
\pgfpathlineto{\pgfqpoint{3.839127in}{0.709980in}}%
\pgfpathlineto{\pgfqpoint{3.843636in}{0.716351in}}%
\pgfpathlineto{\pgfqpoint{3.848145in}{0.711853in}}%
\pgfpathlineto{\pgfqpoint{3.852655in}{0.713404in}}%
\pgfpathlineto{\pgfqpoint{3.857164in}{0.710010in}}%
\pgfpathlineto{\pgfqpoint{3.866182in}{0.713955in}}%
\pgfpathlineto{\pgfqpoint{3.870691in}{0.710508in}}%
\pgfpathlineto{\pgfqpoint{3.879709in}{0.710800in}}%
\pgfpathlineto{\pgfqpoint{3.884218in}{0.722764in}}%
\pgfpathlineto{\pgfqpoint{3.888727in}{0.710654in}}%
\pgfpathlineto{\pgfqpoint{3.893236in}{0.710338in}}%
\pgfpathlineto{\pgfqpoint{3.897745in}{0.717251in}}%
\pgfpathlineto{\pgfqpoint{3.902255in}{0.713710in}}%
\pgfpathlineto{\pgfqpoint{3.906764in}{0.721462in}}%
\pgfpathlineto{\pgfqpoint{3.911273in}{0.713460in}}%
\pgfpathlineto{\pgfqpoint{3.920291in}{0.717523in}}%
\pgfpathlineto{\pgfqpoint{3.924800in}{0.710046in}}%
\pgfpathlineto{\pgfqpoint{3.933818in}{0.711691in}}%
\pgfpathlineto{\pgfqpoint{3.938327in}{0.710631in}}%
\pgfpathlineto{\pgfqpoint{3.942836in}{0.714554in}}%
\pgfpathlineto{\pgfqpoint{3.947345in}{0.711588in}}%
\pgfpathlineto{\pgfqpoint{3.951855in}{0.711595in}}%
\pgfpathlineto{\pgfqpoint{3.956364in}{0.712809in}}%
\pgfpathlineto{\pgfqpoint{3.960873in}{0.711064in}}%
\pgfpathlineto{\pgfqpoint{3.974400in}{0.711464in}}%
\pgfpathlineto{\pgfqpoint{3.978909in}{0.710687in}}%
\pgfpathlineto{\pgfqpoint{3.983418in}{0.716282in}}%
\pgfpathlineto{\pgfqpoint{3.987927in}{0.714814in}}%
\pgfpathlineto{\pgfqpoint{3.992436in}{0.710641in}}%
\pgfpathlineto{\pgfqpoint{3.996945in}{0.712431in}}%
\pgfpathlineto{\pgfqpoint{4.001455in}{0.711388in}}%
\pgfpathlineto{\pgfqpoint{4.010473in}{0.714905in}}%
\pgfpathlineto{\pgfqpoint{4.014982in}{0.711341in}}%
\pgfpathlineto{\pgfqpoint{4.024000in}{0.715797in}}%
\pgfpathlineto{\pgfqpoint{4.033018in}{0.710947in}}%
\pgfpathlineto{\pgfqpoint{4.042036in}{0.712815in}}%
\pgfpathlineto{\pgfqpoint{4.046545in}{0.711557in}}%
\pgfpathlineto{\pgfqpoint{4.051055in}{0.712680in}}%
\pgfpathlineto{\pgfqpoint{4.060073in}{0.711651in}}%
\pgfpathlineto{\pgfqpoint{4.064582in}{0.712781in}}%
\pgfpathlineto{\pgfqpoint{4.069091in}{0.711907in}}%
\pgfpathlineto{\pgfqpoint{4.073600in}{0.713823in}}%
\pgfpathlineto{\pgfqpoint{4.082618in}{0.712047in}}%
\pgfpathlineto{\pgfqpoint{4.087127in}{0.711719in}}%
\pgfpathlineto{\pgfqpoint{4.091636in}{0.715233in}}%
\pgfpathlineto{\pgfqpoint{4.096145in}{0.715009in}}%
\pgfpathlineto{\pgfqpoint{4.100655in}{0.712265in}}%
\pgfpathlineto{\pgfqpoint{4.105164in}{0.711711in}}%
\pgfpathlineto{\pgfqpoint{4.109673in}{0.712986in}}%
\pgfpathlineto{\pgfqpoint{4.114182in}{0.717581in}}%
\pgfpathlineto{\pgfqpoint{4.118691in}{0.713907in}}%
\pgfpathlineto{\pgfqpoint{4.123200in}{0.717448in}}%
\pgfpathlineto{\pgfqpoint{4.132218in}{0.712100in}}%
\pgfpathlineto{\pgfqpoint{4.136727in}{0.712111in}}%
\pgfpathlineto{\pgfqpoint{4.141236in}{0.715593in}}%
\pgfpathlineto{\pgfqpoint{4.145745in}{0.713786in}}%
\pgfpathlineto{\pgfqpoint{4.150255in}{0.714409in}}%
\pgfpathlineto{\pgfqpoint{4.154764in}{0.711520in}}%
\pgfpathlineto{\pgfqpoint{4.159273in}{0.713767in}}%
\pgfpathlineto{\pgfqpoint{4.163782in}{0.713046in}}%
\pgfpathlineto{\pgfqpoint{4.168291in}{0.719527in}}%
\pgfpathlineto{\pgfqpoint{4.172800in}{0.711538in}}%
\pgfpathlineto{\pgfqpoint{4.177309in}{0.711937in}}%
\pgfpathlineto{\pgfqpoint{4.181818in}{0.721559in}}%
\pgfpathlineto{\pgfqpoint{4.186327in}{0.712736in}}%
\pgfpathlineto{\pgfqpoint{4.190836in}{0.714897in}}%
\pgfpathlineto{\pgfqpoint{4.195345in}{0.714436in}}%
\pgfpathlineto{\pgfqpoint{4.199855in}{0.715208in}}%
\pgfpathlineto{\pgfqpoint{4.204364in}{0.712571in}}%
\pgfpathlineto{\pgfqpoint{4.208873in}{0.712404in}}%
\pgfpathlineto{\pgfqpoint{4.217891in}{0.714069in}}%
\pgfpathlineto{\pgfqpoint{4.222400in}{0.713402in}}%
\pgfpathlineto{\pgfqpoint{4.226909in}{0.715448in}}%
\pgfpathlineto{\pgfqpoint{4.231418in}{0.713261in}}%
\pgfpathlineto{\pgfqpoint{4.235927in}{0.712754in}}%
\pgfpathlineto{\pgfqpoint{4.240436in}{0.714593in}}%
\pgfpathlineto{\pgfqpoint{4.244945in}{0.712973in}}%
\pgfpathlineto{\pgfqpoint{4.249455in}{0.714764in}}%
\pgfpathlineto{\pgfqpoint{4.258473in}{0.712132in}}%
\pgfpathlineto{\pgfqpoint{4.262982in}{0.713122in}}%
\pgfpathlineto{\pgfqpoint{4.267491in}{0.716950in}}%
\pgfpathlineto{\pgfqpoint{4.272000in}{0.711982in}}%
\pgfpathlineto{\pgfqpoint{4.281018in}{0.713601in}}%
\pgfpathlineto{\pgfqpoint{4.285527in}{0.712395in}}%
\pgfpathlineto{\pgfqpoint{4.290036in}{0.714640in}}%
\pgfpathlineto{\pgfqpoint{4.294545in}{0.714664in}}%
\pgfpathlineto{\pgfqpoint{4.299055in}{0.712206in}}%
\pgfpathlineto{\pgfqpoint{4.303564in}{0.717704in}}%
\pgfpathlineto{\pgfqpoint{4.308073in}{0.725939in}}%
\pgfpathlineto{\pgfqpoint{4.312582in}{0.715007in}}%
\pgfpathlineto{\pgfqpoint{4.317091in}{0.715947in}}%
\pgfpathlineto{\pgfqpoint{4.321600in}{0.713076in}}%
\pgfpathlineto{\pgfqpoint{4.335127in}{0.714024in}}%
\pgfpathlineto{\pgfqpoint{4.339636in}{0.716685in}}%
\pgfpathlineto{\pgfqpoint{4.344145in}{0.713353in}}%
\pgfpathlineto{\pgfqpoint{4.348655in}{0.716342in}}%
\pgfpathlineto{\pgfqpoint{4.353164in}{0.713339in}}%
\pgfpathlineto{\pgfqpoint{4.357673in}{0.717096in}}%
\pgfpathlineto{\pgfqpoint{4.366691in}{0.717401in}}%
\pgfpathlineto{\pgfqpoint{4.371200in}{0.713080in}}%
\pgfpathlineto{\pgfqpoint{4.375709in}{0.713005in}}%
\pgfpathlineto{\pgfqpoint{4.380218in}{0.714493in}}%
\pgfpathlineto{\pgfqpoint{4.384727in}{0.714436in}}%
\pgfpathlineto{\pgfqpoint{4.389236in}{0.713183in}}%
\pgfpathlineto{\pgfqpoint{4.393745in}{0.714509in}}%
\pgfpathlineto{\pgfqpoint{4.398255in}{0.713675in}}%
\pgfpathlineto{\pgfqpoint{4.402764in}{0.719359in}}%
\pgfpathlineto{\pgfqpoint{4.411782in}{0.713600in}}%
\pgfpathlineto{\pgfqpoint{4.420800in}{0.716213in}}%
\pgfpathlineto{\pgfqpoint{4.425309in}{0.715566in}}%
\pgfpathlineto{\pgfqpoint{4.429818in}{0.717784in}}%
\pgfpathlineto{\pgfqpoint{4.434327in}{0.715888in}}%
\pgfpathlineto{\pgfqpoint{4.438836in}{0.718285in}}%
\pgfpathlineto{\pgfqpoint{4.443345in}{0.713717in}}%
\pgfpathlineto{\pgfqpoint{4.447855in}{0.721705in}}%
\pgfpathlineto{\pgfqpoint{4.452364in}{0.713792in}}%
\pgfpathlineto{\pgfqpoint{4.456873in}{0.713520in}}%
\pgfpathlineto{\pgfqpoint{4.461382in}{0.716014in}}%
\pgfpathlineto{\pgfqpoint{4.465891in}{0.720849in}}%
\pgfpathlineto{\pgfqpoint{4.470400in}{0.713785in}}%
\pgfpathlineto{\pgfqpoint{4.474909in}{0.715576in}}%
\pgfpathlineto{\pgfqpoint{4.479418in}{0.722428in}}%
\pgfpathlineto{\pgfqpoint{4.483927in}{0.714059in}}%
\pgfpathlineto{\pgfqpoint{4.488436in}{0.714174in}}%
\pgfpathlineto{\pgfqpoint{4.492945in}{0.718080in}}%
\pgfpathlineto{\pgfqpoint{4.497455in}{0.714984in}}%
\pgfpathlineto{\pgfqpoint{4.501964in}{0.717577in}}%
\pgfpathlineto{\pgfqpoint{4.515491in}{0.717558in}}%
\pgfpathlineto{\pgfqpoint{4.520000in}{0.716671in}}%
\pgfpathlineto{\pgfqpoint{4.524509in}{0.721003in}}%
\pgfpathlineto{\pgfqpoint{4.529018in}{0.713702in}}%
\pgfpathlineto{\pgfqpoint{4.533527in}{0.717200in}}%
\pgfpathlineto{\pgfqpoint{4.538036in}{0.714502in}}%
\pgfpathlineto{\pgfqpoint{4.542545in}{0.718531in}}%
\pgfpathlineto{\pgfqpoint{4.547055in}{0.715213in}}%
\pgfpathlineto{\pgfqpoint{4.551564in}{0.718691in}}%
\pgfpathlineto{\pgfqpoint{4.556073in}{0.719348in}}%
\pgfpathlineto{\pgfqpoint{4.560582in}{0.718718in}}%
\pgfpathlineto{\pgfqpoint{4.565091in}{0.714941in}}%
\pgfpathlineto{\pgfqpoint{4.569600in}{0.714556in}}%
\pgfpathlineto{\pgfqpoint{4.574109in}{0.716075in}}%
\pgfpathlineto{\pgfqpoint{4.578618in}{0.713898in}}%
\pgfpathlineto{\pgfqpoint{4.583127in}{0.716630in}}%
\pgfpathlineto{\pgfqpoint{4.587636in}{0.717837in}}%
\pgfpathlineto{\pgfqpoint{4.592145in}{0.715456in}}%
\pgfpathlineto{\pgfqpoint{4.596655in}{0.715410in}}%
\pgfpathlineto{\pgfqpoint{4.601164in}{0.725942in}}%
\pgfpathlineto{\pgfqpoint{4.605673in}{0.715034in}}%
\pgfpathlineto{\pgfqpoint{4.610182in}{0.718027in}}%
\pgfpathlineto{\pgfqpoint{4.614691in}{0.713923in}}%
\pgfpathlineto{\pgfqpoint{4.619200in}{0.716335in}}%
\pgfpathlineto{\pgfqpoint{4.623709in}{0.722582in}}%
\pgfpathlineto{\pgfqpoint{4.628218in}{0.743852in}}%
\pgfpathlineto{\pgfqpoint{4.632727in}{0.714476in}}%
\pgfpathlineto{\pgfqpoint{4.637236in}{0.716641in}}%
\pgfpathlineto{\pgfqpoint{4.641745in}{0.715617in}}%
\pgfpathlineto{\pgfqpoint{4.646255in}{0.728717in}}%
\pgfpathlineto{\pgfqpoint{4.650764in}{0.719640in}}%
\pgfpathlineto{\pgfqpoint{4.655273in}{0.716520in}}%
\pgfpathlineto{\pgfqpoint{4.659782in}{0.718607in}}%
\pgfpathlineto{\pgfqpoint{4.664291in}{0.725838in}}%
\pgfpathlineto{\pgfqpoint{4.668800in}{0.744455in}}%
\pgfpathlineto{\pgfqpoint{4.673309in}{0.724512in}}%
\pgfpathlineto{\pgfqpoint{4.677818in}{0.716690in}}%
\pgfpathlineto{\pgfqpoint{4.682327in}{0.722453in}}%
\pgfpathlineto{\pgfqpoint{4.686836in}{0.715766in}}%
\pgfpathlineto{\pgfqpoint{4.695855in}{0.717328in}}%
\pgfpathlineto{\pgfqpoint{4.700364in}{0.726070in}}%
\pgfpathlineto{\pgfqpoint{4.704873in}{0.716526in}}%
\pgfpathlineto{\pgfqpoint{4.709382in}{0.755708in}}%
\pgfpathlineto{\pgfqpoint{4.713891in}{0.716014in}}%
\pgfpathlineto{\pgfqpoint{4.718400in}{0.728777in}}%
\pgfpathlineto{\pgfqpoint{4.722909in}{0.724789in}}%
\pgfpathlineto{\pgfqpoint{4.727418in}{0.716002in}}%
\pgfpathlineto{\pgfqpoint{4.731927in}{0.720237in}}%
\pgfpathlineto{\pgfqpoint{4.736436in}{0.717427in}}%
\pgfpathlineto{\pgfqpoint{4.740945in}{0.715975in}}%
\pgfpathlineto{\pgfqpoint{4.745455in}{0.735012in}}%
\pgfpathlineto{\pgfqpoint{4.749964in}{0.725292in}}%
\pgfpathlineto{\pgfqpoint{4.754473in}{0.727905in}}%
\pgfpathlineto{\pgfqpoint{4.758982in}{0.714732in}}%
\pgfpathlineto{\pgfqpoint{4.768000in}{0.714718in}}%
\pgfpathlineto{\pgfqpoint{4.799564in}{0.715063in}}%
\pgfpathlineto{\pgfqpoint{4.804073in}{0.741234in}}%
\pgfpathlineto{\pgfqpoint{4.808582in}{0.715092in}}%
\pgfpathlineto{\pgfqpoint{4.822109in}{0.715557in}}%
\pgfpathlineto{\pgfqpoint{4.826618in}{0.736442in}}%
\pgfpathlineto{\pgfqpoint{4.831127in}{0.715862in}}%
\pgfpathlineto{\pgfqpoint{4.835636in}{0.718158in}}%
\pgfpathlineto{\pgfqpoint{4.840145in}{0.718749in}}%
\pgfpathlineto{\pgfqpoint{4.844655in}{0.717136in}}%
\pgfpathlineto{\pgfqpoint{4.849164in}{0.717891in}}%
\pgfpathlineto{\pgfqpoint{4.853673in}{0.722534in}}%
\pgfpathlineto{\pgfqpoint{4.858182in}{0.717919in}}%
\pgfpathlineto{\pgfqpoint{4.862691in}{0.720537in}}%
\pgfpathlineto{\pgfqpoint{4.867200in}{0.752759in}}%
\pgfpathlineto{\pgfqpoint{4.876218in}{0.715544in}}%
\pgfpathlineto{\pgfqpoint{4.880727in}{0.728312in}}%
\pgfpathlineto{\pgfqpoint{4.885236in}{0.716208in}}%
\pgfpathlineto{\pgfqpoint{4.889745in}{0.719566in}}%
\pgfpathlineto{\pgfqpoint{4.898764in}{0.716008in}}%
\pgfpathlineto{\pgfqpoint{4.907782in}{0.723219in}}%
\pgfpathlineto{\pgfqpoint{4.912291in}{0.716051in}}%
\pgfpathlineto{\pgfqpoint{4.916800in}{0.716402in}}%
\pgfpathlineto{\pgfqpoint{4.921309in}{0.730514in}}%
\pgfpathlineto{\pgfqpoint{4.925818in}{0.719663in}}%
\pgfpathlineto{\pgfqpoint{4.930327in}{0.717129in}}%
\pgfpathlineto{\pgfqpoint{4.934836in}{0.716896in}}%
\pgfpathlineto{\pgfqpoint{4.939345in}{0.725115in}}%
\pgfpathlineto{\pgfqpoint{4.943855in}{0.717169in}}%
\pgfpathlineto{\pgfqpoint{4.948364in}{0.724209in}}%
\pgfpathlineto{\pgfqpoint{4.957382in}{0.725271in}}%
\pgfpathlineto{\pgfqpoint{4.961891in}{0.715979in}}%
\pgfpathlineto{\pgfqpoint{4.970909in}{0.716788in}}%
\pgfpathlineto{\pgfqpoint{4.988945in}{0.717649in}}%
\pgfpathlineto{\pgfqpoint{4.993455in}{0.716287in}}%
\pgfpathlineto{\pgfqpoint{4.997964in}{0.718926in}}%
\pgfpathlineto{\pgfqpoint{5.002473in}{0.719493in}}%
\pgfpathlineto{\pgfqpoint{5.006982in}{0.722725in}}%
\pgfpathlineto{\pgfqpoint{5.011491in}{0.721260in}}%
\pgfpathlineto{\pgfqpoint{5.016000in}{0.716613in}}%
\pgfpathlineto{\pgfqpoint{5.025018in}{0.716432in}}%
\pgfpathlineto{\pgfqpoint{5.029527in}{0.721338in}}%
\pgfpathlineto{\pgfqpoint{5.034036in}{0.716618in}}%
\pgfpathlineto{\pgfqpoint{5.047564in}{0.718745in}}%
\pgfpathlineto{\pgfqpoint{5.052073in}{0.718422in}}%
\pgfpathlineto{\pgfqpoint{5.056582in}{0.721445in}}%
\pgfpathlineto{\pgfqpoint{5.061091in}{0.717341in}}%
\pgfpathlineto{\pgfqpoint{5.065600in}{0.717743in}}%
\pgfpathlineto{\pgfqpoint{5.070109in}{0.723224in}}%
\pgfpathlineto{\pgfqpoint{5.074618in}{0.717570in}}%
\pgfpathlineto{\pgfqpoint{5.083636in}{0.717108in}}%
\pgfpathlineto{\pgfqpoint{5.088145in}{0.722066in}}%
\pgfpathlineto{\pgfqpoint{5.092655in}{0.718520in}}%
\pgfpathlineto{\pgfqpoint{5.097164in}{0.733331in}}%
\pgfpathlineto{\pgfqpoint{5.101673in}{0.721229in}}%
\pgfpathlineto{\pgfqpoint{5.106182in}{0.717633in}}%
\pgfpathlineto{\pgfqpoint{5.110691in}{0.717148in}}%
\pgfpathlineto{\pgfqpoint{5.115200in}{0.725261in}}%
\pgfpathlineto{\pgfqpoint{5.119709in}{0.720585in}}%
\pgfpathlineto{\pgfqpoint{5.124218in}{0.721270in}}%
\pgfpathlineto{\pgfqpoint{5.128727in}{0.719437in}}%
\pgfpathlineto{\pgfqpoint{5.133236in}{0.727791in}}%
\pgfpathlineto{\pgfqpoint{5.137745in}{0.718072in}}%
\pgfpathlineto{\pgfqpoint{5.142255in}{0.717787in}}%
\pgfpathlineto{\pgfqpoint{5.146764in}{0.724798in}}%
\pgfpathlineto{\pgfqpoint{5.151273in}{0.725653in}}%
\pgfpathlineto{\pgfqpoint{5.155782in}{0.744654in}}%
\pgfpathlineto{\pgfqpoint{5.160291in}{0.727234in}}%
\pgfpathlineto{\pgfqpoint{5.164800in}{0.729435in}}%
\pgfpathlineto{\pgfqpoint{5.169309in}{0.718195in}}%
\pgfpathlineto{\pgfqpoint{5.173818in}{0.718478in}}%
\pgfpathlineto{\pgfqpoint{5.178327in}{0.719898in}}%
\pgfpathlineto{\pgfqpoint{5.182836in}{0.717720in}}%
\pgfpathlineto{\pgfqpoint{5.187345in}{0.719631in}}%
\pgfpathlineto{\pgfqpoint{5.191855in}{0.717546in}}%
\pgfpathlineto{\pgfqpoint{5.196364in}{0.729095in}}%
\pgfpathlineto{\pgfqpoint{5.200873in}{0.719568in}}%
\pgfpathlineto{\pgfqpoint{5.205382in}{0.718254in}}%
\pgfpathlineto{\pgfqpoint{5.209891in}{0.719111in}}%
\pgfpathlineto{\pgfqpoint{5.214400in}{0.729168in}}%
\pgfpathlineto{\pgfqpoint{5.223418in}{0.719555in}}%
\pgfpathlineto{\pgfqpoint{5.227927in}{0.729190in}}%
\pgfpathlineto{\pgfqpoint{5.232436in}{0.724405in}}%
\pgfpathlineto{\pgfqpoint{5.236945in}{0.722667in}}%
\pgfpathlineto{\pgfqpoint{5.241455in}{0.718654in}}%
\pgfpathlineto{\pgfqpoint{5.245964in}{0.730454in}}%
\pgfpathlineto{\pgfqpoint{5.250473in}{0.719012in}}%
\pgfpathlineto{\pgfqpoint{5.254982in}{0.726163in}}%
\pgfpathlineto{\pgfqpoint{5.259491in}{0.719872in}}%
\pgfpathlineto{\pgfqpoint{5.264000in}{0.732958in}}%
\pgfpathlineto{\pgfqpoint{5.268509in}{0.724594in}}%
\pgfpathlineto{\pgfqpoint{5.273018in}{0.730601in}}%
\pgfpathlineto{\pgfqpoint{5.277527in}{0.725114in}}%
\pgfpathlineto{\pgfqpoint{5.282036in}{0.732011in}}%
\pgfpathlineto{\pgfqpoint{5.286545in}{0.722231in}}%
\pgfpathlineto{\pgfqpoint{5.291055in}{0.722355in}}%
\pgfpathlineto{\pgfqpoint{5.295564in}{0.719197in}}%
\pgfpathlineto{\pgfqpoint{5.300073in}{0.720483in}}%
\pgfpathlineto{\pgfqpoint{5.304582in}{0.718349in}}%
\pgfpathlineto{\pgfqpoint{5.309091in}{0.719085in}}%
\pgfpathlineto{\pgfqpoint{5.322618in}{0.718634in}}%
\pgfpathlineto{\pgfqpoint{5.349673in}{0.718847in}}%
\pgfpathlineto{\pgfqpoint{5.354182in}{0.720635in}}%
\pgfpathlineto{\pgfqpoint{5.358691in}{0.718951in}}%
\pgfpathlineto{\pgfqpoint{5.367709in}{0.726403in}}%
\pgfpathlineto{\pgfqpoint{5.372218in}{0.718843in}}%
\pgfpathlineto{\pgfqpoint{5.376727in}{0.733800in}}%
\pgfpathlineto{\pgfqpoint{5.381236in}{0.719444in}}%
\pgfpathlineto{\pgfqpoint{5.385745in}{0.719136in}}%
\pgfpathlineto{\pgfqpoint{5.390255in}{0.721698in}}%
\pgfpathlineto{\pgfqpoint{5.394764in}{0.718389in}}%
\pgfpathlineto{\pgfqpoint{5.399273in}{0.719260in}}%
\pgfpathlineto{\pgfqpoint{5.403782in}{0.729777in}}%
\pgfpathlineto{\pgfqpoint{5.408291in}{0.718943in}}%
\pgfpathlineto{\pgfqpoint{5.412800in}{0.723452in}}%
\pgfpathlineto{\pgfqpoint{5.417309in}{0.726225in}}%
\pgfpathlineto{\pgfqpoint{5.421818in}{0.721953in}}%
\pgfpathlineto{\pgfqpoint{5.426327in}{0.726506in}}%
\pgfpathlineto{\pgfqpoint{5.430836in}{0.719479in}}%
\pgfpathlineto{\pgfqpoint{5.435345in}{0.728948in}}%
\pgfpathlineto{\pgfqpoint{5.439855in}{0.719896in}}%
\pgfpathlineto{\pgfqpoint{5.444364in}{0.719976in}}%
\pgfpathlineto{\pgfqpoint{5.448873in}{0.721813in}}%
\pgfpathlineto{\pgfqpoint{5.457891in}{0.719256in}}%
\pgfpathlineto{\pgfqpoint{5.462400in}{0.721251in}}%
\pgfpathlineto{\pgfqpoint{5.466909in}{0.724688in}}%
\pgfpathlineto{\pgfqpoint{5.471418in}{0.723726in}}%
\pgfpathlineto{\pgfqpoint{5.475927in}{0.719144in}}%
\pgfpathlineto{\pgfqpoint{5.480436in}{0.720124in}}%
\pgfpathlineto{\pgfqpoint{5.493964in}{0.719095in}}%
\pgfpathlineto{\pgfqpoint{5.498473in}{0.723581in}}%
\pgfpathlineto{\pgfqpoint{5.502982in}{0.719709in}}%
\pgfpathlineto{\pgfqpoint{5.507491in}{0.729356in}}%
\pgfpathlineto{\pgfqpoint{5.512000in}{0.722074in}}%
\pgfpathlineto{\pgfqpoint{5.516509in}{0.719208in}}%
\pgfpathlineto{\pgfqpoint{5.521018in}{0.721642in}}%
\pgfpathlineto{\pgfqpoint{5.525527in}{0.719681in}}%
\pgfpathlineto{\pgfqpoint{5.530036in}{0.719919in}}%
\pgfpathlineto{\pgfqpoint{5.534545in}{0.719031in}}%
\pgfpathlineto{\pgfqpoint{5.534545in}{0.719031in}}%
\pgfusepath{stroke}%
\end{pgfscope}%
\begin{pgfscope}%
\pgfpathrectangle{\pgfqpoint{0.800000in}{0.528000in}}{\pgfqpoint{4.960000in}{3.696000in}}%
\pgfusepath{clip}%
\pgfsetrectcap%
\pgfsetroundjoin%
\pgfsetlinewidth{1.505625pt}%
\definecolor{currentstroke}{rgb}{1.000000,0.000000,1.000000}%
\pgfsetstrokecolor{currentstroke}%
\pgfsetdash{}{0pt}%
\pgfpathmoveto{\pgfqpoint{1.025455in}{0.696064in}}%
\pgfpathlineto{\pgfqpoint{1.034473in}{0.696151in}}%
\pgfpathlineto{\pgfqpoint{1.043491in}{0.698174in}}%
\pgfpathlineto{\pgfqpoint{1.048000in}{0.696499in}}%
\pgfpathlineto{\pgfqpoint{1.057018in}{0.696308in}}%
\pgfpathlineto{\pgfqpoint{1.061527in}{0.697649in}}%
\pgfpathlineto{\pgfqpoint{1.066036in}{0.696347in}}%
\pgfpathlineto{\pgfqpoint{1.106618in}{0.696592in}}%
\pgfpathlineto{\pgfqpoint{1.115636in}{0.696444in}}%
\pgfpathlineto{\pgfqpoint{1.129164in}{0.696585in}}%
\pgfpathlineto{\pgfqpoint{1.133673in}{0.698030in}}%
\pgfpathlineto{\pgfqpoint{1.138182in}{0.696598in}}%
\pgfpathlineto{\pgfqpoint{1.165236in}{0.696745in}}%
\pgfpathlineto{\pgfqpoint{1.169745in}{0.698333in}}%
\pgfpathlineto{\pgfqpoint{1.174255in}{0.696789in}}%
\pgfpathlineto{\pgfqpoint{1.183273in}{0.696983in}}%
\pgfpathlineto{\pgfqpoint{1.187782in}{0.698318in}}%
\pgfpathlineto{\pgfqpoint{1.219345in}{0.696922in}}%
\pgfpathlineto{\pgfqpoint{1.223855in}{0.699552in}}%
\pgfpathlineto{\pgfqpoint{1.228364in}{0.697093in}}%
\pgfpathlineto{\pgfqpoint{1.232873in}{0.697152in}}%
\pgfpathlineto{\pgfqpoint{1.237382in}{0.700179in}}%
\pgfpathlineto{\pgfqpoint{1.241891in}{0.700160in}}%
\pgfpathlineto{\pgfqpoint{1.246400in}{0.697296in}}%
\pgfpathlineto{\pgfqpoint{1.286982in}{0.697720in}}%
\pgfpathlineto{\pgfqpoint{1.300509in}{0.697660in}}%
\pgfpathlineto{\pgfqpoint{1.318545in}{0.697453in}}%
\pgfpathlineto{\pgfqpoint{1.336582in}{0.698300in}}%
\pgfpathlineto{\pgfqpoint{1.341091in}{0.700848in}}%
\pgfpathlineto{\pgfqpoint{1.350109in}{0.698612in}}%
\pgfpathlineto{\pgfqpoint{1.359127in}{0.705082in}}%
\pgfpathlineto{\pgfqpoint{1.363636in}{0.706335in}}%
\pgfpathlineto{\pgfqpoint{1.372655in}{0.699989in}}%
\pgfpathlineto{\pgfqpoint{1.377164in}{0.699784in}}%
\pgfpathlineto{\pgfqpoint{1.381673in}{0.701112in}}%
\pgfpathlineto{\pgfqpoint{1.386182in}{0.698717in}}%
\pgfpathlineto{\pgfqpoint{1.390691in}{0.704439in}}%
\pgfpathlineto{\pgfqpoint{1.395200in}{0.700639in}}%
\pgfpathlineto{\pgfqpoint{1.399709in}{0.703845in}}%
\pgfpathlineto{\pgfqpoint{1.404218in}{0.699050in}}%
\pgfpathlineto{\pgfqpoint{1.408727in}{0.699979in}}%
\pgfpathlineto{\pgfqpoint{1.413236in}{0.697941in}}%
\pgfpathlineto{\pgfqpoint{1.426764in}{0.698295in}}%
\pgfpathlineto{\pgfqpoint{1.440291in}{0.698155in}}%
\pgfpathlineto{\pgfqpoint{1.498909in}{0.698621in}}%
\pgfpathlineto{\pgfqpoint{1.507927in}{0.704218in}}%
\pgfpathlineto{\pgfqpoint{1.512436in}{0.699437in}}%
\pgfpathlineto{\pgfqpoint{1.521455in}{0.698796in}}%
\pgfpathlineto{\pgfqpoint{1.539491in}{0.699121in}}%
\pgfpathlineto{\pgfqpoint{1.544000in}{0.699103in}}%
\pgfpathlineto{\pgfqpoint{1.548509in}{0.700825in}}%
\pgfpathlineto{\pgfqpoint{1.553018in}{0.698876in}}%
\pgfpathlineto{\pgfqpoint{1.575564in}{0.699034in}}%
\pgfpathlineto{\pgfqpoint{1.580073in}{0.702365in}}%
\pgfpathlineto{\pgfqpoint{1.584582in}{0.700182in}}%
\pgfpathlineto{\pgfqpoint{1.589091in}{0.701719in}}%
\pgfpathlineto{\pgfqpoint{1.602618in}{0.700084in}}%
\pgfpathlineto{\pgfqpoint{1.607127in}{0.703546in}}%
\pgfpathlineto{\pgfqpoint{1.620655in}{0.704002in}}%
\pgfpathlineto{\pgfqpoint{1.625164in}{0.706753in}}%
\pgfpathlineto{\pgfqpoint{1.629673in}{0.699420in}}%
\pgfpathlineto{\pgfqpoint{1.661236in}{0.699448in}}%
\pgfpathlineto{\pgfqpoint{1.665745in}{0.703656in}}%
\pgfpathlineto{\pgfqpoint{1.670255in}{0.703246in}}%
\pgfpathlineto{\pgfqpoint{1.674764in}{0.700687in}}%
\pgfpathlineto{\pgfqpoint{1.679273in}{0.705402in}}%
\pgfpathlineto{\pgfqpoint{1.683782in}{0.699239in}}%
\pgfpathlineto{\pgfqpoint{1.688291in}{0.700587in}}%
\pgfpathlineto{\pgfqpoint{1.692800in}{0.699880in}}%
\pgfpathlineto{\pgfqpoint{1.701818in}{0.700578in}}%
\pgfpathlineto{\pgfqpoint{1.706327in}{0.700680in}}%
\pgfpathlineto{\pgfqpoint{1.710836in}{0.702256in}}%
\pgfpathlineto{\pgfqpoint{1.715345in}{0.705232in}}%
\pgfpathlineto{\pgfqpoint{1.719855in}{0.699672in}}%
\pgfpathlineto{\pgfqpoint{1.724364in}{0.699708in}}%
\pgfpathlineto{\pgfqpoint{1.728873in}{0.701268in}}%
\pgfpathlineto{\pgfqpoint{1.733382in}{0.700263in}}%
\pgfpathlineto{\pgfqpoint{1.746909in}{0.700527in}}%
\pgfpathlineto{\pgfqpoint{1.751418in}{0.705839in}}%
\pgfpathlineto{\pgfqpoint{1.755927in}{0.703363in}}%
\pgfpathlineto{\pgfqpoint{1.760436in}{0.703046in}}%
\pgfpathlineto{\pgfqpoint{1.764945in}{0.700999in}}%
\pgfpathlineto{\pgfqpoint{1.778473in}{0.701482in}}%
\pgfpathlineto{\pgfqpoint{1.782982in}{0.711653in}}%
\pgfpathlineto{\pgfqpoint{1.787491in}{0.702836in}}%
\pgfpathlineto{\pgfqpoint{1.792000in}{0.702293in}}%
\pgfpathlineto{\pgfqpoint{1.796509in}{0.706590in}}%
\pgfpathlineto{\pgfqpoint{1.801018in}{0.701869in}}%
\pgfpathlineto{\pgfqpoint{1.805527in}{0.706295in}}%
\pgfpathlineto{\pgfqpoint{1.810036in}{0.700972in}}%
\pgfpathlineto{\pgfqpoint{1.814545in}{0.701332in}}%
\pgfpathlineto{\pgfqpoint{1.819055in}{0.705065in}}%
\pgfpathlineto{\pgfqpoint{1.823564in}{0.702356in}}%
\pgfpathlineto{\pgfqpoint{1.828073in}{0.701104in}}%
\pgfpathlineto{\pgfqpoint{1.832582in}{0.704531in}}%
\pgfpathlineto{\pgfqpoint{1.837091in}{0.706354in}}%
\pgfpathlineto{\pgfqpoint{1.841600in}{0.702797in}}%
\pgfpathlineto{\pgfqpoint{1.846109in}{0.703565in}}%
\pgfpathlineto{\pgfqpoint{1.850618in}{0.709165in}}%
\pgfpathlineto{\pgfqpoint{1.855127in}{0.707673in}}%
\pgfpathlineto{\pgfqpoint{1.859636in}{0.701235in}}%
\pgfpathlineto{\pgfqpoint{1.877673in}{0.701620in}}%
\pgfpathlineto{\pgfqpoint{1.882182in}{0.707946in}}%
\pgfpathlineto{\pgfqpoint{1.886691in}{0.701716in}}%
\pgfpathlineto{\pgfqpoint{1.895709in}{0.701348in}}%
\pgfpathlineto{\pgfqpoint{1.900218in}{0.709466in}}%
\pgfpathlineto{\pgfqpoint{1.904727in}{0.704664in}}%
\pgfpathlineto{\pgfqpoint{1.909236in}{0.701526in}}%
\pgfpathlineto{\pgfqpoint{1.940800in}{0.702189in}}%
\pgfpathlineto{\pgfqpoint{1.949818in}{0.701951in}}%
\pgfpathlineto{\pgfqpoint{1.954327in}{0.704268in}}%
\pgfpathlineto{\pgfqpoint{1.972364in}{0.702820in}}%
\pgfpathlineto{\pgfqpoint{1.985891in}{0.704089in}}%
\pgfpathlineto{\pgfqpoint{1.990400in}{0.702087in}}%
\pgfpathlineto{\pgfqpoint{2.003927in}{0.702509in}}%
\pgfpathlineto{\pgfqpoint{2.008436in}{0.705314in}}%
\pgfpathlineto{\pgfqpoint{2.021964in}{0.706940in}}%
\pgfpathlineto{\pgfqpoint{2.026473in}{0.702630in}}%
\pgfpathlineto{\pgfqpoint{2.035491in}{0.704293in}}%
\pgfpathlineto{\pgfqpoint{2.040000in}{0.706412in}}%
\pgfpathlineto{\pgfqpoint{2.044509in}{0.704923in}}%
\pgfpathlineto{\pgfqpoint{2.049018in}{0.709448in}}%
\pgfpathlineto{\pgfqpoint{2.062545in}{0.703577in}}%
\pgfpathlineto{\pgfqpoint{2.071564in}{0.702828in}}%
\pgfpathlineto{\pgfqpoint{2.076073in}{0.704331in}}%
\pgfpathlineto{\pgfqpoint{2.080582in}{0.703371in}}%
\pgfpathlineto{\pgfqpoint{2.085091in}{0.710688in}}%
\pgfpathlineto{\pgfqpoint{2.089600in}{0.702975in}}%
\pgfpathlineto{\pgfqpoint{2.094109in}{0.707780in}}%
\pgfpathlineto{\pgfqpoint{2.098618in}{0.703827in}}%
\pgfpathlineto{\pgfqpoint{2.107636in}{0.709482in}}%
\pgfpathlineto{\pgfqpoint{2.112145in}{0.703504in}}%
\pgfpathlineto{\pgfqpoint{2.116655in}{0.704619in}}%
\pgfpathlineto{\pgfqpoint{2.121164in}{0.718132in}}%
\pgfpathlineto{\pgfqpoint{2.125673in}{0.708727in}}%
\pgfpathlineto{\pgfqpoint{2.130182in}{0.716805in}}%
\pgfpathlineto{\pgfqpoint{2.134691in}{0.712708in}}%
\pgfpathlineto{\pgfqpoint{2.139200in}{0.719535in}}%
\pgfpathlineto{\pgfqpoint{2.143709in}{0.704406in}}%
\pgfpathlineto{\pgfqpoint{2.152727in}{0.703186in}}%
\pgfpathlineto{\pgfqpoint{2.161745in}{0.704989in}}%
\pgfpathlineto{\pgfqpoint{2.166255in}{0.721846in}}%
\pgfpathlineto{\pgfqpoint{2.170764in}{0.705711in}}%
\pgfpathlineto{\pgfqpoint{2.179782in}{0.720371in}}%
\pgfpathlineto{\pgfqpoint{2.184291in}{0.703752in}}%
\pgfpathlineto{\pgfqpoint{2.188800in}{0.718324in}}%
\pgfpathlineto{\pgfqpoint{2.193309in}{0.714927in}}%
\pgfpathlineto{\pgfqpoint{2.197818in}{0.722696in}}%
\pgfpathlineto{\pgfqpoint{2.206836in}{0.705213in}}%
\pgfpathlineto{\pgfqpoint{2.211345in}{0.713832in}}%
\pgfpathlineto{\pgfqpoint{2.215855in}{0.705854in}}%
\pgfpathlineto{\pgfqpoint{2.220364in}{0.705993in}}%
\pgfpathlineto{\pgfqpoint{2.224873in}{0.704945in}}%
\pgfpathlineto{\pgfqpoint{2.229382in}{0.708884in}}%
\pgfpathlineto{\pgfqpoint{2.238400in}{0.704092in}}%
\pgfpathlineto{\pgfqpoint{2.242909in}{0.708275in}}%
\pgfpathlineto{\pgfqpoint{2.247418in}{0.704324in}}%
\pgfpathlineto{\pgfqpoint{2.256436in}{0.704785in}}%
\pgfpathlineto{\pgfqpoint{2.260945in}{0.712047in}}%
\pgfpathlineto{\pgfqpoint{2.265455in}{0.704556in}}%
\pgfpathlineto{\pgfqpoint{2.269964in}{0.713083in}}%
\pgfpathlineto{\pgfqpoint{2.274473in}{0.707956in}}%
\pgfpathlineto{\pgfqpoint{2.278982in}{0.705177in}}%
\pgfpathlineto{\pgfqpoint{2.283491in}{0.704402in}}%
\pgfpathlineto{\pgfqpoint{2.288000in}{0.708642in}}%
\pgfpathlineto{\pgfqpoint{2.292509in}{0.704525in}}%
\pgfpathlineto{\pgfqpoint{2.297018in}{0.705616in}}%
\pgfpathlineto{\pgfqpoint{2.301527in}{0.708573in}}%
\pgfpathlineto{\pgfqpoint{2.306036in}{0.709085in}}%
\pgfpathlineto{\pgfqpoint{2.310545in}{0.705020in}}%
\pgfpathlineto{\pgfqpoint{2.315055in}{0.710001in}}%
\pgfpathlineto{\pgfqpoint{2.319564in}{0.710365in}}%
\pgfpathlineto{\pgfqpoint{2.324073in}{0.704762in}}%
\pgfpathlineto{\pgfqpoint{2.328582in}{0.706177in}}%
\pgfpathlineto{\pgfqpoint{2.333091in}{0.705647in}}%
\pgfpathlineto{\pgfqpoint{2.337600in}{0.706622in}}%
\pgfpathlineto{\pgfqpoint{2.342109in}{0.709436in}}%
\pgfpathlineto{\pgfqpoint{2.346618in}{0.704905in}}%
\pgfpathlineto{\pgfqpoint{2.360145in}{0.705604in}}%
\pgfpathlineto{\pgfqpoint{2.364655in}{0.709352in}}%
\pgfpathlineto{\pgfqpoint{2.369164in}{0.705712in}}%
\pgfpathlineto{\pgfqpoint{2.373673in}{0.710072in}}%
\pgfpathlineto{\pgfqpoint{2.387200in}{0.710433in}}%
\pgfpathlineto{\pgfqpoint{2.391709in}{0.707069in}}%
\pgfpathlineto{\pgfqpoint{2.400727in}{0.705555in}}%
\pgfpathlineto{\pgfqpoint{2.405236in}{0.709422in}}%
\pgfpathlineto{\pgfqpoint{2.409745in}{0.706745in}}%
\pgfpathlineto{\pgfqpoint{2.414255in}{0.705824in}}%
\pgfpathlineto{\pgfqpoint{2.418764in}{0.707459in}}%
\pgfpathlineto{\pgfqpoint{2.423273in}{0.705830in}}%
\pgfpathlineto{\pgfqpoint{2.427782in}{0.707763in}}%
\pgfpathlineto{\pgfqpoint{2.432291in}{0.715277in}}%
\pgfpathlineto{\pgfqpoint{2.436800in}{0.706789in}}%
\pgfpathlineto{\pgfqpoint{2.441309in}{0.712029in}}%
\pgfpathlineto{\pgfqpoint{2.445818in}{0.710095in}}%
\pgfpathlineto{\pgfqpoint{2.450327in}{0.715233in}}%
\pgfpathlineto{\pgfqpoint{2.454836in}{0.706256in}}%
\pgfpathlineto{\pgfqpoint{2.459345in}{0.708537in}}%
\pgfpathlineto{\pgfqpoint{2.463855in}{0.705922in}}%
\pgfpathlineto{\pgfqpoint{2.468364in}{0.711375in}}%
\pgfpathlineto{\pgfqpoint{2.472873in}{0.713425in}}%
\pgfpathlineto{\pgfqpoint{2.477382in}{0.708954in}}%
\pgfpathlineto{\pgfqpoint{2.481891in}{0.706238in}}%
\pgfpathlineto{\pgfqpoint{2.486400in}{0.711590in}}%
\pgfpathlineto{\pgfqpoint{2.490909in}{0.708697in}}%
\pgfpathlineto{\pgfqpoint{2.495418in}{0.707189in}}%
\pgfpathlineto{\pgfqpoint{2.504436in}{0.707386in}}%
\pgfpathlineto{\pgfqpoint{2.508945in}{0.709291in}}%
\pgfpathlineto{\pgfqpoint{2.513455in}{0.706620in}}%
\pgfpathlineto{\pgfqpoint{2.522473in}{0.713218in}}%
\pgfpathlineto{\pgfqpoint{2.526982in}{0.706643in}}%
\pgfpathlineto{\pgfqpoint{2.531491in}{0.708590in}}%
\pgfpathlineto{\pgfqpoint{2.536000in}{0.706458in}}%
\pgfpathlineto{\pgfqpoint{2.540509in}{0.707442in}}%
\pgfpathlineto{\pgfqpoint{2.545018in}{0.711003in}}%
\pgfpathlineto{\pgfqpoint{2.549527in}{0.710534in}}%
\pgfpathlineto{\pgfqpoint{2.554036in}{0.706822in}}%
\pgfpathlineto{\pgfqpoint{2.558545in}{0.707232in}}%
\pgfpathlineto{\pgfqpoint{2.563055in}{0.706235in}}%
\pgfpathlineto{\pgfqpoint{2.567564in}{0.708146in}}%
\pgfpathlineto{\pgfqpoint{2.581091in}{0.707340in}}%
\pgfpathlineto{\pgfqpoint{2.585600in}{0.713094in}}%
\pgfpathlineto{\pgfqpoint{2.594618in}{0.706749in}}%
\pgfpathlineto{\pgfqpoint{2.599127in}{0.708809in}}%
\pgfpathlineto{\pgfqpoint{2.603636in}{0.708045in}}%
\pgfpathlineto{\pgfqpoint{2.608145in}{0.709683in}}%
\pgfpathlineto{\pgfqpoint{2.612655in}{0.707952in}}%
\pgfpathlineto{\pgfqpoint{2.617164in}{0.707862in}}%
\pgfpathlineto{\pgfqpoint{2.621673in}{0.714167in}}%
\pgfpathlineto{\pgfqpoint{2.626182in}{0.708319in}}%
\pgfpathlineto{\pgfqpoint{2.630691in}{0.710087in}}%
\pgfpathlineto{\pgfqpoint{2.635200in}{0.707891in}}%
\pgfpathlineto{\pgfqpoint{2.639709in}{0.707931in}}%
\pgfpathlineto{\pgfqpoint{2.644218in}{0.710067in}}%
\pgfpathlineto{\pgfqpoint{2.653236in}{0.709502in}}%
\pgfpathlineto{\pgfqpoint{2.657745in}{0.708150in}}%
\pgfpathlineto{\pgfqpoint{2.662255in}{0.708203in}}%
\pgfpathlineto{\pgfqpoint{2.666764in}{0.706994in}}%
\pgfpathlineto{\pgfqpoint{2.671273in}{0.709484in}}%
\pgfpathlineto{\pgfqpoint{2.675782in}{0.708249in}}%
\pgfpathlineto{\pgfqpoint{2.684800in}{0.710776in}}%
\pgfpathlineto{\pgfqpoint{2.689309in}{0.707748in}}%
\pgfpathlineto{\pgfqpoint{2.693818in}{0.707698in}}%
\pgfpathlineto{\pgfqpoint{2.698327in}{0.714351in}}%
\pgfpathlineto{\pgfqpoint{2.707345in}{0.720508in}}%
\pgfpathlineto{\pgfqpoint{2.711855in}{0.710383in}}%
\pgfpathlineto{\pgfqpoint{2.716364in}{0.710430in}}%
\pgfpathlineto{\pgfqpoint{2.720873in}{0.715845in}}%
\pgfpathlineto{\pgfqpoint{2.725382in}{0.709114in}}%
\pgfpathlineto{\pgfqpoint{2.752436in}{0.709479in}}%
\pgfpathlineto{\pgfqpoint{2.756945in}{0.712767in}}%
\pgfpathlineto{\pgfqpoint{2.761455in}{0.713270in}}%
\pgfpathlineto{\pgfqpoint{2.765964in}{0.710370in}}%
\pgfpathlineto{\pgfqpoint{2.770473in}{0.709275in}}%
\pgfpathlineto{\pgfqpoint{2.774982in}{0.710795in}}%
\pgfpathlineto{\pgfqpoint{2.779491in}{0.710159in}}%
\pgfpathlineto{\pgfqpoint{2.784000in}{0.712085in}}%
\pgfpathlineto{\pgfqpoint{2.788509in}{0.709728in}}%
\pgfpathlineto{\pgfqpoint{2.793018in}{0.711382in}}%
\pgfpathlineto{\pgfqpoint{2.797527in}{0.710929in}}%
\pgfpathlineto{\pgfqpoint{2.802036in}{0.713032in}}%
\pgfpathlineto{\pgfqpoint{2.806545in}{0.712244in}}%
\pgfpathlineto{\pgfqpoint{2.815564in}{0.713308in}}%
\pgfpathlineto{\pgfqpoint{2.820073in}{0.710083in}}%
\pgfpathlineto{\pgfqpoint{2.824582in}{0.709552in}}%
\pgfpathlineto{\pgfqpoint{2.829091in}{0.716091in}}%
\pgfpathlineto{\pgfqpoint{2.833600in}{0.709181in}}%
\pgfpathlineto{\pgfqpoint{2.838109in}{0.710031in}}%
\pgfpathlineto{\pgfqpoint{2.842618in}{0.708953in}}%
\pgfpathlineto{\pgfqpoint{2.860655in}{0.711801in}}%
\pgfpathlineto{\pgfqpoint{2.865164in}{0.709134in}}%
\pgfpathlineto{\pgfqpoint{2.878691in}{0.711064in}}%
\pgfpathlineto{\pgfqpoint{2.887709in}{0.709292in}}%
\pgfpathlineto{\pgfqpoint{2.892218in}{0.726444in}}%
\pgfpathlineto{\pgfqpoint{2.896727in}{0.723907in}}%
\pgfpathlineto{\pgfqpoint{2.901236in}{0.714220in}}%
\pgfpathlineto{\pgfqpoint{2.905745in}{0.711122in}}%
\pgfpathlineto{\pgfqpoint{2.914764in}{0.710013in}}%
\pgfpathlineto{\pgfqpoint{2.919273in}{0.722247in}}%
\pgfpathlineto{\pgfqpoint{2.923782in}{0.710815in}}%
\pgfpathlineto{\pgfqpoint{2.928291in}{0.710505in}}%
\pgfpathlineto{\pgfqpoint{2.932800in}{0.719079in}}%
\pgfpathlineto{\pgfqpoint{2.937309in}{0.713957in}}%
\pgfpathlineto{\pgfqpoint{2.941818in}{0.712639in}}%
\pgfpathlineto{\pgfqpoint{2.946327in}{0.713900in}}%
\pgfpathlineto{\pgfqpoint{2.950836in}{0.710339in}}%
\pgfpathlineto{\pgfqpoint{2.955345in}{0.709630in}}%
\pgfpathlineto{\pgfqpoint{2.959855in}{0.714229in}}%
\pgfpathlineto{\pgfqpoint{2.964364in}{0.711409in}}%
\pgfpathlineto{\pgfqpoint{2.968873in}{0.714300in}}%
\pgfpathlineto{\pgfqpoint{2.973382in}{0.712296in}}%
\pgfpathlineto{\pgfqpoint{2.977891in}{0.711594in}}%
\pgfpathlineto{\pgfqpoint{2.982400in}{0.713778in}}%
\pgfpathlineto{\pgfqpoint{2.986909in}{0.721410in}}%
\pgfpathlineto{\pgfqpoint{2.991418in}{0.718376in}}%
\pgfpathlineto{\pgfqpoint{2.995927in}{0.712196in}}%
\pgfpathlineto{\pgfqpoint{3.000436in}{0.711897in}}%
\pgfpathlineto{\pgfqpoint{3.004945in}{0.713941in}}%
\pgfpathlineto{\pgfqpoint{3.009455in}{0.713585in}}%
\pgfpathlineto{\pgfqpoint{3.013964in}{0.712054in}}%
\pgfpathlineto{\pgfqpoint{3.022982in}{0.711893in}}%
\pgfpathlineto{\pgfqpoint{3.027491in}{0.714019in}}%
\pgfpathlineto{\pgfqpoint{3.032000in}{0.713365in}}%
\pgfpathlineto{\pgfqpoint{3.036509in}{0.710142in}}%
\pgfpathlineto{\pgfqpoint{3.041018in}{0.713200in}}%
\pgfpathlineto{\pgfqpoint{3.045527in}{0.714636in}}%
\pgfpathlineto{\pgfqpoint{3.050036in}{0.711320in}}%
\pgfpathlineto{\pgfqpoint{3.054545in}{0.715174in}}%
\pgfpathlineto{\pgfqpoint{3.059055in}{0.711533in}}%
\pgfpathlineto{\pgfqpoint{3.072582in}{0.711683in}}%
\pgfpathlineto{\pgfqpoint{3.077091in}{0.715615in}}%
\pgfpathlineto{\pgfqpoint{3.081600in}{0.714650in}}%
\pgfpathlineto{\pgfqpoint{3.086109in}{0.711200in}}%
\pgfpathlineto{\pgfqpoint{3.090618in}{0.711312in}}%
\pgfpathlineto{\pgfqpoint{3.095127in}{0.712545in}}%
\pgfpathlineto{\pgfqpoint{3.099636in}{0.716123in}}%
\pgfpathlineto{\pgfqpoint{3.108655in}{0.713728in}}%
\pgfpathlineto{\pgfqpoint{3.113164in}{0.711878in}}%
\pgfpathlineto{\pgfqpoint{3.117673in}{0.716082in}}%
\pgfpathlineto{\pgfqpoint{3.122182in}{0.715284in}}%
\pgfpathlineto{\pgfqpoint{3.126691in}{0.713002in}}%
\pgfpathlineto{\pgfqpoint{3.131200in}{0.716094in}}%
\pgfpathlineto{\pgfqpoint{3.135709in}{0.716770in}}%
\pgfpathlineto{\pgfqpoint{3.140218in}{0.711932in}}%
\pgfpathlineto{\pgfqpoint{3.149236in}{0.716393in}}%
\pgfpathlineto{\pgfqpoint{3.153745in}{0.713764in}}%
\pgfpathlineto{\pgfqpoint{3.158255in}{0.715890in}}%
\pgfpathlineto{\pgfqpoint{3.162764in}{0.711903in}}%
\pgfpathlineto{\pgfqpoint{3.171782in}{0.714053in}}%
\pgfpathlineto{\pgfqpoint{3.176291in}{0.717934in}}%
\pgfpathlineto{\pgfqpoint{3.180800in}{0.714425in}}%
\pgfpathlineto{\pgfqpoint{3.189818in}{0.711434in}}%
\pgfpathlineto{\pgfqpoint{3.194327in}{0.715982in}}%
\pgfpathlineto{\pgfqpoint{3.198836in}{0.712412in}}%
\pgfpathlineto{\pgfqpoint{3.203345in}{0.717822in}}%
\pgfpathlineto{\pgfqpoint{3.207855in}{0.720208in}}%
\pgfpathlineto{\pgfqpoint{3.212364in}{0.718485in}}%
\pgfpathlineto{\pgfqpoint{3.216873in}{0.713307in}}%
\pgfpathlineto{\pgfqpoint{3.221382in}{0.722285in}}%
\pgfpathlineto{\pgfqpoint{3.225891in}{0.714314in}}%
\pgfpathlineto{\pgfqpoint{3.234909in}{0.717123in}}%
\pgfpathlineto{\pgfqpoint{3.239418in}{0.713122in}}%
\pgfpathlineto{\pgfqpoint{3.243927in}{0.712000in}}%
\pgfpathlineto{\pgfqpoint{3.248436in}{0.715796in}}%
\pgfpathlineto{\pgfqpoint{3.257455in}{0.713393in}}%
\pgfpathlineto{\pgfqpoint{3.261964in}{0.716937in}}%
\pgfpathlineto{\pgfqpoint{3.266473in}{0.713540in}}%
\pgfpathlineto{\pgfqpoint{3.275491in}{0.715110in}}%
\pgfpathlineto{\pgfqpoint{3.280000in}{0.719155in}}%
\pgfpathlineto{\pgfqpoint{3.284509in}{0.713615in}}%
\pgfpathlineto{\pgfqpoint{3.289018in}{0.712523in}}%
\pgfpathlineto{\pgfqpoint{3.293527in}{0.717418in}}%
\pgfpathlineto{\pgfqpoint{3.298036in}{0.718185in}}%
\pgfpathlineto{\pgfqpoint{3.302545in}{0.714712in}}%
\pgfpathlineto{\pgfqpoint{3.311564in}{0.713633in}}%
\pgfpathlineto{\pgfqpoint{3.316073in}{0.717443in}}%
\pgfpathlineto{\pgfqpoint{3.320582in}{0.713777in}}%
\pgfpathlineto{\pgfqpoint{3.325091in}{0.723500in}}%
\pgfpathlineto{\pgfqpoint{3.329600in}{0.719372in}}%
\pgfpathlineto{\pgfqpoint{3.334109in}{0.713514in}}%
\pgfpathlineto{\pgfqpoint{3.338618in}{0.714267in}}%
\pgfpathlineto{\pgfqpoint{3.343127in}{0.717165in}}%
\pgfpathlineto{\pgfqpoint{3.347636in}{0.712952in}}%
\pgfpathlineto{\pgfqpoint{3.352145in}{0.714830in}}%
\pgfpathlineto{\pgfqpoint{3.361164in}{0.712654in}}%
\pgfpathlineto{\pgfqpoint{3.365673in}{0.718012in}}%
\pgfpathlineto{\pgfqpoint{3.370182in}{0.713737in}}%
\pgfpathlineto{\pgfqpoint{3.374691in}{0.714797in}}%
\pgfpathlineto{\pgfqpoint{3.379200in}{0.714613in}}%
\pgfpathlineto{\pgfqpoint{3.383709in}{0.716488in}}%
\pgfpathlineto{\pgfqpoint{3.388218in}{0.720922in}}%
\pgfpathlineto{\pgfqpoint{3.392727in}{0.714732in}}%
\pgfpathlineto{\pgfqpoint{3.401745in}{0.713753in}}%
\pgfpathlineto{\pgfqpoint{3.406255in}{0.733844in}}%
\pgfpathlineto{\pgfqpoint{3.410764in}{0.721669in}}%
\pgfpathlineto{\pgfqpoint{3.419782in}{0.717865in}}%
\pgfpathlineto{\pgfqpoint{3.428800in}{0.714709in}}%
\pgfpathlineto{\pgfqpoint{3.437818in}{0.714692in}}%
\pgfpathlineto{\pgfqpoint{3.442327in}{0.732911in}}%
\pgfpathlineto{\pgfqpoint{3.446836in}{0.715693in}}%
\pgfpathlineto{\pgfqpoint{3.451345in}{0.718642in}}%
\pgfpathlineto{\pgfqpoint{3.455855in}{0.716788in}}%
\pgfpathlineto{\pgfqpoint{3.460364in}{0.720174in}}%
\pgfpathlineto{\pgfqpoint{3.464873in}{0.716409in}}%
\pgfpathlineto{\pgfqpoint{3.469382in}{0.717330in}}%
\pgfpathlineto{\pgfqpoint{3.473891in}{0.719905in}}%
\pgfpathlineto{\pgfqpoint{3.478400in}{0.715647in}}%
\pgfpathlineto{\pgfqpoint{3.482909in}{0.720203in}}%
\pgfpathlineto{\pgfqpoint{3.487418in}{0.716303in}}%
\pgfpathlineto{\pgfqpoint{3.491927in}{0.713928in}}%
\pgfpathlineto{\pgfqpoint{3.496436in}{0.719298in}}%
\pgfpathlineto{\pgfqpoint{3.500945in}{0.715501in}}%
\pgfpathlineto{\pgfqpoint{3.505455in}{0.715064in}}%
\pgfpathlineto{\pgfqpoint{3.509964in}{0.719271in}}%
\pgfpathlineto{\pgfqpoint{3.514473in}{0.718559in}}%
\pgfpathlineto{\pgfqpoint{3.518982in}{0.720824in}}%
\pgfpathlineto{\pgfqpoint{3.523491in}{0.716039in}}%
\pgfpathlineto{\pgfqpoint{3.528000in}{0.714783in}}%
\pgfpathlineto{\pgfqpoint{3.537018in}{0.718934in}}%
\pgfpathlineto{\pgfqpoint{3.541527in}{0.715147in}}%
\pgfpathlineto{\pgfqpoint{3.550545in}{0.715749in}}%
\pgfpathlineto{\pgfqpoint{3.555055in}{0.716981in}}%
\pgfpathlineto{\pgfqpoint{3.559564in}{0.715312in}}%
\pgfpathlineto{\pgfqpoint{3.564073in}{0.717729in}}%
\pgfpathlineto{\pgfqpoint{3.568582in}{0.717979in}}%
\pgfpathlineto{\pgfqpoint{3.573091in}{0.721418in}}%
\pgfpathlineto{\pgfqpoint{3.577600in}{0.718200in}}%
\pgfpathlineto{\pgfqpoint{3.586618in}{0.715669in}}%
\pgfpathlineto{\pgfqpoint{3.591127in}{0.720320in}}%
\pgfpathlineto{\pgfqpoint{3.595636in}{0.717840in}}%
\pgfpathlineto{\pgfqpoint{3.600145in}{0.727281in}}%
\pgfpathlineto{\pgfqpoint{3.604655in}{0.718030in}}%
\pgfpathlineto{\pgfqpoint{3.609164in}{0.723877in}}%
\pgfpathlineto{\pgfqpoint{3.613673in}{0.716780in}}%
\pgfpathlineto{\pgfqpoint{3.618182in}{0.718156in}}%
\pgfpathlineto{\pgfqpoint{3.622691in}{0.720737in}}%
\pgfpathlineto{\pgfqpoint{3.627200in}{0.717233in}}%
\pgfpathlineto{\pgfqpoint{3.631709in}{0.722934in}}%
\pgfpathlineto{\pgfqpoint{3.636218in}{0.717360in}}%
\pgfpathlineto{\pgfqpoint{3.645236in}{0.716122in}}%
\pgfpathlineto{\pgfqpoint{3.649745in}{0.717884in}}%
\pgfpathlineto{\pgfqpoint{3.654255in}{0.721335in}}%
\pgfpathlineto{\pgfqpoint{3.658764in}{0.732356in}}%
\pgfpathlineto{\pgfqpoint{3.663273in}{0.717814in}}%
\pgfpathlineto{\pgfqpoint{3.667782in}{0.716415in}}%
\pgfpathlineto{\pgfqpoint{3.672291in}{0.723625in}}%
\pgfpathlineto{\pgfqpoint{3.681309in}{0.715187in}}%
\pgfpathlineto{\pgfqpoint{3.685818in}{0.720218in}}%
\pgfpathlineto{\pgfqpoint{3.690327in}{0.732766in}}%
\pgfpathlineto{\pgfqpoint{3.694836in}{0.719373in}}%
\pgfpathlineto{\pgfqpoint{3.699345in}{0.721837in}}%
\pgfpathlineto{\pgfqpoint{3.703855in}{0.720707in}}%
\pgfpathlineto{\pgfqpoint{3.708364in}{0.715804in}}%
\pgfpathlineto{\pgfqpoint{3.712873in}{0.718656in}}%
\pgfpathlineto{\pgfqpoint{3.717382in}{0.716213in}}%
\pgfpathlineto{\pgfqpoint{3.721891in}{0.719395in}}%
\pgfpathlineto{\pgfqpoint{3.730909in}{0.718727in}}%
\pgfpathlineto{\pgfqpoint{3.735418in}{0.719955in}}%
\pgfpathlineto{\pgfqpoint{3.744436in}{0.717851in}}%
\pgfpathlineto{\pgfqpoint{3.748945in}{0.718865in}}%
\pgfpathlineto{\pgfqpoint{3.753455in}{0.716881in}}%
\pgfpathlineto{\pgfqpoint{3.762473in}{0.716319in}}%
\pgfpathlineto{\pgfqpoint{3.771491in}{0.723247in}}%
\pgfpathlineto{\pgfqpoint{3.776000in}{0.723239in}}%
\pgfpathlineto{\pgfqpoint{3.780509in}{0.718466in}}%
\pgfpathlineto{\pgfqpoint{3.785018in}{0.728169in}}%
\pgfpathlineto{\pgfqpoint{3.789527in}{0.754416in}}%
\pgfpathlineto{\pgfqpoint{3.794036in}{0.723789in}}%
\pgfpathlineto{\pgfqpoint{3.798545in}{0.729571in}}%
\pgfpathlineto{\pgfqpoint{3.803055in}{0.730533in}}%
\pgfpathlineto{\pgfqpoint{3.807564in}{0.729506in}}%
\pgfpathlineto{\pgfqpoint{3.812073in}{0.726078in}}%
\pgfpathlineto{\pgfqpoint{3.816582in}{0.742546in}}%
\pgfpathlineto{\pgfqpoint{3.821091in}{0.719203in}}%
\pgfpathlineto{\pgfqpoint{3.825600in}{0.730283in}}%
\pgfpathlineto{\pgfqpoint{3.830109in}{0.733527in}}%
\pgfpathlineto{\pgfqpoint{3.834618in}{0.717367in}}%
\pgfpathlineto{\pgfqpoint{3.839127in}{0.719175in}}%
\pgfpathlineto{\pgfqpoint{3.843636in}{0.739928in}}%
\pgfpathlineto{\pgfqpoint{3.848145in}{0.721988in}}%
\pgfpathlineto{\pgfqpoint{3.852655in}{0.719873in}}%
\pgfpathlineto{\pgfqpoint{3.857164in}{0.720784in}}%
\pgfpathlineto{\pgfqpoint{3.866182in}{0.719916in}}%
\pgfpathlineto{\pgfqpoint{3.870691in}{0.733218in}}%
\pgfpathlineto{\pgfqpoint{3.875200in}{0.723285in}}%
\pgfpathlineto{\pgfqpoint{3.884218in}{0.718614in}}%
\pgfpathlineto{\pgfqpoint{3.893236in}{0.724415in}}%
\pgfpathlineto{\pgfqpoint{3.897745in}{0.718333in}}%
\pgfpathlineto{\pgfqpoint{3.902255in}{0.721729in}}%
\pgfpathlineto{\pgfqpoint{3.906764in}{0.727972in}}%
\pgfpathlineto{\pgfqpoint{3.915782in}{0.718423in}}%
\pgfpathlineto{\pgfqpoint{3.920291in}{0.722896in}}%
\pgfpathlineto{\pgfqpoint{3.924800in}{0.722137in}}%
\pgfpathlineto{\pgfqpoint{3.929309in}{0.718823in}}%
\pgfpathlineto{\pgfqpoint{3.933818in}{0.720854in}}%
\pgfpathlineto{\pgfqpoint{3.938327in}{0.724735in}}%
\pgfpathlineto{\pgfqpoint{3.942836in}{0.719035in}}%
\pgfpathlineto{\pgfqpoint{3.947345in}{0.725718in}}%
\pgfpathlineto{\pgfqpoint{3.951855in}{0.720661in}}%
\pgfpathlineto{\pgfqpoint{3.956364in}{0.719108in}}%
\pgfpathlineto{\pgfqpoint{3.960873in}{0.720678in}}%
\pgfpathlineto{\pgfqpoint{3.965382in}{0.725871in}}%
\pgfpathlineto{\pgfqpoint{3.969891in}{0.728464in}}%
\pgfpathlineto{\pgfqpoint{3.974400in}{0.719435in}}%
\pgfpathlineto{\pgfqpoint{3.978909in}{0.726002in}}%
\pgfpathlineto{\pgfqpoint{3.987927in}{0.720289in}}%
\pgfpathlineto{\pgfqpoint{3.992436in}{0.724254in}}%
\pgfpathlineto{\pgfqpoint{3.996945in}{0.721403in}}%
\pgfpathlineto{\pgfqpoint{4.005964in}{0.720406in}}%
\pgfpathlineto{\pgfqpoint{4.010473in}{0.721937in}}%
\pgfpathlineto{\pgfqpoint{4.014982in}{0.725329in}}%
\pgfpathlineto{\pgfqpoint{4.019491in}{0.720065in}}%
\pgfpathlineto{\pgfqpoint{4.024000in}{0.720105in}}%
\pgfpathlineto{\pgfqpoint{4.028509in}{0.721789in}}%
\pgfpathlineto{\pgfqpoint{4.033018in}{0.725950in}}%
\pgfpathlineto{\pgfqpoint{4.037527in}{0.737845in}}%
\pgfpathlineto{\pgfqpoint{4.042036in}{0.722408in}}%
\pgfpathlineto{\pgfqpoint{4.046545in}{0.723096in}}%
\pgfpathlineto{\pgfqpoint{4.051055in}{0.726083in}}%
\pgfpathlineto{\pgfqpoint{4.055564in}{0.720400in}}%
\pgfpathlineto{\pgfqpoint{4.060073in}{0.724874in}}%
\pgfpathlineto{\pgfqpoint{4.064582in}{0.727042in}}%
\pgfpathlineto{\pgfqpoint{4.069091in}{0.720392in}}%
\pgfpathlineto{\pgfqpoint{4.073600in}{0.726226in}}%
\pgfpathlineto{\pgfqpoint{4.078109in}{0.722361in}}%
\pgfpathlineto{\pgfqpoint{4.082618in}{0.723879in}}%
\pgfpathlineto{\pgfqpoint{4.087127in}{0.721276in}}%
\pgfpathlineto{\pgfqpoint{4.091636in}{0.724288in}}%
\pgfpathlineto{\pgfqpoint{4.096145in}{0.720790in}}%
\pgfpathlineto{\pgfqpoint{4.100655in}{0.721283in}}%
\pgfpathlineto{\pgfqpoint{4.109673in}{0.733141in}}%
\pgfpathlineto{\pgfqpoint{4.114182in}{0.745618in}}%
\pgfpathlineto{\pgfqpoint{4.118691in}{0.723908in}}%
\pgfpathlineto{\pgfqpoint{4.123200in}{0.726189in}}%
\pgfpathlineto{\pgfqpoint{4.132218in}{0.721199in}}%
\pgfpathlineto{\pgfqpoint{4.136727in}{0.724995in}}%
\pgfpathlineto{\pgfqpoint{4.141236in}{0.726796in}}%
\pgfpathlineto{\pgfqpoint{4.145745in}{0.719925in}}%
\pgfpathlineto{\pgfqpoint{4.150255in}{0.726735in}}%
\pgfpathlineto{\pgfqpoint{4.154764in}{0.725250in}}%
\pgfpathlineto{\pgfqpoint{4.159273in}{0.721844in}}%
\pgfpathlineto{\pgfqpoint{4.163782in}{0.722693in}}%
\pgfpathlineto{\pgfqpoint{4.168291in}{0.737548in}}%
\pgfpathlineto{\pgfqpoint{4.172800in}{0.721821in}}%
\pgfpathlineto{\pgfqpoint{4.177309in}{0.730002in}}%
\pgfpathlineto{\pgfqpoint{4.181818in}{0.722108in}}%
\pgfpathlineto{\pgfqpoint{4.186327in}{0.725357in}}%
\pgfpathlineto{\pgfqpoint{4.190836in}{0.723658in}}%
\pgfpathlineto{\pgfqpoint{4.195345in}{0.726459in}}%
\pgfpathlineto{\pgfqpoint{4.204364in}{0.724127in}}%
\pgfpathlineto{\pgfqpoint{4.208873in}{0.722265in}}%
\pgfpathlineto{\pgfqpoint{4.213382in}{0.727041in}}%
\pgfpathlineto{\pgfqpoint{4.217891in}{0.724510in}}%
\pgfpathlineto{\pgfqpoint{4.222400in}{0.737279in}}%
\pgfpathlineto{\pgfqpoint{4.226909in}{0.721292in}}%
\pgfpathlineto{\pgfqpoint{4.231418in}{0.724320in}}%
\pgfpathlineto{\pgfqpoint{4.235927in}{0.722963in}}%
\pgfpathlineto{\pgfqpoint{4.240436in}{0.725441in}}%
\pgfpathlineto{\pgfqpoint{4.244945in}{0.723010in}}%
\pgfpathlineto{\pgfqpoint{4.253964in}{0.725119in}}%
\pgfpathlineto{\pgfqpoint{4.258473in}{0.722161in}}%
\pgfpathlineto{\pgfqpoint{4.262982in}{0.726780in}}%
\pgfpathlineto{\pgfqpoint{4.267491in}{0.721448in}}%
\pgfpathlineto{\pgfqpoint{4.272000in}{0.726793in}}%
\pgfpathlineto{\pgfqpoint{4.276509in}{0.722793in}}%
\pgfpathlineto{\pgfqpoint{4.281018in}{0.721817in}}%
\pgfpathlineto{\pgfqpoint{4.294545in}{0.722342in}}%
\pgfpathlineto{\pgfqpoint{4.299055in}{0.724803in}}%
\pgfpathlineto{\pgfqpoint{4.303564in}{0.723261in}}%
\pgfpathlineto{\pgfqpoint{4.308073in}{0.727355in}}%
\pgfpathlineto{\pgfqpoint{4.312582in}{0.724428in}}%
\pgfpathlineto{\pgfqpoint{4.317091in}{0.727654in}}%
\pgfpathlineto{\pgfqpoint{4.321600in}{0.722947in}}%
\pgfpathlineto{\pgfqpoint{4.330618in}{0.731508in}}%
\pgfpathlineto{\pgfqpoint{4.335127in}{0.731461in}}%
\pgfpathlineto{\pgfqpoint{4.339636in}{0.726857in}}%
\pgfpathlineto{\pgfqpoint{4.344145in}{0.728414in}}%
\pgfpathlineto{\pgfqpoint{4.348655in}{0.724023in}}%
\pgfpathlineto{\pgfqpoint{4.353164in}{0.726606in}}%
\pgfpathlineto{\pgfqpoint{4.357673in}{0.722168in}}%
\pgfpathlineto{\pgfqpoint{4.366691in}{0.728917in}}%
\pgfpathlineto{\pgfqpoint{4.371200in}{0.724179in}}%
\pgfpathlineto{\pgfqpoint{4.375709in}{0.724335in}}%
\pgfpathlineto{\pgfqpoint{4.380218in}{0.727999in}}%
\pgfpathlineto{\pgfqpoint{4.384727in}{0.742976in}}%
\pgfpathlineto{\pgfqpoint{4.389236in}{0.728998in}}%
\pgfpathlineto{\pgfqpoint{4.393745in}{0.724810in}}%
\pgfpathlineto{\pgfqpoint{4.398255in}{0.726601in}}%
\pgfpathlineto{\pgfqpoint{4.402764in}{0.723575in}}%
\pgfpathlineto{\pgfqpoint{4.407273in}{0.726803in}}%
\pgfpathlineto{\pgfqpoint{4.411782in}{0.724093in}}%
\pgfpathlineto{\pgfqpoint{4.416291in}{0.727491in}}%
\pgfpathlineto{\pgfqpoint{4.420800in}{0.726978in}}%
\pgfpathlineto{\pgfqpoint{4.425309in}{0.724500in}}%
\pgfpathlineto{\pgfqpoint{4.429818in}{0.723796in}}%
\pgfpathlineto{\pgfqpoint{4.434327in}{0.731349in}}%
\pgfpathlineto{\pgfqpoint{4.438836in}{0.730486in}}%
\pgfpathlineto{\pgfqpoint{4.443345in}{0.725521in}}%
\pgfpathlineto{\pgfqpoint{4.447855in}{0.725897in}}%
\pgfpathlineto{\pgfqpoint{4.452364in}{0.729617in}}%
\pgfpathlineto{\pgfqpoint{4.461382in}{0.726338in}}%
\pgfpathlineto{\pgfqpoint{4.470400in}{0.729342in}}%
\pgfpathlineto{\pgfqpoint{4.474909in}{0.723377in}}%
\pgfpathlineto{\pgfqpoint{4.483927in}{0.727353in}}%
\pgfpathlineto{\pgfqpoint{4.488436in}{0.727040in}}%
\pgfpathlineto{\pgfqpoint{4.492945in}{0.733136in}}%
\pgfpathlineto{\pgfqpoint{4.497455in}{0.723805in}}%
\pgfpathlineto{\pgfqpoint{4.506473in}{0.726474in}}%
\pgfpathlineto{\pgfqpoint{4.510982in}{0.723988in}}%
\pgfpathlineto{\pgfqpoint{4.524509in}{0.728581in}}%
\pgfpathlineto{\pgfqpoint{4.538036in}{0.730671in}}%
\pgfpathlineto{\pgfqpoint{4.542545in}{0.732677in}}%
\pgfpathlineto{\pgfqpoint{4.556073in}{0.724382in}}%
\pgfpathlineto{\pgfqpoint{4.560582in}{0.728406in}}%
\pgfpathlineto{\pgfqpoint{4.565091in}{0.735899in}}%
\pgfpathlineto{\pgfqpoint{4.569600in}{0.731102in}}%
\pgfpathlineto{\pgfqpoint{4.574109in}{0.730127in}}%
\pgfpathlineto{\pgfqpoint{4.578618in}{0.727503in}}%
\pgfpathlineto{\pgfqpoint{4.583127in}{0.723568in}}%
\pgfpathlineto{\pgfqpoint{4.587636in}{0.726253in}}%
\pgfpathlineto{\pgfqpoint{4.592145in}{0.724630in}}%
\pgfpathlineto{\pgfqpoint{4.596655in}{0.725957in}}%
\pgfpathlineto{\pgfqpoint{4.601164in}{0.723965in}}%
\pgfpathlineto{\pgfqpoint{4.605673in}{0.734934in}}%
\pgfpathlineto{\pgfqpoint{4.610182in}{0.738660in}}%
\pgfpathlineto{\pgfqpoint{4.614691in}{0.726200in}}%
\pgfpathlineto{\pgfqpoint{4.619200in}{0.736311in}}%
\pgfpathlineto{\pgfqpoint{4.623709in}{0.732210in}}%
\pgfpathlineto{\pgfqpoint{4.628218in}{0.774149in}}%
\pgfpathlineto{\pgfqpoint{4.632727in}{0.725361in}}%
\pgfpathlineto{\pgfqpoint{4.637236in}{0.725649in}}%
\pgfpathlineto{\pgfqpoint{4.641745in}{0.727974in}}%
\pgfpathlineto{\pgfqpoint{4.646255in}{0.728335in}}%
\pgfpathlineto{\pgfqpoint{4.650764in}{0.727146in}}%
\pgfpathlineto{\pgfqpoint{4.655273in}{0.736052in}}%
\pgfpathlineto{\pgfqpoint{4.659782in}{0.730145in}}%
\pgfpathlineto{\pgfqpoint{4.664291in}{0.751308in}}%
\pgfpathlineto{\pgfqpoint{4.668800in}{0.782042in}}%
\pgfpathlineto{\pgfqpoint{4.673309in}{0.731255in}}%
\pgfpathlineto{\pgfqpoint{4.677818in}{0.743494in}}%
\pgfpathlineto{\pgfqpoint{4.682327in}{0.724868in}}%
\pgfpathlineto{\pgfqpoint{4.686836in}{0.730679in}}%
\pgfpathlineto{\pgfqpoint{4.691345in}{0.725659in}}%
\pgfpathlineto{\pgfqpoint{4.695855in}{0.732960in}}%
\pgfpathlineto{\pgfqpoint{4.700364in}{0.728098in}}%
\pgfpathlineto{\pgfqpoint{4.704873in}{0.733388in}}%
\pgfpathlineto{\pgfqpoint{4.709382in}{0.777052in}}%
\pgfpathlineto{\pgfqpoint{4.713891in}{0.738470in}}%
\pgfpathlineto{\pgfqpoint{4.718400in}{0.735672in}}%
\pgfpathlineto{\pgfqpoint{4.722909in}{0.750149in}}%
\pgfpathlineto{\pgfqpoint{4.727418in}{0.733650in}}%
\pgfpathlineto{\pgfqpoint{4.731927in}{0.736199in}}%
\pgfpathlineto{\pgfqpoint{4.736436in}{0.734635in}}%
\pgfpathlineto{\pgfqpoint{4.740945in}{0.736563in}}%
\pgfpathlineto{\pgfqpoint{4.745455in}{0.749337in}}%
\pgfpathlineto{\pgfqpoint{4.749964in}{0.730944in}}%
\pgfpathlineto{\pgfqpoint{4.754473in}{0.747342in}}%
\pgfpathlineto{\pgfqpoint{4.758982in}{0.724119in}}%
\pgfpathlineto{\pgfqpoint{4.763491in}{0.726667in}}%
\pgfpathlineto{\pgfqpoint{4.777018in}{0.724728in}}%
\pgfpathlineto{\pgfqpoint{4.781527in}{0.727881in}}%
\pgfpathlineto{\pgfqpoint{4.786036in}{0.724737in}}%
\pgfpathlineto{\pgfqpoint{4.790545in}{0.727029in}}%
\pgfpathlineto{\pgfqpoint{4.799564in}{0.725241in}}%
\pgfpathlineto{\pgfqpoint{4.804073in}{0.762468in}}%
\pgfpathlineto{\pgfqpoint{4.808582in}{0.726811in}}%
\pgfpathlineto{\pgfqpoint{4.817600in}{0.725132in}}%
\pgfpathlineto{\pgfqpoint{4.822109in}{0.729274in}}%
\pgfpathlineto{\pgfqpoint{4.826618in}{0.760512in}}%
\pgfpathlineto{\pgfqpoint{4.831127in}{0.729045in}}%
\pgfpathlineto{\pgfqpoint{4.840145in}{0.730926in}}%
\pgfpathlineto{\pgfqpoint{4.844655in}{0.746710in}}%
\pgfpathlineto{\pgfqpoint{4.849164in}{0.731129in}}%
\pgfpathlineto{\pgfqpoint{4.853673in}{0.731330in}}%
\pgfpathlineto{\pgfqpoint{4.858182in}{0.733059in}}%
\pgfpathlineto{\pgfqpoint{4.862691in}{0.739205in}}%
\pgfpathlineto{\pgfqpoint{4.867200in}{0.764483in}}%
\pgfpathlineto{\pgfqpoint{4.876218in}{0.731545in}}%
\pgfpathlineto{\pgfqpoint{4.880727in}{0.729987in}}%
\pgfpathlineto{\pgfqpoint{4.885236in}{0.740211in}}%
\pgfpathlineto{\pgfqpoint{4.889745in}{0.727003in}}%
\pgfpathlineto{\pgfqpoint{4.894255in}{0.731838in}}%
\pgfpathlineto{\pgfqpoint{4.903273in}{0.727997in}}%
\pgfpathlineto{\pgfqpoint{4.907782in}{0.730308in}}%
\pgfpathlineto{\pgfqpoint{4.912291in}{0.730928in}}%
\pgfpathlineto{\pgfqpoint{4.916800in}{0.734941in}}%
\pgfpathlineto{\pgfqpoint{4.921309in}{0.727189in}}%
\pgfpathlineto{\pgfqpoint{4.925818in}{0.727798in}}%
\pgfpathlineto{\pgfqpoint{4.930327in}{0.731784in}}%
\pgfpathlineto{\pgfqpoint{4.934836in}{0.727642in}}%
\pgfpathlineto{\pgfqpoint{4.939345in}{0.730854in}}%
\pgfpathlineto{\pgfqpoint{4.943855in}{0.731308in}}%
\pgfpathlineto{\pgfqpoint{4.948364in}{0.733388in}}%
\pgfpathlineto{\pgfqpoint{4.952873in}{0.728207in}}%
\pgfpathlineto{\pgfqpoint{4.957382in}{0.728005in}}%
\pgfpathlineto{\pgfqpoint{4.966400in}{0.744557in}}%
\pgfpathlineto{\pgfqpoint{4.970909in}{0.729103in}}%
\pgfpathlineto{\pgfqpoint{4.975418in}{0.728634in}}%
\pgfpathlineto{\pgfqpoint{4.979927in}{0.736650in}}%
\pgfpathlineto{\pgfqpoint{4.984436in}{0.728877in}}%
\pgfpathlineto{\pgfqpoint{4.993455in}{0.730932in}}%
\pgfpathlineto{\pgfqpoint{4.997964in}{0.728273in}}%
\pgfpathlineto{\pgfqpoint{5.002473in}{0.733609in}}%
\pgfpathlineto{\pgfqpoint{5.006982in}{0.728790in}}%
\pgfpathlineto{\pgfqpoint{5.011491in}{0.731012in}}%
\pgfpathlineto{\pgfqpoint{5.016000in}{0.729945in}}%
\pgfpathlineto{\pgfqpoint{5.020509in}{0.738711in}}%
\pgfpathlineto{\pgfqpoint{5.025018in}{0.740837in}}%
\pgfpathlineto{\pgfqpoint{5.029527in}{0.727255in}}%
\pgfpathlineto{\pgfqpoint{5.034036in}{0.731491in}}%
\pgfpathlineto{\pgfqpoint{5.043055in}{0.731814in}}%
\pgfpathlineto{\pgfqpoint{5.047564in}{0.731727in}}%
\pgfpathlineto{\pgfqpoint{5.052073in}{0.730402in}}%
\pgfpathlineto{\pgfqpoint{5.056582in}{0.739673in}}%
\pgfpathlineto{\pgfqpoint{5.061091in}{0.739538in}}%
\pgfpathlineto{\pgfqpoint{5.065600in}{0.732432in}}%
\pgfpathlineto{\pgfqpoint{5.070109in}{0.732976in}}%
\pgfpathlineto{\pgfqpoint{5.074618in}{0.739094in}}%
\pgfpathlineto{\pgfqpoint{5.079127in}{0.732579in}}%
\pgfpathlineto{\pgfqpoint{5.083636in}{0.745240in}}%
\pgfpathlineto{\pgfqpoint{5.088145in}{0.732926in}}%
\pgfpathlineto{\pgfqpoint{5.092655in}{0.732075in}}%
\pgfpathlineto{\pgfqpoint{5.097164in}{0.746100in}}%
\pgfpathlineto{\pgfqpoint{5.101673in}{0.740278in}}%
\pgfpathlineto{\pgfqpoint{5.106182in}{0.729626in}}%
\pgfpathlineto{\pgfqpoint{5.110691in}{0.729213in}}%
\pgfpathlineto{\pgfqpoint{5.115200in}{0.733463in}}%
\pgfpathlineto{\pgfqpoint{5.124218in}{0.728836in}}%
\pgfpathlineto{\pgfqpoint{5.128727in}{0.738738in}}%
\pgfpathlineto{\pgfqpoint{5.133236in}{0.728974in}}%
\pgfpathlineto{\pgfqpoint{5.142255in}{0.732658in}}%
\pgfpathlineto{\pgfqpoint{5.146764in}{0.743660in}}%
\pgfpathlineto{\pgfqpoint{5.151273in}{0.742767in}}%
\pgfpathlineto{\pgfqpoint{5.155782in}{0.782190in}}%
\pgfpathlineto{\pgfqpoint{5.160291in}{0.746517in}}%
\pgfpathlineto{\pgfqpoint{5.164800in}{0.745707in}}%
\pgfpathlineto{\pgfqpoint{5.169309in}{0.734679in}}%
\pgfpathlineto{\pgfqpoint{5.178327in}{0.742022in}}%
\pgfpathlineto{\pgfqpoint{5.182836in}{0.739090in}}%
\pgfpathlineto{\pgfqpoint{5.187345in}{0.738050in}}%
\pgfpathlineto{\pgfqpoint{5.191855in}{0.742399in}}%
\pgfpathlineto{\pgfqpoint{5.196364in}{0.745033in}}%
\pgfpathlineto{\pgfqpoint{5.200873in}{0.730238in}}%
\pgfpathlineto{\pgfqpoint{5.209891in}{0.738674in}}%
\pgfpathlineto{\pgfqpoint{5.214400in}{0.735454in}}%
\pgfpathlineto{\pgfqpoint{5.218909in}{0.730381in}}%
\pgfpathlineto{\pgfqpoint{5.223418in}{0.740016in}}%
\pgfpathlineto{\pgfqpoint{5.227927in}{0.732994in}}%
\pgfpathlineto{\pgfqpoint{5.232436in}{0.737400in}}%
\pgfpathlineto{\pgfqpoint{5.236945in}{0.732867in}}%
\pgfpathlineto{\pgfqpoint{5.245964in}{0.758963in}}%
\pgfpathlineto{\pgfqpoint{5.250473in}{0.747958in}}%
\pgfpathlineto{\pgfqpoint{5.254982in}{0.731330in}}%
\pgfpathlineto{\pgfqpoint{5.259491in}{0.795252in}}%
\pgfpathlineto{\pgfqpoint{5.264000in}{0.747516in}}%
\pgfpathlineto{\pgfqpoint{5.268509in}{0.749757in}}%
\pgfpathlineto{\pgfqpoint{5.273018in}{0.731841in}}%
\pgfpathlineto{\pgfqpoint{5.277527in}{0.732261in}}%
\pgfpathlineto{\pgfqpoint{5.282036in}{0.746383in}}%
\pgfpathlineto{\pgfqpoint{5.286545in}{0.741182in}}%
\pgfpathlineto{\pgfqpoint{5.291055in}{0.748127in}}%
\pgfpathlineto{\pgfqpoint{5.295564in}{0.730129in}}%
\pgfpathlineto{\pgfqpoint{5.304582in}{0.732086in}}%
\pgfpathlineto{\pgfqpoint{5.309091in}{0.741256in}}%
\pgfpathlineto{\pgfqpoint{5.313600in}{0.755401in}}%
\pgfpathlineto{\pgfqpoint{5.318109in}{0.736103in}}%
\pgfpathlineto{\pgfqpoint{5.322618in}{0.738546in}}%
\pgfpathlineto{\pgfqpoint{5.331636in}{0.735132in}}%
\pgfpathlineto{\pgfqpoint{5.336145in}{0.737326in}}%
\pgfpathlineto{\pgfqpoint{5.340655in}{0.743354in}}%
\pgfpathlineto{\pgfqpoint{5.345164in}{0.742641in}}%
\pgfpathlineto{\pgfqpoint{5.349673in}{0.739285in}}%
\pgfpathlineto{\pgfqpoint{5.354182in}{0.737422in}}%
\pgfpathlineto{\pgfqpoint{5.358691in}{0.759932in}}%
\pgfpathlineto{\pgfqpoint{5.363200in}{0.731788in}}%
\pgfpathlineto{\pgfqpoint{5.367709in}{0.777786in}}%
\pgfpathlineto{\pgfqpoint{5.372218in}{0.740162in}}%
\pgfpathlineto{\pgfqpoint{5.376727in}{0.733168in}}%
\pgfpathlineto{\pgfqpoint{5.381236in}{0.733532in}}%
\pgfpathlineto{\pgfqpoint{5.385745in}{0.737926in}}%
\pgfpathlineto{\pgfqpoint{5.390255in}{0.731157in}}%
\pgfpathlineto{\pgfqpoint{5.394764in}{0.736301in}}%
\pgfpathlineto{\pgfqpoint{5.399273in}{0.732791in}}%
\pgfpathlineto{\pgfqpoint{5.403782in}{0.738412in}}%
\pgfpathlineto{\pgfqpoint{5.408291in}{0.735663in}}%
\pgfpathlineto{\pgfqpoint{5.412800in}{0.735864in}}%
\pgfpathlineto{\pgfqpoint{5.417309in}{0.746790in}}%
\pgfpathlineto{\pgfqpoint{5.421818in}{0.731681in}}%
\pgfpathlineto{\pgfqpoint{5.426327in}{0.733104in}}%
\pgfpathlineto{\pgfqpoint{5.430836in}{0.750726in}}%
\pgfpathlineto{\pgfqpoint{5.435345in}{0.733689in}}%
\pgfpathlineto{\pgfqpoint{5.439855in}{0.736229in}}%
\pgfpathlineto{\pgfqpoint{5.444364in}{0.732832in}}%
\pgfpathlineto{\pgfqpoint{5.448873in}{0.742462in}}%
\pgfpathlineto{\pgfqpoint{5.453382in}{0.741105in}}%
\pgfpathlineto{\pgfqpoint{5.457891in}{0.745506in}}%
\pgfpathlineto{\pgfqpoint{5.462400in}{0.732824in}}%
\pgfpathlineto{\pgfqpoint{5.466909in}{0.734154in}}%
\pgfpathlineto{\pgfqpoint{5.471418in}{0.737085in}}%
\pgfpathlineto{\pgfqpoint{5.475927in}{0.744307in}}%
\pgfpathlineto{\pgfqpoint{5.480436in}{0.732969in}}%
\pgfpathlineto{\pgfqpoint{5.484945in}{0.743189in}}%
\pgfpathlineto{\pgfqpoint{5.489455in}{0.742083in}}%
\pgfpathlineto{\pgfqpoint{5.493964in}{0.739746in}}%
\pgfpathlineto{\pgfqpoint{5.498473in}{0.734663in}}%
\pgfpathlineto{\pgfqpoint{5.502982in}{0.739805in}}%
\pgfpathlineto{\pgfqpoint{5.507491in}{0.733726in}}%
\pgfpathlineto{\pgfqpoint{5.512000in}{0.732698in}}%
\pgfpathlineto{\pgfqpoint{5.516509in}{0.733066in}}%
\pgfpathlineto{\pgfqpoint{5.521018in}{0.732134in}}%
\pgfpathlineto{\pgfqpoint{5.525527in}{0.733955in}}%
\pgfpathlineto{\pgfqpoint{5.530036in}{0.732581in}}%
\pgfpathlineto{\pgfqpoint{5.534545in}{0.748178in}}%
\pgfpathlineto{\pgfqpoint{5.534545in}{0.748178in}}%
\pgfusepath{stroke}%
\end{pgfscope}%
\begin{pgfscope}%
\pgfsetrectcap%
\pgfsetmiterjoin%
\pgfsetlinewidth{0.803000pt}%
\definecolor{currentstroke}{rgb}{0.000000,0.000000,0.000000}%
\pgfsetstrokecolor{currentstroke}%
\pgfsetdash{}{0pt}%
\pgfpathmoveto{\pgfqpoint{0.800000in}{0.528000in}}%
\pgfpathlineto{\pgfqpoint{0.800000in}{4.224000in}}%
\pgfusepath{stroke}%
\end{pgfscope}%
\begin{pgfscope}%
\pgfsetrectcap%
\pgfsetmiterjoin%
\pgfsetlinewidth{0.803000pt}%
\definecolor{currentstroke}{rgb}{0.000000,0.000000,0.000000}%
\pgfsetstrokecolor{currentstroke}%
\pgfsetdash{}{0pt}%
\pgfpathmoveto{\pgfqpoint{5.760000in}{0.528000in}}%
\pgfpathlineto{\pgfqpoint{5.760000in}{4.224000in}}%
\pgfusepath{stroke}%
\end{pgfscope}%
\begin{pgfscope}%
\pgfsetrectcap%
\pgfsetmiterjoin%
\pgfsetlinewidth{0.803000pt}%
\definecolor{currentstroke}{rgb}{0.000000,0.000000,0.000000}%
\pgfsetstrokecolor{currentstroke}%
\pgfsetdash{}{0pt}%
\pgfpathmoveto{\pgfqpoint{0.800000in}{0.528000in}}%
\pgfpathlineto{\pgfqpoint{5.760000in}{0.528000in}}%
\pgfusepath{stroke}%
\end{pgfscope}%
\begin{pgfscope}%
\pgfsetrectcap%
\pgfsetmiterjoin%
\pgfsetlinewidth{0.803000pt}%
\definecolor{currentstroke}{rgb}{0.000000,0.000000,0.000000}%
\pgfsetstrokecolor{currentstroke}%
\pgfsetdash{}{0pt}%
\pgfpathmoveto{\pgfqpoint{0.800000in}{4.224000in}}%
\pgfpathlineto{\pgfqpoint{5.760000in}{4.224000in}}%
\pgfusepath{stroke}%
\end{pgfscope}%
\begin{pgfscope}%
\definecolor{textcolor}{rgb}{0.000000,0.000000,0.000000}%
\pgfsetstrokecolor{textcolor}%
\pgfsetfillcolor{textcolor}%
\pgftext[x=3.280000in,y=4.307333in,,base]{\color{textcolor}\ttfamily\fontsize{12.000000}{14.400000}\selectfont Time vs Input size}%
\end{pgfscope}%
\begin{pgfscope}%
\pgfsetbuttcap%
\pgfsetmiterjoin%
\definecolor{currentfill}{rgb}{1.000000,1.000000,1.000000}%
\pgfsetfillcolor{currentfill}%
\pgfsetfillopacity{0.800000}%
\pgfsetlinewidth{1.003750pt}%
\definecolor{currentstroke}{rgb}{0.800000,0.800000,0.800000}%
\pgfsetstrokecolor{currentstroke}%
\pgfsetstrokeopacity{0.800000}%
\pgfsetdash{}{0pt}%
\pgfpathmoveto{\pgfqpoint{0.897222in}{3.088923in}}%
\pgfpathlineto{\pgfqpoint{2.094230in}{3.088923in}}%
\pgfpathquadraticcurveto{\pgfqpoint{2.122008in}{3.088923in}}{\pgfqpoint{2.122008in}{3.116701in}}%
\pgfpathlineto{\pgfqpoint{2.122008in}{4.126778in}}%
\pgfpathquadraticcurveto{\pgfqpoint{2.122008in}{4.154556in}}{\pgfqpoint{2.094230in}{4.154556in}}%
\pgfpathlineto{\pgfqpoint{0.897222in}{4.154556in}}%
\pgfpathquadraticcurveto{\pgfqpoint{0.869444in}{4.154556in}}{\pgfqpoint{0.869444in}{4.126778in}}%
\pgfpathlineto{\pgfqpoint{0.869444in}{3.116701in}}%
\pgfpathquadraticcurveto{\pgfqpoint{0.869444in}{3.088923in}}{\pgfqpoint{0.897222in}{3.088923in}}%
\pgfpathlineto{\pgfqpoint{0.897222in}{3.088923in}}%
\pgfpathclose%
\pgfusepath{stroke,fill}%
\end{pgfscope}%
\begin{pgfscope}%
\pgfsetrectcap%
\pgfsetroundjoin%
\pgfsetlinewidth{1.505625pt}%
\definecolor{currentstroke}{rgb}{1.000000,0.000000,0.000000}%
\pgfsetstrokecolor{currentstroke}%
\pgfsetdash{}{0pt}%
\pgfpathmoveto{\pgfqpoint{0.925000in}{4.041342in}}%
\pgfpathlineto{\pgfqpoint{1.063889in}{4.041342in}}%
\pgfpathlineto{\pgfqpoint{1.202778in}{4.041342in}}%
\pgfusepath{stroke}%
\end{pgfscope}%
\begin{pgfscope}%
\definecolor{textcolor}{rgb}{0.000000,0.000000,0.000000}%
\pgfsetstrokecolor{textcolor}%
\pgfsetfillcolor{textcolor}%
\pgftext[x=1.313889in,y=3.992731in,left,base]{\color{textcolor}\ttfamily\fontsize{10.000000}{12.000000}\selectfont Bubble}%
\end{pgfscope}%
\begin{pgfscope}%
\pgfsetrectcap%
\pgfsetroundjoin%
\pgfsetlinewidth{1.505625pt}%
\definecolor{currentstroke}{rgb}{0.486275,0.988235,0.000000}%
\pgfsetstrokecolor{currentstroke}%
\pgfsetdash{}{0pt}%
\pgfpathmoveto{\pgfqpoint{0.925000in}{3.836739in}}%
\pgfpathlineto{\pgfqpoint{1.063889in}{3.836739in}}%
\pgfpathlineto{\pgfqpoint{1.202778in}{3.836739in}}%
\pgfusepath{stroke}%
\end{pgfscope}%
\begin{pgfscope}%
\definecolor{textcolor}{rgb}{0.000000,0.000000,0.000000}%
\pgfsetstrokecolor{textcolor}%
\pgfsetfillcolor{textcolor}%
\pgftext[x=1.313889in,y=3.788128in,left,base]{\color{textcolor}\ttfamily\fontsize{10.000000}{12.000000}\selectfont Selection}%
\end{pgfscope}%
\begin{pgfscope}%
\pgfsetrectcap%
\pgfsetroundjoin%
\pgfsetlinewidth{1.505625pt}%
\definecolor{currentstroke}{rgb}{0.000000,1.000000,0.498039}%
\pgfsetstrokecolor{currentstroke}%
\pgfsetdash{}{0pt}%
\pgfpathmoveto{\pgfqpoint{0.925000in}{3.632136in}}%
\pgfpathlineto{\pgfqpoint{1.063889in}{3.632136in}}%
\pgfpathlineto{\pgfqpoint{1.202778in}{3.632136in}}%
\pgfusepath{stroke}%
\end{pgfscope}%
\begin{pgfscope}%
\definecolor{textcolor}{rgb}{0.000000,0.000000,0.000000}%
\pgfsetstrokecolor{textcolor}%
\pgfsetfillcolor{textcolor}%
\pgftext[x=1.313889in,y=3.583525in,left,base]{\color{textcolor}\ttfamily\fontsize{10.000000}{12.000000}\selectfont Insertion}%
\end{pgfscope}%
\begin{pgfscope}%
\pgfsetrectcap%
\pgfsetroundjoin%
\pgfsetlinewidth{1.505625pt}%
\definecolor{currentstroke}{rgb}{0.000000,1.000000,1.000000}%
\pgfsetstrokecolor{currentstroke}%
\pgfsetdash{}{0pt}%
\pgfpathmoveto{\pgfqpoint{0.925000in}{3.427532in}}%
\pgfpathlineto{\pgfqpoint{1.063889in}{3.427532in}}%
\pgfpathlineto{\pgfqpoint{1.202778in}{3.427532in}}%
\pgfusepath{stroke}%
\end{pgfscope}%
\begin{pgfscope}%
\definecolor{textcolor}{rgb}{0.000000,0.000000,0.000000}%
\pgfsetstrokecolor{textcolor}%
\pgfsetfillcolor{textcolor}%
\pgftext[x=1.313889in,y=3.378921in,left,base]{\color{textcolor}\ttfamily\fontsize{10.000000}{12.000000}\selectfont Merge}%
\end{pgfscope}%
\begin{pgfscope}%
\pgfsetrectcap%
\pgfsetroundjoin%
\pgfsetlinewidth{1.505625pt}%
\definecolor{currentstroke}{rgb}{1.000000,0.000000,1.000000}%
\pgfsetstrokecolor{currentstroke}%
\pgfsetdash{}{0pt}%
\pgfpathmoveto{\pgfqpoint{0.925000in}{3.221980in}}%
\pgfpathlineto{\pgfqpoint{1.063889in}{3.221980in}}%
\pgfpathlineto{\pgfqpoint{1.202778in}{3.221980in}}%
\pgfusepath{stroke}%
\end{pgfscope}%
\begin{pgfscope}%
\definecolor{textcolor}{rgb}{0.000000,0.000000,0.000000}%
\pgfsetstrokecolor{textcolor}%
\pgfsetfillcolor{textcolor}%
\pgftext[x=1.313889in,y=3.173369in,left,base]{\color{textcolor}\ttfamily\fontsize{10.000000}{12.000000}\selectfont Quick}%
\end{pgfscope}%
\end{pgfpicture}%
\makeatother%
\endgroup%

%% Creator: Matplotlib, PGF backend
%%
%% To include the figure in your LaTeX document, write
%%   \input{<filename>.pgf}
%%
%% Make sure the required packages are loaded in your preamble
%%   \usepackage{pgf}
%%
%% Also ensure that all the required font packages are loaded; for instance,
%% the lmodern package is sometimes necessary when using math font.
%%   \usepackage{lmodern}
%%
%% Figures using additional raster images can only be included by \input if
%% they are in the same directory as the main LaTeX file. For loading figures
%% from other directories you can use the `import` package
%%   \usepackage{import}
%%
%% and then include the figures with
%%   \import{<path to file>}{<filename>.pgf}
%%
%% Matplotlib used the following preamble
%%   \usepackage{fontspec}
%%   \setmainfont{DejaVuSerif.ttf}[Path=\detokenize{/home/dbk/.local/lib/python3.10/site-packages/matplotlib/mpl-data/fonts/ttf/}]
%%   \setsansfont{DejaVuSans.ttf}[Path=\detokenize{/home/dbk/.local/lib/python3.10/site-packages/matplotlib/mpl-data/fonts/ttf/}]
%%   \setmonofont{DejaVuSansMono.ttf}[Path=\detokenize{/home/dbk/.local/lib/python3.10/site-packages/matplotlib/mpl-data/fonts/ttf/}]
%%
\begingroup%
\makeatletter%
\begin{pgfpicture}%
\pgfpathrectangle{\pgfpointorigin}{\pgfqpoint{6.400000in}{4.800000in}}%
\pgfusepath{use as bounding box, clip}%
\begin{pgfscope}%
\pgfsetbuttcap%
\pgfsetmiterjoin%
\definecolor{currentfill}{rgb}{1.000000,1.000000,1.000000}%
\pgfsetfillcolor{currentfill}%
\pgfsetlinewidth{0.000000pt}%
\definecolor{currentstroke}{rgb}{1.000000,1.000000,1.000000}%
\pgfsetstrokecolor{currentstroke}%
\pgfsetdash{}{0pt}%
\pgfpathmoveto{\pgfqpoint{0.000000in}{0.000000in}}%
\pgfpathlineto{\pgfqpoint{6.400000in}{0.000000in}}%
\pgfpathlineto{\pgfqpoint{6.400000in}{4.800000in}}%
\pgfpathlineto{\pgfqpoint{0.000000in}{4.800000in}}%
\pgfpathlineto{\pgfqpoint{0.000000in}{0.000000in}}%
\pgfpathclose%
\pgfusepath{fill}%
\end{pgfscope}%
\begin{pgfscope}%
\pgfsetbuttcap%
\pgfsetmiterjoin%
\definecolor{currentfill}{rgb}{1.000000,1.000000,1.000000}%
\pgfsetfillcolor{currentfill}%
\pgfsetlinewidth{0.000000pt}%
\definecolor{currentstroke}{rgb}{0.000000,0.000000,0.000000}%
\pgfsetstrokecolor{currentstroke}%
\pgfsetstrokeopacity{0.000000}%
\pgfsetdash{}{0pt}%
\pgfpathmoveto{\pgfqpoint{0.800000in}{0.528000in}}%
\pgfpathlineto{\pgfqpoint{5.760000in}{0.528000in}}%
\pgfpathlineto{\pgfqpoint{5.760000in}{4.224000in}}%
\pgfpathlineto{\pgfqpoint{0.800000in}{4.224000in}}%
\pgfpathlineto{\pgfqpoint{0.800000in}{0.528000in}}%
\pgfpathclose%
\pgfusepath{fill}%
\end{pgfscope}%
\begin{pgfscope}%
\pgfsetbuttcap%
\pgfsetroundjoin%
\definecolor{currentfill}{rgb}{0.000000,0.000000,0.000000}%
\pgfsetfillcolor{currentfill}%
\pgfsetlinewidth{0.803000pt}%
\definecolor{currentstroke}{rgb}{0.000000,0.000000,0.000000}%
\pgfsetstrokecolor{currentstroke}%
\pgfsetdash{}{0pt}%
\pgfsys@defobject{currentmarker}{\pgfqpoint{0.000000in}{-0.048611in}}{\pgfqpoint{0.000000in}{0.000000in}}{%
\pgfpathmoveto{\pgfqpoint{0.000000in}{0.000000in}}%
\pgfpathlineto{\pgfqpoint{0.000000in}{-0.048611in}}%
\pgfusepath{stroke,fill}%
}%
\begin{pgfscope}%
\pgfsys@transformshift{1.020945in}{0.528000in}%
\pgfsys@useobject{currentmarker}{}%
\end{pgfscope}%
\end{pgfscope}%
\begin{pgfscope}%
\definecolor{textcolor}{rgb}{0.000000,0.000000,0.000000}%
\pgfsetstrokecolor{textcolor}%
\pgfsetfillcolor{textcolor}%
\pgftext[x=1.020945in,y=0.430778in,,top]{\color{textcolor}\ttfamily\fontsize{10.000000}{12.000000}\selectfont 0}%
\end{pgfscope}%
\begin{pgfscope}%
\pgfsetbuttcap%
\pgfsetroundjoin%
\definecolor{currentfill}{rgb}{0.000000,0.000000,0.000000}%
\pgfsetfillcolor{currentfill}%
\pgfsetlinewidth{0.803000pt}%
\definecolor{currentstroke}{rgb}{0.000000,0.000000,0.000000}%
\pgfsetstrokecolor{currentstroke}%
\pgfsetdash{}{0pt}%
\pgfsys@defobject{currentmarker}{\pgfqpoint{0.000000in}{-0.048611in}}{\pgfqpoint{0.000000in}{0.000000in}}{%
\pgfpathmoveto{\pgfqpoint{0.000000in}{0.000000in}}%
\pgfpathlineto{\pgfqpoint{0.000000in}{-0.048611in}}%
\pgfusepath{stroke,fill}%
}%
\begin{pgfscope}%
\pgfsys@transformshift{1.922764in}{0.528000in}%
\pgfsys@useobject{currentmarker}{}%
\end{pgfscope}%
\end{pgfscope}%
\begin{pgfscope}%
\definecolor{textcolor}{rgb}{0.000000,0.000000,0.000000}%
\pgfsetstrokecolor{textcolor}%
\pgfsetfillcolor{textcolor}%
\pgftext[x=1.922764in,y=0.430778in,,top]{\color{textcolor}\ttfamily\fontsize{10.000000}{12.000000}\selectfont 200}%
\end{pgfscope}%
\begin{pgfscope}%
\pgfsetbuttcap%
\pgfsetroundjoin%
\definecolor{currentfill}{rgb}{0.000000,0.000000,0.000000}%
\pgfsetfillcolor{currentfill}%
\pgfsetlinewidth{0.803000pt}%
\definecolor{currentstroke}{rgb}{0.000000,0.000000,0.000000}%
\pgfsetstrokecolor{currentstroke}%
\pgfsetdash{}{0pt}%
\pgfsys@defobject{currentmarker}{\pgfqpoint{0.000000in}{-0.048611in}}{\pgfqpoint{0.000000in}{0.000000in}}{%
\pgfpathmoveto{\pgfqpoint{0.000000in}{0.000000in}}%
\pgfpathlineto{\pgfqpoint{0.000000in}{-0.048611in}}%
\pgfusepath{stroke,fill}%
}%
\begin{pgfscope}%
\pgfsys@transformshift{2.824582in}{0.528000in}%
\pgfsys@useobject{currentmarker}{}%
\end{pgfscope}%
\end{pgfscope}%
\begin{pgfscope}%
\definecolor{textcolor}{rgb}{0.000000,0.000000,0.000000}%
\pgfsetstrokecolor{textcolor}%
\pgfsetfillcolor{textcolor}%
\pgftext[x=2.824582in,y=0.430778in,,top]{\color{textcolor}\ttfamily\fontsize{10.000000}{12.000000}\selectfont 400}%
\end{pgfscope}%
\begin{pgfscope}%
\pgfsetbuttcap%
\pgfsetroundjoin%
\definecolor{currentfill}{rgb}{0.000000,0.000000,0.000000}%
\pgfsetfillcolor{currentfill}%
\pgfsetlinewidth{0.803000pt}%
\definecolor{currentstroke}{rgb}{0.000000,0.000000,0.000000}%
\pgfsetstrokecolor{currentstroke}%
\pgfsetdash{}{0pt}%
\pgfsys@defobject{currentmarker}{\pgfqpoint{0.000000in}{-0.048611in}}{\pgfqpoint{0.000000in}{0.000000in}}{%
\pgfpathmoveto{\pgfqpoint{0.000000in}{0.000000in}}%
\pgfpathlineto{\pgfqpoint{0.000000in}{-0.048611in}}%
\pgfusepath{stroke,fill}%
}%
\begin{pgfscope}%
\pgfsys@transformshift{3.726400in}{0.528000in}%
\pgfsys@useobject{currentmarker}{}%
\end{pgfscope}%
\end{pgfscope}%
\begin{pgfscope}%
\definecolor{textcolor}{rgb}{0.000000,0.000000,0.000000}%
\pgfsetstrokecolor{textcolor}%
\pgfsetfillcolor{textcolor}%
\pgftext[x=3.726400in,y=0.430778in,,top]{\color{textcolor}\ttfamily\fontsize{10.000000}{12.000000}\selectfont 600}%
\end{pgfscope}%
\begin{pgfscope}%
\pgfsetbuttcap%
\pgfsetroundjoin%
\definecolor{currentfill}{rgb}{0.000000,0.000000,0.000000}%
\pgfsetfillcolor{currentfill}%
\pgfsetlinewidth{0.803000pt}%
\definecolor{currentstroke}{rgb}{0.000000,0.000000,0.000000}%
\pgfsetstrokecolor{currentstroke}%
\pgfsetdash{}{0pt}%
\pgfsys@defobject{currentmarker}{\pgfqpoint{0.000000in}{-0.048611in}}{\pgfqpoint{0.000000in}{0.000000in}}{%
\pgfpathmoveto{\pgfqpoint{0.000000in}{0.000000in}}%
\pgfpathlineto{\pgfqpoint{0.000000in}{-0.048611in}}%
\pgfusepath{stroke,fill}%
}%
\begin{pgfscope}%
\pgfsys@transformshift{4.628218in}{0.528000in}%
\pgfsys@useobject{currentmarker}{}%
\end{pgfscope}%
\end{pgfscope}%
\begin{pgfscope}%
\definecolor{textcolor}{rgb}{0.000000,0.000000,0.000000}%
\pgfsetstrokecolor{textcolor}%
\pgfsetfillcolor{textcolor}%
\pgftext[x=4.628218in,y=0.430778in,,top]{\color{textcolor}\ttfamily\fontsize{10.000000}{12.000000}\selectfont 800}%
\end{pgfscope}%
\begin{pgfscope}%
\pgfsetbuttcap%
\pgfsetroundjoin%
\definecolor{currentfill}{rgb}{0.000000,0.000000,0.000000}%
\pgfsetfillcolor{currentfill}%
\pgfsetlinewidth{0.803000pt}%
\definecolor{currentstroke}{rgb}{0.000000,0.000000,0.000000}%
\pgfsetstrokecolor{currentstroke}%
\pgfsetdash{}{0pt}%
\pgfsys@defobject{currentmarker}{\pgfqpoint{0.000000in}{-0.048611in}}{\pgfqpoint{0.000000in}{0.000000in}}{%
\pgfpathmoveto{\pgfqpoint{0.000000in}{0.000000in}}%
\pgfpathlineto{\pgfqpoint{0.000000in}{-0.048611in}}%
\pgfusepath{stroke,fill}%
}%
\begin{pgfscope}%
\pgfsys@transformshift{5.530036in}{0.528000in}%
\pgfsys@useobject{currentmarker}{}%
\end{pgfscope}%
\end{pgfscope}%
\begin{pgfscope}%
\definecolor{textcolor}{rgb}{0.000000,0.000000,0.000000}%
\pgfsetstrokecolor{textcolor}%
\pgfsetfillcolor{textcolor}%
\pgftext[x=5.530036in,y=0.430778in,,top]{\color{textcolor}\ttfamily\fontsize{10.000000}{12.000000}\selectfont 1000}%
\end{pgfscope}%
\begin{pgfscope}%
\definecolor{textcolor}{rgb}{0.000000,0.000000,0.000000}%
\pgfsetstrokecolor{textcolor}%
\pgfsetfillcolor{textcolor}%
\pgftext[x=3.280000in,y=0.240063in,,top]{\color{textcolor}\ttfamily\fontsize{10.000000}{12.000000}\selectfont Size of Array}%
\end{pgfscope}%
\begin{pgfscope}%
\pgfsetbuttcap%
\pgfsetroundjoin%
\definecolor{currentfill}{rgb}{0.000000,0.000000,0.000000}%
\pgfsetfillcolor{currentfill}%
\pgfsetlinewidth{0.803000pt}%
\definecolor{currentstroke}{rgb}{0.000000,0.000000,0.000000}%
\pgfsetstrokecolor{currentstroke}%
\pgfsetdash{}{0pt}%
\pgfsys@defobject{currentmarker}{\pgfqpoint{-0.048611in}{0.000000in}}{\pgfqpoint{-0.000000in}{0.000000in}}{%
\pgfpathmoveto{\pgfqpoint{-0.000000in}{0.000000in}}%
\pgfpathlineto{\pgfqpoint{-0.048611in}{0.000000in}}%
\pgfusepath{stroke,fill}%
}%
\begin{pgfscope}%
\pgfsys@transformshift{0.800000in}{0.640394in}%
\pgfsys@useobject{currentmarker}{}%
\end{pgfscope}%
\end{pgfscope}%
\begin{pgfscope}%
\definecolor{textcolor}{rgb}{0.000000,0.000000,0.000000}%
\pgfsetstrokecolor{textcolor}%
\pgfsetfillcolor{textcolor}%
\pgftext[x=0.619160in, y=0.587260in, left, base]{\color{textcolor}\ttfamily\fontsize{10.000000}{12.000000}\selectfont 0}%
\end{pgfscope}%
\begin{pgfscope}%
\pgfsetbuttcap%
\pgfsetroundjoin%
\definecolor{currentfill}{rgb}{0.000000,0.000000,0.000000}%
\pgfsetfillcolor{currentfill}%
\pgfsetlinewidth{0.803000pt}%
\definecolor{currentstroke}{rgb}{0.000000,0.000000,0.000000}%
\pgfsetstrokecolor{currentstroke}%
\pgfsetdash{}{0pt}%
\pgfsys@defobject{currentmarker}{\pgfqpoint{-0.048611in}{0.000000in}}{\pgfqpoint{-0.000000in}{0.000000in}}{%
\pgfpathmoveto{\pgfqpoint{-0.000000in}{0.000000in}}%
\pgfpathlineto{\pgfqpoint{-0.048611in}{0.000000in}}%
\pgfusepath{stroke,fill}%
}%
\begin{pgfscope}%
\pgfsys@transformshift{0.800000in}{1.231944in}%
\pgfsys@useobject{currentmarker}{}%
\end{pgfscope}%
\end{pgfscope}%
\begin{pgfscope}%
\definecolor{textcolor}{rgb}{0.000000,0.000000,0.000000}%
\pgfsetstrokecolor{textcolor}%
\pgfsetfillcolor{textcolor}%
\pgftext[x=0.368305in, y=1.178809in, left, base]{\color{textcolor}\ttfamily\fontsize{10.000000}{12.000000}\selectfont 2000}%
\end{pgfscope}%
\begin{pgfscope}%
\pgfsetbuttcap%
\pgfsetroundjoin%
\definecolor{currentfill}{rgb}{0.000000,0.000000,0.000000}%
\pgfsetfillcolor{currentfill}%
\pgfsetlinewidth{0.803000pt}%
\definecolor{currentstroke}{rgb}{0.000000,0.000000,0.000000}%
\pgfsetstrokecolor{currentstroke}%
\pgfsetdash{}{0pt}%
\pgfsys@defobject{currentmarker}{\pgfqpoint{-0.048611in}{0.000000in}}{\pgfqpoint{-0.000000in}{0.000000in}}{%
\pgfpathmoveto{\pgfqpoint{-0.000000in}{0.000000in}}%
\pgfpathlineto{\pgfqpoint{-0.048611in}{0.000000in}}%
\pgfusepath{stroke,fill}%
}%
\begin{pgfscope}%
\pgfsys@transformshift{0.800000in}{1.823493in}%
\pgfsys@useobject{currentmarker}{}%
\end{pgfscope}%
\end{pgfscope}%
\begin{pgfscope}%
\definecolor{textcolor}{rgb}{0.000000,0.000000,0.000000}%
\pgfsetstrokecolor{textcolor}%
\pgfsetfillcolor{textcolor}%
\pgftext[x=0.368305in, y=1.770358in, left, base]{\color{textcolor}\ttfamily\fontsize{10.000000}{12.000000}\selectfont 4000}%
\end{pgfscope}%
\begin{pgfscope}%
\pgfsetbuttcap%
\pgfsetroundjoin%
\definecolor{currentfill}{rgb}{0.000000,0.000000,0.000000}%
\pgfsetfillcolor{currentfill}%
\pgfsetlinewidth{0.803000pt}%
\definecolor{currentstroke}{rgb}{0.000000,0.000000,0.000000}%
\pgfsetstrokecolor{currentstroke}%
\pgfsetdash{}{0pt}%
\pgfsys@defobject{currentmarker}{\pgfqpoint{-0.048611in}{0.000000in}}{\pgfqpoint{-0.000000in}{0.000000in}}{%
\pgfpathmoveto{\pgfqpoint{-0.000000in}{0.000000in}}%
\pgfpathlineto{\pgfqpoint{-0.048611in}{0.000000in}}%
\pgfusepath{stroke,fill}%
}%
\begin{pgfscope}%
\pgfsys@transformshift{0.800000in}{2.415042in}%
\pgfsys@useobject{currentmarker}{}%
\end{pgfscope}%
\end{pgfscope}%
\begin{pgfscope}%
\definecolor{textcolor}{rgb}{0.000000,0.000000,0.000000}%
\pgfsetstrokecolor{textcolor}%
\pgfsetfillcolor{textcolor}%
\pgftext[x=0.368305in, y=2.361908in, left, base]{\color{textcolor}\ttfamily\fontsize{10.000000}{12.000000}\selectfont 6000}%
\end{pgfscope}%
\begin{pgfscope}%
\pgfsetbuttcap%
\pgfsetroundjoin%
\definecolor{currentfill}{rgb}{0.000000,0.000000,0.000000}%
\pgfsetfillcolor{currentfill}%
\pgfsetlinewidth{0.803000pt}%
\definecolor{currentstroke}{rgb}{0.000000,0.000000,0.000000}%
\pgfsetstrokecolor{currentstroke}%
\pgfsetdash{}{0pt}%
\pgfsys@defobject{currentmarker}{\pgfqpoint{-0.048611in}{0.000000in}}{\pgfqpoint{-0.000000in}{0.000000in}}{%
\pgfpathmoveto{\pgfqpoint{-0.000000in}{0.000000in}}%
\pgfpathlineto{\pgfqpoint{-0.048611in}{0.000000in}}%
\pgfusepath{stroke,fill}%
}%
\begin{pgfscope}%
\pgfsys@transformshift{0.800000in}{3.006592in}%
\pgfsys@useobject{currentmarker}{}%
\end{pgfscope}%
\end{pgfscope}%
\begin{pgfscope}%
\definecolor{textcolor}{rgb}{0.000000,0.000000,0.000000}%
\pgfsetstrokecolor{textcolor}%
\pgfsetfillcolor{textcolor}%
\pgftext[x=0.368305in, y=2.953457in, left, base]{\color{textcolor}\ttfamily\fontsize{10.000000}{12.000000}\selectfont 8000}%
\end{pgfscope}%
\begin{pgfscope}%
\pgfsetbuttcap%
\pgfsetroundjoin%
\definecolor{currentfill}{rgb}{0.000000,0.000000,0.000000}%
\pgfsetfillcolor{currentfill}%
\pgfsetlinewidth{0.803000pt}%
\definecolor{currentstroke}{rgb}{0.000000,0.000000,0.000000}%
\pgfsetstrokecolor{currentstroke}%
\pgfsetdash{}{0pt}%
\pgfsys@defobject{currentmarker}{\pgfqpoint{-0.048611in}{0.000000in}}{\pgfqpoint{-0.000000in}{0.000000in}}{%
\pgfpathmoveto{\pgfqpoint{-0.000000in}{0.000000in}}%
\pgfpathlineto{\pgfqpoint{-0.048611in}{0.000000in}}%
\pgfusepath{stroke,fill}%
}%
\begin{pgfscope}%
\pgfsys@transformshift{0.800000in}{3.598141in}%
\pgfsys@useobject{currentmarker}{}%
\end{pgfscope}%
\end{pgfscope}%
\begin{pgfscope}%
\definecolor{textcolor}{rgb}{0.000000,0.000000,0.000000}%
\pgfsetstrokecolor{textcolor}%
\pgfsetfillcolor{textcolor}%
\pgftext[x=0.284687in, y=3.545006in, left, base]{\color{textcolor}\ttfamily\fontsize{10.000000}{12.000000}\selectfont 10000}%
\end{pgfscope}%
\begin{pgfscope}%
\pgfsetbuttcap%
\pgfsetroundjoin%
\definecolor{currentfill}{rgb}{0.000000,0.000000,0.000000}%
\pgfsetfillcolor{currentfill}%
\pgfsetlinewidth{0.803000pt}%
\definecolor{currentstroke}{rgb}{0.000000,0.000000,0.000000}%
\pgfsetstrokecolor{currentstroke}%
\pgfsetdash{}{0pt}%
\pgfsys@defobject{currentmarker}{\pgfqpoint{-0.048611in}{0.000000in}}{\pgfqpoint{-0.000000in}{0.000000in}}{%
\pgfpathmoveto{\pgfqpoint{-0.000000in}{0.000000in}}%
\pgfpathlineto{\pgfqpoint{-0.048611in}{0.000000in}}%
\pgfusepath{stroke,fill}%
}%
\begin{pgfscope}%
\pgfsys@transformshift{0.800000in}{4.189690in}%
\pgfsys@useobject{currentmarker}{}%
\end{pgfscope}%
\end{pgfscope}%
\begin{pgfscope}%
\definecolor{textcolor}{rgb}{0.000000,0.000000,0.000000}%
\pgfsetstrokecolor{textcolor}%
\pgfsetfillcolor{textcolor}%
\pgftext[x=0.284687in, y=4.136556in, left, base]{\color{textcolor}\ttfamily\fontsize{10.000000}{12.000000}\selectfont 12000}%
\end{pgfscope}%
\begin{pgfscope}%
\definecolor{textcolor}{rgb}{0.000000,0.000000,0.000000}%
\pgfsetstrokecolor{textcolor}%
\pgfsetfillcolor{textcolor}%
\pgftext[x=0.229131in,y=2.376000in,,bottom,rotate=90.000000]{\color{textcolor}\ttfamily\fontsize{10.000000}{12.000000}\selectfont Memory}%
\end{pgfscope}%
\begin{pgfscope}%
\pgfpathrectangle{\pgfqpoint{0.800000in}{0.528000in}}{\pgfqpoint{4.960000in}{3.696000in}}%
\pgfusepath{clip}%
\pgfsetrectcap%
\pgfsetroundjoin%
\pgfsetlinewidth{1.505625pt}%
\definecolor{currentstroke}{rgb}{1.000000,0.000000,0.000000}%
\pgfsetstrokecolor{currentstroke}%
\pgfsetdash{}{0pt}%
\pgfpathmoveto{\pgfqpoint{1.025455in}{0.724394in}}%
\pgfpathlineto{\pgfqpoint{1.733382in}{0.724394in}}%
\pgfpathlineto{\pgfqpoint{1.746909in}{0.749239in}}%
\pgfpathlineto{\pgfqpoint{2.892218in}{0.749239in}}%
\pgfpathlineto{\pgfqpoint{2.896727in}{0.757521in}}%
\pgfpathlineto{\pgfqpoint{5.534545in}{0.757521in}}%
\pgfpathlineto{\pgfqpoint{5.534545in}{0.757521in}}%
\pgfusepath{stroke}%
\end{pgfscope}%
\begin{pgfscope}%
\pgfpathrectangle{\pgfqpoint{0.800000in}{0.528000in}}{\pgfqpoint{4.960000in}{3.696000in}}%
\pgfusepath{clip}%
\pgfsetrectcap%
\pgfsetroundjoin%
\pgfsetlinewidth{1.505625pt}%
\definecolor{currentstroke}{rgb}{0.486275,0.988235,0.000000}%
\pgfsetstrokecolor{currentstroke}%
\pgfsetdash{}{0pt}%
\pgfpathmoveto{\pgfqpoint{1.025455in}{0.709014in}}%
\pgfpathlineto{\pgfqpoint{1.728873in}{0.709014in}}%
\pgfpathlineto{\pgfqpoint{1.733382in}{0.725577in}}%
\pgfpathlineto{\pgfqpoint{1.737891in}{0.750423in}}%
\pgfpathlineto{\pgfqpoint{2.892218in}{0.750423in}}%
\pgfpathlineto{\pgfqpoint{2.896727in}{0.758704in}}%
\pgfpathlineto{\pgfqpoint{5.534545in}{0.758704in}}%
\pgfpathlineto{\pgfqpoint{5.534545in}{0.758704in}}%
\pgfusepath{stroke}%
\end{pgfscope}%
\begin{pgfscope}%
\pgfpathrectangle{\pgfqpoint{0.800000in}{0.528000in}}{\pgfqpoint{4.960000in}{3.696000in}}%
\pgfusepath{clip}%
\pgfsetrectcap%
\pgfsetroundjoin%
\pgfsetlinewidth{1.505625pt}%
\definecolor{currentstroke}{rgb}{0.000000,1.000000,0.498039}%
\pgfsetstrokecolor{currentstroke}%
\pgfsetdash{}{0pt}%
\pgfpathmoveto{\pgfqpoint{1.025455in}{0.696000in}}%
\pgfpathlineto{\pgfqpoint{1.733382in}{0.696000in}}%
\pgfpathlineto{\pgfqpoint{1.737891in}{0.719662in}}%
\pgfpathlineto{\pgfqpoint{1.742400in}{0.727944in}}%
\pgfpathlineto{\pgfqpoint{5.534545in}{0.727944in}}%
\pgfpathlineto{\pgfqpoint{5.534545in}{0.727944in}}%
\pgfusepath{stroke}%
\end{pgfscope}%
\begin{pgfscope}%
\pgfpathrectangle{\pgfqpoint{0.800000in}{0.528000in}}{\pgfqpoint{4.960000in}{3.696000in}}%
\pgfusepath{clip}%
\pgfsetrectcap%
\pgfsetroundjoin%
\pgfsetlinewidth{1.505625pt}%
\definecolor{currentstroke}{rgb}{0.000000,1.000000,1.000000}%
\pgfsetstrokecolor{currentstroke}%
\pgfsetdash{}{0pt}%
\pgfpathmoveto{\pgfqpoint{1.025455in}{2.517972in}}%
\pgfpathlineto{\pgfqpoint{1.029964in}{1.812845in}}%
\pgfpathlineto{\pgfqpoint{1.043491in}{1.401127in}}%
\pgfpathlineto{\pgfqpoint{1.048000in}{1.566465in}}%
\pgfpathlineto{\pgfqpoint{1.052509in}{1.405859in}}%
\pgfpathlineto{\pgfqpoint{1.066036in}{1.412958in}}%
\pgfpathlineto{\pgfqpoint{1.070545in}{1.554930in}}%
\pgfpathlineto{\pgfqpoint{1.075055in}{1.417690in}}%
\pgfpathlineto{\pgfqpoint{1.079564in}{1.559662in}}%
\pgfpathlineto{\pgfqpoint{1.084073in}{1.422423in}}%
\pgfpathlineto{\pgfqpoint{1.088582in}{1.564394in}}%
\pgfpathlineto{\pgfqpoint{1.093091in}{1.427155in}}%
\pgfpathlineto{\pgfqpoint{1.111127in}{1.436620in}}%
\pgfpathlineto{\pgfqpoint{1.115636in}{1.578592in}}%
\pgfpathlineto{\pgfqpoint{1.120145in}{1.441352in}}%
\pgfpathlineto{\pgfqpoint{1.133673in}{1.448451in}}%
\pgfpathlineto{\pgfqpoint{1.138182in}{1.590423in}}%
\pgfpathlineto{\pgfqpoint{1.142691in}{1.453183in}}%
\pgfpathlineto{\pgfqpoint{1.151709in}{1.457915in}}%
\pgfpathlineto{\pgfqpoint{1.156218in}{1.526535in}}%
\pgfpathlineto{\pgfqpoint{1.169745in}{1.533634in}}%
\pgfpathlineto{\pgfqpoint{1.174255in}{1.544282in}}%
\pgfpathlineto{\pgfqpoint{1.336582in}{1.629465in}}%
\pgfpathlineto{\pgfqpoint{1.341091in}{1.911042in}}%
\pgfpathlineto{\pgfqpoint{1.345600in}{1.634197in}}%
\pgfpathlineto{\pgfqpoint{1.377164in}{1.650761in}}%
\pgfpathlineto{\pgfqpoint{1.381673in}{1.792732in}}%
\pgfpathlineto{\pgfqpoint{1.386182in}{1.655493in}}%
\pgfpathlineto{\pgfqpoint{1.598109in}{1.766704in}}%
\pgfpathlineto{\pgfqpoint{1.602618in}{1.908676in}}%
\pgfpathlineto{\pgfqpoint{1.607127in}{1.771437in}}%
\pgfpathlineto{\pgfqpoint{1.728873in}{1.835324in}}%
\pgfpathlineto{\pgfqpoint{1.733382in}{1.903944in}}%
\pgfpathlineto{\pgfqpoint{1.796509in}{1.937070in}}%
\pgfpathlineto{\pgfqpoint{1.801018in}{2.079042in}}%
\pgfpathlineto{\pgfqpoint{1.805527in}{1.941803in}}%
\pgfpathlineto{\pgfqpoint{2.089600in}{2.090873in}}%
\pgfpathlineto{\pgfqpoint{2.094109in}{2.232845in}}%
\pgfpathlineto{\pgfqpoint{2.098618in}{2.095606in}}%
\pgfpathlineto{\pgfqpoint{2.306036in}{2.204451in}}%
\pgfpathlineto{\pgfqpoint{2.310545in}{2.215099in}}%
\pgfpathlineto{\pgfqpoint{2.315055in}{2.209183in}}%
\pgfpathlineto{\pgfqpoint{2.324073in}{2.213915in}}%
\pgfpathlineto{\pgfqpoint{2.328582in}{2.224563in}}%
\pgfpathlineto{\pgfqpoint{2.333091in}{2.218648in}}%
\pgfpathlineto{\pgfqpoint{2.342109in}{2.223380in}}%
\pgfpathlineto{\pgfqpoint{2.346618in}{2.234028in}}%
\pgfpathlineto{\pgfqpoint{2.360145in}{2.241127in}}%
\pgfpathlineto{\pgfqpoint{2.364655in}{2.235211in}}%
\pgfpathlineto{\pgfqpoint{2.369164in}{2.239944in}}%
\pgfpathlineto{\pgfqpoint{2.373673in}{2.248225in}}%
\pgfpathlineto{\pgfqpoint{2.378182in}{2.242310in}}%
\pgfpathlineto{\pgfqpoint{2.382691in}{2.252958in}}%
\pgfpathlineto{\pgfqpoint{2.387200in}{2.255324in}}%
\pgfpathlineto{\pgfqpoint{2.391709in}{2.249408in}}%
\pgfpathlineto{\pgfqpoint{2.396218in}{2.260056in}}%
\pgfpathlineto{\pgfqpoint{2.414255in}{2.269521in}}%
\pgfpathlineto{\pgfqpoint{2.418764in}{2.263606in}}%
\pgfpathlineto{\pgfqpoint{2.423273in}{2.274254in}}%
\pgfpathlineto{\pgfqpoint{2.463855in}{2.295549in}}%
\pgfpathlineto{\pgfqpoint{2.468364in}{2.289634in}}%
\pgfpathlineto{\pgfqpoint{2.472873in}{2.300282in}}%
\pgfpathlineto{\pgfqpoint{2.563055in}{2.347606in}}%
\pgfpathlineto{\pgfqpoint{2.567564in}{2.489577in}}%
\pgfpathlineto{\pgfqpoint{2.572073in}{2.352338in}}%
\pgfpathlineto{\pgfqpoint{2.693818in}{2.416225in}}%
\pgfpathlineto{\pgfqpoint{2.698327in}{2.558197in}}%
\pgfpathlineto{\pgfqpoint{2.702836in}{2.420958in}}%
\pgfpathlineto{\pgfqpoint{2.883200in}{2.515606in}}%
\pgfpathlineto{\pgfqpoint{2.887709in}{2.584225in}}%
\pgfpathlineto{\pgfqpoint{2.892218in}{2.594873in}}%
\pgfpathlineto{\pgfqpoint{4.168291in}{3.264507in}}%
\pgfpathlineto{\pgfqpoint{4.172800in}{3.406479in}}%
\pgfpathlineto{\pgfqpoint{4.177309in}{3.269239in}}%
\pgfpathlineto{\pgfqpoint{4.695855in}{3.541352in}}%
\pgfpathlineto{\pgfqpoint{4.700364in}{3.683324in}}%
\pgfpathlineto{\pgfqpoint{4.704873in}{3.546085in}}%
\pgfpathlineto{\pgfqpoint{4.898764in}{3.647831in}}%
\pgfpathlineto{\pgfqpoint{4.903273in}{3.789803in}}%
\pgfpathlineto{\pgfqpoint{4.907782in}{3.652563in}}%
\pgfpathlineto{\pgfqpoint{5.191855in}{3.801634in}}%
\pgfpathlineto{\pgfqpoint{5.196364in}{3.870254in}}%
\pgfpathlineto{\pgfqpoint{5.200873in}{3.872620in}}%
\pgfpathlineto{\pgfqpoint{5.205382in}{3.883268in}}%
\pgfpathlineto{\pgfqpoint{5.534545in}{4.056000in}}%
\pgfpathlineto{\pgfqpoint{5.534545in}{4.056000in}}%
\pgfusepath{stroke}%
\end{pgfscope}%
\begin{pgfscope}%
\pgfpathrectangle{\pgfqpoint{0.800000in}{0.528000in}}{\pgfqpoint{4.960000in}{3.696000in}}%
\pgfusepath{clip}%
\pgfsetrectcap%
\pgfsetroundjoin%
\pgfsetlinewidth{1.505625pt}%
\definecolor{currentstroke}{rgb}{1.000000,0.000000,1.000000}%
\pgfsetstrokecolor{currentstroke}%
\pgfsetdash{}{0pt}%
\pgfpathmoveto{\pgfqpoint{1.025455in}{1.872000in}}%
\pgfpathlineto{\pgfqpoint{1.029964in}{0.946817in}}%
\pgfpathlineto{\pgfqpoint{1.052509in}{0.958648in}}%
\pgfpathlineto{\pgfqpoint{1.061527in}{1.095887in}}%
\pgfpathlineto{\pgfqpoint{1.066036in}{0.974028in}}%
\pgfpathlineto{\pgfqpoint{1.070545in}{0.968113in}}%
\pgfpathlineto{\pgfqpoint{1.079564in}{0.972845in}}%
\pgfpathlineto{\pgfqpoint{1.084073in}{1.116000in}}%
\pgfpathlineto{\pgfqpoint{1.088582in}{1.112155in}}%
\pgfpathlineto{\pgfqpoint{1.093091in}{0.988225in}}%
\pgfpathlineto{\pgfqpoint{1.102109in}{0.992958in}}%
\pgfpathlineto{\pgfqpoint{1.106618in}{1.525352in}}%
\pgfpathlineto{\pgfqpoint{1.111127in}{0.997690in}}%
\pgfpathlineto{\pgfqpoint{1.354618in}{1.125465in}}%
\pgfpathlineto{\pgfqpoint{1.359127in}{1.260338in}}%
\pgfpathlineto{\pgfqpoint{1.363636in}{1.130197in}}%
\pgfpathlineto{\pgfqpoint{1.471855in}{1.186986in}}%
\pgfpathlineto{\pgfqpoint{1.476364in}{1.197634in}}%
\pgfpathlineto{\pgfqpoint{1.480873in}{1.200000in}}%
\pgfpathlineto{\pgfqpoint{1.485382in}{1.194085in}}%
\pgfpathlineto{\pgfqpoint{1.489891in}{1.196451in}}%
\pgfpathlineto{\pgfqpoint{1.494400in}{1.207099in}}%
\pgfpathlineto{\pgfqpoint{1.498909in}{1.201183in}}%
\pgfpathlineto{\pgfqpoint{1.503418in}{1.211831in}}%
\pgfpathlineto{\pgfqpoint{1.566545in}{1.244958in}}%
\pgfpathlineto{\pgfqpoint{1.571055in}{1.379831in}}%
\pgfpathlineto{\pgfqpoint{1.575564in}{1.249690in}}%
\pgfpathlineto{\pgfqpoint{1.616145in}{1.270986in}}%
\pgfpathlineto{\pgfqpoint{1.620655in}{1.538366in}}%
\pgfpathlineto{\pgfqpoint{1.625164in}{1.275718in}}%
\pgfpathlineto{\pgfqpoint{1.733382in}{1.332507in}}%
\pgfpathlineto{\pgfqpoint{1.737891in}{1.351437in}}%
\pgfpathlineto{\pgfqpoint{1.755927in}{1.394028in}}%
\pgfpathlineto{\pgfqpoint{1.760436in}{1.421239in}}%
\pgfpathlineto{\pgfqpoint{1.764945in}{1.423606in}}%
\pgfpathlineto{\pgfqpoint{1.769455in}{1.409408in}}%
\pgfpathlineto{\pgfqpoint{1.778473in}{1.430704in}}%
\pgfpathlineto{\pgfqpoint{1.782982in}{1.457915in}}%
\pgfpathlineto{\pgfqpoint{1.787491in}{1.460282in}}%
\pgfpathlineto{\pgfqpoint{1.792000in}{1.470930in}}%
\pgfpathlineto{\pgfqpoint{1.796509in}{1.456732in}}%
\pgfpathlineto{\pgfqpoint{1.801018in}{1.475662in}}%
\pgfpathlineto{\pgfqpoint{1.805527in}{1.453183in}}%
\pgfpathlineto{\pgfqpoint{1.810036in}{1.463831in}}%
\pgfpathlineto{\pgfqpoint{1.814545in}{1.466197in}}%
\pgfpathlineto{\pgfqpoint{1.828073in}{1.498141in}}%
\pgfpathlineto{\pgfqpoint{1.832582in}{1.483944in}}%
\pgfpathlineto{\pgfqpoint{1.837091in}{1.511155in}}%
\pgfpathlineto{\pgfqpoint{1.841600in}{1.505239in}}%
\pgfpathlineto{\pgfqpoint{1.846109in}{1.515887in}}%
\pgfpathlineto{\pgfqpoint{1.850618in}{1.518254in}}%
\pgfpathlineto{\pgfqpoint{1.855127in}{1.504056in}}%
\pgfpathlineto{\pgfqpoint{1.859636in}{1.514704in}}%
\pgfpathlineto{\pgfqpoint{1.868655in}{1.519437in}}%
\pgfpathlineto{\pgfqpoint{1.873164in}{1.513521in}}%
\pgfpathlineto{\pgfqpoint{1.877673in}{1.532451in}}%
\pgfpathlineto{\pgfqpoint{1.886691in}{1.553746in}}%
\pgfpathlineto{\pgfqpoint{1.891200in}{1.572676in}}%
\pgfpathlineto{\pgfqpoint{1.895709in}{1.541915in}}%
\pgfpathlineto{\pgfqpoint{1.904727in}{1.546648in}}%
\pgfpathlineto{\pgfqpoint{1.909236in}{1.573859in}}%
\pgfpathlineto{\pgfqpoint{1.918255in}{1.562028in}}%
\pgfpathlineto{\pgfqpoint{1.927273in}{1.616451in}}%
\pgfpathlineto{\pgfqpoint{1.931782in}{1.618817in}}%
\pgfpathlineto{\pgfqpoint{1.936291in}{1.588056in}}%
\pgfpathlineto{\pgfqpoint{1.940800in}{1.565577in}}%
\pgfpathlineto{\pgfqpoint{1.945309in}{1.584507in}}%
\pgfpathlineto{\pgfqpoint{1.949818in}{1.628282in}}%
\pgfpathlineto{\pgfqpoint{1.954327in}{1.605803in}}%
\pgfpathlineto{\pgfqpoint{1.958836in}{1.616451in}}%
\pgfpathlineto{\pgfqpoint{1.963345in}{1.651944in}}%
\pgfpathlineto{\pgfqpoint{1.967855in}{1.596338in}}%
\pgfpathlineto{\pgfqpoint{1.972364in}{1.623549in}}%
\pgfpathlineto{\pgfqpoint{1.976873in}{1.609352in}}%
\pgfpathlineto{\pgfqpoint{1.981382in}{1.661408in}}%
\pgfpathlineto{\pgfqpoint{1.985891in}{1.647211in}}%
\pgfpathlineto{\pgfqpoint{1.990400in}{1.608169in}}%
\pgfpathlineto{\pgfqpoint{1.994909in}{1.618817in}}%
\pgfpathlineto{\pgfqpoint{1.999418in}{1.621183in}}%
\pgfpathlineto{\pgfqpoint{2.003927in}{1.681521in}}%
\pgfpathlineto{\pgfqpoint{2.008436in}{1.650761in}}%
\pgfpathlineto{\pgfqpoint{2.012945in}{1.644845in}}%
\pgfpathlineto{\pgfqpoint{2.017455in}{1.663775in}}%
\pgfpathlineto{\pgfqpoint{2.021964in}{1.633014in}}%
\pgfpathlineto{\pgfqpoint{2.026473in}{1.651944in}}%
\pgfpathlineto{\pgfqpoint{2.030982in}{1.679155in}}%
\pgfpathlineto{\pgfqpoint{2.035491in}{1.656676in}}%
\pgfpathlineto{\pgfqpoint{2.040000in}{1.683887in}}%
\pgfpathlineto{\pgfqpoint{2.044509in}{1.669690in}}%
\pgfpathlineto{\pgfqpoint{2.049018in}{1.672056in}}%
\pgfpathlineto{\pgfqpoint{2.053527in}{1.682704in}}%
\pgfpathlineto{\pgfqpoint{2.058036in}{1.651944in}}%
\pgfpathlineto{\pgfqpoint{2.071564in}{1.708732in}}%
\pgfpathlineto{\pgfqpoint{2.076073in}{1.694535in}}%
\pgfpathlineto{\pgfqpoint{2.080582in}{1.705183in}}%
\pgfpathlineto{\pgfqpoint{2.085091in}{1.699268in}}%
\pgfpathlineto{\pgfqpoint{2.089600in}{1.718197in}}%
\pgfpathlineto{\pgfqpoint{2.094109in}{1.679155in}}%
\pgfpathlineto{\pgfqpoint{2.098618in}{1.714648in}}%
\pgfpathlineto{\pgfqpoint{2.103127in}{1.692169in}}%
\pgfpathlineto{\pgfqpoint{2.107636in}{1.686254in}}%
\pgfpathlineto{\pgfqpoint{2.112145in}{1.705183in}}%
\pgfpathlineto{\pgfqpoint{2.116655in}{1.699268in}}%
\pgfpathlineto{\pgfqpoint{2.121164in}{1.726479in}}%
\pgfpathlineto{\pgfqpoint{2.125673in}{1.695718in}}%
\pgfpathlineto{\pgfqpoint{2.130182in}{1.689803in}}%
\pgfpathlineto{\pgfqpoint{2.134691in}{1.741859in}}%
\pgfpathlineto{\pgfqpoint{2.139200in}{1.727662in}}%
\pgfpathlineto{\pgfqpoint{2.143709in}{1.721746in}}%
\pgfpathlineto{\pgfqpoint{2.148218in}{1.724113in}}%
\pgfpathlineto{\pgfqpoint{2.152727in}{1.767887in}}%
\pgfpathlineto{\pgfqpoint{2.157236in}{1.745408in}}%
\pgfpathlineto{\pgfqpoint{2.161745in}{1.822310in}}%
\pgfpathlineto{\pgfqpoint{2.166255in}{1.733577in}}%
\pgfpathlineto{\pgfqpoint{2.170764in}{1.802197in}}%
\pgfpathlineto{\pgfqpoint{2.175273in}{1.746592in}}%
\pgfpathlineto{\pgfqpoint{2.179782in}{1.765521in}}%
\pgfpathlineto{\pgfqpoint{2.184291in}{1.743042in}}%
\pgfpathlineto{\pgfqpoint{2.188800in}{1.819944in}}%
\pgfpathlineto{\pgfqpoint{2.193309in}{1.772620in}}%
\pgfpathlineto{\pgfqpoint{2.197818in}{1.774986in}}%
\pgfpathlineto{\pgfqpoint{2.202327in}{1.785634in}}%
\pgfpathlineto{\pgfqpoint{2.206836in}{1.779718in}}%
\pgfpathlineto{\pgfqpoint{2.211345in}{1.765521in}}%
\pgfpathlineto{\pgfqpoint{2.215855in}{1.776169in}}%
\pgfpathlineto{\pgfqpoint{2.220364in}{1.761972in}}%
\pgfpathlineto{\pgfqpoint{2.224873in}{1.764338in}}%
\pgfpathlineto{\pgfqpoint{2.229382in}{1.758423in}}%
\pgfpathlineto{\pgfqpoint{2.233891in}{1.785634in}}%
\pgfpathlineto{\pgfqpoint{2.238400in}{1.779718in}}%
\pgfpathlineto{\pgfqpoint{2.242909in}{1.815211in}}%
\pgfpathlineto{\pgfqpoint{2.247418in}{1.784451in}}%
\pgfpathlineto{\pgfqpoint{2.251927in}{1.819944in}}%
\pgfpathlineto{\pgfqpoint{2.256436in}{1.830592in}}%
\pgfpathlineto{\pgfqpoint{2.260945in}{1.766704in}}%
\pgfpathlineto{\pgfqpoint{2.265455in}{1.810479in}}%
\pgfpathlineto{\pgfqpoint{2.269964in}{1.870817in}}%
\pgfpathlineto{\pgfqpoint{2.274473in}{1.806930in}}%
\pgfpathlineto{\pgfqpoint{2.278982in}{1.842423in}}%
\pgfpathlineto{\pgfqpoint{2.283491in}{1.803380in}}%
\pgfpathlineto{\pgfqpoint{2.288000in}{1.838873in}}%
\pgfpathlineto{\pgfqpoint{2.292509in}{1.832958in}}%
\pgfpathlineto{\pgfqpoint{2.297018in}{1.818761in}}%
\pgfpathlineto{\pgfqpoint{2.301527in}{1.854254in}}%
\pgfpathlineto{\pgfqpoint{2.306036in}{1.831775in}}%
\pgfpathlineto{\pgfqpoint{2.310545in}{1.867268in}}%
\pgfpathlineto{\pgfqpoint{2.315055in}{1.869634in}}%
\pgfpathlineto{\pgfqpoint{2.319564in}{1.838873in}}%
\pgfpathlineto{\pgfqpoint{2.324073in}{1.832958in}}%
\pgfpathlineto{\pgfqpoint{2.328582in}{1.868451in}}%
\pgfpathlineto{\pgfqpoint{2.333091in}{1.870817in}}%
\pgfpathlineto{\pgfqpoint{2.337600in}{1.889746in}}%
\pgfpathlineto{\pgfqpoint{2.342109in}{1.850704in}}%
\pgfpathlineto{\pgfqpoint{2.346618in}{1.886197in}}%
\pgfpathlineto{\pgfqpoint{2.351127in}{1.888563in}}%
\pgfpathlineto{\pgfqpoint{2.355636in}{1.882648in}}%
\pgfpathlineto{\pgfqpoint{2.360145in}{1.918141in}}%
\pgfpathlineto{\pgfqpoint{2.364655in}{1.854254in}}%
\pgfpathlineto{\pgfqpoint{2.369164in}{1.873183in}}%
\pgfpathlineto{\pgfqpoint{2.373673in}{1.858986in}}%
\pgfpathlineto{\pgfqpoint{2.378182in}{1.902761in}}%
\pgfpathlineto{\pgfqpoint{2.382691in}{1.855437in}}%
\pgfpathlineto{\pgfqpoint{2.391709in}{1.942986in}}%
\pgfpathlineto{\pgfqpoint{2.396218in}{1.870817in}}%
\pgfpathlineto{\pgfqpoint{2.400727in}{1.889746in}}%
\pgfpathlineto{\pgfqpoint{2.405236in}{1.950085in}}%
\pgfpathlineto{\pgfqpoint{2.409745in}{1.911042in}}%
\pgfpathlineto{\pgfqpoint{2.414255in}{1.929972in}}%
\pgfpathlineto{\pgfqpoint{2.418764in}{1.907493in}}%
\pgfpathlineto{\pgfqpoint{2.423273in}{1.934704in}}%
\pgfpathlineto{\pgfqpoint{2.427782in}{1.953634in}}%
\pgfpathlineto{\pgfqpoint{2.432291in}{1.898028in}}%
\pgfpathlineto{\pgfqpoint{2.436800in}{1.925239in}}%
\pgfpathlineto{\pgfqpoint{2.445818in}{2.228113in}}%
\pgfpathlineto{\pgfqpoint{2.450327in}{1.965465in}}%
\pgfpathlineto{\pgfqpoint{2.454836in}{1.934704in}}%
\pgfpathlineto{\pgfqpoint{2.459345in}{1.953634in}}%
\pgfpathlineto{\pgfqpoint{2.463855in}{1.947718in}}%
\pgfpathlineto{\pgfqpoint{2.477382in}{1.954817in}}%
\pgfpathlineto{\pgfqpoint{2.481891in}{1.924056in}}%
\pgfpathlineto{\pgfqpoint{2.486400in}{1.942986in}}%
\pgfpathlineto{\pgfqpoint{2.490909in}{1.986761in}}%
\pgfpathlineto{\pgfqpoint{2.495418in}{1.997408in}}%
\pgfpathlineto{\pgfqpoint{2.499927in}{1.933521in}}%
\pgfpathlineto{\pgfqpoint{2.504436in}{1.960732in}}%
\pgfpathlineto{\pgfqpoint{2.513455in}{1.982028in}}%
\pgfpathlineto{\pgfqpoint{2.517964in}{1.959549in}}%
\pgfpathlineto{\pgfqpoint{2.522473in}{1.978479in}}%
\pgfpathlineto{\pgfqpoint{2.526982in}{1.989127in}}%
\pgfpathlineto{\pgfqpoint{2.531491in}{1.991493in}}%
\pgfpathlineto{\pgfqpoint{2.536000in}{1.969014in}}%
\pgfpathlineto{\pgfqpoint{2.545018in}{1.973746in}}%
\pgfpathlineto{\pgfqpoint{2.549527in}{2.058930in}}%
\pgfpathlineto{\pgfqpoint{2.554036in}{1.986761in}}%
\pgfpathlineto{\pgfqpoint{2.558545in}{1.980845in}}%
\pgfpathlineto{\pgfqpoint{2.563055in}{1.950085in}}%
\pgfpathlineto{\pgfqpoint{2.567564in}{2.035268in}}%
\pgfpathlineto{\pgfqpoint{2.572073in}{1.996225in}}%
\pgfpathlineto{\pgfqpoint{2.576582in}{1.973746in}}%
\pgfpathlineto{\pgfqpoint{2.581091in}{1.976113in}}%
\pgfpathlineto{\pgfqpoint{2.585600in}{2.053014in}}%
\pgfpathlineto{\pgfqpoint{2.590109in}{1.997408in}}%
\pgfpathlineto{\pgfqpoint{2.594618in}{2.024620in}}%
\pgfpathlineto{\pgfqpoint{2.599127in}{2.002141in}}%
\pgfpathlineto{\pgfqpoint{2.603636in}{1.996225in}}%
\pgfpathlineto{\pgfqpoint{2.608145in}{2.048282in}}%
\pgfpathlineto{\pgfqpoint{2.612655in}{2.017521in}}%
\pgfpathlineto{\pgfqpoint{2.617164in}{2.044732in}}%
\pgfpathlineto{\pgfqpoint{2.621673in}{2.047099in}}%
\pgfpathlineto{\pgfqpoint{2.626182in}{2.016338in}}%
\pgfpathlineto{\pgfqpoint{2.630691in}{2.051831in}}%
\pgfpathlineto{\pgfqpoint{2.635200in}{2.037634in}}%
\pgfpathlineto{\pgfqpoint{2.639709in}{2.064845in}}%
\pgfpathlineto{\pgfqpoint{2.644218in}{2.058930in}}%
\pgfpathlineto{\pgfqpoint{2.648727in}{2.036451in}}%
\pgfpathlineto{\pgfqpoint{2.653236in}{2.080225in}}%
\pgfpathlineto{\pgfqpoint{2.657745in}{2.066028in}}%
\pgfpathlineto{\pgfqpoint{2.662255in}{2.060113in}}%
\pgfpathlineto{\pgfqpoint{2.666764in}{2.045915in}}%
\pgfpathlineto{\pgfqpoint{2.671273in}{2.122817in}}%
\pgfpathlineto{\pgfqpoint{2.675782in}{2.067211in}}%
\pgfpathlineto{\pgfqpoint{2.680291in}{2.094423in}}%
\pgfpathlineto{\pgfqpoint{2.684800in}{2.047099in}}%
\pgfpathlineto{\pgfqpoint{2.689309in}{2.057746in}}%
\pgfpathlineto{\pgfqpoint{2.693818in}{2.060113in}}%
\pgfpathlineto{\pgfqpoint{2.698327in}{2.070761in}}%
\pgfpathlineto{\pgfqpoint{2.702836in}{2.106254in}}%
\pgfpathlineto{\pgfqpoint{2.707345in}{2.050648in}}%
\pgfpathlineto{\pgfqpoint{2.711855in}{2.086141in}}%
\pgfpathlineto{\pgfqpoint{2.720873in}{2.090873in}}%
\pgfpathlineto{\pgfqpoint{2.725382in}{2.084958in}}%
\pgfpathlineto{\pgfqpoint{2.729891in}{2.153577in}}%
\pgfpathlineto{\pgfqpoint{2.734400in}{2.081408in}}%
\pgfpathlineto{\pgfqpoint{2.738909in}{2.141746in}}%
\pgfpathlineto{\pgfqpoint{2.743418in}{2.069577in}}%
\pgfpathlineto{\pgfqpoint{2.747927in}{2.096789in}}%
\pgfpathlineto{\pgfqpoint{2.752436in}{2.115718in}}%
\pgfpathlineto{\pgfqpoint{2.756945in}{2.076676in}}%
\pgfpathlineto{\pgfqpoint{2.761455in}{2.120451in}}%
\pgfpathlineto{\pgfqpoint{2.765964in}{2.147662in}}%
\pgfpathlineto{\pgfqpoint{2.770473in}{2.125183in}}%
\pgfpathlineto{\pgfqpoint{2.774982in}{2.152394in}}%
\pgfpathlineto{\pgfqpoint{2.779491in}{2.121634in}}%
\pgfpathlineto{\pgfqpoint{2.784000in}{2.165408in}}%
\pgfpathlineto{\pgfqpoint{2.788509in}{2.134648in}}%
\pgfpathlineto{\pgfqpoint{2.793018in}{2.161859in}}%
\pgfpathlineto{\pgfqpoint{2.797527in}{2.180789in}}%
\pgfpathlineto{\pgfqpoint{2.802036in}{2.166592in}}%
\pgfpathlineto{\pgfqpoint{2.806545in}{2.110986in}}%
\pgfpathlineto{\pgfqpoint{2.811055in}{2.138197in}}%
\pgfpathlineto{\pgfqpoint{2.815564in}{2.140563in}}%
\pgfpathlineto{\pgfqpoint{2.820073in}{2.118085in}}%
\pgfpathlineto{\pgfqpoint{2.824582in}{2.178423in}}%
\pgfpathlineto{\pgfqpoint{2.829091in}{2.131099in}}%
\pgfpathlineto{\pgfqpoint{2.833600in}{2.150028in}}%
\pgfpathlineto{\pgfqpoint{2.838109in}{2.202085in}}%
\pgfpathlineto{\pgfqpoint{2.842618in}{2.146479in}}%
\pgfpathlineto{\pgfqpoint{2.847127in}{2.181972in}}%
\pgfpathlineto{\pgfqpoint{2.851636in}{2.151211in}}%
\pgfpathlineto{\pgfqpoint{2.856145in}{2.161859in}}%
\pgfpathlineto{\pgfqpoint{2.860655in}{2.147662in}}%
\pgfpathlineto{\pgfqpoint{2.865164in}{2.141746in}}%
\pgfpathlineto{\pgfqpoint{2.869673in}{2.152394in}}%
\pgfpathlineto{\pgfqpoint{2.874182in}{2.187887in}}%
\pgfpathlineto{\pgfqpoint{2.878691in}{2.165408in}}%
\pgfpathlineto{\pgfqpoint{2.883200in}{2.250592in}}%
\pgfpathlineto{\pgfqpoint{2.887709in}{2.194986in}}%
\pgfpathlineto{\pgfqpoint{2.892218in}{2.164225in}}%
\pgfpathlineto{\pgfqpoint{2.896727in}{2.224563in}}%
\pgfpathlineto{\pgfqpoint{2.901236in}{2.193803in}}%
\pgfpathlineto{\pgfqpoint{2.905745in}{2.254141in}}%
\pgfpathlineto{\pgfqpoint{2.910255in}{2.206817in}}%
\pgfpathlineto{\pgfqpoint{2.914764in}{2.217465in}}%
\pgfpathlineto{\pgfqpoint{2.919273in}{2.178423in}}%
\pgfpathlineto{\pgfqpoint{2.923782in}{2.271887in}}%
\pgfpathlineto{\pgfqpoint{2.928291in}{2.249408in}}%
\pgfpathlineto{\pgfqpoint{2.932800in}{2.218648in}}%
\pgfpathlineto{\pgfqpoint{2.937309in}{2.204451in}}%
\pgfpathlineto{\pgfqpoint{2.941818in}{2.223380in}}%
\pgfpathlineto{\pgfqpoint{2.946327in}{2.217465in}}%
\pgfpathlineto{\pgfqpoint{2.955345in}{2.222197in}}%
\pgfpathlineto{\pgfqpoint{2.959855in}{2.232845in}}%
\pgfpathlineto{\pgfqpoint{2.964364in}{2.235211in}}%
\pgfpathlineto{\pgfqpoint{2.968873in}{2.262423in}}%
\pgfpathlineto{\pgfqpoint{2.973382in}{2.215099in}}%
\pgfpathlineto{\pgfqpoint{2.977891in}{2.234028in}}%
\pgfpathlineto{\pgfqpoint{2.986909in}{2.238761in}}%
\pgfpathlineto{\pgfqpoint{2.991418in}{2.249408in}}%
\pgfpathlineto{\pgfqpoint{2.995927in}{2.268338in}}%
\pgfpathlineto{\pgfqpoint{3.000436in}{2.221014in}}%
\pgfpathlineto{\pgfqpoint{3.004945in}{2.256507in}}%
\pgfpathlineto{\pgfqpoint{3.009455in}{2.267155in}}%
\pgfpathlineto{\pgfqpoint{3.013964in}{2.269521in}}%
\pgfpathlineto{\pgfqpoint{3.018473in}{2.238761in}}%
\pgfpathlineto{\pgfqpoint{3.022982in}{2.274254in}}%
\pgfpathlineto{\pgfqpoint{3.027491in}{2.235211in}}%
\pgfpathlineto{\pgfqpoint{3.032000in}{2.262423in}}%
\pgfpathlineto{\pgfqpoint{3.036509in}{2.264789in}}%
\pgfpathlineto{\pgfqpoint{3.041018in}{2.250592in}}%
\pgfpathlineto{\pgfqpoint{3.045527in}{2.294366in}}%
\pgfpathlineto{\pgfqpoint{3.050036in}{2.247042in}}%
\pgfpathlineto{\pgfqpoint{3.063564in}{2.303831in}}%
\pgfpathlineto{\pgfqpoint{3.068073in}{2.347606in}}%
\pgfpathlineto{\pgfqpoint{3.072582in}{2.267155in}}%
\pgfpathlineto{\pgfqpoint{3.077091in}{2.302648in}}%
\pgfpathlineto{\pgfqpoint{3.081600in}{2.313296in}}%
\pgfpathlineto{\pgfqpoint{3.086109in}{2.315662in}}%
\pgfpathlineto{\pgfqpoint{3.090618in}{2.284901in}}%
\pgfpathlineto{\pgfqpoint{3.095127in}{2.336958in}}%
\pgfpathlineto{\pgfqpoint{3.099636in}{2.297915in}}%
\pgfpathlineto{\pgfqpoint{3.108655in}{2.302648in}}%
\pgfpathlineto{\pgfqpoint{3.113164in}{2.379549in}}%
\pgfpathlineto{\pgfqpoint{3.117673in}{2.332225in}}%
\pgfpathlineto{\pgfqpoint{3.122182in}{2.326310in}}%
\pgfpathlineto{\pgfqpoint{3.126691in}{2.361803in}}%
\pgfpathlineto{\pgfqpoint{3.135709in}{2.300282in}}%
\pgfpathlineto{\pgfqpoint{3.140218in}{2.310930in}}%
\pgfpathlineto{\pgfqpoint{3.144727in}{2.338141in}}%
\pgfpathlineto{\pgfqpoint{3.149236in}{2.332225in}}%
\pgfpathlineto{\pgfqpoint{3.153745in}{2.342873in}}%
\pgfpathlineto{\pgfqpoint{3.158255in}{2.345239in}}%
\pgfpathlineto{\pgfqpoint{3.162764in}{2.372451in}}%
\pgfpathlineto{\pgfqpoint{3.171782in}{2.344056in}}%
\pgfpathlineto{\pgfqpoint{3.176291in}{2.429239in}}%
\pgfpathlineto{\pgfqpoint{3.180800in}{2.406761in}}%
\pgfpathlineto{\pgfqpoint{3.189818in}{2.345239in}}%
\pgfpathlineto{\pgfqpoint{3.194327in}{2.347606in}}%
\pgfpathlineto{\pgfqpoint{3.198836in}{2.374817in}}%
\pgfpathlineto{\pgfqpoint{3.203345in}{2.393746in}}%
\pgfpathlineto{\pgfqpoint{3.207855in}{2.404394in}}%
\pgfpathlineto{\pgfqpoint{3.212364in}{2.390197in}}%
\pgfpathlineto{\pgfqpoint{3.216873in}{2.433972in}}%
\pgfpathlineto{\pgfqpoint{3.221382in}{2.386648in}}%
\pgfpathlineto{\pgfqpoint{3.225891in}{2.380732in}}%
\pgfpathlineto{\pgfqpoint{3.230400in}{2.358254in}}%
\pgfpathlineto{\pgfqpoint{3.234909in}{2.418592in}}%
\pgfpathlineto{\pgfqpoint{3.239418in}{2.429239in}}%
\pgfpathlineto{\pgfqpoint{3.243927in}{2.390197in}}%
\pgfpathlineto{\pgfqpoint{3.248436in}{2.367718in}}%
\pgfpathlineto{\pgfqpoint{3.257455in}{2.471831in}}%
\pgfpathlineto{\pgfqpoint{3.261964in}{2.407944in}}%
\pgfpathlineto{\pgfqpoint{3.266473in}{2.435155in}}%
\pgfpathlineto{\pgfqpoint{3.275491in}{2.456451in}}%
\pgfpathlineto{\pgfqpoint{3.280000in}{2.392563in}}%
\pgfpathlineto{\pgfqpoint{3.284509in}{2.452901in}}%
\pgfpathlineto{\pgfqpoint{3.289018in}{2.397296in}}%
\pgfpathlineto{\pgfqpoint{3.293527in}{2.474197in}}%
\pgfpathlineto{\pgfqpoint{3.298036in}{2.402028in}}%
\pgfpathlineto{\pgfqpoint{3.302545in}{2.420958in}}%
\pgfpathlineto{\pgfqpoint{3.307055in}{2.423324in}}%
\pgfpathlineto{\pgfqpoint{3.311564in}{2.400845in}}%
\pgfpathlineto{\pgfqpoint{3.316073in}{2.444620in}}%
\pgfpathlineto{\pgfqpoint{3.320582in}{2.438704in}}%
\pgfpathlineto{\pgfqpoint{3.325091in}{2.407944in}}%
\pgfpathlineto{\pgfqpoint{3.329600in}{2.451718in}}%
\pgfpathlineto{\pgfqpoint{3.334109in}{2.437521in}}%
\pgfpathlineto{\pgfqpoint{3.338618in}{2.448169in}}%
\pgfpathlineto{\pgfqpoint{3.347636in}{2.419775in}}%
\pgfpathlineto{\pgfqpoint{3.352145in}{2.455268in}}%
\pgfpathlineto{\pgfqpoint{3.361164in}{2.476563in}}%
\pgfpathlineto{\pgfqpoint{3.365673in}{2.495493in}}%
\pgfpathlineto{\pgfqpoint{3.370182in}{2.464732in}}%
\pgfpathlineto{\pgfqpoint{3.374691in}{2.425690in}}%
\pgfpathlineto{\pgfqpoint{3.379200in}{2.486028in}}%
\pgfpathlineto{\pgfqpoint{3.383709in}{2.488394in}}%
\pgfpathlineto{\pgfqpoint{3.388218in}{2.507324in}}%
\pgfpathlineto{\pgfqpoint{3.392727in}{2.493127in}}%
\pgfpathlineto{\pgfqpoint{3.397236in}{2.503775in}}%
\pgfpathlineto{\pgfqpoint{3.401745in}{2.448169in}}%
\pgfpathlineto{\pgfqpoint{3.406255in}{2.433972in}}%
\pgfpathlineto{\pgfqpoint{3.410764in}{2.510873in}}%
\pgfpathlineto{\pgfqpoint{3.415273in}{2.554648in}}%
\pgfpathlineto{\pgfqpoint{3.419782in}{2.441070in}}%
\pgfpathlineto{\pgfqpoint{3.424291in}{2.509690in}}%
\pgfpathlineto{\pgfqpoint{3.428800in}{2.487211in}}%
\pgfpathlineto{\pgfqpoint{3.433309in}{2.522704in}}%
\pgfpathlineto{\pgfqpoint{3.437818in}{2.533352in}}%
\pgfpathlineto{\pgfqpoint{3.442327in}{2.535718in}}%
\pgfpathlineto{\pgfqpoint{3.446836in}{2.504958in}}%
\pgfpathlineto{\pgfqpoint{3.451345in}{2.532169in}}%
\pgfpathlineto{\pgfqpoint{3.455855in}{2.526254in}}%
\pgfpathlineto{\pgfqpoint{3.460364in}{2.536901in}}%
\pgfpathlineto{\pgfqpoint{3.464873in}{2.539268in}}%
\pgfpathlineto{\pgfqpoint{3.469382in}{2.500225in}}%
\pgfpathlineto{\pgfqpoint{3.473891in}{2.535718in}}%
\pgfpathlineto{\pgfqpoint{3.478400in}{2.513239in}}%
\pgfpathlineto{\pgfqpoint{3.482909in}{2.515606in}}%
\pgfpathlineto{\pgfqpoint{3.487418in}{2.584225in}}%
\pgfpathlineto{\pgfqpoint{3.491927in}{2.520338in}}%
\pgfpathlineto{\pgfqpoint{3.496436in}{2.497859in}}%
\pgfpathlineto{\pgfqpoint{3.500945in}{2.558197in}}%
\pgfpathlineto{\pgfqpoint{3.505455in}{2.560563in}}%
\pgfpathlineto{\pgfqpoint{3.509964in}{2.538085in}}%
\pgfpathlineto{\pgfqpoint{3.514473in}{2.548732in}}%
\pgfpathlineto{\pgfqpoint{3.518982in}{2.592507in}}%
\pgfpathlineto{\pgfqpoint{3.528000in}{2.530986in}}%
\pgfpathlineto{\pgfqpoint{3.532509in}{2.558197in}}%
\pgfpathlineto{\pgfqpoint{3.537018in}{2.568845in}}%
\pgfpathlineto{\pgfqpoint{3.541527in}{2.529803in}}%
\pgfpathlineto{\pgfqpoint{3.546036in}{2.565296in}}%
\pgfpathlineto{\pgfqpoint{3.550545in}{2.517972in}}%
\pgfpathlineto{\pgfqpoint{3.559564in}{2.630366in}}%
\pgfpathlineto{\pgfqpoint{3.564073in}{2.574761in}}%
\pgfpathlineto{\pgfqpoint{3.568582in}{2.544000in}}%
\pgfpathlineto{\pgfqpoint{3.573091in}{2.571211in}}%
\pgfpathlineto{\pgfqpoint{3.577600in}{2.639831in}}%
\pgfpathlineto{\pgfqpoint{3.582109in}{2.575944in}}%
\pgfpathlineto{\pgfqpoint{3.586618in}{2.586592in}}%
\pgfpathlineto{\pgfqpoint{3.591127in}{2.555831in}}%
\pgfpathlineto{\pgfqpoint{3.595636in}{2.599606in}}%
\pgfpathlineto{\pgfqpoint{3.600145in}{2.659944in}}%
\pgfpathlineto{\pgfqpoint{3.604655in}{2.587775in}}%
\pgfpathlineto{\pgfqpoint{3.609164in}{2.631549in}}%
\pgfpathlineto{\pgfqpoint{3.613673in}{2.575944in}}%
\pgfpathlineto{\pgfqpoint{3.618182in}{2.586592in}}%
\pgfpathlineto{\pgfqpoint{3.622691in}{2.622085in}}%
\pgfpathlineto{\pgfqpoint{3.627200in}{2.599606in}}%
\pgfpathlineto{\pgfqpoint{3.631709in}{2.610254in}}%
\pgfpathlineto{\pgfqpoint{3.636218in}{2.587775in}}%
\pgfpathlineto{\pgfqpoint{3.640727in}{2.581859in}}%
\pgfpathlineto{\pgfqpoint{3.645236in}{2.609070in}}%
\pgfpathlineto{\pgfqpoint{3.649745in}{2.652845in}}%
\pgfpathlineto{\pgfqpoint{3.654255in}{2.671775in}}%
\pgfpathlineto{\pgfqpoint{3.658764in}{2.599606in}}%
\pgfpathlineto{\pgfqpoint{3.663273in}{2.643380in}}%
\pgfpathlineto{\pgfqpoint{3.667782in}{2.629183in}}%
\pgfpathlineto{\pgfqpoint{3.672291in}{2.672958in}}%
\pgfpathlineto{\pgfqpoint{3.676800in}{2.667042in}}%
\pgfpathlineto{\pgfqpoint{3.681309in}{2.611437in}}%
\pgfpathlineto{\pgfqpoint{3.685818in}{2.646930in}}%
\pgfpathlineto{\pgfqpoint{3.690327in}{2.624451in}}%
\pgfpathlineto{\pgfqpoint{3.694836in}{2.734479in}}%
\pgfpathlineto{\pgfqpoint{3.699345in}{2.703718in}}%
\pgfpathlineto{\pgfqpoint{3.703855in}{2.664676in}}%
\pgfpathlineto{\pgfqpoint{3.708364in}{2.642197in}}%
\pgfpathlineto{\pgfqpoint{3.712873in}{2.661127in}}%
\pgfpathlineto{\pgfqpoint{3.717382in}{2.646930in}}%
\pgfpathlineto{\pgfqpoint{3.721891in}{2.682423in}}%
\pgfpathlineto{\pgfqpoint{3.726400in}{2.668225in}}%
\pgfpathlineto{\pgfqpoint{3.730909in}{2.687155in}}%
\pgfpathlineto{\pgfqpoint{3.735418in}{2.656394in}}%
\pgfpathlineto{\pgfqpoint{3.739927in}{2.667042in}}%
\pgfpathlineto{\pgfqpoint{3.744436in}{2.710817in}}%
\pgfpathlineto{\pgfqpoint{3.748945in}{2.696620in}}%
\pgfpathlineto{\pgfqpoint{3.753455in}{2.649296in}}%
\pgfpathlineto{\pgfqpoint{3.757964in}{2.668225in}}%
\pgfpathlineto{\pgfqpoint{3.762473in}{2.654028in}}%
\pgfpathlineto{\pgfqpoint{3.771491in}{2.691887in}}%
\pgfpathlineto{\pgfqpoint{3.776000in}{2.652845in}}%
\pgfpathlineto{\pgfqpoint{3.780509in}{2.704901in}}%
\pgfpathlineto{\pgfqpoint{3.785018in}{2.690704in}}%
\pgfpathlineto{\pgfqpoint{3.789527in}{2.726197in}}%
\pgfpathlineto{\pgfqpoint{3.794036in}{2.720282in}}%
\pgfpathlineto{\pgfqpoint{3.798545in}{2.697803in}}%
\pgfpathlineto{\pgfqpoint{3.803055in}{2.700169in}}%
\pgfpathlineto{\pgfqpoint{3.807564in}{2.727380in}}%
\pgfpathlineto{\pgfqpoint{3.812073in}{2.704901in}}%
\pgfpathlineto{\pgfqpoint{3.821091in}{2.726197in}}%
\pgfpathlineto{\pgfqpoint{3.825600in}{2.703718in}}%
\pgfpathlineto{\pgfqpoint{3.830109in}{2.739211in}}%
\pgfpathlineto{\pgfqpoint{3.834618in}{2.741577in}}%
\pgfpathlineto{\pgfqpoint{3.839127in}{2.694254in}}%
\pgfpathlineto{\pgfqpoint{3.843636in}{2.787718in}}%
\pgfpathlineto{\pgfqpoint{3.848145in}{2.756958in}}%
\pgfpathlineto{\pgfqpoint{3.852655in}{2.742761in}}%
\pgfpathlineto{\pgfqpoint{3.857164in}{2.720282in}}%
\pgfpathlineto{\pgfqpoint{3.861673in}{2.788901in}}%
\pgfpathlineto{\pgfqpoint{3.866182in}{2.749859in}}%
\pgfpathlineto{\pgfqpoint{3.870691in}{2.777070in}}%
\pgfpathlineto{\pgfqpoint{3.875200in}{2.754592in}}%
\pgfpathlineto{\pgfqpoint{3.879709in}{2.773521in}}%
\pgfpathlineto{\pgfqpoint{3.884218in}{2.734479in}}%
\pgfpathlineto{\pgfqpoint{3.893236in}{2.739211in}}%
\pgfpathlineto{\pgfqpoint{3.897745in}{2.758141in}}%
\pgfpathlineto{\pgfqpoint{3.902255in}{2.735662in}}%
\pgfpathlineto{\pgfqpoint{3.906764in}{2.787718in}}%
\pgfpathlineto{\pgfqpoint{3.911273in}{2.748676in}}%
\pgfpathlineto{\pgfqpoint{3.915782in}{2.767606in}}%
\pgfpathlineto{\pgfqpoint{3.920291in}{2.761690in}}%
\pgfpathlineto{\pgfqpoint{3.924800in}{2.805465in}}%
\pgfpathlineto{\pgfqpoint{3.929309in}{2.758141in}}%
\pgfpathlineto{\pgfqpoint{3.938327in}{2.779437in}}%
\pgfpathlineto{\pgfqpoint{3.947345in}{2.751042in}}%
\pgfpathlineto{\pgfqpoint{3.951855in}{2.778254in}}%
\pgfpathlineto{\pgfqpoint{3.956364in}{2.772338in}}%
\pgfpathlineto{\pgfqpoint{3.960873in}{2.782986in}}%
\pgfpathlineto{\pgfqpoint{3.965382in}{2.785352in}}%
\pgfpathlineto{\pgfqpoint{3.969891in}{2.754592in}}%
\pgfpathlineto{\pgfqpoint{3.974400in}{2.881183in}}%
\pgfpathlineto{\pgfqpoint{3.983418in}{2.794817in}}%
\pgfpathlineto{\pgfqpoint{3.987927in}{2.797183in}}%
\pgfpathlineto{\pgfqpoint{3.992436in}{2.816113in}}%
\pgfpathlineto{\pgfqpoint{3.996945in}{2.793634in}}%
\pgfpathlineto{\pgfqpoint{4.001455in}{2.845690in}}%
\pgfpathlineto{\pgfqpoint{4.005964in}{2.848056in}}%
\pgfpathlineto{\pgfqpoint{4.010473in}{2.792451in}}%
\pgfpathlineto{\pgfqpoint{4.014982in}{2.794817in}}%
\pgfpathlineto{\pgfqpoint{4.019491in}{2.805465in}}%
\pgfpathlineto{\pgfqpoint{4.024000in}{2.857521in}}%
\pgfpathlineto{\pgfqpoint{4.028509in}{2.859887in}}%
\pgfpathlineto{\pgfqpoint{4.033018in}{2.804282in}}%
\pgfpathlineto{\pgfqpoint{4.037527in}{2.872901in}}%
\pgfpathlineto{\pgfqpoint{4.042036in}{2.858704in}}%
\pgfpathlineto{\pgfqpoint{4.046545in}{2.827944in}}%
\pgfpathlineto{\pgfqpoint{4.051055in}{2.846873in}}%
\pgfpathlineto{\pgfqpoint{4.055564in}{2.857521in}}%
\pgfpathlineto{\pgfqpoint{4.060073in}{2.934423in}}%
\pgfpathlineto{\pgfqpoint{4.064582in}{2.887099in}}%
\pgfpathlineto{\pgfqpoint{4.069091in}{2.881183in}}%
\pgfpathlineto{\pgfqpoint{4.073600in}{2.825577in}}%
\pgfpathlineto{\pgfqpoint{4.078109in}{2.877634in}}%
\pgfpathlineto{\pgfqpoint{4.082618in}{2.846873in}}%
\pgfpathlineto{\pgfqpoint{4.091636in}{2.868169in}}%
\pgfpathlineto{\pgfqpoint{4.096145in}{2.845690in}}%
\pgfpathlineto{\pgfqpoint{4.100655in}{2.864620in}}%
\pgfpathlineto{\pgfqpoint{4.105164in}{2.891831in}}%
\pgfpathlineto{\pgfqpoint{4.109673in}{2.869352in}}%
\pgfpathlineto{\pgfqpoint{4.114182in}{2.871718in}}%
\pgfpathlineto{\pgfqpoint{4.118691in}{2.865803in}}%
\pgfpathlineto{\pgfqpoint{4.123200in}{2.884732in}}%
\pgfpathlineto{\pgfqpoint{4.127709in}{2.870535in}}%
\pgfpathlineto{\pgfqpoint{4.132218in}{2.930873in}}%
\pgfpathlineto{\pgfqpoint{4.136727in}{2.866986in}}%
\pgfpathlineto{\pgfqpoint{4.141236in}{2.869352in}}%
\pgfpathlineto{\pgfqpoint{4.145745in}{2.863437in}}%
\pgfpathlineto{\pgfqpoint{4.150255in}{2.956901in}}%
\pgfpathlineto{\pgfqpoint{4.154764in}{2.868169in}}%
\pgfpathlineto{\pgfqpoint{4.163782in}{2.947437in}}%
\pgfpathlineto{\pgfqpoint{4.168291in}{2.908394in}}%
\pgfpathlineto{\pgfqpoint{4.172800in}{2.919042in}}%
\pgfpathlineto{\pgfqpoint{4.181818in}{2.890648in}}%
\pgfpathlineto{\pgfqpoint{4.186327in}{2.934423in}}%
\pgfpathlineto{\pgfqpoint{4.190836in}{2.945070in}}%
\pgfpathlineto{\pgfqpoint{4.195345in}{2.922592in}}%
\pgfpathlineto{\pgfqpoint{4.199855in}{2.908394in}}%
\pgfpathlineto{\pgfqpoint{4.204364in}{2.968732in}}%
\pgfpathlineto{\pgfqpoint{4.208873in}{2.921408in}}%
\pgfpathlineto{\pgfqpoint{4.213382in}{2.948620in}}%
\pgfpathlineto{\pgfqpoint{4.217891in}{2.934423in}}%
\pgfpathlineto{\pgfqpoint{4.222400in}{2.945070in}}%
\pgfpathlineto{\pgfqpoint{4.226909in}{2.964000in}}%
\pgfpathlineto{\pgfqpoint{4.231418in}{2.949803in}}%
\pgfpathlineto{\pgfqpoint{4.235927in}{2.993577in}}%
\pgfpathlineto{\pgfqpoint{4.240436in}{2.979380in}}%
\pgfpathlineto{\pgfqpoint{4.244945in}{2.932056in}}%
\pgfpathlineto{\pgfqpoint{4.249455in}{3.008958in}}%
\pgfpathlineto{\pgfqpoint{4.253964in}{2.945070in}}%
\pgfpathlineto{\pgfqpoint{4.258473in}{2.964000in}}%
\pgfpathlineto{\pgfqpoint{4.262982in}{3.016056in}}%
\pgfpathlineto{\pgfqpoint{4.267491in}{3.001859in}}%
\pgfpathlineto{\pgfqpoint{4.272000in}{2.954535in}}%
\pgfpathlineto{\pgfqpoint{4.276509in}{2.940338in}}%
\pgfpathlineto{\pgfqpoint{4.285527in}{2.928507in}}%
\pgfpathlineto{\pgfqpoint{4.290036in}{3.013690in}}%
\pgfpathlineto{\pgfqpoint{4.299055in}{2.968732in}}%
\pgfpathlineto{\pgfqpoint{4.303564in}{3.012507in}}%
\pgfpathlineto{\pgfqpoint{4.308073in}{2.990028in}}%
\pgfpathlineto{\pgfqpoint{4.312582in}{3.017239in}}%
\pgfpathlineto{\pgfqpoint{4.317091in}{2.986479in}}%
\pgfpathlineto{\pgfqpoint{4.321600in}{2.964000in}}%
\pgfpathlineto{\pgfqpoint{4.326109in}{3.049183in}}%
\pgfpathlineto{\pgfqpoint{4.330618in}{2.985296in}}%
\pgfpathlineto{\pgfqpoint{4.339636in}{3.064563in}}%
\pgfpathlineto{\pgfqpoint{4.344145in}{2.984113in}}%
\pgfpathlineto{\pgfqpoint{4.348655in}{3.011324in}}%
\pgfpathlineto{\pgfqpoint{4.353164in}{2.997127in}}%
\pgfpathlineto{\pgfqpoint{4.357673in}{2.991211in}}%
\pgfpathlineto{\pgfqpoint{4.362182in}{3.001859in}}%
\pgfpathlineto{\pgfqpoint{4.366691in}{3.045634in}}%
\pgfpathlineto{\pgfqpoint{4.371200in}{3.014873in}}%
\pgfpathlineto{\pgfqpoint{4.375709in}{3.091775in}}%
\pgfpathlineto{\pgfqpoint{4.380218in}{3.036169in}}%
\pgfpathlineto{\pgfqpoint{4.384727in}{3.063380in}}%
\pgfpathlineto{\pgfqpoint{4.389236in}{3.049183in}}%
\pgfpathlineto{\pgfqpoint{4.393745in}{3.059831in}}%
\pgfpathlineto{\pgfqpoint{4.398255in}{3.037352in}}%
\pgfpathlineto{\pgfqpoint{4.402764in}{3.039718in}}%
\pgfpathlineto{\pgfqpoint{4.407273in}{3.050366in}}%
\pgfpathlineto{\pgfqpoint{4.411782in}{3.036169in}}%
\pgfpathlineto{\pgfqpoint{4.416291in}{3.046817in}}%
\pgfpathlineto{\pgfqpoint{4.420800in}{3.032620in}}%
\pgfpathlineto{\pgfqpoint{4.425309in}{3.084676in}}%
\pgfpathlineto{\pgfqpoint{4.429818in}{3.045634in}}%
\pgfpathlineto{\pgfqpoint{4.434327in}{3.048000in}}%
\pgfpathlineto{\pgfqpoint{4.438836in}{3.075211in}}%
\pgfpathlineto{\pgfqpoint{4.443345in}{3.094141in}}%
\pgfpathlineto{\pgfqpoint{4.447855in}{3.071662in}}%
\pgfpathlineto{\pgfqpoint{4.452364in}{3.057465in}}%
\pgfpathlineto{\pgfqpoint{4.456873in}{3.076394in}}%
\pgfpathlineto{\pgfqpoint{4.461382in}{3.087042in}}%
\pgfpathlineto{\pgfqpoint{4.465891in}{3.089408in}}%
\pgfpathlineto{\pgfqpoint{4.470400in}{3.066930in}}%
\pgfpathlineto{\pgfqpoint{4.474909in}{3.069296in}}%
\pgfpathlineto{\pgfqpoint{4.479418in}{3.079944in}}%
\pgfpathlineto{\pgfqpoint{4.483927in}{3.082310in}}%
\pgfpathlineto{\pgfqpoint{4.488436in}{3.092958in}}%
\pgfpathlineto{\pgfqpoint{4.492945in}{3.087042in}}%
\pgfpathlineto{\pgfqpoint{4.497455in}{3.105972in}}%
\pgfpathlineto{\pgfqpoint{4.501964in}{3.083493in}}%
\pgfpathlineto{\pgfqpoint{4.506473in}{3.094141in}}%
\pgfpathlineto{\pgfqpoint{4.510982in}{3.088225in}}%
\pgfpathlineto{\pgfqpoint{4.515491in}{3.098873in}}%
\pgfpathlineto{\pgfqpoint{4.520000in}{3.076394in}}%
\pgfpathlineto{\pgfqpoint{4.524509in}{3.087042in}}%
\pgfpathlineto{\pgfqpoint{4.529018in}{3.139099in}}%
\pgfpathlineto{\pgfqpoint{4.533527in}{3.091775in}}%
\pgfpathlineto{\pgfqpoint{4.538036in}{3.143831in}}%
\pgfpathlineto{\pgfqpoint{4.542545in}{3.113070in}}%
\pgfpathlineto{\pgfqpoint{4.547055in}{3.165127in}}%
\pgfpathlineto{\pgfqpoint{4.551564in}{3.101239in}}%
\pgfpathlineto{\pgfqpoint{4.556073in}{3.103606in}}%
\pgfpathlineto{\pgfqpoint{4.560582in}{3.346141in}}%
\pgfpathlineto{\pgfqpoint{4.565091in}{3.149746in}}%
\pgfpathlineto{\pgfqpoint{4.569600in}{3.176958in}}%
\pgfpathlineto{\pgfqpoint{4.574109in}{3.129634in}}%
\pgfpathlineto{\pgfqpoint{4.587636in}{3.111887in}}%
\pgfpathlineto{\pgfqpoint{4.592145in}{3.139099in}}%
\pgfpathlineto{\pgfqpoint{4.605673in}{3.146197in}}%
\pgfpathlineto{\pgfqpoint{4.610182in}{3.181690in}}%
\pgfpathlineto{\pgfqpoint{4.614691in}{3.200620in}}%
\pgfpathlineto{\pgfqpoint{4.619200in}{3.194704in}}%
\pgfpathlineto{\pgfqpoint{4.623709in}{3.122535in}}%
\pgfpathlineto{\pgfqpoint{4.628218in}{3.133183in}}%
\pgfpathlineto{\pgfqpoint{4.632727in}{3.152113in}}%
\pgfpathlineto{\pgfqpoint{4.637236in}{3.129634in}}%
\pgfpathlineto{\pgfqpoint{4.641745in}{3.165127in}}%
\pgfpathlineto{\pgfqpoint{4.646255in}{3.167493in}}%
\pgfpathlineto{\pgfqpoint{4.650764in}{3.211268in}}%
\pgfpathlineto{\pgfqpoint{4.655273in}{3.155662in}}%
\pgfpathlineto{\pgfqpoint{4.659782in}{3.240845in}}%
\pgfpathlineto{\pgfqpoint{4.664291in}{3.152113in}}%
\pgfpathlineto{\pgfqpoint{4.668800in}{3.179324in}}%
\pgfpathlineto{\pgfqpoint{4.673309in}{3.198254in}}%
\pgfpathlineto{\pgfqpoint{4.677818in}{3.250310in}}%
\pgfpathlineto{\pgfqpoint{4.682327in}{3.186423in}}%
\pgfpathlineto{\pgfqpoint{4.686836in}{3.238479in}}%
\pgfpathlineto{\pgfqpoint{4.691345in}{3.174592in}}%
\pgfpathlineto{\pgfqpoint{4.695855in}{3.168676in}}%
\pgfpathlineto{\pgfqpoint{4.700364in}{3.179324in}}%
\pgfpathlineto{\pgfqpoint{4.704873in}{3.231380in}}%
\pgfpathlineto{\pgfqpoint{4.709382in}{3.225465in}}%
\pgfpathlineto{\pgfqpoint{4.713891in}{3.252676in}}%
\pgfpathlineto{\pgfqpoint{4.718400in}{3.238479in}}%
\pgfpathlineto{\pgfqpoint{4.727418in}{3.226648in}}%
\pgfpathlineto{\pgfqpoint{4.731927in}{3.195887in}}%
\pgfpathlineto{\pgfqpoint{4.736436in}{3.181690in}}%
\pgfpathlineto{\pgfqpoint{4.740945in}{3.184056in}}%
\pgfpathlineto{\pgfqpoint{4.745455in}{3.211268in}}%
\pgfpathlineto{\pgfqpoint{4.749964in}{3.271606in}}%
\pgfpathlineto{\pgfqpoint{4.754473in}{3.207718in}}%
\pgfpathlineto{\pgfqpoint{4.763491in}{3.212451in}}%
\pgfpathlineto{\pgfqpoint{4.772509in}{3.275155in}}%
\pgfpathlineto{\pgfqpoint{4.777018in}{3.219549in}}%
\pgfpathlineto{\pgfqpoint{4.781527in}{3.279887in}}%
\pgfpathlineto{\pgfqpoint{4.786036in}{3.224282in}}%
\pgfpathlineto{\pgfqpoint{4.790545in}{3.251493in}}%
\pgfpathlineto{\pgfqpoint{4.804073in}{3.258592in}}%
\pgfpathlineto{\pgfqpoint{4.808582in}{3.302366in}}%
\pgfpathlineto{\pgfqpoint{4.813091in}{3.271606in}}%
\pgfpathlineto{\pgfqpoint{4.817600in}{3.290535in}}%
\pgfpathlineto{\pgfqpoint{4.822109in}{3.268056in}}%
\pgfpathlineto{\pgfqpoint{4.826618in}{3.253859in}}%
\pgfpathlineto{\pgfqpoint{4.831127in}{3.281070in}}%
\pgfpathlineto{\pgfqpoint{4.835636in}{3.233746in}}%
\pgfpathlineto{\pgfqpoint{4.840145in}{3.302366in}}%
\pgfpathlineto{\pgfqpoint{4.844655in}{3.313014in}}%
\pgfpathlineto{\pgfqpoint{4.849164in}{3.282254in}}%
\pgfpathlineto{\pgfqpoint{4.853673in}{3.359155in}}%
\pgfpathlineto{\pgfqpoint{4.858182in}{3.237296in}}%
\pgfpathlineto{\pgfqpoint{4.867200in}{3.366254in}}%
\pgfpathlineto{\pgfqpoint{4.871709in}{3.302366in}}%
\pgfpathlineto{\pgfqpoint{4.876218in}{3.288169in}}%
\pgfpathlineto{\pgfqpoint{4.880727in}{3.290535in}}%
\pgfpathlineto{\pgfqpoint{4.885236in}{3.317746in}}%
\pgfpathlineto{\pgfqpoint{4.889745in}{3.278704in}}%
\pgfpathlineto{\pgfqpoint{4.894255in}{3.831211in}}%
\pgfpathlineto{\pgfqpoint{4.898764in}{3.333127in}}%
\pgfpathlineto{\pgfqpoint{4.903273in}{3.310648in}}%
\pgfpathlineto{\pgfqpoint{4.907782in}{3.395831in}}%
\pgfpathlineto{\pgfqpoint{4.912291in}{3.356789in}}%
\pgfpathlineto{\pgfqpoint{4.916800in}{3.392282in}}%
\pgfpathlineto{\pgfqpoint{4.921309in}{3.311831in}}%
\pgfpathlineto{\pgfqpoint{4.925818in}{3.322479in}}%
\pgfpathlineto{\pgfqpoint{4.930327in}{3.324845in}}%
\pgfpathlineto{\pgfqpoint{4.934836in}{3.318930in}}%
\pgfpathlineto{\pgfqpoint{4.939345in}{3.337859in}}%
\pgfpathlineto{\pgfqpoint{4.943855in}{3.365070in}}%
\pgfpathlineto{\pgfqpoint{4.952873in}{3.336676in}}%
\pgfpathlineto{\pgfqpoint{4.957382in}{3.330761in}}%
\pgfpathlineto{\pgfqpoint{4.961891in}{3.316563in}}%
\pgfpathlineto{\pgfqpoint{4.966400in}{3.368620in}}%
\pgfpathlineto{\pgfqpoint{4.970909in}{3.329577in}}%
\pgfpathlineto{\pgfqpoint{4.975418in}{3.340225in}}%
\pgfpathlineto{\pgfqpoint{4.979927in}{3.367437in}}%
\pgfpathlineto{\pgfqpoint{4.984436in}{3.386366in}}%
\pgfpathlineto{\pgfqpoint{4.988945in}{3.330761in}}%
\pgfpathlineto{\pgfqpoint{4.997964in}{3.368620in}}%
\pgfpathlineto{\pgfqpoint{5.002473in}{3.362704in}}%
\pgfpathlineto{\pgfqpoint{5.006982in}{3.365070in}}%
\pgfpathlineto{\pgfqpoint{5.011491in}{3.375718in}}%
\pgfpathlineto{\pgfqpoint{5.016000in}{3.402930in}}%
\pgfpathlineto{\pgfqpoint{5.020509in}{3.380451in}}%
\pgfpathlineto{\pgfqpoint{5.025018in}{3.424225in}}%
\pgfpathlineto{\pgfqpoint{5.029527in}{3.352056in}}%
\pgfpathlineto{\pgfqpoint{5.034036in}{3.395831in}}%
\pgfpathlineto{\pgfqpoint{5.038545in}{3.373352in}}%
\pgfpathlineto{\pgfqpoint{5.043055in}{3.392282in}}%
\pgfpathlineto{\pgfqpoint{5.047564in}{3.394648in}}%
\pgfpathlineto{\pgfqpoint{5.052073in}{3.388732in}}%
\pgfpathlineto{\pgfqpoint{5.056582in}{3.349690in}}%
\pgfpathlineto{\pgfqpoint{5.061091in}{3.376901in}}%
\pgfpathlineto{\pgfqpoint{5.065600in}{3.420676in}}%
\pgfpathlineto{\pgfqpoint{5.070109in}{3.398197in}}%
\pgfpathlineto{\pgfqpoint{5.074618in}{3.367437in}}%
\pgfpathlineto{\pgfqpoint{5.079127in}{3.452620in}}%
\pgfpathlineto{\pgfqpoint{5.083636in}{3.388732in}}%
\pgfpathlineto{\pgfqpoint{5.088145in}{3.399380in}}%
\pgfpathlineto{\pgfqpoint{5.092655in}{3.401746in}}%
\pgfpathlineto{\pgfqpoint{5.097164in}{3.387549in}}%
\pgfpathlineto{\pgfqpoint{5.101673in}{3.439606in}}%
\pgfpathlineto{\pgfqpoint{5.106182in}{3.425408in}}%
\pgfpathlineto{\pgfqpoint{5.110691in}{3.427775in}}%
\pgfpathlineto{\pgfqpoint{5.115200in}{3.463268in}}%
\pgfpathlineto{\pgfqpoint{5.119709in}{3.473915in}}%
\pgfpathlineto{\pgfqpoint{5.124218in}{3.451437in}}%
\pgfpathlineto{\pgfqpoint{5.128727in}{3.462085in}}%
\pgfpathlineto{\pgfqpoint{5.133236in}{3.406479in}}%
\pgfpathlineto{\pgfqpoint{5.137745in}{3.433690in}}%
\pgfpathlineto{\pgfqpoint{5.142255in}{3.419493in}}%
\pgfpathlineto{\pgfqpoint{5.146764in}{3.463268in}}%
\pgfpathlineto{\pgfqpoint{5.155782in}{3.434873in}}%
\pgfpathlineto{\pgfqpoint{5.160291in}{3.428958in}}%
\pgfpathlineto{\pgfqpoint{5.169309in}{3.516507in}}%
\pgfpathlineto{\pgfqpoint{5.173818in}{3.518873in}}%
\pgfpathlineto{\pgfqpoint{5.178327in}{3.488113in}}%
\pgfpathlineto{\pgfqpoint{5.182836in}{3.482197in}}%
\pgfpathlineto{\pgfqpoint{5.187345in}{3.492845in}}%
\pgfpathlineto{\pgfqpoint{5.191855in}{3.478648in}}%
\pgfpathlineto{\pgfqpoint{5.196364in}{3.481014in}}%
\pgfpathlineto{\pgfqpoint{5.200873in}{3.466817in}}%
\pgfpathlineto{\pgfqpoint{5.205382in}{3.485746in}}%
\pgfpathlineto{\pgfqpoint{5.209891in}{3.471549in}}%
\pgfpathlineto{\pgfqpoint{5.214400in}{3.556732in}}%
\pgfpathlineto{\pgfqpoint{5.218909in}{3.468000in}}%
\pgfpathlineto{\pgfqpoint{5.223418in}{3.495211in}}%
\pgfpathlineto{\pgfqpoint{5.227927in}{3.497577in}}%
\pgfpathlineto{\pgfqpoint{5.232436in}{3.516507in}}%
\pgfpathlineto{\pgfqpoint{5.236945in}{3.460901in}}%
\pgfpathlineto{\pgfqpoint{5.241455in}{3.521239in}}%
\pgfpathlineto{\pgfqpoint{5.245964in}{3.457352in}}%
\pgfpathlineto{\pgfqpoint{5.250473in}{3.492845in}}%
\pgfpathlineto{\pgfqpoint{5.254982in}{3.453803in}}%
\pgfpathlineto{\pgfqpoint{5.259491in}{3.530704in}}%
\pgfpathlineto{\pgfqpoint{5.264000in}{3.499944in}}%
\pgfpathlineto{\pgfqpoint{5.268509in}{3.494028in}}%
\pgfpathlineto{\pgfqpoint{5.277527in}{3.556732in}}%
\pgfpathlineto{\pgfqpoint{5.282036in}{3.525972in}}%
\pgfpathlineto{\pgfqpoint{5.286545in}{3.511775in}}%
\pgfpathlineto{\pgfqpoint{5.291055in}{3.514141in}}%
\pgfpathlineto{\pgfqpoint{5.295564in}{3.499944in}}%
\pgfpathlineto{\pgfqpoint{5.300073in}{3.527155in}}%
\pgfpathlineto{\pgfqpoint{5.304582in}{3.562648in}}%
\pgfpathlineto{\pgfqpoint{5.309091in}{3.515324in}}%
\pgfpathlineto{\pgfqpoint{5.313600in}{3.534254in}}%
\pgfpathlineto{\pgfqpoint{5.318109in}{3.528338in}}%
\pgfpathlineto{\pgfqpoint{5.322618in}{3.547268in}}%
\pgfpathlineto{\pgfqpoint{5.327127in}{3.574479in}}%
\pgfpathlineto{\pgfqpoint{5.331636in}{3.552000in}}%
\pgfpathlineto{\pgfqpoint{5.336145in}{3.521239in}}%
\pgfpathlineto{\pgfqpoint{5.340655in}{3.523606in}}%
\pgfpathlineto{\pgfqpoint{5.345164in}{3.559099in}}%
\pgfpathlineto{\pgfqpoint{5.349673in}{3.569746in}}%
\pgfpathlineto{\pgfqpoint{5.354182in}{3.596958in}}%
\pgfpathlineto{\pgfqpoint{5.358691in}{3.640732in}}%
\pgfpathlineto{\pgfqpoint{5.363200in}{3.601690in}}%
\pgfpathlineto{\pgfqpoint{5.367709in}{3.537803in}}%
\pgfpathlineto{\pgfqpoint{5.376727in}{3.633634in}}%
\pgfpathlineto{\pgfqpoint{5.381236in}{3.578028in}}%
\pgfpathlineto{\pgfqpoint{5.385745in}{3.572113in}}%
\pgfpathlineto{\pgfqpoint{5.390255in}{3.533070in}}%
\pgfpathlineto{\pgfqpoint{5.394764in}{3.618254in}}%
\pgfpathlineto{\pgfqpoint{5.399273in}{3.570930in}}%
\pgfpathlineto{\pgfqpoint{5.403782in}{3.589859in}}%
\pgfpathlineto{\pgfqpoint{5.408291in}{3.567380in}}%
\pgfpathlineto{\pgfqpoint{5.412800in}{3.536620in}}%
\pgfpathlineto{\pgfqpoint{5.417309in}{3.663211in}}%
\pgfpathlineto{\pgfqpoint{5.421818in}{3.557915in}}%
\pgfpathlineto{\pgfqpoint{5.426327in}{3.585127in}}%
\pgfpathlineto{\pgfqpoint{5.430836in}{3.570930in}}%
\pgfpathlineto{\pgfqpoint{5.435345in}{3.581577in}}%
\pgfpathlineto{\pgfqpoint{5.439855in}{3.608789in}}%
\pgfpathlineto{\pgfqpoint{5.444364in}{3.627718in}}%
\pgfpathlineto{\pgfqpoint{5.448873in}{3.572113in}}%
\pgfpathlineto{\pgfqpoint{5.453382in}{3.632451in}}%
\pgfpathlineto{\pgfqpoint{5.457891in}{3.651380in}}%
\pgfpathlineto{\pgfqpoint{5.462400in}{3.587493in}}%
\pgfpathlineto{\pgfqpoint{5.466909in}{3.680958in}}%
\pgfpathlineto{\pgfqpoint{5.471418in}{3.617070in}}%
\pgfpathlineto{\pgfqpoint{5.475927in}{3.660845in}}%
\pgfpathlineto{\pgfqpoint{5.480436in}{3.630085in}}%
\pgfpathlineto{\pgfqpoint{5.484945in}{3.649014in}}%
\pgfpathlineto{\pgfqpoint{5.489455in}{3.609972in}}%
\pgfpathlineto{\pgfqpoint{5.493964in}{3.670310in}}%
\pgfpathlineto{\pgfqpoint{5.498473in}{3.606423in}}%
\pgfpathlineto{\pgfqpoint{5.502982in}{3.675042in}}%
\pgfpathlineto{\pgfqpoint{5.507491in}{3.611155in}}%
\pgfpathlineto{\pgfqpoint{5.512000in}{3.638366in}}%
\pgfpathlineto{\pgfqpoint{5.516509in}{3.640732in}}%
\pgfpathlineto{\pgfqpoint{5.521018in}{3.593408in}}%
\pgfpathlineto{\pgfqpoint{5.525527in}{3.695155in}}%
\pgfpathlineto{\pgfqpoint{5.530036in}{3.614704in}}%
\pgfpathlineto{\pgfqpoint{5.534545in}{3.741296in}}%
\pgfpathlineto{\pgfqpoint{5.534545in}{3.741296in}}%
\pgfusepath{stroke}%
\end{pgfscope}%
\begin{pgfscope}%
\pgfsetrectcap%
\pgfsetmiterjoin%
\pgfsetlinewidth{0.803000pt}%
\definecolor{currentstroke}{rgb}{0.000000,0.000000,0.000000}%
\pgfsetstrokecolor{currentstroke}%
\pgfsetdash{}{0pt}%
\pgfpathmoveto{\pgfqpoint{0.800000in}{0.528000in}}%
\pgfpathlineto{\pgfqpoint{0.800000in}{4.224000in}}%
\pgfusepath{stroke}%
\end{pgfscope}%
\begin{pgfscope}%
\pgfsetrectcap%
\pgfsetmiterjoin%
\pgfsetlinewidth{0.803000pt}%
\definecolor{currentstroke}{rgb}{0.000000,0.000000,0.000000}%
\pgfsetstrokecolor{currentstroke}%
\pgfsetdash{}{0pt}%
\pgfpathmoveto{\pgfqpoint{5.760000in}{0.528000in}}%
\pgfpathlineto{\pgfqpoint{5.760000in}{4.224000in}}%
\pgfusepath{stroke}%
\end{pgfscope}%
\begin{pgfscope}%
\pgfsetrectcap%
\pgfsetmiterjoin%
\pgfsetlinewidth{0.803000pt}%
\definecolor{currentstroke}{rgb}{0.000000,0.000000,0.000000}%
\pgfsetstrokecolor{currentstroke}%
\pgfsetdash{}{0pt}%
\pgfpathmoveto{\pgfqpoint{0.800000in}{0.528000in}}%
\pgfpathlineto{\pgfqpoint{5.760000in}{0.528000in}}%
\pgfusepath{stroke}%
\end{pgfscope}%
\begin{pgfscope}%
\pgfsetrectcap%
\pgfsetmiterjoin%
\pgfsetlinewidth{0.803000pt}%
\definecolor{currentstroke}{rgb}{0.000000,0.000000,0.000000}%
\pgfsetstrokecolor{currentstroke}%
\pgfsetdash{}{0pt}%
\pgfpathmoveto{\pgfqpoint{0.800000in}{4.224000in}}%
\pgfpathlineto{\pgfqpoint{5.760000in}{4.224000in}}%
\pgfusepath{stroke}%
\end{pgfscope}%
\begin{pgfscope}%
\definecolor{textcolor}{rgb}{0.000000,0.000000,0.000000}%
\pgfsetstrokecolor{textcolor}%
\pgfsetfillcolor{textcolor}%
\pgftext[x=3.280000in,y=4.307333in,,base]{\color{textcolor}\ttfamily\fontsize{12.000000}{14.400000}\selectfont Memory vs Input size}%
\end{pgfscope}%
\begin{pgfscope}%
\pgfsetbuttcap%
\pgfsetmiterjoin%
\definecolor{currentfill}{rgb}{1.000000,1.000000,1.000000}%
\pgfsetfillcolor{currentfill}%
\pgfsetfillopacity{0.800000}%
\pgfsetlinewidth{1.003750pt}%
\definecolor{currentstroke}{rgb}{0.800000,0.800000,0.800000}%
\pgfsetstrokecolor{currentstroke}%
\pgfsetstrokeopacity{0.800000}%
\pgfsetdash{}{0pt}%
\pgfpathmoveto{\pgfqpoint{0.897222in}{3.088923in}}%
\pgfpathlineto{\pgfqpoint{2.094230in}{3.088923in}}%
\pgfpathquadraticcurveto{\pgfqpoint{2.122008in}{3.088923in}}{\pgfqpoint{2.122008in}{3.116701in}}%
\pgfpathlineto{\pgfqpoint{2.122008in}{4.126778in}}%
\pgfpathquadraticcurveto{\pgfqpoint{2.122008in}{4.154556in}}{\pgfqpoint{2.094230in}{4.154556in}}%
\pgfpathlineto{\pgfqpoint{0.897222in}{4.154556in}}%
\pgfpathquadraticcurveto{\pgfqpoint{0.869444in}{4.154556in}}{\pgfqpoint{0.869444in}{4.126778in}}%
\pgfpathlineto{\pgfqpoint{0.869444in}{3.116701in}}%
\pgfpathquadraticcurveto{\pgfqpoint{0.869444in}{3.088923in}}{\pgfqpoint{0.897222in}{3.088923in}}%
\pgfpathlineto{\pgfqpoint{0.897222in}{3.088923in}}%
\pgfpathclose%
\pgfusepath{stroke,fill}%
\end{pgfscope}%
\begin{pgfscope}%
\pgfsetrectcap%
\pgfsetroundjoin%
\pgfsetlinewidth{1.505625pt}%
\definecolor{currentstroke}{rgb}{1.000000,0.000000,0.000000}%
\pgfsetstrokecolor{currentstroke}%
\pgfsetdash{}{0pt}%
\pgfpathmoveto{\pgfqpoint{0.925000in}{4.041342in}}%
\pgfpathlineto{\pgfqpoint{1.063889in}{4.041342in}}%
\pgfpathlineto{\pgfqpoint{1.202778in}{4.041342in}}%
\pgfusepath{stroke}%
\end{pgfscope}%
\begin{pgfscope}%
\definecolor{textcolor}{rgb}{0.000000,0.000000,0.000000}%
\pgfsetstrokecolor{textcolor}%
\pgfsetfillcolor{textcolor}%
\pgftext[x=1.313889in,y=3.992731in,left,base]{\color{textcolor}\ttfamily\fontsize{10.000000}{12.000000}\selectfont Bubble}%
\end{pgfscope}%
\begin{pgfscope}%
\pgfsetrectcap%
\pgfsetroundjoin%
\pgfsetlinewidth{1.505625pt}%
\definecolor{currentstroke}{rgb}{0.486275,0.988235,0.000000}%
\pgfsetstrokecolor{currentstroke}%
\pgfsetdash{}{0pt}%
\pgfpathmoveto{\pgfqpoint{0.925000in}{3.836739in}}%
\pgfpathlineto{\pgfqpoint{1.063889in}{3.836739in}}%
\pgfpathlineto{\pgfqpoint{1.202778in}{3.836739in}}%
\pgfusepath{stroke}%
\end{pgfscope}%
\begin{pgfscope}%
\definecolor{textcolor}{rgb}{0.000000,0.000000,0.000000}%
\pgfsetstrokecolor{textcolor}%
\pgfsetfillcolor{textcolor}%
\pgftext[x=1.313889in,y=3.788128in,left,base]{\color{textcolor}\ttfamily\fontsize{10.000000}{12.000000}\selectfont Selection}%
\end{pgfscope}%
\begin{pgfscope}%
\pgfsetrectcap%
\pgfsetroundjoin%
\pgfsetlinewidth{1.505625pt}%
\definecolor{currentstroke}{rgb}{0.000000,1.000000,0.498039}%
\pgfsetstrokecolor{currentstroke}%
\pgfsetdash{}{0pt}%
\pgfpathmoveto{\pgfqpoint{0.925000in}{3.632136in}}%
\pgfpathlineto{\pgfqpoint{1.063889in}{3.632136in}}%
\pgfpathlineto{\pgfqpoint{1.202778in}{3.632136in}}%
\pgfusepath{stroke}%
\end{pgfscope}%
\begin{pgfscope}%
\definecolor{textcolor}{rgb}{0.000000,0.000000,0.000000}%
\pgfsetstrokecolor{textcolor}%
\pgfsetfillcolor{textcolor}%
\pgftext[x=1.313889in,y=3.583525in,left,base]{\color{textcolor}\ttfamily\fontsize{10.000000}{12.000000}\selectfont Insertion}%
\end{pgfscope}%
\begin{pgfscope}%
\pgfsetrectcap%
\pgfsetroundjoin%
\pgfsetlinewidth{1.505625pt}%
\definecolor{currentstroke}{rgb}{0.000000,1.000000,1.000000}%
\pgfsetstrokecolor{currentstroke}%
\pgfsetdash{}{0pt}%
\pgfpathmoveto{\pgfqpoint{0.925000in}{3.427532in}}%
\pgfpathlineto{\pgfqpoint{1.063889in}{3.427532in}}%
\pgfpathlineto{\pgfqpoint{1.202778in}{3.427532in}}%
\pgfusepath{stroke}%
\end{pgfscope}%
\begin{pgfscope}%
\definecolor{textcolor}{rgb}{0.000000,0.000000,0.000000}%
\pgfsetstrokecolor{textcolor}%
\pgfsetfillcolor{textcolor}%
\pgftext[x=1.313889in,y=3.378921in,left,base]{\color{textcolor}\ttfamily\fontsize{10.000000}{12.000000}\selectfont Merge}%
\end{pgfscope}%
\begin{pgfscope}%
\pgfsetrectcap%
\pgfsetroundjoin%
\pgfsetlinewidth{1.505625pt}%
\definecolor{currentstroke}{rgb}{1.000000,0.000000,1.000000}%
\pgfsetstrokecolor{currentstroke}%
\pgfsetdash{}{0pt}%
\pgfpathmoveto{\pgfqpoint{0.925000in}{3.221980in}}%
\pgfpathlineto{\pgfqpoint{1.063889in}{3.221980in}}%
\pgfpathlineto{\pgfqpoint{1.202778in}{3.221980in}}%
\pgfusepath{stroke}%
\end{pgfscope}%
\begin{pgfscope}%
\definecolor{textcolor}{rgb}{0.000000,0.000000,0.000000}%
\pgfsetstrokecolor{textcolor}%
\pgfsetfillcolor{textcolor}%
\pgftext[x=1.313889in,y=3.173369in,left,base]{\color{textcolor}\ttfamily\fontsize{10.000000}{12.000000}\selectfont Quick}%
\end{pgfscope}%
\end{pgfpicture}%
\makeatother%
\endgroup%

%% Creator: Matplotlib, PGF backend
%%
%% To include the figure in your LaTeX document, write
%%   \input{<filename>.pgf}
%%
%% Make sure the required packages are loaded in your preamble
%%   \usepackage{pgf}
%%
%% Also ensure that all the required font packages are loaded; for instance,
%% the lmodern package is sometimes necessary when using math font.
%%   \usepackage{lmodern}
%%
%% Figures using additional raster images can only be included by \input if
%% they are in the same directory as the main LaTeX file. For loading figures
%% from other directories you can use the `import` package
%%   \usepackage{import}
%%
%% and then include the figures with
%%   \import{<path to file>}{<filename>.pgf}
%%
%% Matplotlib used the following preamble
%%   \usepackage{fontspec}
%%   \setmainfont{DejaVuSerif.ttf}[Path=\detokenize{/home/dbk/.local/lib/python3.10/site-packages/matplotlib/mpl-data/fonts/ttf/}]
%%   \setsansfont{DejaVuSans.ttf}[Path=\detokenize{/home/dbk/.local/lib/python3.10/site-packages/matplotlib/mpl-data/fonts/ttf/}]
%%   \setmonofont{DejaVuSansMono.ttf}[Path=\detokenize{/home/dbk/.local/lib/python3.10/site-packages/matplotlib/mpl-data/fonts/ttf/}]
%%
\begingroup%
\makeatletter%
\begin{pgfpicture}%
\pgfpathrectangle{\pgfpointorigin}{\pgfqpoint{6.400000in}{4.800000in}}%
\pgfusepath{use as bounding box, clip}%
\begin{pgfscope}%
\pgfsetbuttcap%
\pgfsetmiterjoin%
\definecolor{currentfill}{rgb}{1.000000,1.000000,1.000000}%
\pgfsetfillcolor{currentfill}%
\pgfsetlinewidth{0.000000pt}%
\definecolor{currentstroke}{rgb}{1.000000,1.000000,1.000000}%
\pgfsetstrokecolor{currentstroke}%
\pgfsetdash{}{0pt}%
\pgfpathmoveto{\pgfqpoint{0.000000in}{0.000000in}}%
\pgfpathlineto{\pgfqpoint{6.400000in}{0.000000in}}%
\pgfpathlineto{\pgfqpoint{6.400000in}{4.800000in}}%
\pgfpathlineto{\pgfqpoint{0.000000in}{4.800000in}}%
\pgfpathlineto{\pgfqpoint{0.000000in}{0.000000in}}%
\pgfpathclose%
\pgfusepath{fill}%
\end{pgfscope}%
\begin{pgfscope}%
\pgfsetbuttcap%
\pgfsetmiterjoin%
\definecolor{currentfill}{rgb}{1.000000,1.000000,1.000000}%
\pgfsetfillcolor{currentfill}%
\pgfsetlinewidth{0.000000pt}%
\definecolor{currentstroke}{rgb}{0.000000,0.000000,0.000000}%
\pgfsetstrokecolor{currentstroke}%
\pgfsetstrokeopacity{0.000000}%
\pgfsetdash{}{0pt}%
\pgfpathmoveto{\pgfqpoint{0.800000in}{0.528000in}}%
\pgfpathlineto{\pgfqpoint{5.760000in}{0.528000in}}%
\pgfpathlineto{\pgfqpoint{5.760000in}{4.224000in}}%
\pgfpathlineto{\pgfqpoint{0.800000in}{4.224000in}}%
\pgfpathlineto{\pgfqpoint{0.800000in}{0.528000in}}%
\pgfpathclose%
\pgfusepath{fill}%
\end{pgfscope}%
\begin{pgfscope}%
\pgfsetbuttcap%
\pgfsetroundjoin%
\definecolor{currentfill}{rgb}{0.000000,0.000000,0.000000}%
\pgfsetfillcolor{currentfill}%
\pgfsetlinewidth{0.803000pt}%
\definecolor{currentstroke}{rgb}{0.000000,0.000000,0.000000}%
\pgfsetstrokecolor{currentstroke}%
\pgfsetdash{}{0pt}%
\pgfsys@defobject{currentmarker}{\pgfqpoint{0.000000in}{-0.048611in}}{\pgfqpoint{0.000000in}{0.000000in}}{%
\pgfpathmoveto{\pgfqpoint{0.000000in}{0.000000in}}%
\pgfpathlineto{\pgfqpoint{0.000000in}{-0.048611in}}%
\pgfusepath{stroke,fill}%
}%
\begin{pgfscope}%
\pgfsys@transformshift{1.020945in}{0.528000in}%
\pgfsys@useobject{currentmarker}{}%
\end{pgfscope}%
\end{pgfscope}%
\begin{pgfscope}%
\definecolor{textcolor}{rgb}{0.000000,0.000000,0.000000}%
\pgfsetstrokecolor{textcolor}%
\pgfsetfillcolor{textcolor}%
\pgftext[x=1.020945in,y=0.430778in,,top]{\color{textcolor}\ttfamily\fontsize{10.000000}{12.000000}\selectfont 0}%
\end{pgfscope}%
\begin{pgfscope}%
\pgfsetbuttcap%
\pgfsetroundjoin%
\definecolor{currentfill}{rgb}{0.000000,0.000000,0.000000}%
\pgfsetfillcolor{currentfill}%
\pgfsetlinewidth{0.803000pt}%
\definecolor{currentstroke}{rgb}{0.000000,0.000000,0.000000}%
\pgfsetstrokecolor{currentstroke}%
\pgfsetdash{}{0pt}%
\pgfsys@defobject{currentmarker}{\pgfqpoint{0.000000in}{-0.048611in}}{\pgfqpoint{0.000000in}{0.000000in}}{%
\pgfpathmoveto{\pgfqpoint{0.000000in}{0.000000in}}%
\pgfpathlineto{\pgfqpoint{0.000000in}{-0.048611in}}%
\pgfusepath{stroke,fill}%
}%
\begin{pgfscope}%
\pgfsys@transformshift{1.922764in}{0.528000in}%
\pgfsys@useobject{currentmarker}{}%
\end{pgfscope}%
\end{pgfscope}%
\begin{pgfscope}%
\definecolor{textcolor}{rgb}{0.000000,0.000000,0.000000}%
\pgfsetstrokecolor{textcolor}%
\pgfsetfillcolor{textcolor}%
\pgftext[x=1.922764in,y=0.430778in,,top]{\color{textcolor}\ttfamily\fontsize{10.000000}{12.000000}\selectfont 200}%
\end{pgfscope}%
\begin{pgfscope}%
\pgfsetbuttcap%
\pgfsetroundjoin%
\definecolor{currentfill}{rgb}{0.000000,0.000000,0.000000}%
\pgfsetfillcolor{currentfill}%
\pgfsetlinewidth{0.803000pt}%
\definecolor{currentstroke}{rgb}{0.000000,0.000000,0.000000}%
\pgfsetstrokecolor{currentstroke}%
\pgfsetdash{}{0pt}%
\pgfsys@defobject{currentmarker}{\pgfqpoint{0.000000in}{-0.048611in}}{\pgfqpoint{0.000000in}{0.000000in}}{%
\pgfpathmoveto{\pgfqpoint{0.000000in}{0.000000in}}%
\pgfpathlineto{\pgfqpoint{0.000000in}{-0.048611in}}%
\pgfusepath{stroke,fill}%
}%
\begin{pgfscope}%
\pgfsys@transformshift{2.824582in}{0.528000in}%
\pgfsys@useobject{currentmarker}{}%
\end{pgfscope}%
\end{pgfscope}%
\begin{pgfscope}%
\definecolor{textcolor}{rgb}{0.000000,0.000000,0.000000}%
\pgfsetstrokecolor{textcolor}%
\pgfsetfillcolor{textcolor}%
\pgftext[x=2.824582in,y=0.430778in,,top]{\color{textcolor}\ttfamily\fontsize{10.000000}{12.000000}\selectfont 400}%
\end{pgfscope}%
\begin{pgfscope}%
\pgfsetbuttcap%
\pgfsetroundjoin%
\definecolor{currentfill}{rgb}{0.000000,0.000000,0.000000}%
\pgfsetfillcolor{currentfill}%
\pgfsetlinewidth{0.803000pt}%
\definecolor{currentstroke}{rgb}{0.000000,0.000000,0.000000}%
\pgfsetstrokecolor{currentstroke}%
\pgfsetdash{}{0pt}%
\pgfsys@defobject{currentmarker}{\pgfqpoint{0.000000in}{-0.048611in}}{\pgfqpoint{0.000000in}{0.000000in}}{%
\pgfpathmoveto{\pgfqpoint{0.000000in}{0.000000in}}%
\pgfpathlineto{\pgfqpoint{0.000000in}{-0.048611in}}%
\pgfusepath{stroke,fill}%
}%
\begin{pgfscope}%
\pgfsys@transformshift{3.726400in}{0.528000in}%
\pgfsys@useobject{currentmarker}{}%
\end{pgfscope}%
\end{pgfscope}%
\begin{pgfscope}%
\definecolor{textcolor}{rgb}{0.000000,0.000000,0.000000}%
\pgfsetstrokecolor{textcolor}%
\pgfsetfillcolor{textcolor}%
\pgftext[x=3.726400in,y=0.430778in,,top]{\color{textcolor}\ttfamily\fontsize{10.000000}{12.000000}\selectfont 600}%
\end{pgfscope}%
\begin{pgfscope}%
\pgfsetbuttcap%
\pgfsetroundjoin%
\definecolor{currentfill}{rgb}{0.000000,0.000000,0.000000}%
\pgfsetfillcolor{currentfill}%
\pgfsetlinewidth{0.803000pt}%
\definecolor{currentstroke}{rgb}{0.000000,0.000000,0.000000}%
\pgfsetstrokecolor{currentstroke}%
\pgfsetdash{}{0pt}%
\pgfsys@defobject{currentmarker}{\pgfqpoint{0.000000in}{-0.048611in}}{\pgfqpoint{0.000000in}{0.000000in}}{%
\pgfpathmoveto{\pgfqpoint{0.000000in}{0.000000in}}%
\pgfpathlineto{\pgfqpoint{0.000000in}{-0.048611in}}%
\pgfusepath{stroke,fill}%
}%
\begin{pgfscope}%
\pgfsys@transformshift{4.628218in}{0.528000in}%
\pgfsys@useobject{currentmarker}{}%
\end{pgfscope}%
\end{pgfscope}%
\begin{pgfscope}%
\definecolor{textcolor}{rgb}{0.000000,0.000000,0.000000}%
\pgfsetstrokecolor{textcolor}%
\pgfsetfillcolor{textcolor}%
\pgftext[x=4.628218in,y=0.430778in,,top]{\color{textcolor}\ttfamily\fontsize{10.000000}{12.000000}\selectfont 800}%
\end{pgfscope}%
\begin{pgfscope}%
\pgfsetbuttcap%
\pgfsetroundjoin%
\definecolor{currentfill}{rgb}{0.000000,0.000000,0.000000}%
\pgfsetfillcolor{currentfill}%
\pgfsetlinewidth{0.803000pt}%
\definecolor{currentstroke}{rgb}{0.000000,0.000000,0.000000}%
\pgfsetstrokecolor{currentstroke}%
\pgfsetdash{}{0pt}%
\pgfsys@defobject{currentmarker}{\pgfqpoint{0.000000in}{-0.048611in}}{\pgfqpoint{0.000000in}{0.000000in}}{%
\pgfpathmoveto{\pgfqpoint{0.000000in}{0.000000in}}%
\pgfpathlineto{\pgfqpoint{0.000000in}{-0.048611in}}%
\pgfusepath{stroke,fill}%
}%
\begin{pgfscope}%
\pgfsys@transformshift{5.530036in}{0.528000in}%
\pgfsys@useobject{currentmarker}{}%
\end{pgfscope}%
\end{pgfscope}%
\begin{pgfscope}%
\definecolor{textcolor}{rgb}{0.000000,0.000000,0.000000}%
\pgfsetstrokecolor{textcolor}%
\pgfsetfillcolor{textcolor}%
\pgftext[x=5.530036in,y=0.430778in,,top]{\color{textcolor}\ttfamily\fontsize{10.000000}{12.000000}\selectfont 1000}%
\end{pgfscope}%
\begin{pgfscope}%
\definecolor{textcolor}{rgb}{0.000000,0.000000,0.000000}%
\pgfsetstrokecolor{textcolor}%
\pgfsetfillcolor{textcolor}%
\pgftext[x=3.280000in,y=0.240063in,,top]{\color{textcolor}\ttfamily\fontsize{10.000000}{12.000000}\selectfont Size of Array}%
\end{pgfscope}%
\begin{pgfscope}%
\pgfsetbuttcap%
\pgfsetroundjoin%
\definecolor{currentfill}{rgb}{0.000000,0.000000,0.000000}%
\pgfsetfillcolor{currentfill}%
\pgfsetlinewidth{0.803000pt}%
\definecolor{currentstroke}{rgb}{0.000000,0.000000,0.000000}%
\pgfsetstrokecolor{currentstroke}%
\pgfsetdash{}{0pt}%
\pgfsys@defobject{currentmarker}{\pgfqpoint{-0.048611in}{0.000000in}}{\pgfqpoint{-0.000000in}{0.000000in}}{%
\pgfpathmoveto{\pgfqpoint{-0.000000in}{0.000000in}}%
\pgfpathlineto{\pgfqpoint{-0.048611in}{0.000000in}}%
\pgfusepath{stroke,fill}%
}%
\begin{pgfscope}%
\pgfsys@transformshift{0.800000in}{0.695450in}%
\pgfsys@useobject{currentmarker}{}%
\end{pgfscope}%
\end{pgfscope}%
\begin{pgfscope}%
\definecolor{textcolor}{rgb}{0.000000,0.000000,0.000000}%
\pgfsetstrokecolor{textcolor}%
\pgfsetfillcolor{textcolor}%
\pgftext[x=0.619160in, y=0.642315in, left, base]{\color{textcolor}\ttfamily\fontsize{10.000000}{12.000000}\selectfont 0}%
\end{pgfscope}%
\begin{pgfscope}%
\pgfsetbuttcap%
\pgfsetroundjoin%
\definecolor{currentfill}{rgb}{0.000000,0.000000,0.000000}%
\pgfsetfillcolor{currentfill}%
\pgfsetlinewidth{0.803000pt}%
\definecolor{currentstroke}{rgb}{0.000000,0.000000,0.000000}%
\pgfsetstrokecolor{currentstroke}%
\pgfsetdash{}{0pt}%
\pgfsys@defobject{currentmarker}{\pgfqpoint{-0.048611in}{0.000000in}}{\pgfqpoint{-0.000000in}{0.000000in}}{%
\pgfpathmoveto{\pgfqpoint{-0.000000in}{0.000000in}}%
\pgfpathlineto{\pgfqpoint{-0.048611in}{0.000000in}}%
\pgfusepath{stroke,fill}%
}%
\begin{pgfscope}%
\pgfsys@transformshift{0.800000in}{1.251418in}%
\pgfsys@useobject{currentmarker}{}%
\end{pgfscope}%
\end{pgfscope}%
\begin{pgfscope}%
\definecolor{textcolor}{rgb}{0.000000,0.000000,0.000000}%
\pgfsetstrokecolor{textcolor}%
\pgfsetfillcolor{textcolor}%
\pgftext[x=0.201069in, y=1.198283in, left, base]{\color{textcolor}\ttfamily\fontsize{10.000000}{12.000000}\selectfont 100000}%
\end{pgfscope}%
\begin{pgfscope}%
\pgfsetbuttcap%
\pgfsetroundjoin%
\definecolor{currentfill}{rgb}{0.000000,0.000000,0.000000}%
\pgfsetfillcolor{currentfill}%
\pgfsetlinewidth{0.803000pt}%
\definecolor{currentstroke}{rgb}{0.000000,0.000000,0.000000}%
\pgfsetstrokecolor{currentstroke}%
\pgfsetdash{}{0pt}%
\pgfsys@defobject{currentmarker}{\pgfqpoint{-0.048611in}{0.000000in}}{\pgfqpoint{-0.000000in}{0.000000in}}{%
\pgfpathmoveto{\pgfqpoint{-0.000000in}{0.000000in}}%
\pgfpathlineto{\pgfqpoint{-0.048611in}{0.000000in}}%
\pgfusepath{stroke,fill}%
}%
\begin{pgfscope}%
\pgfsys@transformshift{0.800000in}{1.807386in}%
\pgfsys@useobject{currentmarker}{}%
\end{pgfscope}%
\end{pgfscope}%
\begin{pgfscope}%
\definecolor{textcolor}{rgb}{0.000000,0.000000,0.000000}%
\pgfsetstrokecolor{textcolor}%
\pgfsetfillcolor{textcolor}%
\pgftext[x=0.201069in, y=1.754252in, left, base]{\color{textcolor}\ttfamily\fontsize{10.000000}{12.000000}\selectfont 200000}%
\end{pgfscope}%
\begin{pgfscope}%
\pgfsetbuttcap%
\pgfsetroundjoin%
\definecolor{currentfill}{rgb}{0.000000,0.000000,0.000000}%
\pgfsetfillcolor{currentfill}%
\pgfsetlinewidth{0.803000pt}%
\definecolor{currentstroke}{rgb}{0.000000,0.000000,0.000000}%
\pgfsetstrokecolor{currentstroke}%
\pgfsetdash{}{0pt}%
\pgfsys@defobject{currentmarker}{\pgfqpoint{-0.048611in}{0.000000in}}{\pgfqpoint{-0.000000in}{0.000000in}}{%
\pgfpathmoveto{\pgfqpoint{-0.000000in}{0.000000in}}%
\pgfpathlineto{\pgfqpoint{-0.048611in}{0.000000in}}%
\pgfusepath{stroke,fill}%
}%
\begin{pgfscope}%
\pgfsys@transformshift{0.800000in}{2.363355in}%
\pgfsys@useobject{currentmarker}{}%
\end{pgfscope}%
\end{pgfscope}%
\begin{pgfscope}%
\definecolor{textcolor}{rgb}{0.000000,0.000000,0.000000}%
\pgfsetstrokecolor{textcolor}%
\pgfsetfillcolor{textcolor}%
\pgftext[x=0.201069in, y=2.310220in, left, base]{\color{textcolor}\ttfamily\fontsize{10.000000}{12.000000}\selectfont 300000}%
\end{pgfscope}%
\begin{pgfscope}%
\pgfsetbuttcap%
\pgfsetroundjoin%
\definecolor{currentfill}{rgb}{0.000000,0.000000,0.000000}%
\pgfsetfillcolor{currentfill}%
\pgfsetlinewidth{0.803000pt}%
\definecolor{currentstroke}{rgb}{0.000000,0.000000,0.000000}%
\pgfsetstrokecolor{currentstroke}%
\pgfsetdash{}{0pt}%
\pgfsys@defobject{currentmarker}{\pgfqpoint{-0.048611in}{0.000000in}}{\pgfqpoint{-0.000000in}{0.000000in}}{%
\pgfpathmoveto{\pgfqpoint{-0.000000in}{0.000000in}}%
\pgfpathlineto{\pgfqpoint{-0.048611in}{0.000000in}}%
\pgfusepath{stroke,fill}%
}%
\begin{pgfscope}%
\pgfsys@transformshift{0.800000in}{2.919323in}%
\pgfsys@useobject{currentmarker}{}%
\end{pgfscope}%
\end{pgfscope}%
\begin{pgfscope}%
\definecolor{textcolor}{rgb}{0.000000,0.000000,0.000000}%
\pgfsetstrokecolor{textcolor}%
\pgfsetfillcolor{textcolor}%
\pgftext[x=0.201069in, y=2.866188in, left, base]{\color{textcolor}\ttfamily\fontsize{10.000000}{12.000000}\selectfont 400000}%
\end{pgfscope}%
\begin{pgfscope}%
\pgfsetbuttcap%
\pgfsetroundjoin%
\definecolor{currentfill}{rgb}{0.000000,0.000000,0.000000}%
\pgfsetfillcolor{currentfill}%
\pgfsetlinewidth{0.803000pt}%
\definecolor{currentstroke}{rgb}{0.000000,0.000000,0.000000}%
\pgfsetstrokecolor{currentstroke}%
\pgfsetdash{}{0pt}%
\pgfsys@defobject{currentmarker}{\pgfqpoint{-0.048611in}{0.000000in}}{\pgfqpoint{-0.000000in}{0.000000in}}{%
\pgfpathmoveto{\pgfqpoint{-0.000000in}{0.000000in}}%
\pgfpathlineto{\pgfqpoint{-0.048611in}{0.000000in}}%
\pgfusepath{stroke,fill}%
}%
\begin{pgfscope}%
\pgfsys@transformshift{0.800000in}{3.475291in}%
\pgfsys@useobject{currentmarker}{}%
\end{pgfscope}%
\end{pgfscope}%
\begin{pgfscope}%
\definecolor{textcolor}{rgb}{0.000000,0.000000,0.000000}%
\pgfsetstrokecolor{textcolor}%
\pgfsetfillcolor{textcolor}%
\pgftext[x=0.201069in, y=3.422157in, left, base]{\color{textcolor}\ttfamily\fontsize{10.000000}{12.000000}\selectfont 500000}%
\end{pgfscope}%
\begin{pgfscope}%
\pgfsetbuttcap%
\pgfsetroundjoin%
\definecolor{currentfill}{rgb}{0.000000,0.000000,0.000000}%
\pgfsetfillcolor{currentfill}%
\pgfsetlinewidth{0.803000pt}%
\definecolor{currentstroke}{rgb}{0.000000,0.000000,0.000000}%
\pgfsetstrokecolor{currentstroke}%
\pgfsetdash{}{0pt}%
\pgfsys@defobject{currentmarker}{\pgfqpoint{-0.048611in}{0.000000in}}{\pgfqpoint{-0.000000in}{0.000000in}}{%
\pgfpathmoveto{\pgfqpoint{-0.000000in}{0.000000in}}%
\pgfpathlineto{\pgfqpoint{-0.048611in}{0.000000in}}%
\pgfusepath{stroke,fill}%
}%
\begin{pgfscope}%
\pgfsys@transformshift{0.800000in}{4.031259in}%
\pgfsys@useobject{currentmarker}{}%
\end{pgfscope}%
\end{pgfscope}%
\begin{pgfscope}%
\definecolor{textcolor}{rgb}{0.000000,0.000000,0.000000}%
\pgfsetstrokecolor{textcolor}%
\pgfsetfillcolor{textcolor}%
\pgftext[x=0.201069in, y=3.978125in, left, base]{\color{textcolor}\ttfamily\fontsize{10.000000}{12.000000}\selectfont 600000}%
\end{pgfscope}%
\begin{pgfscope}%
\definecolor{textcolor}{rgb}{0.000000,0.000000,0.000000}%
\pgfsetstrokecolor{textcolor}%
\pgfsetfillcolor{textcolor}%
\pgftext[x=0.145513in,y=2.376000in,,bottom,rotate=90.000000]{\color{textcolor}\ttfamily\fontsize{10.000000}{12.000000}\selectfont Comparisons}%
\end{pgfscope}%
\begin{pgfscope}%
\pgfpathrectangle{\pgfqpoint{0.800000in}{0.528000in}}{\pgfqpoint{4.960000in}{3.696000in}}%
\pgfusepath{clip}%
\pgfsetrectcap%
\pgfsetroundjoin%
\pgfsetlinewidth{1.505625pt}%
\definecolor{currentstroke}{rgb}{1.000000,0.000000,0.000000}%
\pgfsetstrokecolor{currentstroke}%
\pgfsetdash{}{0pt}%
\pgfpathmoveto{\pgfqpoint{1.025455in}{0.722970in}}%
\pgfpathlineto{\pgfqpoint{1.115636in}{0.735146in}}%
\pgfpathlineto{\pgfqpoint{1.205818in}{0.749545in}}%
\pgfpathlineto{\pgfqpoint{1.296000in}{0.766169in}}%
\pgfpathlineto{\pgfqpoint{1.386182in}{0.785016in}}%
\pgfpathlineto{\pgfqpoint{1.476364in}{0.806087in}}%
\pgfpathlineto{\pgfqpoint{1.566545in}{0.829382in}}%
\pgfpathlineto{\pgfqpoint{1.656727in}{0.854901in}}%
\pgfpathlineto{\pgfqpoint{1.746909in}{0.882644in}}%
\pgfpathlineto{\pgfqpoint{1.837091in}{0.912611in}}%
\pgfpathlineto{\pgfqpoint{1.931782in}{0.946469in}}%
\pgfpathlineto{\pgfqpoint{2.026473in}{0.982780in}}%
\pgfpathlineto{\pgfqpoint{2.121164in}{1.021542in}}%
\pgfpathlineto{\pgfqpoint{2.215855in}{1.062756in}}%
\pgfpathlineto{\pgfqpoint{2.310545in}{1.106421in}}%
\pgfpathlineto{\pgfqpoint{2.405236in}{1.152539in}}%
\pgfpathlineto{\pgfqpoint{2.499927in}{1.201108in}}%
\pgfpathlineto{\pgfqpoint{2.594618in}{1.252130in}}%
\pgfpathlineto{\pgfqpoint{2.689309in}{1.305603in}}%
\pgfpathlineto{\pgfqpoint{2.784000in}{1.361527in}}%
\pgfpathlineto{\pgfqpoint{2.883200in}{1.422745in}}%
\pgfpathlineto{\pgfqpoint{2.982400in}{1.486654in}}%
\pgfpathlineto{\pgfqpoint{3.081600in}{1.553253in}}%
\pgfpathlineto{\pgfqpoint{3.180800in}{1.622543in}}%
\pgfpathlineto{\pgfqpoint{3.280000in}{1.694525in}}%
\pgfpathlineto{\pgfqpoint{3.379200in}{1.769197in}}%
\pgfpathlineto{\pgfqpoint{3.478400in}{1.846560in}}%
\pgfpathlineto{\pgfqpoint{3.582109in}{1.930316in}}%
\pgfpathlineto{\pgfqpoint{3.685818in}{2.017014in}}%
\pgfpathlineto{\pgfqpoint{3.789527in}{2.106653in}}%
\pgfpathlineto{\pgfqpoint{3.893236in}{2.199233in}}%
\pgfpathlineto{\pgfqpoint{3.996945in}{2.294754in}}%
\pgfpathlineto{\pgfqpoint{4.100655in}{2.393216in}}%
\pgfpathlineto{\pgfqpoint{4.208873in}{2.499094in}}%
\pgfpathlineto{\pgfqpoint{4.317091in}{2.608175in}}%
\pgfpathlineto{\pgfqpoint{4.425309in}{2.720459in}}%
\pgfpathlineto{\pgfqpoint{4.533527in}{2.835944in}}%
\pgfpathlineto{\pgfqpoint{4.641745in}{2.954632in}}%
\pgfpathlineto{\pgfqpoint{4.749964in}{3.076523in}}%
\pgfpathlineto{\pgfqpoint{4.862691in}{3.206897in}}%
\pgfpathlineto{\pgfqpoint{4.975418in}{3.340747in}}%
\pgfpathlineto{\pgfqpoint{5.088145in}{3.478071in}}%
\pgfpathlineto{\pgfqpoint{5.200873in}{3.618870in}}%
\pgfpathlineto{\pgfqpoint{5.313600in}{3.763144in}}%
\pgfpathlineto{\pgfqpoint{5.430836in}{3.916874in}}%
\pgfpathlineto{\pgfqpoint{5.534545in}{4.056000in}}%
\pgfpathlineto{\pgfqpoint{5.534545in}{4.056000in}}%
\pgfusepath{stroke}%
\end{pgfscope}%
\begin{pgfscope}%
\pgfpathrectangle{\pgfqpoint{0.800000in}{0.528000in}}{\pgfqpoint{4.960000in}{3.696000in}}%
\pgfusepath{clip}%
\pgfsetrectcap%
\pgfsetroundjoin%
\pgfsetlinewidth{1.505625pt}%
\definecolor{currentstroke}{rgb}{0.486275,0.988235,0.000000}%
\pgfsetstrokecolor{currentstroke}%
\pgfsetdash{}{0pt}%
\pgfpathmoveto{\pgfqpoint{1.025455in}{0.722970in}}%
\pgfpathlineto{\pgfqpoint{1.115636in}{0.735146in}}%
\pgfpathlineto{\pgfqpoint{1.205818in}{0.749545in}}%
\pgfpathlineto{\pgfqpoint{1.296000in}{0.766169in}}%
\pgfpathlineto{\pgfqpoint{1.386182in}{0.785016in}}%
\pgfpathlineto{\pgfqpoint{1.476364in}{0.806087in}}%
\pgfpathlineto{\pgfqpoint{1.566545in}{0.829382in}}%
\pgfpathlineto{\pgfqpoint{1.656727in}{0.854901in}}%
\pgfpathlineto{\pgfqpoint{1.746909in}{0.882644in}}%
\pgfpathlineto{\pgfqpoint{1.837091in}{0.912611in}}%
\pgfpathlineto{\pgfqpoint{1.931782in}{0.946469in}}%
\pgfpathlineto{\pgfqpoint{2.026473in}{0.982780in}}%
\pgfpathlineto{\pgfqpoint{2.121164in}{1.021542in}}%
\pgfpathlineto{\pgfqpoint{2.215855in}{1.062756in}}%
\pgfpathlineto{\pgfqpoint{2.310545in}{1.106421in}}%
\pgfpathlineto{\pgfqpoint{2.405236in}{1.152539in}}%
\pgfpathlineto{\pgfqpoint{2.499927in}{1.201108in}}%
\pgfpathlineto{\pgfqpoint{2.594618in}{1.252130in}}%
\pgfpathlineto{\pgfqpoint{2.689309in}{1.305603in}}%
\pgfpathlineto{\pgfqpoint{2.784000in}{1.361527in}}%
\pgfpathlineto{\pgfqpoint{2.883200in}{1.422745in}}%
\pgfpathlineto{\pgfqpoint{2.982400in}{1.486654in}}%
\pgfpathlineto{\pgfqpoint{3.081600in}{1.553253in}}%
\pgfpathlineto{\pgfqpoint{3.180800in}{1.622543in}}%
\pgfpathlineto{\pgfqpoint{3.280000in}{1.694525in}}%
\pgfpathlineto{\pgfqpoint{3.379200in}{1.769197in}}%
\pgfpathlineto{\pgfqpoint{3.478400in}{1.846560in}}%
\pgfpathlineto{\pgfqpoint{3.582109in}{1.930316in}}%
\pgfpathlineto{\pgfqpoint{3.685818in}{2.017014in}}%
\pgfpathlineto{\pgfqpoint{3.789527in}{2.106653in}}%
\pgfpathlineto{\pgfqpoint{3.893236in}{2.199233in}}%
\pgfpathlineto{\pgfqpoint{3.996945in}{2.294754in}}%
\pgfpathlineto{\pgfqpoint{4.100655in}{2.393216in}}%
\pgfpathlineto{\pgfqpoint{4.208873in}{2.499094in}}%
\pgfpathlineto{\pgfqpoint{4.317091in}{2.608175in}}%
\pgfpathlineto{\pgfqpoint{4.425309in}{2.720459in}}%
\pgfpathlineto{\pgfqpoint{4.533527in}{2.835944in}}%
\pgfpathlineto{\pgfqpoint{4.641745in}{2.954632in}}%
\pgfpathlineto{\pgfqpoint{4.749964in}{3.076523in}}%
\pgfpathlineto{\pgfqpoint{4.862691in}{3.206897in}}%
\pgfpathlineto{\pgfqpoint{4.975418in}{3.340747in}}%
\pgfpathlineto{\pgfqpoint{5.088145in}{3.478071in}}%
\pgfpathlineto{\pgfqpoint{5.200873in}{3.618870in}}%
\pgfpathlineto{\pgfqpoint{5.313600in}{3.763144in}}%
\pgfpathlineto{\pgfqpoint{5.430836in}{3.916874in}}%
\pgfpathlineto{\pgfqpoint{5.534545in}{4.056000in}}%
\pgfpathlineto{\pgfqpoint{5.534545in}{4.056000in}}%
\pgfusepath{stroke}%
\end{pgfscope}%
\begin{pgfscope}%
\pgfpathrectangle{\pgfqpoint{0.800000in}{0.528000in}}{\pgfqpoint{4.960000in}{3.696000in}}%
\pgfusepath{clip}%
\pgfsetrectcap%
\pgfsetroundjoin%
\pgfsetlinewidth{1.505625pt}%
\definecolor{currentstroke}{rgb}{0.000000,1.000000,0.498039}%
\pgfsetstrokecolor{currentstroke}%
\pgfsetdash{}{0pt}%
\pgfpathmoveto{\pgfqpoint{1.025455in}{0.696000in}}%
\pgfpathlineto{\pgfqpoint{5.534545in}{0.701560in}}%
\pgfpathlineto{\pgfqpoint{5.534545in}{0.701560in}}%
\pgfusepath{stroke}%
\end{pgfscope}%
\begin{pgfscope}%
\pgfpathrectangle{\pgfqpoint{0.800000in}{0.528000in}}{\pgfqpoint{4.960000in}{3.696000in}}%
\pgfusepath{clip}%
\pgfsetrectcap%
\pgfsetroundjoin%
\pgfsetlinewidth{1.505625pt}%
\definecolor{currentstroke}{rgb}{0.000000,1.000000,1.000000}%
\pgfsetstrokecolor{currentstroke}%
\pgfsetdash{}{0pt}%
\pgfpathmoveto{\pgfqpoint{1.025455in}{0.699464in}}%
\pgfpathlineto{\pgfqpoint{1.120145in}{0.700542in}}%
\pgfpathlineto{\pgfqpoint{1.219345in}{0.701565in}}%
\pgfpathlineto{\pgfqpoint{1.318545in}{0.702677in}}%
\pgfpathlineto{\pgfqpoint{1.715345in}{0.707636in}}%
\pgfpathlineto{\pgfqpoint{1.778473in}{0.708348in}}%
\pgfpathlineto{\pgfqpoint{1.819055in}{0.708715in}}%
\pgfpathlineto{\pgfqpoint{1.967855in}{0.710583in}}%
\pgfpathlineto{\pgfqpoint{2.098618in}{0.712195in}}%
\pgfpathlineto{\pgfqpoint{2.143709in}{0.712918in}}%
\pgfpathlineto{\pgfqpoint{2.409745in}{0.716499in}}%
\pgfpathlineto{\pgfqpoint{2.468364in}{0.717066in}}%
\pgfpathlineto{\pgfqpoint{2.499927in}{0.717622in}}%
\pgfpathlineto{\pgfqpoint{2.567564in}{0.718556in}}%
\pgfpathlineto{\pgfqpoint{2.761455in}{0.721291in}}%
\pgfpathlineto{\pgfqpoint{2.838109in}{0.722314in}}%
\pgfpathlineto{\pgfqpoint{2.869673in}{0.722870in}}%
\pgfpathlineto{\pgfqpoint{2.919273in}{0.723548in}}%
\pgfpathlineto{\pgfqpoint{2.941818in}{0.723704in}}%
\pgfpathlineto{\pgfqpoint{3.591127in}{0.732444in}}%
\pgfpathlineto{\pgfqpoint{3.604655in}{0.732677in}}%
\pgfpathlineto{\pgfqpoint{3.654255in}{0.733155in}}%
\pgfpathlineto{\pgfqpoint{3.676800in}{0.733711in}}%
\pgfpathlineto{\pgfqpoint{3.717382in}{0.734023in}}%
\pgfpathlineto{\pgfqpoint{3.748945in}{0.734779in}}%
\pgfpathlineto{\pgfqpoint{3.825600in}{0.735857in}}%
\pgfpathlineto{\pgfqpoint{5.534545in}{0.760987in}}%
\pgfpathlineto{\pgfqpoint{5.534545in}{0.760987in}}%
\pgfusepath{stroke}%
\end{pgfscope}%
\begin{pgfscope}%
\pgfpathrectangle{\pgfqpoint{0.800000in}{0.528000in}}{\pgfqpoint{4.960000in}{3.696000in}}%
\pgfusepath{clip}%
\pgfsetrectcap%
\pgfsetroundjoin%
\pgfsetlinewidth{1.505625pt}%
\definecolor{currentstroke}{rgb}{1.000000,0.000000,1.000000}%
\pgfsetstrokecolor{currentstroke}%
\pgfsetdash{}{0pt}%
\pgfpathmoveto{\pgfqpoint{1.025455in}{0.701282in}}%
\pgfpathlineto{\pgfqpoint{1.129164in}{0.703255in}}%
\pgfpathlineto{\pgfqpoint{1.133673in}{0.704306in}}%
\pgfpathlineto{\pgfqpoint{1.138182in}{0.703205in}}%
\pgfpathlineto{\pgfqpoint{1.151709in}{0.704039in}}%
\pgfpathlineto{\pgfqpoint{1.165236in}{0.703889in}}%
\pgfpathlineto{\pgfqpoint{1.169745in}{0.704706in}}%
\pgfpathlineto{\pgfqpoint{1.178764in}{0.704067in}}%
\pgfpathlineto{\pgfqpoint{1.192291in}{0.704729in}}%
\pgfpathlineto{\pgfqpoint{1.205818in}{0.704401in}}%
\pgfpathlineto{\pgfqpoint{1.255418in}{0.705913in}}%
\pgfpathlineto{\pgfqpoint{1.259927in}{0.704929in}}%
\pgfpathlineto{\pgfqpoint{1.268945in}{0.705340in}}%
\pgfpathlineto{\pgfqpoint{1.291491in}{0.706864in}}%
\pgfpathlineto{\pgfqpoint{1.296000in}{0.706419in}}%
\pgfpathlineto{\pgfqpoint{1.300509in}{0.707653in}}%
\pgfpathlineto{\pgfqpoint{1.305018in}{0.705963in}}%
\pgfpathlineto{\pgfqpoint{1.314036in}{0.706869in}}%
\pgfpathlineto{\pgfqpoint{1.323055in}{0.706714in}}%
\pgfpathlineto{\pgfqpoint{1.327564in}{0.706219in}}%
\pgfpathlineto{\pgfqpoint{1.332073in}{0.708676in}}%
\pgfpathlineto{\pgfqpoint{1.354618in}{0.707114in}}%
\pgfpathlineto{\pgfqpoint{1.359127in}{0.709071in}}%
\pgfpathlineto{\pgfqpoint{1.363636in}{0.707414in}}%
\pgfpathlineto{\pgfqpoint{1.395200in}{0.708192in}}%
\pgfpathlineto{\pgfqpoint{1.404218in}{0.707403in}}%
\pgfpathlineto{\pgfqpoint{1.408727in}{0.708787in}}%
\pgfpathlineto{\pgfqpoint{1.417745in}{0.708376in}}%
\pgfpathlineto{\pgfqpoint{1.485382in}{0.709643in}}%
\pgfpathlineto{\pgfqpoint{1.494400in}{0.709265in}}%
\pgfpathlineto{\pgfqpoint{1.498909in}{0.710817in}}%
\pgfpathlineto{\pgfqpoint{1.562036in}{0.711689in}}%
\pgfpathlineto{\pgfqpoint{1.575564in}{0.711801in}}%
\pgfpathlineto{\pgfqpoint{1.584582in}{0.711356in}}%
\pgfpathlineto{\pgfqpoint{1.598109in}{0.712635in}}%
\pgfpathlineto{\pgfqpoint{1.602618in}{0.711512in}}%
\pgfpathlineto{\pgfqpoint{1.607127in}{0.713174in}}%
\pgfpathlineto{\pgfqpoint{1.616145in}{0.712195in}}%
\pgfpathlineto{\pgfqpoint{1.620655in}{0.714831in}}%
\pgfpathlineto{\pgfqpoint{1.625164in}{0.712023in}}%
\pgfpathlineto{\pgfqpoint{1.634182in}{0.712757in}}%
\pgfpathlineto{\pgfqpoint{1.638691in}{0.714686in}}%
\pgfpathlineto{\pgfqpoint{1.643200in}{0.712279in}}%
\pgfpathlineto{\pgfqpoint{1.674764in}{0.712940in}}%
\pgfpathlineto{\pgfqpoint{1.683782in}{0.713229in}}%
\pgfpathlineto{\pgfqpoint{1.701818in}{0.712901in}}%
\pgfpathlineto{\pgfqpoint{1.706327in}{0.715114in}}%
\pgfpathlineto{\pgfqpoint{1.715345in}{0.713997in}}%
\pgfpathlineto{\pgfqpoint{1.724364in}{0.714086in}}%
\pgfpathlineto{\pgfqpoint{1.742400in}{0.715915in}}%
\pgfpathlineto{\pgfqpoint{1.751418in}{0.714024in}}%
\pgfpathlineto{\pgfqpoint{1.755927in}{0.715203in}}%
\pgfpathlineto{\pgfqpoint{1.760436in}{0.714575in}}%
\pgfpathlineto{\pgfqpoint{1.773964in}{0.715648in}}%
\pgfpathlineto{\pgfqpoint{1.778473in}{0.716082in}}%
\pgfpathlineto{\pgfqpoint{1.782982in}{0.714786in}}%
\pgfpathlineto{\pgfqpoint{1.787491in}{0.716660in}}%
\pgfpathlineto{\pgfqpoint{1.819055in}{0.715987in}}%
\pgfpathlineto{\pgfqpoint{1.823564in}{0.717299in}}%
\pgfpathlineto{\pgfqpoint{1.828073in}{0.716221in}}%
\pgfpathlineto{\pgfqpoint{1.895709in}{0.716832in}}%
\pgfpathlineto{\pgfqpoint{1.900218in}{0.718906in}}%
\pgfpathlineto{\pgfqpoint{1.904727in}{0.719317in}}%
\pgfpathlineto{\pgfqpoint{1.909236in}{0.717455in}}%
\pgfpathlineto{\pgfqpoint{1.918255in}{0.718283in}}%
\pgfpathlineto{\pgfqpoint{1.927273in}{0.718066in}}%
\pgfpathlineto{\pgfqpoint{1.931782in}{0.719100in}}%
\pgfpathlineto{\pgfqpoint{1.936291in}{0.717588in}}%
\pgfpathlineto{\pgfqpoint{1.940800in}{0.719273in}}%
\pgfpathlineto{\pgfqpoint{1.967855in}{0.719084in}}%
\pgfpathlineto{\pgfqpoint{1.972364in}{0.721046in}}%
\pgfpathlineto{\pgfqpoint{1.976873in}{0.719162in}}%
\pgfpathlineto{\pgfqpoint{1.981382in}{0.721258in}}%
\pgfpathlineto{\pgfqpoint{1.990400in}{0.719073in}}%
\pgfpathlineto{\pgfqpoint{1.994909in}{0.719979in}}%
\pgfpathlineto{\pgfqpoint{1.999418in}{0.718984in}}%
\pgfpathlineto{\pgfqpoint{2.003927in}{0.720229in}}%
\pgfpathlineto{\pgfqpoint{2.012945in}{0.720101in}}%
\pgfpathlineto{\pgfqpoint{2.021964in}{0.721007in}}%
\pgfpathlineto{\pgfqpoint{2.026473in}{0.719929in}}%
\pgfpathlineto{\pgfqpoint{2.030982in}{0.722370in}}%
\pgfpathlineto{\pgfqpoint{2.035491in}{0.720085in}}%
\pgfpathlineto{\pgfqpoint{2.044509in}{0.720090in}}%
\pgfpathlineto{\pgfqpoint{2.049018in}{0.723548in}}%
\pgfpathlineto{\pgfqpoint{2.062545in}{0.721469in}}%
\pgfpathlineto{\pgfqpoint{2.080582in}{0.721975in}}%
\pgfpathlineto{\pgfqpoint{2.089600in}{0.721013in}}%
\pgfpathlineto{\pgfqpoint{2.094109in}{0.720729in}}%
\pgfpathlineto{\pgfqpoint{2.098618in}{0.722570in}}%
\pgfpathlineto{\pgfqpoint{2.116655in}{0.721497in}}%
\pgfpathlineto{\pgfqpoint{2.121164in}{0.723343in}}%
\pgfpathlineto{\pgfqpoint{2.125673in}{0.721891in}}%
\pgfpathlineto{\pgfqpoint{2.143709in}{0.723120in}}%
\pgfpathlineto{\pgfqpoint{2.148218in}{0.722798in}}%
\pgfpathlineto{\pgfqpoint{2.152727in}{0.721174in}}%
\pgfpathlineto{\pgfqpoint{2.157236in}{0.724960in}}%
\pgfpathlineto{\pgfqpoint{2.161745in}{0.726645in}}%
\pgfpathlineto{\pgfqpoint{2.170764in}{0.723748in}}%
\pgfpathlineto{\pgfqpoint{2.175273in}{0.722881in}}%
\pgfpathlineto{\pgfqpoint{2.188800in}{0.724732in}}%
\pgfpathlineto{\pgfqpoint{2.193309in}{0.723370in}}%
\pgfpathlineto{\pgfqpoint{2.202327in}{0.724449in}}%
\pgfpathlineto{\pgfqpoint{2.206836in}{0.726161in}}%
\pgfpathlineto{\pgfqpoint{2.211345in}{0.723910in}}%
\pgfpathlineto{\pgfqpoint{2.215855in}{0.724343in}}%
\pgfpathlineto{\pgfqpoint{2.220364in}{0.727017in}}%
\pgfpathlineto{\pgfqpoint{2.224873in}{0.723343in}}%
\pgfpathlineto{\pgfqpoint{2.233891in}{0.727201in}}%
\pgfpathlineto{\pgfqpoint{2.238400in}{0.723759in}}%
\pgfpathlineto{\pgfqpoint{2.242909in}{0.727807in}}%
\pgfpathlineto{\pgfqpoint{2.247418in}{0.723609in}}%
\pgfpathlineto{\pgfqpoint{2.269964in}{0.727017in}}%
\pgfpathlineto{\pgfqpoint{2.274473in}{0.729792in}}%
\pgfpathlineto{\pgfqpoint{2.283491in}{0.724260in}}%
\pgfpathlineto{\pgfqpoint{2.292509in}{0.726028in}}%
\pgfpathlineto{\pgfqpoint{2.297018in}{0.728124in}}%
\pgfpathlineto{\pgfqpoint{2.301527in}{0.725639in}}%
\pgfpathlineto{\pgfqpoint{2.306036in}{0.725238in}}%
\pgfpathlineto{\pgfqpoint{2.319564in}{0.726962in}}%
\pgfpathlineto{\pgfqpoint{2.324073in}{0.726195in}}%
\pgfpathlineto{\pgfqpoint{2.328582in}{0.729703in}}%
\pgfpathlineto{\pgfqpoint{2.333091in}{0.726878in}}%
\pgfpathlineto{\pgfqpoint{2.342109in}{0.728608in}}%
\pgfpathlineto{\pgfqpoint{2.346618in}{0.727051in}}%
\pgfpathlineto{\pgfqpoint{2.351127in}{0.727001in}}%
\pgfpathlineto{\pgfqpoint{2.355636in}{0.731576in}}%
\pgfpathlineto{\pgfqpoint{2.360145in}{0.727890in}}%
\pgfpathlineto{\pgfqpoint{2.364655in}{0.729675in}}%
\pgfpathlineto{\pgfqpoint{2.373673in}{0.728396in}}%
\pgfpathlineto{\pgfqpoint{2.378182in}{0.729280in}}%
\pgfpathlineto{\pgfqpoint{2.382691in}{0.727457in}}%
\pgfpathlineto{\pgfqpoint{2.387200in}{0.727084in}}%
\pgfpathlineto{\pgfqpoint{2.396218in}{0.729575in}}%
\pgfpathlineto{\pgfqpoint{2.400727in}{0.727840in}}%
\pgfpathlineto{\pgfqpoint{2.405236in}{0.727868in}}%
\pgfpathlineto{\pgfqpoint{2.409745in}{0.730198in}}%
\pgfpathlineto{\pgfqpoint{2.423273in}{0.727623in}}%
\pgfpathlineto{\pgfqpoint{2.427782in}{0.730470in}}%
\pgfpathlineto{\pgfqpoint{2.432291in}{0.730742in}}%
\pgfpathlineto{\pgfqpoint{2.436800in}{0.728696in}}%
\pgfpathlineto{\pgfqpoint{2.441309in}{0.733178in}}%
\pgfpathlineto{\pgfqpoint{2.445818in}{0.732711in}}%
\pgfpathlineto{\pgfqpoint{2.450327in}{0.729280in}}%
\pgfpathlineto{\pgfqpoint{2.454836in}{0.729052in}}%
\pgfpathlineto{\pgfqpoint{2.459345in}{0.731866in}}%
\pgfpathlineto{\pgfqpoint{2.463855in}{0.728007in}}%
\pgfpathlineto{\pgfqpoint{2.468364in}{0.730314in}}%
\pgfpathlineto{\pgfqpoint{2.472873in}{0.728969in}}%
\pgfpathlineto{\pgfqpoint{2.477382in}{0.731588in}}%
\pgfpathlineto{\pgfqpoint{2.481891in}{0.729291in}}%
\pgfpathlineto{\pgfqpoint{2.495418in}{0.731426in}}%
\pgfpathlineto{\pgfqpoint{2.499927in}{0.730815in}}%
\pgfpathlineto{\pgfqpoint{2.504436in}{0.732138in}}%
\pgfpathlineto{\pgfqpoint{2.508945in}{0.728708in}}%
\pgfpathlineto{\pgfqpoint{2.513455in}{0.730576in}}%
\pgfpathlineto{\pgfqpoint{2.517964in}{0.730209in}}%
\pgfpathlineto{\pgfqpoint{2.522473in}{0.733250in}}%
\pgfpathlineto{\pgfqpoint{2.526982in}{0.730748in}}%
\pgfpathlineto{\pgfqpoint{2.531491in}{0.731910in}}%
\pgfpathlineto{\pgfqpoint{2.536000in}{0.730270in}}%
\pgfpathlineto{\pgfqpoint{2.540509in}{0.732916in}}%
\pgfpathlineto{\pgfqpoint{2.545018in}{0.731332in}}%
\pgfpathlineto{\pgfqpoint{2.549527in}{0.732021in}}%
\pgfpathlineto{\pgfqpoint{2.554036in}{0.730470in}}%
\pgfpathlineto{\pgfqpoint{2.558545in}{0.733539in}}%
\pgfpathlineto{\pgfqpoint{2.563055in}{0.729308in}}%
\pgfpathlineto{\pgfqpoint{2.567564in}{0.733495in}}%
\pgfpathlineto{\pgfqpoint{2.572073in}{0.731282in}}%
\pgfpathlineto{\pgfqpoint{2.581091in}{0.732494in}}%
\pgfpathlineto{\pgfqpoint{2.585600in}{0.737564in}}%
\pgfpathlineto{\pgfqpoint{2.590109in}{0.736558in}}%
\pgfpathlineto{\pgfqpoint{2.594618in}{0.730159in}}%
\pgfpathlineto{\pgfqpoint{2.599127in}{0.732877in}}%
\pgfpathlineto{\pgfqpoint{2.603636in}{0.733255in}}%
\pgfpathlineto{\pgfqpoint{2.608145in}{0.732032in}}%
\pgfpathlineto{\pgfqpoint{2.612655in}{0.734562in}}%
\pgfpathlineto{\pgfqpoint{2.617164in}{0.733289in}}%
\pgfpathlineto{\pgfqpoint{2.621673in}{0.734395in}}%
\pgfpathlineto{\pgfqpoint{2.626182in}{0.732049in}}%
\pgfpathlineto{\pgfqpoint{2.630691in}{0.733667in}}%
\pgfpathlineto{\pgfqpoint{2.635200in}{0.731899in}}%
\pgfpathlineto{\pgfqpoint{2.648727in}{0.733873in}}%
\pgfpathlineto{\pgfqpoint{2.657745in}{0.732572in}}%
\pgfpathlineto{\pgfqpoint{2.662255in}{0.734840in}}%
\pgfpathlineto{\pgfqpoint{2.666764in}{0.731771in}}%
\pgfpathlineto{\pgfqpoint{2.671273in}{0.736752in}}%
\pgfpathlineto{\pgfqpoint{2.675782in}{0.734973in}}%
\pgfpathlineto{\pgfqpoint{2.680291in}{0.735251in}}%
\pgfpathlineto{\pgfqpoint{2.684800in}{0.733000in}}%
\pgfpathlineto{\pgfqpoint{2.698327in}{0.732900in}}%
\pgfpathlineto{\pgfqpoint{2.702836in}{0.738009in}}%
\pgfpathlineto{\pgfqpoint{2.707345in}{0.737353in}}%
\pgfpathlineto{\pgfqpoint{2.711855in}{0.734245in}}%
\pgfpathlineto{\pgfqpoint{2.725382in}{0.737314in}}%
\pgfpathlineto{\pgfqpoint{2.734400in}{0.735496in}}%
\pgfpathlineto{\pgfqpoint{2.743418in}{0.735085in}}%
\pgfpathlineto{\pgfqpoint{2.747927in}{0.739466in}}%
\pgfpathlineto{\pgfqpoint{2.761455in}{0.737142in}}%
\pgfpathlineto{\pgfqpoint{2.765964in}{0.740311in}}%
\pgfpathlineto{\pgfqpoint{2.774982in}{0.738432in}}%
\pgfpathlineto{\pgfqpoint{2.788509in}{0.736469in}}%
\pgfpathlineto{\pgfqpoint{2.793018in}{0.737636in}}%
\pgfpathlineto{\pgfqpoint{2.797527in}{0.737492in}}%
\pgfpathlineto{\pgfqpoint{2.802036in}{0.740472in}}%
\pgfpathlineto{\pgfqpoint{2.806545in}{0.738076in}}%
\pgfpathlineto{\pgfqpoint{2.824582in}{0.738704in}}%
\pgfpathlineto{\pgfqpoint{2.829091in}{0.737169in}}%
\pgfpathlineto{\pgfqpoint{2.865164in}{0.737903in}}%
\pgfpathlineto{\pgfqpoint{2.869673in}{0.739482in}}%
\pgfpathlineto{\pgfqpoint{2.874182in}{0.738276in}}%
\pgfpathlineto{\pgfqpoint{2.878691in}{0.740277in}}%
\pgfpathlineto{\pgfqpoint{2.887709in}{0.736786in}}%
\pgfpathlineto{\pgfqpoint{2.892218in}{0.741250in}}%
\pgfpathlineto{\pgfqpoint{2.896727in}{0.741011in}}%
\pgfpathlineto{\pgfqpoint{2.901236in}{0.739115in}}%
\pgfpathlineto{\pgfqpoint{2.910255in}{0.740066in}}%
\pgfpathlineto{\pgfqpoint{2.919273in}{0.738893in}}%
\pgfpathlineto{\pgfqpoint{2.923782in}{0.742240in}}%
\pgfpathlineto{\pgfqpoint{2.932800in}{0.741072in}}%
\pgfpathlineto{\pgfqpoint{2.937309in}{0.740689in}}%
\pgfpathlineto{\pgfqpoint{2.941818in}{0.738748in}}%
\pgfpathlineto{\pgfqpoint{2.950836in}{0.740311in}}%
\pgfpathlineto{\pgfqpoint{2.955345in}{0.737453in}}%
\pgfpathlineto{\pgfqpoint{2.959855in}{0.744019in}}%
\pgfpathlineto{\pgfqpoint{2.964364in}{0.738598in}}%
\pgfpathlineto{\pgfqpoint{2.968873in}{0.742562in}}%
\pgfpathlineto{\pgfqpoint{2.973382in}{0.738826in}}%
\pgfpathlineto{\pgfqpoint{2.977891in}{0.739516in}}%
\pgfpathlineto{\pgfqpoint{2.982400in}{0.744592in}}%
\pgfpathlineto{\pgfqpoint{2.991418in}{0.738704in}}%
\pgfpathlineto{\pgfqpoint{2.995927in}{0.741790in}}%
\pgfpathlineto{\pgfqpoint{3.000436in}{0.741317in}}%
\pgfpathlineto{\pgfqpoint{3.004945in}{0.744781in}}%
\pgfpathlineto{\pgfqpoint{3.009455in}{0.738887in}}%
\pgfpathlineto{\pgfqpoint{3.013964in}{0.744931in}}%
\pgfpathlineto{\pgfqpoint{3.027491in}{0.739705in}}%
\pgfpathlineto{\pgfqpoint{3.032000in}{0.741373in}}%
\pgfpathlineto{\pgfqpoint{3.036509in}{0.740027in}}%
\pgfpathlineto{\pgfqpoint{3.041018in}{0.742662in}}%
\pgfpathlineto{\pgfqpoint{3.050036in}{0.741595in}}%
\pgfpathlineto{\pgfqpoint{3.054545in}{0.743874in}}%
\pgfpathlineto{\pgfqpoint{3.059055in}{0.742690in}}%
\pgfpathlineto{\pgfqpoint{3.063564in}{0.742846in}}%
\pgfpathlineto{\pgfqpoint{3.068073in}{0.746332in}}%
\pgfpathlineto{\pgfqpoint{3.072582in}{0.743118in}}%
\pgfpathlineto{\pgfqpoint{3.077091in}{0.743474in}}%
\pgfpathlineto{\pgfqpoint{3.081600in}{0.745431in}}%
\pgfpathlineto{\pgfqpoint{3.086109in}{0.743463in}}%
\pgfpathlineto{\pgfqpoint{3.090618in}{0.742829in}}%
\pgfpathlineto{\pgfqpoint{3.095127in}{0.745737in}}%
\pgfpathlineto{\pgfqpoint{3.099636in}{0.742051in}}%
\pgfpathlineto{\pgfqpoint{3.108655in}{0.748583in}}%
\pgfpathlineto{\pgfqpoint{3.113164in}{0.744631in}}%
\pgfpathlineto{\pgfqpoint{3.117673in}{0.746460in}}%
\pgfpathlineto{\pgfqpoint{3.122182in}{0.743441in}}%
\pgfpathlineto{\pgfqpoint{3.126691in}{0.745948in}}%
\pgfpathlineto{\pgfqpoint{3.131200in}{0.744341in}}%
\pgfpathlineto{\pgfqpoint{3.140218in}{0.743341in}}%
\pgfpathlineto{\pgfqpoint{3.144727in}{0.741628in}}%
\pgfpathlineto{\pgfqpoint{3.149236in}{0.745681in}}%
\pgfpathlineto{\pgfqpoint{3.153745in}{0.744547in}}%
\pgfpathlineto{\pgfqpoint{3.158255in}{0.748228in}}%
\pgfpathlineto{\pgfqpoint{3.162764in}{0.743313in}}%
\pgfpathlineto{\pgfqpoint{3.167273in}{0.745809in}}%
\pgfpathlineto{\pgfqpoint{3.176291in}{0.748167in}}%
\pgfpathlineto{\pgfqpoint{3.180800in}{0.747399in}}%
\pgfpathlineto{\pgfqpoint{3.189818in}{0.743129in}}%
\pgfpathlineto{\pgfqpoint{3.194327in}{0.746171in}}%
\pgfpathlineto{\pgfqpoint{3.198836in}{0.745348in}}%
\pgfpathlineto{\pgfqpoint{3.203345in}{0.747321in}}%
\pgfpathlineto{\pgfqpoint{3.207855in}{0.747772in}}%
\pgfpathlineto{\pgfqpoint{3.221382in}{0.746393in}}%
\pgfpathlineto{\pgfqpoint{3.225891in}{0.747594in}}%
\pgfpathlineto{\pgfqpoint{3.234909in}{0.747449in}}%
\pgfpathlineto{\pgfqpoint{3.239418in}{0.748205in}}%
\pgfpathlineto{\pgfqpoint{3.243927in}{0.745020in}}%
\pgfpathlineto{\pgfqpoint{3.248436in}{0.747866in}}%
\pgfpathlineto{\pgfqpoint{3.252945in}{0.746004in}}%
\pgfpathlineto{\pgfqpoint{3.266473in}{0.749379in}}%
\pgfpathlineto{\pgfqpoint{3.270982in}{0.745003in}}%
\pgfpathlineto{\pgfqpoint{3.275491in}{0.749390in}}%
\pgfpathlineto{\pgfqpoint{3.280000in}{0.751597in}}%
\pgfpathlineto{\pgfqpoint{3.284509in}{0.750207in}}%
\pgfpathlineto{\pgfqpoint{3.289018in}{0.746560in}}%
\pgfpathlineto{\pgfqpoint{3.293527in}{0.748800in}}%
\pgfpathlineto{\pgfqpoint{3.298036in}{0.746082in}}%
\pgfpathlineto{\pgfqpoint{3.302545in}{0.749156in}}%
\pgfpathlineto{\pgfqpoint{3.307055in}{0.747494in}}%
\pgfpathlineto{\pgfqpoint{3.320582in}{0.747305in}}%
\pgfpathlineto{\pgfqpoint{3.325091in}{0.751797in}}%
\pgfpathlineto{\pgfqpoint{3.329600in}{0.751013in}}%
\pgfpathlineto{\pgfqpoint{3.334109in}{0.746465in}}%
\pgfpathlineto{\pgfqpoint{3.338618in}{0.749445in}}%
\pgfpathlineto{\pgfqpoint{3.343127in}{0.750012in}}%
\pgfpathlineto{\pgfqpoint{3.347636in}{0.748094in}}%
\pgfpathlineto{\pgfqpoint{3.352145in}{0.749379in}}%
\pgfpathlineto{\pgfqpoint{3.361164in}{0.747110in}}%
\pgfpathlineto{\pgfqpoint{3.365673in}{0.752208in}}%
\pgfpathlineto{\pgfqpoint{3.370182in}{0.747227in}}%
\pgfpathlineto{\pgfqpoint{3.388218in}{0.750268in}}%
\pgfpathlineto{\pgfqpoint{3.397236in}{0.747377in}}%
\pgfpathlineto{\pgfqpoint{3.401745in}{0.750351in}}%
\pgfpathlineto{\pgfqpoint{3.406255in}{0.747266in}}%
\pgfpathlineto{\pgfqpoint{3.410764in}{0.750652in}}%
\pgfpathlineto{\pgfqpoint{3.428800in}{0.750296in}}%
\pgfpathlineto{\pgfqpoint{3.433309in}{0.752069in}}%
\pgfpathlineto{\pgfqpoint{3.437818in}{0.750396in}}%
\pgfpathlineto{\pgfqpoint{3.442327in}{0.753810in}}%
\pgfpathlineto{\pgfqpoint{3.446836in}{0.752386in}}%
\pgfpathlineto{\pgfqpoint{3.455855in}{0.752742in}}%
\pgfpathlineto{\pgfqpoint{3.460364in}{0.753837in}}%
\pgfpathlineto{\pgfqpoint{3.469382in}{0.752826in}}%
\pgfpathlineto{\pgfqpoint{3.478400in}{0.752314in}}%
\pgfpathlineto{\pgfqpoint{3.482909in}{0.755939in}}%
\pgfpathlineto{\pgfqpoint{3.487418in}{0.754460in}}%
\pgfpathlineto{\pgfqpoint{3.491927in}{0.750529in}}%
\pgfpathlineto{\pgfqpoint{3.496436in}{0.749851in}}%
\pgfpathlineto{\pgfqpoint{3.514473in}{0.755761in}}%
\pgfpathlineto{\pgfqpoint{3.518982in}{0.757768in}}%
\pgfpathlineto{\pgfqpoint{3.523491in}{0.751146in}}%
\pgfpathlineto{\pgfqpoint{3.541527in}{0.752937in}}%
\pgfpathlineto{\pgfqpoint{3.550545in}{0.750774in}}%
\pgfpathlineto{\pgfqpoint{3.555055in}{0.754427in}}%
\pgfpathlineto{\pgfqpoint{3.559564in}{0.753470in}}%
\pgfpathlineto{\pgfqpoint{3.564073in}{0.755806in}}%
\pgfpathlineto{\pgfqpoint{3.573091in}{0.753821in}}%
\pgfpathlineto{\pgfqpoint{3.577600in}{0.758402in}}%
\pgfpathlineto{\pgfqpoint{3.586618in}{0.754855in}}%
\pgfpathlineto{\pgfqpoint{3.591127in}{0.756689in}}%
\pgfpathlineto{\pgfqpoint{3.595636in}{0.756578in}}%
\pgfpathlineto{\pgfqpoint{3.600145in}{0.753982in}}%
\pgfpathlineto{\pgfqpoint{3.604655in}{0.756979in}}%
\pgfpathlineto{\pgfqpoint{3.609164in}{0.755305in}}%
\pgfpathlineto{\pgfqpoint{3.613673in}{0.757807in}}%
\pgfpathlineto{\pgfqpoint{3.618182in}{0.756617in}}%
\pgfpathlineto{\pgfqpoint{3.622691in}{0.754260in}}%
\pgfpathlineto{\pgfqpoint{3.627200in}{0.756083in}}%
\pgfpathlineto{\pgfqpoint{3.631709in}{0.759536in}}%
\pgfpathlineto{\pgfqpoint{3.636218in}{0.752798in}}%
\pgfpathlineto{\pgfqpoint{3.640727in}{0.757168in}}%
\pgfpathlineto{\pgfqpoint{3.645236in}{0.753404in}}%
\pgfpathlineto{\pgfqpoint{3.649745in}{0.759063in}}%
\pgfpathlineto{\pgfqpoint{3.654255in}{0.759180in}}%
\pgfpathlineto{\pgfqpoint{3.658764in}{0.756523in}}%
\pgfpathlineto{\pgfqpoint{3.676800in}{0.755055in}}%
\pgfpathlineto{\pgfqpoint{3.681309in}{0.753771in}}%
\pgfpathlineto{\pgfqpoint{3.685818in}{0.757479in}}%
\pgfpathlineto{\pgfqpoint{3.690327in}{0.755183in}}%
\pgfpathlineto{\pgfqpoint{3.694836in}{0.757312in}}%
\pgfpathlineto{\pgfqpoint{3.699345in}{0.761643in}}%
\pgfpathlineto{\pgfqpoint{3.703855in}{0.757012in}}%
\pgfpathlineto{\pgfqpoint{3.708364in}{0.754138in}}%
\pgfpathlineto{\pgfqpoint{3.712873in}{0.756695in}}%
\pgfpathlineto{\pgfqpoint{3.717382in}{0.755578in}}%
\pgfpathlineto{\pgfqpoint{3.721891in}{0.759619in}}%
\pgfpathlineto{\pgfqpoint{3.726400in}{0.755522in}}%
\pgfpathlineto{\pgfqpoint{3.730909in}{0.758580in}}%
\pgfpathlineto{\pgfqpoint{3.744436in}{0.757062in}}%
\pgfpathlineto{\pgfqpoint{3.748945in}{0.762549in}}%
\pgfpathlineto{\pgfqpoint{3.753455in}{0.757334in}}%
\pgfpathlineto{\pgfqpoint{3.762473in}{0.756267in}}%
\pgfpathlineto{\pgfqpoint{3.766982in}{0.758380in}}%
\pgfpathlineto{\pgfqpoint{3.771491in}{0.757446in}}%
\pgfpathlineto{\pgfqpoint{3.776000in}{0.755038in}}%
\pgfpathlineto{\pgfqpoint{3.780509in}{0.760854in}}%
\pgfpathlineto{\pgfqpoint{3.785018in}{0.758241in}}%
\pgfpathlineto{\pgfqpoint{3.789527in}{0.762105in}}%
\pgfpathlineto{\pgfqpoint{3.794036in}{0.756973in}}%
\pgfpathlineto{\pgfqpoint{3.798545in}{0.759686in}}%
\pgfpathlineto{\pgfqpoint{3.803055in}{0.757268in}}%
\pgfpathlineto{\pgfqpoint{3.807564in}{0.765802in}}%
\pgfpathlineto{\pgfqpoint{3.816582in}{0.758413in}}%
\pgfpathlineto{\pgfqpoint{3.821091in}{0.758046in}}%
\pgfpathlineto{\pgfqpoint{3.825600in}{0.763011in}}%
\pgfpathlineto{\pgfqpoint{3.830109in}{0.759486in}}%
\pgfpathlineto{\pgfqpoint{3.834618in}{0.759408in}}%
\pgfpathlineto{\pgfqpoint{3.839127in}{0.757801in}}%
\pgfpathlineto{\pgfqpoint{3.843636in}{0.762622in}}%
\pgfpathlineto{\pgfqpoint{3.848145in}{0.763778in}}%
\pgfpathlineto{\pgfqpoint{3.852655in}{0.759280in}}%
\pgfpathlineto{\pgfqpoint{3.857164in}{0.761254in}}%
\pgfpathlineto{\pgfqpoint{3.861673in}{0.765001in}}%
\pgfpathlineto{\pgfqpoint{3.866182in}{0.760776in}}%
\pgfpathlineto{\pgfqpoint{3.879709in}{0.761421in}}%
\pgfpathlineto{\pgfqpoint{3.897745in}{0.758708in}}%
\pgfpathlineto{\pgfqpoint{3.902255in}{0.763850in}}%
\pgfpathlineto{\pgfqpoint{3.906764in}{0.760631in}}%
\pgfpathlineto{\pgfqpoint{3.911273in}{0.761449in}}%
\pgfpathlineto{\pgfqpoint{3.915782in}{0.757585in}}%
\pgfpathlineto{\pgfqpoint{3.920291in}{0.763917in}}%
\pgfpathlineto{\pgfqpoint{3.924800in}{0.760476in}}%
\pgfpathlineto{\pgfqpoint{3.929309in}{0.760481in}}%
\pgfpathlineto{\pgfqpoint{3.933818in}{0.765363in}}%
\pgfpathlineto{\pgfqpoint{3.938327in}{0.759264in}}%
\pgfpathlineto{\pgfqpoint{3.942836in}{0.760520in}}%
\pgfpathlineto{\pgfqpoint{3.947345in}{0.763089in}}%
\pgfpathlineto{\pgfqpoint{3.956364in}{0.762177in}}%
\pgfpathlineto{\pgfqpoint{3.960873in}{0.765329in}}%
\pgfpathlineto{\pgfqpoint{3.965382in}{0.762055in}}%
\pgfpathlineto{\pgfqpoint{3.969891in}{0.768615in}}%
\pgfpathlineto{\pgfqpoint{3.978909in}{0.763700in}}%
\pgfpathlineto{\pgfqpoint{3.987927in}{0.764662in}}%
\pgfpathlineto{\pgfqpoint{3.992436in}{0.761449in}}%
\pgfpathlineto{\pgfqpoint{3.996945in}{0.763717in}}%
\pgfpathlineto{\pgfqpoint{4.001455in}{0.767864in}}%
\pgfpathlineto{\pgfqpoint{4.005964in}{0.763650in}}%
\pgfpathlineto{\pgfqpoint{4.010473in}{0.763061in}}%
\pgfpathlineto{\pgfqpoint{4.014982in}{0.767414in}}%
\pgfpathlineto{\pgfqpoint{4.019491in}{0.763912in}}%
\pgfpathlineto{\pgfqpoint{4.024000in}{0.763962in}}%
\pgfpathlineto{\pgfqpoint{4.028509in}{0.762566in}}%
\pgfpathlineto{\pgfqpoint{4.037527in}{0.767776in}}%
\pgfpathlineto{\pgfqpoint{4.046545in}{0.766363in}}%
\pgfpathlineto{\pgfqpoint{4.051055in}{0.763584in}}%
\pgfpathlineto{\pgfqpoint{4.055564in}{0.766558in}}%
\pgfpathlineto{\pgfqpoint{4.064582in}{0.767687in}}%
\pgfpathlineto{\pgfqpoint{4.069091in}{0.765385in}}%
\pgfpathlineto{\pgfqpoint{4.073600in}{0.768315in}}%
\pgfpathlineto{\pgfqpoint{4.078109in}{0.766619in}}%
\pgfpathlineto{\pgfqpoint{4.082618in}{0.766853in}}%
\pgfpathlineto{\pgfqpoint{4.087127in}{0.768721in}}%
\pgfpathlineto{\pgfqpoint{4.091636in}{0.768498in}}%
\pgfpathlineto{\pgfqpoint{4.096145in}{0.764017in}}%
\pgfpathlineto{\pgfqpoint{4.100655in}{0.769404in}}%
\pgfpathlineto{\pgfqpoint{4.105164in}{0.768465in}}%
\pgfpathlineto{\pgfqpoint{4.109673in}{0.765629in}}%
\pgfpathlineto{\pgfqpoint{4.118691in}{0.764996in}}%
\pgfpathlineto{\pgfqpoint{4.123200in}{0.766447in}}%
\pgfpathlineto{\pgfqpoint{4.127709in}{0.763728in}}%
\pgfpathlineto{\pgfqpoint{4.132218in}{0.767853in}}%
\pgfpathlineto{\pgfqpoint{4.136727in}{0.766369in}}%
\pgfpathlineto{\pgfqpoint{4.141236in}{0.766513in}}%
\pgfpathlineto{\pgfqpoint{4.145745in}{0.765313in}}%
\pgfpathlineto{\pgfqpoint{4.150255in}{0.774319in}}%
\pgfpathlineto{\pgfqpoint{4.154764in}{0.769605in}}%
\pgfpathlineto{\pgfqpoint{4.163782in}{0.769527in}}%
\pgfpathlineto{\pgfqpoint{4.172800in}{0.770550in}}%
\pgfpathlineto{\pgfqpoint{4.177309in}{0.765880in}}%
\pgfpathlineto{\pgfqpoint{4.181818in}{0.769872in}}%
\pgfpathlineto{\pgfqpoint{4.186327in}{0.769416in}}%
\pgfpathlineto{\pgfqpoint{4.190836in}{0.774047in}}%
\pgfpathlineto{\pgfqpoint{4.199855in}{0.769594in}}%
\pgfpathlineto{\pgfqpoint{4.204364in}{0.773068in}}%
\pgfpathlineto{\pgfqpoint{4.208873in}{0.768398in}}%
\pgfpathlineto{\pgfqpoint{4.213382in}{0.776521in}}%
\pgfpathlineto{\pgfqpoint{4.217891in}{0.777416in}}%
\pgfpathlineto{\pgfqpoint{4.222400in}{0.770183in}}%
\pgfpathlineto{\pgfqpoint{4.226909in}{0.769827in}}%
\pgfpathlineto{\pgfqpoint{4.231418in}{0.771934in}}%
\pgfpathlineto{\pgfqpoint{4.235927in}{0.770339in}}%
\pgfpathlineto{\pgfqpoint{4.240436in}{0.771311in}}%
\pgfpathlineto{\pgfqpoint{4.244945in}{0.768009in}}%
\pgfpathlineto{\pgfqpoint{4.249455in}{0.769538in}}%
\pgfpathlineto{\pgfqpoint{4.253964in}{0.773663in}}%
\pgfpathlineto{\pgfqpoint{4.258473in}{0.769215in}}%
\pgfpathlineto{\pgfqpoint{4.262982in}{0.774475in}}%
\pgfpathlineto{\pgfqpoint{4.267491in}{0.768142in}}%
\pgfpathlineto{\pgfqpoint{4.272000in}{0.773647in}}%
\pgfpathlineto{\pgfqpoint{4.276509in}{0.770906in}}%
\pgfpathlineto{\pgfqpoint{4.281018in}{0.769655in}}%
\pgfpathlineto{\pgfqpoint{4.299055in}{0.770516in}}%
\pgfpathlineto{\pgfqpoint{4.303564in}{0.774781in}}%
\pgfpathlineto{\pgfqpoint{4.308073in}{0.768876in}}%
\pgfpathlineto{\pgfqpoint{4.312582in}{0.767598in}}%
\pgfpathlineto{\pgfqpoint{4.317091in}{0.774442in}}%
\pgfpathlineto{\pgfqpoint{4.321600in}{0.770572in}}%
\pgfpathlineto{\pgfqpoint{4.326109in}{0.779623in}}%
\pgfpathlineto{\pgfqpoint{4.330618in}{0.770455in}}%
\pgfpathlineto{\pgfqpoint{4.335127in}{0.775648in}}%
\pgfpathlineto{\pgfqpoint{4.339636in}{0.778528in}}%
\pgfpathlineto{\pgfqpoint{4.344145in}{0.772173in}}%
\pgfpathlineto{\pgfqpoint{4.348655in}{0.770377in}}%
\pgfpathlineto{\pgfqpoint{4.353164in}{0.773580in}}%
\pgfpathlineto{\pgfqpoint{4.357673in}{0.771095in}}%
\pgfpathlineto{\pgfqpoint{4.362182in}{0.774025in}}%
\pgfpathlineto{\pgfqpoint{4.366691in}{0.775376in}}%
\pgfpathlineto{\pgfqpoint{4.371200in}{0.770222in}}%
\pgfpathlineto{\pgfqpoint{4.375709in}{0.775965in}}%
\pgfpathlineto{\pgfqpoint{4.380218in}{0.774931in}}%
\pgfpathlineto{\pgfqpoint{4.384727in}{0.777071in}}%
\pgfpathlineto{\pgfqpoint{4.389236in}{0.771545in}}%
\pgfpathlineto{\pgfqpoint{4.393745in}{0.774480in}}%
\pgfpathlineto{\pgfqpoint{4.398255in}{0.772918in}}%
\pgfpathlineto{\pgfqpoint{4.402764in}{0.775214in}}%
\pgfpathlineto{\pgfqpoint{4.407273in}{0.772201in}}%
\pgfpathlineto{\pgfqpoint{4.411782in}{0.773730in}}%
\pgfpathlineto{\pgfqpoint{4.434327in}{0.774992in}}%
\pgfpathlineto{\pgfqpoint{4.438836in}{0.779073in}}%
\pgfpathlineto{\pgfqpoint{4.443345in}{0.772990in}}%
\pgfpathlineto{\pgfqpoint{4.447855in}{0.777399in}}%
\pgfpathlineto{\pgfqpoint{4.452364in}{0.775225in}}%
\pgfpathlineto{\pgfqpoint{4.456873in}{0.775459in}}%
\pgfpathlineto{\pgfqpoint{4.461382in}{0.772868in}}%
\pgfpathlineto{\pgfqpoint{4.465891in}{0.778561in}}%
\pgfpathlineto{\pgfqpoint{4.470400in}{0.774469in}}%
\pgfpathlineto{\pgfqpoint{4.474909in}{0.775709in}}%
\pgfpathlineto{\pgfqpoint{4.479418in}{0.774258in}}%
\pgfpathlineto{\pgfqpoint{4.488436in}{0.777572in}}%
\pgfpathlineto{\pgfqpoint{4.492945in}{0.780474in}}%
\pgfpathlineto{\pgfqpoint{4.497455in}{0.773702in}}%
\pgfpathlineto{\pgfqpoint{4.501964in}{0.777916in}}%
\pgfpathlineto{\pgfqpoint{4.506473in}{0.779607in}}%
\pgfpathlineto{\pgfqpoint{4.510982in}{0.774803in}}%
\pgfpathlineto{\pgfqpoint{4.515491in}{0.780958in}}%
\pgfpathlineto{\pgfqpoint{4.520000in}{0.772801in}}%
\pgfpathlineto{\pgfqpoint{4.524509in}{0.779907in}}%
\pgfpathlineto{\pgfqpoint{4.529018in}{0.777244in}}%
\pgfpathlineto{\pgfqpoint{4.533527in}{0.773129in}}%
\pgfpathlineto{\pgfqpoint{4.538036in}{0.777577in}}%
\pgfpathlineto{\pgfqpoint{4.547055in}{0.774236in}}%
\pgfpathlineto{\pgfqpoint{4.551564in}{0.777021in}}%
\pgfpathlineto{\pgfqpoint{4.556073in}{0.775759in}}%
\pgfpathlineto{\pgfqpoint{4.560582in}{0.780268in}}%
\pgfpathlineto{\pgfqpoint{4.565091in}{0.786801in}}%
\pgfpathlineto{\pgfqpoint{4.569600in}{0.779240in}}%
\pgfpathlineto{\pgfqpoint{4.574109in}{0.775898in}}%
\pgfpathlineto{\pgfqpoint{4.587636in}{0.776221in}}%
\pgfpathlineto{\pgfqpoint{4.592145in}{0.773963in}}%
\pgfpathlineto{\pgfqpoint{4.596655in}{0.779634in}}%
\pgfpathlineto{\pgfqpoint{4.601164in}{0.777961in}}%
\pgfpathlineto{\pgfqpoint{4.605673in}{0.785516in}}%
\pgfpathlineto{\pgfqpoint{4.610182in}{0.779429in}}%
\pgfpathlineto{\pgfqpoint{4.623709in}{0.780579in}}%
\pgfpathlineto{\pgfqpoint{4.628218in}{0.778383in}}%
\pgfpathlineto{\pgfqpoint{4.637236in}{0.779634in}}%
\pgfpathlineto{\pgfqpoint{4.646255in}{0.775765in}}%
\pgfpathlineto{\pgfqpoint{4.650764in}{0.782520in}}%
\pgfpathlineto{\pgfqpoint{4.655273in}{0.784293in}}%
\pgfpathlineto{\pgfqpoint{4.659782in}{0.783265in}}%
\pgfpathlineto{\pgfqpoint{4.664291in}{0.778550in}}%
\pgfpathlineto{\pgfqpoint{4.668800in}{0.781180in}}%
\pgfpathlineto{\pgfqpoint{4.673309in}{0.787790in}}%
\pgfpathlineto{\pgfqpoint{4.677818in}{0.791638in}}%
\pgfpathlineto{\pgfqpoint{4.682327in}{0.777388in}}%
\pgfpathlineto{\pgfqpoint{4.686836in}{0.790932in}}%
\pgfpathlineto{\pgfqpoint{4.691345in}{0.777055in}}%
\pgfpathlineto{\pgfqpoint{4.695855in}{0.780591in}}%
\pgfpathlineto{\pgfqpoint{4.700364in}{0.780057in}}%
\pgfpathlineto{\pgfqpoint{4.704873in}{0.784933in}}%
\pgfpathlineto{\pgfqpoint{4.709382in}{0.783721in}}%
\pgfpathlineto{\pgfqpoint{4.718400in}{0.784794in}}%
\pgfpathlineto{\pgfqpoint{4.731927in}{0.783771in}}%
\pgfpathlineto{\pgfqpoint{4.736436in}{0.778278in}}%
\pgfpathlineto{\pgfqpoint{4.740945in}{0.781747in}}%
\pgfpathlineto{\pgfqpoint{4.749964in}{0.781447in}}%
\pgfpathlineto{\pgfqpoint{4.754473in}{0.782019in}}%
\pgfpathlineto{\pgfqpoint{4.758982in}{0.778672in}}%
\pgfpathlineto{\pgfqpoint{4.763491in}{0.785778in}}%
\pgfpathlineto{\pgfqpoint{4.768000in}{0.783020in}}%
\pgfpathlineto{\pgfqpoint{4.772509in}{0.782175in}}%
\pgfpathlineto{\pgfqpoint{4.777018in}{0.778962in}}%
\pgfpathlineto{\pgfqpoint{4.781527in}{0.781964in}}%
\pgfpathlineto{\pgfqpoint{4.786036in}{0.779228in}}%
\pgfpathlineto{\pgfqpoint{4.790545in}{0.785188in}}%
\pgfpathlineto{\pgfqpoint{4.799564in}{0.781614in}}%
\pgfpathlineto{\pgfqpoint{4.804073in}{0.787190in}}%
\pgfpathlineto{\pgfqpoint{4.808582in}{0.783220in}}%
\pgfpathlineto{\pgfqpoint{4.813091in}{0.784282in}}%
\pgfpathlineto{\pgfqpoint{4.817600in}{0.781397in}}%
\pgfpathlineto{\pgfqpoint{4.822109in}{0.784143in}}%
\pgfpathlineto{\pgfqpoint{4.826618in}{0.782214in}}%
\pgfpathlineto{\pgfqpoint{4.831127in}{0.782892in}}%
\pgfpathlineto{\pgfqpoint{4.835636in}{0.786706in}}%
\pgfpathlineto{\pgfqpoint{4.840145in}{0.785650in}}%
\pgfpathlineto{\pgfqpoint{4.849164in}{0.785694in}}%
\pgfpathlineto{\pgfqpoint{4.858182in}{0.782570in}}%
\pgfpathlineto{\pgfqpoint{4.862691in}{0.782887in}}%
\pgfpathlineto{\pgfqpoint{4.867200in}{0.785939in}}%
\pgfpathlineto{\pgfqpoint{4.871709in}{0.783732in}}%
\pgfpathlineto{\pgfqpoint{4.876218in}{0.786439in}}%
\pgfpathlineto{\pgfqpoint{4.885236in}{0.784816in}}%
\pgfpathlineto{\pgfqpoint{4.889745in}{0.780429in}}%
\pgfpathlineto{\pgfqpoint{4.894255in}{0.794034in}}%
\pgfpathlineto{\pgfqpoint{4.898764in}{0.786223in}}%
\pgfpathlineto{\pgfqpoint{4.903273in}{0.786122in}}%
\pgfpathlineto{\pgfqpoint{4.907782in}{0.792388in}}%
\pgfpathlineto{\pgfqpoint{4.912291in}{0.783832in}}%
\pgfpathlineto{\pgfqpoint{4.916800in}{0.790943in}}%
\pgfpathlineto{\pgfqpoint{4.921309in}{0.783871in}}%
\pgfpathlineto{\pgfqpoint{4.925818in}{0.785561in}}%
\pgfpathlineto{\pgfqpoint{4.930327in}{0.785778in}}%
\pgfpathlineto{\pgfqpoint{4.934836in}{0.784699in}}%
\pgfpathlineto{\pgfqpoint{4.939345in}{0.788786in}}%
\pgfpathlineto{\pgfqpoint{4.943855in}{0.786034in}}%
\pgfpathlineto{\pgfqpoint{4.948364in}{0.792199in}}%
\pgfpathlineto{\pgfqpoint{4.952873in}{0.787524in}}%
\pgfpathlineto{\pgfqpoint{4.961891in}{0.786506in}}%
\pgfpathlineto{\pgfqpoint{4.966400in}{0.796441in}}%
\pgfpathlineto{\pgfqpoint{4.970909in}{0.786111in}}%
\pgfpathlineto{\pgfqpoint{4.975418in}{0.784683in}}%
\pgfpathlineto{\pgfqpoint{4.979927in}{0.789542in}}%
\pgfpathlineto{\pgfqpoint{4.984436in}{0.789586in}}%
\pgfpathlineto{\pgfqpoint{4.993455in}{0.783887in}}%
\pgfpathlineto{\pgfqpoint{4.997964in}{0.788046in}}%
\pgfpathlineto{\pgfqpoint{5.002473in}{0.785200in}}%
\pgfpathlineto{\pgfqpoint{5.006982in}{0.787529in}}%
\pgfpathlineto{\pgfqpoint{5.011491in}{0.787435in}}%
\pgfpathlineto{\pgfqpoint{5.016000in}{0.791382in}}%
\pgfpathlineto{\pgfqpoint{5.029527in}{0.784972in}}%
\pgfpathlineto{\pgfqpoint{5.034036in}{0.785011in}}%
\pgfpathlineto{\pgfqpoint{5.038545in}{0.788719in}}%
\pgfpathlineto{\pgfqpoint{5.043055in}{0.787312in}}%
\pgfpathlineto{\pgfqpoint{5.047564in}{0.789119in}}%
\pgfpathlineto{\pgfqpoint{5.056582in}{0.785105in}}%
\pgfpathlineto{\pgfqpoint{5.065600in}{0.788519in}}%
\pgfpathlineto{\pgfqpoint{5.070109in}{0.791443in}}%
\pgfpathlineto{\pgfqpoint{5.079127in}{0.791198in}}%
\pgfpathlineto{\pgfqpoint{5.083636in}{0.788658in}}%
\pgfpathlineto{\pgfqpoint{5.088145in}{0.792372in}}%
\pgfpathlineto{\pgfqpoint{5.092655in}{0.791193in}}%
\pgfpathlineto{\pgfqpoint{5.097164in}{0.791371in}}%
\pgfpathlineto{\pgfqpoint{5.101673in}{0.789041in}}%
\pgfpathlineto{\pgfqpoint{5.106182in}{0.790431in}}%
\pgfpathlineto{\pgfqpoint{5.110691in}{0.790370in}}%
\pgfpathlineto{\pgfqpoint{5.115200in}{0.795780in}}%
\pgfpathlineto{\pgfqpoint{5.119709in}{0.793834in}}%
\pgfpathlineto{\pgfqpoint{5.124218in}{0.788235in}}%
\pgfpathlineto{\pgfqpoint{5.128727in}{0.796591in}}%
\pgfpathlineto{\pgfqpoint{5.133236in}{0.790042in}}%
\pgfpathlineto{\pgfqpoint{5.137745in}{0.789820in}}%
\pgfpathlineto{\pgfqpoint{5.142255in}{0.790843in}}%
\pgfpathlineto{\pgfqpoint{5.146764in}{0.793506in}}%
\pgfpathlineto{\pgfqpoint{5.151273in}{0.792549in}}%
\pgfpathlineto{\pgfqpoint{5.160291in}{0.788130in}}%
\pgfpathlineto{\pgfqpoint{5.164800in}{0.795440in}}%
\pgfpathlineto{\pgfqpoint{5.173818in}{0.792933in}}%
\pgfpathlineto{\pgfqpoint{5.178327in}{0.795179in}}%
\pgfpathlineto{\pgfqpoint{5.187345in}{0.792394in}}%
\pgfpathlineto{\pgfqpoint{5.191855in}{0.796274in}}%
\pgfpathlineto{\pgfqpoint{5.196364in}{0.792916in}}%
\pgfpathlineto{\pgfqpoint{5.200873in}{0.791315in}}%
\pgfpathlineto{\pgfqpoint{5.209891in}{0.797036in}}%
\pgfpathlineto{\pgfqpoint{5.218909in}{0.791299in}}%
\pgfpathlineto{\pgfqpoint{5.223418in}{0.794273in}}%
\pgfpathlineto{\pgfqpoint{5.227927in}{0.792049in}}%
\pgfpathlineto{\pgfqpoint{5.241455in}{0.794784in}}%
\pgfpathlineto{\pgfqpoint{5.250473in}{0.790971in}}%
\pgfpathlineto{\pgfqpoint{5.254982in}{0.794551in}}%
\pgfpathlineto{\pgfqpoint{5.259491in}{0.802307in}}%
\pgfpathlineto{\pgfqpoint{5.264000in}{0.795046in}}%
\pgfpathlineto{\pgfqpoint{5.268509in}{0.795735in}}%
\pgfpathlineto{\pgfqpoint{5.273018in}{0.791532in}}%
\pgfpathlineto{\pgfqpoint{5.282036in}{0.801606in}}%
\pgfpathlineto{\pgfqpoint{5.286545in}{0.794329in}}%
\pgfpathlineto{\pgfqpoint{5.295564in}{0.793122in}}%
\pgfpathlineto{\pgfqpoint{5.304582in}{0.798137in}}%
\pgfpathlineto{\pgfqpoint{5.309091in}{0.792972in}}%
\pgfpathlineto{\pgfqpoint{5.313600in}{0.799888in}}%
\pgfpathlineto{\pgfqpoint{5.318109in}{0.794195in}}%
\pgfpathlineto{\pgfqpoint{5.322618in}{0.804264in}}%
\pgfpathlineto{\pgfqpoint{5.331636in}{0.795340in}}%
\pgfpathlineto{\pgfqpoint{5.336145in}{0.790214in}}%
\pgfpathlineto{\pgfqpoint{5.345164in}{0.796708in}}%
\pgfpathlineto{\pgfqpoint{5.349673in}{0.794851in}}%
\pgfpathlineto{\pgfqpoint{5.354182in}{0.801662in}}%
\pgfpathlineto{\pgfqpoint{5.363200in}{0.794834in}}%
\pgfpathlineto{\pgfqpoint{5.367709in}{0.797481in}}%
\pgfpathlineto{\pgfqpoint{5.372218in}{0.798532in}}%
\pgfpathlineto{\pgfqpoint{5.376727in}{0.801139in}}%
\pgfpathlineto{\pgfqpoint{5.381236in}{0.801428in}}%
\pgfpathlineto{\pgfqpoint{5.385745in}{0.795685in}}%
\pgfpathlineto{\pgfqpoint{5.394764in}{0.799777in}}%
\pgfpathlineto{\pgfqpoint{5.399273in}{0.793311in}}%
\pgfpathlineto{\pgfqpoint{5.403782in}{0.800478in}}%
\pgfpathlineto{\pgfqpoint{5.408291in}{0.796230in}}%
\pgfpathlineto{\pgfqpoint{5.412800in}{0.796035in}}%
\pgfpathlineto{\pgfqpoint{5.417309in}{0.800922in}}%
\pgfpathlineto{\pgfqpoint{5.421818in}{0.795668in}}%
\pgfpathlineto{\pgfqpoint{5.426327in}{0.799293in}}%
\pgfpathlineto{\pgfqpoint{5.430836in}{0.798120in}}%
\pgfpathlineto{\pgfqpoint{5.435345in}{0.793767in}}%
\pgfpathlineto{\pgfqpoint{5.439855in}{0.799438in}}%
\pgfpathlineto{\pgfqpoint{5.444364in}{0.798498in}}%
\pgfpathlineto{\pgfqpoint{5.448873in}{0.804364in}}%
\pgfpathlineto{\pgfqpoint{5.453382in}{0.805726in}}%
\pgfpathlineto{\pgfqpoint{5.457891in}{0.802101in}}%
\pgfpathlineto{\pgfqpoint{5.462400in}{0.795963in}}%
\pgfpathlineto{\pgfqpoint{5.466909in}{0.803196in}}%
\pgfpathlineto{\pgfqpoint{5.471418in}{0.799438in}}%
\pgfpathlineto{\pgfqpoint{5.475927in}{0.805437in}}%
\pgfpathlineto{\pgfqpoint{5.480436in}{0.797837in}}%
\pgfpathlineto{\pgfqpoint{5.484945in}{0.802646in}}%
\pgfpathlineto{\pgfqpoint{5.493964in}{0.799788in}}%
\pgfpathlineto{\pgfqpoint{5.498473in}{0.803875in}}%
\pgfpathlineto{\pgfqpoint{5.502982in}{0.801656in}}%
\pgfpathlineto{\pgfqpoint{5.507491in}{0.802473in}}%
\pgfpathlineto{\pgfqpoint{5.512000in}{0.799371in}}%
\pgfpathlineto{\pgfqpoint{5.516509in}{0.802079in}}%
\pgfpathlineto{\pgfqpoint{5.521018in}{0.797915in}}%
\pgfpathlineto{\pgfqpoint{5.525527in}{0.801862in}}%
\pgfpathlineto{\pgfqpoint{5.530036in}{0.799794in}}%
\pgfpathlineto{\pgfqpoint{5.534545in}{0.804836in}}%
\pgfpathlineto{\pgfqpoint{5.534545in}{0.804836in}}%
\pgfusepath{stroke}%
\end{pgfscope}%
\begin{pgfscope}%
\pgfsetrectcap%
\pgfsetmiterjoin%
\pgfsetlinewidth{0.803000pt}%
\definecolor{currentstroke}{rgb}{0.000000,0.000000,0.000000}%
\pgfsetstrokecolor{currentstroke}%
\pgfsetdash{}{0pt}%
\pgfpathmoveto{\pgfqpoint{0.800000in}{0.528000in}}%
\pgfpathlineto{\pgfqpoint{0.800000in}{4.224000in}}%
\pgfusepath{stroke}%
\end{pgfscope}%
\begin{pgfscope}%
\pgfsetrectcap%
\pgfsetmiterjoin%
\pgfsetlinewidth{0.803000pt}%
\definecolor{currentstroke}{rgb}{0.000000,0.000000,0.000000}%
\pgfsetstrokecolor{currentstroke}%
\pgfsetdash{}{0pt}%
\pgfpathmoveto{\pgfqpoint{5.760000in}{0.528000in}}%
\pgfpathlineto{\pgfqpoint{5.760000in}{4.224000in}}%
\pgfusepath{stroke}%
\end{pgfscope}%
\begin{pgfscope}%
\pgfsetrectcap%
\pgfsetmiterjoin%
\pgfsetlinewidth{0.803000pt}%
\definecolor{currentstroke}{rgb}{0.000000,0.000000,0.000000}%
\pgfsetstrokecolor{currentstroke}%
\pgfsetdash{}{0pt}%
\pgfpathmoveto{\pgfqpoint{0.800000in}{0.528000in}}%
\pgfpathlineto{\pgfqpoint{5.760000in}{0.528000in}}%
\pgfusepath{stroke}%
\end{pgfscope}%
\begin{pgfscope}%
\pgfsetrectcap%
\pgfsetmiterjoin%
\pgfsetlinewidth{0.803000pt}%
\definecolor{currentstroke}{rgb}{0.000000,0.000000,0.000000}%
\pgfsetstrokecolor{currentstroke}%
\pgfsetdash{}{0pt}%
\pgfpathmoveto{\pgfqpoint{0.800000in}{4.224000in}}%
\pgfpathlineto{\pgfqpoint{5.760000in}{4.224000in}}%
\pgfusepath{stroke}%
\end{pgfscope}%
\begin{pgfscope}%
\definecolor{textcolor}{rgb}{0.000000,0.000000,0.000000}%
\pgfsetstrokecolor{textcolor}%
\pgfsetfillcolor{textcolor}%
\pgftext[x=3.280000in,y=4.307333in,,base]{\color{textcolor}\ttfamily\fontsize{12.000000}{14.400000}\selectfont Comparisons vs Input size}%
\end{pgfscope}%
\begin{pgfscope}%
\pgfsetbuttcap%
\pgfsetmiterjoin%
\definecolor{currentfill}{rgb}{1.000000,1.000000,1.000000}%
\pgfsetfillcolor{currentfill}%
\pgfsetfillopacity{0.800000}%
\pgfsetlinewidth{1.003750pt}%
\definecolor{currentstroke}{rgb}{0.800000,0.800000,0.800000}%
\pgfsetstrokecolor{currentstroke}%
\pgfsetstrokeopacity{0.800000}%
\pgfsetdash{}{0pt}%
\pgfpathmoveto{\pgfqpoint{0.897222in}{3.088923in}}%
\pgfpathlineto{\pgfqpoint{2.094230in}{3.088923in}}%
\pgfpathquadraticcurveto{\pgfqpoint{2.122008in}{3.088923in}}{\pgfqpoint{2.122008in}{3.116701in}}%
\pgfpathlineto{\pgfqpoint{2.122008in}{4.126778in}}%
\pgfpathquadraticcurveto{\pgfqpoint{2.122008in}{4.154556in}}{\pgfqpoint{2.094230in}{4.154556in}}%
\pgfpathlineto{\pgfqpoint{0.897222in}{4.154556in}}%
\pgfpathquadraticcurveto{\pgfqpoint{0.869444in}{4.154556in}}{\pgfqpoint{0.869444in}{4.126778in}}%
\pgfpathlineto{\pgfqpoint{0.869444in}{3.116701in}}%
\pgfpathquadraticcurveto{\pgfqpoint{0.869444in}{3.088923in}}{\pgfqpoint{0.897222in}{3.088923in}}%
\pgfpathlineto{\pgfqpoint{0.897222in}{3.088923in}}%
\pgfpathclose%
\pgfusepath{stroke,fill}%
\end{pgfscope}%
\begin{pgfscope}%
\pgfsetrectcap%
\pgfsetroundjoin%
\pgfsetlinewidth{1.505625pt}%
\definecolor{currentstroke}{rgb}{1.000000,0.000000,0.000000}%
\pgfsetstrokecolor{currentstroke}%
\pgfsetdash{}{0pt}%
\pgfpathmoveto{\pgfqpoint{0.925000in}{4.041342in}}%
\pgfpathlineto{\pgfqpoint{1.063889in}{4.041342in}}%
\pgfpathlineto{\pgfqpoint{1.202778in}{4.041342in}}%
\pgfusepath{stroke}%
\end{pgfscope}%
\begin{pgfscope}%
\definecolor{textcolor}{rgb}{0.000000,0.000000,0.000000}%
\pgfsetstrokecolor{textcolor}%
\pgfsetfillcolor{textcolor}%
\pgftext[x=1.313889in,y=3.992731in,left,base]{\color{textcolor}\ttfamily\fontsize{10.000000}{12.000000}\selectfont Bubble}%
\end{pgfscope}%
\begin{pgfscope}%
\pgfsetrectcap%
\pgfsetroundjoin%
\pgfsetlinewidth{1.505625pt}%
\definecolor{currentstroke}{rgb}{0.486275,0.988235,0.000000}%
\pgfsetstrokecolor{currentstroke}%
\pgfsetdash{}{0pt}%
\pgfpathmoveto{\pgfqpoint{0.925000in}{3.836739in}}%
\pgfpathlineto{\pgfqpoint{1.063889in}{3.836739in}}%
\pgfpathlineto{\pgfqpoint{1.202778in}{3.836739in}}%
\pgfusepath{stroke}%
\end{pgfscope}%
\begin{pgfscope}%
\definecolor{textcolor}{rgb}{0.000000,0.000000,0.000000}%
\pgfsetstrokecolor{textcolor}%
\pgfsetfillcolor{textcolor}%
\pgftext[x=1.313889in,y=3.788128in,left,base]{\color{textcolor}\ttfamily\fontsize{10.000000}{12.000000}\selectfont Selection}%
\end{pgfscope}%
\begin{pgfscope}%
\pgfsetrectcap%
\pgfsetroundjoin%
\pgfsetlinewidth{1.505625pt}%
\definecolor{currentstroke}{rgb}{0.000000,1.000000,0.498039}%
\pgfsetstrokecolor{currentstroke}%
\pgfsetdash{}{0pt}%
\pgfpathmoveto{\pgfqpoint{0.925000in}{3.632136in}}%
\pgfpathlineto{\pgfqpoint{1.063889in}{3.632136in}}%
\pgfpathlineto{\pgfqpoint{1.202778in}{3.632136in}}%
\pgfusepath{stroke}%
\end{pgfscope}%
\begin{pgfscope}%
\definecolor{textcolor}{rgb}{0.000000,0.000000,0.000000}%
\pgfsetstrokecolor{textcolor}%
\pgfsetfillcolor{textcolor}%
\pgftext[x=1.313889in,y=3.583525in,left,base]{\color{textcolor}\ttfamily\fontsize{10.000000}{12.000000}\selectfont Insertion}%
\end{pgfscope}%
\begin{pgfscope}%
\pgfsetrectcap%
\pgfsetroundjoin%
\pgfsetlinewidth{1.505625pt}%
\definecolor{currentstroke}{rgb}{0.000000,1.000000,1.000000}%
\pgfsetstrokecolor{currentstroke}%
\pgfsetdash{}{0pt}%
\pgfpathmoveto{\pgfqpoint{0.925000in}{3.427532in}}%
\pgfpathlineto{\pgfqpoint{1.063889in}{3.427532in}}%
\pgfpathlineto{\pgfqpoint{1.202778in}{3.427532in}}%
\pgfusepath{stroke}%
\end{pgfscope}%
\begin{pgfscope}%
\definecolor{textcolor}{rgb}{0.000000,0.000000,0.000000}%
\pgfsetstrokecolor{textcolor}%
\pgfsetfillcolor{textcolor}%
\pgftext[x=1.313889in,y=3.378921in,left,base]{\color{textcolor}\ttfamily\fontsize{10.000000}{12.000000}\selectfont Merge}%
\end{pgfscope}%
\begin{pgfscope}%
\pgfsetrectcap%
\pgfsetroundjoin%
\pgfsetlinewidth{1.505625pt}%
\definecolor{currentstroke}{rgb}{1.000000,0.000000,1.000000}%
\pgfsetstrokecolor{currentstroke}%
\pgfsetdash{}{0pt}%
\pgfpathmoveto{\pgfqpoint{0.925000in}{3.221980in}}%
\pgfpathlineto{\pgfqpoint{1.063889in}{3.221980in}}%
\pgfpathlineto{\pgfqpoint{1.202778in}{3.221980in}}%
\pgfusepath{stroke}%
\end{pgfscope}%
\begin{pgfscope}%
\definecolor{textcolor}{rgb}{0.000000,0.000000,0.000000}%
\pgfsetstrokecolor{textcolor}%
\pgfsetfillcolor{textcolor}%
\pgftext[x=1.313889in,y=3.173369in,left,base]{\color{textcolor}\ttfamily\fontsize{10.000000}{12.000000}\selectfont Quick}%
\end{pgfscope}%
\end{pgfpicture}%
\makeatother%
\endgroup%

%% Creator: Matplotlib, PGF backend
%%
%% To include the figure in your LaTeX document, write
%%   \input{<filename>.pgf}
%%
%% Make sure the required packages are loaded in your preamble
%%   \usepackage{pgf}
%%
%% Also ensure that all the required font packages are loaded; for instance,
%% the lmodern package is sometimes necessary when using math font.
%%   \usepackage{lmodern}
%%
%% Figures using additional raster images can only be included by \input if
%% they are in the same directory as the main LaTeX file. For loading figures
%% from other directories you can use the `import` package
%%   \usepackage{import}
%%
%% and then include the figures with
%%   \import{<path to file>}{<filename>.pgf}
%%
%% Matplotlib used the following preamble
%%   \usepackage{fontspec}
%%   \setmainfont{DejaVuSerif.ttf}[Path=\detokenize{/home/dbk/.local/lib/python3.10/site-packages/matplotlib/mpl-data/fonts/ttf/}]
%%   \setsansfont{DejaVuSans.ttf}[Path=\detokenize{/home/dbk/.local/lib/python3.10/site-packages/matplotlib/mpl-data/fonts/ttf/}]
%%   \setmonofont{DejaVuSansMono.ttf}[Path=\detokenize{/home/dbk/.local/lib/python3.10/site-packages/matplotlib/mpl-data/fonts/ttf/}]
%%
\begingroup%
\makeatletter%
\begin{pgfpicture}%
\pgfpathrectangle{\pgfpointorigin}{\pgfqpoint{6.400000in}{4.800000in}}%
\pgfusepath{use as bounding box, clip}%
\begin{pgfscope}%
\pgfsetbuttcap%
\pgfsetmiterjoin%
\definecolor{currentfill}{rgb}{1.000000,1.000000,1.000000}%
\pgfsetfillcolor{currentfill}%
\pgfsetlinewidth{0.000000pt}%
\definecolor{currentstroke}{rgb}{1.000000,1.000000,1.000000}%
\pgfsetstrokecolor{currentstroke}%
\pgfsetdash{}{0pt}%
\pgfpathmoveto{\pgfqpoint{0.000000in}{0.000000in}}%
\pgfpathlineto{\pgfqpoint{6.400000in}{0.000000in}}%
\pgfpathlineto{\pgfqpoint{6.400000in}{4.800000in}}%
\pgfpathlineto{\pgfqpoint{0.000000in}{4.800000in}}%
\pgfpathlineto{\pgfqpoint{0.000000in}{0.000000in}}%
\pgfpathclose%
\pgfusepath{fill}%
\end{pgfscope}%
\begin{pgfscope}%
\pgfsetbuttcap%
\pgfsetmiterjoin%
\definecolor{currentfill}{rgb}{1.000000,1.000000,1.000000}%
\pgfsetfillcolor{currentfill}%
\pgfsetlinewidth{0.000000pt}%
\definecolor{currentstroke}{rgb}{0.000000,0.000000,0.000000}%
\pgfsetstrokecolor{currentstroke}%
\pgfsetstrokeopacity{0.000000}%
\pgfsetdash{}{0pt}%
\pgfpathmoveto{\pgfqpoint{0.800000in}{0.528000in}}%
\pgfpathlineto{\pgfqpoint{5.760000in}{0.528000in}}%
\pgfpathlineto{\pgfqpoint{5.760000in}{4.224000in}}%
\pgfpathlineto{\pgfqpoint{0.800000in}{4.224000in}}%
\pgfpathlineto{\pgfqpoint{0.800000in}{0.528000in}}%
\pgfpathclose%
\pgfusepath{fill}%
\end{pgfscope}%
\begin{pgfscope}%
\pgfsetbuttcap%
\pgfsetroundjoin%
\definecolor{currentfill}{rgb}{0.000000,0.000000,0.000000}%
\pgfsetfillcolor{currentfill}%
\pgfsetlinewidth{0.803000pt}%
\definecolor{currentstroke}{rgb}{0.000000,0.000000,0.000000}%
\pgfsetstrokecolor{currentstroke}%
\pgfsetdash{}{0pt}%
\pgfsys@defobject{currentmarker}{\pgfqpoint{0.000000in}{-0.048611in}}{\pgfqpoint{0.000000in}{0.000000in}}{%
\pgfpathmoveto{\pgfqpoint{0.000000in}{0.000000in}}%
\pgfpathlineto{\pgfqpoint{0.000000in}{-0.048611in}}%
\pgfusepath{stroke,fill}%
}%
\begin{pgfscope}%
\pgfsys@transformshift{1.020945in}{0.528000in}%
\pgfsys@useobject{currentmarker}{}%
\end{pgfscope}%
\end{pgfscope}%
\begin{pgfscope}%
\definecolor{textcolor}{rgb}{0.000000,0.000000,0.000000}%
\pgfsetstrokecolor{textcolor}%
\pgfsetfillcolor{textcolor}%
\pgftext[x=1.020945in,y=0.430778in,,top]{\color{textcolor}\ttfamily\fontsize{10.000000}{12.000000}\selectfont 0}%
\end{pgfscope}%
\begin{pgfscope}%
\pgfsetbuttcap%
\pgfsetroundjoin%
\definecolor{currentfill}{rgb}{0.000000,0.000000,0.000000}%
\pgfsetfillcolor{currentfill}%
\pgfsetlinewidth{0.803000pt}%
\definecolor{currentstroke}{rgb}{0.000000,0.000000,0.000000}%
\pgfsetstrokecolor{currentstroke}%
\pgfsetdash{}{0pt}%
\pgfsys@defobject{currentmarker}{\pgfqpoint{0.000000in}{-0.048611in}}{\pgfqpoint{0.000000in}{0.000000in}}{%
\pgfpathmoveto{\pgfqpoint{0.000000in}{0.000000in}}%
\pgfpathlineto{\pgfqpoint{0.000000in}{-0.048611in}}%
\pgfusepath{stroke,fill}%
}%
\begin{pgfscope}%
\pgfsys@transformshift{1.922764in}{0.528000in}%
\pgfsys@useobject{currentmarker}{}%
\end{pgfscope}%
\end{pgfscope}%
\begin{pgfscope}%
\definecolor{textcolor}{rgb}{0.000000,0.000000,0.000000}%
\pgfsetstrokecolor{textcolor}%
\pgfsetfillcolor{textcolor}%
\pgftext[x=1.922764in,y=0.430778in,,top]{\color{textcolor}\ttfamily\fontsize{10.000000}{12.000000}\selectfont 200}%
\end{pgfscope}%
\begin{pgfscope}%
\pgfsetbuttcap%
\pgfsetroundjoin%
\definecolor{currentfill}{rgb}{0.000000,0.000000,0.000000}%
\pgfsetfillcolor{currentfill}%
\pgfsetlinewidth{0.803000pt}%
\definecolor{currentstroke}{rgb}{0.000000,0.000000,0.000000}%
\pgfsetstrokecolor{currentstroke}%
\pgfsetdash{}{0pt}%
\pgfsys@defobject{currentmarker}{\pgfqpoint{0.000000in}{-0.048611in}}{\pgfqpoint{0.000000in}{0.000000in}}{%
\pgfpathmoveto{\pgfqpoint{0.000000in}{0.000000in}}%
\pgfpathlineto{\pgfqpoint{0.000000in}{-0.048611in}}%
\pgfusepath{stroke,fill}%
}%
\begin{pgfscope}%
\pgfsys@transformshift{2.824582in}{0.528000in}%
\pgfsys@useobject{currentmarker}{}%
\end{pgfscope}%
\end{pgfscope}%
\begin{pgfscope}%
\definecolor{textcolor}{rgb}{0.000000,0.000000,0.000000}%
\pgfsetstrokecolor{textcolor}%
\pgfsetfillcolor{textcolor}%
\pgftext[x=2.824582in,y=0.430778in,,top]{\color{textcolor}\ttfamily\fontsize{10.000000}{12.000000}\selectfont 400}%
\end{pgfscope}%
\begin{pgfscope}%
\pgfsetbuttcap%
\pgfsetroundjoin%
\definecolor{currentfill}{rgb}{0.000000,0.000000,0.000000}%
\pgfsetfillcolor{currentfill}%
\pgfsetlinewidth{0.803000pt}%
\definecolor{currentstroke}{rgb}{0.000000,0.000000,0.000000}%
\pgfsetstrokecolor{currentstroke}%
\pgfsetdash{}{0pt}%
\pgfsys@defobject{currentmarker}{\pgfqpoint{0.000000in}{-0.048611in}}{\pgfqpoint{0.000000in}{0.000000in}}{%
\pgfpathmoveto{\pgfqpoint{0.000000in}{0.000000in}}%
\pgfpathlineto{\pgfqpoint{0.000000in}{-0.048611in}}%
\pgfusepath{stroke,fill}%
}%
\begin{pgfscope}%
\pgfsys@transformshift{3.726400in}{0.528000in}%
\pgfsys@useobject{currentmarker}{}%
\end{pgfscope}%
\end{pgfscope}%
\begin{pgfscope}%
\definecolor{textcolor}{rgb}{0.000000,0.000000,0.000000}%
\pgfsetstrokecolor{textcolor}%
\pgfsetfillcolor{textcolor}%
\pgftext[x=3.726400in,y=0.430778in,,top]{\color{textcolor}\ttfamily\fontsize{10.000000}{12.000000}\selectfont 600}%
\end{pgfscope}%
\begin{pgfscope}%
\pgfsetbuttcap%
\pgfsetroundjoin%
\definecolor{currentfill}{rgb}{0.000000,0.000000,0.000000}%
\pgfsetfillcolor{currentfill}%
\pgfsetlinewidth{0.803000pt}%
\definecolor{currentstroke}{rgb}{0.000000,0.000000,0.000000}%
\pgfsetstrokecolor{currentstroke}%
\pgfsetdash{}{0pt}%
\pgfsys@defobject{currentmarker}{\pgfqpoint{0.000000in}{-0.048611in}}{\pgfqpoint{0.000000in}{0.000000in}}{%
\pgfpathmoveto{\pgfqpoint{0.000000in}{0.000000in}}%
\pgfpathlineto{\pgfqpoint{0.000000in}{-0.048611in}}%
\pgfusepath{stroke,fill}%
}%
\begin{pgfscope}%
\pgfsys@transformshift{4.628218in}{0.528000in}%
\pgfsys@useobject{currentmarker}{}%
\end{pgfscope}%
\end{pgfscope}%
\begin{pgfscope}%
\definecolor{textcolor}{rgb}{0.000000,0.000000,0.000000}%
\pgfsetstrokecolor{textcolor}%
\pgfsetfillcolor{textcolor}%
\pgftext[x=4.628218in,y=0.430778in,,top]{\color{textcolor}\ttfamily\fontsize{10.000000}{12.000000}\selectfont 800}%
\end{pgfscope}%
\begin{pgfscope}%
\pgfsetbuttcap%
\pgfsetroundjoin%
\definecolor{currentfill}{rgb}{0.000000,0.000000,0.000000}%
\pgfsetfillcolor{currentfill}%
\pgfsetlinewidth{0.803000pt}%
\definecolor{currentstroke}{rgb}{0.000000,0.000000,0.000000}%
\pgfsetstrokecolor{currentstroke}%
\pgfsetdash{}{0pt}%
\pgfsys@defobject{currentmarker}{\pgfqpoint{0.000000in}{-0.048611in}}{\pgfqpoint{0.000000in}{0.000000in}}{%
\pgfpathmoveto{\pgfqpoint{0.000000in}{0.000000in}}%
\pgfpathlineto{\pgfqpoint{0.000000in}{-0.048611in}}%
\pgfusepath{stroke,fill}%
}%
\begin{pgfscope}%
\pgfsys@transformshift{5.530036in}{0.528000in}%
\pgfsys@useobject{currentmarker}{}%
\end{pgfscope}%
\end{pgfscope}%
\begin{pgfscope}%
\definecolor{textcolor}{rgb}{0.000000,0.000000,0.000000}%
\pgfsetstrokecolor{textcolor}%
\pgfsetfillcolor{textcolor}%
\pgftext[x=5.530036in,y=0.430778in,,top]{\color{textcolor}\ttfamily\fontsize{10.000000}{12.000000}\selectfont 1000}%
\end{pgfscope}%
\begin{pgfscope}%
\definecolor{textcolor}{rgb}{0.000000,0.000000,0.000000}%
\pgfsetstrokecolor{textcolor}%
\pgfsetfillcolor{textcolor}%
\pgftext[x=3.280000in,y=0.240063in,,top]{\color{textcolor}\ttfamily\fontsize{10.000000}{12.000000}\selectfont Size of Array}%
\end{pgfscope}%
\begin{pgfscope}%
\pgfsetbuttcap%
\pgfsetroundjoin%
\definecolor{currentfill}{rgb}{0.000000,0.000000,0.000000}%
\pgfsetfillcolor{currentfill}%
\pgfsetlinewidth{0.803000pt}%
\definecolor{currentstroke}{rgb}{0.000000,0.000000,0.000000}%
\pgfsetstrokecolor{currentstroke}%
\pgfsetdash{}{0pt}%
\pgfsys@defobject{currentmarker}{\pgfqpoint{-0.048611in}{0.000000in}}{\pgfqpoint{-0.000000in}{0.000000in}}{%
\pgfpathmoveto{\pgfqpoint{-0.000000in}{0.000000in}}%
\pgfpathlineto{\pgfqpoint{-0.048611in}{0.000000in}}%
\pgfusepath{stroke,fill}%
}%
\begin{pgfscope}%
\pgfsys@transformshift{0.800000in}{0.695329in}%
\pgfsys@useobject{currentmarker}{}%
\end{pgfscope}%
\end{pgfscope}%
\begin{pgfscope}%
\definecolor{textcolor}{rgb}{0.000000,0.000000,0.000000}%
\pgfsetstrokecolor{textcolor}%
\pgfsetfillcolor{textcolor}%
\pgftext[x=0.619160in, y=0.642194in, left, base]{\color{textcolor}\ttfamily\fontsize{10.000000}{12.000000}\selectfont 0}%
\end{pgfscope}%
\begin{pgfscope}%
\pgfsetbuttcap%
\pgfsetroundjoin%
\definecolor{currentfill}{rgb}{0.000000,0.000000,0.000000}%
\pgfsetfillcolor{currentfill}%
\pgfsetlinewidth{0.803000pt}%
\definecolor{currentstroke}{rgb}{0.000000,0.000000,0.000000}%
\pgfsetstrokecolor{currentstroke}%
\pgfsetdash{}{0pt}%
\pgfsys@defobject{currentmarker}{\pgfqpoint{-0.048611in}{0.000000in}}{\pgfqpoint{-0.000000in}{0.000000in}}{%
\pgfpathmoveto{\pgfqpoint{-0.000000in}{0.000000in}}%
\pgfpathlineto{\pgfqpoint{-0.048611in}{0.000000in}}%
\pgfusepath{stroke,fill}%
}%
\begin{pgfscope}%
\pgfsys@transformshift{0.800000in}{1.245650in}%
\pgfsys@useobject{currentmarker}{}%
\end{pgfscope}%
\end{pgfscope}%
\begin{pgfscope}%
\definecolor{textcolor}{rgb}{0.000000,0.000000,0.000000}%
\pgfsetstrokecolor{textcolor}%
\pgfsetfillcolor{textcolor}%
\pgftext[x=0.284687in, y=1.192516in, left, base]{\color{textcolor}\ttfamily\fontsize{10.000000}{12.000000}\selectfont 50000}%
\end{pgfscope}%
\begin{pgfscope}%
\pgfsetbuttcap%
\pgfsetroundjoin%
\definecolor{currentfill}{rgb}{0.000000,0.000000,0.000000}%
\pgfsetfillcolor{currentfill}%
\pgfsetlinewidth{0.803000pt}%
\definecolor{currentstroke}{rgb}{0.000000,0.000000,0.000000}%
\pgfsetstrokecolor{currentstroke}%
\pgfsetdash{}{0pt}%
\pgfsys@defobject{currentmarker}{\pgfqpoint{-0.048611in}{0.000000in}}{\pgfqpoint{-0.000000in}{0.000000in}}{%
\pgfpathmoveto{\pgfqpoint{-0.000000in}{0.000000in}}%
\pgfpathlineto{\pgfqpoint{-0.048611in}{0.000000in}}%
\pgfusepath{stroke,fill}%
}%
\begin{pgfscope}%
\pgfsys@transformshift{0.800000in}{1.795972in}%
\pgfsys@useobject{currentmarker}{}%
\end{pgfscope}%
\end{pgfscope}%
\begin{pgfscope}%
\definecolor{textcolor}{rgb}{0.000000,0.000000,0.000000}%
\pgfsetstrokecolor{textcolor}%
\pgfsetfillcolor{textcolor}%
\pgftext[x=0.201069in, y=1.742837in, left, base]{\color{textcolor}\ttfamily\fontsize{10.000000}{12.000000}\selectfont 100000}%
\end{pgfscope}%
\begin{pgfscope}%
\pgfsetbuttcap%
\pgfsetroundjoin%
\definecolor{currentfill}{rgb}{0.000000,0.000000,0.000000}%
\pgfsetfillcolor{currentfill}%
\pgfsetlinewidth{0.803000pt}%
\definecolor{currentstroke}{rgb}{0.000000,0.000000,0.000000}%
\pgfsetstrokecolor{currentstroke}%
\pgfsetdash{}{0pt}%
\pgfsys@defobject{currentmarker}{\pgfqpoint{-0.048611in}{0.000000in}}{\pgfqpoint{-0.000000in}{0.000000in}}{%
\pgfpathmoveto{\pgfqpoint{-0.000000in}{0.000000in}}%
\pgfpathlineto{\pgfqpoint{-0.048611in}{0.000000in}}%
\pgfusepath{stroke,fill}%
}%
\begin{pgfscope}%
\pgfsys@transformshift{0.800000in}{2.346294in}%
\pgfsys@useobject{currentmarker}{}%
\end{pgfscope}%
\end{pgfscope}%
\begin{pgfscope}%
\definecolor{textcolor}{rgb}{0.000000,0.000000,0.000000}%
\pgfsetstrokecolor{textcolor}%
\pgfsetfillcolor{textcolor}%
\pgftext[x=0.201069in, y=2.293159in, left, base]{\color{textcolor}\ttfamily\fontsize{10.000000}{12.000000}\selectfont 150000}%
\end{pgfscope}%
\begin{pgfscope}%
\pgfsetbuttcap%
\pgfsetroundjoin%
\definecolor{currentfill}{rgb}{0.000000,0.000000,0.000000}%
\pgfsetfillcolor{currentfill}%
\pgfsetlinewidth{0.803000pt}%
\definecolor{currentstroke}{rgb}{0.000000,0.000000,0.000000}%
\pgfsetstrokecolor{currentstroke}%
\pgfsetdash{}{0pt}%
\pgfsys@defobject{currentmarker}{\pgfqpoint{-0.048611in}{0.000000in}}{\pgfqpoint{-0.000000in}{0.000000in}}{%
\pgfpathmoveto{\pgfqpoint{-0.000000in}{0.000000in}}%
\pgfpathlineto{\pgfqpoint{-0.048611in}{0.000000in}}%
\pgfusepath{stroke,fill}%
}%
\begin{pgfscope}%
\pgfsys@transformshift{0.800000in}{2.896615in}%
\pgfsys@useobject{currentmarker}{}%
\end{pgfscope}%
\end{pgfscope}%
\begin{pgfscope}%
\definecolor{textcolor}{rgb}{0.000000,0.000000,0.000000}%
\pgfsetstrokecolor{textcolor}%
\pgfsetfillcolor{textcolor}%
\pgftext[x=0.201069in, y=2.843481in, left, base]{\color{textcolor}\ttfamily\fontsize{10.000000}{12.000000}\selectfont 200000}%
\end{pgfscope}%
\begin{pgfscope}%
\pgfsetbuttcap%
\pgfsetroundjoin%
\definecolor{currentfill}{rgb}{0.000000,0.000000,0.000000}%
\pgfsetfillcolor{currentfill}%
\pgfsetlinewidth{0.803000pt}%
\definecolor{currentstroke}{rgb}{0.000000,0.000000,0.000000}%
\pgfsetstrokecolor{currentstroke}%
\pgfsetdash{}{0pt}%
\pgfsys@defobject{currentmarker}{\pgfqpoint{-0.048611in}{0.000000in}}{\pgfqpoint{-0.000000in}{0.000000in}}{%
\pgfpathmoveto{\pgfqpoint{-0.000000in}{0.000000in}}%
\pgfpathlineto{\pgfqpoint{-0.048611in}{0.000000in}}%
\pgfusepath{stroke,fill}%
}%
\begin{pgfscope}%
\pgfsys@transformshift{0.800000in}{3.446937in}%
\pgfsys@useobject{currentmarker}{}%
\end{pgfscope}%
\end{pgfscope}%
\begin{pgfscope}%
\definecolor{textcolor}{rgb}{0.000000,0.000000,0.000000}%
\pgfsetstrokecolor{textcolor}%
\pgfsetfillcolor{textcolor}%
\pgftext[x=0.201069in, y=3.393802in, left, base]{\color{textcolor}\ttfamily\fontsize{10.000000}{12.000000}\selectfont 250000}%
\end{pgfscope}%
\begin{pgfscope}%
\pgfsetbuttcap%
\pgfsetroundjoin%
\definecolor{currentfill}{rgb}{0.000000,0.000000,0.000000}%
\pgfsetfillcolor{currentfill}%
\pgfsetlinewidth{0.803000pt}%
\definecolor{currentstroke}{rgb}{0.000000,0.000000,0.000000}%
\pgfsetstrokecolor{currentstroke}%
\pgfsetdash{}{0pt}%
\pgfsys@defobject{currentmarker}{\pgfqpoint{-0.048611in}{0.000000in}}{\pgfqpoint{-0.000000in}{0.000000in}}{%
\pgfpathmoveto{\pgfqpoint{-0.000000in}{0.000000in}}%
\pgfpathlineto{\pgfqpoint{-0.048611in}{0.000000in}}%
\pgfusepath{stroke,fill}%
}%
\begin{pgfscope}%
\pgfsys@transformshift{0.800000in}{3.997259in}%
\pgfsys@useobject{currentmarker}{}%
\end{pgfscope}%
\end{pgfscope}%
\begin{pgfscope}%
\definecolor{textcolor}{rgb}{0.000000,0.000000,0.000000}%
\pgfsetstrokecolor{textcolor}%
\pgfsetfillcolor{textcolor}%
\pgftext[x=0.201069in, y=3.944124in, left, base]{\color{textcolor}\ttfamily\fontsize{10.000000}{12.000000}\selectfont 300000}%
\end{pgfscope}%
\begin{pgfscope}%
\definecolor{textcolor}{rgb}{0.000000,0.000000,0.000000}%
\pgfsetstrokecolor{textcolor}%
\pgfsetfillcolor{textcolor}%
\pgftext[x=0.145513in,y=2.376000in,,bottom,rotate=90.000000]{\color{textcolor}\ttfamily\fontsize{10.000000}{12.000000}\selectfont Swaps}%
\end{pgfscope}%
\begin{pgfscope}%
\pgfpathrectangle{\pgfqpoint{0.800000in}{0.528000in}}{\pgfqpoint{4.960000in}{3.696000in}}%
\pgfusepath{clip}%
\pgfsetrectcap%
\pgfsetroundjoin%
\pgfsetlinewidth{1.505625pt}%
\definecolor{currentstroke}{rgb}{1.000000,0.000000,0.000000}%
\pgfsetstrokecolor{currentstroke}%
\pgfsetdash{}{0pt}%
\pgfpathmoveto{\pgfqpoint{1.025455in}{0.715933in}}%
\pgfpathlineto{\pgfqpoint{1.034473in}{0.723483in}}%
\pgfpathlineto{\pgfqpoint{1.038982in}{0.723164in}}%
\pgfpathlineto{\pgfqpoint{1.043491in}{0.725321in}}%
\pgfpathlineto{\pgfqpoint{1.048000in}{0.729504in}}%
\pgfpathlineto{\pgfqpoint{1.052509in}{0.724529in}}%
\pgfpathlineto{\pgfqpoint{1.057018in}{0.722757in}}%
\pgfpathlineto{\pgfqpoint{1.066036in}{0.729548in}}%
\pgfpathlineto{\pgfqpoint{1.070545in}{0.729911in}}%
\pgfpathlineto{\pgfqpoint{1.075055in}{0.725761in}}%
\pgfpathlineto{\pgfqpoint{1.079564in}{0.727346in}}%
\pgfpathlineto{\pgfqpoint{1.088582in}{0.732993in}}%
\pgfpathlineto{\pgfqpoint{1.097600in}{0.729195in}}%
\pgfpathlineto{\pgfqpoint{1.102109in}{0.731342in}}%
\pgfpathlineto{\pgfqpoint{1.106618in}{0.736240in}}%
\pgfpathlineto{\pgfqpoint{1.115636in}{0.734467in}}%
\pgfpathlineto{\pgfqpoint{1.120145in}{0.739442in}}%
\pgfpathlineto{\pgfqpoint{1.129164in}{0.733004in}}%
\pgfpathlineto{\pgfqpoint{1.133673in}{0.735326in}}%
\pgfpathlineto{\pgfqpoint{1.138182in}{0.741236in}}%
\pgfpathlineto{\pgfqpoint{1.142691in}{0.738595in}}%
\pgfpathlineto{\pgfqpoint{1.147200in}{0.743405in}}%
\pgfpathlineto{\pgfqpoint{1.151709in}{0.736306in}}%
\pgfpathlineto{\pgfqpoint{1.156218in}{0.743163in}}%
\pgfpathlineto{\pgfqpoint{1.160727in}{0.743041in}}%
\pgfpathlineto{\pgfqpoint{1.165236in}{0.741093in}}%
\pgfpathlineto{\pgfqpoint{1.169745in}{0.746960in}}%
\pgfpathlineto{\pgfqpoint{1.174255in}{0.745881in}}%
\pgfpathlineto{\pgfqpoint{1.178764in}{0.739883in}}%
\pgfpathlineto{\pgfqpoint{1.183273in}{0.747411in}}%
\pgfpathlineto{\pgfqpoint{1.187782in}{0.740928in}}%
\pgfpathlineto{\pgfqpoint{1.196800in}{0.748082in}}%
\pgfpathlineto{\pgfqpoint{1.201309in}{0.745034in}}%
\pgfpathlineto{\pgfqpoint{1.205818in}{0.749447in}}%
\pgfpathlineto{\pgfqpoint{1.214836in}{0.749954in}}%
\pgfpathlineto{\pgfqpoint{1.219345in}{0.754543in}}%
\pgfpathlineto{\pgfqpoint{1.228364in}{0.750570in}}%
\pgfpathlineto{\pgfqpoint{1.232873in}{0.757889in}}%
\pgfpathlineto{\pgfqpoint{1.237382in}{0.754939in}}%
\pgfpathlineto{\pgfqpoint{1.246400in}{0.751527in}}%
\pgfpathlineto{\pgfqpoint{1.250909in}{0.757306in}}%
\pgfpathlineto{\pgfqpoint{1.255418in}{0.757933in}}%
\pgfpathlineto{\pgfqpoint{1.259927in}{0.761235in}}%
\pgfpathlineto{\pgfqpoint{1.264436in}{0.753355in}}%
\pgfpathlineto{\pgfqpoint{1.268945in}{0.761037in}}%
\pgfpathlineto{\pgfqpoint{1.273455in}{0.759639in}}%
\pgfpathlineto{\pgfqpoint{1.277964in}{0.759628in}}%
\pgfpathlineto{\pgfqpoint{1.282473in}{0.763866in}}%
\pgfpathlineto{\pgfqpoint{1.291491in}{0.761962in}}%
\pgfpathlineto{\pgfqpoint{1.296000in}{0.776237in}}%
\pgfpathlineto{\pgfqpoint{1.300509in}{0.764834in}}%
\pgfpathlineto{\pgfqpoint{1.305018in}{0.760421in}}%
\pgfpathlineto{\pgfqpoint{1.318545in}{0.773573in}}%
\pgfpathlineto{\pgfqpoint{1.323055in}{0.773738in}}%
\pgfpathlineto{\pgfqpoint{1.327564in}{0.762325in}}%
\pgfpathlineto{\pgfqpoint{1.332073in}{0.777448in}}%
\pgfpathlineto{\pgfqpoint{1.336582in}{0.774333in}}%
\pgfpathlineto{\pgfqpoint{1.341091in}{0.779330in}}%
\pgfpathlineto{\pgfqpoint{1.345600in}{0.777987in}}%
\pgfpathlineto{\pgfqpoint{1.350109in}{0.779715in}}%
\pgfpathlineto{\pgfqpoint{1.354618in}{0.770888in}}%
\pgfpathlineto{\pgfqpoint{1.359127in}{0.777800in}}%
\pgfpathlineto{\pgfqpoint{1.363636in}{0.777580in}}%
\pgfpathlineto{\pgfqpoint{1.368145in}{0.783259in}}%
\pgfpathlineto{\pgfqpoint{1.372655in}{0.783336in}}%
\pgfpathlineto{\pgfqpoint{1.377164in}{0.774927in}}%
\pgfpathlineto{\pgfqpoint{1.381673in}{0.780893in}}%
\pgfpathlineto{\pgfqpoint{1.386182in}{0.782720in}}%
\pgfpathlineto{\pgfqpoint{1.395200in}{0.777073in}}%
\pgfpathlineto{\pgfqpoint{1.399709in}{0.782698in}}%
\pgfpathlineto{\pgfqpoint{1.404218in}{0.791206in}}%
\pgfpathlineto{\pgfqpoint{1.408727in}{0.787089in}}%
\pgfpathlineto{\pgfqpoint{1.422255in}{0.792185in}}%
\pgfpathlineto{\pgfqpoint{1.426764in}{0.790391in}}%
\pgfpathlineto{\pgfqpoint{1.431273in}{0.785229in}}%
\pgfpathlineto{\pgfqpoint{1.435782in}{0.793187in}}%
\pgfpathlineto{\pgfqpoint{1.444800in}{0.799020in}}%
\pgfpathlineto{\pgfqpoint{1.449309in}{0.806505in}}%
\pgfpathlineto{\pgfqpoint{1.458327in}{0.800352in}}%
\pgfpathlineto{\pgfqpoint{1.462836in}{0.796841in}}%
\pgfpathlineto{\pgfqpoint{1.467345in}{0.798063in}}%
\pgfpathlineto{\pgfqpoint{1.471855in}{0.801618in}}%
\pgfpathlineto{\pgfqpoint{1.476364in}{0.811634in}}%
\pgfpathlineto{\pgfqpoint{1.485382in}{0.799692in}}%
\pgfpathlineto{\pgfqpoint{1.489891in}{0.804534in}}%
\pgfpathlineto{\pgfqpoint{1.494400in}{0.800924in}}%
\pgfpathlineto{\pgfqpoint{1.498909in}{0.812173in}}%
\pgfpathlineto{\pgfqpoint{1.503418in}{0.812316in}}%
\pgfpathlineto{\pgfqpoint{1.507927in}{0.806714in}}%
\pgfpathlineto{\pgfqpoint{1.512436in}{0.819349in}}%
\pgfpathlineto{\pgfqpoint{1.516945in}{0.810676in}}%
\pgfpathlineto{\pgfqpoint{1.521455in}{0.814264in}}%
\pgfpathlineto{\pgfqpoint{1.525964in}{0.811457in}}%
\pgfpathlineto{\pgfqpoint{1.530473in}{0.813097in}}%
\pgfpathlineto{\pgfqpoint{1.534982in}{0.819294in}}%
\pgfpathlineto{\pgfqpoint{1.539491in}{0.815805in}}%
\pgfpathlineto{\pgfqpoint{1.544000in}{0.826569in}}%
\pgfpathlineto{\pgfqpoint{1.548509in}{0.827131in}}%
\pgfpathlineto{\pgfqpoint{1.553018in}{0.830113in}}%
\pgfpathlineto{\pgfqpoint{1.557527in}{0.826679in}}%
\pgfpathlineto{\pgfqpoint{1.562036in}{0.828011in}}%
\pgfpathlineto{\pgfqpoint{1.571055in}{0.827450in}}%
\pgfpathlineto{\pgfqpoint{1.575564in}{0.833988in}}%
\pgfpathlineto{\pgfqpoint{1.580073in}{0.821991in}}%
\pgfpathlineto{\pgfqpoint{1.584582in}{0.828716in}}%
\pgfpathlineto{\pgfqpoint{1.589091in}{0.839645in}}%
\pgfpathlineto{\pgfqpoint{1.593600in}{0.842253in}}%
\pgfpathlineto{\pgfqpoint{1.598109in}{0.836552in}}%
\pgfpathlineto{\pgfqpoint{1.602618in}{0.839447in}}%
\pgfpathlineto{\pgfqpoint{1.607127in}{0.849661in}}%
\pgfpathlineto{\pgfqpoint{1.611636in}{0.836937in}}%
\pgfpathlineto{\pgfqpoint{1.616145in}{0.833250in}}%
\pgfpathlineto{\pgfqpoint{1.620655in}{0.847911in}}%
\pgfpathlineto{\pgfqpoint{1.625164in}{0.836574in}}%
\pgfpathlineto{\pgfqpoint{1.629673in}{0.835650in}}%
\pgfpathlineto{\pgfqpoint{1.634182in}{0.854955in}}%
\pgfpathlineto{\pgfqpoint{1.638691in}{0.851136in}}%
\pgfpathlineto{\pgfqpoint{1.643200in}{0.851730in}}%
\pgfpathlineto{\pgfqpoint{1.647709in}{0.853799in}}%
\pgfpathlineto{\pgfqpoint{1.652218in}{0.850299in}}%
\pgfpathlineto{\pgfqpoint{1.656727in}{0.856188in}}%
\pgfpathlineto{\pgfqpoint{1.661236in}{0.851532in}}%
\pgfpathlineto{\pgfqpoint{1.665745in}{0.850607in}}%
\pgfpathlineto{\pgfqpoint{1.670255in}{0.863738in}}%
\pgfpathlineto{\pgfqpoint{1.674764in}{0.863892in}}%
\pgfpathlineto{\pgfqpoint{1.679273in}{0.871321in}}%
\pgfpathlineto{\pgfqpoint{1.683782in}{0.865224in}}%
\pgfpathlineto{\pgfqpoint{1.688291in}{0.842419in}}%
\pgfpathlineto{\pgfqpoint{1.697309in}{0.871277in}}%
\pgfpathlineto{\pgfqpoint{1.701818in}{0.872246in}}%
\pgfpathlineto{\pgfqpoint{1.706327in}{0.875042in}}%
\pgfpathlineto{\pgfqpoint{1.710836in}{0.872411in}}%
\pgfpathlineto{\pgfqpoint{1.715345in}{0.860865in}}%
\pgfpathlineto{\pgfqpoint{1.719855in}{0.868922in}}%
\pgfpathlineto{\pgfqpoint{1.724364in}{0.886378in}}%
\pgfpathlineto{\pgfqpoint{1.728873in}{0.880952in}}%
\pgfpathlineto{\pgfqpoint{1.733382in}{0.878839in}}%
\pgfpathlineto{\pgfqpoint{1.737891in}{0.879499in}}%
\pgfpathlineto{\pgfqpoint{1.742400in}{0.873501in}}%
\pgfpathlineto{\pgfqpoint{1.746909in}{0.874425in}}%
\pgfpathlineto{\pgfqpoint{1.751418in}{0.876638in}}%
\pgfpathlineto{\pgfqpoint{1.755927in}{0.873600in}}%
\pgfpathlineto{\pgfqpoint{1.764945in}{0.895051in}}%
\pgfpathlineto{\pgfqpoint{1.769455in}{0.878641in}}%
\pgfpathlineto{\pgfqpoint{1.773964in}{0.886455in}}%
\pgfpathlineto{\pgfqpoint{1.778473in}{0.887743in}}%
\pgfpathlineto{\pgfqpoint{1.782982in}{0.882746in}}%
\pgfpathlineto{\pgfqpoint{1.787491in}{0.896592in}}%
\pgfpathlineto{\pgfqpoint{1.792000in}{0.888723in}}%
\pgfpathlineto{\pgfqpoint{1.801018in}{0.909921in}}%
\pgfpathlineto{\pgfqpoint{1.805527in}{0.886587in}}%
\pgfpathlineto{\pgfqpoint{1.810036in}{0.903449in}}%
\pgfpathlineto{\pgfqpoint{1.814545in}{0.901072in}}%
\pgfpathlineto{\pgfqpoint{1.819055in}{0.912122in}}%
\pgfpathlineto{\pgfqpoint{1.828073in}{0.914940in}}%
\pgfpathlineto{\pgfqpoint{1.832582in}{0.914423in}}%
\pgfpathlineto{\pgfqpoint{1.837091in}{0.911451in}}%
\pgfpathlineto{\pgfqpoint{1.841600in}{0.910394in}}%
\pgfpathlineto{\pgfqpoint{1.846109in}{0.901776in}}%
\pgfpathlineto{\pgfqpoint{1.850618in}{0.905254in}}%
\pgfpathlineto{\pgfqpoint{1.855127in}{0.921709in}}%
\pgfpathlineto{\pgfqpoint{1.859636in}{0.925836in}}%
\pgfpathlineto{\pgfqpoint{1.864145in}{0.927509in}}%
\pgfpathlineto{\pgfqpoint{1.868655in}{0.926970in}}%
\pgfpathlineto{\pgfqpoint{1.873164in}{0.914698in}}%
\pgfpathlineto{\pgfqpoint{1.877673in}{0.917780in}}%
\pgfpathlineto{\pgfqpoint{1.886691in}{0.942489in}}%
\pgfpathlineto{\pgfqpoint{1.891200in}{0.917604in}}%
\pgfpathlineto{\pgfqpoint{1.895709in}{0.931505in}}%
\pgfpathlineto{\pgfqpoint{1.900218in}{0.936358in}}%
\pgfpathlineto{\pgfqpoint{1.904727in}{0.943898in}}%
\pgfpathlineto{\pgfqpoint{1.909236in}{0.925407in}}%
\pgfpathlineto{\pgfqpoint{1.913745in}{0.945769in}}%
\pgfpathlineto{\pgfqpoint{1.918255in}{0.947904in}}%
\pgfpathlineto{\pgfqpoint{1.922764in}{0.940002in}}%
\pgfpathlineto{\pgfqpoint{1.927273in}{0.952065in}}%
\pgfpathlineto{\pgfqpoint{1.931782in}{0.949621in}}%
\pgfpathlineto{\pgfqpoint{1.936291in}{0.954134in}}%
\pgfpathlineto{\pgfqpoint{1.940800in}{0.941829in}}%
\pgfpathlineto{\pgfqpoint{1.945309in}{0.959890in}}%
\pgfpathlineto{\pgfqpoint{1.949818in}{0.957887in}}%
\pgfpathlineto{\pgfqpoint{1.958836in}{0.956478in}}%
\pgfpathlineto{\pgfqpoint{1.963345in}{0.955224in}}%
\pgfpathlineto{\pgfqpoint{1.967855in}{0.971425in}}%
\pgfpathlineto{\pgfqpoint{1.972364in}{0.963060in}}%
\pgfpathlineto{\pgfqpoint{1.976873in}{0.972746in}}%
\pgfpathlineto{\pgfqpoint{1.981382in}{0.970842in}}%
\pgfpathlineto{\pgfqpoint{1.985891in}{0.972933in}}%
\pgfpathlineto{\pgfqpoint{1.990400in}{0.976653in}}%
\pgfpathlineto{\pgfqpoint{1.999418in}{0.959318in}}%
\pgfpathlineto{\pgfqpoint{2.003927in}{0.961068in}}%
\pgfpathlineto{\pgfqpoint{2.008436in}{0.966714in}}%
\pgfpathlineto{\pgfqpoint{2.012945in}{0.981606in}}%
\pgfpathlineto{\pgfqpoint{2.017455in}{0.987593in}}%
\pgfpathlineto{\pgfqpoint{2.021964in}{0.970434in}}%
\pgfpathlineto{\pgfqpoint{2.026473in}{0.962488in}}%
\pgfpathlineto{\pgfqpoint{2.030982in}{0.990972in}}%
\pgfpathlineto{\pgfqpoint{2.035491in}{1.004951in}}%
\pgfpathlineto{\pgfqpoint{2.040000in}{0.973406in}}%
\pgfpathlineto{\pgfqpoint{2.044509in}{0.976763in}}%
\pgfpathlineto{\pgfqpoint{2.049018in}{0.992282in}}%
\pgfpathlineto{\pgfqpoint{2.053527in}{0.985832in}}%
\pgfpathlineto{\pgfqpoint{2.058036in}{0.996938in}}%
\pgfpathlineto{\pgfqpoint{2.062545in}{0.999866in}}%
\pgfpathlineto{\pgfqpoint{2.067055in}{1.000526in}}%
\pgfpathlineto{\pgfqpoint{2.071564in}{0.988034in}}%
\pgfpathlineto{\pgfqpoint{2.076073in}{1.029891in}}%
\pgfpathlineto{\pgfqpoint{2.080582in}{0.986372in}}%
\pgfpathlineto{\pgfqpoint{2.089600in}{0.987836in}}%
\pgfpathlineto{\pgfqpoint{2.094109in}{1.010729in}}%
\pgfpathlineto{\pgfqpoint{2.098618in}{1.008396in}}%
\pgfpathlineto{\pgfqpoint{2.103127in}{1.020040in}}%
\pgfpathlineto{\pgfqpoint{2.107636in}{1.012479in}}%
\pgfpathlineto{\pgfqpoint{2.112145in}{0.999877in}}%
\pgfpathlineto{\pgfqpoint{2.116655in}{1.037497in}}%
\pgfpathlineto{\pgfqpoint{2.121164in}{1.024949in}}%
\pgfpathlineto{\pgfqpoint{2.125673in}{1.028185in}}%
\pgfpathlineto{\pgfqpoint{2.130182in}{1.040138in}}%
\pgfpathlineto{\pgfqpoint{2.134691in}{1.017762in}}%
\pgfpathlineto{\pgfqpoint{2.139200in}{1.031674in}}%
\pgfpathlineto{\pgfqpoint{2.143709in}{1.025114in}}%
\pgfpathlineto{\pgfqpoint{2.148218in}{1.038113in}}%
\pgfpathlineto{\pgfqpoint{2.152727in}{1.020998in}}%
\pgfpathlineto{\pgfqpoint{2.161745in}{1.018742in}}%
\pgfpathlineto{\pgfqpoint{2.166255in}{1.003586in}}%
\pgfpathlineto{\pgfqpoint{2.170764in}{1.045531in}}%
\pgfpathlineto{\pgfqpoint{2.175273in}{1.051244in}}%
\pgfpathlineto{\pgfqpoint{2.179782in}{1.050704in}}%
\pgfpathlineto{\pgfqpoint{2.184291in}{1.044530in}}%
\pgfpathlineto{\pgfqpoint{2.188800in}{1.047435in}}%
\pgfpathlineto{\pgfqpoint{2.193309in}{1.054028in}}%
\pgfpathlineto{\pgfqpoint{2.197818in}{1.038047in}}%
\pgfpathlineto{\pgfqpoint{2.202327in}{1.041426in}}%
\pgfpathlineto{\pgfqpoint{2.206836in}{1.067225in}}%
\pgfpathlineto{\pgfqpoint{2.211345in}{1.040545in}}%
\pgfpathlineto{\pgfqpoint{2.215855in}{1.062096in}}%
\pgfpathlineto{\pgfqpoint{2.220364in}{1.049549in}}%
\pgfpathlineto{\pgfqpoint{2.224873in}{1.045091in}}%
\pgfpathlineto{\pgfqpoint{2.229382in}{1.090845in}}%
\pgfpathlineto{\pgfqpoint{2.233891in}{1.062316in}}%
\pgfpathlineto{\pgfqpoint{2.238400in}{1.061656in}}%
\pgfpathlineto{\pgfqpoint{2.242909in}{1.084593in}}%
\pgfpathlineto{\pgfqpoint{2.247418in}{1.064958in}}%
\pgfpathlineto{\pgfqpoint{2.251927in}{1.090184in}}%
\pgfpathlineto{\pgfqpoint{2.256436in}{1.055360in}}%
\pgfpathlineto{\pgfqpoint{2.260945in}{1.074236in}}%
\pgfpathlineto{\pgfqpoint{2.269964in}{1.098439in}}%
\pgfpathlineto{\pgfqpoint{2.274473in}{1.071429in}}%
\pgfpathlineto{\pgfqpoint{2.278982in}{1.099738in}}%
\pgfpathlineto{\pgfqpoint{2.283491in}{1.086871in}}%
\pgfpathlineto{\pgfqpoint{2.288000in}{1.069624in}}%
\pgfpathlineto{\pgfqpoint{2.292509in}{1.081798in}}%
\pgfpathlineto{\pgfqpoint{2.297018in}{1.105197in}}%
\pgfpathlineto{\pgfqpoint{2.301527in}{1.109369in}}%
\pgfpathlineto{\pgfqpoint{2.310545in}{1.123160in}}%
\pgfpathlineto{\pgfqpoint{2.315055in}{1.085254in}}%
\pgfpathlineto{\pgfqpoint{2.319564in}{1.098692in}}%
\pgfpathlineto{\pgfqpoint{2.324073in}{1.093673in}}%
\pgfpathlineto{\pgfqpoint{2.328582in}{1.108433in}}%
\pgfpathlineto{\pgfqpoint{2.333091in}{1.113573in}}%
\pgfpathlineto{\pgfqpoint{2.337600in}{1.128201in}}%
\pgfpathlineto{\pgfqpoint{2.342109in}{1.114850in}}%
\pgfpathlineto{\pgfqpoint{2.346618in}{1.124293in}}%
\pgfpathlineto{\pgfqpoint{2.351127in}{1.098142in}}%
\pgfpathlineto{\pgfqpoint{2.355636in}{1.107806in}}%
\pgfpathlineto{\pgfqpoint{2.360145in}{1.140121in}}%
\pgfpathlineto{\pgfqpoint{2.364655in}{1.147484in}}%
\pgfpathlineto{\pgfqpoint{2.369164in}{1.104669in}}%
\pgfpathlineto{\pgfqpoint{2.373673in}{1.136191in}}%
\pgfpathlineto{\pgfqpoint{2.378182in}{1.148629in}}%
\pgfpathlineto{\pgfqpoint{2.382691in}{1.131569in}}%
\pgfpathlineto{\pgfqpoint{2.387200in}{1.136334in}}%
\pgfpathlineto{\pgfqpoint{2.391709in}{1.144545in}}%
\pgfpathlineto{\pgfqpoint{2.396218in}{1.129422in}}%
\pgfpathlineto{\pgfqpoint{2.400727in}{1.165215in}}%
\pgfpathlineto{\pgfqpoint{2.405236in}{1.144677in}}%
\pgfpathlineto{\pgfqpoint{2.409745in}{1.136797in}}%
\pgfpathlineto{\pgfqpoint{2.414255in}{1.172039in}}%
\pgfpathlineto{\pgfqpoint{2.418764in}{1.130644in}}%
\pgfpathlineto{\pgfqpoint{2.423273in}{1.132218in}}%
\pgfpathlineto{\pgfqpoint{2.432291in}{1.152415in}}%
\pgfpathlineto{\pgfqpoint{2.436800in}{1.152349in}}%
\pgfpathlineto{\pgfqpoint{2.441309in}{1.165215in}}%
\pgfpathlineto{\pgfqpoint{2.445818in}{1.163278in}}%
\pgfpathlineto{\pgfqpoint{2.450327in}{1.176959in}}%
\pgfpathlineto{\pgfqpoint{2.454836in}{1.172557in}}%
\pgfpathlineto{\pgfqpoint{2.459345in}{1.151523in}}%
\pgfpathlineto{\pgfqpoint{2.463855in}{1.177179in}}%
\pgfpathlineto{\pgfqpoint{2.468364in}{1.166646in}}%
\pgfpathlineto{\pgfqpoint{2.472873in}{1.185401in}}%
\pgfpathlineto{\pgfqpoint{2.477382in}{1.192126in}}%
\pgfpathlineto{\pgfqpoint{2.481891in}{1.173041in}}%
\pgfpathlineto{\pgfqpoint{2.486400in}{1.182660in}}%
\pgfpathlineto{\pgfqpoint{2.490909in}{1.217265in}}%
\pgfpathlineto{\pgfqpoint{2.495418in}{1.181516in}}%
\pgfpathlineto{\pgfqpoint{2.499927in}{1.188230in}}%
\pgfpathlineto{\pgfqpoint{2.504436in}{1.241743in}}%
\pgfpathlineto{\pgfqpoint{2.508945in}{1.205488in}}%
\pgfpathlineto{\pgfqpoint{2.513455in}{1.201702in}}%
\pgfpathlineto{\pgfqpoint{2.517964in}{1.223836in}}%
\pgfpathlineto{\pgfqpoint{2.522473in}{1.194195in}}%
\pgfpathlineto{\pgfqpoint{2.526982in}{1.185984in}}%
\pgfpathlineto{\pgfqpoint{2.531491in}{1.205895in}}%
\pgfpathlineto{\pgfqpoint{2.536000in}{1.202681in}}%
\pgfpathlineto{\pgfqpoint{2.540509in}{1.211178in}}%
\pgfpathlineto{\pgfqpoint{2.545018in}{1.232839in}}%
\pgfpathlineto{\pgfqpoint{2.549527in}{1.235557in}}%
\pgfpathlineto{\pgfqpoint{2.554036in}{1.268313in}}%
\pgfpathlineto{\pgfqpoint{2.558545in}{1.193084in}}%
\pgfpathlineto{\pgfqpoint{2.563055in}{1.206688in}}%
\pgfpathlineto{\pgfqpoint{2.567564in}{1.232905in}}%
\pgfpathlineto{\pgfqpoint{2.572073in}{1.223329in}}%
\pgfpathlineto{\pgfqpoint{2.576582in}{1.252771in}}%
\pgfpathlineto{\pgfqpoint{2.581091in}{1.228469in}}%
\pgfpathlineto{\pgfqpoint{2.585600in}{1.219730in}}%
\pgfpathlineto{\pgfqpoint{2.590109in}{1.250625in}}%
\pgfpathlineto{\pgfqpoint{2.594618in}{1.231331in}}%
\pgfpathlineto{\pgfqpoint{2.603636in}{1.254521in}}%
\pgfpathlineto{\pgfqpoint{2.608145in}{1.258319in}}%
\pgfpathlineto{\pgfqpoint{2.612655in}{1.286407in}}%
\pgfpathlineto{\pgfqpoint{2.617164in}{1.293011in}}%
\pgfpathlineto{\pgfqpoint{2.621673in}{1.229031in}}%
\pgfpathlineto{\pgfqpoint{2.626182in}{1.291613in}}%
\pgfpathlineto{\pgfqpoint{2.630691in}{1.303093in}}%
\pgfpathlineto{\pgfqpoint{2.635200in}{1.254191in}}%
\pgfpathlineto{\pgfqpoint{2.639709in}{1.254037in}}%
\pgfpathlineto{\pgfqpoint{2.644218in}{1.265671in}}%
\pgfpathlineto{\pgfqpoint{2.648727in}{1.282643in}}%
\pgfpathlineto{\pgfqpoint{2.653236in}{1.284140in}}%
\pgfpathlineto{\pgfqpoint{2.657745in}{1.294563in}}%
\pgfpathlineto{\pgfqpoint{2.662255in}{1.259948in}}%
\pgfpathlineto{\pgfqpoint{2.666764in}{1.276997in}}%
\pgfpathlineto{\pgfqpoint{2.671273in}{1.274718in}}%
\pgfpathlineto{\pgfqpoint{2.675782in}{1.286000in}}%
\pgfpathlineto{\pgfqpoint{2.680291in}{1.304535in}}%
\pgfpathlineto{\pgfqpoint{2.684800in}{1.284888in}}%
\pgfpathlineto{\pgfqpoint{2.689309in}{1.292857in}}%
\pgfpathlineto{\pgfqpoint{2.693818in}{1.310170in}}%
\pgfpathlineto{\pgfqpoint{2.698327in}{1.333074in}}%
\pgfpathlineto{\pgfqpoint{2.702836in}{1.268434in}}%
\pgfpathlineto{\pgfqpoint{2.707345in}{1.352424in}}%
\pgfpathlineto{\pgfqpoint{2.711855in}{1.325810in}}%
\pgfpathlineto{\pgfqpoint{2.716364in}{1.291789in}}%
\pgfpathlineto{\pgfqpoint{2.720873in}{1.333350in}}%
\pgfpathlineto{\pgfqpoint{2.725382in}{1.337169in}}%
\pgfpathlineto{\pgfqpoint{2.729891in}{1.325304in}}%
\pgfpathlineto{\pgfqpoint{2.734400in}{1.349309in}}%
\pgfpathlineto{\pgfqpoint{2.738909in}{1.332469in}}%
\pgfpathlineto{\pgfqpoint{2.743418in}{1.328375in}}%
\pgfpathlineto{\pgfqpoint{2.747927in}{1.340757in}}%
\pgfpathlineto{\pgfqpoint{2.752436in}{1.338225in}}%
\pgfpathlineto{\pgfqpoint{2.756945in}{1.322189in}}%
\pgfpathlineto{\pgfqpoint{2.761455in}{1.347306in}}%
\pgfpathlineto{\pgfqpoint{2.765964in}{1.345248in}}%
\pgfpathlineto{\pgfqpoint{2.770473in}{1.301299in}}%
\pgfpathlineto{\pgfqpoint{2.774982in}{1.332909in}}%
\pgfpathlineto{\pgfqpoint{2.779491in}{1.342540in}}%
\pgfpathlineto{\pgfqpoint{2.784000in}{1.333052in}}%
\pgfpathlineto{\pgfqpoint{2.793018in}{1.337400in}}%
\pgfpathlineto{\pgfqpoint{2.797527in}{1.358444in}}%
\pgfpathlineto{\pgfqpoint{2.806545in}{1.387160in}}%
\pgfpathlineto{\pgfqpoint{2.811055in}{1.390198in}}%
\pgfpathlineto{\pgfqpoint{2.815564in}{1.368790in}}%
\pgfpathlineto{\pgfqpoint{2.820073in}{1.408744in}}%
\pgfpathlineto{\pgfqpoint{2.824582in}{1.390429in}}%
\pgfpathlineto{\pgfqpoint{2.829091in}{1.347416in}}%
\pgfpathlineto{\pgfqpoint{2.833600in}{1.384265in}}%
\pgfpathlineto{\pgfqpoint{2.838109in}{1.397550in}}%
\pgfpathlineto{\pgfqpoint{2.842618in}{1.427939in}}%
\pgfpathlineto{\pgfqpoint{2.847127in}{1.416162in}}%
\pgfpathlineto{\pgfqpoint{2.851636in}{1.382009in}}%
\pgfpathlineto{\pgfqpoint{2.856145in}{1.360734in}}%
\pgfpathlineto{\pgfqpoint{2.860655in}{1.419607in}}%
\pgfpathlineto{\pgfqpoint{2.865164in}{1.413443in}}%
\pgfpathlineto{\pgfqpoint{2.869673in}{1.418528in}}%
\pgfpathlineto{\pgfqpoint{2.874182in}{1.403174in}}%
\pgfpathlineto{\pgfqpoint{2.878691in}{1.422678in}}%
\pgfpathlineto{\pgfqpoint{2.883200in}{1.403383in}}%
\pgfpathlineto{\pgfqpoint{2.887709in}{1.393841in}}%
\pgfpathlineto{\pgfqpoint{2.892218in}{1.407797in}}%
\pgfpathlineto{\pgfqpoint{2.901236in}{1.429216in}}%
\pgfpathlineto{\pgfqpoint{2.905745in}{1.419673in}}%
\pgfpathlineto{\pgfqpoint{2.910255in}{1.435225in}}%
\pgfpathlineto{\pgfqpoint{2.914764in}{1.428621in}}%
\pgfpathlineto{\pgfqpoint{2.919273in}{1.450832in}}%
\pgfpathlineto{\pgfqpoint{2.923782in}{1.437239in}}%
\pgfpathlineto{\pgfqpoint{2.928291in}{1.444988in}}%
\pgfpathlineto{\pgfqpoint{2.932800in}{1.425881in}}%
\pgfpathlineto{\pgfqpoint{2.937309in}{1.395382in}}%
\pgfpathlineto{\pgfqpoint{2.941818in}{1.458328in}}%
\pgfpathlineto{\pgfqpoint{2.946327in}{1.472581in}}%
\pgfpathlineto{\pgfqpoint{2.950836in}{1.426662in}}%
\pgfpathlineto{\pgfqpoint{2.955345in}{1.434091in}}%
\pgfpathlineto{\pgfqpoint{2.959855in}{1.437316in}}%
\pgfpathlineto{\pgfqpoint{2.964364in}{1.476609in}}%
\pgfpathlineto{\pgfqpoint{2.968873in}{1.466571in}}%
\pgfpathlineto{\pgfqpoint{2.973382in}{1.533402in}}%
\pgfpathlineto{\pgfqpoint{2.977891in}{1.488892in}}%
\pgfpathlineto{\pgfqpoint{2.982400in}{1.481463in}}%
\pgfpathlineto{\pgfqpoint{2.986909in}{1.422876in}}%
\pgfpathlineto{\pgfqpoint{2.991418in}{1.489300in}}%
\pgfpathlineto{\pgfqpoint{2.995927in}{1.501748in}}%
\pgfpathlineto{\pgfqpoint{3.000436in}{1.423558in}}%
\pgfpathlineto{\pgfqpoint{3.004945in}{1.475410in}}%
\pgfpathlineto{\pgfqpoint{3.009455in}{1.510058in}}%
\pgfpathlineto{\pgfqpoint{3.013964in}{1.498831in}}%
\pgfpathlineto{\pgfqpoint{3.018473in}{1.558145in}}%
\pgfpathlineto{\pgfqpoint{3.022982in}{1.541426in}}%
\pgfpathlineto{\pgfqpoint{3.027491in}{1.468663in}}%
\pgfpathlineto{\pgfqpoint{3.032000in}{1.529077in}}%
\pgfpathlineto{\pgfqpoint{3.036509in}{1.528802in}}%
\pgfpathlineto{\pgfqpoint{3.041018in}{1.536859in}}%
\pgfpathlineto{\pgfqpoint{3.045527in}{1.535461in}}%
\pgfpathlineto{\pgfqpoint{3.050036in}{1.531036in}}%
\pgfpathlineto{\pgfqpoint{3.054545in}{1.489938in}}%
\pgfpathlineto{\pgfqpoint{3.059055in}{1.508880in}}%
\pgfpathlineto{\pgfqpoint{3.063564in}{1.562074in}}%
\pgfpathlineto{\pgfqpoint{3.068073in}{1.562382in}}%
\pgfpathlineto{\pgfqpoint{3.072582in}{1.550892in}}%
\pgfpathlineto{\pgfqpoint{3.077091in}{1.496960in}}%
\pgfpathlineto{\pgfqpoint{3.081600in}{1.547006in}}%
\pgfpathlineto{\pgfqpoint{3.086109in}{1.530519in}}%
\pgfpathlineto{\pgfqpoint{3.090618in}{1.539676in}}%
\pgfpathlineto{\pgfqpoint{3.095127in}{1.589249in}}%
\pgfpathlineto{\pgfqpoint{3.099636in}{1.496630in}}%
\pgfpathlineto{\pgfqpoint{3.104145in}{1.522066in}}%
\pgfpathlineto{\pgfqpoint{3.108655in}{1.584659in}}%
\pgfpathlineto{\pgfqpoint{3.113164in}{1.593575in}}%
\pgfpathlineto{\pgfqpoint{3.117673in}{1.586420in}}%
\pgfpathlineto{\pgfqpoint{3.122182in}{1.565750in}}%
\pgfpathlineto{\pgfqpoint{3.126691in}{1.522385in}}%
\pgfpathlineto{\pgfqpoint{3.131200in}{1.575535in}}%
\pgfpathlineto{\pgfqpoint{3.135709in}{1.536473in}}%
\pgfpathlineto{\pgfqpoint{3.140218in}{1.585089in}}%
\pgfpathlineto{\pgfqpoint{3.144727in}{1.535648in}}%
\pgfpathlineto{\pgfqpoint{3.149236in}{1.584814in}}%
\pgfpathlineto{\pgfqpoint{3.153745in}{1.573510in}}%
\pgfpathlineto{\pgfqpoint{3.158255in}{1.577824in}}%
\pgfpathlineto{\pgfqpoint{3.162764in}{1.626759in}}%
\pgfpathlineto{\pgfqpoint{3.167273in}{1.618856in}}%
\pgfpathlineto{\pgfqpoint{3.171782in}{1.630446in}}%
\pgfpathlineto{\pgfqpoint{3.176291in}{1.624140in}}%
\pgfpathlineto{\pgfqpoint{3.180800in}{1.589073in}}%
\pgfpathlineto{\pgfqpoint{3.185309in}{1.604867in}}%
\pgfpathlineto{\pgfqpoint{3.189818in}{1.655244in}}%
\pgfpathlineto{\pgfqpoint{3.194327in}{1.631239in}}%
\pgfpathlineto{\pgfqpoint{3.198836in}{1.647231in}}%
\pgfpathlineto{\pgfqpoint{3.203345in}{1.608301in}}%
\pgfpathlineto{\pgfqpoint{3.207855in}{1.617437in}}%
\pgfpathlineto{\pgfqpoint{3.212364in}{1.637655in}}%
\pgfpathlineto{\pgfqpoint{3.216873in}{1.636830in}}%
\pgfpathlineto{\pgfqpoint{3.221382in}{1.674857in}}%
\pgfpathlineto{\pgfqpoint{3.225891in}{1.615246in}}%
\pgfpathlineto{\pgfqpoint{3.230400in}{1.625537in}}%
\pgfpathlineto{\pgfqpoint{3.234909in}{1.642069in}}%
\pgfpathlineto{\pgfqpoint{3.239418in}{1.663245in}}%
\pgfpathlineto{\pgfqpoint{3.243927in}{1.612539in}}%
\pgfpathlineto{\pgfqpoint{3.252945in}{1.693513in}}%
\pgfpathlineto{\pgfqpoint{3.257455in}{1.646384in}}%
\pgfpathlineto{\pgfqpoint{3.261964in}{1.720644in}}%
\pgfpathlineto{\pgfqpoint{3.266473in}{1.676684in}}%
\pgfpathlineto{\pgfqpoint{3.275491in}{1.684312in}}%
\pgfpathlineto{\pgfqpoint{3.280000in}{1.701614in}}%
\pgfpathlineto{\pgfqpoint{3.284509in}{1.676486in}}%
\pgfpathlineto{\pgfqpoint{3.289018in}{1.721271in}}%
\pgfpathlineto{\pgfqpoint{3.293527in}{1.710386in}}%
\pgfpathlineto{\pgfqpoint{3.298036in}{1.677224in}}%
\pgfpathlineto{\pgfqpoint{3.302545in}{1.763151in}}%
\pgfpathlineto{\pgfqpoint{3.307055in}{1.745992in}}%
\pgfpathlineto{\pgfqpoint{3.316073in}{1.722801in}}%
\pgfpathlineto{\pgfqpoint{3.320582in}{1.760641in}}%
\pgfpathlineto{\pgfqpoint{3.325091in}{1.702021in}}%
\pgfpathlineto{\pgfqpoint{3.329600in}{1.684631in}}%
\pgfpathlineto{\pgfqpoint{3.334109in}{1.753740in}}%
\pgfpathlineto{\pgfqpoint{3.343127in}{1.702935in}}%
\pgfpathlineto{\pgfqpoint{3.347636in}{1.737902in}}%
\pgfpathlineto{\pgfqpoint{3.352145in}{1.732773in}}%
\pgfpathlineto{\pgfqpoint{3.356655in}{1.768159in}}%
\pgfpathlineto{\pgfqpoint{3.361164in}{1.744473in}}%
\pgfpathlineto{\pgfqpoint{3.365673in}{1.757758in}}%
\pgfpathlineto{\pgfqpoint{3.370182in}{1.781598in}}%
\pgfpathlineto{\pgfqpoint{3.374691in}{1.687889in}}%
\pgfpathlineto{\pgfqpoint{3.379200in}{1.784800in}}%
\pgfpathlineto{\pgfqpoint{3.383709in}{1.752100in}}%
\pgfpathlineto{\pgfqpoint{3.388218in}{1.732949in}}%
\pgfpathlineto{\pgfqpoint{3.401745in}{1.806934in}}%
\pgfpathlineto{\pgfqpoint{3.406255in}{1.812867in}}%
\pgfpathlineto{\pgfqpoint{3.410764in}{1.754511in}}%
\pgfpathlineto{\pgfqpoint{3.415273in}{1.753058in}}%
\pgfpathlineto{\pgfqpoint{3.419782in}{1.803941in}}%
\pgfpathlineto{\pgfqpoint{3.424291in}{1.797062in}}%
\pgfpathlineto{\pgfqpoint{3.428800in}{1.820428in}}%
\pgfpathlineto{\pgfqpoint{3.433309in}{1.825810in}}%
\pgfpathlineto{\pgfqpoint{3.437818in}{1.834439in}}%
\pgfpathlineto{\pgfqpoint{3.442327in}{1.825139in}}%
\pgfpathlineto{\pgfqpoint{3.446836in}{1.773937in}}%
\pgfpathlineto{\pgfqpoint{3.451345in}{1.785846in}}%
\pgfpathlineto{\pgfqpoint{3.455855in}{1.866567in}}%
\pgfpathlineto{\pgfqpoint{3.464873in}{1.803291in}}%
\pgfpathlineto{\pgfqpoint{3.469382in}{1.829641in}}%
\pgfpathlineto{\pgfqpoint{3.473891in}{1.866809in}}%
\pgfpathlineto{\pgfqpoint{3.478400in}{1.840603in}}%
\pgfpathlineto{\pgfqpoint{3.482909in}{1.772363in}}%
\pgfpathlineto{\pgfqpoint{3.487418in}{1.841131in}}%
\pgfpathlineto{\pgfqpoint{3.491927in}{1.886874in}}%
\pgfpathlineto{\pgfqpoint{3.496436in}{1.878168in}}%
\pgfpathlineto{\pgfqpoint{3.500945in}{1.785538in}}%
\pgfpathlineto{\pgfqpoint{3.505455in}{1.845875in}}%
\pgfpathlineto{\pgfqpoint{3.509964in}{1.815475in}}%
\pgfpathlineto{\pgfqpoint{3.514473in}{1.866534in}}%
\pgfpathlineto{\pgfqpoint{3.518982in}{1.892818in}}%
\pgfpathlineto{\pgfqpoint{3.523491in}{1.877034in}}%
\pgfpathlineto{\pgfqpoint{3.528000in}{1.867712in}}%
\pgfpathlineto{\pgfqpoint{3.532509in}{1.925276in}}%
\pgfpathlineto{\pgfqpoint{3.537018in}{1.889890in}}%
\pgfpathlineto{\pgfqpoint{3.541527in}{1.928346in}}%
\pgfpathlineto{\pgfqpoint{3.546036in}{1.925232in}}%
\pgfpathlineto{\pgfqpoint{3.550545in}{1.875097in}}%
\pgfpathlineto{\pgfqpoint{3.555055in}{1.872037in}}%
\pgfpathlineto{\pgfqpoint{3.559564in}{1.940442in}}%
\pgfpathlineto{\pgfqpoint{3.564073in}{1.880622in}}%
\pgfpathlineto{\pgfqpoint{3.568582in}{1.909459in}}%
\pgfpathlineto{\pgfqpoint{3.573091in}{1.890165in}}%
\pgfpathlineto{\pgfqpoint{3.577600in}{1.924208in}}%
\pgfpathlineto{\pgfqpoint{3.582109in}{1.904297in}}%
\pgfpathlineto{\pgfqpoint{3.586618in}{1.903549in}}%
\pgfpathlineto{\pgfqpoint{3.591127in}{1.940938in}}%
\pgfpathlineto{\pgfqpoint{3.595636in}{1.909756in}}%
\pgfpathlineto{\pgfqpoint{3.604655in}{1.950601in}}%
\pgfpathlineto{\pgfqpoint{3.609164in}{1.892179in}}%
\pgfpathlineto{\pgfqpoint{3.613673in}{1.943029in}}%
\pgfpathlineto{\pgfqpoint{3.618182in}{1.944614in}}%
\pgfpathlineto{\pgfqpoint{3.622691in}{1.961465in}}%
\pgfpathlineto{\pgfqpoint{3.627200in}{1.933409in}}%
\pgfpathlineto{\pgfqpoint{3.631709in}{1.989388in}}%
\pgfpathlineto{\pgfqpoint{3.636218in}{1.936876in}}%
\pgfpathlineto{\pgfqpoint{3.640727in}{1.937757in}}%
\pgfpathlineto{\pgfqpoint{3.645236in}{1.971503in}}%
\pgfpathlineto{\pgfqpoint{3.649745in}{1.965647in}}%
\pgfpathlineto{\pgfqpoint{3.654255in}{1.952417in}}%
\pgfpathlineto{\pgfqpoint{3.658764in}{2.008308in}}%
\pgfpathlineto{\pgfqpoint{3.663273in}{1.955686in}}%
\pgfpathlineto{\pgfqpoint{3.672291in}{2.009585in}}%
\pgfpathlineto{\pgfqpoint{3.681309in}{1.960166in}}%
\pgfpathlineto{\pgfqpoint{3.685818in}{2.064804in}}%
\pgfpathlineto{\pgfqpoint{3.690327in}{2.013987in}}%
\pgfpathlineto{\pgfqpoint{3.694836in}{1.995342in}}%
\pgfpathlineto{\pgfqpoint{3.699345in}{2.063252in}}%
\pgfpathlineto{\pgfqpoint{3.703855in}{2.001484in}}%
\pgfpathlineto{\pgfqpoint{3.708364in}{2.008924in}}%
\pgfpathlineto{\pgfqpoint{3.712873in}{2.029936in}}%
\pgfpathlineto{\pgfqpoint{3.717382in}{1.987352in}}%
\pgfpathlineto{\pgfqpoint{3.721891in}{2.081380in}}%
\pgfpathlineto{\pgfqpoint{3.726400in}{2.122555in}}%
\pgfpathlineto{\pgfqpoint{3.730909in}{2.126275in}}%
\pgfpathlineto{\pgfqpoint{3.735418in}{2.051706in}}%
\pgfpathlineto{\pgfqpoint{3.739927in}{2.030057in}}%
\pgfpathlineto{\pgfqpoint{3.744436in}{2.051068in}}%
\pgfpathlineto{\pgfqpoint{3.748945in}{2.018665in}}%
\pgfpathlineto{\pgfqpoint{3.753455in}{2.032962in}}%
\pgfpathlineto{\pgfqpoint{3.757964in}{2.075590in}}%
\pgfpathlineto{\pgfqpoint{3.762473in}{2.075921in}}%
\pgfpathlineto{\pgfqpoint{3.766982in}{2.050804in}}%
\pgfpathlineto{\pgfqpoint{3.771491in}{2.099606in}}%
\pgfpathlineto{\pgfqpoint{3.776000in}{2.055922in}}%
\pgfpathlineto{\pgfqpoint{3.780509in}{2.070792in}}%
\pgfpathlineto{\pgfqpoint{3.785018in}{2.103723in}}%
\pgfpathlineto{\pgfqpoint{3.789527in}{2.102963in}}%
\pgfpathlineto{\pgfqpoint{3.798545in}{2.142488in}}%
\pgfpathlineto{\pgfqpoint{3.803055in}{2.140627in}}%
\pgfpathlineto{\pgfqpoint{3.807564in}{2.110283in}}%
\pgfpathlineto{\pgfqpoint{3.816582in}{2.152041in}}%
\pgfpathlineto{\pgfqpoint{3.821091in}{2.113959in}}%
\pgfpathlineto{\pgfqpoint{3.830109in}{2.065883in}}%
\pgfpathlineto{\pgfqpoint{3.839127in}{2.107025in}}%
\pgfpathlineto{\pgfqpoint{3.843636in}{2.146274in}}%
\pgfpathlineto{\pgfqpoint{3.848145in}{2.169861in}}%
\pgfpathlineto{\pgfqpoint{3.852655in}{2.184697in}}%
\pgfpathlineto{\pgfqpoint{3.857164in}{2.159415in}}%
\pgfpathlineto{\pgfqpoint{3.861673in}{2.170444in}}%
\pgfpathlineto{\pgfqpoint{3.866182in}{2.121102in}}%
\pgfpathlineto{\pgfqpoint{3.870691in}{2.169398in}}%
\pgfpathlineto{\pgfqpoint{3.875200in}{2.171787in}}%
\pgfpathlineto{\pgfqpoint{3.879709in}{2.189298in}}%
\pgfpathlineto{\pgfqpoint{3.884218in}{2.165766in}}%
\pgfpathlineto{\pgfqpoint{3.888727in}{2.129280in}}%
\pgfpathlineto{\pgfqpoint{3.893236in}{2.221745in}}%
\pgfpathlineto{\pgfqpoint{3.897745in}{2.183266in}}%
\pgfpathlineto{\pgfqpoint{3.902255in}{2.159184in}}%
\pgfpathlineto{\pgfqpoint{3.906764in}{2.165392in}}%
\pgfpathlineto{\pgfqpoint{3.911273in}{2.250945in}}%
\pgfpathlineto{\pgfqpoint{3.915782in}{2.151535in}}%
\pgfpathlineto{\pgfqpoint{3.920291in}{2.251286in}}%
\pgfpathlineto{\pgfqpoint{3.924800in}{2.158128in}}%
\pgfpathlineto{\pgfqpoint{3.929309in}{2.155717in}}%
\pgfpathlineto{\pgfqpoint{3.933818in}{2.203067in}}%
\pgfpathlineto{\pgfqpoint{3.938327in}{2.287409in}}%
\pgfpathlineto{\pgfqpoint{3.942836in}{2.290942in}}%
\pgfpathlineto{\pgfqpoint{3.947345in}{2.325635in}}%
\pgfpathlineto{\pgfqpoint{3.951855in}{2.258374in}}%
\pgfpathlineto{\pgfqpoint{3.956364in}{2.239036in}}%
\pgfpathlineto{\pgfqpoint{3.960873in}{2.230737in}}%
\pgfpathlineto{\pgfqpoint{3.965382in}{2.216319in}}%
\pgfpathlineto{\pgfqpoint{3.969891in}{2.282522in}}%
\pgfpathlineto{\pgfqpoint{3.974400in}{2.292362in}}%
\pgfpathlineto{\pgfqpoint{3.978909in}{2.262337in}}%
\pgfpathlineto{\pgfqpoint{3.983418in}{2.198004in}}%
\pgfpathlineto{\pgfqpoint{3.987927in}{2.250890in}}%
\pgfpathlineto{\pgfqpoint{3.992436in}{2.224695in}}%
\pgfpathlineto{\pgfqpoint{3.996945in}{2.316719in}}%
\pgfpathlineto{\pgfqpoint{4.001455in}{2.255854in}}%
\pgfpathlineto{\pgfqpoint{4.005964in}{2.235305in}}%
\pgfpathlineto{\pgfqpoint{4.010473in}{2.296104in}}%
\pgfpathlineto{\pgfqpoint{4.014982in}{2.259189in}}%
\pgfpathlineto{\pgfqpoint{4.019491in}{2.335716in}}%
\pgfpathlineto{\pgfqpoint{4.024000in}{2.259475in}}%
\pgfpathlineto{\pgfqpoint{4.028509in}{2.275676in}}%
\pgfpathlineto{\pgfqpoint{4.033018in}{2.244704in}}%
\pgfpathlineto{\pgfqpoint{4.037527in}{2.337709in}}%
\pgfpathlineto{\pgfqpoint{4.042036in}{2.346888in}}%
\pgfpathlineto{\pgfqpoint{4.046545in}{2.299395in}}%
\pgfpathlineto{\pgfqpoint{4.051055in}{2.376429in}}%
\pgfpathlineto{\pgfqpoint{4.055564in}{2.382670in}}%
\pgfpathlineto{\pgfqpoint{4.060073in}{2.340317in}}%
\pgfpathlineto{\pgfqpoint{4.064582in}{2.284856in}}%
\pgfpathlineto{\pgfqpoint{4.069091in}{2.353173in}}%
\pgfpathlineto{\pgfqpoint{4.073600in}{2.328992in}}%
\pgfpathlineto{\pgfqpoint{4.078109in}{2.335397in}}%
\pgfpathlineto{\pgfqpoint{4.082618in}{2.329091in}}%
\pgfpathlineto{\pgfqpoint{4.087127in}{2.340482in}}%
\pgfpathlineto{\pgfqpoint{4.091636in}{2.392455in}}%
\pgfpathlineto{\pgfqpoint{4.096145in}{2.344081in}}%
\pgfpathlineto{\pgfqpoint{4.105164in}{2.456600in}}%
\pgfpathlineto{\pgfqpoint{4.109673in}{2.339261in}}%
\pgfpathlineto{\pgfqpoint{4.118691in}{2.437438in}}%
\pgfpathlineto{\pgfqpoint{4.123200in}{2.414644in}}%
\pgfpathlineto{\pgfqpoint{4.127709in}{2.458031in}}%
\pgfpathlineto{\pgfqpoint{4.132218in}{2.326801in}}%
\pgfpathlineto{\pgfqpoint{4.136727in}{2.392565in}}%
\pgfpathlineto{\pgfqpoint{4.145745in}{2.477887in}}%
\pgfpathlineto{\pgfqpoint{4.150255in}{2.455610in}}%
\pgfpathlineto{\pgfqpoint{4.154764in}{2.402063in}}%
\pgfpathlineto{\pgfqpoint{4.159273in}{2.466792in}}%
\pgfpathlineto{\pgfqpoint{4.163782in}{2.465835in}}%
\pgfpathlineto{\pgfqpoint{4.168291in}{2.407489in}}%
\pgfpathlineto{\pgfqpoint{4.172800in}{2.437471in}}%
\pgfpathlineto{\pgfqpoint{4.177309in}{2.430955in}}%
\pgfpathlineto{\pgfqpoint{4.181818in}{2.442611in}}%
\pgfpathlineto{\pgfqpoint{4.195345in}{2.564837in}}%
\pgfpathlineto{\pgfqpoint{4.199855in}{2.455345in}}%
\pgfpathlineto{\pgfqpoint{4.204364in}{2.527460in}}%
\pgfpathlineto{\pgfqpoint{4.208873in}{2.403494in}}%
\pgfpathlineto{\pgfqpoint{4.213382in}{2.501088in}}%
\pgfpathlineto{\pgfqpoint{4.217891in}{2.510080in}}%
\pgfpathlineto{\pgfqpoint{4.226909in}{2.466605in}}%
\pgfpathlineto{\pgfqpoint{4.231418in}{2.507967in}}%
\pgfpathlineto{\pgfqpoint{4.235927in}{2.466627in}}%
\pgfpathlineto{\pgfqpoint{4.240436in}{2.518214in}}%
\pgfpathlineto{\pgfqpoint{4.244945in}{2.493626in}}%
\pgfpathlineto{\pgfqpoint{4.249455in}{2.504808in}}%
\pgfpathlineto{\pgfqpoint{4.253964in}{2.552378in}}%
\pgfpathlineto{\pgfqpoint{4.258473in}{2.560160in}}%
\pgfpathlineto{\pgfqpoint{4.262982in}{2.525665in}}%
\pgfpathlineto{\pgfqpoint{4.267491in}{2.541482in}}%
\pgfpathlineto{\pgfqpoint{4.272000in}{2.611868in}}%
\pgfpathlineto{\pgfqpoint{4.276509in}{2.545367in}}%
\pgfpathlineto{\pgfqpoint{4.281018in}{2.447399in}}%
\pgfpathlineto{\pgfqpoint{4.285527in}{2.582382in}}%
\pgfpathlineto{\pgfqpoint{4.290036in}{2.614025in}}%
\pgfpathlineto{\pgfqpoint{4.294545in}{2.547546in}}%
\pgfpathlineto{\pgfqpoint{4.299055in}{2.625208in}}%
\pgfpathlineto{\pgfqpoint{4.303564in}{2.637821in}}%
\pgfpathlineto{\pgfqpoint{4.308073in}{2.624701in}}%
\pgfpathlineto{\pgfqpoint{4.312582in}{2.547931in}}%
\pgfpathlineto{\pgfqpoint{4.321600in}{2.599299in}}%
\pgfpathlineto{\pgfqpoint{4.326109in}{2.682441in}}%
\pgfpathlineto{\pgfqpoint{4.330618in}{2.563429in}}%
\pgfpathlineto{\pgfqpoint{4.335127in}{2.638019in}}%
\pgfpathlineto{\pgfqpoint{4.339636in}{2.592001in}}%
\pgfpathlineto{\pgfqpoint{4.344145in}{2.511126in}}%
\pgfpathlineto{\pgfqpoint{4.348655in}{2.613849in}}%
\pgfpathlineto{\pgfqpoint{4.353164in}{2.674120in}}%
\pgfpathlineto{\pgfqpoint{4.357673in}{2.566697in}}%
\pgfpathlineto{\pgfqpoint{4.362182in}{2.557903in}}%
\pgfpathlineto{\pgfqpoint{4.366691in}{2.630810in}}%
\pgfpathlineto{\pgfqpoint{4.371200in}{2.597273in}}%
\pgfpathlineto{\pgfqpoint{4.375709in}{2.647892in}}%
\pgfpathlineto{\pgfqpoint{4.380218in}{2.676421in}}%
\pgfpathlineto{\pgfqpoint{4.384727in}{2.651161in}}%
\pgfpathlineto{\pgfqpoint{4.389236in}{2.680042in}}%
\pgfpathlineto{\pgfqpoint{4.393745in}{2.675617in}}%
\pgfpathlineto{\pgfqpoint{4.398255in}{2.595644in}}%
\pgfpathlineto{\pgfqpoint{4.402764in}{2.612209in}}%
\pgfpathlineto{\pgfqpoint{4.407273in}{2.696298in}}%
\pgfpathlineto{\pgfqpoint{4.416291in}{2.728426in}}%
\pgfpathlineto{\pgfqpoint{4.420800in}{2.707129in}}%
\pgfpathlineto{\pgfqpoint{4.425309in}{2.649697in}}%
\pgfpathlineto{\pgfqpoint{4.429818in}{2.708471in}}%
\pgfpathlineto{\pgfqpoint{4.434327in}{2.697597in}}%
\pgfpathlineto{\pgfqpoint{4.438836in}{2.691609in}}%
\pgfpathlineto{\pgfqpoint{4.443345in}{2.705235in}}%
\pgfpathlineto{\pgfqpoint{4.447855in}{2.699380in}}%
\pgfpathlineto{\pgfqpoint{4.452364in}{2.755304in}}%
\pgfpathlineto{\pgfqpoint{4.456873in}{2.709418in}}%
\pgfpathlineto{\pgfqpoint{4.461382in}{2.772364in}}%
\pgfpathlineto{\pgfqpoint{4.465891in}{2.679910in}}%
\pgfpathlineto{\pgfqpoint{4.470400in}{2.691004in}}%
\pgfpathlineto{\pgfqpoint{4.474909in}{2.665590in}}%
\pgfpathlineto{\pgfqpoint{4.479418in}{2.805064in}}%
\pgfpathlineto{\pgfqpoint{4.483927in}{2.713160in}}%
\pgfpathlineto{\pgfqpoint{4.488436in}{2.769612in}}%
\pgfpathlineto{\pgfqpoint{4.492945in}{2.903274in}}%
\pgfpathlineto{\pgfqpoint{4.497455in}{2.831523in}}%
\pgfpathlineto{\pgfqpoint{4.501964in}{2.828100in}}%
\pgfpathlineto{\pgfqpoint{4.506473in}{2.827671in}}%
\pgfpathlineto{\pgfqpoint{4.510982in}{2.837929in}}%
\pgfpathlineto{\pgfqpoint{4.515491in}{2.747346in}}%
\pgfpathlineto{\pgfqpoint{4.520000in}{2.785406in}}%
\pgfpathlineto{\pgfqpoint{4.524509in}{2.780475in}}%
\pgfpathlineto{\pgfqpoint{4.529018in}{2.822366in}}%
\pgfpathlineto{\pgfqpoint{4.533527in}{2.826218in}}%
\pgfpathlineto{\pgfqpoint{4.538036in}{2.804822in}}%
\pgfpathlineto{\pgfqpoint{4.542545in}{2.865368in}}%
\pgfpathlineto{\pgfqpoint{4.547055in}{2.877640in}}%
\pgfpathlineto{\pgfqpoint{4.551564in}{2.823478in}}%
\pgfpathlineto{\pgfqpoint{4.556073in}{2.729516in}}%
\pgfpathlineto{\pgfqpoint{4.565091in}{2.863035in}}%
\pgfpathlineto{\pgfqpoint{4.569600in}{2.834143in}}%
\pgfpathlineto{\pgfqpoint{4.574109in}{2.848990in}}%
\pgfpathlineto{\pgfqpoint{4.578618in}{2.826174in}}%
\pgfpathlineto{\pgfqpoint{4.583127in}{2.838237in}}%
\pgfpathlineto{\pgfqpoint{4.587636in}{2.953618in}}%
\pgfpathlineto{\pgfqpoint{4.592145in}{2.847582in}}%
\pgfpathlineto{\pgfqpoint{4.596655in}{2.911760in}}%
\pgfpathlineto{\pgfqpoint{4.601164in}{2.826009in}}%
\pgfpathlineto{\pgfqpoint{4.605673in}{2.870123in}}%
\pgfpathlineto{\pgfqpoint{4.610182in}{2.876088in}}%
\pgfpathlineto{\pgfqpoint{4.614691in}{2.870662in}}%
\pgfpathlineto{\pgfqpoint{4.619200in}{2.881371in}}%
\pgfpathlineto{\pgfqpoint{4.623709in}{2.929293in}}%
\pgfpathlineto{\pgfqpoint{4.628218in}{2.952319in}}%
\pgfpathlineto{\pgfqpoint{4.632727in}{2.940003in}}%
\pgfpathlineto{\pgfqpoint{4.637236in}{2.871069in}}%
\pgfpathlineto{\pgfqpoint{4.641745in}{2.919916in}}%
\pgfpathlineto{\pgfqpoint{4.646255in}{2.844445in}}%
\pgfpathlineto{\pgfqpoint{4.650764in}{2.879104in}}%
\pgfpathlineto{\pgfqpoint{4.655273in}{2.878257in}}%
\pgfpathlineto{\pgfqpoint{4.659782in}{2.964613in}}%
\pgfpathlineto{\pgfqpoint{4.664291in}{2.933113in}}%
\pgfpathlineto{\pgfqpoint{4.668800in}{2.993274in}}%
\pgfpathlineto{\pgfqpoint{4.673309in}{2.920059in}}%
\pgfpathlineto{\pgfqpoint{4.677818in}{3.030288in}}%
\pgfpathlineto{\pgfqpoint{4.682327in}{2.987572in}}%
\pgfpathlineto{\pgfqpoint{4.686836in}{2.988530in}}%
\pgfpathlineto{\pgfqpoint{4.691345in}{2.962214in}}%
\pgfpathlineto{\pgfqpoint{4.695855in}{2.993230in}}%
\pgfpathlineto{\pgfqpoint{4.704873in}{2.950118in}}%
\pgfpathlineto{\pgfqpoint{4.709382in}{2.999437in}}%
\pgfpathlineto{\pgfqpoint{4.713891in}{3.008419in}}%
\pgfpathlineto{\pgfqpoint{4.718400in}{3.035814in}}%
\pgfpathlineto{\pgfqpoint{4.722909in}{2.994143in}}%
\pgfpathlineto{\pgfqpoint{4.727418in}{2.998480in}}%
\pgfpathlineto{\pgfqpoint{4.731927in}{2.941279in}}%
\pgfpathlineto{\pgfqpoint{4.736436in}{3.113904in}}%
\pgfpathlineto{\pgfqpoint{4.740945in}{3.072047in}}%
\pgfpathlineto{\pgfqpoint{4.745455in}{3.002090in}}%
\pgfpathlineto{\pgfqpoint{4.749964in}{3.034174in}}%
\pgfpathlineto{\pgfqpoint{4.754473in}{3.023773in}}%
\pgfpathlineto{\pgfqpoint{4.758982in}{3.042572in}}%
\pgfpathlineto{\pgfqpoint{4.763491in}{3.196871in}}%
\pgfpathlineto{\pgfqpoint{4.772509in}{3.074986in}}%
\pgfpathlineto{\pgfqpoint{4.777018in}{3.036859in}}%
\pgfpathlineto{\pgfqpoint{4.781527in}{3.128642in}}%
\pgfpathlineto{\pgfqpoint{4.786036in}{3.002222in}}%
\pgfpathlineto{\pgfqpoint{4.790545in}{2.996014in}}%
\pgfpathlineto{\pgfqpoint{4.795055in}{3.164028in}}%
\pgfpathlineto{\pgfqpoint{4.799564in}{3.009222in}}%
\pgfpathlineto{\pgfqpoint{4.804073in}{3.153759in}}%
\pgfpathlineto{\pgfqpoint{4.808582in}{3.123711in}}%
\pgfpathlineto{\pgfqpoint{4.813091in}{3.146010in}}%
\pgfpathlineto{\pgfqpoint{4.817600in}{3.130722in}}%
\pgfpathlineto{\pgfqpoint{4.822109in}{3.149070in}}%
\pgfpathlineto{\pgfqpoint{4.826618in}{3.232774in}}%
\pgfpathlineto{\pgfqpoint{4.831127in}{3.247511in}}%
\pgfpathlineto{\pgfqpoint{4.835636in}{3.155399in}}%
\pgfpathlineto{\pgfqpoint{4.840145in}{3.138713in}}%
\pgfpathlineto{\pgfqpoint{4.844655in}{3.128587in}}%
\pgfpathlineto{\pgfqpoint{4.849164in}{3.199942in}}%
\pgfpathlineto{\pgfqpoint{4.853673in}{3.171105in}}%
\pgfpathlineto{\pgfqpoint{4.858182in}{3.132637in}}%
\pgfpathlineto{\pgfqpoint{4.862691in}{3.186107in}}%
\pgfpathlineto{\pgfqpoint{4.867200in}{3.253378in}}%
\pgfpathlineto{\pgfqpoint{4.871709in}{3.251364in}}%
\pgfpathlineto{\pgfqpoint{4.876218in}{3.179492in}}%
\pgfpathlineto{\pgfqpoint{4.880727in}{3.191412in}}%
\pgfpathlineto{\pgfqpoint{4.885236in}{3.318404in}}%
\pgfpathlineto{\pgfqpoint{4.894255in}{3.134442in}}%
\pgfpathlineto{\pgfqpoint{4.898764in}{3.281763in}}%
\pgfpathlineto{\pgfqpoint{4.903273in}{3.266894in}}%
\pgfpathlineto{\pgfqpoint{4.907782in}{3.291625in}}%
\pgfpathlineto{\pgfqpoint{4.912291in}{3.258001in}}%
\pgfpathlineto{\pgfqpoint{4.916800in}{3.204872in}}%
\pgfpathlineto{\pgfqpoint{4.921309in}{3.308234in}}%
\pgfpathlineto{\pgfqpoint{4.930327in}{3.216352in}}%
\pgfpathlineto{\pgfqpoint{4.934836in}{3.292341in}}%
\pgfpathlineto{\pgfqpoint{4.939345in}{3.253609in}}%
\pgfpathlineto{\pgfqpoint{4.943855in}{3.268446in}}%
\pgfpathlineto{\pgfqpoint{4.948364in}{3.348672in}}%
\pgfpathlineto{\pgfqpoint{4.952873in}{3.250219in}}%
\pgfpathlineto{\pgfqpoint{4.957382in}{3.229681in}}%
\pgfpathlineto{\pgfqpoint{4.961891in}{3.337687in}}%
\pgfpathlineto{\pgfqpoint{4.966400in}{3.260466in}}%
\pgfpathlineto{\pgfqpoint{4.970909in}{3.377453in}}%
\pgfpathlineto{\pgfqpoint{4.975418in}{3.391057in}}%
\pgfpathlineto{\pgfqpoint{4.979927in}{3.275875in}}%
\pgfpathlineto{\pgfqpoint{4.984436in}{3.381163in}}%
\pgfpathlineto{\pgfqpoint{4.988945in}{3.293100in}}%
\pgfpathlineto{\pgfqpoint{4.993455in}{3.389571in}}%
\pgfpathlineto{\pgfqpoint{4.997964in}{3.324678in}}%
\pgfpathlineto{\pgfqpoint{5.002473in}{3.357862in}}%
\pgfpathlineto{\pgfqpoint{5.006982in}{3.401800in}}%
\pgfpathlineto{\pgfqpoint{5.011491in}{3.201119in}}%
\pgfpathlineto{\pgfqpoint{5.016000in}{3.468510in}}%
\pgfpathlineto{\pgfqpoint{5.020509in}{3.401921in}}%
\pgfpathlineto{\pgfqpoint{5.025018in}{3.294960in}}%
\pgfpathlineto{\pgfqpoint{5.029527in}{3.423647in}}%
\pgfpathlineto{\pgfqpoint{5.034036in}{3.405696in}}%
\pgfpathlineto{\pgfqpoint{5.038545in}{3.382153in}}%
\pgfpathlineto{\pgfqpoint{5.043055in}{3.405124in}}%
\pgfpathlineto{\pgfqpoint{5.047564in}{3.419740in}}%
\pgfpathlineto{\pgfqpoint{5.052073in}{3.385257in}}%
\pgfpathlineto{\pgfqpoint{5.056582in}{3.316115in}}%
\pgfpathlineto{\pgfqpoint{5.061091in}{3.307254in}}%
\pgfpathlineto{\pgfqpoint{5.065600in}{3.462269in}}%
\pgfpathlineto{\pgfqpoint{5.070109in}{3.379170in}}%
\pgfpathlineto{\pgfqpoint{5.074618in}{3.436140in}}%
\pgfpathlineto{\pgfqpoint{5.079127in}{3.458461in}}%
\pgfpathlineto{\pgfqpoint{5.083636in}{3.460178in}}%
\pgfpathlineto{\pgfqpoint{5.088145in}{3.467497in}}%
\pgfpathlineto{\pgfqpoint{5.092655in}{3.525699in}}%
\pgfpathlineto{\pgfqpoint{5.097164in}{3.543034in}}%
\pgfpathlineto{\pgfqpoint{5.101673in}{3.483467in}}%
\pgfpathlineto{\pgfqpoint{5.106182in}{3.390287in}}%
\pgfpathlineto{\pgfqpoint{5.110691in}{3.422272in}}%
\pgfpathlineto{\pgfqpoint{5.115200in}{3.524191in}}%
\pgfpathlineto{\pgfqpoint{5.119709in}{3.382384in}}%
\pgfpathlineto{\pgfqpoint{5.128727in}{3.539061in}}%
\pgfpathlineto{\pgfqpoint{5.133236in}{3.502530in}}%
\pgfpathlineto{\pgfqpoint{5.137745in}{3.456656in}}%
\pgfpathlineto{\pgfqpoint{5.142255in}{3.527042in}}%
\pgfpathlineto{\pgfqpoint{5.146764in}{3.509520in}}%
\pgfpathlineto{\pgfqpoint{5.151273in}{3.512579in}}%
\pgfpathlineto{\pgfqpoint{5.155782in}{3.488949in}}%
\pgfpathlineto{\pgfqpoint{5.164800in}{3.639230in}}%
\pgfpathlineto{\pgfqpoint{5.169309in}{3.555207in}}%
\pgfpathlineto{\pgfqpoint{5.173818in}{3.541141in}}%
\pgfpathlineto{\pgfqpoint{5.178327in}{3.566896in}}%
\pgfpathlineto{\pgfqpoint{5.182836in}{3.571695in}}%
\pgfpathlineto{\pgfqpoint{5.187345in}{3.543871in}}%
\pgfpathlineto{\pgfqpoint{5.191855in}{3.596437in}}%
\pgfpathlineto{\pgfqpoint{5.196364in}{3.539600in}}%
\pgfpathlineto{\pgfqpoint{5.200873in}{3.612122in}}%
\pgfpathlineto{\pgfqpoint{5.205382in}{3.756570in}}%
\pgfpathlineto{\pgfqpoint{5.209891in}{3.657622in}}%
\pgfpathlineto{\pgfqpoint{5.214400in}{3.598198in}}%
\pgfpathlineto{\pgfqpoint{5.218909in}{3.620674in}}%
\pgfpathlineto{\pgfqpoint{5.223418in}{3.611934in}}%
\pgfpathlineto{\pgfqpoint{5.227927in}{3.521748in}}%
\pgfpathlineto{\pgfqpoint{5.232436in}{3.696717in}}%
\pgfpathlineto{\pgfqpoint{5.236945in}{3.694846in}}%
\pgfpathlineto{\pgfqpoint{5.241455in}{3.591473in}}%
\pgfpathlineto{\pgfqpoint{5.245964in}{3.596691in}}%
\pgfpathlineto{\pgfqpoint{5.250473in}{3.776205in}}%
\pgfpathlineto{\pgfqpoint{5.254982in}{3.606222in}}%
\pgfpathlineto{\pgfqpoint{5.259491in}{3.586289in}}%
\pgfpathlineto{\pgfqpoint{5.264000in}{3.737100in}}%
\pgfpathlineto{\pgfqpoint{5.268509in}{3.655091in}}%
\pgfpathlineto{\pgfqpoint{5.273018in}{3.636864in}}%
\pgfpathlineto{\pgfqpoint{5.277527in}{3.749614in}}%
\pgfpathlineto{\pgfqpoint{5.282036in}{3.677489in}}%
\pgfpathlineto{\pgfqpoint{5.286545in}{3.721647in}}%
\pgfpathlineto{\pgfqpoint{5.291055in}{3.718113in}}%
\pgfpathlineto{\pgfqpoint{5.295564in}{3.700602in}}%
\pgfpathlineto{\pgfqpoint{5.300073in}{3.731332in}}%
\pgfpathlineto{\pgfqpoint{5.304582in}{3.745509in}}%
\pgfpathlineto{\pgfqpoint{5.309091in}{3.741194in}}%
\pgfpathlineto{\pgfqpoint{5.313600in}{3.772771in}}%
\pgfpathlineto{\pgfqpoint{5.318109in}{3.703222in}}%
\pgfpathlineto{\pgfqpoint{5.322618in}{3.773575in}}%
\pgfpathlineto{\pgfqpoint{5.327127in}{3.700932in}}%
\pgfpathlineto{\pgfqpoint{5.336145in}{3.846250in}}%
\pgfpathlineto{\pgfqpoint{5.340655in}{3.679943in}}%
\pgfpathlineto{\pgfqpoint{5.345164in}{3.756130in}}%
\pgfpathlineto{\pgfqpoint{5.349673in}{3.759916in}}%
\pgfpathlineto{\pgfqpoint{5.354182in}{3.871180in}}%
\pgfpathlineto{\pgfqpoint{5.358691in}{3.936140in}}%
\pgfpathlineto{\pgfqpoint{5.363200in}{3.834749in}}%
\pgfpathlineto{\pgfqpoint{5.367709in}{3.829807in}}%
\pgfpathlineto{\pgfqpoint{5.372218in}{3.845832in}}%
\pgfpathlineto{\pgfqpoint{5.376727in}{3.830225in}}%
\pgfpathlineto{\pgfqpoint{5.381236in}{3.753367in}}%
\pgfpathlineto{\pgfqpoint{5.385745in}{3.924649in}}%
\pgfpathlineto{\pgfqpoint{5.390255in}{3.696431in}}%
\pgfpathlineto{\pgfqpoint{5.394764in}{3.848815in}}%
\pgfpathlineto{\pgfqpoint{5.399273in}{3.811492in}}%
\pgfpathlineto{\pgfqpoint{5.403782in}{3.877608in}}%
\pgfpathlineto{\pgfqpoint{5.408291in}{3.849112in}}%
\pgfpathlineto{\pgfqpoint{5.412800in}{3.834870in}}%
\pgfpathlineto{\pgfqpoint{5.417309in}{3.928557in}}%
\pgfpathlineto{\pgfqpoint{5.421818in}{3.831326in}}%
\pgfpathlineto{\pgfqpoint{5.426327in}{3.930196in}}%
\pgfpathlineto{\pgfqpoint{5.430836in}{3.926212in}}%
\pgfpathlineto{\pgfqpoint{5.435345in}{3.972978in}}%
\pgfpathlineto{\pgfqpoint{5.439855in}{3.851269in}}%
\pgfpathlineto{\pgfqpoint{5.444364in}{3.696057in}}%
\pgfpathlineto{\pgfqpoint{5.448873in}{3.957173in}}%
\pgfpathlineto{\pgfqpoint{5.453382in}{3.884619in}}%
\pgfpathlineto{\pgfqpoint{5.457891in}{3.890210in}}%
\pgfpathlineto{\pgfqpoint{5.466909in}{4.033778in}}%
\pgfpathlineto{\pgfqpoint{5.475927in}{3.860790in}}%
\pgfpathlineto{\pgfqpoint{5.480436in}{3.994859in}}%
\pgfpathlineto{\pgfqpoint{5.484945in}{4.017180in}}%
\pgfpathlineto{\pgfqpoint{5.489455in}{4.029463in}}%
\pgfpathlineto{\pgfqpoint{5.493964in}{3.895779in}}%
\pgfpathlineto{\pgfqpoint{5.498473in}{3.917605in}}%
\pgfpathlineto{\pgfqpoint{5.502982in}{3.978416in}}%
\pgfpathlineto{\pgfqpoint{5.507491in}{3.980595in}}%
\pgfpathlineto{\pgfqpoint{5.512000in}{4.056000in}}%
\pgfpathlineto{\pgfqpoint{5.516509in}{3.928424in}}%
\pgfpathlineto{\pgfqpoint{5.521018in}{3.974288in}}%
\pgfpathlineto{\pgfqpoint{5.525527in}{3.965263in}}%
\pgfpathlineto{\pgfqpoint{5.530036in}{3.870465in}}%
\pgfpathlineto{\pgfqpoint{5.534545in}{4.016993in}}%
\pgfpathlineto{\pgfqpoint{5.534545in}{4.016993in}}%
\pgfusepath{stroke}%
\end{pgfscope}%
\begin{pgfscope}%
\pgfpathrectangle{\pgfqpoint{0.800000in}{0.528000in}}{\pgfqpoint{4.960000in}{3.696000in}}%
\pgfusepath{clip}%
\pgfsetrectcap%
\pgfsetroundjoin%
\pgfsetlinewidth{1.505625pt}%
\definecolor{currentstroke}{rgb}{0.486275,0.988235,0.000000}%
\pgfsetstrokecolor{currentstroke}%
\pgfsetdash{}{0pt}%
\pgfpathmoveto{\pgfqpoint{1.025455in}{0.696429in}}%
\pgfpathlineto{\pgfqpoint{5.534545in}{0.707436in}}%
\pgfpathlineto{\pgfqpoint{5.534545in}{0.707436in}}%
\pgfusepath{stroke}%
\end{pgfscope}%
\begin{pgfscope}%
\pgfpathrectangle{\pgfqpoint{0.800000in}{0.528000in}}{\pgfqpoint{4.960000in}{3.696000in}}%
\pgfusepath{clip}%
\pgfsetrectcap%
\pgfsetroundjoin%
\pgfsetlinewidth{1.505625pt}%
\definecolor{currentstroke}{rgb}{0.000000,1.000000,0.498039}%
\pgfsetstrokecolor{currentstroke}%
\pgfsetdash{}{0pt}%
\pgfpathmoveto{\pgfqpoint{1.025455in}{0.715933in}}%
\pgfpathlineto{\pgfqpoint{1.034473in}{0.723483in}}%
\pgfpathlineto{\pgfqpoint{1.038982in}{0.723164in}}%
\pgfpathlineto{\pgfqpoint{1.043491in}{0.725321in}}%
\pgfpathlineto{\pgfqpoint{1.048000in}{0.729504in}}%
\pgfpathlineto{\pgfqpoint{1.052509in}{0.724529in}}%
\pgfpathlineto{\pgfqpoint{1.057018in}{0.722757in}}%
\pgfpathlineto{\pgfqpoint{1.066036in}{0.729548in}}%
\pgfpathlineto{\pgfqpoint{1.070545in}{0.729911in}}%
\pgfpathlineto{\pgfqpoint{1.075055in}{0.725761in}}%
\pgfpathlineto{\pgfqpoint{1.079564in}{0.727346in}}%
\pgfpathlineto{\pgfqpoint{1.088582in}{0.732993in}}%
\pgfpathlineto{\pgfqpoint{1.097600in}{0.729195in}}%
\pgfpathlineto{\pgfqpoint{1.102109in}{0.731342in}}%
\pgfpathlineto{\pgfqpoint{1.106618in}{0.736240in}}%
\pgfpathlineto{\pgfqpoint{1.115636in}{0.734467in}}%
\pgfpathlineto{\pgfqpoint{1.120145in}{0.739442in}}%
\pgfpathlineto{\pgfqpoint{1.129164in}{0.733004in}}%
\pgfpathlineto{\pgfqpoint{1.133673in}{0.735326in}}%
\pgfpathlineto{\pgfqpoint{1.138182in}{0.741236in}}%
\pgfpathlineto{\pgfqpoint{1.142691in}{0.738595in}}%
\pgfpathlineto{\pgfqpoint{1.147200in}{0.743405in}}%
\pgfpathlineto{\pgfqpoint{1.151709in}{0.736306in}}%
\pgfpathlineto{\pgfqpoint{1.156218in}{0.743163in}}%
\pgfpathlineto{\pgfqpoint{1.160727in}{0.743041in}}%
\pgfpathlineto{\pgfqpoint{1.165236in}{0.741093in}}%
\pgfpathlineto{\pgfqpoint{1.169745in}{0.746960in}}%
\pgfpathlineto{\pgfqpoint{1.174255in}{0.745881in}}%
\pgfpathlineto{\pgfqpoint{1.178764in}{0.739883in}}%
\pgfpathlineto{\pgfqpoint{1.183273in}{0.747411in}}%
\pgfpathlineto{\pgfqpoint{1.187782in}{0.740928in}}%
\pgfpathlineto{\pgfqpoint{1.196800in}{0.748082in}}%
\pgfpathlineto{\pgfqpoint{1.201309in}{0.745034in}}%
\pgfpathlineto{\pgfqpoint{1.205818in}{0.749447in}}%
\pgfpathlineto{\pgfqpoint{1.214836in}{0.749954in}}%
\pgfpathlineto{\pgfqpoint{1.219345in}{0.754543in}}%
\pgfpathlineto{\pgfqpoint{1.228364in}{0.750570in}}%
\pgfpathlineto{\pgfqpoint{1.232873in}{0.757889in}}%
\pgfpathlineto{\pgfqpoint{1.237382in}{0.754939in}}%
\pgfpathlineto{\pgfqpoint{1.246400in}{0.751527in}}%
\pgfpathlineto{\pgfqpoint{1.250909in}{0.757306in}}%
\pgfpathlineto{\pgfqpoint{1.255418in}{0.757933in}}%
\pgfpathlineto{\pgfqpoint{1.259927in}{0.761235in}}%
\pgfpathlineto{\pgfqpoint{1.264436in}{0.753355in}}%
\pgfpathlineto{\pgfqpoint{1.268945in}{0.761037in}}%
\pgfpathlineto{\pgfqpoint{1.273455in}{0.759639in}}%
\pgfpathlineto{\pgfqpoint{1.277964in}{0.759628in}}%
\pgfpathlineto{\pgfqpoint{1.282473in}{0.763866in}}%
\pgfpathlineto{\pgfqpoint{1.291491in}{0.761962in}}%
\pgfpathlineto{\pgfqpoint{1.296000in}{0.776237in}}%
\pgfpathlineto{\pgfqpoint{1.300509in}{0.764834in}}%
\pgfpathlineto{\pgfqpoint{1.305018in}{0.760421in}}%
\pgfpathlineto{\pgfqpoint{1.318545in}{0.773573in}}%
\pgfpathlineto{\pgfqpoint{1.323055in}{0.773738in}}%
\pgfpathlineto{\pgfqpoint{1.327564in}{0.762325in}}%
\pgfpathlineto{\pgfqpoint{1.332073in}{0.777448in}}%
\pgfpathlineto{\pgfqpoint{1.336582in}{0.774333in}}%
\pgfpathlineto{\pgfqpoint{1.341091in}{0.779330in}}%
\pgfpathlineto{\pgfqpoint{1.345600in}{0.777987in}}%
\pgfpathlineto{\pgfqpoint{1.350109in}{0.779715in}}%
\pgfpathlineto{\pgfqpoint{1.354618in}{0.770888in}}%
\pgfpathlineto{\pgfqpoint{1.359127in}{0.777800in}}%
\pgfpathlineto{\pgfqpoint{1.363636in}{0.777580in}}%
\pgfpathlineto{\pgfqpoint{1.368145in}{0.783259in}}%
\pgfpathlineto{\pgfqpoint{1.372655in}{0.783336in}}%
\pgfpathlineto{\pgfqpoint{1.377164in}{0.774927in}}%
\pgfpathlineto{\pgfqpoint{1.381673in}{0.780893in}}%
\pgfpathlineto{\pgfqpoint{1.386182in}{0.782720in}}%
\pgfpathlineto{\pgfqpoint{1.395200in}{0.777073in}}%
\pgfpathlineto{\pgfqpoint{1.399709in}{0.782698in}}%
\pgfpathlineto{\pgfqpoint{1.404218in}{0.791206in}}%
\pgfpathlineto{\pgfqpoint{1.408727in}{0.787089in}}%
\pgfpathlineto{\pgfqpoint{1.422255in}{0.792185in}}%
\pgfpathlineto{\pgfqpoint{1.426764in}{0.790391in}}%
\pgfpathlineto{\pgfqpoint{1.431273in}{0.785229in}}%
\pgfpathlineto{\pgfqpoint{1.435782in}{0.793187in}}%
\pgfpathlineto{\pgfqpoint{1.444800in}{0.799020in}}%
\pgfpathlineto{\pgfqpoint{1.449309in}{0.806505in}}%
\pgfpathlineto{\pgfqpoint{1.458327in}{0.800352in}}%
\pgfpathlineto{\pgfqpoint{1.462836in}{0.796841in}}%
\pgfpathlineto{\pgfqpoint{1.467345in}{0.798063in}}%
\pgfpathlineto{\pgfqpoint{1.471855in}{0.801618in}}%
\pgfpathlineto{\pgfqpoint{1.476364in}{0.811634in}}%
\pgfpathlineto{\pgfqpoint{1.485382in}{0.799692in}}%
\pgfpathlineto{\pgfqpoint{1.489891in}{0.804534in}}%
\pgfpathlineto{\pgfqpoint{1.494400in}{0.800924in}}%
\pgfpathlineto{\pgfqpoint{1.498909in}{0.812173in}}%
\pgfpathlineto{\pgfqpoint{1.503418in}{0.812316in}}%
\pgfpathlineto{\pgfqpoint{1.507927in}{0.806714in}}%
\pgfpathlineto{\pgfqpoint{1.512436in}{0.819349in}}%
\pgfpathlineto{\pgfqpoint{1.516945in}{0.810676in}}%
\pgfpathlineto{\pgfqpoint{1.521455in}{0.814264in}}%
\pgfpathlineto{\pgfqpoint{1.525964in}{0.811457in}}%
\pgfpathlineto{\pgfqpoint{1.530473in}{0.813097in}}%
\pgfpathlineto{\pgfqpoint{1.534982in}{0.819294in}}%
\pgfpathlineto{\pgfqpoint{1.539491in}{0.815805in}}%
\pgfpathlineto{\pgfqpoint{1.544000in}{0.826569in}}%
\pgfpathlineto{\pgfqpoint{1.548509in}{0.827131in}}%
\pgfpathlineto{\pgfqpoint{1.553018in}{0.830113in}}%
\pgfpathlineto{\pgfqpoint{1.557527in}{0.826679in}}%
\pgfpathlineto{\pgfqpoint{1.562036in}{0.828011in}}%
\pgfpathlineto{\pgfqpoint{1.571055in}{0.827450in}}%
\pgfpathlineto{\pgfqpoint{1.575564in}{0.833988in}}%
\pgfpathlineto{\pgfqpoint{1.580073in}{0.821991in}}%
\pgfpathlineto{\pgfqpoint{1.584582in}{0.828716in}}%
\pgfpathlineto{\pgfqpoint{1.589091in}{0.839645in}}%
\pgfpathlineto{\pgfqpoint{1.593600in}{0.842253in}}%
\pgfpathlineto{\pgfqpoint{1.598109in}{0.836552in}}%
\pgfpathlineto{\pgfqpoint{1.602618in}{0.839447in}}%
\pgfpathlineto{\pgfqpoint{1.607127in}{0.849661in}}%
\pgfpathlineto{\pgfqpoint{1.611636in}{0.836937in}}%
\pgfpathlineto{\pgfqpoint{1.616145in}{0.833250in}}%
\pgfpathlineto{\pgfqpoint{1.620655in}{0.847911in}}%
\pgfpathlineto{\pgfqpoint{1.625164in}{0.836574in}}%
\pgfpathlineto{\pgfqpoint{1.629673in}{0.835650in}}%
\pgfpathlineto{\pgfqpoint{1.634182in}{0.854955in}}%
\pgfpathlineto{\pgfqpoint{1.638691in}{0.851136in}}%
\pgfpathlineto{\pgfqpoint{1.643200in}{0.851730in}}%
\pgfpathlineto{\pgfqpoint{1.647709in}{0.853799in}}%
\pgfpathlineto{\pgfqpoint{1.652218in}{0.850299in}}%
\pgfpathlineto{\pgfqpoint{1.656727in}{0.856188in}}%
\pgfpathlineto{\pgfqpoint{1.661236in}{0.851532in}}%
\pgfpathlineto{\pgfqpoint{1.665745in}{0.850607in}}%
\pgfpathlineto{\pgfqpoint{1.670255in}{0.863738in}}%
\pgfpathlineto{\pgfqpoint{1.674764in}{0.863892in}}%
\pgfpathlineto{\pgfqpoint{1.679273in}{0.871321in}}%
\pgfpathlineto{\pgfqpoint{1.683782in}{0.865224in}}%
\pgfpathlineto{\pgfqpoint{1.688291in}{0.842419in}}%
\pgfpathlineto{\pgfqpoint{1.697309in}{0.871277in}}%
\pgfpathlineto{\pgfqpoint{1.701818in}{0.872246in}}%
\pgfpathlineto{\pgfqpoint{1.706327in}{0.875042in}}%
\pgfpathlineto{\pgfqpoint{1.710836in}{0.872411in}}%
\pgfpathlineto{\pgfqpoint{1.715345in}{0.860865in}}%
\pgfpathlineto{\pgfqpoint{1.719855in}{0.868922in}}%
\pgfpathlineto{\pgfqpoint{1.724364in}{0.886378in}}%
\pgfpathlineto{\pgfqpoint{1.728873in}{0.880952in}}%
\pgfpathlineto{\pgfqpoint{1.733382in}{0.878839in}}%
\pgfpathlineto{\pgfqpoint{1.737891in}{0.879499in}}%
\pgfpathlineto{\pgfqpoint{1.742400in}{0.873501in}}%
\pgfpathlineto{\pgfqpoint{1.746909in}{0.874425in}}%
\pgfpathlineto{\pgfqpoint{1.751418in}{0.876638in}}%
\pgfpathlineto{\pgfqpoint{1.755927in}{0.873600in}}%
\pgfpathlineto{\pgfqpoint{1.764945in}{0.895051in}}%
\pgfpathlineto{\pgfqpoint{1.769455in}{0.878641in}}%
\pgfpathlineto{\pgfqpoint{1.773964in}{0.886455in}}%
\pgfpathlineto{\pgfqpoint{1.778473in}{0.887743in}}%
\pgfpathlineto{\pgfqpoint{1.782982in}{0.882746in}}%
\pgfpathlineto{\pgfqpoint{1.787491in}{0.896592in}}%
\pgfpathlineto{\pgfqpoint{1.792000in}{0.888723in}}%
\pgfpathlineto{\pgfqpoint{1.801018in}{0.909921in}}%
\pgfpathlineto{\pgfqpoint{1.805527in}{0.886587in}}%
\pgfpathlineto{\pgfqpoint{1.810036in}{0.903449in}}%
\pgfpathlineto{\pgfqpoint{1.814545in}{0.901072in}}%
\pgfpathlineto{\pgfqpoint{1.819055in}{0.912122in}}%
\pgfpathlineto{\pgfqpoint{1.828073in}{0.914940in}}%
\pgfpathlineto{\pgfqpoint{1.832582in}{0.914423in}}%
\pgfpathlineto{\pgfqpoint{1.837091in}{0.911451in}}%
\pgfpathlineto{\pgfqpoint{1.841600in}{0.910394in}}%
\pgfpathlineto{\pgfqpoint{1.846109in}{0.901776in}}%
\pgfpathlineto{\pgfqpoint{1.850618in}{0.905254in}}%
\pgfpathlineto{\pgfqpoint{1.855127in}{0.921709in}}%
\pgfpathlineto{\pgfqpoint{1.859636in}{0.925836in}}%
\pgfpathlineto{\pgfqpoint{1.864145in}{0.927509in}}%
\pgfpathlineto{\pgfqpoint{1.868655in}{0.926970in}}%
\pgfpathlineto{\pgfqpoint{1.873164in}{0.914698in}}%
\pgfpathlineto{\pgfqpoint{1.877673in}{0.917780in}}%
\pgfpathlineto{\pgfqpoint{1.886691in}{0.942489in}}%
\pgfpathlineto{\pgfqpoint{1.891200in}{0.917604in}}%
\pgfpathlineto{\pgfqpoint{1.895709in}{0.931505in}}%
\pgfpathlineto{\pgfqpoint{1.900218in}{0.936358in}}%
\pgfpathlineto{\pgfqpoint{1.904727in}{0.943898in}}%
\pgfpathlineto{\pgfqpoint{1.909236in}{0.925407in}}%
\pgfpathlineto{\pgfqpoint{1.913745in}{0.945769in}}%
\pgfpathlineto{\pgfqpoint{1.918255in}{0.947904in}}%
\pgfpathlineto{\pgfqpoint{1.922764in}{0.940002in}}%
\pgfpathlineto{\pgfqpoint{1.927273in}{0.952065in}}%
\pgfpathlineto{\pgfqpoint{1.931782in}{0.949621in}}%
\pgfpathlineto{\pgfqpoint{1.936291in}{0.954134in}}%
\pgfpathlineto{\pgfqpoint{1.940800in}{0.941829in}}%
\pgfpathlineto{\pgfqpoint{1.945309in}{0.959890in}}%
\pgfpathlineto{\pgfqpoint{1.949818in}{0.957887in}}%
\pgfpathlineto{\pgfqpoint{1.958836in}{0.956478in}}%
\pgfpathlineto{\pgfqpoint{1.963345in}{0.955224in}}%
\pgfpathlineto{\pgfqpoint{1.967855in}{0.971425in}}%
\pgfpathlineto{\pgfqpoint{1.972364in}{0.963060in}}%
\pgfpathlineto{\pgfqpoint{1.976873in}{0.972746in}}%
\pgfpathlineto{\pgfqpoint{1.981382in}{0.970842in}}%
\pgfpathlineto{\pgfqpoint{1.985891in}{0.972933in}}%
\pgfpathlineto{\pgfqpoint{1.990400in}{0.976653in}}%
\pgfpathlineto{\pgfqpoint{1.999418in}{0.959318in}}%
\pgfpathlineto{\pgfqpoint{2.003927in}{0.961068in}}%
\pgfpathlineto{\pgfqpoint{2.008436in}{0.966714in}}%
\pgfpathlineto{\pgfqpoint{2.012945in}{0.981606in}}%
\pgfpathlineto{\pgfqpoint{2.017455in}{0.987593in}}%
\pgfpathlineto{\pgfqpoint{2.021964in}{0.970434in}}%
\pgfpathlineto{\pgfqpoint{2.026473in}{0.962488in}}%
\pgfpathlineto{\pgfqpoint{2.030982in}{0.990972in}}%
\pgfpathlineto{\pgfqpoint{2.035491in}{1.004951in}}%
\pgfpathlineto{\pgfqpoint{2.040000in}{0.973406in}}%
\pgfpathlineto{\pgfqpoint{2.044509in}{0.976763in}}%
\pgfpathlineto{\pgfqpoint{2.049018in}{0.992282in}}%
\pgfpathlineto{\pgfqpoint{2.053527in}{0.985832in}}%
\pgfpathlineto{\pgfqpoint{2.058036in}{0.996938in}}%
\pgfpathlineto{\pgfqpoint{2.062545in}{0.999866in}}%
\pgfpathlineto{\pgfqpoint{2.067055in}{1.000526in}}%
\pgfpathlineto{\pgfqpoint{2.071564in}{0.988034in}}%
\pgfpathlineto{\pgfqpoint{2.076073in}{1.029891in}}%
\pgfpathlineto{\pgfqpoint{2.080582in}{0.986372in}}%
\pgfpathlineto{\pgfqpoint{2.089600in}{0.987836in}}%
\pgfpathlineto{\pgfqpoint{2.094109in}{1.010729in}}%
\pgfpathlineto{\pgfqpoint{2.098618in}{1.008396in}}%
\pgfpathlineto{\pgfqpoint{2.103127in}{1.020040in}}%
\pgfpathlineto{\pgfqpoint{2.107636in}{1.012479in}}%
\pgfpathlineto{\pgfqpoint{2.112145in}{0.999877in}}%
\pgfpathlineto{\pgfqpoint{2.116655in}{1.037497in}}%
\pgfpathlineto{\pgfqpoint{2.121164in}{1.024949in}}%
\pgfpathlineto{\pgfqpoint{2.125673in}{1.028185in}}%
\pgfpathlineto{\pgfqpoint{2.130182in}{1.040138in}}%
\pgfpathlineto{\pgfqpoint{2.134691in}{1.017762in}}%
\pgfpathlineto{\pgfqpoint{2.139200in}{1.031674in}}%
\pgfpathlineto{\pgfqpoint{2.143709in}{1.025114in}}%
\pgfpathlineto{\pgfqpoint{2.148218in}{1.038113in}}%
\pgfpathlineto{\pgfqpoint{2.152727in}{1.020998in}}%
\pgfpathlineto{\pgfqpoint{2.161745in}{1.018742in}}%
\pgfpathlineto{\pgfqpoint{2.166255in}{1.003586in}}%
\pgfpathlineto{\pgfqpoint{2.170764in}{1.045531in}}%
\pgfpathlineto{\pgfqpoint{2.175273in}{1.051244in}}%
\pgfpathlineto{\pgfqpoint{2.179782in}{1.050704in}}%
\pgfpathlineto{\pgfqpoint{2.184291in}{1.044530in}}%
\pgfpathlineto{\pgfqpoint{2.188800in}{1.047435in}}%
\pgfpathlineto{\pgfqpoint{2.193309in}{1.054028in}}%
\pgfpathlineto{\pgfqpoint{2.197818in}{1.038047in}}%
\pgfpathlineto{\pgfqpoint{2.202327in}{1.041426in}}%
\pgfpathlineto{\pgfqpoint{2.206836in}{1.067225in}}%
\pgfpathlineto{\pgfqpoint{2.211345in}{1.040545in}}%
\pgfpathlineto{\pgfqpoint{2.215855in}{1.062096in}}%
\pgfpathlineto{\pgfqpoint{2.220364in}{1.049549in}}%
\pgfpathlineto{\pgfqpoint{2.224873in}{1.045091in}}%
\pgfpathlineto{\pgfqpoint{2.229382in}{1.090845in}}%
\pgfpathlineto{\pgfqpoint{2.233891in}{1.062316in}}%
\pgfpathlineto{\pgfqpoint{2.238400in}{1.061656in}}%
\pgfpathlineto{\pgfqpoint{2.242909in}{1.084593in}}%
\pgfpathlineto{\pgfqpoint{2.247418in}{1.064958in}}%
\pgfpathlineto{\pgfqpoint{2.251927in}{1.090184in}}%
\pgfpathlineto{\pgfqpoint{2.256436in}{1.055360in}}%
\pgfpathlineto{\pgfqpoint{2.260945in}{1.074236in}}%
\pgfpathlineto{\pgfqpoint{2.269964in}{1.098439in}}%
\pgfpathlineto{\pgfqpoint{2.274473in}{1.071429in}}%
\pgfpathlineto{\pgfqpoint{2.278982in}{1.099738in}}%
\pgfpathlineto{\pgfqpoint{2.283491in}{1.086871in}}%
\pgfpathlineto{\pgfqpoint{2.288000in}{1.069624in}}%
\pgfpathlineto{\pgfqpoint{2.292509in}{1.081798in}}%
\pgfpathlineto{\pgfqpoint{2.297018in}{1.105197in}}%
\pgfpathlineto{\pgfqpoint{2.301527in}{1.109369in}}%
\pgfpathlineto{\pgfqpoint{2.310545in}{1.123160in}}%
\pgfpathlineto{\pgfqpoint{2.315055in}{1.085254in}}%
\pgfpathlineto{\pgfqpoint{2.319564in}{1.098692in}}%
\pgfpathlineto{\pgfqpoint{2.324073in}{1.093673in}}%
\pgfpathlineto{\pgfqpoint{2.328582in}{1.108433in}}%
\pgfpathlineto{\pgfqpoint{2.333091in}{1.113573in}}%
\pgfpathlineto{\pgfqpoint{2.337600in}{1.128201in}}%
\pgfpathlineto{\pgfqpoint{2.342109in}{1.114850in}}%
\pgfpathlineto{\pgfqpoint{2.346618in}{1.124293in}}%
\pgfpathlineto{\pgfqpoint{2.351127in}{1.098142in}}%
\pgfpathlineto{\pgfqpoint{2.355636in}{1.107806in}}%
\pgfpathlineto{\pgfqpoint{2.360145in}{1.140121in}}%
\pgfpathlineto{\pgfqpoint{2.364655in}{1.147484in}}%
\pgfpathlineto{\pgfqpoint{2.369164in}{1.104669in}}%
\pgfpathlineto{\pgfqpoint{2.373673in}{1.136191in}}%
\pgfpathlineto{\pgfqpoint{2.378182in}{1.148629in}}%
\pgfpathlineto{\pgfqpoint{2.382691in}{1.131569in}}%
\pgfpathlineto{\pgfqpoint{2.387200in}{1.136334in}}%
\pgfpathlineto{\pgfqpoint{2.391709in}{1.144545in}}%
\pgfpathlineto{\pgfqpoint{2.396218in}{1.129422in}}%
\pgfpathlineto{\pgfqpoint{2.400727in}{1.165215in}}%
\pgfpathlineto{\pgfqpoint{2.405236in}{1.144677in}}%
\pgfpathlineto{\pgfqpoint{2.409745in}{1.136797in}}%
\pgfpathlineto{\pgfqpoint{2.414255in}{1.172039in}}%
\pgfpathlineto{\pgfqpoint{2.418764in}{1.130644in}}%
\pgfpathlineto{\pgfqpoint{2.423273in}{1.132218in}}%
\pgfpathlineto{\pgfqpoint{2.432291in}{1.152415in}}%
\pgfpathlineto{\pgfqpoint{2.436800in}{1.152349in}}%
\pgfpathlineto{\pgfqpoint{2.441309in}{1.165215in}}%
\pgfpathlineto{\pgfqpoint{2.445818in}{1.163278in}}%
\pgfpathlineto{\pgfqpoint{2.450327in}{1.176959in}}%
\pgfpathlineto{\pgfqpoint{2.454836in}{1.172557in}}%
\pgfpathlineto{\pgfqpoint{2.459345in}{1.151523in}}%
\pgfpathlineto{\pgfqpoint{2.463855in}{1.177179in}}%
\pgfpathlineto{\pgfqpoint{2.468364in}{1.166646in}}%
\pgfpathlineto{\pgfqpoint{2.472873in}{1.185401in}}%
\pgfpathlineto{\pgfqpoint{2.477382in}{1.192126in}}%
\pgfpathlineto{\pgfqpoint{2.481891in}{1.173041in}}%
\pgfpathlineto{\pgfqpoint{2.486400in}{1.182660in}}%
\pgfpathlineto{\pgfqpoint{2.490909in}{1.217265in}}%
\pgfpathlineto{\pgfqpoint{2.495418in}{1.181516in}}%
\pgfpathlineto{\pgfqpoint{2.499927in}{1.188230in}}%
\pgfpathlineto{\pgfqpoint{2.504436in}{1.241743in}}%
\pgfpathlineto{\pgfqpoint{2.508945in}{1.205488in}}%
\pgfpathlineto{\pgfqpoint{2.513455in}{1.201702in}}%
\pgfpathlineto{\pgfqpoint{2.517964in}{1.223836in}}%
\pgfpathlineto{\pgfqpoint{2.522473in}{1.194195in}}%
\pgfpathlineto{\pgfqpoint{2.526982in}{1.185984in}}%
\pgfpathlineto{\pgfqpoint{2.531491in}{1.205895in}}%
\pgfpathlineto{\pgfqpoint{2.536000in}{1.202681in}}%
\pgfpathlineto{\pgfqpoint{2.540509in}{1.211178in}}%
\pgfpathlineto{\pgfqpoint{2.545018in}{1.232839in}}%
\pgfpathlineto{\pgfqpoint{2.549527in}{1.235557in}}%
\pgfpathlineto{\pgfqpoint{2.554036in}{1.268313in}}%
\pgfpathlineto{\pgfqpoint{2.558545in}{1.193084in}}%
\pgfpathlineto{\pgfqpoint{2.563055in}{1.206688in}}%
\pgfpathlineto{\pgfqpoint{2.567564in}{1.232905in}}%
\pgfpathlineto{\pgfqpoint{2.572073in}{1.223329in}}%
\pgfpathlineto{\pgfqpoint{2.576582in}{1.252771in}}%
\pgfpathlineto{\pgfqpoint{2.581091in}{1.228469in}}%
\pgfpathlineto{\pgfqpoint{2.585600in}{1.219730in}}%
\pgfpathlineto{\pgfqpoint{2.590109in}{1.250625in}}%
\pgfpathlineto{\pgfqpoint{2.594618in}{1.231331in}}%
\pgfpathlineto{\pgfqpoint{2.603636in}{1.254521in}}%
\pgfpathlineto{\pgfqpoint{2.608145in}{1.258319in}}%
\pgfpathlineto{\pgfqpoint{2.612655in}{1.286407in}}%
\pgfpathlineto{\pgfqpoint{2.617164in}{1.293011in}}%
\pgfpathlineto{\pgfqpoint{2.621673in}{1.229031in}}%
\pgfpathlineto{\pgfqpoint{2.626182in}{1.291613in}}%
\pgfpathlineto{\pgfqpoint{2.630691in}{1.303093in}}%
\pgfpathlineto{\pgfqpoint{2.635200in}{1.254191in}}%
\pgfpathlineto{\pgfqpoint{2.639709in}{1.254037in}}%
\pgfpathlineto{\pgfqpoint{2.644218in}{1.265671in}}%
\pgfpathlineto{\pgfqpoint{2.648727in}{1.282643in}}%
\pgfpathlineto{\pgfqpoint{2.653236in}{1.284140in}}%
\pgfpathlineto{\pgfqpoint{2.657745in}{1.294563in}}%
\pgfpathlineto{\pgfqpoint{2.662255in}{1.259948in}}%
\pgfpathlineto{\pgfqpoint{2.666764in}{1.276997in}}%
\pgfpathlineto{\pgfqpoint{2.671273in}{1.274718in}}%
\pgfpathlineto{\pgfqpoint{2.675782in}{1.286000in}}%
\pgfpathlineto{\pgfqpoint{2.680291in}{1.304535in}}%
\pgfpathlineto{\pgfqpoint{2.684800in}{1.284888in}}%
\pgfpathlineto{\pgfqpoint{2.689309in}{1.292857in}}%
\pgfpathlineto{\pgfqpoint{2.693818in}{1.310170in}}%
\pgfpathlineto{\pgfqpoint{2.698327in}{1.333074in}}%
\pgfpathlineto{\pgfqpoint{2.702836in}{1.268434in}}%
\pgfpathlineto{\pgfqpoint{2.707345in}{1.352424in}}%
\pgfpathlineto{\pgfqpoint{2.711855in}{1.325810in}}%
\pgfpathlineto{\pgfqpoint{2.716364in}{1.291789in}}%
\pgfpathlineto{\pgfqpoint{2.720873in}{1.333350in}}%
\pgfpathlineto{\pgfqpoint{2.725382in}{1.337169in}}%
\pgfpathlineto{\pgfqpoint{2.729891in}{1.325304in}}%
\pgfpathlineto{\pgfqpoint{2.734400in}{1.349309in}}%
\pgfpathlineto{\pgfqpoint{2.738909in}{1.332469in}}%
\pgfpathlineto{\pgfqpoint{2.743418in}{1.328375in}}%
\pgfpathlineto{\pgfqpoint{2.747927in}{1.340757in}}%
\pgfpathlineto{\pgfqpoint{2.752436in}{1.338225in}}%
\pgfpathlineto{\pgfqpoint{2.756945in}{1.322189in}}%
\pgfpathlineto{\pgfqpoint{2.761455in}{1.347306in}}%
\pgfpathlineto{\pgfqpoint{2.765964in}{1.345248in}}%
\pgfpathlineto{\pgfqpoint{2.770473in}{1.301299in}}%
\pgfpathlineto{\pgfqpoint{2.774982in}{1.332909in}}%
\pgfpathlineto{\pgfqpoint{2.779491in}{1.342540in}}%
\pgfpathlineto{\pgfqpoint{2.784000in}{1.333052in}}%
\pgfpathlineto{\pgfqpoint{2.793018in}{1.337400in}}%
\pgfpathlineto{\pgfqpoint{2.797527in}{1.358444in}}%
\pgfpathlineto{\pgfqpoint{2.806545in}{1.387160in}}%
\pgfpathlineto{\pgfqpoint{2.811055in}{1.390198in}}%
\pgfpathlineto{\pgfqpoint{2.815564in}{1.368790in}}%
\pgfpathlineto{\pgfqpoint{2.820073in}{1.408744in}}%
\pgfpathlineto{\pgfqpoint{2.824582in}{1.390429in}}%
\pgfpathlineto{\pgfqpoint{2.829091in}{1.347416in}}%
\pgfpathlineto{\pgfqpoint{2.833600in}{1.384265in}}%
\pgfpathlineto{\pgfqpoint{2.838109in}{1.397550in}}%
\pgfpathlineto{\pgfqpoint{2.842618in}{1.427939in}}%
\pgfpathlineto{\pgfqpoint{2.847127in}{1.416162in}}%
\pgfpathlineto{\pgfqpoint{2.851636in}{1.382009in}}%
\pgfpathlineto{\pgfqpoint{2.856145in}{1.360734in}}%
\pgfpathlineto{\pgfqpoint{2.860655in}{1.419607in}}%
\pgfpathlineto{\pgfqpoint{2.865164in}{1.413443in}}%
\pgfpathlineto{\pgfqpoint{2.869673in}{1.418528in}}%
\pgfpathlineto{\pgfqpoint{2.874182in}{1.403174in}}%
\pgfpathlineto{\pgfqpoint{2.878691in}{1.422678in}}%
\pgfpathlineto{\pgfqpoint{2.883200in}{1.403383in}}%
\pgfpathlineto{\pgfqpoint{2.887709in}{1.393841in}}%
\pgfpathlineto{\pgfqpoint{2.892218in}{1.407797in}}%
\pgfpathlineto{\pgfqpoint{2.901236in}{1.429216in}}%
\pgfpathlineto{\pgfqpoint{2.905745in}{1.419673in}}%
\pgfpathlineto{\pgfqpoint{2.910255in}{1.435225in}}%
\pgfpathlineto{\pgfqpoint{2.914764in}{1.428621in}}%
\pgfpathlineto{\pgfqpoint{2.919273in}{1.450832in}}%
\pgfpathlineto{\pgfqpoint{2.923782in}{1.437239in}}%
\pgfpathlineto{\pgfqpoint{2.928291in}{1.444988in}}%
\pgfpathlineto{\pgfqpoint{2.932800in}{1.425881in}}%
\pgfpathlineto{\pgfqpoint{2.937309in}{1.395382in}}%
\pgfpathlineto{\pgfqpoint{2.941818in}{1.458328in}}%
\pgfpathlineto{\pgfqpoint{2.946327in}{1.472581in}}%
\pgfpathlineto{\pgfqpoint{2.950836in}{1.426662in}}%
\pgfpathlineto{\pgfqpoint{2.955345in}{1.434091in}}%
\pgfpathlineto{\pgfqpoint{2.959855in}{1.437316in}}%
\pgfpathlineto{\pgfqpoint{2.964364in}{1.476609in}}%
\pgfpathlineto{\pgfqpoint{2.968873in}{1.466571in}}%
\pgfpathlineto{\pgfqpoint{2.973382in}{1.533402in}}%
\pgfpathlineto{\pgfqpoint{2.977891in}{1.488892in}}%
\pgfpathlineto{\pgfqpoint{2.982400in}{1.481463in}}%
\pgfpathlineto{\pgfqpoint{2.986909in}{1.422876in}}%
\pgfpathlineto{\pgfqpoint{2.991418in}{1.489300in}}%
\pgfpathlineto{\pgfqpoint{2.995927in}{1.501748in}}%
\pgfpathlineto{\pgfqpoint{3.000436in}{1.423558in}}%
\pgfpathlineto{\pgfqpoint{3.004945in}{1.475410in}}%
\pgfpathlineto{\pgfqpoint{3.009455in}{1.510058in}}%
\pgfpathlineto{\pgfqpoint{3.013964in}{1.498831in}}%
\pgfpathlineto{\pgfqpoint{3.018473in}{1.558145in}}%
\pgfpathlineto{\pgfqpoint{3.022982in}{1.541426in}}%
\pgfpathlineto{\pgfqpoint{3.027491in}{1.468663in}}%
\pgfpathlineto{\pgfqpoint{3.032000in}{1.529077in}}%
\pgfpathlineto{\pgfqpoint{3.036509in}{1.528802in}}%
\pgfpathlineto{\pgfqpoint{3.041018in}{1.536859in}}%
\pgfpathlineto{\pgfqpoint{3.045527in}{1.535461in}}%
\pgfpathlineto{\pgfqpoint{3.050036in}{1.531036in}}%
\pgfpathlineto{\pgfqpoint{3.054545in}{1.489938in}}%
\pgfpathlineto{\pgfqpoint{3.059055in}{1.508880in}}%
\pgfpathlineto{\pgfqpoint{3.063564in}{1.562074in}}%
\pgfpathlineto{\pgfqpoint{3.068073in}{1.562382in}}%
\pgfpathlineto{\pgfqpoint{3.072582in}{1.550892in}}%
\pgfpathlineto{\pgfqpoint{3.077091in}{1.496960in}}%
\pgfpathlineto{\pgfqpoint{3.081600in}{1.547006in}}%
\pgfpathlineto{\pgfqpoint{3.086109in}{1.530519in}}%
\pgfpathlineto{\pgfqpoint{3.090618in}{1.539676in}}%
\pgfpathlineto{\pgfqpoint{3.095127in}{1.589249in}}%
\pgfpathlineto{\pgfqpoint{3.099636in}{1.496630in}}%
\pgfpathlineto{\pgfqpoint{3.104145in}{1.522066in}}%
\pgfpathlineto{\pgfqpoint{3.108655in}{1.584659in}}%
\pgfpathlineto{\pgfqpoint{3.113164in}{1.593575in}}%
\pgfpathlineto{\pgfqpoint{3.117673in}{1.586420in}}%
\pgfpathlineto{\pgfqpoint{3.122182in}{1.565750in}}%
\pgfpathlineto{\pgfqpoint{3.126691in}{1.522385in}}%
\pgfpathlineto{\pgfqpoint{3.131200in}{1.575535in}}%
\pgfpathlineto{\pgfqpoint{3.135709in}{1.536473in}}%
\pgfpathlineto{\pgfqpoint{3.140218in}{1.585089in}}%
\pgfpathlineto{\pgfqpoint{3.144727in}{1.535648in}}%
\pgfpathlineto{\pgfqpoint{3.149236in}{1.584814in}}%
\pgfpathlineto{\pgfqpoint{3.153745in}{1.573510in}}%
\pgfpathlineto{\pgfqpoint{3.158255in}{1.577824in}}%
\pgfpathlineto{\pgfqpoint{3.162764in}{1.626759in}}%
\pgfpathlineto{\pgfqpoint{3.167273in}{1.618856in}}%
\pgfpathlineto{\pgfqpoint{3.171782in}{1.630446in}}%
\pgfpathlineto{\pgfqpoint{3.176291in}{1.624140in}}%
\pgfpathlineto{\pgfqpoint{3.180800in}{1.589073in}}%
\pgfpathlineto{\pgfqpoint{3.185309in}{1.604867in}}%
\pgfpathlineto{\pgfqpoint{3.189818in}{1.655244in}}%
\pgfpathlineto{\pgfqpoint{3.194327in}{1.631239in}}%
\pgfpathlineto{\pgfqpoint{3.198836in}{1.647231in}}%
\pgfpathlineto{\pgfqpoint{3.203345in}{1.608301in}}%
\pgfpathlineto{\pgfqpoint{3.207855in}{1.617437in}}%
\pgfpathlineto{\pgfqpoint{3.212364in}{1.637655in}}%
\pgfpathlineto{\pgfqpoint{3.216873in}{1.636830in}}%
\pgfpathlineto{\pgfqpoint{3.221382in}{1.674857in}}%
\pgfpathlineto{\pgfqpoint{3.225891in}{1.615246in}}%
\pgfpathlineto{\pgfqpoint{3.230400in}{1.625537in}}%
\pgfpathlineto{\pgfqpoint{3.234909in}{1.642069in}}%
\pgfpathlineto{\pgfqpoint{3.239418in}{1.663245in}}%
\pgfpathlineto{\pgfqpoint{3.243927in}{1.612539in}}%
\pgfpathlineto{\pgfqpoint{3.252945in}{1.693513in}}%
\pgfpathlineto{\pgfqpoint{3.257455in}{1.646384in}}%
\pgfpathlineto{\pgfqpoint{3.261964in}{1.720644in}}%
\pgfpathlineto{\pgfqpoint{3.266473in}{1.676684in}}%
\pgfpathlineto{\pgfqpoint{3.275491in}{1.684312in}}%
\pgfpathlineto{\pgfqpoint{3.280000in}{1.701614in}}%
\pgfpathlineto{\pgfqpoint{3.284509in}{1.676486in}}%
\pgfpathlineto{\pgfqpoint{3.289018in}{1.721271in}}%
\pgfpathlineto{\pgfqpoint{3.293527in}{1.710386in}}%
\pgfpathlineto{\pgfqpoint{3.298036in}{1.677224in}}%
\pgfpathlineto{\pgfqpoint{3.302545in}{1.763151in}}%
\pgfpathlineto{\pgfqpoint{3.307055in}{1.745992in}}%
\pgfpathlineto{\pgfqpoint{3.316073in}{1.722801in}}%
\pgfpathlineto{\pgfqpoint{3.320582in}{1.760641in}}%
\pgfpathlineto{\pgfqpoint{3.325091in}{1.702021in}}%
\pgfpathlineto{\pgfqpoint{3.329600in}{1.684631in}}%
\pgfpathlineto{\pgfqpoint{3.334109in}{1.753740in}}%
\pgfpathlineto{\pgfqpoint{3.343127in}{1.702935in}}%
\pgfpathlineto{\pgfqpoint{3.347636in}{1.737902in}}%
\pgfpathlineto{\pgfqpoint{3.352145in}{1.732773in}}%
\pgfpathlineto{\pgfqpoint{3.356655in}{1.768159in}}%
\pgfpathlineto{\pgfqpoint{3.361164in}{1.744473in}}%
\pgfpathlineto{\pgfqpoint{3.365673in}{1.757758in}}%
\pgfpathlineto{\pgfqpoint{3.370182in}{1.781598in}}%
\pgfpathlineto{\pgfqpoint{3.374691in}{1.687889in}}%
\pgfpathlineto{\pgfqpoint{3.379200in}{1.784800in}}%
\pgfpathlineto{\pgfqpoint{3.383709in}{1.752100in}}%
\pgfpathlineto{\pgfqpoint{3.388218in}{1.732949in}}%
\pgfpathlineto{\pgfqpoint{3.401745in}{1.806934in}}%
\pgfpathlineto{\pgfqpoint{3.406255in}{1.812867in}}%
\pgfpathlineto{\pgfqpoint{3.410764in}{1.754511in}}%
\pgfpathlineto{\pgfqpoint{3.415273in}{1.753058in}}%
\pgfpathlineto{\pgfqpoint{3.419782in}{1.803941in}}%
\pgfpathlineto{\pgfqpoint{3.424291in}{1.797062in}}%
\pgfpathlineto{\pgfqpoint{3.428800in}{1.820428in}}%
\pgfpathlineto{\pgfqpoint{3.433309in}{1.825810in}}%
\pgfpathlineto{\pgfqpoint{3.437818in}{1.834439in}}%
\pgfpathlineto{\pgfqpoint{3.442327in}{1.825139in}}%
\pgfpathlineto{\pgfqpoint{3.446836in}{1.773937in}}%
\pgfpathlineto{\pgfqpoint{3.451345in}{1.785846in}}%
\pgfpathlineto{\pgfqpoint{3.455855in}{1.866567in}}%
\pgfpathlineto{\pgfqpoint{3.464873in}{1.803291in}}%
\pgfpathlineto{\pgfqpoint{3.469382in}{1.829641in}}%
\pgfpathlineto{\pgfqpoint{3.473891in}{1.866809in}}%
\pgfpathlineto{\pgfqpoint{3.478400in}{1.840603in}}%
\pgfpathlineto{\pgfqpoint{3.482909in}{1.772363in}}%
\pgfpathlineto{\pgfqpoint{3.487418in}{1.841131in}}%
\pgfpathlineto{\pgfqpoint{3.491927in}{1.886874in}}%
\pgfpathlineto{\pgfqpoint{3.496436in}{1.878168in}}%
\pgfpathlineto{\pgfqpoint{3.500945in}{1.785538in}}%
\pgfpathlineto{\pgfqpoint{3.505455in}{1.845875in}}%
\pgfpathlineto{\pgfqpoint{3.509964in}{1.815475in}}%
\pgfpathlineto{\pgfqpoint{3.514473in}{1.866534in}}%
\pgfpathlineto{\pgfqpoint{3.518982in}{1.892818in}}%
\pgfpathlineto{\pgfqpoint{3.523491in}{1.877034in}}%
\pgfpathlineto{\pgfqpoint{3.528000in}{1.867712in}}%
\pgfpathlineto{\pgfqpoint{3.532509in}{1.925276in}}%
\pgfpathlineto{\pgfqpoint{3.537018in}{1.889890in}}%
\pgfpathlineto{\pgfqpoint{3.541527in}{1.928346in}}%
\pgfpathlineto{\pgfqpoint{3.546036in}{1.925232in}}%
\pgfpathlineto{\pgfqpoint{3.550545in}{1.875097in}}%
\pgfpathlineto{\pgfqpoint{3.555055in}{1.872037in}}%
\pgfpathlineto{\pgfqpoint{3.559564in}{1.940442in}}%
\pgfpathlineto{\pgfqpoint{3.564073in}{1.880622in}}%
\pgfpathlineto{\pgfqpoint{3.568582in}{1.909459in}}%
\pgfpathlineto{\pgfqpoint{3.573091in}{1.890165in}}%
\pgfpathlineto{\pgfqpoint{3.577600in}{1.924208in}}%
\pgfpathlineto{\pgfqpoint{3.582109in}{1.904297in}}%
\pgfpathlineto{\pgfqpoint{3.586618in}{1.903549in}}%
\pgfpathlineto{\pgfqpoint{3.591127in}{1.940938in}}%
\pgfpathlineto{\pgfqpoint{3.595636in}{1.909756in}}%
\pgfpathlineto{\pgfqpoint{3.604655in}{1.950601in}}%
\pgfpathlineto{\pgfqpoint{3.609164in}{1.892179in}}%
\pgfpathlineto{\pgfqpoint{3.613673in}{1.943029in}}%
\pgfpathlineto{\pgfqpoint{3.618182in}{1.944614in}}%
\pgfpathlineto{\pgfqpoint{3.622691in}{1.961465in}}%
\pgfpathlineto{\pgfqpoint{3.627200in}{1.933409in}}%
\pgfpathlineto{\pgfqpoint{3.631709in}{1.989388in}}%
\pgfpathlineto{\pgfqpoint{3.636218in}{1.936876in}}%
\pgfpathlineto{\pgfqpoint{3.640727in}{1.937757in}}%
\pgfpathlineto{\pgfqpoint{3.645236in}{1.971503in}}%
\pgfpathlineto{\pgfqpoint{3.649745in}{1.965647in}}%
\pgfpathlineto{\pgfqpoint{3.654255in}{1.952417in}}%
\pgfpathlineto{\pgfqpoint{3.658764in}{2.008308in}}%
\pgfpathlineto{\pgfqpoint{3.663273in}{1.955686in}}%
\pgfpathlineto{\pgfqpoint{3.672291in}{2.009585in}}%
\pgfpathlineto{\pgfqpoint{3.681309in}{1.960166in}}%
\pgfpathlineto{\pgfqpoint{3.685818in}{2.064804in}}%
\pgfpathlineto{\pgfqpoint{3.690327in}{2.013987in}}%
\pgfpathlineto{\pgfqpoint{3.694836in}{1.995342in}}%
\pgfpathlineto{\pgfqpoint{3.699345in}{2.063252in}}%
\pgfpathlineto{\pgfqpoint{3.703855in}{2.001484in}}%
\pgfpathlineto{\pgfqpoint{3.708364in}{2.008924in}}%
\pgfpathlineto{\pgfqpoint{3.712873in}{2.029936in}}%
\pgfpathlineto{\pgfqpoint{3.717382in}{1.987352in}}%
\pgfpathlineto{\pgfqpoint{3.721891in}{2.081380in}}%
\pgfpathlineto{\pgfqpoint{3.726400in}{2.122555in}}%
\pgfpathlineto{\pgfqpoint{3.730909in}{2.126275in}}%
\pgfpathlineto{\pgfqpoint{3.735418in}{2.051706in}}%
\pgfpathlineto{\pgfqpoint{3.739927in}{2.030057in}}%
\pgfpathlineto{\pgfqpoint{3.744436in}{2.051068in}}%
\pgfpathlineto{\pgfqpoint{3.748945in}{2.018665in}}%
\pgfpathlineto{\pgfqpoint{3.753455in}{2.032962in}}%
\pgfpathlineto{\pgfqpoint{3.757964in}{2.075590in}}%
\pgfpathlineto{\pgfqpoint{3.762473in}{2.075921in}}%
\pgfpathlineto{\pgfqpoint{3.766982in}{2.050804in}}%
\pgfpathlineto{\pgfqpoint{3.771491in}{2.099606in}}%
\pgfpathlineto{\pgfqpoint{3.776000in}{2.055922in}}%
\pgfpathlineto{\pgfqpoint{3.780509in}{2.070792in}}%
\pgfpathlineto{\pgfqpoint{3.785018in}{2.103723in}}%
\pgfpathlineto{\pgfqpoint{3.789527in}{2.102963in}}%
\pgfpathlineto{\pgfqpoint{3.798545in}{2.142488in}}%
\pgfpathlineto{\pgfqpoint{3.803055in}{2.140627in}}%
\pgfpathlineto{\pgfqpoint{3.807564in}{2.110283in}}%
\pgfpathlineto{\pgfqpoint{3.816582in}{2.152041in}}%
\pgfpathlineto{\pgfqpoint{3.821091in}{2.113959in}}%
\pgfpathlineto{\pgfqpoint{3.830109in}{2.065883in}}%
\pgfpathlineto{\pgfqpoint{3.839127in}{2.107025in}}%
\pgfpathlineto{\pgfqpoint{3.843636in}{2.146274in}}%
\pgfpathlineto{\pgfqpoint{3.848145in}{2.169861in}}%
\pgfpathlineto{\pgfqpoint{3.852655in}{2.184697in}}%
\pgfpathlineto{\pgfqpoint{3.857164in}{2.159415in}}%
\pgfpathlineto{\pgfqpoint{3.861673in}{2.170444in}}%
\pgfpathlineto{\pgfqpoint{3.866182in}{2.121102in}}%
\pgfpathlineto{\pgfqpoint{3.870691in}{2.169398in}}%
\pgfpathlineto{\pgfqpoint{3.875200in}{2.171787in}}%
\pgfpathlineto{\pgfqpoint{3.879709in}{2.189298in}}%
\pgfpathlineto{\pgfqpoint{3.884218in}{2.165766in}}%
\pgfpathlineto{\pgfqpoint{3.888727in}{2.129280in}}%
\pgfpathlineto{\pgfqpoint{3.893236in}{2.221745in}}%
\pgfpathlineto{\pgfqpoint{3.897745in}{2.183266in}}%
\pgfpathlineto{\pgfqpoint{3.902255in}{2.159184in}}%
\pgfpathlineto{\pgfqpoint{3.906764in}{2.165392in}}%
\pgfpathlineto{\pgfqpoint{3.911273in}{2.250945in}}%
\pgfpathlineto{\pgfqpoint{3.915782in}{2.151535in}}%
\pgfpathlineto{\pgfqpoint{3.920291in}{2.251286in}}%
\pgfpathlineto{\pgfqpoint{3.924800in}{2.158128in}}%
\pgfpathlineto{\pgfqpoint{3.929309in}{2.155717in}}%
\pgfpathlineto{\pgfqpoint{3.933818in}{2.203067in}}%
\pgfpathlineto{\pgfqpoint{3.938327in}{2.287409in}}%
\pgfpathlineto{\pgfqpoint{3.942836in}{2.290942in}}%
\pgfpathlineto{\pgfqpoint{3.947345in}{2.325635in}}%
\pgfpathlineto{\pgfqpoint{3.951855in}{2.258374in}}%
\pgfpathlineto{\pgfqpoint{3.956364in}{2.239036in}}%
\pgfpathlineto{\pgfqpoint{3.960873in}{2.230737in}}%
\pgfpathlineto{\pgfqpoint{3.965382in}{2.216319in}}%
\pgfpathlineto{\pgfqpoint{3.969891in}{2.282522in}}%
\pgfpathlineto{\pgfqpoint{3.974400in}{2.292362in}}%
\pgfpathlineto{\pgfqpoint{3.978909in}{2.262337in}}%
\pgfpathlineto{\pgfqpoint{3.983418in}{2.198004in}}%
\pgfpathlineto{\pgfqpoint{3.987927in}{2.250890in}}%
\pgfpathlineto{\pgfqpoint{3.992436in}{2.224695in}}%
\pgfpathlineto{\pgfqpoint{3.996945in}{2.316719in}}%
\pgfpathlineto{\pgfqpoint{4.001455in}{2.255854in}}%
\pgfpathlineto{\pgfqpoint{4.005964in}{2.235305in}}%
\pgfpathlineto{\pgfqpoint{4.010473in}{2.296104in}}%
\pgfpathlineto{\pgfqpoint{4.014982in}{2.259189in}}%
\pgfpathlineto{\pgfqpoint{4.019491in}{2.335716in}}%
\pgfpathlineto{\pgfqpoint{4.024000in}{2.259475in}}%
\pgfpathlineto{\pgfqpoint{4.028509in}{2.275676in}}%
\pgfpathlineto{\pgfqpoint{4.033018in}{2.244704in}}%
\pgfpathlineto{\pgfqpoint{4.037527in}{2.337709in}}%
\pgfpathlineto{\pgfqpoint{4.042036in}{2.346888in}}%
\pgfpathlineto{\pgfqpoint{4.046545in}{2.299395in}}%
\pgfpathlineto{\pgfqpoint{4.051055in}{2.376429in}}%
\pgfpathlineto{\pgfqpoint{4.055564in}{2.382670in}}%
\pgfpathlineto{\pgfqpoint{4.060073in}{2.340317in}}%
\pgfpathlineto{\pgfqpoint{4.064582in}{2.284856in}}%
\pgfpathlineto{\pgfqpoint{4.069091in}{2.353173in}}%
\pgfpathlineto{\pgfqpoint{4.073600in}{2.328992in}}%
\pgfpathlineto{\pgfqpoint{4.078109in}{2.335397in}}%
\pgfpathlineto{\pgfqpoint{4.082618in}{2.329091in}}%
\pgfpathlineto{\pgfqpoint{4.087127in}{2.340482in}}%
\pgfpathlineto{\pgfqpoint{4.091636in}{2.392455in}}%
\pgfpathlineto{\pgfqpoint{4.096145in}{2.344081in}}%
\pgfpathlineto{\pgfqpoint{4.105164in}{2.456600in}}%
\pgfpathlineto{\pgfqpoint{4.109673in}{2.339261in}}%
\pgfpathlineto{\pgfqpoint{4.118691in}{2.437438in}}%
\pgfpathlineto{\pgfqpoint{4.123200in}{2.414644in}}%
\pgfpathlineto{\pgfqpoint{4.127709in}{2.458031in}}%
\pgfpathlineto{\pgfqpoint{4.132218in}{2.326801in}}%
\pgfpathlineto{\pgfqpoint{4.136727in}{2.392565in}}%
\pgfpathlineto{\pgfqpoint{4.145745in}{2.477887in}}%
\pgfpathlineto{\pgfqpoint{4.150255in}{2.455610in}}%
\pgfpathlineto{\pgfqpoint{4.154764in}{2.402063in}}%
\pgfpathlineto{\pgfqpoint{4.159273in}{2.466792in}}%
\pgfpathlineto{\pgfqpoint{4.163782in}{2.465835in}}%
\pgfpathlineto{\pgfqpoint{4.168291in}{2.407489in}}%
\pgfpathlineto{\pgfqpoint{4.172800in}{2.437471in}}%
\pgfpathlineto{\pgfqpoint{4.177309in}{2.430955in}}%
\pgfpathlineto{\pgfqpoint{4.181818in}{2.442611in}}%
\pgfpathlineto{\pgfqpoint{4.195345in}{2.564837in}}%
\pgfpathlineto{\pgfqpoint{4.199855in}{2.455345in}}%
\pgfpathlineto{\pgfqpoint{4.204364in}{2.527460in}}%
\pgfpathlineto{\pgfqpoint{4.208873in}{2.403494in}}%
\pgfpathlineto{\pgfqpoint{4.213382in}{2.501088in}}%
\pgfpathlineto{\pgfqpoint{4.217891in}{2.510080in}}%
\pgfpathlineto{\pgfqpoint{4.226909in}{2.466605in}}%
\pgfpathlineto{\pgfqpoint{4.231418in}{2.507967in}}%
\pgfpathlineto{\pgfqpoint{4.235927in}{2.466627in}}%
\pgfpathlineto{\pgfqpoint{4.240436in}{2.518214in}}%
\pgfpathlineto{\pgfqpoint{4.244945in}{2.493626in}}%
\pgfpathlineto{\pgfqpoint{4.249455in}{2.504808in}}%
\pgfpathlineto{\pgfqpoint{4.253964in}{2.552378in}}%
\pgfpathlineto{\pgfqpoint{4.258473in}{2.560160in}}%
\pgfpathlineto{\pgfqpoint{4.262982in}{2.525665in}}%
\pgfpathlineto{\pgfqpoint{4.267491in}{2.541482in}}%
\pgfpathlineto{\pgfqpoint{4.272000in}{2.611868in}}%
\pgfpathlineto{\pgfqpoint{4.276509in}{2.545367in}}%
\pgfpathlineto{\pgfqpoint{4.281018in}{2.447399in}}%
\pgfpathlineto{\pgfqpoint{4.285527in}{2.582382in}}%
\pgfpathlineto{\pgfqpoint{4.290036in}{2.614025in}}%
\pgfpathlineto{\pgfqpoint{4.294545in}{2.547546in}}%
\pgfpathlineto{\pgfqpoint{4.299055in}{2.625208in}}%
\pgfpathlineto{\pgfqpoint{4.303564in}{2.637821in}}%
\pgfpathlineto{\pgfqpoint{4.308073in}{2.624701in}}%
\pgfpathlineto{\pgfqpoint{4.312582in}{2.547931in}}%
\pgfpathlineto{\pgfqpoint{4.321600in}{2.599299in}}%
\pgfpathlineto{\pgfqpoint{4.326109in}{2.682441in}}%
\pgfpathlineto{\pgfqpoint{4.330618in}{2.563429in}}%
\pgfpathlineto{\pgfqpoint{4.335127in}{2.638019in}}%
\pgfpathlineto{\pgfqpoint{4.339636in}{2.592001in}}%
\pgfpathlineto{\pgfqpoint{4.344145in}{2.511126in}}%
\pgfpathlineto{\pgfqpoint{4.348655in}{2.613849in}}%
\pgfpathlineto{\pgfqpoint{4.353164in}{2.674120in}}%
\pgfpathlineto{\pgfqpoint{4.357673in}{2.566697in}}%
\pgfpathlineto{\pgfqpoint{4.362182in}{2.557903in}}%
\pgfpathlineto{\pgfqpoint{4.366691in}{2.630810in}}%
\pgfpathlineto{\pgfqpoint{4.371200in}{2.597273in}}%
\pgfpathlineto{\pgfqpoint{4.375709in}{2.647892in}}%
\pgfpathlineto{\pgfqpoint{4.380218in}{2.676421in}}%
\pgfpathlineto{\pgfqpoint{4.384727in}{2.651161in}}%
\pgfpathlineto{\pgfqpoint{4.389236in}{2.680042in}}%
\pgfpathlineto{\pgfqpoint{4.393745in}{2.675617in}}%
\pgfpathlineto{\pgfqpoint{4.398255in}{2.595644in}}%
\pgfpathlineto{\pgfqpoint{4.402764in}{2.612209in}}%
\pgfpathlineto{\pgfqpoint{4.407273in}{2.696298in}}%
\pgfpathlineto{\pgfqpoint{4.416291in}{2.728426in}}%
\pgfpathlineto{\pgfqpoint{4.420800in}{2.707129in}}%
\pgfpathlineto{\pgfqpoint{4.425309in}{2.649697in}}%
\pgfpathlineto{\pgfqpoint{4.429818in}{2.708471in}}%
\pgfpathlineto{\pgfqpoint{4.434327in}{2.697597in}}%
\pgfpathlineto{\pgfqpoint{4.438836in}{2.691609in}}%
\pgfpathlineto{\pgfqpoint{4.443345in}{2.705235in}}%
\pgfpathlineto{\pgfqpoint{4.447855in}{2.699380in}}%
\pgfpathlineto{\pgfqpoint{4.452364in}{2.755304in}}%
\pgfpathlineto{\pgfqpoint{4.456873in}{2.709418in}}%
\pgfpathlineto{\pgfqpoint{4.461382in}{2.772364in}}%
\pgfpathlineto{\pgfqpoint{4.465891in}{2.679910in}}%
\pgfpathlineto{\pgfqpoint{4.470400in}{2.691004in}}%
\pgfpathlineto{\pgfqpoint{4.474909in}{2.665590in}}%
\pgfpathlineto{\pgfqpoint{4.479418in}{2.805064in}}%
\pgfpathlineto{\pgfqpoint{4.483927in}{2.713160in}}%
\pgfpathlineto{\pgfqpoint{4.488436in}{2.769612in}}%
\pgfpathlineto{\pgfqpoint{4.492945in}{2.903274in}}%
\pgfpathlineto{\pgfqpoint{4.497455in}{2.831523in}}%
\pgfpathlineto{\pgfqpoint{4.501964in}{2.828100in}}%
\pgfpathlineto{\pgfqpoint{4.506473in}{2.827671in}}%
\pgfpathlineto{\pgfqpoint{4.510982in}{2.837929in}}%
\pgfpathlineto{\pgfqpoint{4.515491in}{2.747346in}}%
\pgfpathlineto{\pgfqpoint{4.520000in}{2.785406in}}%
\pgfpathlineto{\pgfqpoint{4.524509in}{2.780475in}}%
\pgfpathlineto{\pgfqpoint{4.529018in}{2.822366in}}%
\pgfpathlineto{\pgfqpoint{4.533527in}{2.826218in}}%
\pgfpathlineto{\pgfqpoint{4.538036in}{2.804822in}}%
\pgfpathlineto{\pgfqpoint{4.542545in}{2.865368in}}%
\pgfpathlineto{\pgfqpoint{4.547055in}{2.877640in}}%
\pgfpathlineto{\pgfqpoint{4.551564in}{2.823478in}}%
\pgfpathlineto{\pgfqpoint{4.556073in}{2.729516in}}%
\pgfpathlineto{\pgfqpoint{4.565091in}{2.863035in}}%
\pgfpathlineto{\pgfqpoint{4.569600in}{2.834143in}}%
\pgfpathlineto{\pgfqpoint{4.574109in}{2.848990in}}%
\pgfpathlineto{\pgfqpoint{4.578618in}{2.826174in}}%
\pgfpathlineto{\pgfqpoint{4.583127in}{2.838237in}}%
\pgfpathlineto{\pgfqpoint{4.587636in}{2.953618in}}%
\pgfpathlineto{\pgfqpoint{4.592145in}{2.847582in}}%
\pgfpathlineto{\pgfqpoint{4.596655in}{2.911760in}}%
\pgfpathlineto{\pgfqpoint{4.601164in}{2.826009in}}%
\pgfpathlineto{\pgfqpoint{4.605673in}{2.870123in}}%
\pgfpathlineto{\pgfqpoint{4.610182in}{2.876088in}}%
\pgfpathlineto{\pgfqpoint{4.614691in}{2.870662in}}%
\pgfpathlineto{\pgfqpoint{4.619200in}{2.881371in}}%
\pgfpathlineto{\pgfqpoint{4.623709in}{2.929293in}}%
\pgfpathlineto{\pgfqpoint{4.628218in}{2.952319in}}%
\pgfpathlineto{\pgfqpoint{4.632727in}{2.940003in}}%
\pgfpathlineto{\pgfqpoint{4.637236in}{2.871069in}}%
\pgfpathlineto{\pgfqpoint{4.641745in}{2.919916in}}%
\pgfpathlineto{\pgfqpoint{4.646255in}{2.844445in}}%
\pgfpathlineto{\pgfqpoint{4.650764in}{2.879104in}}%
\pgfpathlineto{\pgfqpoint{4.655273in}{2.878257in}}%
\pgfpathlineto{\pgfqpoint{4.659782in}{2.964613in}}%
\pgfpathlineto{\pgfqpoint{4.664291in}{2.933113in}}%
\pgfpathlineto{\pgfqpoint{4.668800in}{2.993274in}}%
\pgfpathlineto{\pgfqpoint{4.673309in}{2.920059in}}%
\pgfpathlineto{\pgfqpoint{4.677818in}{3.030288in}}%
\pgfpathlineto{\pgfqpoint{4.682327in}{2.987572in}}%
\pgfpathlineto{\pgfqpoint{4.686836in}{2.988530in}}%
\pgfpathlineto{\pgfqpoint{4.691345in}{2.962214in}}%
\pgfpathlineto{\pgfqpoint{4.695855in}{2.993230in}}%
\pgfpathlineto{\pgfqpoint{4.704873in}{2.950118in}}%
\pgfpathlineto{\pgfqpoint{4.709382in}{2.999437in}}%
\pgfpathlineto{\pgfqpoint{4.713891in}{3.008419in}}%
\pgfpathlineto{\pgfqpoint{4.718400in}{3.035814in}}%
\pgfpathlineto{\pgfqpoint{4.722909in}{2.994143in}}%
\pgfpathlineto{\pgfqpoint{4.727418in}{2.998480in}}%
\pgfpathlineto{\pgfqpoint{4.731927in}{2.941279in}}%
\pgfpathlineto{\pgfqpoint{4.736436in}{3.113904in}}%
\pgfpathlineto{\pgfqpoint{4.740945in}{3.072047in}}%
\pgfpathlineto{\pgfqpoint{4.745455in}{3.002090in}}%
\pgfpathlineto{\pgfqpoint{4.749964in}{3.034174in}}%
\pgfpathlineto{\pgfqpoint{4.754473in}{3.023773in}}%
\pgfpathlineto{\pgfqpoint{4.758982in}{3.042572in}}%
\pgfpathlineto{\pgfqpoint{4.763491in}{3.196871in}}%
\pgfpathlineto{\pgfqpoint{4.772509in}{3.074986in}}%
\pgfpathlineto{\pgfqpoint{4.777018in}{3.036859in}}%
\pgfpathlineto{\pgfqpoint{4.781527in}{3.128642in}}%
\pgfpathlineto{\pgfqpoint{4.786036in}{3.002222in}}%
\pgfpathlineto{\pgfqpoint{4.790545in}{2.996014in}}%
\pgfpathlineto{\pgfqpoint{4.795055in}{3.164028in}}%
\pgfpathlineto{\pgfqpoint{4.799564in}{3.009222in}}%
\pgfpathlineto{\pgfqpoint{4.804073in}{3.153759in}}%
\pgfpathlineto{\pgfqpoint{4.808582in}{3.123711in}}%
\pgfpathlineto{\pgfqpoint{4.813091in}{3.146010in}}%
\pgfpathlineto{\pgfqpoint{4.817600in}{3.130722in}}%
\pgfpathlineto{\pgfqpoint{4.822109in}{3.149070in}}%
\pgfpathlineto{\pgfqpoint{4.826618in}{3.232774in}}%
\pgfpathlineto{\pgfqpoint{4.831127in}{3.247511in}}%
\pgfpathlineto{\pgfqpoint{4.835636in}{3.155399in}}%
\pgfpathlineto{\pgfqpoint{4.840145in}{3.138713in}}%
\pgfpathlineto{\pgfqpoint{4.844655in}{3.128587in}}%
\pgfpathlineto{\pgfqpoint{4.849164in}{3.199942in}}%
\pgfpathlineto{\pgfqpoint{4.853673in}{3.171105in}}%
\pgfpathlineto{\pgfqpoint{4.858182in}{3.132637in}}%
\pgfpathlineto{\pgfqpoint{4.862691in}{3.186107in}}%
\pgfpathlineto{\pgfqpoint{4.867200in}{3.253378in}}%
\pgfpathlineto{\pgfqpoint{4.871709in}{3.251364in}}%
\pgfpathlineto{\pgfqpoint{4.876218in}{3.179492in}}%
\pgfpathlineto{\pgfqpoint{4.880727in}{3.191412in}}%
\pgfpathlineto{\pgfqpoint{4.885236in}{3.318404in}}%
\pgfpathlineto{\pgfqpoint{4.894255in}{3.134442in}}%
\pgfpathlineto{\pgfqpoint{4.898764in}{3.281763in}}%
\pgfpathlineto{\pgfqpoint{4.903273in}{3.266894in}}%
\pgfpathlineto{\pgfqpoint{4.907782in}{3.291625in}}%
\pgfpathlineto{\pgfqpoint{4.912291in}{3.258001in}}%
\pgfpathlineto{\pgfqpoint{4.916800in}{3.204872in}}%
\pgfpathlineto{\pgfqpoint{4.921309in}{3.308234in}}%
\pgfpathlineto{\pgfqpoint{4.930327in}{3.216352in}}%
\pgfpathlineto{\pgfqpoint{4.934836in}{3.292341in}}%
\pgfpathlineto{\pgfqpoint{4.939345in}{3.253609in}}%
\pgfpathlineto{\pgfqpoint{4.943855in}{3.268446in}}%
\pgfpathlineto{\pgfqpoint{4.948364in}{3.348672in}}%
\pgfpathlineto{\pgfqpoint{4.952873in}{3.250219in}}%
\pgfpathlineto{\pgfqpoint{4.957382in}{3.229681in}}%
\pgfpathlineto{\pgfqpoint{4.961891in}{3.337687in}}%
\pgfpathlineto{\pgfqpoint{4.966400in}{3.260466in}}%
\pgfpathlineto{\pgfqpoint{4.970909in}{3.377453in}}%
\pgfpathlineto{\pgfqpoint{4.975418in}{3.391057in}}%
\pgfpathlineto{\pgfqpoint{4.979927in}{3.275875in}}%
\pgfpathlineto{\pgfqpoint{4.984436in}{3.381163in}}%
\pgfpathlineto{\pgfqpoint{4.988945in}{3.293100in}}%
\pgfpathlineto{\pgfqpoint{4.993455in}{3.389571in}}%
\pgfpathlineto{\pgfqpoint{4.997964in}{3.324678in}}%
\pgfpathlineto{\pgfqpoint{5.002473in}{3.357862in}}%
\pgfpathlineto{\pgfqpoint{5.006982in}{3.401800in}}%
\pgfpathlineto{\pgfqpoint{5.011491in}{3.201119in}}%
\pgfpathlineto{\pgfqpoint{5.016000in}{3.468510in}}%
\pgfpathlineto{\pgfqpoint{5.020509in}{3.401921in}}%
\pgfpathlineto{\pgfqpoint{5.025018in}{3.294960in}}%
\pgfpathlineto{\pgfqpoint{5.029527in}{3.423647in}}%
\pgfpathlineto{\pgfqpoint{5.034036in}{3.405696in}}%
\pgfpathlineto{\pgfqpoint{5.038545in}{3.382153in}}%
\pgfpathlineto{\pgfqpoint{5.043055in}{3.405124in}}%
\pgfpathlineto{\pgfqpoint{5.047564in}{3.419740in}}%
\pgfpathlineto{\pgfqpoint{5.052073in}{3.385257in}}%
\pgfpathlineto{\pgfqpoint{5.056582in}{3.316115in}}%
\pgfpathlineto{\pgfqpoint{5.061091in}{3.307254in}}%
\pgfpathlineto{\pgfqpoint{5.065600in}{3.462269in}}%
\pgfpathlineto{\pgfqpoint{5.070109in}{3.379170in}}%
\pgfpathlineto{\pgfqpoint{5.074618in}{3.436140in}}%
\pgfpathlineto{\pgfqpoint{5.079127in}{3.458461in}}%
\pgfpathlineto{\pgfqpoint{5.083636in}{3.460178in}}%
\pgfpathlineto{\pgfqpoint{5.088145in}{3.467497in}}%
\pgfpathlineto{\pgfqpoint{5.092655in}{3.525699in}}%
\pgfpathlineto{\pgfqpoint{5.097164in}{3.543034in}}%
\pgfpathlineto{\pgfqpoint{5.101673in}{3.483467in}}%
\pgfpathlineto{\pgfqpoint{5.106182in}{3.390287in}}%
\pgfpathlineto{\pgfqpoint{5.110691in}{3.422272in}}%
\pgfpathlineto{\pgfqpoint{5.115200in}{3.524191in}}%
\pgfpathlineto{\pgfqpoint{5.119709in}{3.382384in}}%
\pgfpathlineto{\pgfqpoint{5.128727in}{3.539061in}}%
\pgfpathlineto{\pgfqpoint{5.133236in}{3.502530in}}%
\pgfpathlineto{\pgfqpoint{5.137745in}{3.456656in}}%
\pgfpathlineto{\pgfqpoint{5.142255in}{3.527042in}}%
\pgfpathlineto{\pgfqpoint{5.146764in}{3.509520in}}%
\pgfpathlineto{\pgfqpoint{5.151273in}{3.512579in}}%
\pgfpathlineto{\pgfqpoint{5.155782in}{3.488949in}}%
\pgfpathlineto{\pgfqpoint{5.164800in}{3.639230in}}%
\pgfpathlineto{\pgfqpoint{5.169309in}{3.555207in}}%
\pgfpathlineto{\pgfqpoint{5.173818in}{3.541141in}}%
\pgfpathlineto{\pgfqpoint{5.178327in}{3.566896in}}%
\pgfpathlineto{\pgfqpoint{5.182836in}{3.571695in}}%
\pgfpathlineto{\pgfqpoint{5.187345in}{3.543871in}}%
\pgfpathlineto{\pgfqpoint{5.191855in}{3.596437in}}%
\pgfpathlineto{\pgfqpoint{5.196364in}{3.539600in}}%
\pgfpathlineto{\pgfqpoint{5.200873in}{3.612122in}}%
\pgfpathlineto{\pgfqpoint{5.205382in}{3.756570in}}%
\pgfpathlineto{\pgfqpoint{5.209891in}{3.657622in}}%
\pgfpathlineto{\pgfqpoint{5.214400in}{3.598198in}}%
\pgfpathlineto{\pgfqpoint{5.218909in}{3.620674in}}%
\pgfpathlineto{\pgfqpoint{5.223418in}{3.611934in}}%
\pgfpathlineto{\pgfqpoint{5.227927in}{3.521748in}}%
\pgfpathlineto{\pgfqpoint{5.232436in}{3.696717in}}%
\pgfpathlineto{\pgfqpoint{5.236945in}{3.694846in}}%
\pgfpathlineto{\pgfqpoint{5.241455in}{3.591473in}}%
\pgfpathlineto{\pgfqpoint{5.245964in}{3.596691in}}%
\pgfpathlineto{\pgfqpoint{5.250473in}{3.776205in}}%
\pgfpathlineto{\pgfqpoint{5.254982in}{3.606222in}}%
\pgfpathlineto{\pgfqpoint{5.259491in}{3.586289in}}%
\pgfpathlineto{\pgfqpoint{5.264000in}{3.737100in}}%
\pgfpathlineto{\pgfqpoint{5.268509in}{3.655091in}}%
\pgfpathlineto{\pgfqpoint{5.273018in}{3.636864in}}%
\pgfpathlineto{\pgfqpoint{5.277527in}{3.749614in}}%
\pgfpathlineto{\pgfqpoint{5.282036in}{3.677489in}}%
\pgfpathlineto{\pgfqpoint{5.286545in}{3.721647in}}%
\pgfpathlineto{\pgfqpoint{5.291055in}{3.718113in}}%
\pgfpathlineto{\pgfqpoint{5.295564in}{3.700602in}}%
\pgfpathlineto{\pgfqpoint{5.300073in}{3.731332in}}%
\pgfpathlineto{\pgfqpoint{5.304582in}{3.745509in}}%
\pgfpathlineto{\pgfqpoint{5.309091in}{3.741194in}}%
\pgfpathlineto{\pgfqpoint{5.313600in}{3.772771in}}%
\pgfpathlineto{\pgfqpoint{5.318109in}{3.703222in}}%
\pgfpathlineto{\pgfqpoint{5.322618in}{3.773575in}}%
\pgfpathlineto{\pgfqpoint{5.327127in}{3.700932in}}%
\pgfpathlineto{\pgfqpoint{5.336145in}{3.846250in}}%
\pgfpathlineto{\pgfqpoint{5.340655in}{3.679943in}}%
\pgfpathlineto{\pgfqpoint{5.345164in}{3.756130in}}%
\pgfpathlineto{\pgfqpoint{5.349673in}{3.759916in}}%
\pgfpathlineto{\pgfqpoint{5.354182in}{3.871180in}}%
\pgfpathlineto{\pgfqpoint{5.358691in}{3.936140in}}%
\pgfpathlineto{\pgfqpoint{5.363200in}{3.834749in}}%
\pgfpathlineto{\pgfqpoint{5.367709in}{3.829807in}}%
\pgfpathlineto{\pgfqpoint{5.372218in}{3.845832in}}%
\pgfpathlineto{\pgfqpoint{5.376727in}{3.830225in}}%
\pgfpathlineto{\pgfqpoint{5.381236in}{3.753367in}}%
\pgfpathlineto{\pgfqpoint{5.385745in}{3.924649in}}%
\pgfpathlineto{\pgfqpoint{5.390255in}{3.696431in}}%
\pgfpathlineto{\pgfqpoint{5.394764in}{3.848815in}}%
\pgfpathlineto{\pgfqpoint{5.399273in}{3.811492in}}%
\pgfpathlineto{\pgfqpoint{5.403782in}{3.877608in}}%
\pgfpathlineto{\pgfqpoint{5.408291in}{3.849112in}}%
\pgfpathlineto{\pgfqpoint{5.412800in}{3.834870in}}%
\pgfpathlineto{\pgfqpoint{5.417309in}{3.928557in}}%
\pgfpathlineto{\pgfqpoint{5.421818in}{3.831326in}}%
\pgfpathlineto{\pgfqpoint{5.426327in}{3.930196in}}%
\pgfpathlineto{\pgfqpoint{5.430836in}{3.926212in}}%
\pgfpathlineto{\pgfqpoint{5.435345in}{3.972978in}}%
\pgfpathlineto{\pgfqpoint{5.439855in}{3.851269in}}%
\pgfpathlineto{\pgfqpoint{5.444364in}{3.696057in}}%
\pgfpathlineto{\pgfqpoint{5.448873in}{3.957173in}}%
\pgfpathlineto{\pgfqpoint{5.453382in}{3.884619in}}%
\pgfpathlineto{\pgfqpoint{5.457891in}{3.890210in}}%
\pgfpathlineto{\pgfqpoint{5.466909in}{4.033778in}}%
\pgfpathlineto{\pgfqpoint{5.475927in}{3.860790in}}%
\pgfpathlineto{\pgfqpoint{5.480436in}{3.994859in}}%
\pgfpathlineto{\pgfqpoint{5.484945in}{4.017180in}}%
\pgfpathlineto{\pgfqpoint{5.489455in}{4.029463in}}%
\pgfpathlineto{\pgfqpoint{5.493964in}{3.895779in}}%
\pgfpathlineto{\pgfqpoint{5.498473in}{3.917605in}}%
\pgfpathlineto{\pgfqpoint{5.502982in}{3.978416in}}%
\pgfpathlineto{\pgfqpoint{5.507491in}{3.980595in}}%
\pgfpathlineto{\pgfqpoint{5.512000in}{4.056000in}}%
\pgfpathlineto{\pgfqpoint{5.516509in}{3.928424in}}%
\pgfpathlineto{\pgfqpoint{5.521018in}{3.974288in}}%
\pgfpathlineto{\pgfqpoint{5.525527in}{3.965263in}}%
\pgfpathlineto{\pgfqpoint{5.530036in}{3.870465in}}%
\pgfpathlineto{\pgfqpoint{5.534545in}{4.016993in}}%
\pgfpathlineto{\pgfqpoint{5.534545in}{4.016993in}}%
\pgfusepath{stroke}%
\end{pgfscope}%
\begin{pgfscope}%
\pgfpathrectangle{\pgfqpoint{0.800000in}{0.528000in}}{\pgfqpoint{4.960000in}{3.696000in}}%
\pgfusepath{clip}%
\pgfsetrectcap%
\pgfsetroundjoin%
\pgfsetlinewidth{1.505625pt}%
\definecolor{currentstroke}{rgb}{0.000000,1.000000,1.000000}%
\pgfsetstrokecolor{currentstroke}%
\pgfsetdash{}{0pt}%
\pgfpathmoveto{\pgfqpoint{1.025455in}{0.696000in}}%
\pgfpathlineto{\pgfqpoint{1.129164in}{0.696165in}}%
\pgfpathlineto{\pgfqpoint{1.336582in}{0.696583in}}%
\pgfpathlineto{\pgfqpoint{1.395200in}{0.696671in}}%
\pgfpathlineto{\pgfqpoint{2.044509in}{0.697728in}}%
\pgfpathlineto{\pgfqpoint{2.112145in}{0.697827in}}%
\pgfpathlineto{\pgfqpoint{2.130182in}{0.697761in}}%
\pgfpathlineto{\pgfqpoint{2.265455in}{0.697992in}}%
\pgfpathlineto{\pgfqpoint{3.379200in}{0.699962in}}%
\pgfpathlineto{\pgfqpoint{3.392727in}{0.699830in}}%
\pgfpathlineto{\pgfqpoint{4.465891in}{0.701492in}}%
\pgfpathlineto{\pgfqpoint{4.483927in}{0.701547in}}%
\pgfpathlineto{\pgfqpoint{4.605673in}{0.701756in}}%
\pgfpathlineto{\pgfqpoint{4.650764in}{0.701756in}}%
\pgfpathlineto{\pgfqpoint{4.704873in}{0.702054in}}%
\pgfpathlineto{\pgfqpoint{4.727418in}{0.701613in}}%
\pgfpathlineto{\pgfqpoint{4.754473in}{0.701932in}}%
\pgfpathlineto{\pgfqpoint{4.808582in}{0.701932in}}%
\pgfpathlineto{\pgfqpoint{5.534545in}{0.703033in}}%
\pgfpathlineto{\pgfqpoint{5.534545in}{0.703033in}}%
\pgfusepath{stroke}%
\end{pgfscope}%
\begin{pgfscope}%
\pgfpathrectangle{\pgfqpoint{0.800000in}{0.528000in}}{\pgfqpoint{4.960000in}{3.696000in}}%
\pgfusepath{clip}%
\pgfsetrectcap%
\pgfsetroundjoin%
\pgfsetlinewidth{1.505625pt}%
\definecolor{currentstroke}{rgb}{1.000000,0.000000,1.000000}%
\pgfsetstrokecolor{currentstroke}%
\pgfsetdash{}{0pt}%
\pgfpathmoveto{\pgfqpoint{1.025455in}{0.697794in}}%
\pgfpathlineto{\pgfqpoint{1.115636in}{0.698476in}}%
\pgfpathlineto{\pgfqpoint{1.300509in}{0.699577in}}%
\pgfpathlineto{\pgfqpoint{1.498909in}{0.700953in}}%
\pgfpathlineto{\pgfqpoint{1.525964in}{0.701382in}}%
\pgfpathlineto{\pgfqpoint{1.602618in}{0.701778in}}%
\pgfpathlineto{\pgfqpoint{1.629673in}{0.702010in}}%
\pgfpathlineto{\pgfqpoint{1.841600in}{0.703506in}}%
\pgfpathlineto{\pgfqpoint{2.220364in}{0.705972in}}%
\pgfpathlineto{\pgfqpoint{2.229382in}{0.706126in}}%
\pgfpathlineto{\pgfqpoint{2.324073in}{0.706896in}}%
\pgfpathlineto{\pgfqpoint{2.563055in}{0.709131in}}%
\pgfpathlineto{\pgfqpoint{2.572073in}{0.708834in}}%
\pgfpathlineto{\pgfqpoint{2.653236in}{0.709549in}}%
\pgfpathlineto{\pgfqpoint{2.693818in}{0.709692in}}%
\pgfpathlineto{\pgfqpoint{2.702836in}{0.710110in}}%
\pgfpathlineto{\pgfqpoint{2.716364in}{0.710033in}}%
\pgfpathlineto{\pgfqpoint{2.752436in}{0.709813in}}%
\pgfpathlineto{\pgfqpoint{2.765964in}{0.710143in}}%
\pgfpathlineto{\pgfqpoint{2.784000in}{0.710407in}}%
\pgfpathlineto{\pgfqpoint{2.811055in}{0.710826in}}%
\pgfpathlineto{\pgfqpoint{2.829091in}{0.710220in}}%
\pgfpathlineto{\pgfqpoint{2.838109in}{0.711189in}}%
\pgfpathlineto{\pgfqpoint{2.937309in}{0.711200in}}%
\pgfpathlineto{\pgfqpoint{2.946327in}{0.712113in}}%
\pgfpathlineto{\pgfqpoint{2.959855in}{0.711926in}}%
\pgfpathlineto{\pgfqpoint{2.986909in}{0.711772in}}%
\pgfpathlineto{\pgfqpoint{2.995927in}{0.712378in}}%
\pgfpathlineto{\pgfqpoint{3.009455in}{0.712157in}}%
\pgfpathlineto{\pgfqpoint{3.045527in}{0.712961in}}%
\pgfpathlineto{\pgfqpoint{3.068073in}{0.712212in}}%
\pgfpathlineto{\pgfqpoint{3.072582in}{0.713016in}}%
\pgfpathlineto{\pgfqpoint{3.081600in}{0.712543in}}%
\pgfpathlineto{\pgfqpoint{3.113164in}{0.713148in}}%
\pgfpathlineto{\pgfqpoint{3.135709in}{0.713093in}}%
\pgfpathlineto{\pgfqpoint{3.140218in}{0.713720in}}%
\pgfpathlineto{\pgfqpoint{3.149236in}{0.713511in}}%
\pgfpathlineto{\pgfqpoint{3.198836in}{0.714007in}}%
\pgfpathlineto{\pgfqpoint{3.212364in}{0.713742in}}%
\pgfpathlineto{\pgfqpoint{3.230400in}{0.713951in}}%
\pgfpathlineto{\pgfqpoint{3.234909in}{0.713335in}}%
\pgfpathlineto{\pgfqpoint{3.243927in}{0.714117in}}%
\pgfpathlineto{\pgfqpoint{3.252945in}{0.714150in}}%
\pgfpathlineto{\pgfqpoint{3.280000in}{0.714403in}}%
\pgfpathlineto{\pgfqpoint{3.293527in}{0.714238in}}%
\pgfpathlineto{\pgfqpoint{3.307055in}{0.714799in}}%
\pgfpathlineto{\pgfqpoint{3.347636in}{0.715030in}}%
\pgfpathlineto{\pgfqpoint{3.365673in}{0.715195in}}%
\pgfpathlineto{\pgfqpoint{3.388218in}{0.714931in}}%
\pgfpathlineto{\pgfqpoint{3.406255in}{0.715933in}}%
\pgfpathlineto{\pgfqpoint{3.419782in}{0.715217in}}%
\pgfpathlineto{\pgfqpoint{3.428800in}{0.715503in}}%
\pgfpathlineto{\pgfqpoint{3.464873in}{0.715481in}}%
\pgfpathlineto{\pgfqpoint{3.473891in}{0.716065in}}%
\pgfpathlineto{\pgfqpoint{3.487418in}{0.715657in}}%
\pgfpathlineto{\pgfqpoint{3.496436in}{0.716274in}}%
\pgfpathlineto{\pgfqpoint{3.518982in}{0.716230in}}%
\pgfpathlineto{\pgfqpoint{3.528000in}{0.716527in}}%
\pgfpathlineto{\pgfqpoint{3.555055in}{0.716296in}}%
\pgfpathlineto{\pgfqpoint{3.568582in}{0.716813in}}%
\pgfpathlineto{\pgfqpoint{3.586618in}{0.716428in}}%
\pgfpathlineto{\pgfqpoint{3.744436in}{0.718299in}}%
\pgfpathlineto{\pgfqpoint{3.748945in}{0.717562in}}%
\pgfpathlineto{\pgfqpoint{3.757964in}{0.718112in}}%
\pgfpathlineto{\pgfqpoint{3.839127in}{0.718915in}}%
\pgfpathlineto{\pgfqpoint{3.848145in}{0.718629in}}%
\pgfpathlineto{\pgfqpoint{3.857164in}{0.719191in}}%
\pgfpathlineto{\pgfqpoint{3.870691in}{0.718860in}}%
\pgfpathlineto{\pgfqpoint{3.884218in}{0.719422in}}%
\pgfpathlineto{\pgfqpoint{3.911273in}{0.719510in}}%
\pgfpathlineto{\pgfqpoint{3.924800in}{0.719708in}}%
\pgfpathlineto{\pgfqpoint{3.938327in}{0.720060in}}%
\pgfpathlineto{\pgfqpoint{3.942836in}{0.720368in}}%
\pgfpathlineto{\pgfqpoint{3.951855in}{0.719191in}}%
\pgfpathlineto{\pgfqpoint{4.001455in}{0.720225in}}%
\pgfpathlineto{\pgfqpoint{4.028509in}{0.720423in}}%
\pgfpathlineto{\pgfqpoint{4.114182in}{0.721315in}}%
\pgfpathlineto{\pgfqpoint{4.141236in}{0.721579in}}%
\pgfpathlineto{\pgfqpoint{4.150255in}{0.721139in}}%
\pgfpathlineto{\pgfqpoint{4.168291in}{0.721590in}}%
\pgfpathlineto{\pgfqpoint{4.177309in}{0.721161in}}%
\pgfpathlineto{\pgfqpoint{4.199855in}{0.721821in}}%
\pgfpathlineto{\pgfqpoint{4.208873in}{0.721601in}}%
\pgfpathlineto{\pgfqpoint{4.226909in}{0.721942in}}%
\pgfpathlineto{\pgfqpoint{4.235927in}{0.721469in}}%
\pgfpathlineto{\pgfqpoint{4.244945in}{0.722008in}}%
\pgfpathlineto{\pgfqpoint{4.321600in}{0.722669in}}%
\pgfpathlineto{\pgfqpoint{4.326109in}{0.722019in}}%
\pgfpathlineto{\pgfqpoint{4.330618in}{0.723032in}}%
\pgfpathlineto{\pgfqpoint{4.335127in}{0.722184in}}%
\pgfpathlineto{\pgfqpoint{4.348655in}{0.723065in}}%
\pgfpathlineto{\pgfqpoint{4.398255in}{0.723087in}}%
\pgfpathlineto{\pgfqpoint{4.420800in}{0.723571in}}%
\pgfpathlineto{\pgfqpoint{4.461382in}{0.723835in}}%
\pgfpathlineto{\pgfqpoint{4.465891in}{0.723318in}}%
\pgfpathlineto{\pgfqpoint{4.470400in}{0.724198in}}%
\pgfpathlineto{\pgfqpoint{4.492945in}{0.723736in}}%
\pgfpathlineto{\pgfqpoint{4.497455in}{0.724529in}}%
\pgfpathlineto{\pgfqpoint{4.506473in}{0.724220in}}%
\pgfpathlineto{\pgfqpoint{4.533527in}{0.724958in}}%
\pgfpathlineto{\pgfqpoint{4.542545in}{0.724320in}}%
\pgfpathlineto{\pgfqpoint{4.547055in}{0.725101in}}%
\pgfpathlineto{\pgfqpoint{4.560582in}{0.724198in}}%
\pgfpathlineto{\pgfqpoint{4.578618in}{0.724914in}}%
\pgfpathlineto{\pgfqpoint{4.596655in}{0.724705in}}%
\pgfpathlineto{\pgfqpoint{4.610182in}{0.724936in}}%
\pgfpathlineto{\pgfqpoint{4.619200in}{0.725035in}}%
\pgfpathlineto{\pgfqpoint{4.632727in}{0.725222in}}%
\pgfpathlineto{\pgfqpoint{4.668800in}{0.725398in}}%
\pgfpathlineto{\pgfqpoint{4.677818in}{0.724617in}}%
\pgfpathlineto{\pgfqpoint{4.682327in}{0.726268in}}%
\pgfpathlineto{\pgfqpoint{4.686836in}{0.724881in}}%
\pgfpathlineto{\pgfqpoint{4.695855in}{0.725982in}}%
\pgfpathlineto{\pgfqpoint{4.713891in}{0.725464in}}%
\pgfpathlineto{\pgfqpoint{4.731927in}{0.726136in}}%
\pgfpathlineto{\pgfqpoint{4.808582in}{0.726631in}}%
\pgfpathlineto{\pgfqpoint{4.813091in}{0.726169in}}%
\pgfpathlineto{\pgfqpoint{4.817600in}{0.726939in}}%
\pgfpathlineto{\pgfqpoint{4.822109in}{0.726444in}}%
\pgfpathlineto{\pgfqpoint{4.831127in}{0.727357in}}%
\pgfpathlineto{\pgfqpoint{4.835636in}{0.726510in}}%
\pgfpathlineto{\pgfqpoint{4.840145in}{0.727126in}}%
\pgfpathlineto{\pgfqpoint{4.844655in}{0.726565in}}%
\pgfpathlineto{\pgfqpoint{4.849164in}{0.727137in}}%
\pgfpathlineto{\pgfqpoint{4.853673in}{0.726521in}}%
\pgfpathlineto{\pgfqpoint{4.858182in}{0.727412in}}%
\pgfpathlineto{\pgfqpoint{4.871709in}{0.727313in}}%
\pgfpathlineto{\pgfqpoint{4.876218in}{0.727996in}}%
\pgfpathlineto{\pgfqpoint{4.885236in}{0.727082in}}%
\pgfpathlineto{\pgfqpoint{4.889745in}{0.727732in}}%
\pgfpathlineto{\pgfqpoint{4.894255in}{0.726763in}}%
\pgfpathlineto{\pgfqpoint{4.903273in}{0.727643in}}%
\pgfpathlineto{\pgfqpoint{4.912291in}{0.727390in}}%
\pgfpathlineto{\pgfqpoint{4.916800in}{0.726917in}}%
\pgfpathlineto{\pgfqpoint{4.921309in}{0.728062in}}%
\pgfpathlineto{\pgfqpoint{4.948364in}{0.727666in}}%
\pgfpathlineto{\pgfqpoint{4.957382in}{0.728095in}}%
\pgfpathlineto{\pgfqpoint{4.966400in}{0.727324in}}%
\pgfpathlineto{\pgfqpoint{4.975418in}{0.728161in}}%
\pgfpathlineto{\pgfqpoint{4.979927in}{0.727666in}}%
\pgfpathlineto{\pgfqpoint{4.988945in}{0.728491in}}%
\pgfpathlineto{\pgfqpoint{4.997964in}{0.728425in}}%
\pgfpathlineto{\pgfqpoint{5.029527in}{0.728667in}}%
\pgfpathlineto{\pgfqpoint{5.034036in}{0.729283in}}%
\pgfpathlineto{\pgfqpoint{5.038545in}{0.728326in}}%
\pgfpathlineto{\pgfqpoint{5.056582in}{0.728843in}}%
\pgfpathlineto{\pgfqpoint{5.065600in}{0.728491in}}%
\pgfpathlineto{\pgfqpoint{5.079127in}{0.728964in}}%
\pgfpathlineto{\pgfqpoint{5.088145in}{0.729030in}}%
\pgfpathlineto{\pgfqpoint{5.092655in}{0.729647in}}%
\pgfpathlineto{\pgfqpoint{5.106182in}{0.728920in}}%
\pgfpathlineto{\pgfqpoint{5.173818in}{0.729713in}}%
\pgfpathlineto{\pgfqpoint{5.182836in}{0.729206in}}%
\pgfpathlineto{\pgfqpoint{5.187345in}{0.730098in}}%
\pgfpathlineto{\pgfqpoint{5.209891in}{0.729658in}}%
\pgfpathlineto{\pgfqpoint{5.223418in}{0.730197in}}%
\pgfpathlineto{\pgfqpoint{5.245964in}{0.730560in}}%
\pgfpathlineto{\pgfqpoint{5.250473in}{0.731276in}}%
\pgfpathlineto{\pgfqpoint{5.268509in}{0.730329in}}%
\pgfpathlineto{\pgfqpoint{5.277527in}{0.730483in}}%
\pgfpathlineto{\pgfqpoint{5.291055in}{0.730461in}}%
\pgfpathlineto{\pgfqpoint{5.295564in}{0.731320in}}%
\pgfpathlineto{\pgfqpoint{5.304582in}{0.730527in}}%
\pgfpathlineto{\pgfqpoint{5.313600in}{0.730890in}}%
\pgfpathlineto{\pgfqpoint{5.331636in}{0.731089in}}%
\pgfpathlineto{\pgfqpoint{5.336145in}{0.731848in}}%
\pgfpathlineto{\pgfqpoint{5.354182in}{0.731066in}}%
\pgfpathlineto{\pgfqpoint{5.430836in}{0.732431in}}%
\pgfpathlineto{\pgfqpoint{5.444364in}{0.732211in}}%
\pgfpathlineto{\pgfqpoint{5.457891in}{0.731298in}}%
\pgfpathlineto{\pgfqpoint{5.462400in}{0.732563in}}%
\pgfpathlineto{\pgfqpoint{5.475927in}{0.731782in}}%
\pgfpathlineto{\pgfqpoint{5.480436in}{0.732563in}}%
\pgfpathlineto{\pgfqpoint{5.493964in}{0.732398in}}%
\pgfpathlineto{\pgfqpoint{5.507491in}{0.732156in}}%
\pgfpathlineto{\pgfqpoint{5.516509in}{0.732530in}}%
\pgfpathlineto{\pgfqpoint{5.534545in}{0.731881in}}%
\pgfpathlineto{\pgfqpoint{5.534545in}{0.731881in}}%
\pgfusepath{stroke}%
\end{pgfscope}%
\begin{pgfscope}%
\pgfsetrectcap%
\pgfsetmiterjoin%
\pgfsetlinewidth{0.803000pt}%
\definecolor{currentstroke}{rgb}{0.000000,0.000000,0.000000}%
\pgfsetstrokecolor{currentstroke}%
\pgfsetdash{}{0pt}%
\pgfpathmoveto{\pgfqpoint{0.800000in}{0.528000in}}%
\pgfpathlineto{\pgfqpoint{0.800000in}{4.224000in}}%
\pgfusepath{stroke}%
\end{pgfscope}%
\begin{pgfscope}%
\pgfsetrectcap%
\pgfsetmiterjoin%
\pgfsetlinewidth{0.803000pt}%
\definecolor{currentstroke}{rgb}{0.000000,0.000000,0.000000}%
\pgfsetstrokecolor{currentstroke}%
\pgfsetdash{}{0pt}%
\pgfpathmoveto{\pgfqpoint{5.760000in}{0.528000in}}%
\pgfpathlineto{\pgfqpoint{5.760000in}{4.224000in}}%
\pgfusepath{stroke}%
\end{pgfscope}%
\begin{pgfscope}%
\pgfsetrectcap%
\pgfsetmiterjoin%
\pgfsetlinewidth{0.803000pt}%
\definecolor{currentstroke}{rgb}{0.000000,0.000000,0.000000}%
\pgfsetstrokecolor{currentstroke}%
\pgfsetdash{}{0pt}%
\pgfpathmoveto{\pgfqpoint{0.800000in}{0.528000in}}%
\pgfpathlineto{\pgfqpoint{5.760000in}{0.528000in}}%
\pgfusepath{stroke}%
\end{pgfscope}%
\begin{pgfscope}%
\pgfsetrectcap%
\pgfsetmiterjoin%
\pgfsetlinewidth{0.803000pt}%
\definecolor{currentstroke}{rgb}{0.000000,0.000000,0.000000}%
\pgfsetstrokecolor{currentstroke}%
\pgfsetdash{}{0pt}%
\pgfpathmoveto{\pgfqpoint{0.800000in}{4.224000in}}%
\pgfpathlineto{\pgfqpoint{5.760000in}{4.224000in}}%
\pgfusepath{stroke}%
\end{pgfscope}%
\begin{pgfscope}%
\definecolor{textcolor}{rgb}{0.000000,0.000000,0.000000}%
\pgfsetstrokecolor{textcolor}%
\pgfsetfillcolor{textcolor}%
\pgftext[x=3.280000in,y=4.307333in,,base]{\color{textcolor}\ttfamily\fontsize{12.000000}{14.400000}\selectfont Swaps vs Input size}%
\end{pgfscope}%
\begin{pgfscope}%
\pgfsetbuttcap%
\pgfsetmiterjoin%
\definecolor{currentfill}{rgb}{1.000000,1.000000,1.000000}%
\pgfsetfillcolor{currentfill}%
\pgfsetfillopacity{0.800000}%
\pgfsetlinewidth{1.003750pt}%
\definecolor{currentstroke}{rgb}{0.800000,0.800000,0.800000}%
\pgfsetstrokecolor{currentstroke}%
\pgfsetstrokeopacity{0.800000}%
\pgfsetdash{}{0pt}%
\pgfpathmoveto{\pgfqpoint{0.897222in}{3.088923in}}%
\pgfpathlineto{\pgfqpoint{2.094230in}{3.088923in}}%
\pgfpathquadraticcurveto{\pgfqpoint{2.122008in}{3.088923in}}{\pgfqpoint{2.122008in}{3.116701in}}%
\pgfpathlineto{\pgfqpoint{2.122008in}{4.126778in}}%
\pgfpathquadraticcurveto{\pgfqpoint{2.122008in}{4.154556in}}{\pgfqpoint{2.094230in}{4.154556in}}%
\pgfpathlineto{\pgfqpoint{0.897222in}{4.154556in}}%
\pgfpathquadraticcurveto{\pgfqpoint{0.869444in}{4.154556in}}{\pgfqpoint{0.869444in}{4.126778in}}%
\pgfpathlineto{\pgfqpoint{0.869444in}{3.116701in}}%
\pgfpathquadraticcurveto{\pgfqpoint{0.869444in}{3.088923in}}{\pgfqpoint{0.897222in}{3.088923in}}%
\pgfpathlineto{\pgfqpoint{0.897222in}{3.088923in}}%
\pgfpathclose%
\pgfusepath{stroke,fill}%
\end{pgfscope}%
\begin{pgfscope}%
\pgfsetrectcap%
\pgfsetroundjoin%
\pgfsetlinewidth{1.505625pt}%
\definecolor{currentstroke}{rgb}{1.000000,0.000000,0.000000}%
\pgfsetstrokecolor{currentstroke}%
\pgfsetdash{}{0pt}%
\pgfpathmoveto{\pgfqpoint{0.925000in}{4.041342in}}%
\pgfpathlineto{\pgfqpoint{1.063889in}{4.041342in}}%
\pgfpathlineto{\pgfqpoint{1.202778in}{4.041342in}}%
\pgfusepath{stroke}%
\end{pgfscope}%
\begin{pgfscope}%
\definecolor{textcolor}{rgb}{0.000000,0.000000,0.000000}%
\pgfsetstrokecolor{textcolor}%
\pgfsetfillcolor{textcolor}%
\pgftext[x=1.313889in,y=3.992731in,left,base]{\color{textcolor}\ttfamily\fontsize{10.000000}{12.000000}\selectfont Bubble}%
\end{pgfscope}%
\begin{pgfscope}%
\pgfsetrectcap%
\pgfsetroundjoin%
\pgfsetlinewidth{1.505625pt}%
\definecolor{currentstroke}{rgb}{0.486275,0.988235,0.000000}%
\pgfsetstrokecolor{currentstroke}%
\pgfsetdash{}{0pt}%
\pgfpathmoveto{\pgfqpoint{0.925000in}{3.836739in}}%
\pgfpathlineto{\pgfqpoint{1.063889in}{3.836739in}}%
\pgfpathlineto{\pgfqpoint{1.202778in}{3.836739in}}%
\pgfusepath{stroke}%
\end{pgfscope}%
\begin{pgfscope}%
\definecolor{textcolor}{rgb}{0.000000,0.000000,0.000000}%
\pgfsetstrokecolor{textcolor}%
\pgfsetfillcolor{textcolor}%
\pgftext[x=1.313889in,y=3.788128in,left,base]{\color{textcolor}\ttfamily\fontsize{10.000000}{12.000000}\selectfont Selection}%
\end{pgfscope}%
\begin{pgfscope}%
\pgfsetrectcap%
\pgfsetroundjoin%
\pgfsetlinewidth{1.505625pt}%
\definecolor{currentstroke}{rgb}{0.000000,1.000000,0.498039}%
\pgfsetstrokecolor{currentstroke}%
\pgfsetdash{}{0pt}%
\pgfpathmoveto{\pgfqpoint{0.925000in}{3.632136in}}%
\pgfpathlineto{\pgfqpoint{1.063889in}{3.632136in}}%
\pgfpathlineto{\pgfqpoint{1.202778in}{3.632136in}}%
\pgfusepath{stroke}%
\end{pgfscope}%
\begin{pgfscope}%
\definecolor{textcolor}{rgb}{0.000000,0.000000,0.000000}%
\pgfsetstrokecolor{textcolor}%
\pgfsetfillcolor{textcolor}%
\pgftext[x=1.313889in,y=3.583525in,left,base]{\color{textcolor}\ttfamily\fontsize{10.000000}{12.000000}\selectfont Insertion}%
\end{pgfscope}%
\begin{pgfscope}%
\pgfsetrectcap%
\pgfsetroundjoin%
\pgfsetlinewidth{1.505625pt}%
\definecolor{currentstroke}{rgb}{0.000000,1.000000,1.000000}%
\pgfsetstrokecolor{currentstroke}%
\pgfsetdash{}{0pt}%
\pgfpathmoveto{\pgfqpoint{0.925000in}{3.427532in}}%
\pgfpathlineto{\pgfqpoint{1.063889in}{3.427532in}}%
\pgfpathlineto{\pgfqpoint{1.202778in}{3.427532in}}%
\pgfusepath{stroke}%
\end{pgfscope}%
\begin{pgfscope}%
\definecolor{textcolor}{rgb}{0.000000,0.000000,0.000000}%
\pgfsetstrokecolor{textcolor}%
\pgfsetfillcolor{textcolor}%
\pgftext[x=1.313889in,y=3.378921in,left,base]{\color{textcolor}\ttfamily\fontsize{10.000000}{12.000000}\selectfont Merge}%
\end{pgfscope}%
\begin{pgfscope}%
\pgfsetrectcap%
\pgfsetroundjoin%
\pgfsetlinewidth{1.505625pt}%
\definecolor{currentstroke}{rgb}{1.000000,0.000000,1.000000}%
\pgfsetstrokecolor{currentstroke}%
\pgfsetdash{}{0pt}%
\pgfpathmoveto{\pgfqpoint{0.925000in}{3.221980in}}%
\pgfpathlineto{\pgfqpoint{1.063889in}{3.221980in}}%
\pgfpathlineto{\pgfqpoint{1.202778in}{3.221980in}}%
\pgfusepath{stroke}%
\end{pgfscope}%
\begin{pgfscope}%
\definecolor{textcolor}{rgb}{0.000000,0.000000,0.000000}%
\pgfsetstrokecolor{textcolor}%
\pgfsetfillcolor{textcolor}%
\pgftext[x=1.313889in,y=3.173369in,left,base]{\color{textcolor}\ttfamily\fontsize{10.000000}{12.000000}\selectfont Quick}%
\end{pgfscope}%
\end{pgfpicture}%
\makeatother%
\endgroup%

%% Creator: Matplotlib, PGF backend
%%
%% To include the figure in your LaTeX document, write
%%   \input{<filename>.pgf}
%%
%% Make sure the required packages are loaded in your preamble
%%   \usepackage{pgf}
%%
%% Also ensure that all the required font packages are loaded; for instance,
%% the lmodern package is sometimes necessary when using math font.
%%   \usepackage{lmodern}
%%
%% Figures using additional raster images can only be included by \input if
%% they are in the same directory as the main LaTeX file. For loading figures
%% from other directories you can use the `import` package
%%   \usepackage{import}
%%
%% and then include the figures with
%%   \import{<path to file>}{<filename>.pgf}
%%
%% Matplotlib used the following preamble
%%   \usepackage{fontspec}
%%   \setmainfont{DejaVuSerif.ttf}[Path=\detokenize{/home/dbk/.local/lib/python3.10/site-packages/matplotlib/mpl-data/fonts/ttf/}]
%%   \setsansfont{DejaVuSans.ttf}[Path=\detokenize{/home/dbk/.local/lib/python3.10/site-packages/matplotlib/mpl-data/fonts/ttf/}]
%%   \setmonofont{DejaVuSansMono.ttf}[Path=\detokenize{/home/dbk/.local/lib/python3.10/site-packages/matplotlib/mpl-data/fonts/ttf/}]
%%
\begingroup%
\makeatletter%
\begin{pgfpicture}%
\pgfpathrectangle{\pgfpointorigin}{\pgfqpoint{6.400000in}{4.800000in}}%
\pgfusepath{use as bounding box, clip}%
\begin{pgfscope}%
\pgfsetbuttcap%
\pgfsetmiterjoin%
\definecolor{currentfill}{rgb}{1.000000,1.000000,1.000000}%
\pgfsetfillcolor{currentfill}%
\pgfsetlinewidth{0.000000pt}%
\definecolor{currentstroke}{rgb}{1.000000,1.000000,1.000000}%
\pgfsetstrokecolor{currentstroke}%
\pgfsetdash{}{0pt}%
\pgfpathmoveto{\pgfqpoint{0.000000in}{0.000000in}}%
\pgfpathlineto{\pgfqpoint{6.400000in}{0.000000in}}%
\pgfpathlineto{\pgfqpoint{6.400000in}{4.800000in}}%
\pgfpathlineto{\pgfqpoint{0.000000in}{4.800000in}}%
\pgfpathlineto{\pgfqpoint{0.000000in}{0.000000in}}%
\pgfpathclose%
\pgfusepath{fill}%
\end{pgfscope}%
\begin{pgfscope}%
\pgfsetbuttcap%
\pgfsetmiterjoin%
\definecolor{currentfill}{rgb}{1.000000,1.000000,1.000000}%
\pgfsetfillcolor{currentfill}%
\pgfsetlinewidth{0.000000pt}%
\definecolor{currentstroke}{rgb}{0.000000,0.000000,0.000000}%
\pgfsetstrokecolor{currentstroke}%
\pgfsetstrokeopacity{0.000000}%
\pgfsetdash{}{0pt}%
\pgfpathmoveto{\pgfqpoint{0.800000in}{0.528000in}}%
\pgfpathlineto{\pgfqpoint{5.760000in}{0.528000in}}%
\pgfpathlineto{\pgfqpoint{5.760000in}{4.224000in}}%
\pgfpathlineto{\pgfqpoint{0.800000in}{4.224000in}}%
\pgfpathlineto{\pgfqpoint{0.800000in}{0.528000in}}%
\pgfpathclose%
\pgfusepath{fill}%
\end{pgfscope}%
\begin{pgfscope}%
\pgfsetbuttcap%
\pgfsetroundjoin%
\definecolor{currentfill}{rgb}{0.000000,0.000000,0.000000}%
\pgfsetfillcolor{currentfill}%
\pgfsetlinewidth{0.803000pt}%
\definecolor{currentstroke}{rgb}{0.000000,0.000000,0.000000}%
\pgfsetstrokecolor{currentstroke}%
\pgfsetdash{}{0pt}%
\pgfsys@defobject{currentmarker}{\pgfqpoint{0.000000in}{-0.048611in}}{\pgfqpoint{0.000000in}{0.000000in}}{%
\pgfpathmoveto{\pgfqpoint{0.000000in}{0.000000in}}%
\pgfpathlineto{\pgfqpoint{0.000000in}{-0.048611in}}%
\pgfusepath{stroke,fill}%
}%
\begin{pgfscope}%
\pgfsys@transformshift{1.020945in}{0.528000in}%
\pgfsys@useobject{currentmarker}{}%
\end{pgfscope}%
\end{pgfscope}%
\begin{pgfscope}%
\definecolor{textcolor}{rgb}{0.000000,0.000000,0.000000}%
\pgfsetstrokecolor{textcolor}%
\pgfsetfillcolor{textcolor}%
\pgftext[x=1.020945in,y=0.430778in,,top]{\color{textcolor}\ttfamily\fontsize{10.000000}{12.000000}\selectfont 0}%
\end{pgfscope}%
\begin{pgfscope}%
\pgfsetbuttcap%
\pgfsetroundjoin%
\definecolor{currentfill}{rgb}{0.000000,0.000000,0.000000}%
\pgfsetfillcolor{currentfill}%
\pgfsetlinewidth{0.803000pt}%
\definecolor{currentstroke}{rgb}{0.000000,0.000000,0.000000}%
\pgfsetstrokecolor{currentstroke}%
\pgfsetdash{}{0pt}%
\pgfsys@defobject{currentmarker}{\pgfqpoint{0.000000in}{-0.048611in}}{\pgfqpoint{0.000000in}{0.000000in}}{%
\pgfpathmoveto{\pgfqpoint{0.000000in}{0.000000in}}%
\pgfpathlineto{\pgfqpoint{0.000000in}{-0.048611in}}%
\pgfusepath{stroke,fill}%
}%
\begin{pgfscope}%
\pgfsys@transformshift{1.922764in}{0.528000in}%
\pgfsys@useobject{currentmarker}{}%
\end{pgfscope}%
\end{pgfscope}%
\begin{pgfscope}%
\definecolor{textcolor}{rgb}{0.000000,0.000000,0.000000}%
\pgfsetstrokecolor{textcolor}%
\pgfsetfillcolor{textcolor}%
\pgftext[x=1.922764in,y=0.430778in,,top]{\color{textcolor}\ttfamily\fontsize{10.000000}{12.000000}\selectfont 200}%
\end{pgfscope}%
\begin{pgfscope}%
\pgfsetbuttcap%
\pgfsetroundjoin%
\definecolor{currentfill}{rgb}{0.000000,0.000000,0.000000}%
\pgfsetfillcolor{currentfill}%
\pgfsetlinewidth{0.803000pt}%
\definecolor{currentstroke}{rgb}{0.000000,0.000000,0.000000}%
\pgfsetstrokecolor{currentstroke}%
\pgfsetdash{}{0pt}%
\pgfsys@defobject{currentmarker}{\pgfqpoint{0.000000in}{-0.048611in}}{\pgfqpoint{0.000000in}{0.000000in}}{%
\pgfpathmoveto{\pgfqpoint{0.000000in}{0.000000in}}%
\pgfpathlineto{\pgfqpoint{0.000000in}{-0.048611in}}%
\pgfusepath{stroke,fill}%
}%
\begin{pgfscope}%
\pgfsys@transformshift{2.824582in}{0.528000in}%
\pgfsys@useobject{currentmarker}{}%
\end{pgfscope}%
\end{pgfscope}%
\begin{pgfscope}%
\definecolor{textcolor}{rgb}{0.000000,0.000000,0.000000}%
\pgfsetstrokecolor{textcolor}%
\pgfsetfillcolor{textcolor}%
\pgftext[x=2.824582in,y=0.430778in,,top]{\color{textcolor}\ttfamily\fontsize{10.000000}{12.000000}\selectfont 400}%
\end{pgfscope}%
\begin{pgfscope}%
\pgfsetbuttcap%
\pgfsetroundjoin%
\definecolor{currentfill}{rgb}{0.000000,0.000000,0.000000}%
\pgfsetfillcolor{currentfill}%
\pgfsetlinewidth{0.803000pt}%
\definecolor{currentstroke}{rgb}{0.000000,0.000000,0.000000}%
\pgfsetstrokecolor{currentstroke}%
\pgfsetdash{}{0pt}%
\pgfsys@defobject{currentmarker}{\pgfqpoint{0.000000in}{-0.048611in}}{\pgfqpoint{0.000000in}{0.000000in}}{%
\pgfpathmoveto{\pgfqpoint{0.000000in}{0.000000in}}%
\pgfpathlineto{\pgfqpoint{0.000000in}{-0.048611in}}%
\pgfusepath{stroke,fill}%
}%
\begin{pgfscope}%
\pgfsys@transformshift{3.726400in}{0.528000in}%
\pgfsys@useobject{currentmarker}{}%
\end{pgfscope}%
\end{pgfscope}%
\begin{pgfscope}%
\definecolor{textcolor}{rgb}{0.000000,0.000000,0.000000}%
\pgfsetstrokecolor{textcolor}%
\pgfsetfillcolor{textcolor}%
\pgftext[x=3.726400in,y=0.430778in,,top]{\color{textcolor}\ttfamily\fontsize{10.000000}{12.000000}\selectfont 600}%
\end{pgfscope}%
\begin{pgfscope}%
\pgfsetbuttcap%
\pgfsetroundjoin%
\definecolor{currentfill}{rgb}{0.000000,0.000000,0.000000}%
\pgfsetfillcolor{currentfill}%
\pgfsetlinewidth{0.803000pt}%
\definecolor{currentstroke}{rgb}{0.000000,0.000000,0.000000}%
\pgfsetstrokecolor{currentstroke}%
\pgfsetdash{}{0pt}%
\pgfsys@defobject{currentmarker}{\pgfqpoint{0.000000in}{-0.048611in}}{\pgfqpoint{0.000000in}{0.000000in}}{%
\pgfpathmoveto{\pgfqpoint{0.000000in}{0.000000in}}%
\pgfpathlineto{\pgfqpoint{0.000000in}{-0.048611in}}%
\pgfusepath{stroke,fill}%
}%
\begin{pgfscope}%
\pgfsys@transformshift{4.628218in}{0.528000in}%
\pgfsys@useobject{currentmarker}{}%
\end{pgfscope}%
\end{pgfscope}%
\begin{pgfscope}%
\definecolor{textcolor}{rgb}{0.000000,0.000000,0.000000}%
\pgfsetstrokecolor{textcolor}%
\pgfsetfillcolor{textcolor}%
\pgftext[x=4.628218in,y=0.430778in,,top]{\color{textcolor}\ttfamily\fontsize{10.000000}{12.000000}\selectfont 800}%
\end{pgfscope}%
\begin{pgfscope}%
\pgfsetbuttcap%
\pgfsetroundjoin%
\definecolor{currentfill}{rgb}{0.000000,0.000000,0.000000}%
\pgfsetfillcolor{currentfill}%
\pgfsetlinewidth{0.803000pt}%
\definecolor{currentstroke}{rgb}{0.000000,0.000000,0.000000}%
\pgfsetstrokecolor{currentstroke}%
\pgfsetdash{}{0pt}%
\pgfsys@defobject{currentmarker}{\pgfqpoint{0.000000in}{-0.048611in}}{\pgfqpoint{0.000000in}{0.000000in}}{%
\pgfpathmoveto{\pgfqpoint{0.000000in}{0.000000in}}%
\pgfpathlineto{\pgfqpoint{0.000000in}{-0.048611in}}%
\pgfusepath{stroke,fill}%
}%
\begin{pgfscope}%
\pgfsys@transformshift{5.530036in}{0.528000in}%
\pgfsys@useobject{currentmarker}{}%
\end{pgfscope}%
\end{pgfscope}%
\begin{pgfscope}%
\definecolor{textcolor}{rgb}{0.000000,0.000000,0.000000}%
\pgfsetstrokecolor{textcolor}%
\pgfsetfillcolor{textcolor}%
\pgftext[x=5.530036in,y=0.430778in,,top]{\color{textcolor}\ttfamily\fontsize{10.000000}{12.000000}\selectfont 1000}%
\end{pgfscope}%
\begin{pgfscope}%
\definecolor{textcolor}{rgb}{0.000000,0.000000,0.000000}%
\pgfsetstrokecolor{textcolor}%
\pgfsetfillcolor{textcolor}%
\pgftext[x=3.280000in,y=0.240063in,,top]{\color{textcolor}\ttfamily\fontsize{10.000000}{12.000000}\selectfont Size of Array}%
\end{pgfscope}%
\begin{pgfscope}%
\pgfsetbuttcap%
\pgfsetroundjoin%
\definecolor{currentfill}{rgb}{0.000000,0.000000,0.000000}%
\pgfsetfillcolor{currentfill}%
\pgfsetlinewidth{0.803000pt}%
\definecolor{currentstroke}{rgb}{0.000000,0.000000,0.000000}%
\pgfsetstrokecolor{currentstroke}%
\pgfsetdash{}{0pt}%
\pgfsys@defobject{currentmarker}{\pgfqpoint{-0.048611in}{0.000000in}}{\pgfqpoint{-0.000000in}{0.000000in}}{%
\pgfpathmoveto{\pgfqpoint{-0.000000in}{0.000000in}}%
\pgfpathlineto{\pgfqpoint{-0.048611in}{0.000000in}}%
\pgfusepath{stroke,fill}%
}%
\begin{pgfscope}%
\pgfsys@transformshift{0.800000in}{0.695277in}%
\pgfsys@useobject{currentmarker}{}%
\end{pgfscope}%
\end{pgfscope}%
\begin{pgfscope}%
\definecolor{textcolor}{rgb}{0.000000,0.000000,0.000000}%
\pgfsetstrokecolor{textcolor}%
\pgfsetfillcolor{textcolor}%
\pgftext[x=0.619160in, y=0.642143in, left, base]{\color{textcolor}\ttfamily\fontsize{10.000000}{12.000000}\selectfont 0}%
\end{pgfscope}%
\begin{pgfscope}%
\pgfsetbuttcap%
\pgfsetroundjoin%
\definecolor{currentfill}{rgb}{0.000000,0.000000,0.000000}%
\pgfsetfillcolor{currentfill}%
\pgfsetlinewidth{0.803000pt}%
\definecolor{currentstroke}{rgb}{0.000000,0.000000,0.000000}%
\pgfsetstrokecolor{currentstroke}%
\pgfsetdash{}{0pt}%
\pgfsys@defobject{currentmarker}{\pgfqpoint{-0.048611in}{0.000000in}}{\pgfqpoint{-0.000000in}{0.000000in}}{%
\pgfpathmoveto{\pgfqpoint{-0.000000in}{0.000000in}}%
\pgfpathlineto{\pgfqpoint{-0.048611in}{0.000000in}}%
\pgfusepath{stroke,fill}%
}%
\begin{pgfscope}%
\pgfsys@transformshift{0.800000in}{1.251274in}%
\pgfsys@useobject{currentmarker}{}%
\end{pgfscope}%
\end{pgfscope}%
\begin{pgfscope}%
\definecolor{textcolor}{rgb}{0.000000,0.000000,0.000000}%
\pgfsetstrokecolor{textcolor}%
\pgfsetfillcolor{textcolor}%
\pgftext[x=0.201069in, y=1.198140in, left, base]{\color{textcolor}\ttfamily\fontsize{10.000000}{12.000000}\selectfont 100000}%
\end{pgfscope}%
\begin{pgfscope}%
\pgfsetbuttcap%
\pgfsetroundjoin%
\definecolor{currentfill}{rgb}{0.000000,0.000000,0.000000}%
\pgfsetfillcolor{currentfill}%
\pgfsetlinewidth{0.803000pt}%
\definecolor{currentstroke}{rgb}{0.000000,0.000000,0.000000}%
\pgfsetstrokecolor{currentstroke}%
\pgfsetdash{}{0pt}%
\pgfsys@defobject{currentmarker}{\pgfqpoint{-0.048611in}{0.000000in}}{\pgfqpoint{-0.000000in}{0.000000in}}{%
\pgfpathmoveto{\pgfqpoint{-0.000000in}{0.000000in}}%
\pgfpathlineto{\pgfqpoint{-0.048611in}{0.000000in}}%
\pgfusepath{stroke,fill}%
}%
\begin{pgfscope}%
\pgfsys@transformshift{0.800000in}{1.807271in}%
\pgfsys@useobject{currentmarker}{}%
\end{pgfscope}%
\end{pgfscope}%
\begin{pgfscope}%
\definecolor{textcolor}{rgb}{0.000000,0.000000,0.000000}%
\pgfsetstrokecolor{textcolor}%
\pgfsetfillcolor{textcolor}%
\pgftext[x=0.201069in, y=1.754136in, left, base]{\color{textcolor}\ttfamily\fontsize{10.000000}{12.000000}\selectfont 200000}%
\end{pgfscope}%
\begin{pgfscope}%
\pgfsetbuttcap%
\pgfsetroundjoin%
\definecolor{currentfill}{rgb}{0.000000,0.000000,0.000000}%
\pgfsetfillcolor{currentfill}%
\pgfsetlinewidth{0.803000pt}%
\definecolor{currentstroke}{rgb}{0.000000,0.000000,0.000000}%
\pgfsetstrokecolor{currentstroke}%
\pgfsetdash{}{0pt}%
\pgfsys@defobject{currentmarker}{\pgfqpoint{-0.048611in}{0.000000in}}{\pgfqpoint{-0.000000in}{0.000000in}}{%
\pgfpathmoveto{\pgfqpoint{-0.000000in}{0.000000in}}%
\pgfpathlineto{\pgfqpoint{-0.048611in}{0.000000in}}%
\pgfusepath{stroke,fill}%
}%
\begin{pgfscope}%
\pgfsys@transformshift{0.800000in}{2.363268in}%
\pgfsys@useobject{currentmarker}{}%
\end{pgfscope}%
\end{pgfscope}%
\begin{pgfscope}%
\definecolor{textcolor}{rgb}{0.000000,0.000000,0.000000}%
\pgfsetstrokecolor{textcolor}%
\pgfsetfillcolor{textcolor}%
\pgftext[x=0.201069in, y=2.310133in, left, base]{\color{textcolor}\ttfamily\fontsize{10.000000}{12.000000}\selectfont 300000}%
\end{pgfscope}%
\begin{pgfscope}%
\pgfsetbuttcap%
\pgfsetroundjoin%
\definecolor{currentfill}{rgb}{0.000000,0.000000,0.000000}%
\pgfsetfillcolor{currentfill}%
\pgfsetlinewidth{0.803000pt}%
\definecolor{currentstroke}{rgb}{0.000000,0.000000,0.000000}%
\pgfsetstrokecolor{currentstroke}%
\pgfsetdash{}{0pt}%
\pgfsys@defobject{currentmarker}{\pgfqpoint{-0.048611in}{0.000000in}}{\pgfqpoint{-0.000000in}{0.000000in}}{%
\pgfpathmoveto{\pgfqpoint{-0.000000in}{0.000000in}}%
\pgfpathlineto{\pgfqpoint{-0.048611in}{0.000000in}}%
\pgfusepath{stroke,fill}%
}%
\begin{pgfscope}%
\pgfsys@transformshift{0.800000in}{2.919264in}%
\pgfsys@useobject{currentmarker}{}%
\end{pgfscope}%
\end{pgfscope}%
\begin{pgfscope}%
\definecolor{textcolor}{rgb}{0.000000,0.000000,0.000000}%
\pgfsetstrokecolor{textcolor}%
\pgfsetfillcolor{textcolor}%
\pgftext[x=0.201069in, y=2.866130in, left, base]{\color{textcolor}\ttfamily\fontsize{10.000000}{12.000000}\selectfont 400000}%
\end{pgfscope}%
\begin{pgfscope}%
\pgfsetbuttcap%
\pgfsetroundjoin%
\definecolor{currentfill}{rgb}{0.000000,0.000000,0.000000}%
\pgfsetfillcolor{currentfill}%
\pgfsetlinewidth{0.803000pt}%
\definecolor{currentstroke}{rgb}{0.000000,0.000000,0.000000}%
\pgfsetstrokecolor{currentstroke}%
\pgfsetdash{}{0pt}%
\pgfsys@defobject{currentmarker}{\pgfqpoint{-0.048611in}{0.000000in}}{\pgfqpoint{-0.000000in}{0.000000in}}{%
\pgfpathmoveto{\pgfqpoint{-0.000000in}{0.000000in}}%
\pgfpathlineto{\pgfqpoint{-0.048611in}{0.000000in}}%
\pgfusepath{stroke,fill}%
}%
\begin{pgfscope}%
\pgfsys@transformshift{0.800000in}{3.475261in}%
\pgfsys@useobject{currentmarker}{}%
\end{pgfscope}%
\end{pgfscope}%
\begin{pgfscope}%
\definecolor{textcolor}{rgb}{0.000000,0.000000,0.000000}%
\pgfsetstrokecolor{textcolor}%
\pgfsetfillcolor{textcolor}%
\pgftext[x=0.201069in, y=3.422127in, left, base]{\color{textcolor}\ttfamily\fontsize{10.000000}{12.000000}\selectfont 500000}%
\end{pgfscope}%
\begin{pgfscope}%
\pgfsetbuttcap%
\pgfsetroundjoin%
\definecolor{currentfill}{rgb}{0.000000,0.000000,0.000000}%
\pgfsetfillcolor{currentfill}%
\pgfsetlinewidth{0.803000pt}%
\definecolor{currentstroke}{rgb}{0.000000,0.000000,0.000000}%
\pgfsetstrokecolor{currentstroke}%
\pgfsetdash{}{0pt}%
\pgfsys@defobject{currentmarker}{\pgfqpoint{-0.048611in}{0.000000in}}{\pgfqpoint{-0.000000in}{0.000000in}}{%
\pgfpathmoveto{\pgfqpoint{-0.000000in}{0.000000in}}%
\pgfpathlineto{\pgfqpoint{-0.048611in}{0.000000in}}%
\pgfusepath{stroke,fill}%
}%
\begin{pgfscope}%
\pgfsys@transformshift{0.800000in}{4.031258in}%
\pgfsys@useobject{currentmarker}{}%
\end{pgfscope}%
\end{pgfscope}%
\begin{pgfscope}%
\definecolor{textcolor}{rgb}{0.000000,0.000000,0.000000}%
\pgfsetstrokecolor{textcolor}%
\pgfsetfillcolor{textcolor}%
\pgftext[x=0.201069in, y=3.978124in, left, base]{\color{textcolor}\ttfamily\fontsize{10.000000}{12.000000}\selectfont 600000}%
\end{pgfscope}%
\begin{pgfscope}%
\definecolor{textcolor}{rgb}{0.000000,0.000000,0.000000}%
\pgfsetstrokecolor{textcolor}%
\pgfsetfillcolor{textcolor}%
\pgftext[x=0.145513in,y=2.376000in,,bottom,rotate=90.000000]{\color{textcolor}\ttfamily\fontsize{10.000000}{12.000000}\selectfont Iterations}%
\end{pgfscope}%
\begin{pgfscope}%
\pgfpathrectangle{\pgfqpoint{0.800000in}{0.528000in}}{\pgfqpoint{4.960000in}{3.696000in}}%
\pgfusepath{clip}%
\pgfsetrectcap%
\pgfsetroundjoin%
\pgfsetlinewidth{1.505625pt}%
\definecolor{currentstroke}{rgb}{1.000000,0.000000,0.000000}%
\pgfsetstrokecolor{currentstroke}%
\pgfsetdash{}{0pt}%
\pgfpathmoveto{\pgfqpoint{1.025455in}{0.722799in}}%
\pgfpathlineto{\pgfqpoint{1.115636in}{0.734975in}}%
\pgfpathlineto{\pgfqpoint{1.205818in}{0.749376in}}%
\pgfpathlineto{\pgfqpoint{1.296000in}{0.766000in}}%
\pgfpathlineto{\pgfqpoint{1.386182in}{0.784848in}}%
\pgfpathlineto{\pgfqpoint{1.476364in}{0.805921in}}%
\pgfpathlineto{\pgfqpoint{1.566545in}{0.829217in}}%
\pgfpathlineto{\pgfqpoint{1.656727in}{0.854737in}}%
\pgfpathlineto{\pgfqpoint{1.746909in}{0.882481in}}%
\pgfpathlineto{\pgfqpoint{1.837091in}{0.912450in}}%
\pgfpathlineto{\pgfqpoint{1.931782in}{0.946310in}}%
\pgfpathlineto{\pgfqpoint{2.026473in}{0.982622in}}%
\pgfpathlineto{\pgfqpoint{2.121164in}{1.021386in}}%
\pgfpathlineto{\pgfqpoint{2.215855in}{1.062602in}}%
\pgfpathlineto{\pgfqpoint{2.310545in}{1.106270in}}%
\pgfpathlineto{\pgfqpoint{2.405236in}{1.152390in}}%
\pgfpathlineto{\pgfqpoint{2.499927in}{1.200962in}}%
\pgfpathlineto{\pgfqpoint{2.594618in}{1.251986in}}%
\pgfpathlineto{\pgfqpoint{2.689309in}{1.305461in}}%
\pgfpathlineto{\pgfqpoint{2.784000in}{1.361389in}}%
\pgfpathlineto{\pgfqpoint{2.883200in}{1.422610in}}%
\pgfpathlineto{\pgfqpoint{2.982400in}{1.486522in}}%
\pgfpathlineto{\pgfqpoint{3.081600in}{1.553125in}}%
\pgfpathlineto{\pgfqpoint{3.180800in}{1.622419in}}%
\pgfpathlineto{\pgfqpoint{3.280000in}{1.694403in}}%
\pgfpathlineto{\pgfqpoint{3.379200in}{1.769079in}}%
\pgfpathlineto{\pgfqpoint{3.478400in}{1.846446in}}%
\pgfpathlineto{\pgfqpoint{3.582109in}{1.930207in}}%
\pgfpathlineto{\pgfqpoint{3.685818in}{2.016909in}}%
\pgfpathlineto{\pgfqpoint{3.789527in}{2.106553in}}%
\pgfpathlineto{\pgfqpoint{3.893236in}{2.199137in}}%
\pgfpathlineto{\pgfqpoint{3.996945in}{2.294663in}}%
\pgfpathlineto{\pgfqpoint{4.100655in}{2.393130in}}%
\pgfpathlineto{\pgfqpoint{4.208873in}{2.499014in}}%
\pgfpathlineto{\pgfqpoint{4.317091in}{2.608101in}}%
\pgfpathlineto{\pgfqpoint{4.425309in}{2.720390in}}%
\pgfpathlineto{\pgfqpoint{4.533527in}{2.835882in}}%
\pgfpathlineto{\pgfqpoint{4.641745in}{2.954576in}}%
\pgfpathlineto{\pgfqpoint{4.749964in}{3.076473in}}%
\pgfpathlineto{\pgfqpoint{4.862691in}{3.206854in}}%
\pgfpathlineto{\pgfqpoint{4.975418in}{3.340710in}}%
\pgfpathlineto{\pgfqpoint{5.088145in}{3.478041in}}%
\pgfpathlineto{\pgfqpoint{5.200873in}{3.618847in}}%
\pgfpathlineto{\pgfqpoint{5.313600in}{3.763129in}}%
\pgfpathlineto{\pgfqpoint{5.430836in}{3.916867in}}%
\pgfpathlineto{\pgfqpoint{5.534545in}{4.056000in}}%
\pgfpathlineto{\pgfqpoint{5.534545in}{4.056000in}}%
\pgfusepath{stroke}%
\end{pgfscope}%
\begin{pgfscope}%
\pgfpathrectangle{\pgfqpoint{0.800000in}{0.528000in}}{\pgfqpoint{4.960000in}{3.696000in}}%
\pgfusepath{clip}%
\pgfsetrectcap%
\pgfsetroundjoin%
\pgfsetlinewidth{1.505625pt}%
\definecolor{currentstroke}{rgb}{0.486275,0.988235,0.000000}%
\pgfsetstrokecolor{currentstroke}%
\pgfsetdash{}{0pt}%
\pgfpathmoveto{\pgfqpoint{1.025455in}{0.722799in}}%
\pgfpathlineto{\pgfqpoint{1.115636in}{0.734975in}}%
\pgfpathlineto{\pgfqpoint{1.205818in}{0.749376in}}%
\pgfpathlineto{\pgfqpoint{1.296000in}{0.766000in}}%
\pgfpathlineto{\pgfqpoint{1.386182in}{0.784848in}}%
\pgfpathlineto{\pgfqpoint{1.476364in}{0.805921in}}%
\pgfpathlineto{\pgfqpoint{1.566545in}{0.829217in}}%
\pgfpathlineto{\pgfqpoint{1.656727in}{0.854737in}}%
\pgfpathlineto{\pgfqpoint{1.746909in}{0.882481in}}%
\pgfpathlineto{\pgfqpoint{1.837091in}{0.912450in}}%
\pgfpathlineto{\pgfqpoint{1.931782in}{0.946310in}}%
\pgfpathlineto{\pgfqpoint{2.026473in}{0.982622in}}%
\pgfpathlineto{\pgfqpoint{2.121164in}{1.021386in}}%
\pgfpathlineto{\pgfqpoint{2.215855in}{1.062602in}}%
\pgfpathlineto{\pgfqpoint{2.310545in}{1.106270in}}%
\pgfpathlineto{\pgfqpoint{2.405236in}{1.152390in}}%
\pgfpathlineto{\pgfqpoint{2.499927in}{1.200962in}}%
\pgfpathlineto{\pgfqpoint{2.594618in}{1.251986in}}%
\pgfpathlineto{\pgfqpoint{2.689309in}{1.305461in}}%
\pgfpathlineto{\pgfqpoint{2.784000in}{1.361389in}}%
\pgfpathlineto{\pgfqpoint{2.883200in}{1.422610in}}%
\pgfpathlineto{\pgfqpoint{2.982400in}{1.486522in}}%
\pgfpathlineto{\pgfqpoint{3.081600in}{1.553125in}}%
\pgfpathlineto{\pgfqpoint{3.180800in}{1.622419in}}%
\pgfpathlineto{\pgfqpoint{3.280000in}{1.694403in}}%
\pgfpathlineto{\pgfqpoint{3.379200in}{1.769079in}}%
\pgfpathlineto{\pgfqpoint{3.478400in}{1.846446in}}%
\pgfpathlineto{\pgfqpoint{3.582109in}{1.930207in}}%
\pgfpathlineto{\pgfqpoint{3.685818in}{2.016909in}}%
\pgfpathlineto{\pgfqpoint{3.789527in}{2.106553in}}%
\pgfpathlineto{\pgfqpoint{3.893236in}{2.199137in}}%
\pgfpathlineto{\pgfqpoint{3.996945in}{2.294663in}}%
\pgfpathlineto{\pgfqpoint{4.100655in}{2.393130in}}%
\pgfpathlineto{\pgfqpoint{4.208873in}{2.499014in}}%
\pgfpathlineto{\pgfqpoint{4.317091in}{2.608101in}}%
\pgfpathlineto{\pgfqpoint{4.425309in}{2.720390in}}%
\pgfpathlineto{\pgfqpoint{4.533527in}{2.835882in}}%
\pgfpathlineto{\pgfqpoint{4.641745in}{2.954576in}}%
\pgfpathlineto{\pgfqpoint{4.749964in}{3.076473in}}%
\pgfpathlineto{\pgfqpoint{4.862691in}{3.206854in}}%
\pgfpathlineto{\pgfqpoint{4.975418in}{3.340710in}}%
\pgfpathlineto{\pgfqpoint{5.088145in}{3.478041in}}%
\pgfpathlineto{\pgfqpoint{5.200873in}{3.618847in}}%
\pgfpathlineto{\pgfqpoint{5.313600in}{3.763129in}}%
\pgfpathlineto{\pgfqpoint{5.430836in}{3.916867in}}%
\pgfpathlineto{\pgfqpoint{5.534545in}{4.056000in}}%
\pgfpathlineto{\pgfqpoint{5.534545in}{4.056000in}}%
\pgfusepath{stroke}%
\end{pgfscope}%
\begin{pgfscope}%
\pgfpathrectangle{\pgfqpoint{0.800000in}{0.528000in}}{\pgfqpoint{4.960000in}{3.696000in}}%
\pgfusepath{clip}%
\pgfsetrectcap%
\pgfsetroundjoin%
\pgfsetlinewidth{1.505625pt}%
\definecolor{currentstroke}{rgb}{0.000000,1.000000,0.498039}%
\pgfsetstrokecolor{currentstroke}%
\pgfsetdash{}{0pt}%
\pgfpathmoveto{\pgfqpoint{1.025455in}{0.706236in}}%
\pgfpathlineto{\pgfqpoint{1.034473in}{0.710061in}}%
\pgfpathlineto{\pgfqpoint{1.038982in}{0.709905in}}%
\pgfpathlineto{\pgfqpoint{1.048000in}{0.713119in}}%
\pgfpathlineto{\pgfqpoint{1.052509in}{0.710612in}}%
\pgfpathlineto{\pgfqpoint{1.057018in}{0.709722in}}%
\pgfpathlineto{\pgfqpoint{1.070545in}{0.713353in}}%
\pgfpathlineto{\pgfqpoint{1.075055in}{0.711262in}}%
\pgfpathlineto{\pgfqpoint{1.084073in}{0.713519in}}%
\pgfpathlineto{\pgfqpoint{1.088582in}{0.714932in}}%
\pgfpathlineto{\pgfqpoint{1.097600in}{0.713025in}}%
\pgfpathlineto{\pgfqpoint{1.102109in}{0.714114in}}%
\pgfpathlineto{\pgfqpoint{1.106618in}{0.716594in}}%
\pgfpathlineto{\pgfqpoint{1.115636in}{0.715710in}}%
\pgfpathlineto{\pgfqpoint{1.120145in}{0.718229in}}%
\pgfpathlineto{\pgfqpoint{1.129164in}{0.714987in}}%
\pgfpathlineto{\pgfqpoint{1.133673in}{0.716166in}}%
\pgfpathlineto{\pgfqpoint{1.138182in}{0.719157in}}%
\pgfpathlineto{\pgfqpoint{1.142691in}{0.717828in}}%
\pgfpathlineto{\pgfqpoint{1.147200in}{0.720264in}}%
\pgfpathlineto{\pgfqpoint{1.151709in}{0.716683in}}%
\pgfpathlineto{\pgfqpoint{1.156218in}{0.720153in}}%
\pgfpathlineto{\pgfqpoint{1.165236in}{0.719118in}}%
\pgfpathlineto{\pgfqpoint{1.169745in}{0.722087in}}%
\pgfpathlineto{\pgfqpoint{1.174255in}{0.721548in}}%
\pgfpathlineto{\pgfqpoint{1.178764in}{0.718523in}}%
\pgfpathlineto{\pgfqpoint{1.183273in}{0.722332in}}%
\pgfpathlineto{\pgfqpoint{1.187782in}{0.719063in}}%
\pgfpathlineto{\pgfqpoint{1.196800in}{0.722688in}}%
\pgfpathlineto{\pgfqpoint{1.201309in}{0.721153in}}%
\pgfpathlineto{\pgfqpoint{1.205818in}{0.723388in}}%
\pgfpathlineto{\pgfqpoint{1.214836in}{0.723655in}}%
\pgfpathlineto{\pgfqpoint{1.219345in}{0.725979in}}%
\pgfpathlineto{\pgfqpoint{1.228364in}{0.723983in}}%
\pgfpathlineto{\pgfqpoint{1.232873in}{0.727686in}}%
\pgfpathlineto{\pgfqpoint{1.241891in}{0.725457in}}%
\pgfpathlineto{\pgfqpoint{1.246400in}{0.724489in}}%
\pgfpathlineto{\pgfqpoint{1.250909in}{0.727414in}}%
\pgfpathlineto{\pgfqpoint{1.255418in}{0.727736in}}%
\pgfpathlineto{\pgfqpoint{1.259927in}{0.729410in}}%
\pgfpathlineto{\pgfqpoint{1.264436in}{0.725434in}}%
\pgfpathlineto{\pgfqpoint{1.268945in}{0.729321in}}%
\pgfpathlineto{\pgfqpoint{1.277964in}{0.728620in}}%
\pgfpathlineto{\pgfqpoint{1.282473in}{0.730766in}}%
\pgfpathlineto{\pgfqpoint{1.291491in}{0.729816in}}%
\pgfpathlineto{\pgfqpoint{1.296000in}{0.737033in}}%
\pgfpathlineto{\pgfqpoint{1.300509in}{0.731278in}}%
\pgfpathlineto{\pgfqpoint{1.305018in}{0.729054in}}%
\pgfpathlineto{\pgfqpoint{1.318545in}{0.735715in}}%
\pgfpathlineto{\pgfqpoint{1.323055in}{0.735804in}}%
\pgfpathlineto{\pgfqpoint{1.327564in}{0.730044in}}%
\pgfpathlineto{\pgfqpoint{1.332073in}{0.737689in}}%
\pgfpathlineto{\pgfqpoint{1.336582in}{0.736121in}}%
\pgfpathlineto{\pgfqpoint{1.341091in}{0.738651in}}%
\pgfpathlineto{\pgfqpoint{1.345600in}{0.737978in}}%
\pgfpathlineto{\pgfqpoint{1.350109in}{0.738856in}}%
\pgfpathlineto{\pgfqpoint{1.354618in}{0.734403in}}%
\pgfpathlineto{\pgfqpoint{1.359127in}{0.737900in}}%
\pgfpathlineto{\pgfqpoint{1.363636in}{0.737794in}}%
\pgfpathlineto{\pgfqpoint{1.368145in}{0.740669in}}%
\pgfpathlineto{\pgfqpoint{1.372655in}{0.740713in}}%
\pgfpathlineto{\pgfqpoint{1.377164in}{0.736471in}}%
\pgfpathlineto{\pgfqpoint{1.381673in}{0.739490in}}%
\pgfpathlineto{\pgfqpoint{1.386182in}{0.740419in}}%
\pgfpathlineto{\pgfqpoint{1.395200in}{0.737577in}}%
\pgfpathlineto{\pgfqpoint{1.399709in}{0.740424in}}%
\pgfpathlineto{\pgfqpoint{1.404218in}{0.744728in}}%
\pgfpathlineto{\pgfqpoint{1.408727in}{0.742654in}}%
\pgfpathlineto{\pgfqpoint{1.422255in}{0.745245in}}%
\pgfpathlineto{\pgfqpoint{1.426764in}{0.744344in}}%
\pgfpathlineto{\pgfqpoint{1.431273in}{0.741742in}}%
\pgfpathlineto{\pgfqpoint{1.435782in}{0.745767in}}%
\pgfpathlineto{\pgfqpoint{1.444800in}{0.748725in}}%
\pgfpathlineto{\pgfqpoint{1.449309in}{0.752512in}}%
\pgfpathlineto{\pgfqpoint{1.467345in}{0.748269in}}%
\pgfpathlineto{\pgfqpoint{1.471855in}{0.750071in}}%
\pgfpathlineto{\pgfqpoint{1.476364in}{0.755136in}}%
\pgfpathlineto{\pgfqpoint{1.485382in}{0.749114in}}%
\pgfpathlineto{\pgfqpoint{1.489891in}{0.751566in}}%
\pgfpathlineto{\pgfqpoint{1.494400in}{0.749748in}}%
\pgfpathlineto{\pgfqpoint{1.498909in}{0.755436in}}%
\pgfpathlineto{\pgfqpoint{1.503418in}{0.755514in}}%
\pgfpathlineto{\pgfqpoint{1.507927in}{0.752689in}}%
\pgfpathlineto{\pgfqpoint{1.512436in}{0.759078in}}%
\pgfpathlineto{\pgfqpoint{1.516945in}{0.754702in}}%
\pgfpathlineto{\pgfqpoint{1.521455in}{0.756520in}}%
\pgfpathlineto{\pgfqpoint{1.525964in}{0.755108in}}%
\pgfpathlineto{\pgfqpoint{1.530473in}{0.755942in}}%
\pgfpathlineto{\pgfqpoint{1.534982in}{0.759078in}}%
\pgfpathlineto{\pgfqpoint{1.539491in}{0.757321in}}%
\pgfpathlineto{\pgfqpoint{1.544000in}{0.762764in}}%
\pgfpathlineto{\pgfqpoint{1.548509in}{0.763053in}}%
\pgfpathlineto{\pgfqpoint{1.553018in}{0.764566in}}%
\pgfpathlineto{\pgfqpoint{1.557527in}{0.762836in}}%
\pgfpathlineto{\pgfqpoint{1.566545in}{0.763331in}}%
\pgfpathlineto{\pgfqpoint{1.571055in}{0.763242in}}%
\pgfpathlineto{\pgfqpoint{1.575564in}{0.766550in}}%
\pgfpathlineto{\pgfqpoint{1.580073in}{0.760496in}}%
\pgfpathlineto{\pgfqpoint{1.584582in}{0.763898in}}%
\pgfpathlineto{\pgfqpoint{1.589091in}{0.769425in}}%
\pgfpathlineto{\pgfqpoint{1.593600in}{0.770748in}}%
\pgfpathlineto{\pgfqpoint{1.598109in}{0.767874in}}%
\pgfpathlineto{\pgfqpoint{1.602618in}{0.769342in}}%
\pgfpathlineto{\pgfqpoint{1.607127in}{0.774507in}}%
\pgfpathlineto{\pgfqpoint{1.611636in}{0.768085in}}%
\pgfpathlineto{\pgfqpoint{1.616145in}{0.766228in}}%
\pgfpathlineto{\pgfqpoint{1.620655in}{0.773639in}}%
\pgfpathlineto{\pgfqpoint{1.625164in}{0.767918in}}%
\pgfpathlineto{\pgfqpoint{1.629673in}{0.767457in}}%
\pgfpathlineto{\pgfqpoint{1.634182in}{0.777214in}}%
\pgfpathlineto{\pgfqpoint{1.638691in}{0.775291in}}%
\pgfpathlineto{\pgfqpoint{1.647709in}{0.776647in}}%
\pgfpathlineto{\pgfqpoint{1.652218in}{0.774885in}}%
\pgfpathlineto{\pgfqpoint{1.656727in}{0.777865in}}%
\pgfpathlineto{\pgfqpoint{1.661236in}{0.775519in}}%
\pgfpathlineto{\pgfqpoint{1.665745in}{0.775057in}}%
\pgfpathlineto{\pgfqpoint{1.670255in}{0.781696in}}%
\pgfpathlineto{\pgfqpoint{1.674764in}{0.781779in}}%
\pgfpathlineto{\pgfqpoint{1.679273in}{0.785538in}}%
\pgfpathlineto{\pgfqpoint{1.683782in}{0.782463in}}%
\pgfpathlineto{\pgfqpoint{1.688291in}{0.770948in}}%
\pgfpathlineto{\pgfqpoint{1.697309in}{0.785538in}}%
\pgfpathlineto{\pgfqpoint{1.710836in}{0.786127in}}%
\pgfpathlineto{\pgfqpoint{1.715345in}{0.780300in}}%
\pgfpathlineto{\pgfqpoint{1.719855in}{0.784376in}}%
\pgfpathlineto{\pgfqpoint{1.724364in}{0.793199in}}%
\pgfpathlineto{\pgfqpoint{1.728873in}{0.790464in}}%
\pgfpathlineto{\pgfqpoint{1.733382in}{0.789402in}}%
\pgfpathlineto{\pgfqpoint{1.737891in}{0.789741in}}%
\pgfpathlineto{\pgfqpoint{1.742400in}{0.786716in}}%
\pgfpathlineto{\pgfqpoint{1.751418in}{0.788312in}}%
\pgfpathlineto{\pgfqpoint{1.755927in}{0.786783in}}%
\pgfpathlineto{\pgfqpoint{1.764945in}{0.797631in}}%
\pgfpathlineto{\pgfqpoint{1.769455in}{0.789346in}}%
\pgfpathlineto{\pgfqpoint{1.773964in}{0.793299in}}%
\pgfpathlineto{\pgfqpoint{1.778473in}{0.793956in}}%
\pgfpathlineto{\pgfqpoint{1.782982in}{0.791437in}}%
\pgfpathlineto{\pgfqpoint{1.787491in}{0.798437in}}%
\pgfpathlineto{\pgfqpoint{1.792000in}{0.794467in}}%
\pgfpathlineto{\pgfqpoint{1.801018in}{0.805187in}}%
\pgfpathlineto{\pgfqpoint{1.805527in}{0.793405in}}%
\pgfpathlineto{\pgfqpoint{1.810036in}{0.801929in}}%
\pgfpathlineto{\pgfqpoint{1.814545in}{0.800733in}}%
\pgfpathlineto{\pgfqpoint{1.819055in}{0.806321in}}%
\pgfpathlineto{\pgfqpoint{1.828073in}{0.807755in}}%
\pgfpathlineto{\pgfqpoint{1.832582in}{0.807500in}}%
\pgfpathlineto{\pgfqpoint{1.841600in}{0.805476in}}%
\pgfpathlineto{\pgfqpoint{1.846109in}{0.801128in}}%
\pgfpathlineto{\pgfqpoint{1.850618in}{0.802890in}}%
\pgfpathlineto{\pgfqpoint{1.855127in}{0.811208in}}%
\pgfpathlineto{\pgfqpoint{1.859636in}{0.813299in}}%
\pgfpathlineto{\pgfqpoint{1.868655in}{0.813882in}}%
\pgfpathlineto{\pgfqpoint{1.873164in}{0.807689in}}%
\pgfpathlineto{\pgfqpoint{1.877673in}{0.809251in}}%
\pgfpathlineto{\pgfqpoint{1.886691in}{0.821744in}}%
\pgfpathlineto{\pgfqpoint{1.891200in}{0.809179in}}%
\pgfpathlineto{\pgfqpoint{1.895709in}{0.816207in}}%
\pgfpathlineto{\pgfqpoint{1.900218in}{0.818664in}}%
\pgfpathlineto{\pgfqpoint{1.904727in}{0.822478in}}%
\pgfpathlineto{\pgfqpoint{1.909236in}{0.813143in}}%
\pgfpathlineto{\pgfqpoint{1.913745in}{0.823434in}}%
\pgfpathlineto{\pgfqpoint{1.918255in}{0.824519in}}%
\pgfpathlineto{\pgfqpoint{1.922764in}{0.820532in}}%
\pgfpathlineto{\pgfqpoint{1.927273in}{0.826631in}}%
\pgfpathlineto{\pgfqpoint{1.931782in}{0.825403in}}%
\pgfpathlineto{\pgfqpoint{1.936291in}{0.827688in}}%
\pgfpathlineto{\pgfqpoint{1.940800in}{0.821477in}}%
\pgfpathlineto{\pgfqpoint{1.945309in}{0.830607in}}%
\pgfpathlineto{\pgfqpoint{1.954327in}{0.829373in}}%
\pgfpathlineto{\pgfqpoint{1.963345in}{0.828272in}}%
\pgfpathlineto{\pgfqpoint{1.967855in}{0.836461in}}%
\pgfpathlineto{\pgfqpoint{1.972364in}{0.832241in}}%
\pgfpathlineto{\pgfqpoint{1.976873in}{0.837140in}}%
\pgfpathlineto{\pgfqpoint{1.981382in}{0.836183in}}%
\pgfpathlineto{\pgfqpoint{1.990400in}{0.839130in}}%
\pgfpathlineto{\pgfqpoint{1.999418in}{0.830384in}}%
\pgfpathlineto{\pgfqpoint{2.003927in}{0.831274in}}%
\pgfpathlineto{\pgfqpoint{2.008436in}{0.834132in}}%
\pgfpathlineto{\pgfqpoint{2.012945in}{0.841660in}}%
\pgfpathlineto{\pgfqpoint{2.017455in}{0.844690in}}%
\pgfpathlineto{\pgfqpoint{2.021964in}{0.836028in}}%
\pgfpathlineto{\pgfqpoint{2.026473in}{0.832019in}}%
\pgfpathlineto{\pgfqpoint{2.030982in}{0.846414in}}%
\pgfpathlineto{\pgfqpoint{2.035491in}{0.853481in}}%
\pgfpathlineto{\pgfqpoint{2.040000in}{0.837551in}}%
\pgfpathlineto{\pgfqpoint{2.044509in}{0.839253in}}%
\pgfpathlineto{\pgfqpoint{2.049018in}{0.847098in}}%
\pgfpathlineto{\pgfqpoint{2.053527in}{0.843845in}}%
\pgfpathlineto{\pgfqpoint{2.058036in}{0.849461in}}%
\pgfpathlineto{\pgfqpoint{2.067055in}{0.851284in}}%
\pgfpathlineto{\pgfqpoint{2.071564in}{0.844979in}}%
\pgfpathlineto{\pgfqpoint{2.076073in}{0.866129in}}%
\pgfpathlineto{\pgfqpoint{2.080582in}{0.844151in}}%
\pgfpathlineto{\pgfqpoint{2.089600in}{0.844902in}}%
\pgfpathlineto{\pgfqpoint{2.094109in}{0.856472in}}%
\pgfpathlineto{\pgfqpoint{2.098618in}{0.855299in}}%
\pgfpathlineto{\pgfqpoint{2.103127in}{0.861187in}}%
\pgfpathlineto{\pgfqpoint{2.107636in}{0.857373in}}%
\pgfpathlineto{\pgfqpoint{2.112145in}{0.851012in}}%
\pgfpathlineto{\pgfqpoint{2.116655in}{0.870021in}}%
\pgfpathlineto{\pgfqpoint{2.121164in}{0.863689in}}%
\pgfpathlineto{\pgfqpoint{2.125673in}{0.865329in}}%
\pgfpathlineto{\pgfqpoint{2.130182in}{0.871373in}}%
\pgfpathlineto{\pgfqpoint{2.134691in}{0.860075in}}%
\pgfpathlineto{\pgfqpoint{2.139200in}{0.867108in}}%
\pgfpathlineto{\pgfqpoint{2.143709in}{0.863800in}}%
\pgfpathlineto{\pgfqpoint{2.148218in}{0.870372in}}%
\pgfpathlineto{\pgfqpoint{2.152727in}{0.861732in}}%
\pgfpathlineto{\pgfqpoint{2.161745in}{0.860603in}}%
\pgfpathlineto{\pgfqpoint{2.166255in}{0.852952in}}%
\pgfpathlineto{\pgfqpoint{2.170764in}{0.874147in}}%
\pgfpathlineto{\pgfqpoint{2.175273in}{0.877038in}}%
\pgfpathlineto{\pgfqpoint{2.179782in}{0.876771in}}%
\pgfpathlineto{\pgfqpoint{2.184291in}{0.873658in}}%
\pgfpathlineto{\pgfqpoint{2.188800in}{0.875131in}}%
\pgfpathlineto{\pgfqpoint{2.193309in}{0.878467in}}%
\pgfpathlineto{\pgfqpoint{2.197818in}{0.870400in}}%
\pgfpathlineto{\pgfqpoint{2.202327in}{0.872112in}}%
\pgfpathlineto{\pgfqpoint{2.206836in}{0.885150in}}%
\pgfpathlineto{\pgfqpoint{2.211345in}{0.871678in}}%
\pgfpathlineto{\pgfqpoint{2.215855in}{0.882570in}}%
\pgfpathlineto{\pgfqpoint{2.220364in}{0.876237in}}%
\pgfpathlineto{\pgfqpoint{2.224873in}{0.873991in}}%
\pgfpathlineto{\pgfqpoint{2.229382in}{0.897110in}}%
\pgfpathlineto{\pgfqpoint{2.233891in}{0.882704in}}%
\pgfpathlineto{\pgfqpoint{2.238400in}{0.882376in}}%
\pgfpathlineto{\pgfqpoint{2.242909in}{0.893968in}}%
\pgfpathlineto{\pgfqpoint{2.247418in}{0.884055in}}%
\pgfpathlineto{\pgfqpoint{2.251927in}{0.896804in}}%
\pgfpathlineto{\pgfqpoint{2.256436in}{0.879218in}}%
\pgfpathlineto{\pgfqpoint{2.260945in}{0.888759in}}%
\pgfpathlineto{\pgfqpoint{2.269964in}{0.900996in}}%
\pgfpathlineto{\pgfqpoint{2.274473in}{0.887357in}}%
\pgfpathlineto{\pgfqpoint{2.278982in}{0.901663in}}%
\pgfpathlineto{\pgfqpoint{2.283491in}{0.895169in}}%
\pgfpathlineto{\pgfqpoint{2.288000in}{0.886462in}}%
\pgfpathlineto{\pgfqpoint{2.292509in}{0.892617in}}%
\pgfpathlineto{\pgfqpoint{2.297018in}{0.904443in}}%
\pgfpathlineto{\pgfqpoint{2.301527in}{0.906556in}}%
\pgfpathlineto{\pgfqpoint{2.310545in}{0.913534in}}%
\pgfpathlineto{\pgfqpoint{2.315055in}{0.894391in}}%
\pgfpathlineto{\pgfqpoint{2.319564in}{0.901185in}}%
\pgfpathlineto{\pgfqpoint{2.324073in}{0.898655in}}%
\pgfpathlineto{\pgfqpoint{2.328582in}{0.906117in}}%
\pgfpathlineto{\pgfqpoint{2.333091in}{0.908719in}}%
\pgfpathlineto{\pgfqpoint{2.337600in}{0.916114in}}%
\pgfpathlineto{\pgfqpoint{2.342109in}{0.909375in}}%
\pgfpathlineto{\pgfqpoint{2.346618in}{0.914151in}}%
\pgfpathlineto{\pgfqpoint{2.351127in}{0.900946in}}%
\pgfpathlineto{\pgfqpoint{2.355636in}{0.905833in}}%
\pgfpathlineto{\pgfqpoint{2.360145in}{0.922163in}}%
\pgfpathlineto{\pgfqpoint{2.364655in}{0.925888in}}%
\pgfpathlineto{\pgfqpoint{2.369164in}{0.904265in}}%
\pgfpathlineto{\pgfqpoint{2.373673in}{0.920195in}}%
\pgfpathlineto{\pgfqpoint{2.378182in}{0.926483in}}%
\pgfpathlineto{\pgfqpoint{2.382691in}{0.917871in}}%
\pgfpathlineto{\pgfqpoint{2.387200in}{0.920284in}}%
\pgfpathlineto{\pgfqpoint{2.391709in}{0.924437in}}%
\pgfpathlineto{\pgfqpoint{2.396218in}{0.916803in}}%
\pgfpathlineto{\pgfqpoint{2.400727in}{0.934890in}}%
\pgfpathlineto{\pgfqpoint{2.405236in}{0.924520in}}%
\pgfpathlineto{\pgfqpoint{2.409745in}{0.920545in}}%
\pgfpathlineto{\pgfqpoint{2.414255in}{0.938353in}}%
\pgfpathlineto{\pgfqpoint{2.418764in}{0.917448in}}%
\pgfpathlineto{\pgfqpoint{2.423273in}{0.918249in}}%
\pgfpathlineto{\pgfqpoint{2.432291in}{0.928462in}}%
\pgfpathlineto{\pgfqpoint{2.436800in}{0.928434in}}%
\pgfpathlineto{\pgfqpoint{2.441309in}{0.934940in}}%
\pgfpathlineto{\pgfqpoint{2.445818in}{0.933967in}}%
\pgfpathlineto{\pgfqpoint{2.450327in}{0.940883in}}%
\pgfpathlineto{\pgfqpoint{2.454836in}{0.938665in}}%
\pgfpathlineto{\pgfqpoint{2.459345in}{0.928045in}}%
\pgfpathlineto{\pgfqpoint{2.463855in}{0.941011in}}%
\pgfpathlineto{\pgfqpoint{2.468364in}{0.935696in}}%
\pgfpathlineto{\pgfqpoint{2.472873in}{0.945176in}}%
\pgfpathlineto{\pgfqpoint{2.477382in}{0.948578in}}%
\pgfpathlineto{\pgfqpoint{2.481891in}{0.938943in}}%
\pgfpathlineto{\pgfqpoint{2.486400in}{0.943808in}}%
\pgfpathlineto{\pgfqpoint{2.490909in}{0.961294in}}%
\pgfpathlineto{\pgfqpoint{2.495418in}{0.943241in}}%
\pgfpathlineto{\pgfqpoint{2.499927in}{0.946638in}}%
\pgfpathlineto{\pgfqpoint{2.504436in}{0.973676in}}%
\pgfpathlineto{\pgfqpoint{2.508945in}{0.955367in}}%
\pgfpathlineto{\pgfqpoint{2.513455in}{0.953460in}}%
\pgfpathlineto{\pgfqpoint{2.517964in}{0.964647in}}%
\pgfpathlineto{\pgfqpoint{2.522473in}{0.949679in}}%
\pgfpathlineto{\pgfqpoint{2.526982in}{0.945537in}}%
\pgfpathlineto{\pgfqpoint{2.531491in}{0.955600in}}%
\pgfpathlineto{\pgfqpoint{2.536000in}{0.953983in}}%
\pgfpathlineto{\pgfqpoint{2.540509in}{0.958280in}}%
\pgfpathlineto{\pgfqpoint{2.545018in}{0.969228in}}%
\pgfpathlineto{\pgfqpoint{2.549527in}{0.970607in}}%
\pgfpathlineto{\pgfqpoint{2.554036in}{0.987159in}}%
\pgfpathlineto{\pgfqpoint{2.558545in}{0.949162in}}%
\pgfpathlineto{\pgfqpoint{2.563055in}{0.956040in}}%
\pgfpathlineto{\pgfqpoint{2.567564in}{0.969289in}}%
\pgfpathlineto{\pgfqpoint{2.572073in}{0.964458in}}%
\pgfpathlineto{\pgfqpoint{2.576582in}{0.979336in}}%
\pgfpathlineto{\pgfqpoint{2.581091in}{0.967065in}}%
\pgfpathlineto{\pgfqpoint{2.585600in}{0.962656in}}%
\pgfpathlineto{\pgfqpoint{2.590109in}{0.978268in}}%
\pgfpathlineto{\pgfqpoint{2.594618in}{0.968527in}}%
\pgfpathlineto{\pgfqpoint{2.603636in}{0.980253in}}%
\pgfpathlineto{\pgfqpoint{2.608145in}{0.982177in}}%
\pgfpathlineto{\pgfqpoint{2.612655in}{0.996372in}}%
\pgfpathlineto{\pgfqpoint{2.617164in}{0.999713in}}%
\pgfpathlineto{\pgfqpoint{2.621673in}{0.967399in}}%
\pgfpathlineto{\pgfqpoint{2.626182in}{0.999018in}}%
\pgfpathlineto{\pgfqpoint{2.630691in}{1.004823in}}%
\pgfpathlineto{\pgfqpoint{2.635200in}{0.980125in}}%
\pgfpathlineto{\pgfqpoint{2.639709in}{0.980053in}}%
\pgfpathlineto{\pgfqpoint{2.644218in}{0.985936in}}%
\pgfpathlineto{\pgfqpoint{2.648727in}{0.994515in}}%
\pgfpathlineto{\pgfqpoint{2.653236in}{0.995276in}}%
\pgfpathlineto{\pgfqpoint{2.657745in}{1.000547in}}%
\pgfpathlineto{\pgfqpoint{2.662255in}{0.983067in}}%
\pgfpathlineto{\pgfqpoint{2.666764in}{0.991685in}}%
\pgfpathlineto{\pgfqpoint{2.671273in}{0.990539in}}%
\pgfpathlineto{\pgfqpoint{2.675782in}{0.996244in}}%
\pgfpathlineto{\pgfqpoint{2.680291in}{1.005612in}}%
\pgfpathlineto{\pgfqpoint{2.684800in}{0.995693in}}%
\pgfpathlineto{\pgfqpoint{2.689309in}{0.999724in}}%
\pgfpathlineto{\pgfqpoint{2.693818in}{1.008476in}}%
\pgfpathlineto{\pgfqpoint{2.698327in}{1.020052in}}%
\pgfpathlineto{\pgfqpoint{2.702836in}{0.987403in}}%
\pgfpathlineto{\pgfqpoint{2.707345in}{1.029837in}}%
\pgfpathlineto{\pgfqpoint{2.711855in}{1.016399in}}%
\pgfpathlineto{\pgfqpoint{2.716364in}{0.999218in}}%
\pgfpathlineto{\pgfqpoint{2.720873in}{1.020218in}}%
\pgfpathlineto{\pgfqpoint{2.725382in}{1.022153in}}%
\pgfpathlineto{\pgfqpoint{2.729891in}{1.016165in}}%
\pgfpathlineto{\pgfqpoint{2.734400in}{1.028297in}}%
\pgfpathlineto{\pgfqpoint{2.738909in}{1.019796in}}%
\pgfpathlineto{\pgfqpoint{2.743418in}{1.017733in}}%
\pgfpathlineto{\pgfqpoint{2.747927in}{1.023994in}}%
\pgfpathlineto{\pgfqpoint{2.752436in}{1.022720in}}%
\pgfpathlineto{\pgfqpoint{2.756945in}{1.014625in}}%
\pgfpathlineto{\pgfqpoint{2.761455in}{1.027319in}}%
\pgfpathlineto{\pgfqpoint{2.765964in}{1.026284in}}%
\pgfpathlineto{\pgfqpoint{2.770473in}{1.004089in}}%
\pgfpathlineto{\pgfqpoint{2.774982in}{1.020063in}}%
\pgfpathlineto{\pgfqpoint{2.779491in}{1.024933in}}%
\pgfpathlineto{\pgfqpoint{2.784000in}{1.020146in}}%
\pgfpathlineto{\pgfqpoint{2.793018in}{1.022353in}}%
\pgfpathlineto{\pgfqpoint{2.797527in}{1.032990in}}%
\pgfpathlineto{\pgfqpoint{2.806545in}{1.047507in}}%
\pgfpathlineto{\pgfqpoint{2.811055in}{1.049047in}}%
\pgfpathlineto{\pgfqpoint{2.815564in}{1.038238in}}%
\pgfpathlineto{\pgfqpoint{2.820073in}{1.058427in}}%
\pgfpathlineto{\pgfqpoint{2.824582in}{1.049180in}}%
\pgfpathlineto{\pgfqpoint{2.829091in}{1.027458in}}%
\pgfpathlineto{\pgfqpoint{2.833600in}{1.046078in}}%
\pgfpathlineto{\pgfqpoint{2.838109in}{1.052794in}}%
\pgfpathlineto{\pgfqpoint{2.842618in}{1.068151in}}%
\pgfpathlineto{\pgfqpoint{2.847127in}{1.062207in}}%
\pgfpathlineto{\pgfqpoint{2.851636in}{1.044960in}}%
\pgfpathlineto{\pgfqpoint{2.856145in}{1.034218in}}%
\pgfpathlineto{\pgfqpoint{2.860655in}{1.063964in}}%
\pgfpathlineto{\pgfqpoint{2.865164in}{1.060856in}}%
\pgfpathlineto{\pgfqpoint{2.869673in}{1.063431in}}%
\pgfpathlineto{\pgfqpoint{2.874182in}{1.055680in}}%
\pgfpathlineto{\pgfqpoint{2.878691in}{1.065538in}}%
\pgfpathlineto{\pgfqpoint{2.883200in}{1.055797in}}%
\pgfpathlineto{\pgfqpoint{2.887709in}{1.050982in}}%
\pgfpathlineto{\pgfqpoint{2.896727in}{1.063102in}}%
\pgfpathlineto{\pgfqpoint{2.901236in}{1.068868in}}%
\pgfpathlineto{\pgfqpoint{2.905745in}{1.064053in}}%
\pgfpathlineto{\pgfqpoint{2.910255in}{1.071915in}}%
\pgfpathlineto{\pgfqpoint{2.914764in}{1.068585in}}%
\pgfpathlineto{\pgfqpoint{2.919273in}{1.079810in}}%
\pgfpathlineto{\pgfqpoint{2.923782in}{1.072949in}}%
\pgfpathlineto{\pgfqpoint{2.928291in}{1.076869in}}%
\pgfpathlineto{\pgfqpoint{2.932800in}{1.067222in}}%
\pgfpathlineto{\pgfqpoint{2.937309in}{1.051821in}}%
\pgfpathlineto{\pgfqpoint{2.941818in}{1.083624in}}%
\pgfpathlineto{\pgfqpoint{2.946327in}{1.090830in}}%
\pgfpathlineto{\pgfqpoint{2.950836in}{1.067639in}}%
\pgfpathlineto{\pgfqpoint{2.955345in}{1.071398in}}%
\pgfpathlineto{\pgfqpoint{2.959855in}{1.073033in}}%
\pgfpathlineto{\pgfqpoint{2.964364in}{1.092887in}}%
\pgfpathlineto{\pgfqpoint{2.968873in}{1.087822in}}%
\pgfpathlineto{\pgfqpoint{2.973382in}{1.121588in}}%
\pgfpathlineto{\pgfqpoint{2.977891in}{1.099109in}}%
\pgfpathlineto{\pgfqpoint{2.982400in}{1.095361in}}%
\pgfpathlineto{\pgfqpoint{2.986909in}{1.065771in}}%
\pgfpathlineto{\pgfqpoint{2.991418in}{1.099331in}}%
\pgfpathlineto{\pgfqpoint{2.995927in}{1.105625in}}%
\pgfpathlineto{\pgfqpoint{3.000436in}{1.066133in}}%
\pgfpathlineto{\pgfqpoint{3.004945in}{1.092331in}}%
\pgfpathlineto{\pgfqpoint{3.009455in}{1.109840in}}%
\pgfpathlineto{\pgfqpoint{3.013964in}{1.104174in}}%
\pgfpathlineto{\pgfqpoint{3.018473in}{1.134142in}}%
\pgfpathlineto{\pgfqpoint{3.022982in}{1.125702in}}%
\pgfpathlineto{\pgfqpoint{3.027491in}{1.088951in}}%
\pgfpathlineto{\pgfqpoint{3.032000in}{1.119475in}}%
\pgfpathlineto{\pgfqpoint{3.036509in}{1.119342in}}%
\pgfpathlineto{\pgfqpoint{3.041018in}{1.123417in}}%
\pgfpathlineto{\pgfqpoint{3.045527in}{1.122716in}}%
\pgfpathlineto{\pgfqpoint{3.050036in}{1.120487in}}%
\pgfpathlineto{\pgfqpoint{3.054545in}{1.099732in}}%
\pgfpathlineto{\pgfqpoint{3.059055in}{1.109306in}}%
\pgfpathlineto{\pgfqpoint{3.063564in}{1.136183in}}%
\pgfpathlineto{\pgfqpoint{3.068073in}{1.136344in}}%
\pgfpathlineto{\pgfqpoint{3.072582in}{1.130545in}}%
\pgfpathlineto{\pgfqpoint{3.077091in}{1.103307in}}%
\pgfpathlineto{\pgfqpoint{3.081600in}{1.128593in}}%
\pgfpathlineto{\pgfqpoint{3.086109in}{1.120270in}}%
\pgfpathlineto{\pgfqpoint{3.090618in}{1.124902in}}%
\pgfpathlineto{\pgfqpoint{3.095127in}{1.149949in}}%
\pgfpathlineto{\pgfqpoint{3.099636in}{1.103168in}}%
\pgfpathlineto{\pgfqpoint{3.104145in}{1.116022in}}%
\pgfpathlineto{\pgfqpoint{3.108655in}{1.147647in}}%
\pgfpathlineto{\pgfqpoint{3.113164in}{1.152156in}}%
\pgfpathlineto{\pgfqpoint{3.117673in}{1.148548in}}%
\pgfpathlineto{\pgfqpoint{3.122182in}{1.138112in}}%
\pgfpathlineto{\pgfqpoint{3.126691in}{1.116211in}}%
\pgfpathlineto{\pgfqpoint{3.131200in}{1.143066in}}%
\pgfpathlineto{\pgfqpoint{3.135709in}{1.123339in}}%
\pgfpathlineto{\pgfqpoint{3.140218in}{1.147903in}}%
\pgfpathlineto{\pgfqpoint{3.144727in}{1.122933in}}%
\pgfpathlineto{\pgfqpoint{3.149236in}{1.147775in}}%
\pgfpathlineto{\pgfqpoint{3.153745in}{1.142071in}}%
\pgfpathlineto{\pgfqpoint{3.158255in}{1.144256in}}%
\pgfpathlineto{\pgfqpoint{3.162764in}{1.168981in}}%
\pgfpathlineto{\pgfqpoint{3.167273in}{1.164994in}}%
\pgfpathlineto{\pgfqpoint{3.171782in}{1.170855in}}%
\pgfpathlineto{\pgfqpoint{3.176291in}{1.167674in}}%
\pgfpathlineto{\pgfqpoint{3.180800in}{1.149966in}}%
\pgfpathlineto{\pgfqpoint{3.185309in}{1.157950in}}%
\pgfpathlineto{\pgfqpoint{3.189818in}{1.183403in}}%
\pgfpathlineto{\pgfqpoint{3.194327in}{1.171283in}}%
\pgfpathlineto{\pgfqpoint{3.198836in}{1.179367in}}%
\pgfpathlineto{\pgfqpoint{3.203345in}{1.159707in}}%
\pgfpathlineto{\pgfqpoint{3.207855in}{1.164327in}}%
\pgfpathlineto{\pgfqpoint{3.212364in}{1.174546in}}%
\pgfpathlineto{\pgfqpoint{3.216873in}{1.174135in}}%
\pgfpathlineto{\pgfqpoint{3.221382in}{1.193350in}}%
\pgfpathlineto{\pgfqpoint{3.225891in}{1.163243in}}%
\pgfpathlineto{\pgfqpoint{3.230400in}{1.168447in}}%
\pgfpathlineto{\pgfqpoint{3.234909in}{1.176804in}}%
\pgfpathlineto{\pgfqpoint{3.239418in}{1.187507in}}%
\pgfpathlineto{\pgfqpoint{3.243927in}{1.161898in}}%
\pgfpathlineto{\pgfqpoint{3.252945in}{1.202813in}}%
\pgfpathlineto{\pgfqpoint{3.257455in}{1.179011in}}%
\pgfpathlineto{\pgfqpoint{3.261964in}{1.216530in}}%
\pgfpathlineto{\pgfqpoint{3.266473in}{1.194329in}}%
\pgfpathlineto{\pgfqpoint{3.275491in}{1.198193in}}%
\pgfpathlineto{\pgfqpoint{3.280000in}{1.206939in}}%
\pgfpathlineto{\pgfqpoint{3.284509in}{1.194251in}}%
\pgfpathlineto{\pgfqpoint{3.289018in}{1.216880in}}%
\pgfpathlineto{\pgfqpoint{3.293527in}{1.211387in}}%
\pgfpathlineto{\pgfqpoint{3.298036in}{1.194640in}}%
\pgfpathlineto{\pgfqpoint{3.302545in}{1.238052in}}%
\pgfpathlineto{\pgfqpoint{3.307055in}{1.229390in}}%
\pgfpathlineto{\pgfqpoint{3.316073in}{1.217686in}}%
\pgfpathlineto{\pgfqpoint{3.320582in}{1.236807in}}%
\pgfpathlineto{\pgfqpoint{3.325091in}{1.207200in}}%
\pgfpathlineto{\pgfqpoint{3.329600in}{1.198421in}}%
\pgfpathlineto{\pgfqpoint{3.334109in}{1.233338in}}%
\pgfpathlineto{\pgfqpoint{3.343127in}{1.207684in}}%
\pgfpathlineto{\pgfqpoint{3.347636in}{1.225353in}}%
\pgfpathlineto{\pgfqpoint{3.352145in}{1.222768in}}%
\pgfpathlineto{\pgfqpoint{3.356655in}{1.240649in}}%
\pgfpathlineto{\pgfqpoint{3.361164in}{1.228689in}}%
\pgfpathlineto{\pgfqpoint{3.365673in}{1.235406in}}%
\pgfpathlineto{\pgfqpoint{3.370182in}{1.247454in}}%
\pgfpathlineto{\pgfqpoint{3.374691in}{1.200122in}}%
\pgfpathlineto{\pgfqpoint{3.379200in}{1.249083in}}%
\pgfpathlineto{\pgfqpoint{3.383709in}{1.232570in}}%
\pgfpathlineto{\pgfqpoint{3.388218in}{1.222902in}}%
\pgfpathlineto{\pgfqpoint{3.401745in}{1.260292in}}%
\pgfpathlineto{\pgfqpoint{3.406255in}{1.263295in}}%
\pgfpathlineto{\pgfqpoint{3.410764in}{1.233821in}}%
\pgfpathlineto{\pgfqpoint{3.415273in}{1.233093in}}%
\pgfpathlineto{\pgfqpoint{3.419782in}{1.258802in}}%
\pgfpathlineto{\pgfqpoint{3.424291in}{1.255333in}}%
\pgfpathlineto{\pgfqpoint{3.428800in}{1.267142in}}%
\pgfpathlineto{\pgfqpoint{3.433309in}{1.269867in}}%
\pgfpathlineto{\pgfqpoint{3.437818in}{1.274231in}}%
\pgfpathlineto{\pgfqpoint{3.442327in}{1.269539in}}%
\pgfpathlineto{\pgfqpoint{3.446836in}{1.243679in}}%
\pgfpathlineto{\pgfqpoint{3.451345in}{1.249701in}}%
\pgfpathlineto{\pgfqpoint{3.455855in}{1.290483in}}%
\pgfpathlineto{\pgfqpoint{3.464873in}{1.258530in}}%
\pgfpathlineto{\pgfqpoint{3.469382in}{1.271846in}}%
\pgfpathlineto{\pgfqpoint{3.473891in}{1.290627in}}%
\pgfpathlineto{\pgfqpoint{3.478400in}{1.277395in}}%
\pgfpathlineto{\pgfqpoint{3.482909in}{1.242929in}}%
\pgfpathlineto{\pgfqpoint{3.487418in}{1.277673in}}%
\pgfpathlineto{\pgfqpoint{3.491927in}{1.300786in}}%
\pgfpathlineto{\pgfqpoint{3.496436in}{1.296393in}}%
\pgfpathlineto{\pgfqpoint{3.500945in}{1.249606in}}%
\pgfpathlineto{\pgfqpoint{3.505455in}{1.280091in}}%
\pgfpathlineto{\pgfqpoint{3.509964in}{1.264740in}}%
\pgfpathlineto{\pgfqpoint{3.514473in}{1.290539in}}%
\pgfpathlineto{\pgfqpoint{3.518982in}{1.303821in}}%
\pgfpathlineto{\pgfqpoint{3.523491in}{1.295854in}}%
\pgfpathlineto{\pgfqpoint{3.528000in}{1.291150in}}%
\pgfpathlineto{\pgfqpoint{3.532509in}{1.320234in}}%
\pgfpathlineto{\pgfqpoint{3.537018in}{1.302365in}}%
\pgfpathlineto{\pgfqpoint{3.541527in}{1.321797in}}%
\pgfpathlineto{\pgfqpoint{3.546036in}{1.320229in}}%
\pgfpathlineto{\pgfqpoint{3.550545in}{1.294909in}}%
\pgfpathlineto{\pgfqpoint{3.555055in}{1.293369in}}%
\pgfpathlineto{\pgfqpoint{3.559564in}{1.327929in}}%
\pgfpathlineto{\pgfqpoint{3.564073in}{1.297716in}}%
\pgfpathlineto{\pgfqpoint{3.568582in}{1.312289in}}%
\pgfpathlineto{\pgfqpoint{3.573091in}{1.302548in}}%
\pgfpathlineto{\pgfqpoint{3.577600in}{1.319751in}}%
\pgfpathlineto{\pgfqpoint{3.582109in}{1.309698in}}%
\pgfpathlineto{\pgfqpoint{3.586618in}{1.309326in}}%
\pgfpathlineto{\pgfqpoint{3.591127in}{1.328218in}}%
\pgfpathlineto{\pgfqpoint{3.595636in}{1.312473in}}%
\pgfpathlineto{\pgfqpoint{3.604655in}{1.333117in}}%
\pgfpathlineto{\pgfqpoint{3.609164in}{1.303610in}}%
\pgfpathlineto{\pgfqpoint{3.613673in}{1.329303in}}%
\pgfpathlineto{\pgfqpoint{3.618182in}{1.330109in}}%
\pgfpathlineto{\pgfqpoint{3.622691in}{1.338627in}}%
\pgfpathlineto{\pgfqpoint{3.627200in}{1.324460in}}%
\pgfpathlineto{\pgfqpoint{3.631709in}{1.352743in}}%
\pgfpathlineto{\pgfqpoint{3.636218in}{1.326222in}}%
\pgfpathlineto{\pgfqpoint{3.640727in}{1.326673in}}%
\pgfpathlineto{\pgfqpoint{3.645236in}{1.343725in}}%
\pgfpathlineto{\pgfqpoint{3.649745in}{1.340773in}}%
\pgfpathlineto{\pgfqpoint{3.654255in}{1.334095in}}%
\pgfpathlineto{\pgfqpoint{3.658764in}{1.362334in}}%
\pgfpathlineto{\pgfqpoint{3.663273in}{1.335758in}}%
\pgfpathlineto{\pgfqpoint{3.672291in}{1.362996in}}%
\pgfpathlineto{\pgfqpoint{3.681309in}{1.338043in}}%
\pgfpathlineto{\pgfqpoint{3.685818in}{1.390907in}}%
\pgfpathlineto{\pgfqpoint{3.690327in}{1.365242in}}%
\pgfpathlineto{\pgfqpoint{3.694836in}{1.355829in}}%
\pgfpathlineto{\pgfqpoint{3.699345in}{1.390140in}}%
\pgfpathlineto{\pgfqpoint{3.703855in}{1.358943in}}%
\pgfpathlineto{\pgfqpoint{3.708364in}{1.362707in}}%
\pgfpathlineto{\pgfqpoint{3.712873in}{1.373326in}}%
\pgfpathlineto{\pgfqpoint{3.717382in}{1.351820in}}%
\pgfpathlineto{\pgfqpoint{3.721891in}{1.399325in}}%
\pgfpathlineto{\pgfqpoint{3.726400in}{1.420130in}}%
\pgfpathlineto{\pgfqpoint{3.730909in}{1.422015in}}%
\pgfpathlineto{\pgfqpoint{3.735418in}{1.384352in}}%
\pgfpathlineto{\pgfqpoint{3.739927in}{1.373421in}}%
\pgfpathlineto{\pgfqpoint{3.744436in}{1.384041in}}%
\pgfpathlineto{\pgfqpoint{3.748945in}{1.367678in}}%
\pgfpathlineto{\pgfqpoint{3.753455in}{1.374905in}}%
\pgfpathlineto{\pgfqpoint{3.757964in}{1.396445in}}%
\pgfpathlineto{\pgfqpoint{3.762473in}{1.396617in}}%
\pgfpathlineto{\pgfqpoint{3.766982in}{1.383935in}}%
\pgfpathlineto{\pgfqpoint{3.771491in}{1.408593in}}%
\pgfpathlineto{\pgfqpoint{3.776000in}{1.386531in}}%
\pgfpathlineto{\pgfqpoint{3.780509in}{1.394048in}}%
\pgfpathlineto{\pgfqpoint{3.785018in}{1.410689in}}%
\pgfpathlineto{\pgfqpoint{3.789527in}{1.410311in}}%
\pgfpathlineto{\pgfqpoint{3.798545in}{1.430288in}}%
\pgfpathlineto{\pgfqpoint{3.803055in}{1.429354in}}%
\pgfpathlineto{\pgfqpoint{3.807564in}{1.414031in}}%
\pgfpathlineto{\pgfqpoint{3.816582in}{1.435137in}}%
\pgfpathlineto{\pgfqpoint{3.821091in}{1.415905in}}%
\pgfpathlineto{\pgfqpoint{3.830109in}{1.391630in}}%
\pgfpathlineto{\pgfqpoint{3.839127in}{1.412424in}}%
\pgfpathlineto{\pgfqpoint{3.843636in}{1.432257in}}%
\pgfpathlineto{\pgfqpoint{3.848145in}{1.444177in}}%
\pgfpathlineto{\pgfqpoint{3.852655in}{1.451678in}}%
\pgfpathlineto{\pgfqpoint{3.857164in}{1.438912in}}%
\pgfpathlineto{\pgfqpoint{3.861673in}{1.444488in}}%
\pgfpathlineto{\pgfqpoint{3.866182in}{1.419569in}}%
\pgfpathlineto{\pgfqpoint{3.870691in}{1.443971in}}%
\pgfpathlineto{\pgfqpoint{3.875200in}{1.445183in}}%
\pgfpathlineto{\pgfqpoint{3.879709in}{1.454035in}}%
\pgfpathlineto{\pgfqpoint{3.884218in}{1.442153in}}%
\pgfpathlineto{\pgfqpoint{3.888727in}{1.423728in}}%
\pgfpathlineto{\pgfqpoint{3.893236in}{1.470442in}}%
\pgfpathlineto{\pgfqpoint{3.897745in}{1.451010in}}%
\pgfpathlineto{\pgfqpoint{3.902255in}{1.438851in}}%
\pgfpathlineto{\pgfqpoint{3.906764in}{1.441992in}}%
\pgfpathlineto{\pgfqpoint{3.911273in}{1.485215in}}%
\pgfpathlineto{\pgfqpoint{3.915782in}{1.435003in}}%
\pgfpathlineto{\pgfqpoint{3.920291in}{1.485399in}}%
\pgfpathlineto{\pgfqpoint{3.924800in}{1.438345in}}%
\pgfpathlineto{\pgfqpoint{3.929309in}{1.437133in}}%
\pgfpathlineto{\pgfqpoint{3.933818in}{1.461057in}}%
\pgfpathlineto{\pgfqpoint{3.938327in}{1.503669in}}%
\pgfpathlineto{\pgfqpoint{3.942836in}{1.505459in}}%
\pgfpathlineto{\pgfqpoint{3.947345in}{1.522990in}}%
\pgfpathlineto{\pgfqpoint{3.951855in}{1.489018in}}%
\pgfpathlineto{\pgfqpoint{3.956364in}{1.479255in}}%
\pgfpathlineto{\pgfqpoint{3.960873in}{1.475068in}}%
\pgfpathlineto{\pgfqpoint{3.965382in}{1.467790in}}%
\pgfpathlineto{\pgfqpoint{3.969891in}{1.501239in}}%
\pgfpathlineto{\pgfqpoint{3.974400in}{1.506215in}}%
\pgfpathlineto{\pgfqpoint{3.978909in}{1.491053in}}%
\pgfpathlineto{\pgfqpoint{3.983418in}{1.458561in}}%
\pgfpathlineto{\pgfqpoint{3.987927in}{1.485282in}}%
\pgfpathlineto{\pgfqpoint{3.992436in}{1.472055in}}%
\pgfpathlineto{\pgfqpoint{3.996945in}{1.518547in}}%
\pgfpathlineto{\pgfqpoint{4.001455in}{1.487806in}}%
\pgfpathlineto{\pgfqpoint{4.005964in}{1.477431in}}%
\pgfpathlineto{\pgfqpoint{4.010473in}{1.508150in}}%
\pgfpathlineto{\pgfqpoint{4.014982in}{1.489508in}}%
\pgfpathlineto{\pgfqpoint{4.019491in}{1.528172in}}%
\pgfpathlineto{\pgfqpoint{4.024000in}{1.489663in}}%
\pgfpathlineto{\pgfqpoint{4.028509in}{1.497853in}}%
\pgfpathlineto{\pgfqpoint{4.033018in}{1.482213in}}%
\pgfpathlineto{\pgfqpoint{4.037527in}{1.529200in}}%
\pgfpathlineto{\pgfqpoint{4.042036in}{1.533843in}}%
\pgfpathlineto{\pgfqpoint{4.046545in}{1.509857in}}%
\pgfpathlineto{\pgfqpoint{4.051055in}{1.548777in}}%
\pgfpathlineto{\pgfqpoint{4.055564in}{1.551935in}}%
\pgfpathlineto{\pgfqpoint{4.060073in}{1.530546in}}%
\pgfpathlineto{\pgfqpoint{4.064582in}{1.502535in}}%
\pgfpathlineto{\pgfqpoint{4.069091in}{1.537051in}}%
\pgfpathlineto{\pgfqpoint{4.073600in}{1.524841in}}%
\pgfpathlineto{\pgfqpoint{4.078109in}{1.528083in}}%
\pgfpathlineto{\pgfqpoint{4.082618in}{1.524902in}}%
\pgfpathlineto{\pgfqpoint{4.087127in}{1.530662in}}%
\pgfpathlineto{\pgfqpoint{4.091636in}{1.556922in}}%
\pgfpathlineto{\pgfqpoint{4.096145in}{1.532492in}}%
\pgfpathlineto{\pgfqpoint{4.105164in}{1.589342in}}%
\pgfpathlineto{\pgfqpoint{4.109673in}{1.530073in}}%
\pgfpathlineto{\pgfqpoint{4.118691in}{1.579679in}}%
\pgfpathlineto{\pgfqpoint{4.123200in}{1.568170in}}%
\pgfpathlineto{\pgfqpoint{4.127709in}{1.590093in}}%
\pgfpathlineto{\pgfqpoint{4.132218in}{1.523807in}}%
\pgfpathlineto{\pgfqpoint{4.136727in}{1.557033in}}%
\pgfpathlineto{\pgfqpoint{4.145745in}{1.600145in}}%
\pgfpathlineto{\pgfqpoint{4.150255in}{1.588898in}}%
\pgfpathlineto{\pgfqpoint{4.154764in}{1.561854in}}%
\pgfpathlineto{\pgfqpoint{4.159273in}{1.594558in}}%
\pgfpathlineto{\pgfqpoint{4.163782in}{1.594079in}}%
\pgfpathlineto{\pgfqpoint{4.168291in}{1.564612in}}%
\pgfpathlineto{\pgfqpoint{4.172800in}{1.579763in}}%
\pgfpathlineto{\pgfqpoint{4.177309in}{1.576477in}}%
\pgfpathlineto{\pgfqpoint{4.181818in}{1.582370in}}%
\pgfpathlineto{\pgfqpoint{4.195345in}{1.644130in}}%
\pgfpathlineto{\pgfqpoint{4.199855in}{1.588825in}}%
\pgfpathlineto{\pgfqpoint{4.204364in}{1.625260in}}%
\pgfpathlineto{\pgfqpoint{4.208873in}{1.562643in}}%
\pgfpathlineto{\pgfqpoint{4.213382in}{1.611949in}}%
\pgfpathlineto{\pgfqpoint{4.217891in}{1.616497in}}%
\pgfpathlineto{\pgfqpoint{4.226909in}{1.594546in}}%
\pgfpathlineto{\pgfqpoint{4.231418in}{1.615446in}}%
\pgfpathlineto{\pgfqpoint{4.235927in}{1.594569in}}%
\pgfpathlineto{\pgfqpoint{4.240436in}{1.620634in}}%
\pgfpathlineto{\pgfqpoint{4.244945in}{1.608218in}}%
\pgfpathlineto{\pgfqpoint{4.249455in}{1.613873in}}%
\pgfpathlineto{\pgfqpoint{4.253964in}{1.637909in}}%
\pgfpathlineto{\pgfqpoint{4.258473in}{1.641845in}}%
\pgfpathlineto{\pgfqpoint{4.262982in}{1.624426in}}%
\pgfpathlineto{\pgfqpoint{4.267491in}{1.632421in}}%
\pgfpathlineto{\pgfqpoint{4.272000in}{1.667983in}}%
\pgfpathlineto{\pgfqpoint{4.276509in}{1.634395in}}%
\pgfpathlineto{\pgfqpoint{4.281018in}{1.584911in}}%
\pgfpathlineto{\pgfqpoint{4.285527in}{1.653104in}}%
\pgfpathlineto{\pgfqpoint{4.290036in}{1.669095in}}%
\pgfpathlineto{\pgfqpoint{4.294545in}{1.635518in}}%
\pgfpathlineto{\pgfqpoint{4.299055in}{1.674755in}}%
\pgfpathlineto{\pgfqpoint{4.303564in}{1.681132in}}%
\pgfpathlineto{\pgfqpoint{4.308073in}{1.674510in}}%
\pgfpathlineto{\pgfqpoint{4.312582in}{1.635735in}}%
\pgfpathlineto{\pgfqpoint{4.321600in}{1.661694in}}%
\pgfpathlineto{\pgfqpoint{4.326109in}{1.703700in}}%
\pgfpathlineto{\pgfqpoint{4.330618in}{1.643585in}}%
\pgfpathlineto{\pgfqpoint{4.335127in}{1.681271in}}%
\pgfpathlineto{\pgfqpoint{4.339636in}{1.658030in}}%
\pgfpathlineto{\pgfqpoint{4.344145in}{1.617181in}}%
\pgfpathlineto{\pgfqpoint{4.348655in}{1.669078in}}%
\pgfpathlineto{\pgfqpoint{4.353164in}{1.699530in}}%
\pgfpathlineto{\pgfqpoint{4.357673in}{1.645270in}}%
\pgfpathlineto{\pgfqpoint{4.362182in}{1.640833in}}%
\pgfpathlineto{\pgfqpoint{4.366691in}{1.677668in}}%
\pgfpathlineto{\pgfqpoint{4.371200in}{1.660732in}}%
\pgfpathlineto{\pgfqpoint{4.375709in}{1.686308in}}%
\pgfpathlineto{\pgfqpoint{4.380218in}{1.700725in}}%
\pgfpathlineto{\pgfqpoint{4.384727in}{1.687971in}}%
\pgfpathlineto{\pgfqpoint{4.389236in}{1.702566in}}%
\pgfpathlineto{\pgfqpoint{4.393745in}{1.700336in}}%
\pgfpathlineto{\pgfqpoint{4.398255in}{1.659943in}}%
\pgfpathlineto{\pgfqpoint{4.402764in}{1.668316in}}%
\pgfpathlineto{\pgfqpoint{4.407273in}{1.710800in}}%
\pgfpathlineto{\pgfqpoint{4.416291in}{1.727041in}}%
\pgfpathlineto{\pgfqpoint{4.420800in}{1.716288in}}%
\pgfpathlineto{\pgfqpoint{4.425309in}{1.687281in}}%
\pgfpathlineto{\pgfqpoint{4.429818in}{1.716977in}}%
\pgfpathlineto{\pgfqpoint{4.434327in}{1.711489in}}%
\pgfpathlineto{\pgfqpoint{4.438836in}{1.708470in}}%
\pgfpathlineto{\pgfqpoint{4.443345in}{1.715359in}}%
\pgfpathlineto{\pgfqpoint{4.447855in}{1.712407in}}%
\pgfpathlineto{\pgfqpoint{4.452364in}{1.740662in}}%
\pgfpathlineto{\pgfqpoint{4.456873in}{1.717488in}}%
\pgfpathlineto{\pgfqpoint{4.461382in}{1.749292in}}%
\pgfpathlineto{\pgfqpoint{4.465891in}{1.702593in}}%
\pgfpathlineto{\pgfqpoint{4.470400in}{1.708203in}}%
\pgfpathlineto{\pgfqpoint{4.474909in}{1.695371in}}%
\pgfpathlineto{\pgfqpoint{4.479418in}{1.765832in}}%
\pgfpathlineto{\pgfqpoint{4.483927in}{1.719412in}}%
\pgfpathlineto{\pgfqpoint{4.488436in}{1.747935in}}%
\pgfpathlineto{\pgfqpoint{4.492945in}{1.815461in}}%
\pgfpathlineto{\pgfqpoint{4.497455in}{1.779221in}}%
\pgfpathlineto{\pgfqpoint{4.501964in}{1.777497in}}%
\pgfpathlineto{\pgfqpoint{4.506473in}{1.777286in}}%
\pgfpathlineto{\pgfqpoint{4.510982in}{1.782473in}}%
\pgfpathlineto{\pgfqpoint{4.515491in}{1.736720in}}%
\pgfpathlineto{\pgfqpoint{4.520000in}{1.755952in}}%
\pgfpathlineto{\pgfqpoint{4.524509in}{1.753467in}}%
\pgfpathlineto{\pgfqpoint{4.529018in}{1.774634in}}%
\pgfpathlineto{\pgfqpoint{4.533527in}{1.776585in}}%
\pgfpathlineto{\pgfqpoint{4.538036in}{1.765782in}}%
\pgfpathlineto{\pgfqpoint{4.542545in}{1.796373in}}%
\pgfpathlineto{\pgfqpoint{4.547055in}{1.802578in}}%
\pgfpathlineto{\pgfqpoint{4.551564in}{1.775223in}}%
\pgfpathlineto{\pgfqpoint{4.556073in}{1.727763in}}%
\pgfpathlineto{\pgfqpoint{4.565091in}{1.795222in}}%
\pgfpathlineto{\pgfqpoint{4.569600in}{1.780633in}}%
\pgfpathlineto{\pgfqpoint{4.574109in}{1.788139in}}%
\pgfpathlineto{\pgfqpoint{4.578618in}{1.776619in}}%
\pgfpathlineto{\pgfqpoint{4.583127in}{1.782718in}}%
\pgfpathlineto{\pgfqpoint{4.587636in}{1.841009in}}%
\pgfpathlineto{\pgfqpoint{4.592145in}{1.787450in}}%
\pgfpathlineto{\pgfqpoint{4.596655in}{1.819875in}}%
\pgfpathlineto{\pgfqpoint{4.601164in}{1.776563in}}%
\pgfpathlineto{\pgfqpoint{4.605673in}{1.798853in}}%
\pgfpathlineto{\pgfqpoint{4.610182in}{1.801872in}}%
\pgfpathlineto{\pgfqpoint{4.614691in}{1.799137in}}%
\pgfpathlineto{\pgfqpoint{4.619200in}{1.804552in}}%
\pgfpathlineto{\pgfqpoint{4.623709in}{1.828766in}}%
\pgfpathlineto{\pgfqpoint{4.628218in}{1.840403in}}%
\pgfpathlineto{\pgfqpoint{4.632727in}{1.834187in}}%
\pgfpathlineto{\pgfqpoint{4.637236in}{1.799370in}}%
\pgfpathlineto{\pgfqpoint{4.641745in}{1.824051in}}%
\pgfpathlineto{\pgfqpoint{4.646255in}{1.785932in}}%
\pgfpathlineto{\pgfqpoint{4.650764in}{1.803446in}}%
\pgfpathlineto{\pgfqpoint{4.655273in}{1.803023in}}%
\pgfpathlineto{\pgfqpoint{4.659782in}{1.846652in}}%
\pgfpathlineto{\pgfqpoint{4.664291in}{1.830745in}}%
\pgfpathlineto{\pgfqpoint{4.668800in}{1.861141in}}%
\pgfpathlineto{\pgfqpoint{4.673309in}{1.824162in}}%
\pgfpathlineto{\pgfqpoint{4.677818in}{1.879851in}}%
\pgfpathlineto{\pgfqpoint{4.682327in}{1.858278in}}%
\pgfpathlineto{\pgfqpoint{4.686836in}{1.858767in}}%
\pgfpathlineto{\pgfqpoint{4.691345in}{1.845479in}}%
\pgfpathlineto{\pgfqpoint{4.695855in}{1.861153in}}%
\pgfpathlineto{\pgfqpoint{4.704873in}{1.839385in}}%
\pgfpathlineto{\pgfqpoint{4.709382in}{1.864305in}}%
\pgfpathlineto{\pgfqpoint{4.713891in}{1.868847in}}%
\pgfpathlineto{\pgfqpoint{4.718400in}{1.882692in}}%
\pgfpathlineto{\pgfqpoint{4.722909in}{1.861647in}}%
\pgfpathlineto{\pgfqpoint{4.727418in}{1.863844in}}%
\pgfpathlineto{\pgfqpoint{4.731927in}{1.834954in}}%
\pgfpathlineto{\pgfqpoint{4.736436in}{1.922162in}}%
\pgfpathlineto{\pgfqpoint{4.740945in}{1.901023in}}%
\pgfpathlineto{\pgfqpoint{4.745455in}{1.865689in}}%
\pgfpathlineto{\pgfqpoint{4.749964in}{1.881902in}}%
\pgfpathlineto{\pgfqpoint{4.754473in}{1.876654in}}%
\pgfpathlineto{\pgfqpoint{4.758982in}{1.886156in}}%
\pgfpathlineto{\pgfqpoint{4.763491in}{1.964106in}}%
\pgfpathlineto{\pgfqpoint{4.772509in}{1.902546in}}%
\pgfpathlineto{\pgfqpoint{4.777018in}{1.883292in}}%
\pgfpathlineto{\pgfqpoint{4.781527in}{1.929662in}}%
\pgfpathlineto{\pgfqpoint{4.786036in}{1.865806in}}%
\pgfpathlineto{\pgfqpoint{4.790545in}{1.862676in}}%
\pgfpathlineto{\pgfqpoint{4.795055in}{1.947554in}}%
\pgfpathlineto{\pgfqpoint{4.799564in}{1.869359in}}%
\pgfpathlineto{\pgfqpoint{4.804073in}{1.942378in}}%
\pgfpathlineto{\pgfqpoint{4.808582in}{1.927205in}}%
\pgfpathlineto{\pgfqpoint{4.813091in}{1.938475in}}%
\pgfpathlineto{\pgfqpoint{4.817600in}{1.930758in}}%
\pgfpathlineto{\pgfqpoint{4.822109in}{1.940032in}}%
\pgfpathlineto{\pgfqpoint{4.826618in}{1.982321in}}%
\pgfpathlineto{\pgfqpoint{4.831127in}{1.989771in}}%
\pgfpathlineto{\pgfqpoint{4.835636in}{1.943245in}}%
\pgfpathlineto{\pgfqpoint{4.840145in}{1.934822in}}%
\pgfpathlineto{\pgfqpoint{4.844655in}{1.929712in}}%
\pgfpathlineto{\pgfqpoint{4.849164in}{1.965763in}}%
\pgfpathlineto{\pgfqpoint{4.853673in}{1.951202in}}%
\pgfpathlineto{\pgfqpoint{4.858182in}{1.931775in}}%
\pgfpathlineto{\pgfqpoint{4.862691in}{1.958791in}}%
\pgfpathlineto{\pgfqpoint{4.867200in}{1.992779in}}%
\pgfpathlineto{\pgfqpoint{4.871709in}{1.991767in}}%
\pgfpathlineto{\pgfqpoint{4.876218in}{1.955466in}}%
\pgfpathlineto{\pgfqpoint{4.880727in}{1.961493in}}%
\pgfpathlineto{\pgfqpoint{4.885236in}{2.025650in}}%
\pgfpathlineto{\pgfqpoint{4.894255in}{1.932732in}}%
\pgfpathlineto{\pgfqpoint{4.898764in}{2.007157in}}%
\pgfpathlineto{\pgfqpoint{4.903273in}{1.999651in}}%
\pgfpathlineto{\pgfqpoint{4.907782in}{2.012150in}}%
\pgfpathlineto{\pgfqpoint{4.912291in}{1.995170in}}%
\pgfpathlineto{\pgfqpoint{4.916800in}{1.968338in}}%
\pgfpathlineto{\pgfqpoint{4.921309in}{2.020557in}}%
\pgfpathlineto{\pgfqpoint{4.930327in}{1.974153in}}%
\pgfpathlineto{\pgfqpoint{4.934836in}{2.012545in}}%
\pgfpathlineto{\pgfqpoint{4.939345in}{1.992985in}}%
\pgfpathlineto{\pgfqpoint{4.943855in}{2.000485in}}%
\pgfpathlineto{\pgfqpoint{4.948364in}{2.041017in}}%
\pgfpathlineto{\pgfqpoint{4.952873in}{1.991289in}}%
\pgfpathlineto{\pgfqpoint{4.957382in}{1.980920in}}%
\pgfpathlineto{\pgfqpoint{4.961891in}{2.035485in}}%
\pgfpathlineto{\pgfqpoint{4.966400in}{1.996482in}}%
\pgfpathlineto{\pgfqpoint{4.970909in}{2.055585in}}%
\pgfpathlineto{\pgfqpoint{4.975418in}{2.062462in}}%
\pgfpathlineto{\pgfqpoint{4.979927in}{2.004283in}}%
\pgfpathlineto{\pgfqpoint{4.984436in}{2.057475in}}%
\pgfpathlineto{\pgfqpoint{4.988945in}{2.012995in}}%
\pgfpathlineto{\pgfqpoint{4.993455in}{2.061734in}}%
\pgfpathlineto{\pgfqpoint{4.997964in}{2.028958in}}%
\pgfpathlineto{\pgfqpoint{5.002473in}{2.045727in}}%
\pgfpathlineto{\pgfqpoint{5.006982in}{2.067928in}}%
\pgfpathlineto{\pgfqpoint{5.011491in}{1.966558in}}%
\pgfpathlineto{\pgfqpoint{5.016000in}{2.101638in}}%
\pgfpathlineto{\pgfqpoint{5.020509in}{2.068006in}}%
\pgfpathlineto{\pgfqpoint{5.025018in}{2.013979in}}%
\pgfpathlineto{\pgfqpoint{5.029527in}{2.078992in}}%
\pgfpathlineto{\pgfqpoint{5.034036in}{2.069929in}}%
\pgfpathlineto{\pgfqpoint{5.038545in}{2.058042in}}%
\pgfpathlineto{\pgfqpoint{5.043055in}{2.069651in}}%
\pgfpathlineto{\pgfqpoint{5.047564in}{2.077041in}}%
\pgfpathlineto{\pgfqpoint{5.052073in}{2.059627in}}%
\pgfpathlineto{\pgfqpoint{5.056582in}{2.024705in}}%
\pgfpathlineto{\pgfqpoint{5.061091in}{2.020234in}}%
\pgfpathlineto{\pgfqpoint{5.065600in}{2.098546in}}%
\pgfpathlineto{\pgfqpoint{5.070109in}{2.056574in}}%
\pgfpathlineto{\pgfqpoint{5.074618in}{2.085358in}}%
\pgfpathlineto{\pgfqpoint{5.079127in}{2.096639in}}%
\pgfpathlineto{\pgfqpoint{5.083636in}{2.097512in}}%
\pgfpathlineto{\pgfqpoint{5.088145in}{2.101215in}}%
\pgfpathlineto{\pgfqpoint{5.092655in}{2.130622in}}%
\pgfpathlineto{\pgfqpoint{5.097164in}{2.139384in}}%
\pgfpathlineto{\pgfqpoint{5.101673in}{2.109299in}}%
\pgfpathlineto{\pgfqpoint{5.106182in}{2.062234in}}%
\pgfpathlineto{\pgfqpoint{5.110691in}{2.078397in}}%
\pgfpathlineto{\pgfqpoint{5.115200in}{2.129888in}}%
\pgfpathlineto{\pgfqpoint{5.119709in}{2.058259in}}%
\pgfpathlineto{\pgfqpoint{5.128727in}{2.137416in}}%
\pgfpathlineto{\pgfqpoint{5.133236in}{2.118968in}}%
\pgfpathlineto{\pgfqpoint{5.137745in}{2.095800in}}%
\pgfpathlineto{\pgfqpoint{5.142255in}{2.131361in}}%
\pgfpathlineto{\pgfqpoint{5.146764in}{2.122515in}}%
\pgfpathlineto{\pgfqpoint{5.151273in}{2.124067in}}%
\pgfpathlineto{\pgfqpoint{5.155782in}{2.112135in}}%
\pgfpathlineto{\pgfqpoint{5.164800in}{2.188062in}}%
\pgfpathlineto{\pgfqpoint{5.169309in}{2.145623in}}%
\pgfpathlineto{\pgfqpoint{5.173818in}{2.138523in}}%
\pgfpathlineto{\pgfqpoint{5.178327in}{2.151539in}}%
\pgfpathlineto{\pgfqpoint{5.182836in}{2.153968in}}%
\pgfpathlineto{\pgfqpoint{5.187345in}{2.139918in}}%
\pgfpathlineto{\pgfqpoint{5.191855in}{2.166478in}}%
\pgfpathlineto{\pgfqpoint{5.196364in}{2.137772in}}%
\pgfpathlineto{\pgfqpoint{5.200873in}{2.174412in}}%
\pgfpathlineto{\pgfqpoint{5.205382in}{2.247387in}}%
\pgfpathlineto{\pgfqpoint{5.209891in}{2.197408in}}%
\pgfpathlineto{\pgfqpoint{5.214400in}{2.167396in}}%
\pgfpathlineto{\pgfqpoint{5.218909in}{2.178755in}}%
\pgfpathlineto{\pgfqpoint{5.223418in}{2.174346in}}%
\pgfpathlineto{\pgfqpoint{5.227927in}{2.128793in}}%
\pgfpathlineto{\pgfqpoint{5.232436in}{2.217185in}}%
\pgfpathlineto{\pgfqpoint{5.236945in}{2.216245in}}%
\pgfpathlineto{\pgfqpoint{5.241455in}{2.164032in}}%
\pgfpathlineto{\pgfqpoint{5.245964in}{2.166673in}}%
\pgfpathlineto{\pgfqpoint{5.250473in}{2.257361in}}%
\pgfpathlineto{\pgfqpoint{5.254982in}{2.171499in}}%
\pgfpathlineto{\pgfqpoint{5.259491in}{2.161435in}}%
\pgfpathlineto{\pgfqpoint{5.264000in}{2.237624in}}%
\pgfpathlineto{\pgfqpoint{5.268509in}{2.196202in}}%
\pgfpathlineto{\pgfqpoint{5.273018in}{2.187000in}}%
\pgfpathlineto{\pgfqpoint{5.277527in}{2.243962in}}%
\pgfpathlineto{\pgfqpoint{5.282036in}{2.207533in}}%
\pgfpathlineto{\pgfqpoint{5.286545in}{2.229845in}}%
\pgfpathlineto{\pgfqpoint{5.291055in}{2.228066in}}%
\pgfpathlineto{\pgfqpoint{5.295564in}{2.219226in}}%
\pgfpathlineto{\pgfqpoint{5.300073in}{2.234755in}}%
\pgfpathlineto{\pgfqpoint{5.304582in}{2.241921in}}%
\pgfpathlineto{\pgfqpoint{5.309091in}{2.239747in}}%
\pgfpathlineto{\pgfqpoint{5.313600in}{2.255705in}}%
\pgfpathlineto{\pgfqpoint{5.318109in}{2.220577in}}%
\pgfpathlineto{\pgfqpoint{5.322618in}{2.256122in}}%
\pgfpathlineto{\pgfqpoint{5.327127in}{2.219431in}}%
\pgfpathlineto{\pgfqpoint{5.336145in}{2.292851in}}%
\pgfpathlineto{\pgfqpoint{5.340655in}{2.208845in}}%
\pgfpathlineto{\pgfqpoint{5.345164in}{2.247337in}}%
\pgfpathlineto{\pgfqpoint{5.349673in}{2.249255in}}%
\pgfpathlineto{\pgfqpoint{5.354182in}{2.305466in}}%
\pgfpathlineto{\pgfqpoint{5.358691in}{2.338287in}}%
\pgfpathlineto{\pgfqpoint{5.363200in}{2.287074in}}%
\pgfpathlineto{\pgfqpoint{5.367709in}{2.284583in}}%
\pgfpathlineto{\pgfqpoint{5.372218in}{2.292684in}}%
\pgfpathlineto{\pgfqpoint{5.376727in}{2.284805in}}%
\pgfpathlineto{\pgfqpoint{5.381236in}{2.245986in}}%
\pgfpathlineto{\pgfqpoint{5.385745in}{2.332515in}}%
\pgfpathlineto{\pgfqpoint{5.390255in}{2.217235in}}%
\pgfpathlineto{\pgfqpoint{5.394764in}{2.294218in}}%
\pgfpathlineto{\pgfqpoint{5.399273in}{2.275370in}}%
\pgfpathlineto{\pgfqpoint{5.403782in}{2.308774in}}%
\pgfpathlineto{\pgfqpoint{5.408291in}{2.294385in}}%
\pgfpathlineto{\pgfqpoint{5.412800in}{2.287196in}}%
\pgfpathlineto{\pgfqpoint{5.417309in}{2.334528in}}%
\pgfpathlineto{\pgfqpoint{5.421818in}{2.285417in}}%
\pgfpathlineto{\pgfqpoint{5.426327in}{2.335368in}}%
\pgfpathlineto{\pgfqpoint{5.430836in}{2.333361in}}%
\pgfpathlineto{\pgfqpoint{5.435345in}{2.356990in}}%
\pgfpathlineto{\pgfqpoint{5.439855in}{2.295514in}}%
\pgfpathlineto{\pgfqpoint{5.444364in}{2.217113in}}%
\pgfpathlineto{\pgfqpoint{5.448873in}{2.349023in}}%
\pgfpathlineto{\pgfqpoint{5.453382in}{2.312377in}}%
\pgfpathlineto{\pgfqpoint{5.457891in}{2.315207in}}%
\pgfpathlineto{\pgfqpoint{5.466909in}{2.387743in}}%
\pgfpathlineto{\pgfqpoint{5.475927in}{2.300368in}}%
\pgfpathlineto{\pgfqpoint{5.480436in}{2.368099in}}%
\pgfpathlineto{\pgfqpoint{5.484945in}{2.379380in}}%
\pgfpathlineto{\pgfqpoint{5.489455in}{2.385591in}}%
\pgfpathlineto{\pgfqpoint{5.493964in}{2.318065in}}%
\pgfpathlineto{\pgfqpoint{5.498473in}{2.329096in}}%
\pgfpathlineto{\pgfqpoint{5.502982in}{2.359820in}}%
\pgfpathlineto{\pgfqpoint{5.507491in}{2.360927in}}%
\pgfpathlineto{\pgfqpoint{5.512000in}{2.399024in}}%
\pgfpathlineto{\pgfqpoint{5.516509in}{2.334584in}}%
\pgfpathlineto{\pgfqpoint{5.521018in}{2.357758in}}%
\pgfpathlineto{\pgfqpoint{5.525527in}{2.353204in}}%
\pgfpathlineto{\pgfqpoint{5.530036in}{2.305322in}}%
\pgfpathlineto{\pgfqpoint{5.534545in}{2.379347in}}%
\pgfpathlineto{\pgfqpoint{5.534545in}{2.379347in}}%
\pgfusepath{stroke}%
\end{pgfscope}%
\begin{pgfscope}%
\pgfpathrectangle{\pgfqpoint{0.800000in}{0.528000in}}{\pgfqpoint{4.960000in}{3.696000in}}%
\pgfusepath{clip}%
\pgfsetrectcap%
\pgfsetroundjoin%
\pgfsetlinewidth{1.505625pt}%
\definecolor{currentstroke}{rgb}{0.000000,1.000000,1.000000}%
\pgfsetstrokecolor{currentstroke}%
\pgfsetdash{}{0pt}%
\pgfpathmoveto{\pgfqpoint{1.025455in}{0.696378in}}%
\pgfpathlineto{\pgfqpoint{5.534545in}{0.707498in}}%
\pgfpathlineto{\pgfqpoint{5.534545in}{0.707498in}}%
\pgfusepath{stroke}%
\end{pgfscope}%
\begin{pgfscope}%
\pgfpathrectangle{\pgfqpoint{0.800000in}{0.528000in}}{\pgfqpoint{4.960000in}{3.696000in}}%
\pgfusepath{clip}%
\pgfsetrectcap%
\pgfsetroundjoin%
\pgfsetlinewidth{1.505625pt}%
\definecolor{currentstroke}{rgb}{1.000000,0.000000,1.000000}%
\pgfsetstrokecolor{currentstroke}%
\pgfsetdash{}{0pt}%
\pgfpathmoveto{\pgfqpoint{1.025455in}{0.696033in}}%
\pgfpathlineto{\pgfqpoint{1.192291in}{0.696300in}}%
\pgfpathlineto{\pgfqpoint{1.323055in}{0.696523in}}%
\pgfpathlineto{\pgfqpoint{1.422255in}{0.696612in}}%
\pgfpathlineto{\pgfqpoint{1.494400in}{0.696790in}}%
\pgfpathlineto{\pgfqpoint{1.616145in}{0.696934in}}%
\pgfpathlineto{\pgfqpoint{1.724364in}{0.697234in}}%
\pgfpathlineto{\pgfqpoint{1.810036in}{0.697257in}}%
\pgfpathlineto{\pgfqpoint{1.882182in}{0.697457in}}%
\pgfpathlineto{\pgfqpoint{2.012945in}{0.697590in}}%
\pgfpathlineto{\pgfqpoint{3.338618in}{0.699892in}}%
\pgfpathlineto{\pgfqpoint{3.365673in}{0.699925in}}%
\pgfpathlineto{\pgfqpoint{3.397236in}{0.699936in}}%
\pgfpathlineto{\pgfqpoint{4.601164in}{0.701960in}}%
\pgfpathlineto{\pgfqpoint{4.623709in}{0.701916in}}%
\pgfpathlineto{\pgfqpoint{4.957382in}{0.702505in}}%
\pgfpathlineto{\pgfqpoint{4.984436in}{0.702483in}}%
\pgfpathlineto{\pgfqpoint{5.097164in}{0.702694in}}%
\pgfpathlineto{\pgfqpoint{5.191855in}{0.702939in}}%
\pgfpathlineto{\pgfqpoint{5.214400in}{0.702894in}}%
\pgfpathlineto{\pgfqpoint{5.273018in}{0.702994in}}%
\pgfpathlineto{\pgfqpoint{5.534545in}{0.703295in}}%
\pgfpathlineto{\pgfqpoint{5.534545in}{0.703295in}}%
\pgfusepath{stroke}%
\end{pgfscope}%
\begin{pgfscope}%
\pgfsetrectcap%
\pgfsetmiterjoin%
\pgfsetlinewidth{0.803000pt}%
\definecolor{currentstroke}{rgb}{0.000000,0.000000,0.000000}%
\pgfsetstrokecolor{currentstroke}%
\pgfsetdash{}{0pt}%
\pgfpathmoveto{\pgfqpoint{0.800000in}{0.528000in}}%
\pgfpathlineto{\pgfqpoint{0.800000in}{4.224000in}}%
\pgfusepath{stroke}%
\end{pgfscope}%
\begin{pgfscope}%
\pgfsetrectcap%
\pgfsetmiterjoin%
\pgfsetlinewidth{0.803000pt}%
\definecolor{currentstroke}{rgb}{0.000000,0.000000,0.000000}%
\pgfsetstrokecolor{currentstroke}%
\pgfsetdash{}{0pt}%
\pgfpathmoveto{\pgfqpoint{5.760000in}{0.528000in}}%
\pgfpathlineto{\pgfqpoint{5.760000in}{4.224000in}}%
\pgfusepath{stroke}%
\end{pgfscope}%
\begin{pgfscope}%
\pgfsetrectcap%
\pgfsetmiterjoin%
\pgfsetlinewidth{0.803000pt}%
\definecolor{currentstroke}{rgb}{0.000000,0.000000,0.000000}%
\pgfsetstrokecolor{currentstroke}%
\pgfsetdash{}{0pt}%
\pgfpathmoveto{\pgfqpoint{0.800000in}{0.528000in}}%
\pgfpathlineto{\pgfqpoint{5.760000in}{0.528000in}}%
\pgfusepath{stroke}%
\end{pgfscope}%
\begin{pgfscope}%
\pgfsetrectcap%
\pgfsetmiterjoin%
\pgfsetlinewidth{0.803000pt}%
\definecolor{currentstroke}{rgb}{0.000000,0.000000,0.000000}%
\pgfsetstrokecolor{currentstroke}%
\pgfsetdash{}{0pt}%
\pgfpathmoveto{\pgfqpoint{0.800000in}{4.224000in}}%
\pgfpathlineto{\pgfqpoint{5.760000in}{4.224000in}}%
\pgfusepath{stroke}%
\end{pgfscope}%
\begin{pgfscope}%
\definecolor{textcolor}{rgb}{0.000000,0.000000,0.000000}%
\pgfsetstrokecolor{textcolor}%
\pgfsetfillcolor{textcolor}%
\pgftext[x=3.280000in,y=4.307333in,,base]{\color{textcolor}\ttfamily\fontsize{12.000000}{14.400000}\selectfont Iterations vs Input size}%
\end{pgfscope}%
\begin{pgfscope}%
\pgfsetbuttcap%
\pgfsetmiterjoin%
\definecolor{currentfill}{rgb}{1.000000,1.000000,1.000000}%
\pgfsetfillcolor{currentfill}%
\pgfsetfillopacity{0.800000}%
\pgfsetlinewidth{1.003750pt}%
\definecolor{currentstroke}{rgb}{0.800000,0.800000,0.800000}%
\pgfsetstrokecolor{currentstroke}%
\pgfsetstrokeopacity{0.800000}%
\pgfsetdash{}{0pt}%
\pgfpathmoveto{\pgfqpoint{0.897222in}{3.088923in}}%
\pgfpathlineto{\pgfqpoint{2.094230in}{3.088923in}}%
\pgfpathquadraticcurveto{\pgfqpoint{2.122008in}{3.088923in}}{\pgfqpoint{2.122008in}{3.116701in}}%
\pgfpathlineto{\pgfqpoint{2.122008in}{4.126778in}}%
\pgfpathquadraticcurveto{\pgfqpoint{2.122008in}{4.154556in}}{\pgfqpoint{2.094230in}{4.154556in}}%
\pgfpathlineto{\pgfqpoint{0.897222in}{4.154556in}}%
\pgfpathquadraticcurveto{\pgfqpoint{0.869444in}{4.154556in}}{\pgfqpoint{0.869444in}{4.126778in}}%
\pgfpathlineto{\pgfqpoint{0.869444in}{3.116701in}}%
\pgfpathquadraticcurveto{\pgfqpoint{0.869444in}{3.088923in}}{\pgfqpoint{0.897222in}{3.088923in}}%
\pgfpathlineto{\pgfqpoint{0.897222in}{3.088923in}}%
\pgfpathclose%
\pgfusepath{stroke,fill}%
\end{pgfscope}%
\begin{pgfscope}%
\pgfsetrectcap%
\pgfsetroundjoin%
\pgfsetlinewidth{1.505625pt}%
\definecolor{currentstroke}{rgb}{1.000000,0.000000,0.000000}%
\pgfsetstrokecolor{currentstroke}%
\pgfsetdash{}{0pt}%
\pgfpathmoveto{\pgfqpoint{0.925000in}{4.041342in}}%
\pgfpathlineto{\pgfqpoint{1.063889in}{4.041342in}}%
\pgfpathlineto{\pgfqpoint{1.202778in}{4.041342in}}%
\pgfusepath{stroke}%
\end{pgfscope}%
\begin{pgfscope}%
\definecolor{textcolor}{rgb}{0.000000,0.000000,0.000000}%
\pgfsetstrokecolor{textcolor}%
\pgfsetfillcolor{textcolor}%
\pgftext[x=1.313889in,y=3.992731in,left,base]{\color{textcolor}\ttfamily\fontsize{10.000000}{12.000000}\selectfont Bubble}%
\end{pgfscope}%
\begin{pgfscope}%
\pgfsetrectcap%
\pgfsetroundjoin%
\pgfsetlinewidth{1.505625pt}%
\definecolor{currentstroke}{rgb}{0.486275,0.988235,0.000000}%
\pgfsetstrokecolor{currentstroke}%
\pgfsetdash{}{0pt}%
\pgfpathmoveto{\pgfqpoint{0.925000in}{3.836739in}}%
\pgfpathlineto{\pgfqpoint{1.063889in}{3.836739in}}%
\pgfpathlineto{\pgfqpoint{1.202778in}{3.836739in}}%
\pgfusepath{stroke}%
\end{pgfscope}%
\begin{pgfscope}%
\definecolor{textcolor}{rgb}{0.000000,0.000000,0.000000}%
\pgfsetstrokecolor{textcolor}%
\pgfsetfillcolor{textcolor}%
\pgftext[x=1.313889in,y=3.788128in,left,base]{\color{textcolor}\ttfamily\fontsize{10.000000}{12.000000}\selectfont Selection}%
\end{pgfscope}%
\begin{pgfscope}%
\pgfsetrectcap%
\pgfsetroundjoin%
\pgfsetlinewidth{1.505625pt}%
\definecolor{currentstroke}{rgb}{0.000000,1.000000,0.498039}%
\pgfsetstrokecolor{currentstroke}%
\pgfsetdash{}{0pt}%
\pgfpathmoveto{\pgfqpoint{0.925000in}{3.632136in}}%
\pgfpathlineto{\pgfqpoint{1.063889in}{3.632136in}}%
\pgfpathlineto{\pgfqpoint{1.202778in}{3.632136in}}%
\pgfusepath{stroke}%
\end{pgfscope}%
\begin{pgfscope}%
\definecolor{textcolor}{rgb}{0.000000,0.000000,0.000000}%
\pgfsetstrokecolor{textcolor}%
\pgfsetfillcolor{textcolor}%
\pgftext[x=1.313889in,y=3.583525in,left,base]{\color{textcolor}\ttfamily\fontsize{10.000000}{12.000000}\selectfont Insertion}%
\end{pgfscope}%
\begin{pgfscope}%
\pgfsetrectcap%
\pgfsetroundjoin%
\pgfsetlinewidth{1.505625pt}%
\definecolor{currentstroke}{rgb}{0.000000,1.000000,1.000000}%
\pgfsetstrokecolor{currentstroke}%
\pgfsetdash{}{0pt}%
\pgfpathmoveto{\pgfqpoint{0.925000in}{3.427532in}}%
\pgfpathlineto{\pgfqpoint{1.063889in}{3.427532in}}%
\pgfpathlineto{\pgfqpoint{1.202778in}{3.427532in}}%
\pgfusepath{stroke}%
\end{pgfscope}%
\begin{pgfscope}%
\definecolor{textcolor}{rgb}{0.000000,0.000000,0.000000}%
\pgfsetstrokecolor{textcolor}%
\pgfsetfillcolor{textcolor}%
\pgftext[x=1.313889in,y=3.378921in,left,base]{\color{textcolor}\ttfamily\fontsize{10.000000}{12.000000}\selectfont Merge}%
\end{pgfscope}%
\begin{pgfscope}%
\pgfsetrectcap%
\pgfsetroundjoin%
\pgfsetlinewidth{1.505625pt}%
\definecolor{currentstroke}{rgb}{1.000000,0.000000,1.000000}%
\pgfsetstrokecolor{currentstroke}%
\pgfsetdash{}{0pt}%
\pgfpathmoveto{\pgfqpoint{0.925000in}{3.221980in}}%
\pgfpathlineto{\pgfqpoint{1.063889in}{3.221980in}}%
\pgfpathlineto{\pgfqpoint{1.202778in}{3.221980in}}%
\pgfusepath{stroke}%
\end{pgfscope}%
\begin{pgfscope}%
\definecolor{textcolor}{rgb}{0.000000,0.000000,0.000000}%
\pgfsetstrokecolor{textcolor}%
\pgfsetfillcolor{textcolor}%
\pgftext[x=1.313889in,y=3.173369in,left,base]{\color{textcolor}\ttfamily\fontsize{10.000000}{12.000000}\selectfont Quick}%
\end{pgfscope}%
\end{pgfpicture}%
\makeatother%
\endgroup%


\end{document}
